\chapter{Ultralimits}

Here we introduce ultralimits of sequence of points, metric spaces and functions.
The ultralimits of metric spaces can be considered as an extension of Gromov--Hausdorff convergence since the ultralimit is isometric to the Gromov--Hausdorff limit once the later is defined.
Our presentation is based on \cite{kleiner-leeb}.

In this book, the use of ultralimits is very limited; 
we use them only as a canonical way to pass to a convergent subsequence.
(In principle, we could avoid to sell our souls 
to the set-theoretical devel but then we had to say ``pass to convergents subsequence'' too many times.)

\section{Ultrafilters}

We will need existence of one selective ultrafilter $\o$,
you can think that it is fixed once for all.
The existence follows from axiom of choice and continuum hypothesis.

Recall that $\NN$ denotes the set of natural numbers, $\NN=\{1,2,\dots\}$

\begin{thm}{Definition}
A finitely additive measure $\o$ 
on  $\NN$ 
is called \emph{ultrafilter}\index{ultrafilter} if it satisfies the following conditions: 
\begin{subthm}{}
$\o(S)=0$ or $1$ for any subset $S\subset \NN$,
\end{subthm}
An ultrafilter $\o$ is called 
\emph{nonprinciple}\index{ultrafilter!nonprinciple ultrafilter}\index{nonprinciple ultrafilter} if in addition 
\begin{subthm}{}
$\o(F)=0$ for any finite subset $F\subset \NN$.
\end{subthm}
A nonprinciple ultrafilter $\o$ is called 
\emph{selective}\index{ultrafilter!selective ultrafilter}\index{selective ultrafilter} if in addition 
\begin{subthm}{}
For any partition of $\NN$ into sets $\{C_\alpha\}_{\alpha\in\IndexSet}$ such that $\o(C_\alpha)=0$ for each $\alpha$
there is a set $S\subset \NN$ such that $\o(S)=1$ and $S\cap C_\alpha$ is a one-point set for each $\alpha\in\IndexSet$.
\end{subthm}
\end{thm}

More commonly nonprinciple ultrafilter is defined as a collection, say $\mathfrak{F}$ of sets in $\NN$ such that
\begin{enumerate}
\item\label{filter:supset} if $P\in \mathfrak{F}$ and $Q\supset P$ then $Q\in \mathfrak{F}$,
\item\label{filter:cap} if $P, Q\in \mathfrak{F}$ then $P\cap Q\in \mathfrak{F}$,
\item\label{filter:ultra} for any subset $P\subset\NN$, either $P$ or its complement is an element of $\mathfrak{F}$.
\item\label{filter:non-prin} if $F\subset \NN $ is finite then $F\notin \mathfrak{F}$.
\end{enumerate}
Setting $P\in\mathfrak{F}\Leftrightarrow\o(P)=1$, makes these two definitions equivalent.

A nonempty collection of sets $\mathfrak{F}$ which does not include the empty set and satisfy only conditions \ref{filter:supset} and \ref{filter:cap} is called \emph{filter}; 
if in addition it satisfies Condition~\ref{filter:ultra} it is called \emph{ultrafilter}.
From Zorn's lemma, it follows that every filter contains an ultrafilter.
Thus there is an ultrafilter $\mathfrak{F}$ which contained in the filter of all complements of finite sets; clearly it is nonprinciple.

The existence of selective ultrafilter follows from continuum hypothesis;
it was proved by  Rudin in \cite{rudin}.

Given a set $S\subset \NN$, consider subset $\Omega_S$ of all ultrafilters $\o$ such that $\o(S)=1$.
It is straightforward to check that the sets $\Omega_S$ for all $S\subset \NN$ form a topology on the set of ultrafilters on $\NN$. 
The obtained space is called \emph{Stone--\v{C}ech compactification} of $\NN$;
it is usually denoted as $\beta\NN$.
There is a natural embedding $\NN\hookrightarrow\beta\NN$ defined as following
$n\mapsto\o_n$, $\o_n(S)=1$ if and only if $n\in S$. 
(The image of $\NN$ is formed by the principle ultrafilters.)
Thus we can (and will) consider $\NN$ as a subset of $\beta\NN$.

The space $\beta\NN$ is the maximal compact Hausdorff space which contains $\NN$  as an everywhere dense subset;
more precisely, for any compact Hausdorff space $\spc{X}$ 
and a map $f\:\NN\to \spc{X}$ there is unique continuous map $\bar f\:\beta\NN\to X$ such that the restriction $\bar f|\NN$ coincides with $f$. 

\section{Ultralimits of points}
\label{ultralimits}\index{ultralimit}

Fix an ultrafilter $\o$.
Assume $(x_n)$ is a sequence of points in a metric space $\spc{X}$. 
Let us define the \emph{$\o$-limit}\index{$\o$!$\o$-limit} of $(x_n)$ as the point $x_\o$ 
such that for any $\eps>0$, $\o$-almost all elements of $(x_n)$ lie in $\oBall(x_\o,\eps)$; 
i.e.,
\[\o\set{n\in\NN}{\dist{x_\o}{x_n}{}<\eps}=1.\]
In this case, we will write 
\[x_\o=\lim_{n\to\o} x_n
\ \ \text{or}\ \ 
x_n\to x_\o\ \t{as}\ n\to\o.\]

For example if $\o$ is the principle ultrafilter such that $\o(\{n\})=1$ for some $n\in\NN$ then
$x_\o=x_n$.

Note that $\o$-limits of a sequence and its subsequence may differ, say in general
\[\lim_{n\to\o}x_n
\ne
\lim_{n\to\o}x_{2\cdot n}.\]

\begin{thm}{Proposition}\label{prop:ultra/partial}
Let $\o$ be a nonprinciple ultrafilter.
Assume $(x_n)$ is a sequence of points in a metric space $\spc{X}$
and $x_n\to  x_\o$ as $n\to\o$.
Then $x_\o$ is a partial limit of the sequence $(x_n)$;
i.e., there is a subsequence $(x_n)_{n\in S}$ which converges to $x_\o$ in the usual sense.

Moreover, if $\o$ is selective,
then the subsequence $(x_n)_{n\in S}$ can be chosen so that $\o(S)=1$.
\end{thm}

\parit{Proof.}
Given $\eps>0$, 
set $S_\eps=\set{n\in\NN}{\dist{x_n}{x_\o}{}<\eps}$.

Note that $\o(S_\eps)=1$ for any $\eps>0$.
Since $\o$ is nonprinciple, the set $S_\eps$ is infinite.
Therefore we can choose an increasing sequence $(n_\kay)$
such that $n_\kay\in S_{\frac1\kay}$ for each $\kay\in \NN$.
Clearly $x_{n_\kay}\to x_\o$ as $\kay\to\infty$.

Now assume that $\o$ is selective.
Consider the sets
\begin{align*}
C_\kay&=\set{n\in\NN}{\tfrac1{n}<\dist{x_n}{x_\o}{}\le \tfrac1{n-1}},
\intertext{where we assume $\tfrac10=\infty$ and the set }
C_\infty&=\set{n\in\NN}{x_n=x_\o},
\end{align*}

Note that $\o(C_\kay)=0$ for any $\kay\in\NN$.

If $\o(C_\infty)=1$, we can take the sequence $(x_n)_{n\in C_\infty}$.

Otherwise discarding all empty sets among $C_\kay$ and $C_\infty$ gives a partition of $\NN$ into countable collection of $\o$-small sets.
Since $\o$ is selective, we can choose a set $S\subset\NN$ such that
$S$ meets each set of the partition at one point and $\o(S)=1$.
Clearly the sequence $(x_n)_{n\in S}$ converges to $x_o$ in the usual sense.
\qeds

The following proposition 
is analogous to the statement that any sequence in a compact metric space 
has a convergent subsequence;
it can be proved the same way.

\begin{thm}{Proposition}\label{prop:ultra/compact}
Let $\spc{X}$ be a compact metric space.
Then
any sequence of points $(x_n)$ in $\spc{X}$ has unique $\o$-limit $x_\o$.

In particular, a bounded sequence of real numbers has a unique $\o$-limit.
\end{thm}

The following lemma is an ultralimit analog of Cauchy convergence test.

\begin{thm}{Lemma}\label{lem:X-X^w}
Let $(x_n)$ be a sequence of points in a complete space $\spc{X}$. 
Assume for each subsequence $(y_n)$ of $(x_n)$, 
the $\o$-limit 
\[y_\o=\lim_{n\to\o}y_{n}\in \spc{X}\]
is defined and does not depend on the choice of subsequence, 
then the sequence $(x_n)$ converges in the usual sense.
\end{thm}

\parit{Proof.} Assume that $(x_n)$ is not a Cauchy sequence. 
Then for some $\eps>0$, there is a subsequence $(y_n)$ of $(x_n)$ such that $\dist{x_n}{y_n}{}\ge\eps$ for all $n$.

It follows that $\dist{x_\o}{y_\o}{}\ge \eps$, a contradiction.\qeds

\begin{thm}{Corollary}\label{cor:two-geodesics-in-ultrapower}
Assume $\spc{X}$ is a complete length space 
and $p,q\in\spc{X}$ cannot be joined by a geodesic in $\spc{X}$.  Then there are at least two distinct geodesics between $p$ and $q$ 
in the ultrapower $\spc{X}^\o$.
\end{thm}

\section{Ultralimit of spaces}\label{sec:Ultralimit of spaces}


Let $\spc{X}_n$ be a sequence of metric spaces
$\spc{X}_n$ .
Consider all sequences $x_n\in \spc{X}_n$.
On the set of all such sequences,
define a pseudometric on these subsequences by
\[\dist{(x_n)}{(y_n)}{}
=
\lim_{n\to\o} \dist{x_n}{y_n}{}.
\eqlbl{eq:olim-dist}\]
Note that the $\o$-limit on the right hand side is always defined 
and takes a value in $[0,\infty]$. 

Set $\spc{X}_\o$ to be the corresponding metric space; 
i.e., the underlying set of $\spc{X}_\o$ is formed by classes of equivalence of sequences of points $x_n\in\spc{X}_n$ 
defined by 
\[(x_n)\sim(y_n)
\ \Leftrightarrow\ 
\lim_{n\to\o} \dist{x_n}{y_n}{}=0\]
and the distance is defined as in \ref{eq:olim-dist}.

The space $\spc{X}_\o$ is called \emph{$\o$-limit}\index{$\o$-limit space} of $\spc{X}_n$.
We will use $\spc{X}_\o$ as the standard notation for 
$\o$-limit of $\spc{X}_n$;
we may also write  
\[\spc{X}_n\to\spc{X}_\o\ \ \text{as}\ \  n\to\o\ \ \text{or}\ \ \spc{X}_\o=\lim_{n\to\o}\spc{X}_n.\]

Given a sequence $x_n\in \spc{X}_n$,
we will denote by $x_\o$ its equivalence class which is a point in $\spc{X}_\o$;
equivalently we will write
\[x_n\to x_\o \ \ \text{as}\ \  n\to\o\ \ \text{or}\ \ x_\o=\lim_{n\to\o} x_n.\]


\parbf{Gromov--Hausdorff convergence versus ultralimits.}  

\begin{thm}{Theorem}\label{thm:ultra-GH}
Assume $\spc{X}_n$ is a sequence of complete spaces. 
Let $\spc{X}_n\to \spc{X}_\o$ as $n\to\o$
and $\spc{Y}_n\subset \spc{X}_n$ 
is a sequence of subsets such that $\spc{Y}_n\GHto\spc{Y}_\infty$. 
Then there is an distance preserving map 
$\iota:\spc{Y}_\infty\to \spc{X}_\o$.

Moreover:

\begin{subthm}{thm:ultra-GH:a}
If $\spc{X}_n\GHto \spc{X}_\infty$ 
and $\spc{X}_\infty$ is compact then 
$\spc{X}_\infty$ is isometric to $\spc{X}_\o$.
\end{subthm}

\begin{subthm}{thm:ultra-GH:b}
If $\spc{X}_n\GHto \spc{X}_\infty$ 
and $\spc{X}_\infty$ is proper then 
$\spc{X}_\infty$ is isometric to a metric component of $\spc{X}_\o$.
\end{subthm}

\end{thm}

\parit{Proof.} 
For each point $y_\infty\in \spc{Y}_\infty$ 
choose a lifting $y_n\in \spc{Y}_n$.
Pass to the $\o$-limit $y_\o\in \spc{X}_\o$ of $(y_n)$.
Clearly for any $y_\infty,z_\infty\in \spc{Y}_\infty$, 
we have 
\[\dist{y_\infty}{z_\infty}{\spc{Y}_\infty}=\dist{y_\o}{z_\o}{\spc{X}_\o};\] 
i.e., the map $y_\infty\mapsto y_\o$ gives a distance preserving map $\iota:\spc{Y}_\infty\to \spc{X}_\o$. 


\parit{(\ref{SHORT.thm:ultra-GH:a})$+$(\ref{SHORT.thm:ultra-GH:b}).}
Fix $x_\o\in \spc{X}_\o$.
Choose a sequence $x_n\in \spc{X}_n$ 
such that $x_n\to x_\o$ as $n\to\o$. 

Denote by $\bm{X}$ the common space for the convergence $\spc{X}_n\GHto \spc{X}_\infty$;
see the definition of Gromov--Hausdorff convergence (\ref{def:GH}).
Consider the sequence $(x_n)$ 
as a sequence of points in $\bm{X}$.

If the $\o$-limit $x_\infty$ of $(x_n)$ exists, 
it has to lie in $\spc{X}_\infty$. 

The point $x_\infty$, if defined, does not depend on the choice of $(x_n)$.
Indeed, if $y_n\in\spc{X}_n$ be an other sequence such that $y_n\to x_\o$ as $n\to\o$ then 
\[
\dist{y_\infty}{x_\infty}{}=\lim_{n\to\o}\dist{y_n}{x_n}{}=0;
\]
i.e., $x_\infty=y_\infty$.


This way we constructed a map $\nu\:x_\o\to x_\infty$;
it is defined on a subset of $\Dom\nu \subset\spc{X}_\o$.
By construction of $\iota$, 
we get $\iota\circ\nu(x_\o)=x_\o$ for any $x_\o\in \Dom\nu$.

Finally note that if $\spc{X}_\infty$ is compact then $\nu$ is defined on whole $\spc{X}_\o$;
this proves (\ref{SHORT.thm:ultra-GH:a}).

If $\spc{X}_\infty$ is proper, choose any point $z_\infty\in \spc{X}_\infty$
and set $z_\o=\iota(z_\infty)$.
For any point $x_\o\in \spc{X}_\o$ on finite distance from $z_\o$,
for the sequence $x_n$ 
as above we have that $\dist{z_n}{x_n}{}$ is bounded for $\o$-almost all $n$.
Since $\spc{X}_\infty$ is proper $\nu(x_\o)$ is defined;
in other words $\nu$ is defined on the metric component of $z_\o$.
Hence (\ref{SHORT.thm:ultra-GH:b}) follows.
\qeds

\begin{thm}{Corollary} 
\label{cor:ulara-geod}
$\o$-limit of a sequence of complete length spaces is geodesic.
\end{thm}

\parit{Proof.} Given two points $x_\o,y_\o\in \spc{X}_\o$, find two bounded sequences of points $x_n,y_n\in \spc{X}_n$, $x_n\to x_\o$, $y_n\to y_\o$ as $n\to\o$.
Consider a sequence of paths  $\gamma_n\:[0,1]\to \spc{X}_n$ from $_n$ to $y_n$
 such that 
\[\length\gamma_n\le \dist{x_n}{y_n}{}+\tfrac{1}{n}\]
Apply Theorem~\ref{thm:ultra-GH} 
for the images $\spc{Y}_n=\gamma_n([0,1])\subset \spc{X}_n$.
\qeds


\parbf{Ultrapower.} If all the metric spaces in the sequence are identical $\spc{X}_n=\spc{X}$, 
its $\o$-limit 
$\lim_{n\to\o}\spc{X}_n$
is denoted by $\spc{X}^\o$
and called $\o$-power of $\spc{X}$.
 
According to Theorem~\ref{thm:ultra-GH},
there is a distance preserving map
$\iota:\spc{X}\hookrightarrow \spc{X}^\o$, here $\iota(y)$ is the equivalence class of constant sequence $y_n=y$. 

The image $\iota(\spc{X})$ might be proper subset of $\spc{X}^\o$.
For example $\RR^\o$ has pairs of points on distance $\infty$ from each other although each metric component of $\RR^\o$ is isometric to $\RR$.

According to Theorem~\ref{thm:ultra-GH}, 
if $\spc{X}$ is compact then $\iota(\spc{X})=\spc{X}^\o$;
in particular, $\spc{X}^\o$ is isometric to $\spc{X}$.
If $\spc{X}$ is proper then $\iota(\spc{X})$ forms a metric component of $\spc{X}^\o$.

The embedding $\iota$ gives us right to treat $\spc{X}$ as a subset of its ultrapower $\spc{X}^\o$. 

\begin{thm}{Lemma}
Let $\spc{X}$ be a complete metric space. 
Then $\spc{X}^\o$ is geodesic space if and only if $\spc{X}$ is an length space.
\end{thm}

\parbf{Blowups and ultrablowup.}
For a metric space $\spc{X}$ and a positive real number $\lam$,
we will denote by $\lam\blow\spc{X}$ its \emph{$\lam$-blowup}\index{blowup},
which is a metric space with the same underlying set as $\spc{X}$ and the metric multiplied by $\lam$.
The tautological bijection $\spc{X}\to \lam\blow\spc{X}$ will be denoted as $x\mapsto \lam\blow x$, 
so 
\[\dist{(\lam\blow x)}{(\lam\blow y)}{}
=
\lam\cdot\dist[{{}}]{x}{y}{}\] 
for any $x,y\in \spc{X}$.

The $\o$-blowup $\o\blow\spc{X}$ of $\spc{X}$ is defined as the $\o$-limit
of $n\blow\spc{X}$; i.e.,
\[\o\blow\spc{X}
\df
\lim_{n\to\o} n\blow\spc{X}.\]

Given a point $x\in \spc{X}$ we can consider the sequence $n\blow x\in n\blow\spc{X}$;
it corresponds to a point $\o\blow x\in \o\blow\spc{X}$.
Note that if $x\ne y$ then $\dist{\o\blow x}{\o\blow y}{\o\blow\spc{X}}=\infty$;
i.e., 
$\o\blow x$ and $\o\blow y$ 
belong to different metric components of $\o\blow\spc{X}$.




\section{Ultralimits of functions}

Let $\spc{X}_n$ be a sequence of metric spaces with marked points $\star_n\in \spc{X}_n$.
Assume $(\spc{X}_n,\star_n)\to (\spc{X}_\o,\star_\o)$
as $n\to \o$.

Let $f_n\:\spc{X}_n\subto\RR$ be a sequence of subfunctions.
Consider the set $\Omega$ of all points $x_\o\in \spc{X}_\o$ 
such that if $x_n\to x_\o$
as $n\to\o$ 
for a sequence $x_n\in \spc{X}_n$
then $x_n\in \Dom f_n$ for $\o$-almost all $n$.
Note that $\Omega$ is open.

Assume there is a function $f_\o\:\Omega\to\RR$
which satisfies the following condition.
For any $x_\o\in \Omega$
and any sequence of points $x_n\in\spc{X}_n$
if $x_n\to x_\o$ as $n\to\o$ then $f_n(x_n)\to f_\o(x_o)$ as $n\to\o$.
In this case 
the subfunction $f_\o\:\spc{L}_\o\to\RR$ 
is said to be 
$\o$-limit of $f_n\:\spc{L}_n\to\RR$.

The following lemma gives a mild condition on a sequence of functions $f_n$
which guarantee the existence of $\o$-limit.

\begin{thm}{Lemma}
Let $\spc{X}_n$ be a sequence of metric spaces with marked points $\star_n\in \spc{X}_n$.
Assume $(\spc{X}_n,\star_n)\to (\spc{X}_\o,\star_\o)$
as $n\to \o$
and $f_n\:\spc{X}_n\subto\RR$ be a sequence of subfunctions.

Given $\eps>0$, 
consider open set
\[\Omega_n^\eps=\set{x_n\in\spc{X}_n}{\dist{(\spc{L}\backslash\Dom f_n)}{x_n}{}>\eps, \dist{\star_n}{x_n}{}<\tfrac1\eps
}.\]

Then if for any $\eps>0$ the restrictions $f_n|\Omega_n^\eps$ are uniformly bounded and uniformly continuous then $\o$-limit $f_\o\:\spc{L}_\o\to\RR$
of $f_n$ is defined.

In particular if $f_n$ are uniformly bounded and continuous then 
the $\o$-limit is defined.
\end{thm}

The proof is straightforward.

If $f\:\spc{X}\subto\RR$ be a subfunction.
The $\o$-limit of constant sequence $f_n=f$ is called $\o$-power of $f$ and denoted by $f^\o$.
So
\[f^\o\:\spc{X}\subto\RR,\ \ f^\o(x_\o)=\lim_{n\to\o} f(x_n).\]

Recall that we treat $\spc{X}$ as a subset of its $\o$-power $\spc{X}^\o$.
Note that $\Dom f=\spc{X}\cap \Dom f^\o$.
Moreover, 
if $\oBall(x,\eps)_{\spc{X}}\subset \Dom f$
then $\oBall(x,\eps)_{\spc{X}^\o}\subset \Dom f^\o$



\section{Exercises}

The following exercise provides a partial converse for Theorem \ref{thm:ultra-GH}.

\begin{thm}{Exercise}
Let $(\spc{X}_n,\star_n)$ be a sequence of metric spaces with marked points.
Assume
$(\spc{X}_n,\star_n)\to(\spc{X}_\o,\star_\o)$
as $n\to \o$ and the space $\spc{X}_\o$ is a proper;
moreover the isometry class of $(\spc{X}_\o,\star_\o)$ does not depend on choice of ultrafilter $\o$.
Prove that $(\spc{X}_n,\star_n)\xto{}(\spc{X}_\o,\star_\o)$.
\end{thm}

\begin{thm}{Exercise}\label{ex:ultra-unique-geod}
Let $\spc{X}$ be a complete length space.
Assume that for points $p,q\in \spc{X}$ there is unique geodesic $[p q]$ in $\spc{X}^\o$.
Then $[p q]$ lies in $\spc{X}$.
\end{thm}

