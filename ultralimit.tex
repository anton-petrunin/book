%%%!TEX root =the-ultralimit.tex
\chapter{Ultralimits}

Here we introduce ultralimits of sequences of points, metric spaces, and functions.
Our presentation is based on \cite{kleiner-leeb}.

Ultralimits are closely related to Gromov--Hausdorff limits.
We use them only as a canonical way to pass to  convergent subsequences.
We could avoid using them at the cost of saying ``pass to a convergent subsequence'' too many times. This however could become extremely cumbersome in some situations and obscure ideas of the proof such as in the proof of the globalization theorem for general $\Alex{}$ spaces. Also the use of ultralimits is very convenient when dealing with $\CAT{}$ spaces due to the lack of compactness results and is widely used in literature.
%See for example Gromov's book \cite{gromov:asymt-inv} or the above mentioned reference by Kleiner and Leeb.


\section{Ultrafilters}

We will need the existence of a selective ultrafilter $\o$ that will be fixed once and  for all.
The existence follows from the axiom of choice and continuum hypothesis.

\parbf{Measure-theoretic definition.}
Recall that $\NN$ denotes the set of natural numbers, $\NN=\{1,2,\dots\}$.

\begin{thm}{Definition}\label{def:ultrafilter}
A finitely additive measure $\o$ 
on  $\NN$ 
is called an \index{ultrafilter}\emph{ultrafilter} if it satisfies 
\begin{subthm}{}
$\o(S)=0$ or $1$ for any subset $S\subset \NN$.
\end{subthm}
An ultrafilter $\o$ is called 
\index{ultrafilter!nonprincipal ultrafilter}\index{nonprincipal ultrafilter}\emph{nonprincipal} if in addition 
\begin{subthm}{}
$\o(F)=0$ for any finite subset $F\subset \NN$.
\end{subthm}
A nonprincipal ultrafilter $\o$ is called 
\emph{selective}\index{ultrafilter!selective ultrafilter}\index{selective ultrafilter} if in addition 
\begin{subthm}{}
for any partition of $\NN$ into sets $\{C_\alpha\}_{\alpha\in\IndexSet}$ such that $\o(C_\alpha)\z=0$ for each $\alpha$, 
there is a set $S\subset \NN$ such that $\o(S)=1$ and $S\cap C_\alpha$ is a one-point set for each $\alpha\in\IndexSet$.
\end{subthm}
\end{thm}

If $\o(S)=0$ for some subset $S\subset \NN$,
we say that $S$ is \index{$\o$-small}\emph{$\o$-small}. 
If $\o(S)=1$, we say that $S$ contains \index{$\o$-almost all}\emph{$\o$-almost all} elements of $\NN$.

\begin{thm}{Advanced exercise}\label{ex:ultrakatetov}
Let $\o$ be an ultrafilter and $f\:\NN\z\to \NN$.
Suppose that $\o(S)\le \o(f^{-1}(S))$ for any set $S\subset \NN$.
Show that $f(n)=n$ for $\o$-almost all $n\in\NN$.
\end{thm}

\parbf{Classical definition.}
More commonly, a nonprincipal ultrafilter is defined as a collection, say $\mathfrak{F}$, of sets in $\NN$ such that
\begin{enumerate}
\item\label{filter:supset} if $P\in \mathfrak{F}$ and $Q\supset P$, then $Q\in \mathfrak{F}$,
\item\label{filter:cap} if $P, Q\in \mathfrak{F}$, then $P\cap Q\in \mathfrak{F}$,
\item\label{filter:ultra} for any subset $P\subset\NN$, either $P$ or its complement is an element of $\mathfrak{F}$,
\item\label{filter:non-prin} if $F\subset \NN $ is finite, then $F\notin \mathfrak{F}$.
\end{enumerate}

Setting 
\[P\in\mathfrak{F}\quad\iff\quad\o(P)=1\] 
makes these two definitions equivalent.

A nonempty collection of sets $\mathfrak{F}$ that does not include the empty set and satisfies only conditions \ref{filter:supset} and \ref{filter:cap} is called a \index{filter}\emph{filter}; 
if in addition $\mathfrak{F}$ satisfies Condition~\ref{filter:ultra} it is called an \index{ultrafilter}\emph{ultrafilter}.
From Zorn's lemma, it follows that every filter is contained in an ultrafilter.
Thus there is an ultrafilter $\mathfrak{F}$ contained in the filter of all complements of finite sets; clearly this $\mathfrak{F}$ is nonprincipal.

The existence of a selective ultrafilter follows from the continuum hypothesis;
this was proved by Walter Rudin \cite{rudin}.

\parbf{Stone--\v{C}ech compactification.}
Given a set $S\subset \NN$, consider the subset $\Omega_S$ of all ultrafilters $\o$ such that $\o(S)=1$.
It is straightforward to check that the sets $\Omega_S$ for all $S\subset \NN$ form a topology on the set of ultrafilters on~$\NN$. 
The resulting space is called the \index{Stone--\v{C}ech compactification}\emph{Stone--\v{C}ech compactification} of~$\NN$; it is usually denoted by $\beta\NN$\index{ $\beta\NN$} .

There is a natural embedding $\NN\hookrightarrow\beta\NN$ defined by 
$n\mapsto\o_n$, where $\o_n$ is the principal ultrafilter such that $\o_n(S)=1$ if and only if $n\in S$. 
Using this embedding, we can (and will) consider $\NN$ as a subset of $\beta\NN$.

The space $\beta\NN$ is the maximal compact Hausdorff space that contains $\NN$  as an everywhere dense subset.
More precisely, for any compact Hausdorff space $\spc{X}$ 
and a  map $f\:\NN\to \spc{X}$, there is a unique continuous map $\bar f\:\beta\NN\to X$ such that the restriction $\bar f|_\NN$ coincides with $f$. 

\section{Ultralimits of points}
\label{ultralimits}

Fix an ultrafilter $\o$.
Assume $x_n$ is a sequence of points in a metric space $\spc{X}$. 
Define an  \index{ultralimit of points}\emph{$\o$-limit} of $x_n$ to be a point $x_\o$ 
such that for any $\eps>0$, $\o$-almost all elements of $x_n$ lie in $\oBall(x_\o,\eps)$; 
that is,
\[\o\set{n\in\NN}{\dist{x_\o}{x_n}{}<\eps}=1.\]
In this case, we write 
\[x_\o=\lim_{n\to\o} x_n
\quad \text{or}\quad 
x_n\to x_\o\quad \text{as}\quad n\to\o.\]

Also, if $\spc{X}=\RR$ we write $\lim_{n\to\o} x_n=\pm\infty$ if 
\[\o\set{n\in\NN}{\pm x_n>L}=1\] for any $L\in\RR$.


It easily follows from the definition that  $\o$-limits are unique if they exist. 
For example, if $\o$ is the principal ultrafilter such that $\o(\{n\})=1$ for some $n\in\NN$, then
$x_\o=x_n$.

Note that $\o$-limits of a sequence and its subsequences may differ.
For example, in general
\[\lim_{n\to\o}x_n
\ne
\lim_{n\to\o}x_{2\cdot n}.\]

The sequence $x_n$ can be regarded as a map $\NN\to\spc{X}$.
If $\spc{X}$ is compact, then this map can be uniquely extended to a continuous map to the Stone--\v{C}ech compactification $\beta\NN$ of $\NN$.
Then $x_\o$ is the image of~$\o$. 

\begin{thm}{Proposition}\label{prop:ultra/partial}
Let $\o$ be a nonprincipal ultrafilter.
Assume $x_n$ is a sequence of points in a metric space $\spc{X}$
and $x_n\to  x_\o$ as $n\to\o$.
Then there is a subsequence of $x_n$ that converges to $x_\o$ in the usual sense.

Moreover, if $\o$ is selective,
then the subsequence $(x_n)_{n\in S}$ can be chosen so that $\o(S)=1$.
\end{thm}

\parit{Proof.}
Given $\eps>0$, 
let $S_\eps=\set{n\in\NN}{\dist{x_n}{x_\o}{}<\eps}$.

Note that $\o(S_\eps)=1$ for any $\eps>0$.
Since $\o$ is nonprincipal, the set $S_\eps$ is infinite.
Therefore we can choose an increasing sequence $n_\kay$
such that $n_\kay\in S_{\frac1\kay}$ for each $\kay\in \NN$.
Clearly $x_{n_\kay}\to x_\o$ as $\kay\to\infty$.

Now assume that $\o$ is selective.
Consider the sets
\begin{align*}
C_\kay&=\set{n\in\NN}{\tfrac1{\kay}<\dist{x_n}{x_\o}{}\le \tfrac1{\kay-1}},
\intertext{where we assume $\tfrac10=\infty$, and the set }
C_\infty&=\set{n\in\NN}{x_n=x_\o}.
\end{align*}

Note that $\o(C_\kay)=0$ for any $\kay\in\NN$.

If $\o(C_\infty)=1$, we can take the subsequence consisting of the $x_n$, $n\in C_\infty$.

Otherwise, discarding all empty sets among $C_\kay$ and $C_\infty$ gives a partition of $\NN$ into a countable collection of $\o$-small sets.
Since $\o$ is selective, we can choose a set $S\subset\NN$ such that
$S$ meets each set of the partition at one point and $\o(S)=1$.
Clearly the subsequence consisting of the $x_n$, $n\in S$
converges to $x_\o$ in the usual sense.
\qeds

The following proposition 
is analogous to the statement that any sequence in a compact metric space 
has a convergent subsequence;
it can be proved in the same way.

\begin{thm}{Proposition}\label{prop:ultra/compact}
Let $\spc{X}$ be a compact metric space.
Then
any sequence of points $x_n$ in $\spc{X}$ has a unique $\o$-limit $x_\o$.

In particular, a bounded sequence of real numbers has a unique $\o$-limit. 
%
\end{thm}

The following lemma is an ultralimit analog of the Cauchy convergence test.

\begin{thm}{Lemma}\label{lem:X-X^w}
Let $x_n$ be a sequence of points in a complete metric space~$\spc{X}$. 
If for each subsequence $y_n$ of $x_n$, 
the $\o$-limit 
\[y_\o=\lim_{n\to\o}y_{n}\in \spc{X}\]
is defined and does not depend on the choice of a subsequence, 
then the sequence $x_n$ converges in the usual sense.
\end{thm}

\parit{Proof.} Assume the contrary. 
Then for some $\eps>0$, there is a subsequence $y_n$ of $x_n$ such that $\dist{x_n}{y_n}{}\ge\eps$ for all $n$.

It follows that $\dist{x_\o}{y_\o}{}\ge \eps$, a contradiction.
\qeds

\begin{thm}{Exercise}\label{ex:linear}
Recall that $\ell^\infty$ denotes the space of bounded sequences of real numbers.
Show that there is a linear functional $L\:\ell^\infty\to\RR$ such that
for any sequence $\bm{s}=(s_1,s_2,\dots)\in S$ the image $L(\bm{s})$ is a partial limit of $s_1,s_2,\dots$
\end{thm}

\begin{thm}{Exercise}\label{ex:ultrakatetov+}
Suppose that $f\:\NN\to\NN$ is a map such that 
\[\lim_{n\to\o}x_n=\lim_{n\to\o}x_{f(n)}\]
for any bounded sequence $x_n$ of real numbers.
Show that $f(n)=n$ for $\o$-almost all $n\in\NN$.
\end{thm}


\section{Ultralimits of spaces}\label{sec:Ultralimit of spaces}

Fix a selective ultrafilter $\o$ on the set of natural numbers.

Let $\spc{X}_n$ be a sequence of metric spaces.
Consider all sequences
$x_n\in \spc{X}_n$.
On the set of all such sequences,
define a pseudometric  by the formula
\[\dist{(x_n)}{(y_n)}{}
=
\lim_{n\to\o} \dist{x_n}{y_n}{}.
\eqlbl{eq:olim-dist}\]
Note that the $\o$-limit on the right-hand side is always defined 
and takes  value in $[0,\infty]$. 

Let $\spc{X}_\o$ be the corresponding metric space; 
that is, the underlying set of $\spc{X}_\o$ is formed by equivalence  classes of sequences of points $x_n\in\spc{X}_n$ 
defined by the relation
\[(x_n)\sim(y_n)
\quad \iff\quad 
\lim_{n\to\o} \dist{x_n}{y_n}{}=0,\]
and the distance is defined as in \ref{eq:olim-dist}.

The space $\spc{X}_\o$ is called the \index{ultralimit of spaces}\emph{$\o$-limit} of $\spc{X}_n$.
Typically  $\spc{X}_\o$ will denote the  
$\o$-limit of a sequence $\spc{X}_n$;
we may also write  
\[\spc{X}_n\to\spc{X}_\o\quad \text{as}\quad  n\to\o\quad \text{or}\quad \spc{X}_\o=\lim_{n\to\o}\spc{X}_n.\]

Given a sequence  $x_n\in \spc{X}_n$,
we will denote by $x_\o$ its equivalence class, which is a point in $\spc{X}_\o$;
in this case, we may write
\[x_n\to x_\o \quad \text{as}\quad  n\to\o\quad \text{or}\quad x_\o=\lim_{n\to\o} x_n.\]

\begin{thm}{Observation}\label{obs:ultralimit-is-complete}
The $\o$-limit of any sequence of metric spaces is complete. 
\end{thm}

\parit{Proof.}
Let $\spc{X}_n$ be a sequence of metric spaces and $\spc{X}_n\to\spc{X}_\o$ as $n\to\o$.
Choose a Cauchy sequence $x_n$ in $\spc{X}_\o$.
Passing to a subsequence, we can assume that $\dist{x_k}{x_{m}}{\spc{X}_\o}<\tfrac1{k}$ for any $k<m$.

Let us choose points $x_{n,m}\in\spc{X}_n$ such that for any fixed $m$ we have $x_{n,m}\to x_m$ as $n\to\o$.
Note that for any $k<m$ the inequality $\dist{x_{n,k}}{x_{n,m}}{}<\tfrac1{k}$ holds for $\o$-almost all $n$.
It follows that we can choose a nested sequence of sets 
\[\NN= S_1\supset S_2\supset\dots\] 
such that 
\begin{itemize}
\item $\o(S_m)=1$ for each $m$, 
\item $\bigcap_m S_m=\emptyset$, and
\item $\dist{x_{n,k}}{x_{n,l}}{}<\tfrac1{k}$ for $k<l\le m$ and $n\in S_m$.
\end{itemize}

Consider the sequence $y_n=x_{n,m(n)}$, where $m(n)$ is the largest value such that $n\in S_{m(n)}$.
Denote by $y_\o\in \spc{X}_\o$ the $\o$-limit of $y_n$.

Observe that $|y_m-x_{n,m}|<\tfrac1{m}$ for $\o$-almost all $n$.
It follows that $|x_m-y_\o|\le \tfrac1{m}$ for any $m$.
Therefore, $x_n\to y_\o$ as $n\to \infty$.
That is, any Cauchy sequence in $\spc{X}_\o$ converges.
\qeds


\begin{thm}{Observation}\label{obs:ultralimit-is-geodesic}
The $\o$-limit of any sequence of length spaces is geodesic. 
\end{thm}

\parit{Proof.}
If $\spc{X}_n$ is a sequence of length spaces, then for any sequence of pairs $(x_n, y_n)$ in $\spc{X}_n$ there is a sequence of $\tfrac1n$-midpoints $z_n$.

Let $x_n\to x_\o$, $y_n\to y_\o$, and $z_n\to z_\o$ as $n\to \o$.
Note that $z_\o$ is a midpoint between $x_\o$ and $y_\o$ in $\spc{X}_\o$.

By Observation~\ref{obs:ultralimit-is-complete}, $\spc{X}_\o$ is complete.
Applying Lemma~\ref{lem:mid>geod} we obtain the statement.
\qeds

A geodesic space $\spc{T}$ is called a \index{metric tree}\emph{metric tree} if any pair of points in $\spc{T}$ are connected by a unique geodesic,
and the union of any two geodesics $[xy]_{\spc{T}}$, and $[yz]_{\spc{T}}$ contain the geodesic $[xz]_{\spc{T}}$.
The latter means that any triangle in $\spc{T}$ is a tripod;
that is, for any three points $x$, $y$, and $z$ there is a point $p$ such that 
\[[xy]\cup[yz]\cup[zx]=[px]\cup[py]\cup[pz].\]

\begin{thm}{Exercise}\label{ex:Asym(Lob)}
Let $\spc{T}$ be a metric component of the ultralimit of $\Lob2n$ as $n\to\o$.

\begin{subthm}{ex:Asym(Lob):metric-tree}
Show that $\spc{T}$ is a complete metric tree.
\end{subthm}

\begin{subthm}{ex:Asym(Lob):homogeneous}
Show that $\spc{T}$ is homogeneous; that is, given two points $s,t\in \spc{T}$ there is an isometry of $\spc{T}$ that maps $s$ to $t$.
\end{subthm}

\begin{subthm}{ex:Asym(Lob):continuum}
Show that $\spc{T}$ has \index{degree}\emph{continuum degree} at any point;
that is, for any point $t\in \spc{T}$ the set of connected components of the complement $\spc{T}\setminus\{t\}$ has cardinality continuum.
\end{subthm}

\end{thm}

\parbf{Ultrapower.} If all the metric spaces in a sequence are identical, $\spc{X}_n\z=\spc{X}$, 
the $\o$-limit 
$\lim_{n\to\o}\spc{X}_n$
is denoted by $\spc{X}^\o$
and called the \index{ultrapower} $\o$-power of $\spc{X}$.
 
By Theorem~\ref{thm:ultra-GH},
there is a distance-preserving map
$\iota\:\spc{X}\hookrightarrow \spc{X}^\o$, where $\iota(y)$ is the equivalence class of the constant sequence $y_n=y$. 

The image $\iota(\spc{X})$ might be a proper subset of $\spc{X}^\o$.
For example, $\RR^\o$ has pairs of points at distance $\infty$ from each other, while each metric component of $\RR^\o$ is isometric to $\RR$.

According to Theorem~\ref{thm:ultra-GH}, 
if $\spc{X}$ is compact then $\iota(\spc{X})=\spc{X}^\o$;
in particular, $\spc{X}^\o$ is isometric to $\spc{X}$.
If $\spc{X}$ is proper, then $\iota(\spc{X})$ forms a metric component of~$\spc{X}^\o$.

The embedding $\iota$ allows us to treat $\spc{X}$ as a subset of its ultrapower~$\spc{X}^\o$. 

\begin{thm}{Observation}\label{obs:ultrapower-is-geodesic}
Let $\spc{X}$ be a complete metric space. 
Then $\spc{X}^\o$ is a geodesic space if and only if $\spc{X}$ is a length space.
\end{thm}

\parit{Proof.}
Assume $\spc{X}^\o$ is geodesic space.
Then any pair of points $x,y\in \spc{X}$ has a midpoint $z_\o\in\spc{X}^\o$.
Fix a sequence of points $z_n\in  \spc{X}$ such that $z_n\to z_\o$ as $n\to \o$.

Note that 
$\dist{x}{z_n}{\spc{X}}\to \tfrac12\cdot \dist{x}{y}{\spc{X}}$
and 
$\dist{y}{z_n}{\spc{X}}\to \tfrac12\cdot \dist{x}{y}{\spc{X}}$
as 
$n\to\o$.
In particular, for any $\eps>0$, the point $z_n$ is an $\eps$-midpoint between $x$ and $y$ for $\o$-almost all $n$.
It remains to apply Lemma~\ref{lem:mid>geod}.

The if part follows from Observation~\ref{obs:ultralimit-is-geodesic}.
\qeds

Note that the proof above together with Lemma~\ref{lem:X-X^w} imply the following:

\begin{thm}{Corollary}\label{cor:two-geodesics-in-ultrapower}
Assume $\spc{X}$ is a complete length space 
and $p,q\in\spc{X}$ cannot be joined by a geodesic in $\spc{X}$.  
Then there are at least continuum distinct geodesics between $p$ and $q$ 
in the ultrapower $\spc{X}^\o$.
\end{thm}

\begin{thm}{Exercise}\label{ex:isom-ultrapower}
Let $\spc{X}$ be a countable set with discrete metric;
that is $\dist{x}{y}{\spc{X}}=1$ if $x\ne y$.
Show that 

\begin{subthm}{ex:isom-ultrapower:no}
$\spc{X}^\o$ is not isometric to $\spc{X}$.
\end{subthm}

\begin{subthm}{ex:isom-ultrapower:yes}
$\spc{X}^\o$ is  isometric to $(\spc{X}^\o)^\o$.
\end{subthm}

\end{thm}

\begin{thm}{Exercise}\label{ex:ultrapower(ultrapower)}
Given a nonprincipal ultrafilter $\o$, construct an ultrafilter $\o_1$ such that 
\[\spc{X}^{\o_1}\iso(\spc{X}^\o)^\o\]
for any metric space~$\spc{X}$.

\end{thm}

\begin{thm}{Exercise}\label{ex:notproper-limit}
Construct a proper metric space $\spc{X}$ such that $\spc{X}^\o$ is not proper;
that is, there is a point $p\in \spc{X}^\o$ and $R<\infty$ such that the closed ball $\cBall[p,R]_{\spc{X}^\o}$ is not compact.
\end{thm}

{\sloppy

\section{Ultralimits of sets}

Let $\spc{X}_n$ be a sequence of metric spaces and $\spc{X}_n\to \spc{X}_\o$
as $n\to \o$.

For a sequence of sets $\Omega_n\subset \spc{X}_n$,
consider the maximal set $\Omega_\o\subset \spc{X}_\o$ such that 
for any $x_\o\in\Omega_\o$ and any sequence $x_n\in\spc{X}_n$ such that $x_n\to x_\o$ as $n\to \o$, we have $x_n\in\Omega_n$ for $\o$-almost all $n$.

The set $\Omega_\o$ is called the  \index{ultralimit of sets}\emph{open $\o$-limit} of $\Omega_n$;
we could also write  $\Omega_n\to \Omega_\o$ as $n\to\o$ or $\Omega_\o=\lim_{n\to\o}\Omega_n$. 

{\sloppy

Applying Observation~\ref{obs:ultralimit-is-complete} to the sequence of complements $\spc{X}_n\backslash \Omega_n$, we see that $\Omega_\o$ is open for any sequence $\Omega_n$.

This definition can be applied to arbitrary sequences of sets,
but we will apply it only for sequences of open sets.

}

\section{Ultralimits of functions}\label{sec:Ultralimits of functions}

Recall that a family of submaps (see section \ref{sec:submaps}) between metric spaces $\{f_\alpha\co \spc{X}\subto\spc{Y}\}_{\alpha\in\mathcal A}$ is called \index{equicontinuous family}\emph{equicontinuous} if for any $\eps>0$ there is $\delta>0$ such that for any $p,q\in\spc{X}$ with $\dist{p}{q}{}<\delta$ and any $\alpha\in\mathcal A$ we have $\dist{f_\alpha(p)}{f_\alpha(q)}{}<\eps$.

Let $f_n\:\spc{X}_n\subto\RR$ be a sequence of subfunctions.

Set $\Omega_n=\Dom f_n$.
Consider the open $\o$-limit set $\Omega_\o\subset \spc{X}_\o$ of $\Omega_n$.

Assume there is a subfunction $f_\o\:\spc{X}_\o\subto\RR$
that satisfies the following conditions: 
(1) $\Dom f_\o=\Omega_\o$, (2) if $x_n\to x_\o\in \Omega_\o$ for a sequence of points $x_n\in\spc{X}_n$, then $f_n(x_n)\to f_\o(x_\o)$ as $n\to\o$.
In this case, the subfunction $f_\o\:\spc{X}_\o\to\RR$ is said to be the $\o$-limit of $f_n\:\spc{X}_n\to\RR$.

The following lemma gives a mild condition on a sequence of functions $f_n$
guaranteeing the existence of its $\o$-limit.

\begin{thm}{Lemma}
Let $\spc{X}_n$ be a sequence of metric spaces
and $f_n\:\spc{X}_n\subto\RR$ be a sequence of subfunctions.

Assume that  for any positive integer $\kay$, there is an open set $\Omega_n(\kay)\subset \Dom f_n$
such that the restrictions $f_n|_{\Omega_n(\kay)}$ are uniformly bounded and equicontinuous
and the open $\o$-limit of $\Omega_n(n)$ coincides with the open $\o$-limit of $\Dom f_n$.
Then the $\o$-limit $f_\o$ of $f_n$ is defined;
moreover $f_\o$ is a continuous subfunction.

In particular, if the functions $f_n$ are uniformly bounded and equicontinuous, then its $\o$-limit $f_\o$ is defined, bounded and uniformly continuous.
\end{thm}

The proof is straightforward.

{\sloppy

\begin{thm}{Exercise}\label{ex:nonconvex-limit}
Construct a sequence of compact length spaces 
$\spc{X}_n$  
with a converging sequence of $\Lip$-Lipschitz concave (see Definition \ref{def:lam-convex}) functions $f_n\:\spc{X}_n\to\RR$ such that
the $\o$-limit $\spc{X}_\o$ of $\spc{X}_n$ is compact
and the $\o$-limit $f_\o\:\spc{X}_\o\to\RR$ of $f_n$ is not concave.
\end{thm}

}

If $f\:\spc{X}\subto\RR$ is a subfunction, 
the $\o$-limit of the constant sequence $f_n=f$ is called the $\o$-power of $f$ and is denoted by $f^\o$.
So
\[f^\o\:\spc{X}\subto\RR,\quad f^\o(x_\o)=\lim_{n\to\o} f(x_n).\]

Evidently, if $f^\o$ is defined, then $f$ is continuous.

Recall that we treat $\spc{X}$ as a subset of its $\o$-power $\spc{X}^\o$.
Note that $\Dom f=\spc{X}\cap \Dom f^\o$.
Moreover, 
if $\oBall(x,\eps)_{\spc{X}}\subset \Dom f$
then $\oBall(x,\eps)_{\spc{X}^\o}\subset \Dom f^\o$.

