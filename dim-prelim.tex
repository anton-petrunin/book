\chapter{Dimension theory}

In Section~\ref{sec:prelim:dim}, we give definitions of different type of dimension-like invariants of metric spaces and state general relations between them.


\section{Definitions}\label{sec:prelim:dim}

The proof of most of the statements in this section can be found in the book of Witold Hurewicz and Henry Wallman \cite{top-dim}, 
the rest follows directly from the definitions.

\begin{thm}{Hausdorff dimension}
\label{def:HausDim}\index{dimension!Hausdorff dimension}
Let $\spc{X}$ be a metric space. 
Its Hausdorff dimension is defined as
\[\HausDim\spc{X}=\sup\set{\alpha\in\RR}{\HausMes_\alpha(\spc{X})>0},\]
 where $\HausMes_\alpha$ denotes $\alpha$-dimensional Hausdorff measure. %???it is defined in Section~\ref{sec:mes+balls}.
\end{thm}

Let $\spc{X}$ be a metric space and $\{V_\beta\}_{\beta\in\IndexSet[2]}$
 be its open cover.
Let us recall two notions in general topology:
\begin{itemize}
\item The \emph{order}\index{order} of $\{V_\beta\}$ is supremum of all integers $n$ such that there is a collection of $n+1$ elements of $\{V_\beta\}$ with nonempy intersection.
\item An open cover $\{W_\alpha\}_{\alpha\in\IndexSet}$ of $\spc{X}$ is called \emph{refinement}\index{refinement} of  $\{V_\beta\}_{\beta\in\IndexSet[2]}$ if for any $\alpha\in\IndexSet$ there is $\beta\in\IndexSet[2]$ such that $W_\alpha\subset V_\beta$.
\end{itemize}

\begin{thm}{Topological dimension}
\label{def:TopDim}
\index{dimension!topological dimension}
Let $\spc{X}$ be a metric space. 
The topological dimension of $\spc{X}$ is defined as the minimum of nonnegative integers $n$, 
such that for any finite open cover of $\spc{X}$ there is a finite open refinement with order~$n$.

If no such $n$ exists, the topological dimension of $\spc{X}$ is infinite.

The topological dimension of $\spc{X}$ will be denoted by $\TopDim\spc{X}$.
\end{thm}

The invariants satisfying the following two statements \ref{dim-axiom-norm} and \ref{dim-axiom-sigma} are commonly called ``dimension'';
by that reason we call them axioms.

\begin{thm}{Normalization axiom}
\label{dim-axiom-norm} For any $m\in\ZZ_{\ge0}$,
\[\TopDim\EE^m=\HausDim\EE^m=m.\]

\end{thm}

\begin{thm}{Cover axiom}\label{dim-axiom-sigma} 
If $\{A_n\}_{n=1}^\infty$ is a countable closed cover of $\spc{X}$, then
\begin{align*}
\TopDim \spc{X}&=\sup\nolimits_n\{\TopDim A_n\},
\\
\HausDim \spc{X}&=\sup\nolimits_n\{\HausDim A_n\}.
\end{align*}

\end{thm}

\parbf{On product spaces.} Let us mention that the following two inequalities
\begin{align*}
\TopDim  (\spc{X}\times\spc{Y})
&\le 
\TopDim \spc{X}+ \TopDim\spc{Y},
\intertext{and}
\HausDim (\spc{X}\times\spc{Y})
&\ge 
\HausDim \spc{X}+ \HausDim\spc{Y},
\end{align*}
hold for any pair of metric spaces $\spc{X}$ and $\spc{Y}$.

These inequalities in might be strict.
For the topological dimension it holds for apair of Pontryagin surfaces constructed in \cite{pontyagin-surface}.
For Hausdorff dimension, an example was constructed by Abram Besicovitch and Pat Moran in \cite{besicovitch-moran}.

\medskip
 
The following theorem follows from \cite[theorems V 8 and VII 2]{top-dim}.

\begin{thm}{Szpilrajn's theorem}\label{thm:szpilrajn} 
Let $\spc{X}$ be a separable metric space.
Assume $\TopDim\spc{X}\ge m$. Then $\HausMes_m \spc{X}>0$.

In particular, 
$\TopDim\spc{X}\le\HausDim\spc{X}$.
\end{thm}

In fact it is true that for any separable metric space $\spc{X}$ we have
\[\TopDim\spc{X}=\inf\{\HausDim\spc{Y}\},\]
where infimum is taken over all metric spaces $\spc{Y}$  homeomorphic to $\spc{X}$.

\begin{thm}{Definition}
Let $\spc{X}$ be a metric space
and $F\:\spc{X}\to\RR^m$ be  a continuous map.
A point $\bm{z}\in \Im F$ is called a  \emph{stable value}\index{stable value} of $F$
if there is $\eps>0$ such that $\bm{z}\in\Im F'$ 
for any $\eps$-close continuous map $F'\:\spc{X}\to\RR^m$;
that is, $|F'(x)-F(x)|<\eps$ for all $x\in \spc{X}$.
\end{thm}

The following theorem follows from \cite[theorems VI 1$\&$2]{top-dim}.
(This theorem also holds for non-separable metric spaces 
\cite{nagata} or \cite[3.2.10]{engelking}). %??? reference formating

\begin{thm}{Stable value theorem}\label{thm:stable-value}
Let $\spc{X}$ be a separable metric space.
Then $\TopDim\spc{X}\ge m$ if and only if there is a map $F\:\spc{X}\to\RR^{m}$ with a stable value.
\end{thm}



\begin{thm}{Proposition}\label{thm:HausDim+Lip}
Suppose $\spc{X}$ and $\spc{Y}$ are metric spaces 
and $\map \:\spc{X}\to \spc{Y}$ satisfies
\[\dist{\map (x)}{\map (x')}{}\ge \eps\cdot\dist[{{}}]{x}{x'}{}\]
for some fixed $\eps>0$ and any pair $x,x'\in \spc{X}$.
Then
\[\HausDim \spc{X}\le \HausDim \spc{Y}.\]

In particular, if there is a Lipschitz onto map $\spc{Y}\to \spc{X}$, then  
\[\HausDim \spc{X}\le \HausDim \spc{Y}.\]

\end{thm}

\section{Linear dimension}

In addition to $\HausDim$ and $\TopDim$, 
we will use yet another one, which we call \emph{linear dimension}.

Recall that \emph{cone map} is a map between cones respecting the cone multiplication.

\begin{thm}{Definition of linear dimension}\label{def:lin-dim}\index{dimension!Linear dimension}
Let $\spc{X}$ be a metric space. 
The \emph{linear dimension} of $\spc{X}$ (denoted by $\LinDim\spc{X}$\index{$\LinDim$}) is defined exact upper bound on $m\in\ZZ_{\ge0}$
such that there is an isometric cone embedding $\EE^m\hookrightarrow \T_p\spc{X}$
for some $p\in \spc{X}$; here $\EE^m$ denotes $m$-dimensional Euclidean space 
and $\T_p\spc{X}$ denotes tangent space of $\spc{X}$ at $p$ (defined in Section~\ref{sec:tangent-space+directions}).
\end{thm}

Note that $\LinDim$ takes values in $\ZZ_{\ge0}\cup\{\infty\}$.
 
For general metric spaces, $\LinDim$ has almost no  relations to $\HausDim$ and $\TopDim$.
Also, $\LinDim$ does not satisfy the cover axiom
 (\ref{dim-axiom-sigma}).
For $\LinDim$, an inverse of the product inequality holds; that is,
\[\LinDim(\spc{X}\times \spc{Y})
\ge
\LinDim\spc{X}+ \LinDim\spc{Y}
\eqlbl{eq:inverse-product-axiom}\] 
for any two metric spaces $\spc{X}$ and $\spc{Y}$; 
that is easy to check. 
According to the following exercise, the inequality \ref{eq:inverse-product-axiom} might be strict for some spaces \cite{schroeder-foetch}.

\begin{thm}{Exercise}\label{ex:schroeder-foetch}
Construct two norms on $\RR^{10}$ such that 
non of the corresponding metric spaces $\spc{X}$ and $\spc{Y}$
have a isometric copy of $\EE^2$ but
$\spc{X}\times\spc{Y}$ has isometric copy of $\EE^{10}$.
\end{thm}

The linear dimension will be applied only to  Alexandrov spaces and to their open subsets (in both cases of curvature bounded below and above).
As we shall see, in all these cases, $\LinDim$  behaves nicely and  is easy to work with.

\parbf{Remarks.}
Linear dimension was first introduced by Conrad Plaut in \cite{plaut:survey}
under the name \emph{local dimension}\index{local dimension}\index{dimension!local dimension}. 
\emph{Geometric dimension}\index{geometric dimension}\index{dimension!geometric dimension} introduced in by Bruce Kleiner in \cite{kleiner} is closely related; 
it coincides %should we explain why???
 with the linear dimension for $\Alex{}$ and $\CAT{}$ spaces.

One could modify the definition of linear dimension by taking an arbitrary $n$-dimensional Banach space instead of the Euclidean $n$-space.
Such a definition makes more sense for general metric spaces.
This dimension is additive with respect to the direct product (that is, we always have equality in \ref{eq:inverse-product-axiom}). 

For Alexandrov spaces (either CBB or CAT) this modification is equivalent to the definition we use in this book.
