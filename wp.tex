%%!TEX root = the-wp.tex

\chapter{Curvature bounds of warped products}



%%%%%%%%%%%%%%%%%%%%%%%%%%%%%%%%%%%%%%%%%%%%%%%%%%%%%%%%%%%%%%%%%%%%%%%%%%%%%%%%%%%%%%%%%%%
\section{Fiber comparison for warped products.}\label{sec:fiber-comparison}


Let $\spc{B}$ and $\spc{F}$ be  $\Cat{}{\Kappa}$) and  $\Cat{}{\Kappa_\spc{F}}$),
respectively, and
$f:\spc{B} \to \RR_{\ge 0}$\, be
an $\FK$-convex function that is Lipschitz on bounded sets, where
$\Kappa\le 0$. 
%If $X=f^{-1}(0)$ is nonempty and $\Kappa_F>0$, suppose $f'(0^+)\ge \sqrt{\Kappa_F}$ at the footpoint of every minimizer to $X$.
If $X=f^{-1}(0)$ is nonempty, any minimizer
in $\spc{B}\times_f\spc{F}$ between two points not in $X$, whose projections to
$\spc{F}$ are $\ge \pi/\sqrt{\Kappa_\spc{F}}$ apart,  intersects $X$ and consists of
two horizontal segments
joined by a segment in $X$.


Let $\spc{B}$ and $\spc{F}$ be Alexandrov spaces with
CBB by $\Kappa$ and $\Kappa_\spc{F} $ respectively. Suppose $f:\spc{B}\to \RR_{\ge 0}$
is a locally Lipschitz $\FK$-concave function vanishing on
$\partial \spc{B}$.  
If $\partial \spc{B}=f^{-1}(0)$ is nonempty, any minimizer in
$\spc{B}\times_f\spc{F}$ joining two points not in $\partial \spc{B}$,
and intersecting $\partial \spc{B}=f^{-1}(0)$, consists of two horizontal
segments whose
projections to $\spc{F}$ are $\pi/\sqrt{\Kappa_\spc{F}}$ apart, joined by a
point of $\partial \spc{B}$.




We show that a curvature bound for
$\spc{B}\times_f\spc{F}$ can be obtained by comparison to $\spc{B}\times_f\Lob{}{\kappa}$, where $\kappa$ is a curvature bound for the fiber space $\spc{F}$. %This method was introduced by Berestovskii in the case of cones (section \ref{cones-joins}). It was used for arbitrary positive warping function on a $1$-dimensional base by Chen \cite{C1, C2}, and extended to general base and warping function in \cite[Theorem 4.1]{AB1}.

\begin{thm}{Lemma (CBA fiber comparison)}
\label{thm:cba-fiber-comparison}
Suppose $\spc{B}\in \Cat{}{\kappa}$, 
%$\spc{F}\in \Cat{}{\kappa_\spc{F}}$, and  $f:\spc{B}\to\RR_{\ge 0}$, then $\spc{B}\times_f\spc{F}\in \Cat{}{\kappa}$ if and only if $\spc{B}\times_fS_{\kappa_\spc{F}}\in \Cat{}{\kappa}$.  
$\spc{F}_i\in \Cat{}{\kappa(i)}$, and  $f_i:\spc{B}\to\RR_{\ge 0}$.  If  $\spc{B}\times_{(f_1,\ldots,f_k)}(\Lob{}{\kappa(1)}\times\ldots\times \Lob{}{\kappa(k)})\in \Cat{}{\kappa}$, then
$\spc{B}\times_{(f_1,\ldots,f_k)}(\spc{F}_1\times\ldots\times \spc{F}_k)\in \Cat{}{\kappa}$.
 \end{thm} 

\begin{thm}{Lemma (CBB fiber comparison)}
\label{thm:cbb-fiber-comparison}
Suppose $\spc{B}\in \CBB{}{\kappa}$,
%$\spc{F}\in \CBB{}{\kappa_\spc{F}}$, and  $f:\spc{B}\to\RR_{\ge 0}$. Then $\spc{B}\times_f\spc{F}\in \CBB{}{\kappa}$ if and only if $\spc{B}\times_fS_{\kappa_\spc{F}}\in \CBB{}{\kappa}$.
$\spc{F}_i\in \CBB{}{\kappa(i)}$, and  $f_i:\spc{B}\to\RR_{\ge 0}$.   If $\spc{B}\times_{(f_1,\ldots,f_k)}(\Lob{}{\kappa(1)}\times\ldots\times \Lob{}{\kappa(k)})\in \CBB{}{\kappa}$, then
$\spc{B}\times_{(f_1,\ldots,f_k)}(\spc{F}_1\times\ldots\times \spc{F}_k)\in \CBB{}{\kappa}$.
 \end{thm}


%For parts (\ref{cbb-join-curvature}), write 
%\begin{align}\notag
%\Join_\kappa (\spc{M},\spc{N})
%&
%= [0,\varpi_\kappa/2]\times_{(\sn_\kappa,cs_\kappa)}(\spc{M}\times \spc{N})
%\\
%&
%\notag
%=\bigl([0,\varpi_\kappa/2]\times_{sn_\kappa}\spc{M}\bigr)\times_{cs_\kappa} \spc{N}.
%\end{align}


When the warping functions are strictly positive, the fiber comparison theorems depend on Lemma \ref{lem:fiber-independence}.  In addition, generalized cone points, i.e., vanishing points of the warping function, must be addressed.  

\noindent\emph{Proof of Proposition  \ref{prop:modelredA}.}

Since $X=f^{-1}(0)$ is convex, any minimizer
$\gamma$ intersects $X$ in at most an interval.  Therefore $\gamma$
consists of two
horizontal segments joined by a segment in $X$.

\parit{Proof.}Part (1).We  may normalize so $\Kappa_\spc{F}=1$.  By way of
contradiction, assume there
is a minimizer $\gamma$  in $\spc{B}\times_f\spc{F}$ whose
endpoints project to points of the fiber at distance $\ge\pi$ and
which does not intersect $X$.  By Lemma \ref{lem:fiberindep}, there
is such a minimizer also in $\spc{B}\times_f\RR$. Thus it suffices to assume
$\spc{F}=\RR$.

Since $\gamma$ has a subsegment whose  projection to
$\RR$ has length $\pi$, there is a minimizer in
$\spc{B}\times_fS^1=(\spc{B}\times_f\RR)/Z$ not intersecting $X$ and whose
endpoints project to opposite points of $S^1$.  But then there are
two minimizers in $\spc{B}\times_fS^1$  with those endpoints, in
contradiction to Theorem \ref{thm:1dCBA}, which states that
$\spc{B}\times_fS^1$ is $\Cat{}{\Kappa}$).
\qeds

[Part (2).] We show that any triangle
$\triangle=(\triangle_\spc{B},\triangle_\spc{F})$ in
$\spc{B}\times_f\spc{F}$ is $\Kappa$-thin:

(i)  Suppose $\triangle$ does not
intersect $X$ and satisfies $
\triangle_\spc{F}<\pi/\sqrt{\Kappa_\spc{F}}$.  Let $\triangle_\spc{F}'$ be a comparison
triangle in $S_{\Kappa_\spc{F}}$ for
$\triangle_\spc{F}$ with the same parametrization of sides.  By Lemma
\ref{lem:fiberindep}, $\triangle' =
(\triangle_\spc{B},\triangle_\spc{F}')$ is a triangle in $\spc{B}\times_fS_{\Kappa_\spc{F}}$ with the same
sidelengths as $\triangle$.   Let $\bar\gamma=(\bar\gamma_\spc{B},\bar\gamma_\spc{F})$ be a
transversal minimizer for
$\triangle'$.  Since $\triangle'$ is $\Kappa$-thin by hypothesis, it
suffices to construct a curve  in  $\spc{B}\times_f\spc{F}$ joining the
corresponding points
of $\triangle$ and no longer than $\bar\gamma$.  This we do as
follows.  If $\bar\gamma$ does not intersect $X$, then $\bar\gamma_\spc{F}$
is a reparametrized transversal of $\triangle_\spc{F}'$, by Lemma
\ref{lem:wpgeodesics} (a), and so is not shorter than the
corresponding transversal $\gamma_\spc{F}$ of $\triangle_\spc{F}$.  In this case,
pair $\bar\gamma_\spc{B}$ with a
reparametrization of $\gamma_\spc{F}$ proportional to $\bar\gamma_\spc{F}$.
If $\bar\gamma$ intersects $X$, pair $\bar\gamma_\spc{B}$ with the curve
that is indeterminate on the
same interval as $\bar\gamma_\spc{F}$ and takes as its constant values, the
two points
corresponding to the endpoints of $\bar\gamma_\spc{F}$.  Thus $\triangle$
is $\Kappa$-thin.

(ii)  Suppose $\triangle$ has a vertex
in $X$. Clearly $\triangle$ is $\Kappa$-thin if it has a side in $X$,
since then $\triangle$ lies in an isometric copy of $\spc{B}$.  Otherwise,
a vertex pair lies outside $X$.  If the side of $\triangle$  joining
this pair intersects $X$, then the image of $\triangle_\spc{F}$ is their
projections to $\spc{F}$, and otherwise it is the segment joining their
projections to $\spc{F}$.  By Lemma \ref{lem:fiberindep}, we may construct
a triangle $\triangle' = (\triangle_\spc{B},\triangle_\spc{F}')$ in $\spc{B}\times_f\RR$
whose sides have  the same domains, indeterminate subdomains and
lengths as those of $\triangle$, and where the images of
$\triangle_\spc{F}$ and $\triangle_\spc{F}'$ correspond:   either two points the
same distance apart, or segments of the same length.  Therefore the
distances between the projections to $\spc{F}$ and $\RR$ of any
corresponding pairs of points on $\triangle$ and $\triangle'$,
respectively, are equal when they are determinate.  By Lemma
\ref{lem:fiberindep}, it follows that we can construct transversal
minimizers for $\triangle'$ of the same length as the corresponding
ones for $\triangle$.  Since $\spc{B}\times_f\RR$ is $\Cat{}{\Kappa}$ by Theorem
\ref{thm:1dCBA}, $\triangle'$ is $\Kappa$-thin  and  hence so is $\triangle$.

(iii) If a side of $\triangle$ meets $X$, then $\triangle$ may be
subdivided into two triangles of type (ii), which are
therefore thin.  Therefore $\triangle$ is $\Kappa$-thin, by the basic
Alexandrov Lemma (see \cite[p.115]{BBI}).

It remains to deal with the cases where $\triangle$ does not meet $X$ and
$\triangle_\spc{F} \ge \pi/\sqrt{\Kappa_\spc{F}}$:

(iv)  If a minimizer from a vertex of $\triangle$ to a point of
the opposite side meets $X$, then $\triangle$ is subdivided into
two triangles of type (iii). Since each of these is $\Kappa$-thin, so
is $\triangle$.

(v) If the minimizers joining a vertex of $\triangle$ to the
points of its opposite side $\gamma_\spc{B}$ never meet $X$, then these
minimizers have length $<\pi/\sqrt{\Kappa_\spc{F}}$ by part ($1$ ).  Thus by subdividing
$\gamma_\spc{B}$, we may subdivide $\triangle$ into triangles of type
(i). Since each of these is $\Kappa$-thin, so is $\triangle$. \qeds

\noindent\emph{Proof of Proposition \ref{prop:modelredB}.}

A minimizer in $\spc{B}$ either lies in $\partial \spc{B}$
or intersects $\partial \spc{B}$ only at endpoint(s) \cite{Pm}. It follows that if
$\gamma$ is a minimizer in $\spc{B}\times_f\spc{F}$, any subsegment of  $\gamma$
with an endpoint in $\partial \spc{B}$ is horizontal, and either lies in
$\partial \spc{B}$ or intersects it only at one or both endpoints.
Therefore if $\gamma$ intersects $\partial \spc{B}$, it does so either in a
single point or only at the two endpoints.


Part (1) of this theorem  is the only result
from \S \ref{sec:stripsp} - \ref{sec:mainproofs} that will be used in
the proofs of Theorems
\ref{thm:keyCBB} and \ref{thm:keyCBA}, and hence the only one that
must be proved from first principles without reference to those
theorems:


\begin{proof}[Part (1).]

Set $\Kappa_\spc{F}=1$, and hence $\diam \spc{F}\le\pi$.   By Lemma
\ref{lem:fiberindep}, if there is a minimizer
$\gamma=(\gamma_\spc{B},\gamma_\spc{F})$ consisting of two horizontal segments
whose projections to $\spc{F}$ are at distance $<\pi$, then there is a
minimizer in $\spc{B}\times_f[-\pi/2,\pi/2]$ with the same property.
Therefore if suffices to show there are no such minimizers when $\spc{F}=\II$.

By hypothesis, if $p\in\partial \spc{B}$, then $f(p)=0$ and $Df_p\le 1$.
Let $\sn{\kappa}$ be the
solution of $y'' + ky = 0$ such that $\sn{\kappa}(0) = 0$, 
${\sn{\kappa}} '(0) =
1$. Then along any minimizer $\sigma$ in $\spc{B}$ starting from $\partial
\spc{B}$ we have $f(\sigma(s)) \le \sn{\Kappa}(s)$, an immediate consequence
of the $\FK$-concavity of $f$.

Let $\gamma:[0,L]\to \spc{B}\times_f\II$ be a minimizer intersecting
$\partial \spc{B}$, where  $q_1=\gamma_\spc{B}(0)$ and $q_2 = \gamma_\spc{B}(L)$ are
not in $\partial\spc{B}$. 
Let $\~{\spc{B}}$ be
the double of
$\spc{B}$ and $\~q_1, \~q_2$ the reflections of $q_1, q_2$ in the
other half of $\~{\spc{B}}$. 
Then a minimizer from $q_1$ to $\~q_2$
will consist of two segments $\sigma_1$ and $\~\sigma_2$
joined at some point $p \in \partial \spc{B}$; the reflections
$\~\sigma_1, \sigma_2$ of $\sigma_1, \~\sigma_2$ form together
a minimizer from $\~q_1$ to $q_2$. Clearly $\sigma_1$ and
$\sigma_2$ make up a shortest path among those connecting $q_1$ and
$q_2$ which pass through a point of $\partial \spc{B}$. See Figure
\ref{fig:double}.

%\marginpar{fig:double}
%\begin{figure}[h]
%\centering\epsfig{file=double.eps, width=1.5in}
%\caption{}
%\label{fig:double}
%\end{figure}


The direction space $\~\Sigma_p$ of $\~{\spc{B}}$ has two pairs of directions,
$q_1', \~q_2'$ and $q_2', \~q_1'$, with distance each
$\pi$, so that $\~\Sigma_p$ is a spherical suspension over each
pair. Hence there are unique minimizers $q_1^{\prime\frown} q_2'$,
$q_2^{\prime\frown} \~q_1'$, $\~q_1^{\prime\frown}
\~ q_2'$, $\~q_2^{\prime\frown} q_1'$ which together form
a periodic geodesic of length $2\pi$.  This geodesic has an
involutive isometry fixing its intersections with $\partial \Sigma_p$
(the direction space of $\partial \spc{B}$), and hence intersects $\partial
\Sigma_p$ in an opposite pair $v_1, v_2$.

An exception occurs when $q_1' = q_2'$, whereupon $p$ is the nearest
point in $\partial \spc{B}$ to $q_1$ and $q_2$ and the longer of $\sigma_1,
\sigma_2$ is a common minimizer $\sigma: [0, R] \to \spc{B}$ through them to
$p$.  Define a map $\Psi : Y \to \spc{B}
\times_f \II$, where $Y$ is a sector in $S_{\Kappa}$ with polar
coordinates $(r, \theta),\,\,0\le r\le
R,\,\,-\pi/2\le\theta\le\pi/2$, and $\Psi(r,\theta) =
(\sigma(r),\theta)$. Then $\Psi$ is nonexpanding since
$f(\sigma(r)) \le \sn{\Kappa}(r)$; but also $\Psi$
preserves distance along radial geodesics. The image of the minimizer in $Y$
from $(d(p,q_1), \gamma_\II(0))$ to $(d(p,q_2), \gamma_\II(L))$ is a path
connecting $(q_1, \gamma_\II(0))$ and $(q_2, \gamma_\II(L))$ and has
length $\le d(p,q_1) +
d(p,q_2)$, with equality only if $\ell = \pi$. This proves the
exceptional case.

For the general case we define another nonexpanding map $\Psi$ from a
constant-curvature half-ball $Y$ in $S^3_{\Kappa}$ to $\spc{B} \times_f \II$.
Represent $Y$
as $\mathbf{D} \times_\Phi \II$, where $\mathbf{D}$ is
a half-disk with radius $R = \max\{d(p,q_1), d(p, q_2)\}$ and
constant curvature $\Kappa$; when $\Kappa = 0$, take $\Phi = d_\Xi$, where $\Xi$
is the diameter of $\mathbf{D}$; generally, take $\Phi =
\sn{\Kappa}\circ d_\Xi$.  Set
$$\Psi = \exp_p\circ\Psi_\mathbf{D} \times \id:Y=\mathbf{D}
\times_\Phi \II \to \spc{B}\times_f\II.$$
Here, $\Psi_\mathbf{D}:\mathbf{D}\to \spc{B}$ is
the $\Kappa$-cone map into the $\Kappa$-cone over the geodesic in $\Sigma_p$
from $v_1$ to $v_2$ through $q_1'$ and $q_2'$; this first composite is
an isometric injection. (When $\Kappa =0, -1, 1$, these $\Kappa$-cones are
the last three on the list preceding Example \ref{examp:conepts} in
\S \ref{ss:applications}). The map $\exp_p$ is the \emph{gradient
exponential map} defined in \cite{PP} and proved there to be
nonexpanding and isometric along cone radii which correspond to
minimizers to $p$ such as $\sigma_1, \sigma_2$.

We show that $\Psi$ retains the nonexpanding property. It also retains the
isometry for those horizontal minimizers from the center which are
mapped to minimizers from $p$. Consider the functions $f\circ \exp_p
\circ \Psi_\mathbf{D}$ and $\Phi$ on $\mathbf{D}$. For $q \in
\mathbf{D}$ there is a minimizer $\rho$ from $q$ to the diameter of
$\mathbf{D}$, and, since $\Psi_\mathbf{D}$ and $\exp_p$ map boundaries
into boundaries, the image of $\rho$ in $\spc{B}$ is a curve no longer
than $d_\Xi(q)$ from $\exp_p\circ\Psi_\mathbf{D}(q)$ to $\partial \spc{B}$. That is,
$d_{\partial \spc{B}}(\exp_p\circ\Psi_\mathbf{D}(q)) \le d_\Xi(q)$.  Since
$\sn{\Kappa}$ is increasing
for all arguments under consideration,
$f(\exp_p\circ\Psi_\mathbf{D}(q)) \le \sn{\Kappa}(d_{\partial
\spc{B}}(\exp_p\circ\Psi_\mathbf{D}(q))\le\Phi(q).$
So for a curve in $Y$, the base component is contracted by $\exp_p
\circ\Psi_\mathbf{D}$, while the $\II$-component is mapped identically,
but
is shortened more by the warping function $f$ than it was by the
warping function $\Phi$.  Thus there is an image of a geodesic in $Y$
connecting
$(q_1,\gamma_\II(0)), (q_2,\gamma_\II(L))$ and no longer than $d(p,q_1) +
d(p,q_2)$,
with equality only if $\ell = \pi$.
\end{proof}

\begin{proof}[Part(2).]
Now we show that any triangle $\triangle=(\triangle_\spc{B},\triangle_\spc{F})$ in
$\spc{B}\times_f\spc{F}$ is $\Kappa$-thick.

(i) Suppose $\triangle$ does not intersect $\partial \spc{B}$, and
$\triangle_\spc{F}<2\pi/\sqrt{\Kappa_\spc{F}}$.  Then we may proceed as in case
(i) of Proposition
\ref{prop:modelredA} ($2$) to show that $\triangle$ has a comparison
triangle in $\spc{B}\times_fS_{\Kappa_\spc{F}}$ such that the
transversals of the former are no shorter than the corresponding
transversals of the latter. The only change is that, rather than
starting with a transversal of the model triangle, here we start with a
transversal of $\triangle$ and construct a curve in $\spc{B}\times_fS_{\Kappa_\spc{F}}$ that
is no longer.  Since triangles in $\spc{B}\times_fS_{\Kappa_\spc{F}}$ are $\Kappa$-thick,
so is $\triangle$.

(ii) Suppose $\triangle$ has a vertex in $\partial \spc{B}$.  If there are
two such vertices, then all sides of $\triangle$ are horizontal.
Thus the projection of $\triangle$ to $\spc{B}$ is a triangle with the same
sidelengths.  Since this triangle in $\spc{B}$ is $\Kappa$-thick, and the
projection is nonexpanding, $\triangle$ is $\Kappa$-thick.

If exactly one vertex of $\triangle$ lies in
$\partial \spc{B}$, the two adjacent sides are horizontal, hence have
constant projections to $\spc{F}$.  Since $\diam \spc{F} \le \pi/\sqrt{\Kappa_\spc{F}}$, we
may choose two points in $S_{\Kappa_\spc{F}}$ the
same distance apart.  Then, as in case (ii) of Proposition
\ref{prop:modelredA} ($2$), we may construct a comparison triangle
$\triangle'$  in $\spc{B}\times_fS_{\Kappa_\spc{F}}$ whose transversal distances are
equal to those of $\triangle$.  Since $\triangle'$ is $\Kappa$-thick, so
is $\triangle$.

(iii) Suppose $\triangle$ does not intersect $\partial \spc{B}$ and
$\triangle_\spc{F}=2\pi/\sqrt{\Kappa_\spc{F}},\, \Kappa_\spc{F}>0$. Recall that no triangle
in $\spc{F}$ can have larger perimeter \cite{BGP}. If two vertices are
$\pi/\sqrt{\Kappa_\spc{F}}$ apart, $\spc{F}$ is a spherical susension with poles at
these vertices (see \cite[p.369]{BBI}).  Otherwise, $\triangle_\spc{F}$
is spanned in $\spc{F}$ by an isometric copy of a hemisphere in $S_{\Kappa_\spc{F}}$
(see \cite[p.836]{Pl}). Thus in either case, $\triangle_\spc{F}$ is spanned
by an isometric copy of a sector of a hemisphere in $S_{\Kappa_\spc{F}}$.  By
Lemma \ref{lem:fiberindep}, $\triangle$ has a comparison triangle
$\triangle'$ in $\spc{B}\times_fS_{\Kappa_\spc{F}}$ with the same transversal
distances. Since $\triangle'$ is $\Kappa$-thick, so is $\triangle$.

(iv)  The only remaining possibility is that no vertex of $\triangle$
lies in $\partial \spc{B}$ but some side  intersects $\partial \spc{B}$. Since
$\partial \spc{B}\ne\emptyset$, $\Kappa_\spc{F}>0$ by hypothesis.  In this case, the
projection to $\spc{F}$ of the side intersecting $\partial \spc{B}$ consists of
two points at distance $\pi$, by part ($1$).  Therefore $\spc{F}$ is a
spherical suspension with poles at these points.  The image of
$\triangle_\spc{F}$ is either the poles or two segments whose union is  a
minimizer of length $\pi$ joining them.  Let $\triangle_\spc{F}'$ map into
$S_{\Kappa_\spc{F}}$ with the same domains and indeterminate subdomains of the
sides as $\triangle_\spc{F}$, and a corresponding image.  By Lemma
\ref{lem:fiberindep}, $\triangle'=(\triangle_\spc{B},\triangle_\spc{F}')$ is a
model triangle in  $\spc{B}\times_fS_{\Kappa_\spc{F}}$ for $\triangle$ and has the
same transversal distances.  Since $\triangle'$ is $\Kappa$-thick, so is
$\triangle$.
            
\end{proof}

\section{Warped product characterization theorem.}\label{sec:metric-min-surfaces}



