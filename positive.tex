%%!TEX root = all.tex
\chapter{Positively curved spaces}

\section{Sphere theorems}

\begin{thm}{Diameter-suspension  theorem}
Let $\spc{L}$ be an $m$-dimensional complete length $\Alex{1}$ space.
Assume $\diam \spc{L}>\tfrac\pi2$ and $p,q\in \spc{L}$ be two points such that $\dist{p}{q}{}=\diam \spc{L}$.
then $\spc{L}$ is homeomorphic to $\Susp\Sigma_p$.

Moreover $\Sigma_p$ is homeomorphic to $\Sigma_q$.
\end{thm}

\parit{Proof.}
Clearly for any $x\in \spc{L}\backslash\{p,q\}$ we have $\mangle\hinge x p q\ge\angk{1}{x}{p}{q}>\tfrac\pi2$.
Thus $\distfun{p}{}{}$ has only two critical points $p$ and $q$.
Thus according to Morse lemma, $\spc{L}$ is homeomorphic to $\Susp Z_t$, where $Z_t=\distfun[-1]{p}{}{}(t)$, $0<t<\dist{p}{q}{}$.
Finally, according to ???, for all sufficiently small $t$ we have $\Sigma_p$ is homeomorphic to $Z_t$.

To prove the second statement ???.
\qeds


\begin{thm}{Sphere theorem}\label{thm:sphere}
Assume $\spc{L}$ be an $m$-dimensional complete length $\Alex{1}$ space, $\spc{L}$ does not have proper extremal subsets and $\diam \spc{L}>\tfrac\pi2$.
Then $\spc{L}$ is homeomorphic to $\SS^m$. 
\end{thm}

\parit{Proof.} Assume $p,q\in \spc{L}$ realize the diameter of $\spc{L}$.
Since  $\spc{L}$ has no extremal subsets, 
from Example~\ref{ex:t-cone}, 
it  follows that a small spherical neighborhood
of $p\in \spc{L}$ is homeomorphic to $\RR^m$. 
From angle comparison, $\distfun{p}{}{}$ has only two critical points $p$ and $q$. 
Therefore, this theorem follows from the Morse lemma (\ref{lem:morse}) applied to $\distfun{p}{}{}$. \qeds

\begin{thm}{Radius-sphere theorem}\label{thm:rad-sphere}
Assume $\spc{L}$ be an $m$-dimensional complete length $\Alex{1}$ space and $\rad \spc{L}>\tfrac\pi2$.
Then $\spc{L}$ is homeomorphic to a sphere. 
\end{thm}

We give two proofs, both are build on the following lemma.
The first a direct proof;
the second is a reduction to theorem \ref{thm:sphere}.

\begin{thm}{Lemma}\label{lemma:rad-big-U_p}
Let $\spc{L}$  be an $m$-dimensional complete length $\Alex{1}$ space and  $\rad\spc{L}>\tfrac\pi2$. 
Then 
\[\rad\Sigma_p> \tfrac\pi2\] 
for any $p\in \spc{L}$.
\end{thm}

\parit{Proof.} Assume $\rad\Sigma_p\le\tfrac\pi2$ for some $p\in \spc{L}$.
Let $\xi\in \Sigma_p$ be a direction, such that $\cBall[\xi,\tfrac\pi2]=\Sigma_p$. 
Draw a quasigeodesic $\gamma$ from $p$ in direction $\xi$. 

Set $q=\gamma(\tfrac\pi2)$. 
For any point $x\in \spc{L}$ we have  $\mangle(\xi,\dir px)\le\tfrac\pi2$. 
Therefore, by the comparison for $\gamma$ (\ref{k-convex-angle}$'$), $\dist{x}{q}{}\le\tfrac\pi2$. 
That is, $\cBall[q,\tfrac\pi2]=\spc{L}$, a contradiction.
\qeds

\parit{Proof \textnumero 1.} 
From Lemma~\ref{lemma:rad-big-U_p}, $\rad \Sigma_p>\tfrac\pi2$. 
Since $\dim \Sigma_p<m$, by the induction hypothesis we have $\Sigma_p\simeq \SS^{m-1}$.
Now the Morse lemma (\ref{lem:morse}) for
$\distfun{p}{}{}\:\spc{L}\to\RR$ gives that $\spc{L}\simeq\Susp\Sigma_p\simeq \SS^m$, 
here $\Susp\Sigma_p$ denotes a spherical suspension over $\Sigma_p$.
\qeds

\parit{Proof \textnumero 2.} Assume the contrary, then according to Sphere theorem (\ref{thm:sphere}), $\spc{L}$ contains a proper extremal subset $E\subset \spc{L}$.
Thus according to \ref{thm:dist-extr}, there is a point%
\footnote{In fact, according to ???, it is true for any $q\in E$.}
 $q\in E$ such that $\rad\Sigma_q\le\pi$, which contradicts Lemma~\ref{lemma:rad-big-U_p}.
\qeds

\section{Collapsing with positive curvature bound}

The following theorem was proved in \cite{perelman:collapsing}.
The corollary  was obtained earlier:
it follows from more delicate result \ref{thm:vol-rad-gap}, proved in \cite{grove-petersen:rad-sphere}; 
also it was proved directly in \cite{petrunin:master}.

\begin{thm}{Limit of collapse}\label{thm:collapse:k>1}
Assume $(\spc{L}_n)$ is a collasing sequence of $m$-dimensional complete length $\Alex{1}$ spaces and $\spc{L}_n\to \spc{L}$
then either $\spc{L}$ has an extremal subset or $\diam \spc{L}\le\tfrac\pi2$.
\end{thm}

\parit{Proof.}
The conclusion of the theorem is equivalent to the fact that $\Cone \spc{L}$ has an extremal set. 
???
\qeds

\begin{thm}{Corollary} \label{cor:collapse:k>1}
Let $(\spc{L}_n)$ be a collapsing sequence of $m$-dimensional complete length $\Alex{1}$ spaces and $\spc{L}_n\GHto \spc{L}$. 
Then $\rad \spc{L}\le \tfrac\pi2$.
\end{thm}

The direct proof below will use the following 

\begin{thm}{Exercise}
Let $\spc{L}$ be an $m$-dimensional complete length $\Alex{1}$ space and $\rad \spc{L}> \tfrac\pi2$. 
Show that there is an array of $m+2$ points $p,a_0,a_1,\dots,a_m\in \spc{L}$ such that for any $\xi\in\Sigma_p$ there is $i$ such that
\[\mangle(\xi,\dir p{a_i})>\tfrac\pi2.\]
\end{thm}

\parit{Hint:} For smooth nondecreasing function $\phi:\RR\to\RR$ such that 
$\phi(x)=0$ for all $x\le \tfrac\pi2$
and $\phi(x)>0$ for all $x>\tfrac\pi2$, take $p$ to be the minimum point of \[f(y)
=
\int\limits_\spc{L}\phi\dist[{{}}]{x}{y}{}\cdot\d x.
\]

\parit{Direct proof.} From ???, we have $\spc{L}$ is a $\kay$-dimensional complete length $\Alex1$ space and since $\spc{L}_n$ is collapsing, $\kay<m$.

Choose $p,\{a_i\}_{i=0}^\kay\in \spc{L}$ as in the exercise.
Let $p_n,\{a_{i,n}\}_{i=0}^\kay$ in $\spc{L}_n$ be such that $p_n\to p$ and $a_{i,n}\to a_i$ as $\spc{L}_n\GHto \spc{L}$.
Since $m>\kay$ there is a direction $\xi_n\in\Sigma_{p_n}$ such that $\mangle(\xi_n,\dir{p_n}{a_{i,n}})\le \tfrac\pi2$.
Let us shoot quasigeodesics $\gamma_n\:[0,\tfrac\pi2]\to \spc{L}_n$ with the initial data $\gamma^+(0)=\xi_n$ 
and pass to its partial limit $\gamma\:[0,\tfrac\pi2]\to \spc{L}$.
According to ??? $\gamma$ is a quasigeodesic, thus $\gamma^+(0)$ is defined and $|\gamma^+(0)|=1$.
According to ???, $\dist{a_{i,n}}{\gamma_n(t)}{}\le \dist{a_{i,n}}{p_n}{}$ for all $t$;
thus we have $\dist{a_{i}}{\gamma(t)}{}\le \dist{a_{i}}{p}{}$ for all $t$.
Therefore $\mangle(\gamma^+(0),\dir p{a_i})\le \tfrac\pi2$ for each $i$, which contradicts with the exercise.
\qeds


\begin{thm}{Volume-radius gap}\label{thm:vol-rad-gap}
Assume $\spc{L}$ is an $m$-dimensional complete length $\Alex{1}$ space and $\rad \spc{L}_n\le\tfrac\pi2$.
Then $\vol \spc{L}>\eps_m$, where $\eps_m=???$.
\end{thm}

This fact much stronger than ??? and its proof use only Morse lemma from all Alexandrov geometry.
The hart of it lie in Gromov's Filling radius estimate???

\parit{Proof.} 
For any  $p \in \spc{L}$, we can find $q\in \spc{L}$ such that $\dist{p}{q}{}>\tfrac\pi2$. 
Hence, if $\dist{x}{p}{}<\tfrac\pi2$ has no have critical values in $(0,\tfrac\pi2)$;
by Morse lemma (???), all balls $\oBall(p,\rho)$, where $\rho<\tfrac\pi2$ are contractible.

In terminology of \cite{gromov-filling}, that means that $\ConRad\spc{L}\ge\tfrac\pi2$, where $\ConRad\spc{L}$ is contractibility radius of $\spc{L}$.
According to \ref{lem:grove-petersen} $[\spc{L}_n]\not=0\in H_m(\spc{L}_n)$.
Thus, applying \cite[???]{gromov-filling}, the filling radius can be estimated from contractibility, namely
\[\FillRad\spc{L}
>
\frac{1}{2\cdot(m+2)}
\cdot
\ConRad\spc{L}
\ge
\frac\pi{4\cdot(m+2)}.\]
Finally applying celebrated Gromov's result \cite[1.2.A]{gromov-filling}, we have
\[
(\vol_m \spc{L})^{\frac{1}{m}}>\Const_m\FillRad\spc{L},
\]
for some $\Const_m\gg \frac{1}{m^{3m}}$;
and the result follows.
\qeds

\begin{thm}{Open question}
Assume $M_n$ be a sequence of Riemannian manifolds with curvature $\ge 1$ that is collapsing to Riemannian manifold $M$ of positive dimension.
Is it true that 
\[\dim M_n=\dim M+1\]
for all large $n$?
\end{thm}



\section{On area of boundary}

The following probelem was suggested by A.~Lytchak.
It looks simple, but we could not make a proof without use of gradient exponent.

This problem would have followed from Conjecture~\ref{conj:bry} 
(that boundary of an Alexandrov's space is an Alexandrov's space), 
but before this conjecture has been proved, any partial result is of some interest, see also remark on page~\pageref{rem:lyt-prob}.

\begin{thm}{Claim}\label{lyt-prob} 
Let $\spc{L}$ be an $m$-dimensional complete length $\Alex1$ space. 
Then
\[
\vol_{m-1}\partial \spc{L}
\le 
\vol_{m-1}\SS^{m-1}.
\]

\end{thm}

Let us first prepare a proposition:

\begin{thm}{Proposition}\label{prop:unique-gexp-inverse}
Let $\spc{L}$ be an $m$-dimensional complete length $\Alex{}$ space, $p,q\in\spc{L}$.
Then for any point $x\in \l] p q \r[$ the inverse of the gradient exponential map $\gexp^{-1}_{p;\kappa}x$ 
is uniquely defined.
\end{thm}

\parit{Proof.} 
Let $\gamma\:[0,t_0]\to \spc{L}$ be a unit-speed geodesic,
$\gamma(0)=p$, $\gamma(t_0)=q$.
From the angle comparison we get that $|\nabla_x\distfun{p}{}{}|\ge-\cos\angk\kappa x p q$. 
Therefore, for any $\zeta$ we have
\[
\tfrac{d^+}{dt}\dist[{{}}]{p}{\alpha_\zeta(t)}{}
\ge
-|\alpha^+_\zeta(t)|\cdot\cos\angk\kappa{\alpha_\zeta(t)}p q
\ \ \text{and}\ \ 
\tfrac{d^+}{dt}\dist[{{}}]{\alpha_\zeta(t)}{q}{}\ge-|\alpha^+_\zeta(t)|.
\]
It follows that $\angk\kappa q {\alpha_\zeta(t)}p $ is nondecreasing in $t$, hence the result.
\qeds


\parit{Proof of \ref{lyt-prob}}. 
Let $z\in \spc{L}$ be the point at maximal distance from $\partial \spc{L}$; in particular
$z$ realizes maximum of $f=\sin\circ\distfun{\partial \spc{L}}{}{}$.
From Theorem~\ref{thm:dist-to-bry}, $f(z)\le 1$ and $f''+f\le 0$.

Note that $\spc{L}\subset \cBall[z,\tfrac\pi2]$, otherwise if $y\in \spc{L}$ with $\dist{y}{z}{}>\tfrac\pi2$, then applying inequality $f''+f\le 0$ to the geodesic $[z y]$, we get that $(\d_z f)(\dir z y)>0$; that is, $z$ is not a maximum of $f$.

From this it follows that gradient exponent
\[
\gexp\mc1_z:(\cBall[\0,\tfrac{\pi}{2}],\mathfrak s)\to \spc{L}
\]

is a short onto map. 

Moreover,
\[\partial \spc{L}
\subset
\gexp\mc1_z(\partial \cBall[\0,\tfrac{\pi}{2}]).\] 
Indeed, $\gexp$ gives a homotopy equivalence 
$\partial \cBall[\0,{\tfrac\pi2}])\to
\spc{L}\backslash \{z\}$. 
Clearly, $\Sigma_z=\partial (\cBall[\0,{\tfrac\pi2}],\mathfrak s)$ has no boundary, therefore 
$H_{m-1}(\partial \spc{L},\ZZ_2)\not=0$, see \cite[lemma 1]{grove-petersen:rad-sphere}. 
Hence for any point $x\in\partial \spc{L}$, any geodesic $[z x]$ must have
a point of the image $\gexp\mc1_z(\partial \cBall[\0,\tfrac{\pi}{2}])$ but, as it is shown in
Proposition~\ref{prop:unique-gexp-inverse}, it can only be its end $x$. 

Now since 
\[
\gexp\mc1_z:(\cBall[\0,\tfrac{\pi}{2}],\mathfrak s)\to \spc{L}
\]
is short and
$(\partial \cBall[\0,\tfrac{\pi}{2}],\mathfrak s)$ is isometric to $\Sigma_z \spc{L}$ we get
$\vol\partial \spc{L}\le\vol \Sigma_z \spc{L}$ and clearly, $\vol \Sigma_z \spc{L}\le \vol \SS^{m-1}$.\qeds

\parbf{Remark.}\label{rem:lyt-prob}
Among other corollaries of
Conjecture~\ref{conj:bry}, it is expected that if $\spc{L}$ is an $m$-dimensional complete length $\Alex1$ space
then $\partial \spc{L}$, equipped with length-metric, 
admits a noncontracting map to $\SS^{m-1}$. 
In particular, its intrinsic diameter is at most $\pi$
and intrinsic perimeter of any triangle in $\partial \spc{L}$ is at most $2\cdot\pi$. 
This does not follow from the proof above, since in general 
$\gexp\mc1_z\partial \cBall[\0,{\tfrac\pi2}]\not\subset\partial \spc{L}$; that is, $\gexp_{z;1}\partial \cBall[\0,{\tfrac\pi2}]$
might have some creases left inside of $\spc{L}$, which might be used as a shortcut for
curves with ends in $\partial \spc{L}$.

\begin{wrapfigure}{r}{65mm}
\begin{lpic}[t(0mm),b(0mm),r(0mm),l(0mm)]{pics/shperical-triangle(0.25)}
\lbl[t]{4,58;$a$}
\lbl[lb]{199,80;$b$}
\lbl[tl]{199,58;$c$}
\lbl[b]{122,92;$z$}
\lbl[r]{195,69;$p{\longrightarrow}$}
\lbl[tl]{242,62;$a'$}
\lbl[t]{184,1;$r$}
\end{lpic}
\end{wrapfigure}

The existence of such creases one can see already in dimension $2$:
Cut a spherical triangle $\trig a b c]$, with right angle at $c$ and $\dist{a}{b}{}>\tfrac\pi2$.
Let us glue space $\spc{L}$ from two copies of $[a b c]$ along sides $[a b]$ and $[b c]$.

According to doubling theorem ???, $\spc{L}$ is a compact two-dimensional length $\Alex1$ space and its boundary consists of copies of side $[ac]$.
The point $z$, the farest point from the boundary, lies on side $[ab]$ and $\dist{a}{z}{}=\tfrac\pi2$. 
Let us show that 
\[
p
=
\gexp\mc1_z(\tfrac\pi2\cdot \dir zb)\notin\partial \spc{L}.
\]
Clearly $p\in [b c]$, thus we need to show only that $p\not=c$.
One can check it by calculations, but also one can see it using the following additional construction in $\SS^2$:
Let $a'$ be the opposite point to $a$ in $\SS^2$ 
and $r$ be a point on the extension of $[b c]$ behind $c$ such that $\dist{z}{r}{}=\tfrac\pi2$. 
Then $p\in[b c]$ and from the fact that $\gexp\mc1_z$ is short, one can see that $\dist{p}{r}{}=\dist{a'}{r}{}$.
Clearly $\dist{a'}{r}{}>\dist{c}{r}{}$; therefore $p\not=c$; that is, $p\not \in\partial \spc{L}$.








\section{Petersen's problem} 
\label{app-con-con}

\begin{thm}{Petersen's problem}\label{smoothable}
Let $\spc{L}$ be a smoothable Alexandrov's $m$-space; 
that is,
there is a sequence of Riemannian $m$-manifolds $M_n$ with curvature $\ge\kappa$
such that $M_n\GHto \spc{L}$.

Prove that the space of directions $\Sigma_x \spc{L}$ for any point $x\in \spc{L}$ is
homeomorphic to the standard sphere.
\end{thm}

Note that Perelman's stability theorem (see  \cite{perelman:spaces2},
\cite{kapovitch:stability}) only gives that $\Sigma_x \spc{L}$ has to be homotopically
equivalent to the standard sphere.

\bigskip
\noi\textit{Sketch of the proof:}
Fix a big negative $\lambda$ and construct a function $f\:\spc{L}\to\RR$ with $\d_p
f(x)\approx -|x|$ and controlled concavity of type $(\lambda,\kappa)$.
From  \ref{contr-concave}, the liftings  $f_n\:M_n\to\RR$ of  $f$ (see
\ref{lem:lifting}) are strictly concave for large $n$.
Let us slightly smooth the functions $f_n$ keeping them strictly concave.
If $a$ is taken little below the maximum of $f_n$,
then the level sets $f^{-1}_n(a)$
have strictly positive curvature and are diffeomorphic to the standard
sphere\footnote{Since $f$ has only one critical value above $a$ and it is a local maximum.}.

Let us denote by $p_n\in M_n$ a maximum point of $f_n$.
Then it is not hard to choose a sequence $\{a_n\}$ and a sequence of rescalings
$\{\lam_n\}$ so that $(\lam_n M_n,p_n)\GHto (\T_p,0)$ and $\lam_n f^{-1}_n(a_n)\subset \lam_n M_n$ converge to a
convex hypersurface $S$ close to $\Sigma_p\subset \T_p$.
Then, from Perelman's stability theorem, it follows that $S$ and therefore $\Sigma_p$
is homeomorphic to the standard sphere.
\qeds

\noi{\bf Remark.} From this proof it follows that $\Sigma_p$ is
itself smoothable. 
Moreover, there is a non-collapsing sequence of Riemannian metrics $g_n$ on $\SS^{m-1}$ such that $(\SS^{m-1},g_n)\GHto\Sigma_p$. 
This observation makes possible to proof a similar statement for iterated spaces of directions of smoothable Alexandrov space.

\section{Codimension of collaps}

In the case of collapsing, the liftings $f_n$ of a function $f$ with controlled concavity
type do not have the same controlled concavity type.

Nevertheless, the liftings are semiconcave and moreover, as was noted in
\cite{kapovitch:collapsing}, if
$M_n$ is a sequence of $m+k$-dimensional Riemannian manifolds with curvature $\ge
\kappa$, $M_n\GHto \spc{L}$, $\dim \spc{L}=m$, then one has a good control over the sum of
$k+1$ maximal eigenvalues of their Hessians. 
In particular, a construction as in the proof of Theorem~\ref{thm:strictly-concave} 
gives a strictly concave function
on $\spc{L}$ for which the liftings $f_n$ on $\spc{L}_n$ have Morse index $\le k$.
It follows that one can retract an $\eps$-neighborhood of $p_n$ to a $k$-dimensional CW-complex\footnote{it is unknown whether it could be retracted to an $k$-submanifold. If true, it would give some interesting applications}, where $p_n\in \spc{L}_n$ is a maximum point of $f_n$ and $\eps$ does not depend on $n$.
This observation gives a lower bound for the \emph{codimension of
a collapse}\footnote{in our case, it is $k$; the difference between the dimension of spaces from the collapsing sequence and the dimension of the limit space} to particular spaces. 
For example, for any lower curvature bound $\kappa$, the codimension of a collapse to $\Susp\HP^m$\footnote{that is, a spherical suspension over $\HP^m$} is
at least 3, and for $\Susp\CaP^2$  is at least 8 (it is expected to be $\infty$). 
In addition, it yields the following theorem, which seems to be the only sphere theorem that does not assume positiveness of curvature.

\begin{thm}{Funny sphere theorem}
If a $4\cdot(m+1)$ Riemannian manifold $M$ with sectional curvature $\ge\kappa$ is
sufficiently close\footnote{that is, $\eps$-close for some $\eps=\eps(\kappa,m)$} to $\Susp\HP^m$, then it
is homeomorphic to a sphere.
\end{thm}

\section{Convergence}

\begin{thm}{???Theorem}
Let $\spc{L}_n$ be an $m$-dimensional complete length $\Alex1$ space
and $\spc{L}_n\GHto\spc{L}$.
Assume that there is an array of points $\~{\bm{p}}=(\~p_0,\~p_1,\dots,\~p_\kay)$ in $\SS^\kay$ and
a sequence of arrays $\bm{p}_n=(p_{n;0},p_{n;1},\dots,p_{n;\kay})$ in $\spc{L}_n$ 
such that 
for all $i,j$ we have 
$\dist{p_{n;i}}{p_{n;j}}{}\to \dist{p_{i}}{p_{j}}{}$ as $n\to\infty$
and for any $n$ and $i\not=j$, we have $\dist{p_{n;i}}{p_{n;j}}{}>\tfrac\pi2$. 
Assume $\bm{p}$ be the limit array in $\spc{L}$.
Then there is an isometric embedding $\SS^\kay\hookrightarrow\spc{L}$ that contains all points of the array $\bm{p}$ in the image.
\end{thm}