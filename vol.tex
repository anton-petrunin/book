%%!TEX root = the-vol.tex
%%array^

\chapter{Volume}

In this chapter we extend number of results in measure theory of Euclidean space 
to $\Alex{}$ spaces.
The necessary preliminaries are discussed in Chapter~\ref{chap:measure-theorey}.

\section{Spherical density}

Given a real numbers $R>0$ and $\kappa$, 
set 
\[
\lambda_\kappa(R)=
\left[
\begin{matrix}
&1&\text{if}&\kappa\ge 0
\\
&\frac{\sn\kappa R}{R}&\text{if}&\kappa< 0
\end{matrix}
\right.
\]
Note that for any $\kappa$, we have $\lambda_\kappa(R)\to 1$ as $R\to0$.

\begin{thm}{Proposition}\label{prop:bishop}
Let $\spc{L}$ be an $m$-dimensional complete length $\Alex\kappa$ space
and
$p\in\spc{L}$.
Denote by $o$ the tip of the tangent space $\T_p$.
Then

\begin{subthm}{prop:bishop:f}
There is a map $f\:\T_p\to \spc{L}$
such that
\[f(\oBall(o,R)_{\T_p})=\oBall(p,R)_{\spc{L}}\]
for any $R>0$
and
$f$ 
is $\lambda_\kappa(R)$-Lipschitz in $\oBall(o,R)\subset\T_p$.

(In particular, if $\kappa\ge 0$ then $f$ is a short map.)
\end{subthm}

\begin{subthm}{prop:bishop:g}
There is a short map $g\:\EE^m\to \T_p$
such that
\[g[\oBall(0,R)_{\EE^m}]=\oBall(o,R)_{\T_p}\]
for any $R>0$.
\end{subthm}
\end{thm}

The following proof is 
a straightforward application of properties of gradient exponent
stated in Theorem~\ref{thm:prop-gexp}.


\parit{Proof.}
Take $f=\gexp_p\mc\kappa$.
According to \ref{thm:prop-gexp:short}, it satisfy the conditions in (\ref{SHORT.prop:bishop:f}).

Let us construct $g\:\EE^m\to \T_p$.

By Theorem~\ref{thm:tan4finite}, 
$\T_p$ is an $m$-dimensional complete length $\Alex\kappa$ space.
Choose a sequence of points $x_n\in \T_p$
such that $x_n\to o$ and $\T_{x_n}=\T_{x_n}\T_p\iso \EE^m$.
According to \ref{LinDim+-f}, such sequence exists.

Let us identify $\T_{x_n}$ and $\EE^m$ for each $n$.
We get a sequence of short maps $\gexp_{x_n}\:\EE^m\to\T_p$.

Note that
\[
\gexp_{x_n}[\oBall(0,R)_{\EE^m}]
\supset
\oBall(o,R-|x_n|)_{\T_p}.
\eqlbl{eq:ball-in-ball}
\]

Set $g\:\EE^m\to \T_p$ to be a partial limit of $\gexp_{x_n}$
as $n\to\infty$. 
The map $g$ is short since it is a limit of short maps.
From \ref{eq:ball-in-ball}, 
it follows that $g$ meets the other requirement in (\ref{SHORT.prop:bishop:g})
\qeds

\begin{thm}{Corollary}\label{cor:bishop}
Let $\spc{L}$ be an $m$-dimensional complete length $\Alex{\kappa}$ space. 
Then

\begin{subthm}{cor:bishop:doubling} The $m$-th spherical measure $\SphMes_m$ on $\spc{L}$ is doubling.
In particular, 

\begin{enumerate}[(i)]
\item By Proposition~\ref{prop:doubling-measure=>space}, space $\spc{L}$ is doubling. 
\item The $m$-th spherical measure $\SphMes_m$ is a Radon measure on $\spc{L}$.
\end{enumerate}
 
\end{subthm}


\begin{subthm}{cor:bishop:balls}
Assume $B=\oBall(p,R)$ be a ball in $\spc{L}$
and $B_n=\oBall(x_n,r_n)$, $n\in\IndexSet$ be a collection of disjoint balls
such that $B_n\subset B$ for each $n$.
Then 
\[\sum_n r_n^{ m}
\le
\lambda_\kappa^{m}(R)\cdot R^{m}.\]

In particular
\[\sum_n r_n^{m}
\le
R^{ m}+o(R^{ m})\]
and if $\kappa\ge 0$ then 
\[\sum_n r_n^{ m}
\le
R^{m}.\]
\end{subthm}

\begin{subthm}{cor:bishop:vol} For any $p\in \spc{L}$ and any $R>0$, we have
\[\SphMes_m\oBall(p,R)\le \lambda_\kappa^{ m}(R)\cdot R^{m}.\]
In particular
\[\SphMes_m\oBall(p,R)\le R^{ m}+o(R^{m})\]
and if $\kappa\ge 0$ then
\[\SphMes_m\oBall(p,R)\le R^{ m}.\]
\end{subthm}

\end{thm}

\parit{Proof; (\ref{SHORT.cor:bishop:doubling}).}
Fix $p\in$???

Part \ref{SHORT.cor:bishop:doubling} 
follows from Proposition~\ref{prop:bilip-measure} and ???
\qeds

\begin{thm}{Definition}
Let $\spc{L}$ be an $m$-dimensional complete length $\Alex{}$ space.
Given a subset $A\subset\spc{L}$, 
the density of $A$ is defined as
\[\dens_p A
=
\liminf_{\eps\to0+}
\frac{\SphMes_m[A\cap \oBall(p,\eps)]}{\eps^{m}}.\]

\end{thm}

\parbf{N.B.}
We define the density comparing 
the measures of a ball in the space 
and a ball of the same radius in the model space.
Often the density is defined as 
\[\liminf_{\eps\to0+}\frac{\SphMes_m[A\cap\oBall(p,\eps)]}{\SphMes_m[\oBall(p,\eps)]},\]
we are NOT using this definition.

In particular, note that with our convention, in general, $\dens_p \spc{L}\ne1$. It is however true for most $p\in \spc{L}$ and moreover we have






\begin{thm}{Lebesgue's density theorem}\label{thm:dens}
Let $\spc{L}$ be an $m$-dimensional complete length $\Alex\kappa$ space.
Then 
\begin{subthm}{thm:dens:=<1}
$\dens_p\spc{L}\le 1$ for any $p\in \spc{L}$.
\end{subthm}

\begin{subthm}{thm:dens:0or1}
For any set $A\subset \spc{L}$,
for 
$\SphMes_m$-almost all points $a$ in $A$ 
we have
$\dens_aA=1$. In particular, $\dens_p\spc{L}= 1$ for 
$\SphMes_m$-almost all points $p\in\spc{L}$.
\end{subthm}

\end{thm}

\parit{Proof.}
Part (\ref{SHORT.thm:dens:=<1}) follows from \ref{cor:bishop:vol}.

The proof of (\ref{SHORT.thm:dens:0or1}), 
modulo Corollary~\ref{cor:bishop},
is essentially the same as in the Euclidean case. By part  (\ref{SHORT.thm:dens:=<1})  we have that $\dens_xA\le1$ for all $x\in A$.

Assume that $\SphMes_m A'>0$,
where 
\[A'=\set{x\in A}{\dens_xA<1}.\]
Without loss of generality we may assume that $A$ 
and therefore $A'$ are bounded.

Given $\delta>0$ consider the set 
\[A'_\delta=\set{x\in\spc{L}}{\dens_xA<1-2\cdot\delta}.\]
Fix a sufficiently small $\delta$, 
so that 
\[\SphMes_m A'_\delta>0.\]

Cover $A_\delta'$ by a countable collection of balls $B_n=B(x_n,r_n)$
such that $r_n<\delta$ for each $n$ and
\[\SphMes_m  A'_\delta>(1-\delta)\cdot\sum_n r_n^{m}
.\]
It follows that we can find an $n$ such that
\[\SphMes_m [B_n\cap A'_\delta]
>
(1-\delta)\cdot r_n^{m}.\eqlbl{big-ball}\]

Note that the balls $\cBall[y,\rho]$
such that
\[\SphMes_m(A\cap\cBall[y,\rho])<(1-2\cdot\delta)\cdot r_n^{ m}\]
form a Vitali covering of $A'_\delta$.
Applying Vitali covering theorem \ref{thm:vitali},
we get a collection of disjoint closed balls 
$W_\kay=\cBall[y_\kay,\rho_\kay]$ which cover all the spherical measure of $B_n\cap A'_\delta$
and such that 
\[\frac{\SphMes_m [W_\kay\cap A'_\delta]}{\rho_\kay^{m}}
<
(1-2\cdot\delta )\]
for each $\kay$.

Since $r_n<\delta$ and $\delta$ is sufficiently small, 
applying \ref{cor:bishop:balls},
we get
\[\sum_\kay\rho_\kay^{m}\le (1+\tfrac\delta{10})\cdot r_n^{m}.\]
Therefore, 
\begin{align*}
\SphMes_m [B_n\cap A'_\delta]
&=\sum_\kay\SphMes_m [W_\kay\cap A'_\delta]<
\\
&<(1-2\cdot\delta)\cdot\sum_\kay\rho_\kay^{ m}<
\\
&<(1-2\cdot\delta)\cdot(1+\tfrac{\delta}{10})\cdot r_n^{ m}
\end{align*}
The later contradicts \ref{big-ball}.
\qeds

\begin{thm}{Proposition}\label{prop:dans=1=>euclid}
Let $\spc{L}$ be an $m$-dimensional complete length $\Alex{}$ space and $p\in \spc{L}$.
Then $\dens_p\spc{L}=1$ if and only if $\T_p\iso\EE^m$.
\end{thm}

\parit{Proof; $(\Leftarrow)$.}
Assume $\T_p\iso\EE^m$.

Fix $\eps>0$.
According to ???, $p$ admits a neighborhood 
which is $e^{\mp\eps}$-bi-Lipschitz to an open set in $\EE^m$.

It follows that $\dens_p\ge e^{-2\cdot m\cdot\eps}$.
By \ref{thm:dens:=<1}, $\dens_p\spc{L}\le 1$.
Since $\eps>0$ is arbitrary, we get $\dens_p\spc{L}= 1$.

\parit{$(\Rightarrow)$.}
Assume that $\dens_p\spc{L}=1$.
Let us show the following.

\begin{clm}{}\label{SphMes-tip}
For any $x\in \T_p$ and $r>0$ we have
\[\SphMes_m \oBall(x,r)_{\T_p}= r^{m}.
\]
\end{clm}

Indeed, 
let us denote by $o$ the tip of the cone $\T_p$
and set 
\[s=\SphMes_m\oBall(o,1)_{\T_p};\]
that is, $s$ is the spherical measure of the unit ball in $\T_p$
centered at its tip.
By Theorem~\ref{thm:tan4finite}, 
$\T_p$ is an $m$-dimensional complete length $\Alex0$ space.
Therefore by \ref{cor:bishop:vol}, we get that $s\le 1$.

By \ref{prop:bishop:f},
$$\SphMes_m(\oBall(p,R))
\le
s\cdot (R^{\cdot m}+o(R^{\cdot m})),$$
in particular $\dens_p\le s$.
It follows that $s=1$.

Further, by \ref{comp-kappa} the function $f\:\T_p\to\RR$
defined as $f=\tfrac12\cdot\dist[2]{x}{}{}$
is $1$-concave.
By Proposition~\ref{prop:GF-is-lip},
given $t>0$,
the gradient flow $x\mapsto\GF^t_f(x)$ is $e^t$-Lipschitz.
On the other hand if $\bar z\in [xz]$ 
such that $e^t\cdot\dist{x}{\bar z}{}=\dist{x}{z}{}$ for some $t\ge 0$ then $\GF^t_f(\bar z)=z$.
In particular,
\[\GF^t_f[\oBall(x,r)]=\oBall(x,e^t\cdot r)\supset\oBall(o,R),\]
where $R=e^t\cdot r-|x|$.
Therefore 
\[e^{m\cdot t}\SphMes_m[\oBall(x,r)]
\ge
\SphMes_m[\oBall(o,R)]
\]
Since $s=1$, we get
\[\SphMes_m \oBall(o,R)=R^{ m}.\]

and hence
\[\SphMes_m[\oBall(x,r)]
\ge
\frac{\SphMes_m[\oBall(o,R)]}{e^{m\cdot t}}\ge \frac{(e^t\cdot r-|x|)^m}{e^{m\cdot t}}
\]


Therefore, passing to the limit as $t\to\infty$ we get that 
\[\SphMes_m[\oBall(x,r)] \ge r^m\]
The opposite inequality follows by  \ref{thm:dens:=<1} which gives
 \ref{SphMes-tip}.
\claimqeds

Since $\T_p$ is an $m$-dimensional complete length $\Alex0$ space,
by Proposition \ref{prop:bishop:g}
there is a short map $g\:\EE^m\to \T_p$
such that 
\[g(\oBall(0,R)_{\EE^m})=\oBall(o,R)_{\T_p}\] 
for any $R>0$.

By Claim \ref{SphMes-tip}, $g$ has to preserve the spherical measure.

Let $x,y\in \EE^m$.
Set $x'=g(x)$ and $y'=g(y)$.
Since $g$ is short,
we have
\[\dist{x'}{y'}{\T_p}\le \dist{x}{y}{\EE^m}.
\eqlbl{|x'-y'|=<|x-y|}\]
To prove that $g$ is an isometry,
we need to show that we always have equality in \ref{|x'-y'|=<|x-y|}.
Suppose we have a strict inequality in \ref{|x'-y'|=<|x-y|}  for some $x,y\in \EE^m$ .
Set
\begin{align*}
R&\df\tfrac12\cdot\dist{x}{y}{\EE^m}
\\
r&\df
\tfrac12\cdot(\dist{x}{y}{\EE^m}-\dist{x'}{y'}{\T_p}),
\end{align*}
Note that $0<r< R$.


\begin{wrapfigure}{r}{48mm}
\begin{lpic}[t(-0mm),b(-0mm),r(0mm),l(0mm)]{pics/xwz-balls(1)}
\lbl[t]{14,13;$x'$}
\lbl[t]{34,13;$y'$}
\lbl[b]{23.6,15.7;{\small $w$}}%$\swarrow^{{}^{\hbox{$w$}}}$}
\end{lpic}
\end{wrapfigure}

Since $g$ is short,
\[g[\oBall(x,R)\cup\oBall(y,R)]
\subset 
\oBall(x',R)\cup\oBall(y',R).\eqlbl{balls-in-balls}\]

Let $w$ be the midpoint of a geodesic $[x'y']$ in $\T_p$.
Note that 
\[\oBall(x',R)\cap\oBall(y',R)\supset \oBall(w,r).\]
It follows that 
\begin{align*}
\SphMes_m[\oBall(x',R)\cup\oBall(y',R)]
&\le \SphMes_m\oBall(x',R)+\SphMes_m\oBall(y',R)-
\\
& \phantom{\le \SphMes_m\oBall(x',R)\ }-\SphMes_m\oBall(w,r)<
\\
&<\SphMes_m[\oBall(x,R)\cup \oBall(y,R)].
\end{align*}
By \ref{balls-in-balls},
the map $g$ can not preserve spherical measure,
a contradiction.
\qeds

Combining  Lebesgue's density theorem \ref{thm:dens:0or1} and Proposition  \ref{prop:dans=1=>euclid} we immediately obtain the following.

\begin{thm}{Corollary}\label{cor:ae-Euclidean-sph}
Let $\spc{L}$ be an $m$-dimensional complete length $\Alex{}$ space.
Then $\T_p\iso\EE^m$ for
$\SphMes_m$-almost all points $p\in\spc{L}$.

\end{thm}

As another corollary of Lebesgue's density theorem~\ref{thm:dens},
we get the following.

\begin{thm}{Lebesgue differentiation theorem}\label{thm:lebesgue-differentiation}
\label{thm:Lebesgue-derivative}
Let $\spc{L}$ be a complete length $\Alex{}$ space.
Then for any bounded $\SphMes_m$-measurable function $f\:\spc{L}\to \RR$
we have
\[\lim_{\eps\to0}\frac{1}{\eps^m}\cdot\int\limits_{\cBall[x,\eps]}|f(z)-f(x)|\cdot\d_z\SphMes_m=0\]
$\SphMes_m$-almost all $x\in\spc{L}$.

In particular,
\[f(x)
\ae
\lim_{\eps\to0}\frac{1}{\eps^m}\cdot\int\limits_{\cBall[x,\eps]}f(z)\cdot\d_z\SphMes_m.\]

\end{thm}

\parbf{Remark.}
The same holds for any locally $\SphMes_m$-integrable function $f$;
the same argument as in Euclidean space gives a proof;
see for example ???.


\parit{Proof.}
Apply the Lebesgue density theorem for the sets 
\[A_{v,w}=\set{x\in\spc{L}}{v\le f(x)\le w}\] 
for all the rational numbers $v$ and $w$.
\qeds

\section{Uniqueness of volume}

 
\begin{thm}{Definition}
Let $\spc{X}$ be a metric space.
A measure $\mu$ on $\spc{X}$ is called 
\emph{monotonic}\index{monotonic measure}
if for any short submap $f\:\spc{X}\subto \spc{X}$
and any closed set $A\subset\Dom f$
we have 
\[\mu A\ge \mu [f(A)].\]
\end{thm}

Note that for any integer $m\ge 0$,
the measures $\SphMes_m$, $\HausMes_m$ and $\LongMes_m$
defined in Section~\ref{sec:mes+balls} are monotonic.

It is well known that up to a multiplicative constant,
on the Euclidean space
there is a unique Radon measure
which is invariant with respect to translations.
The following theorem provides an analog for the $\Alex{}$ spaces.

\begin{thm}{Theorem}\label{thm:unique-measure-cbb}
Let $\spc{L}$ be an $m$-dimensional complete length $\Alex{}$ space.
Then up to a multiplicative constant, $\SphMes_m$
 is the unique monotonic Radon measure on $\spc{L}$.
\end{thm}

The proof is based on the following proposition.

\begin{thm}{Proposition}\label{prop:almost-const-R^m}
Let $\nu$ be a Radon measure defined on an open subset $\Omega$ of $\EE^m$.
Assume that there is $\lambda>1$ such that 
\[\nu(\cBall[x,r])\le \nu(\cBall[y,\lambda\cdot r])\]
for any two balls $\cBall[x,r], \cBall[y,\lambda\cdot r]\subset\Omega$.

Then there is a constant $c\ge 0$ such that
\[
\Const\cdot\SphMes_m A
\le
\nu A
\le
\Const\cdot\lambda^m\cdot\SphMes_m A\]
for any Borel set $A\subset \Omega$.
\end{thm}

\parit{Proof.}
First note that there is a constant $M$ such that
\[\nu(\cBall[x,r])\le M\cdot r^m
\eqlbl{pre-mu=<SphMes}\]
for any closed ball $\cBall[x,r]\subset \Omega$
with sufficiently small radius $r>0$.
In particular,
\[\nu\le M\cdot\SphMes_m.
\eqlbl{mu=<SphMes}\]

Indeed, without loss of generality, we may assume that $\Omega$ is bounded
and $\nu\Omega=1$.
Note that there is a constant $M $
such that $\Omega$ contains at least $\lfloor\tfrac{2}{M\cdot r^m}\rfloor$ disjoint closed balls of radius $\lambda\cdot r>0$.
In particular, given sufficiently small $r>0$, there is a ball $\cBall[x,\lambda\cdot r]\subset\Omega$ such that
\[\nu(\cBall[x,\lambda\cdot r])\le M\cdot r^m.\]

Applying the assumption, we get that 
\begin{align*}
\nu(\cBall[y,r])
&\le \nu(\cBall[x,\lambda\cdot r])
\le 
\\
&\le 
M\cdot r^m.
\end{align*}
for any ball $\cBall[y,r]\subset \Omega$.
Hence \ref{pre-mu=<SphMes} and consequently \ref{mu=<SphMes} follow.
\claimqeds

Applying the  Radon--Nikodym theorem, we get that 
\[\nu=\rho\cdot\SphMes_m\]
for a bounded $\SphMes_m$-measurable function $\rho\:\Omega\to\RR_{\ge0}$.

Applying Lebesgue differentiation theorem (\ref{thm:lebesgue-differentiation}), we get that 
\[\rho(x)
\ae
\lim_{\eps\to0}\frac{\nu(\cBall[x,\eps])}{\eps^m}.\]
By assumption,
$\nu(\cBall[x,\eps])\le \nu(\cBall[y,\lambda\cdot\eps])$
for any $x,y\in \Omega$ 
and all sufficiently small $\eps$.
Therefore for
$\SphMes_m$-almost all $x$ the inequality
\[\rho(x)\le \lambda^m\cdot\rho(y).
\eqlbl{eq:rho(x)=<rho(y)}\] 
holds for $\SphMes_m$-almost all $y$.
Hence the proposition follows.
\qeds


\parit{Proof of Theorem~\ref{thm:unique-measure-cbb}.}
Let $W$ denotes the subset of points of 
$w\in \spc{L}$ such that $\T_w\iso\EE^m$.
By Corollary~\ref{cor:ae-Euclidean-sph}
the set $W$ has full $\SphMes_m$ measure in $\spc{L}$.


Fix a point $q\in W$ and a sufficiently small $\eps>0$,
say $\eps<\tfrac1{10\cdot m}$.
According to ???,
there 
is an $e^{\mp\eps}$-bi-Lipschitz submap
\[\iota\:\spc{L}\subto \EE^m\] 
such that $q\in\Dom \iota$.

Denote by $\nu$ the push-forward of $\mu$ to $\Im\iota\subset\EE^m$;
that is, 
\[\nu A\df\mu [\iota^{-1} A]\]
for any Borel set $A\subset \Im \iota$. 
Note that the measure $\nu$ satisfies the assumptions of Proposition~\ref{prop:almost-const-R^m} for $\lambda=e^{2\cdot\eps}$ 
and $\Omega=\Im\iota$.
Indeed, if 
$\cBall(x,\lambda\cdot r),\cBall(y,r)
\subset
\Im\iota$
then 
\[h\:z\mapsto y+\tfrac1\lambda\cdot(z-x)\]
maps $\cBall[x,\lambda\cdot r]$ onto $\cBall[y,r]$ 
and it
induces
a short submap 
\[\iota^{-1}\circ h\circ\iota\:\spc{L}\subto \spc{L}.\]
Therefore, by the assumption, 
\begin{align*}
\nu(\cBall[x,\lambda\cdot r])
&=\mu [\iota^{-1}(\cBall[x,\lambda\cdot r])]
\le 
\\
&\le
\mu [\iota^{-1}(\cBall[y,r])]=
\\
&=\nu(\cBall[y,r]).
\end{align*}
By Proposition~\ref{prop:almost-const-R^m},
there is a constant $c\ge 0$ such that
\[
\Const\cdot\SphMes_m Q
\le
\nu Q
\le
\Const\cdot e^{2\cdot m\cdot\eps}\cdot\SphMes_m Q\]
for any Borel set $Q\subset \Im\iota$.

Since $\iota$ is $e^{\mp\eps}$bi-Lipshcitz,
we get
\[
\Const\cdot e^{- 2\cdot m\cdot\eps}\cdot\SphMes_m A
\le
\mu A
\le
\Const\cdot e^{4\cdot m\cdot\eps}\cdot\SphMes_m A
\eqlbl{eq:almost-SphMes}\]
for any Borel set $A\subset \Dom \iota$.

The same argument shows 
 that \ref{eq:almost-SphMes} holds also for any sets which can be covered 
 by the domains $\Dom\iota$ for different submaps $\iota$ as above;
 in particular for any subset $A\subset W$.
Since $\eps>0$ in \ref {eq:almost-SphMes} is arbitrary, we get that there is a constant $\Const\ge 0$ such that
\[
\mu A
=
\Const\cdot\SphMes_m A
\eqlbl{eq:=SphMes}\]
for any Borel set $A\subset W$.

It remains to show that $\mu(\spc{L}\backslash W)=0$.
Since $\SphMes_m(\spc{L}\backslash W)=0$,
the later will follow 
if 
\[\mu \le 10^m\cdot \Const\cdot\SphMes_m.\]
In turn, the later follows if 
\[
\mu (\cBall[p,r])
\le
10^m\cdot\Const\cdot r^m\eqlbl{eq:all-the-balls}\]
for any $p\in\spc{L}$ and all sufficiently small $r>0$.

Fix $p\in\spc{L}$.
Let the maps $f\:\T_p\to \spc{L}$ and $g\:\EE^m\to \T_p$ 
be as in the Proposition~\ref{prop:bishop}.
As before fix $q\in W$ and an $e^{\mp\eps}$-bi-Lipschitz submap
\[\iota\:\spc{L}\subto \EE^m\] 
such that $q\in\Dom \iota$.
Without loss of generality, we may assume that $\iota(q)=0$.

Note that the submap
\[h\:z\mapsto (g\circ f)(\tfrac{\iota(z)}2)\]
is short in a neighborhood of $q$ (provided $\eps$ was chosen sufficiently small)
and
$h(\cBall[q,2\cdot r])\supset \cBall[p, r]$ for all small $r>0$.
Therefore \ref{eq:all-the-balls} follows.
\qeds


\begin{thm}{Definition}\label{def:volume}
Let $\spc{L}$ be a metric space, 
$m$ be a nonnegative integer 
and $A\subset\spc{L}$ be a Borel subset.
The $m$-volume of $A$ is defined as 
\[\vol_m A\df \tfrac{\alpha_m}{2^{m}}\cdot\HausMes_m A.\]
(Recall that  $\alpha_m$ is the Lebesgue measure of the unit ball in $\EE^m$).

Further, if $\spc{L}$ is an $m$-dimensional complete length $\Alex{}$ space, we will say \emph{almost everywhere}\index{almost everywhere} meaning $\vol_m$-almost everywhere. 
\end{thm}
Theorem~\ref{thm:unique-measure-cbb} implies the following
\begin{thm}{Corollary}\label{cor:volume=}
Let $\spc{L}$ be an $m$-dimensional complete length $\Alex{}$ space,
then
\begin{align*}
\vol_m A&=\tfrac{\alpha_m}{2^{\cdot m}}\cdot\HausMes_m A=
\\
&=\alpha_m\cdot\SphMes_m A=
\\
&=\LongMes_m A.
\end{align*}
for any Borel set $A\subset\spc{L}$.
\end{thm}

With this in mind we can now restate Corollary~\ref{cor:ae-Euclidean-sph} as follows

\begin{thm}{Corollary}\label{cor:ae-Euclidean}
Let $\spc{L}$ be an $m$-dimensional complete length $\Alex{}$ space.
Then $\T_p\iso\EE^m$ for almost all points $p\in \spc{L}$.
\end{thm}

\section{Volume convergence}


\begin{thm}{Volume Convergence Theorem}\label{thm:vol-coverge}
Let  $\spc{L}_1, \spc{L}_2,\dots \spc{L}_\infty$ be $m$-dimensional complete length $\Alex\kappa$ spaces
and $\spc{L}_n\GHto \spc{L}_\infty$.
Then $\vol_m$ on $\spc{L}_n$ converges weakly to $\vol_m$ on $\spc{L}_\infty$.

In particular,
if $\spc{L}_\infty$ is compact then 
$\vol_m\spc{L}_n\to \vol_m\spc{L}_\infty$
as $n\to\infty$.
\end{thm}

\parit{Proof.}
Set 
$\mu_\infty=(\vol_m)_{\spc{L}_\infty}$ 
and 
$\mu_n=(\vol_m)_{\spc{L}_n}$.

From ~\ref{cor:bishop:vol} and ~\ref{cor:volume=} it follows that for any $p_n\in\spc{L}_n$ we have
\[\mu_n(\oBall(p_n,r))\le \alpha_m\cdot r^m+o(r^m).\] 
%where $\alpha_m$ denotes the volume of unit ball in $\EE^m$.
Therefore, passing to a subsequence if necessary,
we can assume that $\mu_n$ converges weakly to a measure,
say $\nu$
on $\spc{L}_\infty$
and 
\begin{align*}
\nu&\le \alpha_m\cdot\SphMes_m= 
\\
&=\mu_\infty;
\end{align*}
the last identity follows from Corollary~\ref{cor:volume=}.


Applying Radon--Nikodym theorem, 
we get that 
\[\nu=\rho\cdot\mu_\infty\] 
for some $\mu_\infty$-measurable function $\rho\:\spc{L}\to[0,1]$.
By Lebesgue differentiation theorem (\ref{thm:lebesgue-differentiation})
we can assume that
\[\rho(x)
=
\limsup_{\eps\to0} \frac{\nu(\cBall[x,\eps])}{\alpha_m\cdot\eps^m}.
\eqlbl{eq:rho-lebesgue}\]

Let us prove the following

\begin{clm}{}\label{clm:euclid=>rho=1}
If for $x_\infty\in\spc{L}_\infty$
we have $\T_x\iso\EE^m$ 
then $\rho(x)=1$.
\end{clm}

Fix a sequence of points $x_n\in\spc{L}_n$ such that $x_n\GHto x_\infty$.
Fix a small $\eps>0$.
According to ??? there is a sequence of $e^{\mp\eps}$-bi-Lipschitz submaps
$\map_n\spc{}\:\spc{L}_n\to\EE^m$ which converges to a $e^{\mp\eps}$-bi-Lipschitz submap $\map_\infty\:\spc{}\spc{L}_\infty\to\EE^m$ 
and such that 
$x_\infty\in\Dom\map_\infty$.

In particular, for any fixed small $r>0$, 
we have
$\oBall(x_n,r)\subset \Dom \map_n\subset \spc{L}_n$ for all large $n$.

Since $\Phi_n$ is $e^{\mp\eps}$-bi-Lipschitz,
we get 
\[\mu_n[\oBall(x_n,e^{-\eps}\cdot r)]
\ge e^{-3\cdot m\cdot\eps}\cdot \alpha_m\cdot r^m \]
Therefore 
\[\nu[\oBall(x_\infty,\cdot r)]\ge e^{-3\cdot m\cdot\eps}\cdot \alpha_m\cdot r^m\]
for all sufficiently large $r$.

Note that 
\[\limsup_{n\to\infty}\mu_n[\oBall(x_n,e^{-\eps}\cdot r)]\le \nu[\oBall(x_\infty,\cdot r)].\]
Since $\eps>0$ is arbitrary, \ref{eq:rho-lebesgue} implies Claim \ref{clm:euclid=>rho=1}.

By Corollary \ref{cor:ae-Euclidean},
Claim \ref{clm:euclid=>rho=1}
implies $\rho\ae 1$; that is, $\nu=\mu_\infty$
\qeds

\section{Coarea formula}

In this section we will show that coarea formula holds for Lipschitz maps between $\Alex{}$ spaces.
See Section~\ref{sec:coarea-prelim} for the preliminaries and definition.

Recall that it defines the $m$-th Jacobian of function $f\:\EE^\kay\to\EE^m$ at $x$ as $|(\d_x f)^{\wedge m}|$ if the differential $\d_x f$ is defined and linear.
Let $\spc{L}$ and $\spc{M}$ be complete length $\Alex{}$ spaces,
$\dim \spc{L}=\kay$, $\dim\spc{M}=m$ 
and $f\:\spc{L}\subto\spc{M}$ be a subfunction.
Assume $x\in\Dom f$,
$\T_x\iso\EE^m$,
$\T_{f(x)}=W\oplus W^\bot$,
where $W$ is a subcone of $ \T_{f(x)}$ which is isometric to Euclidean subspace
and $W^\bot$is its orthogonal complement,
$\d_xf$ is defined
and $\Im \d_xf\subset W$ and $\d_xf\:\T_x\to W$.
In this case we say that $|(\d_x f)^{\wedge m}|$ is defined
and 



\begin{thm}{Coarea formula}\label{thm:coarea-cbb}
Let
$\spc{L}$ and $\spc{M}$ be complete length $\Alex{}$ spaces,
$\infty>\dim \spc{L}=\kay>\dim\spc{M}=m$ 
and $f\:\spc{L}\subto\spc{M}$ be a locally Lipschitz submap
then  $|(\d_x f)^{\wedge m}|$ is defined for almost all $x\in \Dom f$ and
\[
\int\limits_Q|(\d_x f)^{\wedge m}|\cdot\d_x\vol_\kay
=
\int\limits_{f(Q)}\vol_{\kay-m}( f^{-1}(y)\cap Q)\cdot\d_y\vol_m;
\eqlbl{eq:coarea-once-more}
\] 
for any Borel set $Q\subset \Dom f$.
\end{thm}

Before getting to the proof of the coarea formula,
let us state and prove two of its corollaries.

Let $\spc{L}$ be a metric space.
Given $p\in \spc{L}$ denote by $S[R,p]$
the sphere of radius $R$ centered a $p$;
that is,\index{$S[p,R]$}
\[S[p,R]=\set{x\in\spc{L}}{\dist{p}{x}{}=R};\]
we may also write $S[p,R]_{\spc{L}}$ if we need to emphasize 
that this is a sphere in $\spc{L}$.


\begin{thm}{Corollary}\label{cor:coarea-for-dist-function}
Let $\spc{L}$ be an $m$-dimensional complete length $\Alex{}$ space
and $Q\subset\spc{L}$ be a Borel subset.
Then 
\[\vol_m Q=\int\limits_0^\infty\vol_{m-1}(Q\cap S[p,R])\cdot\d R.\]

In particular,
\[\vol_m \cBall[R,p]=\int\limits_0^R\vol_{m-1}(S[p,R])\cdot\d R.\]


\end{thm}

\parit{Proof.}
Note that
$(\d_x\dist{p}{}{})(\dir xp)=-1$.
In particular, $|\d_x(\dist{p}{}{})|\ge 1$.
On the other hand, since $\dist{p}{}{}\:\spc{L}\to\RR$ is $1$-Lipschitz,
we have
$|\d_x(\dist{p}{}{})|\le 1$.
Therefore
\[|\d_x(\dist{p}{}{})|= 1\] 
for any $x$.

It remains to apply the coarea formula to the distance function 
\[\dist{p}{}{}\:\spc{L}\to\RR.\]
\qedsf

\begin{thm}{Corollary}\label{cor:coarea-for-dist-map}
Let $\spc{L}$ be an $m$-dimensional complete length $\Alex{}$ space,
$f\:\spc{L}\to\RR$ be a continuous subfunction
and $\bm{a}=(a^1,a^2,\dots,a^m)$ be a point array in $\spc{L}$.
Assume that the distance map $\dist{\bm{a}}{}{}$ 
 is injective in an open set $\Omega\subset\spc{L}$.
Then for any function $f\:\dist{\bm{a}}{}{}(\Omega)\to\RR$ 
we have
\[\int\limits_{\Omega} \frac{f(\dist{\bm{a}}{p}{})}{\sqrt{\det(g_{ij})}}\cdot\d_p\vol_m
=
\int\limits_{\dist{\bm{a}}{}{}(\Omega)} f(\bm{x})\cdot\d x^1\cdot\d x^2\cdots\d x^m,\]
where $g_{ij}=\cos\mangle\hinge p{a^i}{a^j}$.
\end{thm}

\parit{Proof.}
Note that $|(\d_x[\dist{\bm{a}}{}{}])^{\wedge m}|\ae \frac1{\sqrt{\det(g_{ij})}}$.
Then apply the coarea formula.
\qeds

In the proof we will need the following two estimates of Jacobians.
Both easily follow from the definitions in Section~\ref{sec:coarea-prelim}.

\begin{thm}{Proposition}\label{prop:lin-algebra-1}
Let $U$, $V$ and $W$ 
be finite dimensional real vector spaces, equipped with inner products.
Assume $L\:U\to V$ and $M\: V\to U$ be linear operators.
Then 
\begin{subthm}{}
$|L^{\wedge n}|\le |L|^{ n}$ for any positive integer $n$.
\end{subthm}

\begin{subthm}{}
$|(L\circ M)^{\wedge n}|\le |L^{\wedge n}|\cdot |M^{\wedge n}|$ for any positive integer $n$.
\end{subthm}

\end{thm}


\begin{thm}{Proposition}\label{prop:lin-algebra-2}
Let $m\le \kay$ 
and 
$L\:\EE^m\to\EE^\kay$ be a linear map.
Denote by $\proj^1,\dots,\proj^N$ the projections of $\EE^\kay$ to all the coordinate $m$-planes%
\footnote{so, 
$N=\binom{k}{m}\df\tfrac{k!}{m!\cdot(\kay-m)!}$}.
Set 
$$M=\max\set{|(\proj^i\circ L)^{\wedge m}|}{i\in\{1,\dots,N\}}.$$
Then
$$M \le|L^{\wedge m}|
\le
N\cdot M.$$
\end{thm}

The following statement is less trivial.
It says that the coarea formula holds if the volume of the domain vanishes.

\begin{thm}{Proposition}\label{prop:coarea-rough}
Let $\spc{L}$ and $\spc{M}$ be complete length $\Alex\kappa$ space,
$\kay=\dim\spc{L}\ge \dim\spc{M}=m$,
and $f\:\spc{L}\subto\spc{M}$ be a locally Lipschitz submap.

Assume $A\subset \Dom f$ 
is a Borel subset 
such that $\vol_m A=0$
then
\[
\int\limits_{\spc{M}}\vol_{\kay-m}( f^{-1}(y)\cap A)\cdot\d_y\vol_m=0.\]

\end{thm}

\parit{Proof.}
Without loss of generality we may assume that $f$ is (globally) Lipschitz.

Fix sufficiently small $\eps>0$, say $\eps<\tfrac{\varpi\kappa}{10^\kay}$.

Given $y\in \spc{M}$, 
set $A_y= A\cap f^{-1}(y)$ and 
\[h_\eps(y)
\df
\inf
\set{\sum_{n\in\NN}(\diam W_n)^{m-\kay}}
{\begin{aligned}
&\text{all}\  
W_n
\ \text{are closed,}\ \bigcup_{n\in\NN}W_n\supset A_y,
\\
&\text{and}\ 
\sup_{n\in\NN}\{\diam W_n\}<\eps
 \end{aligned}
}.\]

By the definition of volume (\ref{def:volume}), 
we have
\begin{align*}
\vol_{m-\kay}[f^{-1}(y)]
&=
\tfrac{\alpha_m}{2^m}\cdot\HausMes_{m-\kay}[f^{-1}(y)]
=
\\
&=
\tfrac{\alpha_m}{2^m}\cdot\lim_{\eps\to0+}h_\eps(y).
\end{align*}
Therefore, by the monotone convergence theorem, 
it is sufficient to show that
\[\int\limits_{\spc{M}}h_\eps(y)\cdot\d_y\vol_m\to 0
\ \ 
\text{as}
\ \ \eps\to0{+}.
\eqlbl{eq:lim(h_eps)}\]

Let us cover $A$ by a countable collection of sets $Q_1,Q_2,\dots$
such that 
$$\sum_n(\diam Q_n)^{ \kay}<\eps^{\kay};$$
in particular $\diam Q_n<\eps$ for each $n$.

Clearly $\diam (A_y\cap Q_n)\le \diam Q_n$ for each $n$.
Since $\eps$ is sufficiently small, Theorem~\ref{cor:bishop:vol} and Corollary~\ref{cor:volume=}
imply that there is a constant $\Const$ which depends only on $m$, 
such that
$$\vol_m f(Q_n)<\Const\cdot(\lip f \cdot\diam Q_n)^m.$$
Therefore
\begin{align*}
\int\limits_{\spc{M}}h_\eps(y)\cdot\d_y\vol_m
&\le 
\sum_n\int\limits_{\spc{M}}h_\eps( A_y\cap Q_n)\cdot\d_y\vol_m
\le
\\
&\le \Const\cdot(\lip f)^{ m}\cdot \sum_n(\diam Q_n)^{\kay}<
\\
&<\Const\cdot(\lip f)^{ m}\cdot\eps^\kay
\end{align*}
Hence \ref{eq:lim(h_eps)} follows.
\qeds
 

\parit{Proof of \ref{thm:coarea-cbb}.}
Let us subdivide $Q$ into three disjoint subsets $Q^1$, $Q^2$ and $Q^3$ in the following way
\begin{align*}
Q^1&=\set{x\in Q}{\T_x\iso \EE^m\ \text{and}\ \T_{f(x)}\iso \EE^\kay},
\\
Q^2&=\set{x\in Q\backslash Q^1}{\T_{x}\iso \EE^m},
\\
Q^3&=Q\backslash (Q^1\cup Q^2).
\end{align*}

It is sufficient to show that
\[
\int\limits_{Q^i}|(\d_x f)^{\wedge m}|\cdot\d_x\vol_\kay
=
\int\limits_{f(Q^i)}\vol_{\kay-m}( f^{-1}(y)\cap Q^i)\cdot\d_y\vol_m.
\eqlbl{eq:coarea-123}^i\]
for $i=1$, $2$ and $3$;
each case will be proved separately.

\parit{Proof of \ref{eq:coarea-123}$^1$.}
Fix $\eps>0$.
According to ???,
both sets $Q^1\subset\spc{L}$ 
and $f(Q^1)\subset\spc{M}$ can be covered by the domains of a countable collection
of $e^{\mp\eps}$-bi-Lipschitz charts,
say $\phi_n\:\spc{L}\subto\EE^\kay$ and $\psi_n\:\spc{M}\subto\EE^m$.

Applying the Federer's coarea formula (\ref{thm:coarea-federer})
to the maps $\psi_j\circ f\circ\phi^{-1}_i$ for all $i$ and $j$ together with Proposition~\ref{prop:lin-algebra-1},
we get 
\[
\int\limits_{Q^1}|(\d_x f)^{\wedge m}|\cdot\d_x\vol_\kay
\lege
e^{\pm 10m\cdot\eps}\cdot\int\limits_{f(Q^1)}\vol_{\kay-m}( f^{-1}(y)\cap Q^1)\cdot\d_y\vol_m.\eqlbl{eq:coarea-pm}\]
More formally, we can subdivide $Q^1$ into a countable collection of disjoint Borel subsets $X_1$, $X_2,\dots$
in such a way that 
for each $n$ there are $i$ and $j$ 
such that $X_n\subset\Dom \phi_i$ and
$f(X_n)\subset\Dom \psi_j$.
Applying Federer's coarea formula (\ref{thm:coarea-federer}) for the  submap $h=\psi_j\circ f\circ\phi^{-1}_i\:\EE^\kay\subto\EE^m$  and $Y_n=\phi_n(X_n)\subset \EE^\kay$
we get
\[
\int\limits_{Y_n}|(\d_z h)^{\wedge m}|\cdot\d_z\vol_\kay
=
\int\limits_{h(Y_n)}\vol_{\kay-m}( h^{-1}(y)\cap Y_n)\cdot\d_y\vol_m.
\]
By Propositions~\ref{prop:lin-algebra-1} and \ref{prop:bilip-measure} we have
\[
\int\limits_{Y_n}|(\d_z h)^{\wedge m}|\cdot\d_z\vol_\kay
\lege 
e^{\pm 4\cdot m\cdot\eps}\cdot \int\limits_{X_n}|(\d_x f)^{\wedge m}|\cdot\d_x\vol_\kay
\]
and 
\[
\int\limits_{h(Y_n)}\vol_{\kay-m}( h^{-1}(y)\cap Y_n)\cdot\d_y\vol_m
\lege 
e^{\pm 4\cdot m\cdot\eps}
\cdot
\int\limits_{f(X_n)}\vol_{\kay-m}( f^{-1}(y)\cap X_n)\cdot\d_y\vol_m
\]
Combining the above we get
\[
\int\limits_{X_n}|(\d_x f)^{\wedge m}|\cdot\d_x\vol_\kay
\lege
e^{\pm 10\cdot m\cdot\eps}\cdot\int\limits_{f(X_n)}\vol_{\kay-m}( f^{-1}(y)\cap X_n)\cdot\d_y\vol_m.
\]
Summing up these inequalities by $n$, we get \ref{eq:coarea-pm}.

Since $\eps>0$ is arbitrary, 
\ref{eq:coarea-pm} implies \ref{eq:coarea-123}$^1$.

\parit{Proof of \ref{eq:coarea-123}$^2$.}
According to Corollary~\ref{cor:ae-Euclidean}, $\vol_\kay f(Q^2)=0$.
In particular, the right hand side in \ref{eq:coarea-123}$^2$ vanishes,
it remains to show that the left hands side vanishes too.

Note that the set $Q^2\subset\spc{L}$ can be covered by a countable collection
of bi-Lipschitz charts,
say $\phi_n\:\spc{L}\subto\EE^\kay$.

Without loss of generality we may assume that $f(Q^2)$ is bounded.
Therefore, by the theorem on distance embedding (\ref{thm:dist-emb}),
there is a  bi-Lipschitz embedding 
$\iota\: f(Q^2)\hookrightarrow\EE^n$
for some positive integer $n$.

Thus, there is a real constant $\Lip>1$ such that
each $\phi_i$ is $\Lip^{\mp1}$-bi-Lipschitz
and
\[|(\d_pf)^{\wedge m}|
\lege
\Lip^{\pm1}\cdot|(\d_p[\iota\circ f])^{\wedge m}|\]
for $\vol_m$-almost all $p\in Q^2$.

Since $\iota$ is bi-Lipschitz,
$\vol_\kay [\iota\circ f(Q^2)]=0$.

Consider all the orthogonal projections $\proj^1,\dots,\proj^N$ of $\EE^n$ to $m$-dimensional coordinate subspaces,
so $N=\tfrac{n!}{m!\cdot(n-m)!}$.
Note that if $\d_p(\iota\circ f)$ is linear then
so are the differentials $\d_p(\proj^i\circ\iota\circ f)$
for all $i$.
By Propositions~\ref{prop:lin-algebra-2} and \ref{prop:lin-algebra-1},
\[|(\d_p f)^{\wedge m}|
\le
N\cdot\Lip^2
\cdot
\sum_{i=1}^N|(\d_p[\proj^j\circ\iota\circ f\circ\phi_i^{-1}])^{\wedge m}|.
\eqlbl{eq:jac=<LN}\]
for $\vol_\kay$-almost all $p\in Q^2$.

Let us subdivide $Q^2$ into a countable collection of disjoint Borel subsets $X_1$, $X_2,\dots$
such that each $X_n\subset\Dom \phi_n$ for each $n$.
Applying Federer's coarea formula to $\proj^j\circ\iota\circ f\circ\phi_n^{-1}\:\EE^\kay\subto \EE^m$,
we get
\[
\int\limits_{\phi_iX_n}|(\d_x[\proj^j\circ\iota\circ f\circ\phi_n^{-1}])^{\wedge m}|\cdot\d_x\vol_\kay=0
\]
for all $j$ and hence

\[
\int\limits_{X_n}|(\d_x[\proj^j\circ\iota\circ f])^{\wedge m}|\cdot\d_x\vol_\kay=0
\]
for all $j$

Together with \ref{eq:jac=<LN}, this implies 
\[
\int\limits_{X_n}|(\d_x f)^{\wedge m}|\cdot\d_x\vol_\kay=0\]
Summing up for all $X_n$, 
we get that the left hand side in \ref{eq:coarea-123}$^2$ vanishes.

\parit{Proof of \ref{eq:coarea-123}$^3$.}
According to Corollary~\ref{cor:ae-Euclidean}, $\vol_m Q^3=0$.
In particular,  the left hand side in \ref{eq:coarea-123}$^3$ vanishes,
it remains to show that the right hands side also vanishes.
The later follows from Proposition~\ref{prop:coarea-rough}.
\qeds

\section{Bishop--Gromov inequality}

\begin{thm}{Bishop--Gromov inequality}\label{thm:bishop-gromov}
Let $m\ge 2$ and 
$\spc{L}$ is an $m$-dimensional complete length $\Alex\kappa$ space, 
$p\in \spc{L}$.
Set
\begin{align*}
v(R)&=\vol_m\oBall(p,R),
&
\~v(R)&=\vol_m\oBall(R)_{\Lob{m}{\kappa}}.
\end{align*}
Then the function
\[f\:R\mapsto \frac{v(R)}{\~v(R)}\] 
is not increasing and
\[\lim_{R\to0+}f(R)
=
\frac{\vol_{m-1}\Sigma_p}{\vol_{m-1} \SS^{m-1}}.\]

In particular, by Corollary~\ref{thm:simicont-Sigma},
\[v(R)\le \~v(R)\]
for any $R>0$.
\end{thm}


In the proof we will use the proposition below and 
Corollary~\ref{cor:coarea-for-dist-function},
which is a special case of the Coarea formula (\ref{thm:coarea-cbb}).

Recall that $\Sphere[p,R]_{\spc{L}}$ denotes the sphere of radius $R$ centered a $p$ in $\spc{L}$.
Similarly,  $S[R]_{\Lob{m}{\kappa}}$ denotes  a sphere of radius $R$ in ${\Lob{m}{\kappa}}$.

\begin{thm}{Proposition}\label{prop:monotonic-spheres}
Let $m\ge 2$ and 
$\spc{L}$ be an $m$-dimensional complete length $\Alex\kappa$ space, 
$p\in \spc{L}$.
Set
\begin{align*}
s(R)&=\vol_m\Sphere[p,R],
\\
\~s(R)&=\vol_m\Sphere(R)_{\Lob{m}{\kappa}}=(\sn\kappa R)^{m-1}\cdot\vol_{m-1}\SS^{m-1}.
\end{align*}
The function 
\[
h\: R\mapsto \frac {s(R)}{\tilde s(R)}
\]
is non-increasing on $(0,\varpi\kappa)$.
%\[\frac{\vol_{m-1}S[p,R]}{\tilde s(R)}\le\frac{\vol_{m-1}S[p,r]}{\tilde s(r)}.\]
\end{thm}

\parit{Proof.}
Let $0<r<R<\varpi\kappa$.

Let $Q\subset \Sphere[p,r]$ be the set of all points 
which lie on the geodesics from $p$ to the points on $\Sphere[p,R]$
and $f\:Q\to \Sphere[p,R]$ be the map such that $f\:x\mapsto y$ if $x\in [py]$.
Note that $f$ is onto.

The set $Q$ is a closed
and 
\[\vol_{m-1}Q\le s(r).\]

By Angle-sidelength  monotonicity (\ref{cor:monoton:2-sides}), 
for any $\eps>0$ the map $f$ is 
locally $(\tfrac{\sn\kappa R}{\sn\kappa r}+\eps)$-Lipschitz.
Since $\eps>0$ is arbitrary, the proposition follows.
\qeds


\parit{Proof of \ref{thm:bishop-gromov}.}
Set $h(R)= \frac {s(R)}{\tilde s(R)}$.
Consider the measure $\mu=\tilde s(r)\cdot\d r$ on  $(0,\varpi\kappa)$.

According to Corollary~\ref{cor:coarea-for-dist-function},
\begin{align*}
v(R)&=\int\limits_0^R s(r)\cdot\d r=\int\limits_0^R h(r)\cdot\d_r \mu,
\\
\~v(R)&=\int\limits_0^R\tilde s(r)\cdot\d r=\mu(0,R).
\end{align*}
Therefore
\begin{align*}
f(R)&=\frac{v(R)}{\~v(R)}=
\\
&=\frac1{\mu(0,R)}\cdot\int\limits_0^R h(r)\cdot\d_r\mu=
\\
&=\oint\limits_0^R h(r)\cdot\d_r\mu,
\end{align*}
here $\oint$ denotes the mean value.

By Proposition~\ref{prop:monotonic-spheres}, 
$h(R)= \frac {s(R)}{\tilde s(R)}$ is non-increasing.
it follows that $\mu$-mean value of $h$ on the interval  $(0,R)$ 
is non-increasing;
that is, $f(R)$ is non-increasing in $R$ in the interval $(0,\varpi\kappa)$.




It remains to prove that 
\[\lim_{R\to0+}f(R)
=
\frac{\vol_{m-1}\Sigma_p}{\vol_{m-1} \SS^{m-1}}.\]

Note that
\[\lim_{R\to0+}f(R)=\lim_{R\to0+} \frac{\vol_m\oBall(p,R)}{\alpha_m\cdot R^{ m}}.\]
Note that 
\[\frac{\vol_m\oBall(p,R)}{R^{m}}\]
is the volume of unit ball in the blowup 
$\tfrac1R\blow \spc{L}$ centered at $\tfrac1R\blow p$.

Applying the Volume Convergence Theorem (\ref{thm:vol-coverge})
for the canonical convergence $\tfrac1R\blow\spc{L}\GHto \T_p$ as $R\to 0+$,
we get that 
\[\frac{\vol_m[\oBall(p,R)_{\spc{L}}]}{R^m}
\to
\vol_m[\oBall(o,1)_{\T_p}]\ \ \text{as} \ \ R\to0+.\]

Finally, applying Corollary~\ref{cor:coarea-for-dist-function}
for the function $x\mapsto |x|$ on $\T_p$, we get
\begin{align*}
\vol_m[\oBall(o,1)_{\T_p}]
&=
\vol_{m-1}\Sigma_p\cdot\int\limits_0^1 R^{m-1}\cdot\d R=
\\
&=\tfrac1n\cdot\vol_{m-1}\Sigma_p
\end{align*}
Hence the result follows.
\qeds

\section{Exercises}


\begin{thm}{Lytchak's problem}
Let $\spc{L}$ be an $m$-dimensional complete length $\Alex{}$ space and $X\subset\spc{L}$ is a convex everywhere dense set.
Show that $X$ contains almost all points  of $\spc{L}$.
\end{thm}


\begin{thm}{Exercise}
Show that in case $\kappa=0$,
the Bishop--Gromov inequality (\ref{thm:bishop-gromov})
follows directly from ??? without using coarea formula.
\end{thm}

\begin{thm}{Exercise}
Let $\spc{L}$ be an $m$-dimensional complete length $\Alex{}$ space,
set $f(x)=\vol_{m-1}\Sigma_x$.
Show that $f\:\spc{L}\to\RR$ is a lower semi-continuous function;
that is, for any converging sequence $x_n\to x_\infty$ in $\spc{L}$,
we have 
\[\liminf_{n\to\infty}f(x_n)\ge f(x_\infty).\]
\end{thm}

\begin{thm}{Exercise}
Let $\spc{L}_n\GHto\spc{L}_\infty$,
$\spc{L}_n,\spc{L}_\infty$ are complete $\Alex\kappa$ spaces,
$\spc{L}_\infty$ is compact
and $\dim\spc{L}_n=m>\kay=\dim{L}_\infty$.

Consider sequence of probability measures 
$\mu_n=\tfrac{1}{\vol_m\spc{L}_n}\cdot\vol_m$
on $\spc{L}_n$.

Show that after passing to a subsequence $\mu_n$ weakly converges to a probability measure, say $\mu_\infty$, on $\spc{L}_\infty$ and $\mu_\infty$ is absolutely continuous with respect to $\vol_\kay$;
that is, $\mu_\infty=\rho\cdot\vol_\kay$
for some measurable function $\rho$.
\end{thm}








\begin{thm}{Exercise}
Let $\spc{L}$ be an $m$-dimensional complete length $\Alex{}$ space and $\partial \spc{L}\ne\emptyset$.
Show that the $(m-1)$-density of $\partial \spc{L}$ at any poin is at most $1$;
that is, 
for any point $p\in \partial \spc{L}$
we have
\[\limsup_{\eps\to0+}\frac{\vol_{m-1}(\partial\spc{L}\cap\cBall[p,\eps])}{\alpha_{m-1}\cdot \eps^{m-1}}\le 1,\]
where $\alpha_{m-1}$ denotes the volume of the unit ball in $\EE^{m-1}$.
\end{thm}


\begin{thm}{Definition}
Let $\spc{L}$ be a complete length $\Alex{}$ space and $p\in \spc{L}$.
The union of all semiopen geodesic segments $\l[px\r[$
will be called \index{inner locus}\emph{inner locus} of $p$ 
and denoted as $\InnLoc_p$.

The complement of the inner locus
is called 
\index{cut locus}\emph{cut locus}.
The cut locus of point $p$ will be denoted by $\CutLoc_p$;
so $\CutLoc_p=\spc{L}\backslash\InnLoc_p$.
In other words, $q\in \CutLoc_p$ if there is no semiopen geodesic segments $\l[px\r[$ containing $q$. 
\end{thm}


\begin{thm}{Exercise}
Let $\spc{L}$ be an $m$-dimensional complete length $\Alex{}$ space.
Show that
\[\vol_m\CutLoc_p=0\]
for any $p\in \spc{L}$.
\end{thm}

Note that since $\Str_p\supset \InnLoc_p$,
we also get that $\Str_p$ contains $\vol_m$-almost all points of $\spc{L}$.

\begin{thm}{Li--Rong problem}
Let $\spc{L}$ be an $m$-dimensional complete length $\Alex0$ space,
and $\DD$ be the unit ball in $\EE^m$.
Assume $\rad\spc{L}=1$ and $\vol_m\spc{L}=\vol_m\DD$.
Show that $\spc{L}$ is isometric to the quotient space of  $\DD$ 
by isometric involution on its boundary.
\end{thm}

A more general statement is proved by Li and Rong in \cite{li-rong}.

The following exercise generalize Bishop--Gromov inequality.

\begin{thm}{Exercise}\label{thm:bishop-gromov-gen}
Let $m\ge 2$ and 
$\spc{L}$ be an $m$-dimensional complete length $\Alex\kappa$ space, 
$S\subset \spc{L}$ be a compact subset.
Set
\begin{align*}
v(R)&=\vol_m\oBall[S,R],
&
\~v(R)&=\vol_m\oBall[R]_{\Lob{m}{\kappa}}.
\end{align*}
Prove that the function
\[f\:R\mapsto \frac{v(R)}{\~v(R)}\] 
is non-increasing.
\end{thm}
