-%%!TEX root = the-subspaces-cba.tex



\chapter{Subspaces of CBA spaces}

\section{Curves of bounded curvature.}\label{sec:cbc}

Using the Arm Lemma, we prove a curvature globalization theorem for curves.

By a {\it $k$-curve}   we mean a unit-speed curve $\sigma:\II\to\Lob2\kappa$ of constant
curvature $k$ in a model space, where $\II$ is a closed, possibly infinite, interval.  A {\it complete} $k$-curve is a $k$-curve that is maximally extended so as to self-intersect at most at its endpoints: a circle traversed once, or a  geodesic, horocycle, or equidistant curve.

The \emph{chord-length} of a curve $\gamma\:\II\to\spc{X}$ in a metric space $\spc{X}$ is the distance between the endpoints of $\gamma$.  A \emph{chord} is a geodesic joining two of its points, and an \emph{endpoint chord} is a geodesic joining its endpoints.

\begin{thm}{Definition}\label{def:arc-chord}
Let $\gamma$ be a rectifiable curve in a metric space $\spc{X}$. Suppose the length $s$ and chord-length $r$ of $\gamma$ satisfy $s+r <
2\pi/\varpi\kappa$.  The \emph{(global) arc-chord $\kappa$-curvature }$k_{\kappa}( \gamma)$ of $\gamma$ is the unique $ K\in\R$
for which there exists a $ K$-curve $\sigma$ in $\Lob2\kappa$ of length $s_\sigma=s$ and
chord-length $r_\sigma=r$. 
\end{thm}

Note that by the triangle inequality, the length $\bar s$ and chord-length $\bar r$ of any subarc $\bar\gamma$ of $\gamma$ satisfy $\bar s + \bar r \le s+r < 2\varpi\kappa$.

\begin{thm}{Definition}
\label{ptwise-arc-chord}
 The \emph{pointwise arc-chord curvature} of  $\gamma\:\II\to\spc{X}$ at $t_0$ is 
 the upper limit 
 $$ \limsup \,k_{\kappa}( \bar\gamma)$$
  of arc-chord curvatures taken over 
subarcs $ \bar\gamma=\gamma|(t_0-\epsilon,t_0+\epsilon)$ as $\epsilon\to 0$. (If $t_0$ is the left or right endpoint of $\II$, use $[t_0,t_0+\epsilon)$ or $� (t_0-\epsilon,t_0]$ accordingly.)   
\end{thm}

Definition \ref{ptwise-arc-chord} is independent of $\kappa$, as may be seen from the power series expressions of length $s$ in terms of chord-length $r$ for a subarc of a $k$-curve in $\Lob2{\kappa}$.

\begin{thm}{Theorem (Globalization of arc-chord curvature)}\label{thm:global-arc-chord}
  Suppose $\spc{X}\in\Cat{}{\kappa}$. Let $\gamma\:[a,b]\to\spc{X}$ be a rectifiable curve having pointwise arc-chord curvature
$ \le k$, and whose length $s$ and chord-length $r$ satisfy $s+r<2\varpi\kappa$.  
Then
$$k_{\kappa}( \gamma)\le k.$$
\end{thm}

The next corollary  spells out the significance of Theorem \ref{thm:global-arc-chord}, and follows directly from it:

\begin{thm}{Corollary}\label{cor:arc/chord-glob}    Suppose $\spc{X}\in\Cat{}{\kappa}$.  Let $\gamma\:[a,b]\to\spc{X}$ be a rectifiable curve having pointwise arc-chord curvature
$ \le k$, and whose length $s$ and chord-length $r$ satisfy $s+r<2\varpi\kappa$.    Let $\sigma$ be the complete $k$-curve in
$\Lob2\kappa$. 

\begin{subthm}{}  If $\gamma$ is
closed and nonconstant, then $s \ge s_\sigma$.\end{subthm}

\begin{subthm}{} 
 If $s \le\length \sigma$ and $\bar\sigma$ is an arc of $\sigma$ of length $s$, then $r \ge r_{\bar\sigma}$. 
\end{subthm}

\begin{subthm}{}  Suppose $r<\infty$ and  $r \le\diam\sigma$ (the diameter of $\sigma$  in $\Lob2\kappa$); thus a chord of length $r$ 
cuts $\sigma$ into  two arcs of lengths  $s_-\le s_+$ if $\sigma$ is a circle, and three arcs of which only one has finite length $s_-$ otherwise.  
Then 
either $s\le s_-$ or $s\ge s_+$.
The former holds if $\diam\gamma \le \diam\sigma$.
\end{subthm}
\end{thm}
%
%The proof of  Theorem \ref{thm:global-arc-chord} leads as well to the following useful lemma:
%
%\begin{thm}{Lemma}\label{prop:base-angle}
%If \,$s\le$ half the length of a complete $k$-curve in $\Lob2{\kappa}$, then  
%the angle between $\gamma^+(a)$ (respectively $-\gamma^-(b)$)  and the endpoint chord of $\gamma$ is at most equal to the angle between a $k$-curve of length $s$ and  its endpoint chord in $\Lob2{\kappa}$.
%\end{thm}
%
\parit{Proof 
%of Theorem \ref{thm:global-arc-chord} and Lemma \ref{prop:base-angle}.
} 
Let us complete $\gamma$
to a closed curve $\hat\gamma$ by including its chord. Since $s+r<2\varpi\kappa$, then by Reshetnyak
majorization  \ref{thm:major} there is a closed convex curve $\tilde\alpha$ in $S_K$ that
majorizes $\hat\gamma$. The portion of $\tilde\alpha$ that maps to the
chord of $\gamma$ is itself a  geodesic (\ref{lem:majorize-geodesic}).  Therefore
$\tilde\alpha$ consists of a curve $\tilde\gamma$, mapped to $\gamma$, and its
chord, mapped isometrically to the chord of $\gamma$. On subarcs of
$\tilde\alpha$, the majorizing map preserves arclength and does not increase chordlength, so does not decrease arc-chord
curvatures. Hence pointwise arc-chord curvature  of $\tilde\gamma$  is $\le k$  and  $k_{\kappa}(\tilde\gamma)= k_{\kappa}(\gamma)$. The same argument can be applied to every subarc of $\gamma$.
It follows that it suffices to prove Theorem \ref{thm:global-arc-chord} for convex curves in $\Lob2\kappa$.  
%By definition of angle, it similarly suffices to prove  Proposition \ref{prop:base-angle} for convex curves in $\Lob2\kappa$.  

Accordingly, we take $\spc{X}=\Lob2\kappa$, and suppose  $\gamma$ is convex. Let $\sigma$ be the complete $k$-curve  in $\Lob2\kappa$
%\parit{Theorem \ref{thm:global-arc-chord}:} 
We must show that if $s\le s_\sigma$, then a subarc of $\sigma$ of length $s$ has chord-length $\le r$. 

First we verify that $\gamma$ satisifies  the following \emph{local angle condition}: 
Given $\tilde k> k$,
for three sufficiently close
points $x=\gamma (t_1)$, $p=\gamma (t)$, $y=\gamma(t_2)$, $t_1<t<t_2$,  and points $\tilde x, \tilde p, \tilde y$ on a $\tilde k$-curve satisfying 
$\dist{x}{p}{}=
\dist{\tilde x}{\tilde  p}{}$ and $\dist{p}{y}{}=
\dist{\tilde p}{\tilde  y}{}$, we have $\mangle\hinge{p}{x}{y}\ge \mangle\hinge{\tilde p}{\tilde x}{\tilde y}$. 
Indeed, suppose this condition failed for some $\tilde k$. Then we may choose $x$ and $y$ arbitrarily close to $p$, such that $p$ is on the negative side of one of the small $\tilde k$-curves, say $\tilde \sigma$, with $x$ and $y$ as endpoints. 
Now move $\tilde\sigma$  toward $p$ along the bisector of $[xy]$. Sufficiently close to the last point of contact with $\gamma([t_0,t_1])$,  the translates of $\tilde\sigma$ cut off subarcs  of $\gamma$ that are longer than a $ \tilde k$-curve with the same endpoints.  But then $\gamma$ has pointwise arc-chord curvature $\ge \tilde k$ at $p$,  a contradiction.

Now approximate $\gamma$ by an inscribed equilateral broken geodesic $x^0x^1\dots x^{n+1}$  with the same endpoints as $\gamma$ and length $s- \epsilon_n$, 
so that for $i = 1,\ldots,n$, the subarc of $\gamma$ from
$x_{i-1}$ to $x_{i+1}$ satisfies the local angle condition.
Let $\~x^0\~x^1\dots \~x^{n+1}$ be an equilateral broken geodesic having the same number and
length of segments as $x^0x^1\dots x^{n+1}$, and inscribed in a $k$-curve $\tilde\sigma$.  For the subarcs of $\tilde\sigma$
between adjacent vertices, the difference  between their
arclength and chordlength has the uniform bound $C(s-\epsilon _n)^3/
n^3$, where $C$ depends only on $k$ and $\kappa$.  Therefore $\tilde\sigma$ has length at
most $(s- \epsilon_n) + C(s-\epsilon _n)^3/ n^2$, hence is less than the length of $\sigma$  for $n$ sufficiently large.  It follows that $[\~x^0\~x^1\dots \~x^{n+1}]$
is a convex polygon.

By Arm Lemma \ref{lem:arm}, $$\tilde r =\dist{\tilde x^0}{\tilde  x^{n+1}}{}\le  \dist{x^0}{x^{n+1}}{}=r. $$ 
As $n \to\infty$, $\tilde\sigma$ converges to a $k$-arc 
 of length $s$ and
chordlength $\le r$, proving Theorem \ref{thm:global-arc-chord}.
%\parit{Proposition \ref{prop:base-angle}:} 
%Now suppose $\gamma$ has length less than half of a complete $k$-curve.  Since  $\~x^0\~x^1\dots \~x^{n+1}$ is inscribed in a $k$-curve of the same length as $\gamma$ , the angles at $\~x^0$ and $ \~x^{n+1}$ of the convex polygon $[\~x^0\~x^1\dots \~x^{n+1}]$ are acute.
%
%We may  increase the angle at $\~x_i$ of the broken geodesic $\~x^0\~x^1\dots \~x^{n+1}$ to the angle at $x^i$ of $x^0x^1\dots x^{n+1}$, either by moving $\~x^0\~x^1\dots \~x^i$ rigidly while fixing $\~x^i\~x^{i+1}\dots \~x^{n+1}$, or by moving the latter and fixing the former,  respectively. We obtain a new convex polygon that lies, by first variation, in the same halfspace bounded by the complete geodesic through $\~x^0$ and $ \~x^{n+1}$ as does $[\~x^0\~x^1\dots \~x^{n+1}]$.  Therefore the angle at $ \~x^{n+1}$ and at $\~x^0$ respectively is no greater in the new polygon than in $[\~x^0\~x^1\dots \~x^{n+1}]$.
%
%Applying this procedure to each vertex in turn yields a convex polygon isometric to $[x^0x^1\dots x^{n+1}]$ and having angles at $x^0$ and $ x^{n+1}$ respectively no greater then the angles of $[\~x^0\~x^1\dots \~x^{n+1}]$ at $\~x^0$ and $ \~x^{n+1}$.
%
%As before, the claim follows by taking $n\to\infty$.
\qeds

%\begin{thm}{Exercise}(Arc-chord rigidity)
%\label{thm:arc-chord-rigidity}
%Under the hypotheses of Theorem \ref{thm:global-arc-chord}, if $
%\kappa_{K}(\gamma) = k$ 
%then $\gamma$ and its chord bound a totally
%geodesic surface in $\spc{X}$ that is isometric to a domain  in $\Lob2\kappa$
%bounded by a $k$-curve of length $s$ and its chord.
%\end{thm}

\section{Alexandrov's ruled surface theorem}\label{sec:ruled-surf}

\section{Gauss equation.}\label{sec:gauss-equation}

\section{Injectivity radius}\label{inj}

\section{Metric-minimizing surfaces}\label{metric-min}