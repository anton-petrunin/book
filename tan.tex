%%!TEX root = all.tex
\chapter{First order differentiation}\label{chap:tan}



\section{Space of directions and tangent space}
\label{sec:tangent-space+directions}

Recall that according to 
the corollaries \ref{cor:monoton:sup} 
and \ref{cor:monoton-cba:angle=inf},
the  $\Alex{}$ and $\cCat{}{}$ spaces have \index{space with defined angles}\emph{defined angles};
that is the angle $\mangle\hinge pxy$ is defined for any hinge $\hinge pxy$ in the space.

Let $\spc{X}$ be a metric space with defined angles.
Fix a point $p\in \spc{X}$. 

Consider the set $\mathfrak{S}_p$ 
of all nontrivial unit-speed geodesics  which start at $p$.
By \ref{claim:angle-3angle-inq} the triangle inequality holds for $\mangle$ on $\mathfrak{S}_p$;
that is, $(\mathfrak{S}_p,\mangle)$ 
forms a pseudometric space.

The metric space corresponding to  $(\mathfrak{S}_p,\mangle)$ is called \emph{space of geodesic directions} at $p$
and denoted as $\Sigma'_p$ or $\Sigma'_p\spc{X}$.
The elements of $\Sigma'_p$ are called \emph{geodesic directions} at $p$.
Each geodesic direction is formed by an equivalence class of geodesics starting from $p$ 
for the equivalence relation 
\[[px]\sim[py]\ \ \iff\ \ \mangle\hinge pxy=0.\]

\begin{thm}{Exercise}
Assume $\spc{L}$ is a complete length $\Alex{}$ space, and $[px]$, $[py]$ be two geodesics  which correspond to the same geodesic direction $\xi\in \Sigma'_p$.
Show that 
\[[px]\subset [py]\ \ \text{or}\ \ [px]\supset [py].\]

\end{thm}

\begin{thm}{Exercise}
Assume $\spc{U}\in\cCat{}{}$
be a space with extandable geodesics;
that is for any geodesic in $\spc{U}$
is an arc in a both-side infinite local geodesic.

Show that $\Sigma_p'$ is complete for any $p\in \spc{U}$.
\end{thm}

The completion of $\Sigma'_p$ is called \emph{space of directions} at $p$ and is denoted as $\Sigma_p$ or $\Sigma_p\spc{X}$.
The elements of $\Sigma_p$ are called \emph{directions} at $p$.

\begin{thm}{Exercise} 
Let $\spc{L}$ be a complete length $\Alex{1}$ space.
Show that $\spc{L}$ contains at most 3 points with space of directions $\le\tfrac12\cdot\SS^n$.
\end{thm}

The Euclidean cone $\Cone\Sigma_p$ over the space of directions $\Sigma_p$ is called tangent space at  $p$ and denoted as $\T_p$ or $\T_p\spc{X}$.

The tangent space $\T_p$ could be also defined directly, without introducing the space of direction.
To do so consider the set $\mathfrak{T}_p$ of all geodesics starting at $p$, with arbitrary speed.
Given $\alpha,\beta\in \mathfrak{T}_p$,
set 
\[\dist{\alpha}{\beta}{\mathfrak{T}_p}
=
\lim_{\eps\to0} 
\frac{\dist{\alpha(\eps)}{\beta(\eps)}{\spc{X}}}\eps
\eqlbl{eq:dist-in-T_p}\]
Since the angles in $\spc{X}$ are defined, 
\ref{eq:dist-in-T_p}
defines a pseudometric on $\mathfrak{T}_p$.


The corresponding metric space admits a natuaral isometric identification with the cone $\T'_p=\Cone\Sigma'_p$.
The elements of $\T'_p$ are formed by the equivalence classes for the realtion 
\[\alpha\sim\beta\ \ \iff\ \ \dist{\alpha(t)}{\beta(t)}{\spc{X}}=o(t).\]
The completion of $\T'_p$ is therefore  natuaral isometric to $\T_p$.

The elements of $\T_p$ will be called tangent vector at $p$,
despite that $\T_p$ is only cone --- not a vector space.
The elements of $\T'_p$ will be called geodesic tangent vector at $p$.

\section{Velocity of curves}

\begin{thm}{Definition}\label{def:right-derivative}
Assume $\alpha\:[0,a)\to \spc{X}$ for some $a>0$ is a function, not necessary continuous, such that $\alpha(0)=p$.
We say that $v\in\T_p$ is the right derivative of $\alpha$ at $0$,
breifly $\alpha^+(0)=v$ if there is a squence of vectors $v_n\in\T'_p$
with corresponding geodesics $\gamma_n$, 
such that $v_n\to v$ as $n\to\infty$ and 
\[\limsup_{\eps\to0+}\frac{\dist{\alpha(\eps)}{\gamma_n(\eps)}{\spc{X}}}{\eps}\to 0\ \ \text{as}\ \ n\to\infty.\]
\end{thm}

Note that if the condition in the Definition~\ref{def:right-derivative} holds for one choice of the sequences $(v_n)$ and $(\gamma_n)$ then it holds for any other choice of that type.

Note that if $\gamma$ is a geodesic starting at $p$ 
and the tangent vector $v\in\T_p'$ corresponds to $\gamma$ 
then $\gamma^+(0)=v$.

\begin{thm}{Exercise}\label{ex:tangent-vect=o(t)}
Assume $\spc{X}$ is a metric space with defined angles
and $\alpha,\beta\:[0,a)\to\spc{X}$ 
be two maps such that the right derivatives $\alpha^+(0)$, $\beta^+(0)$ are defined and $\alpha^+(0)=\beta^+(0)$.
Show that
\[\dist{\alpha(t)}{\beta(t)}{\spc{X}}=o(t).\]
\end{thm}

\begin{thm}{Proposition}
Let $\spc{X}$ be a metric space with defined angles and $p\in \spc{X}$.
Then for any tangent vector $v\in\T_p\spc{X}$ there is a map $\alpha\:[0,\eps)\to \spc{X}$ such that $\alpha^+(0)=v$.
\end{thm}

\parit{Proof.}
If $v\in \T_p'$ then for the corresponding geodesic $\alpha$ we have $\alpha^+(0)=v$.

Given $v\in \T_p$ construct a sequence $v_n\in\T'_p$ 
such that $v_n\to v$ and let $\gamma_n$ be the sequence of corresponding geodesic.

The needed map $\alpha$ can be found among the maps such that $\alpha(0)=p$ and
\[\alpha(t)=\gamma_n(t)\ \ \text{if}\ \ \eps_{n+1}\le t<\eps_n,\]
where $(\eps_n)$
is a decreasing sequence converging to $0$ as $n\to\infty$.
In order to satify the condition one has to choose the sequence $\eps_n$ converging to $0$ very fast.
\qeds

We define right and left derivatives $\alpha^+(t_0)$ and $\alpha^-(t_0)$
of $\alpha$ at $t_0\in\II$ as 
\[\alpha^\pm(t_0)=\check\alpha^+(0),\] where $\check\alpha(t)=\alpha(t_0\pm t)$.
The sign conversion is not quite standard, 
say if $\alpha$ is a smooth curve in a Riemannian manifold then we have
\[\alpha^+(t)=-\alpha^-(t).\]


\begin{thm}{Definition}\label{def:diff-curv}
Let 
$\spc{X}$ be a metric space 
and $\alpha\:\II\to \spc{X}$ be a curve.

For $t_0\in\II$, 
if $\alpha^+(t_0)$ or $\alpha^-(t_0)$ or both are defined,
we say correspondingly that  $\alpha$ is \emph{right}\index{differentiable!right differentiable} or \emph{left}\index{differentiable!left differentiable} or \emph{both-sided differentiable}\index{differentiable!both-sided differentiable} at $t_0$.
In the exceptional cases where $t_0$ is the left (respectively right) end of $\II$, $\alpha$ is by definition left (respectively right) differentiable at $t_0$.
\end{thm}

\begin{thm}{Exercise}\label{ex:tangent-vect=o(t)}
Assume $\spc{X}$ is a metric space with defined angles
Show that any geodesic $\gamma\:\II\to\spc{X}$ is both-sided differentiable everywhere.
\end{thm}

\section{Differential}

\begin{thm}{Definition}
Let $\spc{X}$ be a metric space with defined angles and
$f\:\spc{X}\subto\RR$ be a subfunction, 
$p\in\Dom f$ and $\II$ be a real interval.
A function $\phi\:\T_p\to\RR$ is called differential of $f$ at $p$
(briefly $\phi=\d_pf$) if for any map $\alpha\:\II\to \spc{X}$ such that $\alpha(0)=p$ and $\alpha^+(0)$ is defined, we have \[(f\circ\alpha)^+(0)=\phi(\alpha^+(0)).\]
\end{thm}

\begin{thm}{Proposition}\label{prop:differential}
Let $f\:\spc{X}\subto\RR$ be a locally Lipschitz semiconcave subfunction.
Then differential $\d_pf$ is uniquely defined for any $p\in\Dom f$. Moreover, 
\begin{subthm}{prop:differential:lip}
The differential $\d_pf\:\T_p\to\RR$ is Lipschitz and the Lipschitz constant of $\d_pf\:\T_p\to\RR$ does not exceed the Lipschitz constant of $f$ in a neighborhood of $p$. 
\end{subthm}

\begin{subthm}{prop:differential:homo}
$\d_pf\:\T_p\to\RR$ is a positive homogenius function;
that is, for any $\lam\ge 0$ and $v\in\T_p$ we have 
\[\lam\cdot\d_pf(v)=\d_pf(\lam\cdot v).\]
\end{subthm}

\end{thm}


\parit{Proof.}
Passing to a subdomain of $f$ if nesessary,
we can assume that $f$ is $\Lip$-Lipschitz and $\lambda$-concave for some $\Lip,\lambda\in\RR$.

Take a geodessic $\gamma$ starting at $p$ which lies in $\Dom f$.
Since $f\circ\gamma$ is semiconcave,
the rigth derivative $(f\circ\gamma)^+(0)$ is defined.
Since $f$ is  $\Lip$-Lipschitz, we have
\[|(f\circ\gamma)^+(0)-(f\circ\gamma_1)^+(0)|
\le
\Lip\cdot\dist[{{}}]{\gamma^+(0)}{\gamma_1^+(0)}{}\eqlbl{gam-bargam}\]
for any other geodesic $\gamma_1$ starting at $p$.

Define $\phi\:\T'_p\to\RR\:\gamma^+(0)\mapsto(f\circ\gamma)^+(0)$.
From \ref{gam-bargam}, $\phi$ is a $\Lip$-Lipschtz function defined on $\T_p'$.
Thus, we can extend $\phi$ to a whole $\T_p$ as a $\Lip$-Lipschitz function. 

It remains to show that $\phi$ is differential of $f$ at $p$.
Assume $\alpha\:[0,a)\to\spc{X}$ is a map such that $\alpha(0)=p$ and $\alpha^+(0)=v\in \T_p$.
Let $\gamma_n\in\Gamma_p$ be a sequence of geodesics as in the definition \ref{def:right-derivative};
that is, if 
\[v_n=\gamma^+_n(0)\ \ \t{and}\ \ a_n= \limsup_{t\to0+}{\dist{\alpha(t)}{\gamma_n(t)}{}}/{t}\] 
then $a_n\to 0$ and $v_n\to v$ as $n\to\infty$.
Then 
\[\phi(v)=\lim_{n\to\infty}\phi(v_n),\] \[f\circ\gamma_n(t)=f(p)+\phi(v_n)\cdot t+o(t),\] 
\[|f\circ\alpha(t)-f\circ\gamma_n(t)|\le\Lip\cdot\dist[{{}}]{\alpha(t)}{\gamma_n(t)}{}.\]
Hence 
\[f\circ\alpha(t)=f(p)+\phi(v)\cdot t+o(t)\]
\qedsf













\section{Ultratangent space} 

Recall that we assume that $\o$ is a once for all fixed choice of a selective ultrafilter.

For a metric space $\spc{X}$ and a positive real number $\lam$,
we will denote by $\lam\cdot\spc{X}$ its \emph{$\lam$-blowup}\index{blowup},
which is a metric space with the same underlying set as $\spc{X}$ and the metric multiplied by $\lam$.
The tautological bijection $\spc{X}\to \lam\cdot\spc{X}$ will be denoted as $x\mapsto x^\lam$, 
so 
\[\dist{x^\lam}{y^\lam}{}
=
\lam\cdot\dist[{{}}]{x}{y}{}\] 
for any $x,y\in \spc{X}$.

The $\o$-blowup $\o\cdot\spc{X}$ of $\spc{X}$ is defined as the $\o$-limit
of $n\cdot\spc{X}$; that is,
\[\o\cdot\spc{X}
\df
\lim_{n\to\o} n\cdot\spc{X}.\]

Given a point $x\in \spc{X}$ we can consider the sequence $x^n\in n\cdot\spc{X}$;
it corresponds to a point $x^\o\in \o\cdot\spc{X}$.
Note that if $x\ne y$ then 
\[\dist{x^\o}{y^\o}{\o\cdot\spc{X}}=\infty;\]
that is, 
$x^\o$ and $y^\o$ 
belong to different metric components of $\o\cdot\spc{X}$.

The metric component of $x^\o$ in $\o\cdot\spc{X}$ is called ultratangent space of $\spc{X}$ at $x$ and it is denoted as $\T^\o_x\spc{X}$.

Equivalently, ultratangent space $\T^\o_x\spc{X}$ can be defined the following way.
Consider all the sequences of points $x_n\in \spc{X}$ such that
the sequence $\ell_n=n\cdot\dist{x}{x_n}{\spc{X}}$ is bounded.
Define the pseudistance between two such sequences as 
\[\dist{(x_n)}{(y_n)}{}
=
\lim_{n\to\o}n\cdot\dist{x_n}{y_n}{\spc{X}}.\]
Then $\T^\o_x\spc{X}$ is the corresponding metric space.

Both, tangent space and ultratangent space, 
generalize the notion of tangent space of Riemannian manifold.
In the simplest cases these two notions define the same space.
In general, they are different and both useful ---
often lack of a property in one is compensated by the other.

It is clear from the definition that tangent space has cone structure.
On the other hand, in general, ultratangent space does not have a cone structure; a Hilbert's cube $\prod_{n=1}^\infty[0,2^{-n}]$ which is $\CBB{}{0}$ as well as $\cCat{}{0}$ gives an example.

\begin{thm}{Theorem}\label{thm:tan-is}
\begin{subthm}{thm:tan-is-CBB}
If $\spc{L}$ is a complete length $\Alex{\kappa}$ space and $p\in \spc{L}$ then $\T^\o_p$ is a complete length $\Alex{0}$ space.
\end{subthm}

\begin{subthm}{thm:tan-is-CBA}
If $\spc{U}\in\cCat{}\kappa$ and $p\in \spc{U}$ then $\T^\o_p\in\cCat{}{0}$.
\end{subthm}

\end{thm}

The proofs of both parts are nearly identical.

\parit{Proof; (\ref{SHORT.thm:tan-is-CBB}).}
Since $\spc{L}$ is a complete length $\Alex{\kappa}$ space, then for its blowup $n\cdot\spc{L}$ is a complete length $\Alex{\kappa/{n^{2}}}$ space.
Since $n\cdot(\spc{L},p)\to(\T_p^\o,\0)$ as $n\to\o$, Proposition~\ref{prp:A^omega} implies ${\T^\o_p}$ is a complete length $\Alex{0}$ space.

\parit{(\ref{SHORT.thm:tan-is-CBA}).}
Since $\spc{U}\in\cCat{}\kappa$, then for its blowup $n\cdot\spc{U}$, we have $n\cdot\spc{L}\in\cCat{}{\kappa/{n^{2}}}$.
Since $n\cdot(\spc{U},p)\to(\T_p^\o,\0)$ as $n\to\o$, Proposition~\ref{prop:CAT^omega} implies ${\T^\o_p}\in\cCat{}{0}$.
\qeds

The next theorem shows that $\T_p$ can be (and often will be) considered as a subset of  $\T^\o_p$.
That also implies that (1+3)-point comparison (\ref{df:cbb1+3}) as well as $n$-point comparison (\ref{thm:pos-config}) holds in $\T_p$.

\begin{thm}{Theorem}
\label{thm:T-in-T^w} 
Let $\spc{X}$ be a metric space with defined angles.
Then for any $p\in \spc{L}$, there is an distance preserving map 
\[\iota:\T_p\hookrightarrow \T^\o_p\] 
such that for any geodesic $\gamma$ starting at $p$
we have $\gamma^+(0)\mapsto \lim_{n\to\o}[\gamma(\tfrac1n)]^n$.
\end{thm}

\parit{Proof.}
Given $v\in \T'_p$ 
choose a geodesic $\gamma$ which starts at $p$ and $\gamma^+(0)=v$.
Set 
\[v^\o=\lim_{n\to\o}[\gamma(\tfrac1n)]^n.\]

Note that the value $v^\o\in\T^\o_p$ does not depend on choice of $\gamma$.
Indeed, if $\gamma_1$ is an other geodesic starting at $p$ such that $\gamma_1^+(0)=v$
then 
\[\dist{\gamma(t)}{\gamma_1(t)}{\spc{X}}=o(t).\]
Therefore 
\[\lim_{n\to\o}[\gamma(\tfrac1n)]^n=\lim_{n\to\o}[\gamma_1(\tfrac1n)]^n.\]

Set $\iota(v)=v^\o$.
Since angles between geodesics in $\spc{X}$ are defined, for any $v,w\in \T_p'$ we have
$n\cdot\dist[{{}}]{v_n}{w_n}{}\to\dist{v}{w}{}$.
Thus $\dist{v_\o}{w_\o}{}=\dist{v}{w}{}$; that is, $\iota$ is a global isometry of $\T_p'$.

Since $\T_p'$ is dense in $\T_p$,
we can extend $\iota$ to a global isometry $\T_p\to \T^\o_p$.
\qeds


\section{Linear subspace of tangent space}

\begin{thm}{Definition}\label{def:opp+Lin}
Let $\spc{L}$ be a complete length $\Alex{\kappa}$ space, $p\in \spc{L}$ and $u,v\in\T_p$.
We say that vectors $u$ and $v$ are \emph{opposite}\index{opposite}\label{def:opposite:page} to each other, (briefly, $u+v=0$) if $|u|=|v|=0$ or $\mangle(u,v)=\pi$ and $|u|=|v|$.

The subcone
\[\Lin_p=\{v\in\T_p\mid\exists\ w\in\T_p\ \ \t{such that}\ \ w+v=0\}\]
will be called \emph{linear subcone}\index{linear subspace} of $\T_p$.
\end{thm}

The reason for choosing the name ``linear subcone'' will be evident after proving Theorem~\ref{thm:lin-subcone}.

Let $u\in \Lin_p$; that is $u+v=0$ for some $v\in\T_p$.
Given $s<0$, set $s\cdot u=(-s)\cdot v$.
This way we define multiplication of any vector in $\Lin_p$ by any real number (positive and negative).
Proposition~\ref{prop:two-opp} implies that such multiplication is uniquely defined.

\begin{thm}{Proposition}\label{prop:opposite}
Let $\spc{L}$ be a complete length $\Alex{}$ space and $p\in \spc{L}$.
Given two vectors $u,v\in\T_p$, the following statements are equivalent:
\begin{subthm}{opposite} $u+v=0$;
\end{subthm}
\begin{subthm}{<x,u>} $\<u,x\>+\<v,x\>=0$ for any $x\in\T_p$;
\end{subthm}
\begin{subthm}{<xi,u>} $\<u,\xi\>+\<v,\xi\>=0$ for any $\xi\in\Sigma_p$.
\end{subthm}
\end{thm}

\parit{Proof.}
The condition $u+v=0$ is equivalent to $\<u,u\>=-\<u,v\>=\<v,v\>$;
thus 
(\ref{SHORT.<x,u>})$\Rightarrow$(\ref{SHORT.opposite}).
Since $\T_p$ is isometric to a subset of $\T^\o_p$,
the splitting theorem (\ref{thm:splitting}) applied for $\T_p^\o$
gives (\ref{SHORT.opposite})$\Rightarrow$(\ref{SHORT.<x,u>}).
The equivalence  (\ref{SHORT.<x,u>})$\Leftrightarrow$(\ref{SHORT.<xi,u>}) is trivial.
\qeds

\begin{thm}{Proposition}\label{prop:two-opp}
Let $\spc{L}$  be a complete length $\Alex{}$ space and $p\in \spc{L}$.
Then for any three vectors $u,v,w\in\T_p$, $u+v=0$ and $u+ w=0$ implies $v=w$.
\end{thm}

\parit{Proof.} From Proposition~\ref{prop:opposite} both $v$ and $w$ satisfy the condition in corollary~\ref{cor:polar}. 
Hence the result.\qeds


\begin{thm}{Theorem}\label{thm:lin-subcone}
Let $\spc{L}$  be a complete length $\Alex{\kappa}$ space and $p\in \spc{L}$. 
Then $\Lin_p$ forms a subcone of $\T_p$ isometric to a Hilbert space.
\end{thm}

Before comming to the proof of the theorem, 
let us give its corollary.

\begin{thm}{Corollary}\label{cor:euclid-subcone}
Let $\spc{L}$  be a complete length $\Alex{\kappa}$ space
and $p\in \Str(x_1,x_2,\dots,x_n)$.
Then there is a subcone $E\subset \T_p$ which isometric to a Euclidean space such that $\ddir p{x_i}\in E$ for every $i$.
\end{thm}

\parit{Proof.} 
By the definition of $\Str$ (\ref{def:straight}), $\ddir{p}{x_i}\in \Lin_p$ for each $i$.
It remains to apply Theorem~\ref{thm:lin-subcone}.
\qeds

The main difficulty in the proof of Theorem~\ref{thm:lin-subcone} comes from the fact that in general $\T_p\notin\CBB{}{0}$;
see Habeisen's example (Section~\ref{halbeisen}).
Otherwise the statement would follow directly from the Splitting theorem (\ref{thm:splitting}).
In fact the proof of this theorem is a far walk around, in the proof we use construction of gradient, as well as Splitting theorem, proof of which use gradient flow.
Thus in order to understand our proof completely one needs to read most of Chapter~\ref{chap:grad}.

First we give a construction of a tangent vector $w$ for given two vectors $u,v$.
If the tangent space is Euclidean, then  $w=-u-v$.




\parit{Proof of \ref{thm:lin-subcone}.}
First we show that $\Lin_p\in \CBB{}{0}$.

Note that $\T^\o_p\in \CBB{}{0}$ (see \ref{thm:tan-is-CBB}) and $\Lin_p$ is a closed subset of $\T^\o_p$.
Thus, it is sufficient to show that the metric on $\Lin_p$ inherited from $\T^\o_p$ is a length-metric.

Fix two vectors $x,y\in\Lin_p$.
Let $u$ and $v$ be such that $u+\tfrac{1}{2}\cdot x=0$ 
and $v+\tfrac{1}{2}\cdot y=0$.
Let us apply Lemma~\ref{lem:minus-sum} 
for vectors $u$ and $v$ to obtain vector $w\in \T_p$.
\begin{clm}{}\label{clm:w-mid(xy)}
$w$ is a midpoint of $[x y]$.
\end{clm}

Indeed, according to Lemma~\ref{lem:minus-sum}, 
\begin{align*}
|w|^2
&=
-\<w,u\>-\<w,v\>
=
\\
&=
\tfrac{1}{2}\cdot\<w,x\>+\tfrac{1}{2}\cdot\<w,y\>.
\\
\intertext{Therefore,}
\dist[2]{x}{w}{}+\dist[2]{w}{y}{}
&=2\cdot|w|^2+|x|^2+|y|^2-2\cdot\<w,x\>-2\cdot\<w,y\>=\\
&=|x|^2+|y|^2-\<w,x\>-\<w,y\>\le\\
&\le |x|^2+|y|^2+\<u,x\>+\<v,x\>+\<u,y\>+\<v,y\>=\\
&=\tfrac{1}{2}\cdot|x|^2+\tfrac{1}{2}\cdot|y|^2-\<x,y\>=\\
&=\tfrac{1}{2}\cdot\dist[2]{x}{y}{}.
\end{align*}
Thus $\dist{x}{w}{}=\dist{w}{y}{}=\tfrac{1}{2}\cdot\dist[{{}}]{x}{y}{}$ and \ref{clm:w-mid(xy)} follows.
\claimqeds

Note that for any $v\in\Lin_p$ there is a line $\ell$ which contains $v$ and $\0$, thus applying \ref{cor:splitting}, we get that $\Lin_p$ is isometric to a Hilbert space.
\qeds

\section{Differential and ultradifferential}\index{differential}

Note that $\dist{p}{}{}\:\spc{L}\to\RR$ is semiconcave in $\spc{L}\backslash\{p\}$.
Thus for any $q\not=p$ one can define differential.
The following claim follows directly from first variation formula (\ref{1st-var+}).

\begin{thm}{Theorem???}\label{thm:differential-of-dist}
Let $\spc{L}$ be a complete length $\Alex{\kappa}$ space
and $p,q\in \spc{L}$ be distinct points. 
Let $\Dir{p}{q}$ denotes the set of all directions of geodesics from $p$ to $q$ in $\spc{L}^\o$.
Then 
\[(\d_q\dist{p}{}{})(v)=-\sup\set{\<\xi,v\>}{\xi\in???\Dir{q}{p}}.\]
In particular, for any geodesic $[qx]$ we have
\[(\d_q\dist{p}{}{})(\dir qx)\ge -\cos\angk\kappa qxp\]
if the right hand side is defined.
\end{thm}


\begin{thm}{Theorem}\label{thm:d_q dist_p(v)=-<dri p q, v>}
Let $\spc{L}$ be a complete length $\Alex{\kappa}$ space 
and $p,q\in \spc{L}$ be to distinct points such that there is unique geodesic $[p q]$ in $\spc{L}^\o$.
Then for any $v\in\T_q$, we have
\[(\d_q\dist{p}{}{})(v)=-\<\dir q p,v\>.\]
In particular, the above identity holds if $q\in \Str(p)$ and if $p\in \Str(q)$.
\end{thm}


In general, as it shown in Halbeisen's example (see section \ref{halbeisen}),  
$\T_p$  might be not a length space; 
thus concavity of $\d_p f$ is meaningless. 
Nevertheless, as the following lemma shows, differential $\d_p f$ of a semiconcave function allways satisfy the following weaker property similar to concavity (compare \cite[136]{plaut:survey}, \cite[4.2]{ohta}).  
Note that in finite dimensional case $\d_p f$ is concave (see ???), 
but again, we use gradient to prove some basic properties of dimension.

\begin{thm}{Lemma}\label{lem:ohta} 
Let $\spc{L}$ be a complete length $\Alex{}$ space
$f\:\spc{L}\subto\RR$ be a locally Lipschitz semiconcave subfunction 
and $p\in \Dom f$.
Then
\[\sup\set{(\d_p f)(\xi)}{\xi\in\Sigma_p}
\ge 
\frac{(\d_p f)(u)+(\d_p f)(v)}{\sqrt{|u|^2+2\cdot\<u,v\> +|v|^2}}\]
for any $u,v\in \T_p$.
\end{thm}

\parit{Proof of \ref{lem:ohta}.}
We can assume that $\alpha=\mangle(u,v)>0$, otherwise the statement is trivial.
Moreover, since $\T'_p=\Cone(\Sigma'_p)$\index{$\T'_{{*}}$} is dense in $\T_p$ and $\d_p f\:\T_p\to\RR$ is Lipschitz, we can assume that $u,v\in \T'_p$; that is, $\exp_p(t\cdot u)$
 and $\exp_p(t\cdot v)$ are defined for all small $t>0$.

\begin{wrapfigure}{r}{50mm}
\begin{lpic}[t(0mm),b(-5mm),r(0mm),l(5mm)]{pics/puvvw(0.4)}
\lbl[r]{1,1;$\~ p$}
\lbl[lb]{14,44;$\~ u$}
\lbl[lb]{53,5;$\~ v$}
\lbl[b]{105,5;$\~ v'$}
\lbl[lb]{34,23;$\~ w$}
\lbl[lb]{11,8;$\alpha$}
\end{lpic}
\end{wrapfigure}

Prepare a model configuration of five points: $\~p,\~u,\~v,\~v',\~w\in\EE^2$ such that
\begin{itemize}
\item $\mangle\hinge{\~p}{\~u}{\~v}=\alpha$, 
\item $\dist{\~p}{\~u}{}=|u|$, 
\item $\dist{\~p}{\~v}{}=|v|$, 
\item $\~v'$ lies on the extension of $[\~p\~v]$ so that $\~v$ is midpoint of $[\~p\~v']$, 
\item $\~w$ is the midpoint for $\~u$ and ${\~v}$.
\end{itemize}
Note that 
\[\dist{\~p}{\~w}{}=\tfrac{1}{2}\cdot\sqrt{|u|^2+2\cdot\<u,v\>+|v|^2}.\]

Assume that $\spc{L}$ is geodesic.

For all small $t>0$, construct points $u_t,v_t,v'_t,w_t\in \spc{L}$ the following way:
\begin{enumerate}[(a)]
\item $v_t=\exp_p(t\cdot v)$,\ \  $v_t'=\exp_p(t\cdot v')$
\item\label{u_t}  $u_t=\exp_p(t\cdot u)$.
\item $w_t$ is the midpoint of $[u_t v_t]$.
\end{enumerate}
Clearly $\dist{p}{u_t}{}=t|u|$, $\dist{p}{v_t}{}=t|v|$, $\dist{p}{v_t'}{}=t|v'|$. 
Since $\mangle(u,v)$ is defined, 
we have $\dist{u_t}{v_t}{}=t\dist{\~u}{\~v}{}+o(t)$ 
and $\dist{u_t}{v_t'}{}=t\dist{\~u}{\~v'}{}+o(t)$ 
(see Theorem~\ref{angle} and Section~\ref{sec:angle}).

From point-on-side and hinge comparison (\ref{point-on-side}$+$\ref{SHORT.angle}), we get that 
\[\angk\kappa{v_t}p{w_t}
\ge
\angk\kappa{v_t}p{u_t}
\ge
\mangle\hinge{\~v}{\~p}{\~u}+\tfrac{o(t)}t\]
and
\[\angk\kappa{v_t}{v_t'}{w_t}
\ge
\angk\kappa{v_t}{v_t'}{u_t}
\ge
\mangle\hinge{\~v}{\~v'}{\~u}+\tfrac{o(t)}t.\]
Clearly, 
$\mangle\hinge{\~v}{\~p}{\~u}+\mangle\hinge{\~v}{\~x}{\~u'}=\pi$. 
From adjacent angle comparison (\ref{2-sum}), 
$\angk\kappa{v_t}p{v_t}\z+\angk\kappa{v_t}{u_t}{v_t'}\le \pi$.
Hence
$\angk\kappa{v_t}p{w_t}
\to
\mangle\hinge{\~v}{\~p}{\~v}$ as $t\to0+$
and thus 
\[\dist{p}{w_t}{}=t\dist{\~p}{\~w}{}+o(t).\]

Since $f$ is $\lambda$-concave we have 
\begin{align*}
2\cdot f(w_t)&\ge f(u_t)+f(v_t)+\tfrac\lambda4\cdot\dist[2]{u_t}{v_t}{}=
\\
&=2\cdot f(p)
+t\cdot [(\d_p f)(u)+(\d_p f)(v)]+o(t).
\end{align*}
 
Applying $\lambda$-concavity of $f$, we get
\[(\d_p f)(\dir p{w_t})
\ge 
\frac{t\cdot[(\d_p f)(u)+(\d_p f)(v)]
+o(t)}{2\cdot t\cdot\dist[{{}}]{\~p}{\~w}{}+o(t)}
\eqlbl{eq:lem:ohta*}\]
and the lemma follows.

\medskip

Finally, if $\spc{L}$ is not geodesic one needs to make two adjustments in the above constructions.
Namely: 
\begin{enumerate}[(i)]
\item For geodesic $[u_t v_t]$ to be defined, one has to take in (\ref{u_t}) $u_t\in \Str(v_t)$, $u_t\approx\exp_p(t\cdot u)$. 
More precicely, 
\[\dist{u_t}{\exp_p(t\cdot u)}{}=o(t).\] 
Thus instead of $\dist{p}{u_t}{}=t\cdot|u|$ we have 
\[\dist{p}{u_t}{}=t|u|+o(t),\] but it is sufficient for the rest of proof.
\item The direction $\dir p{w_t}$ might be undefined.
Thus, in the estimate \ref{eq:lem:ohta*}, instead of $\dir p{w_t}$, one should take $\dir p{w'_t}$ for some point $w_t'\in \Str(p)$ near $w_t$ (that is, $\dist{w_t}{w_t'}{}=o(t)$)
\end{enumerate}
\qedsf

\parbf{Ultradifferential.}
Given a function $f\:\spc{L}\to\RR$, consider sequence of functions $f_n\:n\cdot\spc{L}\to\RR$, defined by 
\[f_n(x^n)=n\cdot(f(x)-f(p)),\]
here $x^n\in n\cdot\spc{L}$ is the point corresopnding to $x\in\spc{L}$.
While $n\cdot(\spc{L},p)\to(\T^\o,\0)$ as $n\to\o$, 
functions $f_n$ converge to $\o$-differential of $f$ at $p$.
It will be denoted by $\d_p^\o f$;
\[\d_p^\o f\:\T_p^\o\to\RR,\ \ \d_p^\o f=\lim_{n\to\o} f_n.\] 

Clearly, the $\o$-differential of a locally Lipschitz subfunction is defined at each point of definition.
%By Theorem~\ref{thm:tan-is}, it is also clear that if $\T^\o_p$ is an Alexandrov space with upper or lower curvature bound, then differential $\d^\o_p f$ of semiconcave locally Lipschitz function $f$ is concave.

\begin{thm}{Proposition}
Assume $\spc{X}$ is a complete length $\Alex{}$ space or $\cCat{}{}$ space
and $f\:\spc{X}\subto\RR$ be a semiconcave locally Lipscitz subfunction.
Then for any $p\in\Dom f$, the ultradifferential $\d^\o_p\:\T^\o_p\to\RR$ is a concave function.
\end{thm}

\parit{Proof.}
Fix a geodesic $[x^\o y^\o]$ in $\T^\o_p$.
It is sufficient to show that for any subarc $[\bar x^\o \bar y^\o]$ of $[x^\o y^\o]$
which does not contains the ends
there is a sequence of geodesics $[\bar x^n\bar y^n]$ in $n\cdot \spc{X}$ which converges to $[\bar x^\o\bar y^\o]$.

Choose any sequences $\bar x^n,\bar y^n\in n\cdot \spc{X}$ such that $\bar x^n\to \bar x^\o$ and $\bar y^n\to \bar y^\o$ as $n\to\o$.
Note taht $[\bar x^n \bar y^n]$ 
converges to $[\bar x^\o \bar y^\o]$
as $n\to\o$.
The latter holds tivially in case $\cCat{}{}$
and in $\Alex{}$ case follows from ???.
\qeds



\section{Halbeisen's example}\label{halbeisen}

Here we give a construction of the following

\begin{thm}{Halbeisen's example}\label{Halbeisen's example}
There is a complete length $\Alex{}$ space $\check{\spc{L}}$
with a point $p\in\check{\spc{L}}$ such that the space of directions $\Sigma_p\check{\spc{L}}$ and therefore the tangent space $\T_p\check{\spc{L}}$ is not a length space. 
\end{thm}

Our construction is just a minor variation of one in \cite{halbeisen}.
If the dimension is finite, such examples do not exist, see \ref{thm:tan4finite}; 
for proper spaces the question is open, see \ref{open:Halb-proper}.

\parit{Construction.}
Let $\HH$ be a Hilbert space formed by infinite sequences of real numbers $\bm{x}=(x_0,x_1,\dots)$ with norm
$|\bm{x}|^2=\sum_i(x_i)^2$. 
Fix $\eps=0.001$ and consider two functions $f,\check f:\HH\to\RR$
\[f(\bm{x})=|\bm{x}|,\]
\[\check f(\bm{x})
=
\max\left\{|\bm{x}|,\max_{n\in\NN}\{(1+\eps)\cdot x_n-\tfrac{1}{n}\}\right\}.\] 
Both of these functions are convex and Lipschitz, therefore their graphs in $\HH\times \RR$ equipped with its length-metric form infinite dimensional Alexandrov spaces, say $\spc{L}$  and $\check{\spc{L}}$ (it is proved formally in \ref{lem:hil-con}).
Set $p$ to be the origin of $\HH\times \RR$.

Note that $\check{\spc{L}}\cap\spc{L}$ is starshaped in $\HH$ with center at $p$.
Further $\check{\spc{L}}\backslash\spc{L}$ consist of countable number of disjointed sets
\[\Omega_n=\set{(\bm{x},\check f(\bm{x}))\in\check{\spc{L}}}{(1+\eps)\cdot x_n-\tfrac{1}{n}>|\bm{x}|}\]
and $\dist{\Omega_n}{p}{}>\tfrac{1}{n}$.
It follows that for any geodesic $[p q]$ in $\check{\spc{L}}$,
a small subinterval $[p \bar q]\subset [p q]$ 
coinsides with a straight segment in $\HH\times\RR$, 
which is also a geodesic in $\spc{L}$.
Thus we can treat $\Sigma_p\spc{L}$ and $\Sigma_p\check{\spc{L}}$ as one set, with two angle metrics $\mangle$ and $\check\mangle$;
let us denote by $\mangle_{\HH\times \RR}$ the angle in $\HH\times\RR$.

The space $\spc{L}$  is isometric to the Euclidean cone
over $\Sigma_p\spc{L}$ with vertex at $p$; 
$\Sigma_p\spc{L}$ is isometric to a sphere in Hilbert space with radius $\frac{1}{\sqrt{2}}$.
In particular, $\mangle$ is length-metric of $\Sigma_p{\spc{L}}$ in $\mangle_{\HH\times\RR}$.



Thus, to show that $\check \mangle$ does not define a length-metric on $\Sigma_p{\spc{L}}$
it is sufficient to construct a pair of directions $(\xi_+,\xi_-)$ such that
\[\check \mangle(\xi_+,\xi_-)<\mangle(\xi_+,\xi_-).\] 
Set $\bm{e}_0=(1,0,0,\dots)$, $\bm{e}_1=(0,1,0,\dots),\dots\in \HH$. 
Consider the following two rays in $\HH\times \RR$
\[\gamma_+(t)
=
\tfrac{t}{\sqrt{2}}\cdot(\bm{e}_0,1)
\ \  \text{and}\ \ 
\gamma_-(t)
=
\tfrac{t}{\sqrt{2}}\cdot(-\bm{e}_0,1),
\ \ t\in[0,+\infty).\] 
They form unit-speed geodesics in both, $\spc{L}$ and $\check{\spc{L}}$.
Let $\xi_\pm$ be the directions of $\gamma_\pm$ at $p$.
Denote by $\sigma_n$ the half-planes in $\HH$ 
spanned by $\bm{e}_0$ and $\bm{e}_n$;
that is, $\sigma_n\z=\set{x\cdot\bm{e}_0+y\cdot\bm{e}_n}{y\ge 0}$.
Consider sequence of $2$-dimensional sectors $Q_n=\check{\spc{L}}\cap (\sigma_n\times \RR)$. 
For each $n$, the sector $Q_n$ intersects $\Omega_n$ and it is bounded by two geodesic rays $\gamma_\pm$.
Note that $Q_n\GHto Q$, where  $Q$ is a solid Euclidean angle
in $\EE^2$ with angle measure $\beta<\mangle(\xi_+,\xi_-)=\tfrac\pi{\sqrt{2}}$.
Indeed, $Q_n$ is isometric to the subset of $\EE^3$ described by
\begin{align*}
 y\ge0 \ \ 
\text{and}\ \  
&
z=\max\l\{\sqrt{x^2+y^2},
(1+\eps)\cdot y-\tfrac{1}{n} \r\}
\intertext{with length-metric.
Thus, its limit $Q$ is isometric to the subset of $\EE^3$ described by}
y\ge0
\ \ \text{and}\ \  
&
z=\max\l\{\sqrt{x^2+y^2},(1+\eps)\cdot y\r\}
\end{align*}
with length-metric.
In particular, for any $t,\tau\ge0$, 
\begin{align*}
\dist{\gamma_+(t)}{\gamma_-(\tau)}{\check{\spc{L}}} 
&\le 
\lim_{n\to\infty}\dist{\gamma_+(t)}{\gamma_-(\tau)}{Q_n}
=
\\ 
&=\side0 \{\beta;t,\tau\}.
\end{align*}
I.e., $\check\mangle(\xi_+,\xi_-) \le \beta<\mangle(\xi_+,\xi_-)$.\qeds

\begin{thm}{Lemma}\label{lem:hil-con}
Let $\HH$ be a Hilbert space,
$f\:\HH\to \RR$ be a convex Lipschitz function 
and $S\subset \HH\times \RR$ be the graph of $f$ 
equipped with length-metric.
Then $S$ is $\Alex{0}$.
\end{thm}

\parit{Proof.} Recall that a subset $X\subset \HH\times \RR$, 
we will denote by $\dist{*}{*}{X}$ the
length-metric on $X$.

Note that according to \ref{thm:buyalo} any convex hypersurface in a Euclidean space equipped with length-metric is non-negatively curved.
Thus it is sufficient to show that for any 4-point set $\{x_0,x_1,x_2,x_3\}\subset S$, 
there is a finite-dimensional subspace $E\subset \HH\times \RR$, 
such that $\{x_i\}\in E$ and $\dist{x_i}{x_j}{S\cap E}$ is arbitrary close to $\dist{x_i}{x_j}{S}$.

Clearly $\dist{x_i}{x_j}{S\cap E}\ge \dist{x_i}{x_j}{S}$; 
thus it is sufficient to show that for given $\eps>0$ one can choose $E$, so that 
\[\dist{x_i}{x_j}{S\cap E}
<
\dist{x_i}{x_j}{S}+\eps
\eqlbl{eq:claim:hil-con*}\]

For each pair $(x_i,x_j)$ choose a broken line $\beta_{i j}$ connecting $x_i,x_j$, which lies under $S$ (that is, outside of $\Conv S$) in $\HH\times \RR$ 
and has length at most $\dist{x_i}{x_j}{S}+\eps$.
Take $E$ to be the affine hull of all the vertexes in all $\beta_{i j}$.
Thus,
\[\dist{x_i}{x_j}{S\cap E}\le \length \beta_{i j}\] 
and \ref{eq:claim:hil-con*} follows.\qeds

\begin{thm}{Exercise}
Construct a non-compact complete geodesic $\Alex{0}$ space which contains no rays.
\end{thm}
















