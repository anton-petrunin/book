%%!TEX root = the-grad.tex
%arrays^%arXiv
\chapter{Gradient flow}\label{chap:grad}

Gradient flow could be considered as a nonsmooth version of first-order ordinary differential equations.
It provides a universal tool in Alexandrov geometry.

We consider only the $\Alex{}{}$ case since the main applications of gradient flow are there.
But the proofs in this chapter admit straightforward extension to locally compact length spaces with defined angles between geodesics, as well as $\CAT{}{}$ spaces.

The technique of gradient flow takes its roots in \index{Sharafutdinov's retraction}\emph{Sharafutdinov's retraction}, 
introduced by Vladimir Sharafutdinov \cite{sharafutdinov}.
It has been used  widely in comparison geometry since then.
In $\Alex{}$ spaces, it was first used by Grigory Perelman and the third author \cite{perelman-petrunin:qg}.
A bit later, independently Jürgen Jost and Uwe Mayer \cite{jost,mayer} 
used the gradient flow in $\CAT{}$ spaces.
Later, Alexander Lytchak unified and generalized these two approaches
to a wide class of metric spaces in \cite{lytchak:open-map}.
It was developed  yet further by Shin-ichi Ohta \cite{ohta} and by Giuseppe Sevar\'e \cite{sevare}.

{\sloppy 
The following exercise is a stripped-down version of Sharfutdinov's retraction;
it gives the idea behind gradient flow.

}

\begin{thm}{Exercise}\label{ex:sharafutdinov}
Assume that a one-parameter family of convex sets $K_t\subset \EE^m$ is nested; 
that is, $K_{t_1}\supset K_{t_2}$ if $t_1\le t_2$.
Show that there is a family of short maps $\phi_t\:\EE^m\to K_t$ 
such that $\phi_t|_{K_t}=\id$ and $\phi_{t_2}\circ\phi_{t_1}=\phi_{t_2}$ for all $t_1\le t_2$.
\end{thm}




\section{Gradient-like curves}\label{sec:gradient-like}


Gradient-like curves will be used later in the construction of gradient curves.
The latter are a special reparametrization of gradient-like curves.

\begin{thm}{Definition}\label{def:grad-like-curve}{\sloppy 
Let $\spc{Z}$ be a complete length space
and $f\:\spc{Z}\subto\RR$ be locally Lipschitz semiconcave subfunction.
Suppose that $\spc{Z}$ is either $\Alex{}$ or $\CAT{}$.

}

A Lipschitz curve $\hat\alpha\:[s_{\min},s_{\max})\to\Dom f$ will be called an  \index{gradient-like curve}\emph{$f$-gradient-like curve} if
\[\hat\alpha^+=\tfrac{1}{|\nabla_{\hat\alpha} f|}\cdot\nabla_{\hat\alpha} f;\]
that is, for any $s\in[s_{\min},s_{\max})$, $\hat\alpha^+(s)$ is defined and
\[\hat\alpha^+(s)=\tfrac{1}{|\nabla_{\hat\alpha(s)} f|}\cdot\nabla_{\hat\alpha(s)} f.\]

\end{thm}

Note that this definition implies that $|\nabla_p f|>0$ for any point $p$ on $\hat\alpha$.

The following theorem  gives a seemingly weaker condition that is equivalent to the definition of gradient-like curve.

\begin{thm}{Theorem}\label{thm:grad-like-2nd-def}
Suppose $\spc{Z}$ is a complete length space, 
$f\:\spc{Z}\subto\RR$ is a locally Lipschitz semiconcave subfunction,
and 
$|\nabla_p f|>0$ for any $p\in\Dom f$.
Suppose that $\spc{Z}$ is either $\Alex{}$ or $\CAT{}$.

A curve $\hat\alpha\:[s_{\min},s_{\max})\to\Dom f$ is an $f$-gradient-like curve if and only if it is $1$-Lipschitz and
\[\liminf_{s\to s_0+}\frac{f\circ\hat\alpha(s)-f\circ\hat\alpha(s_0)}{s-s_0}
\ge 
|\nabla_{\hat\alpha(s_0)} f|
\eqlbl{eq:thm:grad-like-2nd-def-1}\]
for almost all $s_0\in [s_{\min},s_{\max})$.
\end{thm}

\parit{Proof.} The only-if part follows directly from the definition.
To prove the if part, note that for any $s_0\in[s_{\min},s_{\max})$ we have
\begin{align*}
\liminf_{s\to s_0+}\frac{f\circ\hat\alpha(s)-f\circ\hat\alpha(s_0)}{s-s_0}
&\ge 
\liminf_{s\to s_0+}\oint\limits_{s_0}^s|\nabla_{\hat\alpha(\under s)}f|\cdot\dd\under s
\ge
\\
&\ge 
|\nabla_{\hat\alpha(s_0)}f|;
\end{align*}
the first inequality follows from \ref{eq:thm:grad-like-2nd-def-1} 
and the second from lower semicontinuity of the function $x\mapsto|\nabla_x f|$, 
see \ref{cor:gradlim}.
From \ref{lem:alm-grad}, we have 
\[\hat\alpha^+(s_0)=\tfrac{1}{|\nabla_{\hat\alpha(s_0)} f|}\cdot\nabla_{\hat\alpha(s_0)} f.\]
Hence the result.
\qeds

%It might have sense to generalize the following theorem to the solutions $f''\le \phi(f,f')$ or at least to the solutions of $f''\le \lambda-\kappa\cdot f$ ??

Recall that second-order differential inequalities are understood in a barrier sense; see Section~\ref{sec:conv-real}.

\begin{thm}{Theorem} \label{thm:concave}
Let $\spc{Z}$ be a complete length space 
and
$f\:\spc{Z}\subto \RR$ be
locally Lipschitz and $\lambda$-concave. 
Suppose that $\spc{Z}$ is either $\Alex{}$ or $\CAT{}$.
Assume $\hat\alpha\:[0,s_{\max})\to\Dom f$ is an $f$-gradient-like curve.
Then 
\[(f\circ\hat\alpha)''\le\lambda\] 
everywhere on $[0,s_{\max})$.
\end{thm}

{\sloppy 
Closely related statements were proved independently by Uwe Mayer and Shin-ichi Ohta \cite[2.36]{mayer} and \cite[5.7]{ohta}.

}

Before the proof, let us formulate and prove a corollary. 

\begin{thm}{Corollary}\label{cor:right-cont}
Let $\spc{Z}$ be a complete length space,
$f\:\spc{Z}\subto \RR$ be a locally Lipschitz and semiconcave function, 
and $\hat\alpha\:[0,s_{\max})\to\Dom f$ be an $f$-gradient-like curve.
Suppose that $\spc{Z}$ is either $\Alex{}$ or $\CAT{}$.
Then the function $s\mapsto |\nabla_{\hat\alpha(s)}f|$
is right-continuous; 
that is, for any $s_0\in [0,s_{\max})$ we have
\[|\nabla_{\hat\alpha(s_0)}f|
=
\lim_{s\to s_0+} |\nabla_{\hat\alpha(s)}f|.\]

\end{thm}

\parit{Proof.} Applying \ref{thm:concave} locally, we have that $f\circ\hat\alpha(s)$ is semiconcave.
The statement follows since 
\[(f\circ\hat\alpha)^+(s)
=
(\dd_p f)\left(\tfrac{1}{|\nabla_{\hat\alpha(s)}f|}\cdot\nabla_{\hat\alpha(s)}f\right)
=
|\nabla_{\hat\alpha(s)}f|.\]
\qedsf




\parit{Proof of \ref{thm:concave}.} For any $s>s_0$,
\begin{align*}
(f\circ\hat\alpha)^+(s_0)&=|\nabla_{\hat\alpha(s_0)}f|
\ge
\\
&\ge
(d_{\hat\alpha(s_0)}f)(\dir{\hat\alpha(s_0)}{\hat\alpha(s)})
\ge
\\
&\ge
\frac{f\circ\hat\alpha(s)-f\circ\hat\alpha(s_0)}{\dist{\hat\alpha(s)}{\hat\alpha(s_0)}{}}
-
\tfrac\lambda2\cdot\dist[{{}}]{\hat\alpha(s)}{\hat\alpha(s_0)}{}.
\end{align*}
Let $\lambda_+=\max\{0,\lambda\}$. 
Since $s-s_0\ge\dist{\hat\alpha(s)}{\hat\alpha(s_0)}{}$, for any $s>s_0$ we have 
\[(f\circ\hat\alpha)^+(s_0)\ge
\frac{f\circ\hat\alpha(s)-f\circ\hat\alpha(s_0)}{s-s_0}-\tfrac{\lambda_+}2\cdot(s-s_0).
\eqlbl{eq:thm:concave-1}\]
Thus $f\circ\hat\alpha$ is $\lambda_+$-concave.
That finishes the proof for $\lambda\ge 0$.
For $\lambda<0$ we get only that $f\circ\hat\alpha$ is $0$-concave.

Note that $\dist{\hat\alpha(s)}{\hat\alpha(s_0)}{}=s-s_0-o(s-s_0)$. Thus
\[(f\circ\hat\alpha)^+(s_0)\ge
\frac{f\circ\hat\alpha(s)-f\circ\hat\alpha(s_0)}{s-s_0} -\tfrac\lambda2\cdot(s-s_0)+o(s-s_0).
\eqlbl{eq:thm:concave-2}\]
Together, \ref{eq:thm:concave-1} and \ref{eq:thm:concave-2} imply that $f\circ\hat\alpha$ is $\lambda$-concave.
\qeds  




\begin{thm}{Proposition}
\label{prop:grad-like-unique-past}
Let $\spc{L}$ be a complete length $\Alex{\kappa}$ space, $p,q\in\spc{L}$.
Assume $\hat\alpha\:[s_{\min},s_{\max})\to\spc{L}$ is a $\distfun{p}$-gradient-like curve such that $\hat\alpha(s)\to z\in\mathopen{]}p q\mathclose{[}$ as $s\to s_{\max}+$.
Then $\hat\alpha$ is a unit-speed geodesic that lies in $[p q]$.
\end{thm}

\parit{Proof.} 
Clearly,
\[ \tfrac{d^+}{dt}\dist[{{}}]{q}{\hat\alpha(t)}{}
\ge
-1.
\eqlbl{eq:>=-1}
\]
On the other hand,

\[\begin{aligned}
\tfrac{d^+}{dt}\dist[{{}}]{p}{\hat\alpha(t)}{}
&\ge
(\dd_{\hat\alpha(t)}\distfun{p}{}{})(\dir{\hat\alpha(t)}{q})
\ge\\
&\ge
-\cos\angk\kappa{\hat\alpha(t)}p q.
\end{aligned}
\eqlbl{eq:>=-cos}\]
Inequalities \ref{eq:>=-1} and \ref{eq:>=-cos} imply that the function $t\mapsto\angk\kappa q {\hat\alpha(t)}p $ is nondecreasing.
Hence the result.
\qeds










\section{Gradient curves}\label{sec:grad-curves:def}

In this section we define gradient curves 
and tie them tightly to gradient-like curves 
which were introduced in Section~\ref{sec:gradient-like}.


\begin{thm}{Definition}\label{def:grad-curve}{\sloppy 
Let $\spc{Z}$ be a complete length space
and $f\:\spc{Z}\subto\RR$ be a locally Lipschitz and semiconcave function.
Suppose that $\spc{Z}$ is either $\Alex{}$ or $\CAT{}$.

}

A locally Lipschitz curve $\alpha\:[t_{\min},t_{\max})\to\Dom f$ will be called an \index{gradient curve}\emph{$f$-gradient curve} if
\[\alpha^+=\nabla_{\alpha} f;\]
that is, for any $t\in[t_{\min},t_{\max})$, $\alpha^+(t)$ is defined and 
$\alpha^+(t)=\nabla_{\alpha(t)} f$.
\end{thm}

The following exercise describes a global geometric property of a gradient curve without direct reference to its function.
It uses the notion of self-contracting curves which was introduced by Aris Daniilidis, Olivier Ley, St\'ephane Sabourau \cite{daniilidis-ley-sabourau}.

\begin{thm}{Exercise}\label{ex:elf-contracting}
Let 
$\spc{Z}$ be a complete length space,
$f\:\spc{Z}\to\RR$  a concave locally Lipschitz function, 
and $\alpha\:\II\to\spc{Z}$  an $f$-gradient curve.
Suppose that $\spc{Z}$ is either $\Alex{}$ or $\CAT{}$.

Show that $\alpha$ is \index{self-contracting curve}\emph{self-contracting}; that is,
\[\dist{\alpha(t_1)}{\alpha(t_3)}{\spc{Z}}\ge \dist{\alpha(t_2)}{\alpha(t_3)}{\spc{Z}}\]
if $t_1\le t_2\le t_3$.
\end{thm}

The next lemma states that gradient and gradient-like curves are special reparametrizations of each other.

\begin{thm}{Lemma}\label{lem:grad--grad-like}
Let $\spc{Z}$ be a complete length space
and
$f\:\spc{Z}\subto\RR$ be a locally Lipschitz semiconcave subfunction 
such that $|\nabla_p f|>0$ for any $p\in\Dom f$.
Suppose that $\spc{Z}$ is either $\Alex{}$ or $\CAT{}$.

Assume that $\alpha\:[0,t_{\max})\to \Dom f$ is a locally Lipschitz curve 
and $\hat\alpha\:[0,s_{\max})\to \Dom f$ is its reparametrization by arc-length, 
so $\alpha\z=\hat\alpha\circ\varsigma$ for some homeomorphism $\varsigma\:[0,t_{\max})\to [0,s_{\max})$.
Then 
\begin{align*}
\alpha^+&=\nabla_\alpha f
\\
&\Updownarrow
\\
\hat\alpha^+=\frac{1}{|\nabla_{\hat\alpha} f|}\cdot\nabla_{\hat\alpha} f
\quad
&
\text{and}
\quad
\varsigma^{-1}(s)
=
\int\limits_0^s\frac{\dd\under s}{(f\circ\hat\alpha)'(\under s)
 }.
\end{align*}

\end{thm}

\parit{Proof; $(\Downarrow)$.} 
According to \ref{thm:speed},
\[
\begin{aligned}
\varsigma'(t)&\ae|\alpha^+(t)|=
\\
&=|\nabla_{\alpha(t)}f|.
\end{aligned}
\eqlbl{eq:lem:grad--grad-like-1}\]
Note that 
\begin{align*}
(f\circ\alpha)'(t)&\ae (f\circ\alpha)^+(t)=
\\
&=|\nabla_{\alpha(t)} f|^2.
\end{align*}
Setting $s=\varsigma(t)$, we have
\begin{align*}(f\circ\hat\alpha)'(s)
&\ae\frac{(f\circ\alpha)'(t)}{\varsigma'(t)}
\ae
\\
&\ae|\nabla_{\alpha(t)}f|=
\\
&=|\nabla_{\hat\alpha(s)}f|.
\end{align*}

From \ref{thm:grad-like-2nd-def}, it follows that $\hat\alpha(t)$ is an $f$-gradient-like curve; 
that is,
\[\hat\alpha^+=\frac{1}{|\nabla_{\hat\alpha} f|}\cdot\nabla_{\hat\alpha} f.\]
In particular, $(f\circ\hat\alpha)^+(s)=|\nabla_{\hat\alpha^+(s)} f|$, and by \ref{eq:lem:grad--grad-like-1},
\begin{align*}\varsigma^{-1}(s)
&=\int\limits_0^s\frac{\dd\under s}{|\nabla_{\hat\alpha(\under s)} f|}
=
\\
&=
\int\limits_0^s\frac{\dd\under s}{(f\circ\hat\alpha)'(\under s)}.
\end{align*}
\medskip

\noi{$(\Uparrow)$.}
Clearly,
\begin{align*}\varsigma(t)
&=
\int\limits_0^{t}(f\circ\hat\alpha)^+(\varsigma(\under t))\cdot\dd \under t
=
\\
&=
\int\limits_0^{t}|\nabla_{\alpha(\under t)}f|\cdot\dd \under t.
\end{align*}
According to \ref{cor:right-cont}, the function $s\mapsto|\nabla_{\hat\alpha(s)}f|$ is right-continuous.
Therefore so is the function $t\mapsto|\nabla_{\hat\alpha\circ\varsigma(t)}f|=|\nabla_{\alpha(t)}f|$.
Hence, for any $t_0\in[0,t_{\max})$ we have
\begin{align*}\varsigma^+(t_0)
&=
\lim_{t\to t_0+}
\oint\limits_{t_0}^t
|\nabla_{\alpha(\under t)}f|\cdot\dd\under t
=
\\
&=
|\nabla_{\alpha(t_0)}f|.
\end{align*}
Thus, we have 
\begin{align*}\alpha^+(t_0)
&=
\varsigma^+(t_0)\cdot\hat\alpha^+(\varsigma(t_0))
=
\\
&=
\nabla_{\alpha(t_0)} f.
\end{align*}
\qedsf


\begin{thm}{Exercise}\label{ex:grad-curve-condition}
Let $\spc{Z}$ be a complete length space, and 
$f\:\spc{Z}\to \RR$ be a semiconcave locally Lipschitz 
function.
Suppose that $\spc{Z}$ is either $\Alex{}$ or $\CAT{}$.
Assume $\alpha\:\II\to \spc{Z}$ is a Lipschitz curve such that 
\begin{align*}
\alpha^+(t)&\le|\nabla_{\alpha(t)}f|,
\\
(f\circ\alpha)^+(t)&\ge |\nabla_{\alpha(t)}f|^2
\end{align*}
for almost all $t$.
Show that $\alpha$ is an $f$-gradient curve.
\end{thm}



\begin{thm}{Exercise}\label{ex:grad-curve-analitic}
Let $\spc{Z}$ be a complete length space and $f\:\spc{Z}\to\RR$ be a concave locally Lipschitz function.
Suppose that $\spc{Z}$ is either $\Alex{}$ or $\CAT{}$.
Show that $\alpha\:\RR\to\spc{Z}$ is an $f$-gradient curve if and only if
\[\dist[2]{x}{\alpha(t_1)}{\spc{Z}}-\dist[2]{x}{\alpha(t_0)}{\spc{Z}}
\le 
2\cdot(t_1-t_0)\cdot  (f\circ\alpha(t_1)-f(x))\]
for any $t_1>t_0$ and $x\in\spc{Z}$.
\end{thm}



\section*{Distance estimates}\label{sec:grad-curv:dist-est}

\begin{thm}{First distance estimate}\label{thm:dist-est}
Let $\spc{Z}$ be a complete length space, and 
$f\:\spc{Z}\to \RR$ be a locally Lipschitz 
 $\lambda$-concave function.
Suppose that $\spc{Z}$ is either $\Alex{}$ or $\CAT{}$.
Let $\alpha,\beta\:[0,t_{\max})\to \spc{Z}$ be two $f$-gradient curves.
Then
\[\dist{\alpha(t)}{\beta(t)}{}
\le 
e^{\lambda\cdot t}\cdot\dist[{{}}]{\alpha(0)}{\beta(0)}{}\]
for any $t$.

Moreover, the statement holds for a locally Lipschitz $\lambda$-concave subfunction $f\:\spc{Z}\subto \RR$ if  there is a geodesic $[\alpha(t)\,\beta(t)]$ in $\Dom f$ for any~$t$.
\end{thm}

\parit{Proof.} 
If $\spc{Z}$ is not geodesic, then pass to its ultrapower $\spc{Z}^\o$.

Fix a choice of geodesic $[\alpha(t)\,\beta(t)]$ for each $t$.

Setting $\ell(t)=\dist{\alpha(t)}{\beta(t)}{}$, from the first variation inequality (\ref{lem:first-var}) and the estimate in \ref{cor:grad-lip} we get
\[\ell^+(t)\le-\<\dir{\alpha(t)}{\beta(t)},\nabla_{\alpha(t)}f\>-\<\dir{\beta(t)}{\alpha(t)},\nabla_{\beta(t)}f\>\le \lambda\cdot\ell(t).\]
Here one has to choose a midpont $p$ of $[\alpha(t)\,\beta(t)]$, apply the first variation inequality for distance to $p$, and apply the triangle inequality.
Hence the result. 
\qeds

\begin{thm}{Second distance estimate}\label{lem:fg-dist-est}
Let $\spc{Z}$ be a complete length space, 
$\eps>0$,  
and $f,g\:\spc{Z}\to \RR$ be two $\lambda$-concave locally Lipschitz function such that $|f-g|<\eps$.
Suppose that $\spc{Z}$ is either $\Alex{}$ or $\CAT{}$.
Assume
$\alpha,\beta\:[0,t_{\max})\to \spc{Z}$ are respectively $f$- and $g$-gradient curves such that $\alpha(0)=\beta(0)$.
Then 
\[\dist{\alpha(t)}{\beta(t)}{}
\le
\sqrt{\tfrac{1}{2\cdot\eps\cdot\lambda}
\cdot
\left(e^{\frac{t\cdot\lambda}\eps}-1\right)}\]
for any $t\in[0,t_{\max})$.
In particular, if $t_{\max}<\infty$ then
\[\dist{\alpha(t)}{\beta(t)}{}
\le
\Const\cdot\sqrt{\eps\cdot t}\]
for some constant $\Const=\Const(t_{\max},\lambda)$.

Moreover, the same conclusion holds for locally Lipschitz  $\lambda$-concave subfunctions $f,g\:\spc{Z}\subto \RR$ if for any $t\in[0,t_{\max})$ there is a geodesic $[\alpha(t)\,\beta(t)]$ in $\Dom f\cap\Dom g$.
\end{thm}

\parit{Proof.} Set $\ell=\ell(t)=\dist{\alpha(t)}{\beta(t)}{}$.
Fix $t$, and let $p=\alpha(t)$ and $q=\beta(t)$.
From the first variation formula and \ref{lem:grad-lip},
\begin{align*}
 \ell^+
&\le -\<\dir{p}{q},\nabla_{p}f\>
-\<\dir{q}{p},\nabla_{q}g\>
\le
\\
&\le -{\left({f(q)}-{f(p)}-\lambda\cdot\tfrac{\ell^2}2\right)}/{\ell}
-{\left({g(p)}-{g(q)}-\lambda\cdot\tfrac{\ell^2}2\right)}/{\ell}\le
\\
&\le \lambda\cdot\ell+\tfrac{2\cdot\eps}{\ell}
.
\end{align*}
Integrating the above estimate, we get
\[\ell(t)
\le
\sqrt{\tfrac{1}{2\cdot\eps\cdot\lambda}
\cdot\left(e^{\frac{t\cdot\lambda}\eps}-1\right)}.\]
\qedsf




\section*{Existence, uniqueness, convergence}
\label{sec:grad-curv:exist}

In general, the ``past'' of gradient curves can not be determined by the present.
For example, consider the concave function $f\:\RR\to\RR$, $f(x)\z=-|x|$. The 
two curves $\alpha(t)=\min\{0,t\}$ with $\beta(t)=0$
are $f$-gradient with $\alpha(t)\z=\beta(t)\z=0$ for all $t\ge0$; 
however $\alpha(t)\z\ne\beta(t)$ for all $t<0$.

The next theorem shows that a ``future'' gradient curve is unique.

\begin{thm}{Picard's theorem}\label{thm:picard}
Let $\spc{Z}$ be a complete length space
and
$f\:\spc{Z}\subto \RR$ be a semiconcave subfunction.
Suppose that $\spc{Z}$ is either $\Alex{}$ or $\CAT{}$.
Assume $\alpha,\beta\:[0,t_{\max})\to\Dom f$ are two $f$-gradient curves 
such that $\alpha(0)=\beta(0)$.
Then $\alpha(t)=\beta(t)$ for any $t\in[0,t_{\max})$.
\end{thm}

\parit{Proof.} This follows from the first distance estimate (\ref{thm:dist-est}).\qeds

\begin{thm}{Local existence}\label{thm:exist-grad-curv}
Let $\spc{Z}$ be a complete length space 
and $f\:\spc{Z}\subto \RR$ be locally Lipschitz $\lambda$-concave subfunction.
Suppose that $\spc{Z}$ is either $\Alex{}$ or $\CAT{}$.
Then for any $p\in \Dom f$,
\begin{subthm}{}
if $|\nabla_pf|>0$, then for some $\eps>0$, 
there is an $f$-gradient-like curve $\hat\alpha\:[0,\eps)\to\spc{Z}$ that starts at $p$ (that is, $\hat\alpha(0)\z=p$);
\end{subthm}

\begin{subthm}{}for some $\delta>0$, there is an $f$-gradient curve $\alpha\:[0,\delta)\to \spc{Z}$ that starts at $p$ (that is $\alpha(0)=p$).
\end{subthm}
\end{thm}

This theorem was proved by Grigory Perelman and the third author \cite{perelman-petrunin:qg};
we present a simplified proof given by Alexander Lytchak \cite{lytchak:open-map}.

\parit{Proof.} 
If $|\nabla_p f|=0$, then the constant curve $\alpha(t)=p$ is $f$-gradient.

Otherwise, choose $\eps>0$ 
such that $\oBall(p,\eps)\subset\Dom f$,
the restriction $f|_{\oBall(p,\eps)}$ is Lipschitz, 
and $|\nabla_x f|>\eps$ for all $x\in \oBall(p,\eps)$;
the latter is possible due to semicontinuity of \textbar gradient\textbar\ (\ref{cor:gradlim}).

The curves $\hat\alpha$ and $\alpha$ will be constructed in the following three steps.
First we construct an $f^\o$-gradient-like curve $\hat\alpha_\o\:[0,\eps)\to\spc{Z}^\o$ as an $\o$-limit of a certain sequence of polygonal lines in $\spc{Z}$.
Second, we parametrize $\hat\alpha_\o$ as in \ref{lem:grad--grad-like}, to obtain an $f^\o$-gradient curve $\alpha_\o$ in $\spc{Z}^\o$.
Third, applying Picard's theorem (\ref{thm:picard}) together with Lemma~\ref{lem:X-X^w}, we obtain that $\alpha_\o$ lies in $\spc{Z}\subset \spc{Z}^\o$ and therefore one can take $\alpha=\alpha_\o$ and $\hat\alpha=\hat\alpha_\o$.

Note that if $\spc{Z}$ is proper, then $\spc{Z}=\spc{Z}^\o$ and $f^\o=f$.
Thus, in this case, the third step is not necessary.

\parit{Step 1.}
Given $n\in \NN$, 
by an open-closed argument,
we can construct a unit-speed curve $\hat\alpha_n\:[0,\eps] \to \spc{Z}$ starting at $p$, with a partition of $[0,\eps)$ into a countable number of half-open intervals $[\varsigma_i,\bar\varsigma_i)$ 
such that for each $i$ we have 
\begin{enumerate}[(i)]
\item $\hat\alpha_n([\varsigma_i,\bar\varsigma_i])$ is a geodesic and $\bar\varsigma_i-\varsigma_i<\tfrac{1}{n}$,
\item\label{alm-grad} 
$f\circ\hat\alpha_n(\bar\varsigma_i)-f\circ\hat\alpha_n(\varsigma_i)
>
(\bar\varsigma_i-\varsigma_i)
\cdot
(|\nabla_{\hat\alpha_n(\varsigma_i)}f|-\tfrac{1}{n}).$
\end{enumerate}

Passing to a subsequence of $\hat\alpha_n$ such that $f\circ\hat\alpha_n$ uniformly converges, let 
\[h(s)=\lim_{n\to\infty}f\circ\hat\alpha_n(s).\]

Let $\hat\alpha_\o=\lim_{n\to\o}\hat\alpha_{n}$, 
 a curve in $\spc{Z}^\o$ that starts at $p\in \spc{Z}\subset \spc{Z}^\o$.

Clearly $\hat\alpha_\o$ is $1$-Lipschitz.
From (\ref{alm-grad}) and \ref{lem:gradcon}, we have
\[(f^\o\circ\hat\alpha_\o)^+(\varsigma)
\ge
|\nabla_{\hat\alpha_\o(\varsigma)}f^\o|.\]
According to \ref{thm:grad-like-2nd-def}, $\hat\alpha_\o\:[0,\eps)\to \spc{Z}^\o$  is an $f^\o$-gradient-like curve.

\parit{Step 2.}
Clearly $h(s)=f^\o\circ\alpha_\o$. 
Therefore, according to \ref{thm:concave}, $h$ is $\lambda$-concave.
Thus we can define a homeomorphism $\varsigma\:[0,\delta]\to[0,\eps]$ by 
\[{\varsigma^{-1}(s)}
=
\int\limits_0^{s}\frac{\dd\under s}{h'(\under s)},
\eqlbl{eq:thm:exist-grad-curv-1}\]

According to \ref{lem:grad--grad-like}, $\alpha(t)=\hat\alpha\circ\varsigma(t)$ is an $f^\o$-gradient curve in $\spc{Z}^\o$. 

\parit{Step 3.}
Clearly, $\nabla_p f=\nabla_p f^\o$ for any $p\in \spc{Z}\subset \spc{Z}^\o$;
more formally, if $\iota\:\spc{Z}\hookrightarrow\spc{Z}^\o$ is the natural embedding, then
$(\dd_p\iota)(\nabla_p f)=\nabla_p f^\o$.
Thus it is sufficient to show that $\alpha_\o$ lies in $\spc{Z}$.
Assume the contrary; then according to \ref{lem:X-X^w}, there is a subsequence $\hat\alpha_{n_\kay}$ such that
\[\hat\alpha_\o\not
=
\hat\alpha'_\o
\df
\lim_{\kay\to\o}\hat\alpha_{n_\kay}.\]
Clearly $h(s)=f^\o\circ\hat\alpha_\o=f^\o\circ\hat\alpha'_\o$.
Thus for $\varsigma\:[0,\delta]\to[0,\eps]$ defined by \ref{eq:thm:exist-grad-curv-1}, 
we have that both curves
$\hat\alpha_\o\circ\varsigma$ and $\hat\alpha'_\o\circ\varsigma$ are $f^\o$-gradient.
From Picard's theorem (\ref{thm:picard}), we have $\hat\alpha_\o\circ\varsigma=\hat\alpha'_\o\circ\varsigma$.
Therefore $\hat\alpha_\o=\hat\alpha'_\o$, a contradiction.
\qeds

\begin{thm}{Ultralimit of gradient curves}\label{ultr-lim-g-curve}
Assume
\begin{itemize}
\item $\spc{Z}_n$ is a sequence of complete spaces, $\spc{Z}_n \to \spc{Z}_\o$ as $n\to\o$, and $p_n\to p_\omega$ for a sequence of points $p_n\in \spc{Z}_n$,
\item all spaces $\spc{Z}_n$ are either $\Alex\kappa$ or $\CAT\kappa$, 
\item $f_n\:\spc{Z}_n\subto \RR$ are $\Lip$-Lipschitz and $\lambda$-concave,
$f_n\to f_\o$ as $n\to\o$, and $p_\o\in\Dom f_\o$.
\end{itemize}

Then: 

\begin{subthm}{thm:convex-limit-cbb}
$f_\o$ is $\lambda$-concave.
\end{subthm}


\begin{subthm}{lim-grad-like}
If $|\nabla_{p_\o}f_\o|>0$, then there is $\eps>0$ such that, the $f_n$-gradient-like curves $\hat\alpha_n\:[0,\eps)\to\spc{Z}_n$ are defined for $\o$-almost all $n$.
Moreover, a curve $\hat\alpha_\o\:[0,\eps)\to\spc{Z}_\o$ is a gradient-like curve that starts at $p_\o$ if and only if
$\hat\alpha_n(s)\to\hat\alpha_\o(s)$ as $n\to\o$ for all $s\in[0,\eps)$.
\end{subthm}

\begin{subthm}{lim-grad}
For some $\delta>0$, the $f_n$-gradient curves $\alpha_n\:[0,\delta)\to\spc{Z}_n$ are defined for $\o$-almost all $n$.
Moreover, a curve $\alpha_\o\:[0,\delta)\to\spc{Z}_\o$ is a gradient curve that starts at $p_\o$ if and only if
$\alpha_n(t)\to\alpha_\o(t)$  as $n\to\o$ for all $t\in[0,\delta)$.
\end{subthm}
\end{thm}

%A: Do we need a convergence theorem, which would work for radial curves as well??

Note that according to Exercise~\ref{ex:nonconvex-limit}, part (\ref{SHORT.thm:convex-limit-cbb}) does not hold for general metric spaces.
The idea of the proof is the same as in the proof of local existence (\ref{thm:exist-grad-curv}).

\parit{Proof of \ref{ultr-lim-g-curve}; (\ref{SHORT.thm:convex-limit-cbb})}
Fix a geodesic $\gamma_\o\:\II\to \Dom f_\o$;
we need to show that the function 
\[t\mapsto f_\o\circ\gamma_\o(t)-\tfrac\lambda 2\cdot t^2\eqlbl{eq:lambda-concave}\]
is concave.


Since the $f_n$ are $\Lip$-Lipschitz, so is $f_\o$.
Therefore it is sufficient to prove concavity in the interior of $\II$.
In particular we can assume that $\gamma_\o$ is sufficiently short and can be extended behind its ends $p_\o$ and $q_\o$ as a minimizing geodesic.
If $\spc{Z}$ is $\Alex{}$, then by Theorem~\ref{thm:almost.geod}, $\gamma_\o$ is the unique geodesic connecting $p_\o$ to $q_\o$.
The same holds true if $\spc{Z}$ is $\CAT{}$ by the uniqueness of geodesics (\ref{thm:cat-unique}).

Construct two sequences of points $p_n,q_n\in\spc{Z}_n$ such that $p_n\to p_\o$ and $q_n\to q_\o$ as $n\to \o$.
Applying either \ref{thm:almost.geod} or \ref{thm:cat-unique},
we can assume that for each $n$ there is a geodesic $\gamma_n$ from $p_n$ to $q_n$ in $\spc{Z}_n$.

Since $f_n$ is $\lambda$-concave, the function 
\[t\mapsto f_n\circ\gamma_n(t)-\tfrac\lambda 2\cdot t^2\]
is concave.

The $\o$-limit of the sequence $\gamma_n$ is a geodesic in $\spc{Z}_\o$ from $p_\o$ to $q_\o$.
By uniqueness of such geodesics, we have that $\gamma_n\to \gamma_\o$ as $n\to \o$.
Passing to the limit, we have \ref{eq:lambda-concave}.

\parit{If part of (\ref{SHORT.lim-grad-like}).}
Take $\eps>0$ so small that $\oBall(p_\o,\eps)\subset\Dom f_\o$ and $|\nabla_{x_\o}f_\o|>0$ for any $x_\o\in\oBall(p_\o,\eps)$ (this is possible by \ref{cor:gradlim}).

Clearly $\hat\alpha_\o$ is $1$-Lipschitz.
From \ref{lem:gradcon}, we get 
$(f_\o\circ\hat\alpha_\o)^+(s)
\ge
|\nabla_{\hat\alpha_\o(s)}f^\o|$.
According to \ref{thm:grad-like-2nd-def}, $\hat\alpha_\o\:[0,\eps)\to \spc{Z}^\o$  is an $f_\o$-gradient-like curve.

\parit{If part of (\ref{SHORT.lim-grad}).}
Assume first that $|\nabla_{p_\o}f_\o|>0$, 
so we can apply the if part of (\ref{SHORT.lim-grad-like}).
Let $h_n=f_n\circ\hat\alpha_n\:[0,\eps)\to \RR$ 
and $h_\o=f_\o\circ\hat\alpha_\o$.
From \ref{thm:concave}, the $h_n$ are $\lambda$-concave, and clearly $h_n\to h_\o$ as $n\to\o$.
Let us define reparametrizations
\begin{align*}
{\varsigma_n^{-1}(s)}
&=
\int\limits_0^{s}\frac{\dd\under s}{h_n'(\under s)},
&
{\varsigma_\o^{-1}(s)}
&=
\int\limits_0^{s}\frac{\dd\under s}{h_\o'(\under s)}.
\end{align*}
The $\lambda$-convexity of the $h_n$ implies that $\sigma_n\to\sigma_\o$ as $n\to\o$.
By \ref{lem:grad--grad-like}, 
$\alpha_n=\hat\alpha_n\circ\varsigma_n$.
Applying the if part of (\ref{SHORT.lim-grad-like}) together with Lemma~\ref{lem:grad--grad-like},
we get that $\alpha_\o=\hat\alpha_\o\circ\varsigma_\o$ is gradient curve.

The remaining case $|\nabla_{p_\o}f_\o|=0$ can be reduced to the one above using the following trick.
Consider the sequence of spaces $\spc{Z}_n^{\times}=\spc{Z}_n\times\RR$,
with the sequence of subfunctions $f^{\times}_n\:\spc{Z}_n^{\times}\to\RR$ defined by
\[f^{\times}_n(p,t)=f_n(p)+t.\]
Applying either \ref{thm:warp-curv-bound:cbb:E} or \ref{thm:cbb-product},  we have that
$\spc{Z}_n^{\times}$ is a $\Alex{\kappa_-}$ space for $\kappa_-=\min\{\kappa,0\}$, or $\CAT{\kappa_+}$ space for $\kappa_+=\max\{\kappa,0\}$.
Note that the $f_n^{\times}$ are $\lambda_+$-concave
for $\lambda_+=\max\{\lambda,0\}$.
Now let $\spc{Z}_\o^{\times}=\spc{Z}_\o\times\RR$,
and $f^{\times}_\o(p,t)=f_\o(p)+t$.

Clearly 
$\spc{Z}_n^{\times}\to\spc{Z}_\o^{\times}$,
$f_n^{\times}\to f_\o^{\times}$ as $n\to\o$,
and $|\nabla_xf^{\times}_\o|>0$ for any $x\in\Dom f_\o^{\times}$.
Thus for the sequence $f_n^{\times}\:\spc{Z}_n^{\times}\subto\RR$, 
we can apply the if part of (\ref{SHORT.lim-grad-like}).
It remains to note that the curve $\alpha^{\times}_\o(t)=(\alpha_\o(t),t)$ is an $f^{\times}_\o$-gradient curve in $\spc{Z}^{\times}_\o$ 
if and only if $\alpha_\o(t)$ is an $f_\o$-gradient curve.

\parit{Only-if part of (\ref{SHORT.lim-grad}) and (\ref{SHORT.lim-grad-like}).}
The only-if part of (\ref{SHORT.lim-grad}) follows from
the if part of (\ref{SHORT.lim-grad}) and Picard's theorem (\ref{thm:picard}).
Applying Lemma~\ref{lem:grad--grad-like}, we get the only-if part of (\ref{SHORT.lim-grad-like}).
\qeds

Directly from  local existence (\ref{thm:exist-grad-curv}) and the distance estimates (\ref{thm:dist-est}), we obtain:

\begin{thm}{Global existence}\label{thm:glob-exist-grad-curv}
Let $f\:\spc{Z}\subto \RR$ be a locally Lipschitz and $\lambda$-concave subfunction on a complete length space $\spc{Z}$.
Suppose that $\spc{Z}$ is either $\Alex{}$ or $\CAT{}$.
Then for any $p\in \Dom f$, there is $t_{\max}\in(0,\infty]$ such that
there is an $f$-gradient curve $\alpha\:[0,t_{\max})\to \spc{Z}$ with $\alpha(0)=p$.
Moreover, for any sequence $t_n\to t_{\max}-$, the sequence $\alpha(t_n)$ does not have a limit point in $\Dom f$.
\end{thm}


The following theorem guarantees the existence of gradient curves for all times for the special type of semiconcave functions that play important role in the theory.
It follows from \ref{thm:glob-exist-grad-curv},
\ref{thm:concave} and \ref{lem:grad--grad-like}.

\begin{thm}{Theorem}\label{thm:comp-grad-test}
Let $\spc{Z}$ be a complete length space 
and $f\:\spc{Z}\to\RR$ satisfies 
\[f''+\kappa\cdot f\le \lambda\] 
for some real values $\kappa$ and $\lambda$.
Suppose that $\spc{Z}$ is either $\Alex{}$ or $\CAT{}$.
Then $f$ has \emph{complete gradient};
that is, for any $x\in\spc{Z}$ there is a $f$-gradient curve $\alpha\:[0,\infty)\to\spc{Z}$ that starts at~$x$.
\end{thm}



















\section{Gradient flow}\label{sec:Gradient flow}

In this section we define gradient flow for semiconcave subfunctions 
and reformulate theorems obtained earlier in this chapter using this new terminology.

Let $\spc{Z}$ be a complete length space 
and $f\:\spc{Z}\subto \RR$ be a locally Lipschitz semiconcave subfunction.
Suppose that $\spc{Z}$ is either $\Alex{}$ or $\CAT{}$.
For any $t\ge 0$, we write $\GF^t_f(x)=y$ if there is an $f$-gradient curve $\alpha$ such that $\alpha(0)=x$ and $\alpha(t)=y$.
The partially defined map $\GF^t_f$ from $\spc{Z}$ to itself is called the \index{gradient flow}\emph{$f$-gradient flow} for time $t$.
 
From \ref{lem:fg-dist-est}, 
it follows that for any $t\ge 0$, the domain of definition of $\GF^t_f$ is an open subset of $\spc{Z}$; 
that is, $\GF^t_f$ is a submap.
Moreover, if $f$ is defined on all of $\spc{Z}$ and $f''+\Kappa\cdot f\le \lambda$ for some $\Kappa,\lambda\in\RR$, 
then according to \ref{thm:comp-grad-test}, $\GF^t_f(x)$ is defined for all pairs $(x,t)\in\spc{Z}\times\RR_{\ge0}$.

Clearly $\GF^{t_1+t_2}_f=\GF_f^{t_1}\circ\GF_f^{t_2}$;
in other words, gradient flow is given by an action of the semigroup $(\RR_{\ge0},+)$.

From the first distance estimate (\ref{thm:dist-est}),
we have the following:

\begin{thm}{Proposition}\label{prop:GF-is-lip}
Let $\spc{Z}$ be a complete length $\Alex{}$ space 
and $f\:\spc{Z}\to \RR$ be a semiconcave function.
Then the map $x\mapsto\GF^t_f(x)$ is locally Lipschitz.

Moreover, if $f$ is $\lambda$-concave, then $\GF^t_f$ is $e^{\lambda\cdot t}$-Lipschitz.
\end{thm}

The next proposition states that gradient flow is stable under Gromov--Hausdorff convergence.
The proposition follows directly from the proposition on ultralimit of gradient curves~\ref{ultr-lim-g-curve}.

\begin{thm}{Proposition}\label{grad-curve-conv}
If $\spc{Z}_n$ is an $m$-dimensional complete length $\Alex\kappa$ space, $\spc{Z}_n\xto{\GH} \spc{Z}$, and $f_n\:\spc{Z}_n\to\RR$ is a sequence of
$\lambda$-concave functions that converges to $f\:\spc{Z}\to \RR$, then
$\GF_{f_n}^t\:\spc{Z}_n\to \spc{Z}_n$ converges to $\GF_f^t\:\spc{Z}\to \spc{Z}$.
\end{thm}%do we need it for ultralimits?

%\begin{thm}{Exercise}\label{ex:grad-flow-bry}
%Let $\spc{L}$ be an $m$-dimensional complete length $\Alex{}$ space, $\partial\spc{L}=\emptyset$, $K\subset \spc{L}$ be a compact subset, and $f\:\spc{L}\to\RR$ be semiconcave function.
%Assume that for some $t>0$ the gradient flow $\GF^t_f$ is defined everywhere in $K$.
%Prove that 
%$$\partial_{\spc{L}}\GF^t_fK\subset\GF^t_f\Fr K.$$
%\end{thm} %finite-dimensional problem


\section{Line splitting theorem}
 

Let $\spc{X}$ be a metric space and $A,B\subset \spc{X}$.
We will write 
\[\spc{X}=A\oplus B\]
if there exist projections $\proj_A\:\spc{X}\to A$ 
and 
$\proj_B\:\spc{X}\to B$
such that 
\[\dist[2]{x}{y}{}=\dist[2]{\proj_A(x)}{\proj_A(y)}{}+\dist[2]{\proj_B(x)}{\proj_B(y)}{}\]
for any two points $x,y\in \spc{X}$.

Note that if 
\[\spc{X}=A\oplus B\]
then 
\begin{itemize}
\item $A$ intersects $B$ at a single point,
\item both sets $A$ and $B$ are convex sets in $\spc{X}$.
\end{itemize}

Recall that a line in a metric space is a both-sided infinite geodesic; thus it minimizes the length on each segment.

 {\sloppy 

\begin{thm}{Line splitting theorem}\label{thm:splitting}
Let $\spc{L}$  be a complete length $\Alex{0}$ space
and $\gamma$ be a line in $\spc{L}$. 
Then 
\[\spc{L}=X\oplus \gamma(\RR)\]
for some subset $X\subset \spc{L}$.
\end{thm}

}

For smooth $2$-dimensional surfaces, 
this theorem was proved by Stefan Cohn-Vossen \cite{cohn-vossen_line}.
For Riemannian manifolds of higher dimensions 
it was proved by Victor Toponogov \cite{toponogov-globalization+splitting}.
Then it was generalized by Anatoliy Milka \cite{milka-line}
to Alexandrov spaces; almost the same proof is given in \cite[1.5]{burago-burago-ivanov}.

Further generalizations of the splitting theorem for Riemannian manifolds with nonnegative Ricci curvature were obtained by Jeff Cheeger and Detlef Gromoll \cite{cheeger-gromoll-split}.
This was further generalized by Jeff Cheeger and Toby Colding for limits of Riemannian manifolds with almost nonnegative Ricci curvature \cite{cheeger-colding-alm-rigidity} and to their synthetic generalizations, so-called RCD spaces, by Nicola Gigli \cite{gigli2013splitting, gigli-splitting-overview}.
Jost-Hinrich Eschenburg obtained an analogous result for  Lorentzian manifolds \cite{eshenburg-split}, that is, pseudo-Riemannian manifolds of signature $(1,n)$.

We present a proof that uses gradient flow for Busemann functions. 
It is close in spirit to the proof given in \cite{cheeger-gromoll-split}.

Before going into the proof, let us state a few corollaries of the theorem.

\begin{thm}{Corollary}\label{cor:splitting}
Let $\spc{L}$ be a complete length $\Alex{0}$ space. 
Then there is an isometric splitting
\[
\spc{L}=\spc{L}'\oplus H
\]
where $H\subset \spc{L}$ is a subset isometric to a Hilbert space, and $\spc{L}'\subset \spc{L}$ is a convex subset that contains no line. 
\end{thm}

 {\sloppy 

\begin{thm}{Corollary}\label{cor:splitting-vectors}
Let $\spc{K}$ be an $n$-dimensional complete length $\Alex0$ cone and $v_+,v_-\in \spc{K}$ be a pair of opposite vectors 
(that is, $v_+ + v_-=0$, see Definiton~\ref{def:opp+Lin}).
Then there is an isometry $\iota\:K\to K'\times \RR$, where $K'$ is a complete length $\Alex0$ space having a cone structure with tip $\0'$ such that
$\iota:v_\pm\mapsto (\0',\pm|v_\pm|)$.
\end{thm}

}

\begin{thm}{Corollary}\label{cor:splitting-CBB[1]}
Assume $\spc{L}$ is an $m$-dimensional complete length $\Alex1$ space, $m\ge2$, and $\rad\spc{L}=\pi$.
Then \[\spc{L}\iso \mathbb{S}^m.\]
 
\end{thm}

The following lemma is closely relevant to the first distance estimate (\ref{thm:dist-est}); its proof goes along the same lines.

\begin{thm}{Lemma}\label{lem:dist-estimate}
Suppose $f\:\spc{L}\to\RR$ be a concave 1-Lipschitz function.
Consider two $f$-gradient curves $\alpha$ and $\beta$.
Then for any $t, s\ge 0$ we have
\begin{align*}
&\dist[2]{\alpha(s)}{\beta(t)}{}
\le 
\dist[2]{p}{q}{}+
2\cdot(f(p)-f(q))\cdot(s-t)+ (s-t)^2,
\end{align*}
where $p=\alpha(0)$ and $q=\beta(0)$.
\end{thm}

\parit{Proof.}
If $\spc{L}$ is not geodesic, then pass to its ultrapower $\spc{L}^\o$.

Since $f$ is 1-Lipschitz, $|\nabla f|\le1$.
Therefore 
\[f\circ\beta(t)\le f(q)+t\]
for any $t\ge0$.

Set $\ell(t)=\dist{p}{\beta(t)}{}$.
Applying \ref{lem:grad-lip:lam=0} and the first variation inequality (\ref{lem:first-var}), we get
\begin{align*}
\ell^2(t)'
&\le 2\cdot \left(f\circ\beta(t)-f(p)\right)\le 
\\
&\le2\cdot\left(f(q)+t-f(p)\right).
\end{align*}
Therefore 
\[\ell^2(t)-\ell^2(0)\le 2\cdot\left(f(q)-f(p)\right)\cdot t + t^2.\]
It proves the needed inequality in case $s=0$.
Combining it with the first distance estimate (\ref{thm:dist-est}), we get the result in case $s\le t$.
The case $s\ge t$ follows by switching the roles of $s$ and $t$..
\qeds


\parit{Proof of \ref{thm:splitting}.} Consider two Busemann functions, $\bus_+$ and $\bus_-$, asociated with half-lines $\gamma:[0,\infty)\to \spc{L}$ and $\gamma:(-\infty,0]\to \spc{L}$ respectively; that is,
\[
\bus_\pm(x)
=
\lim_{t\to\infty}\dist{\gamma(\pm t)}{x}{}- t.
\]
According to Exercise~\ref{ex:busemann-CBB}, 
both functions $\bus_\pm$ are concave.

Fix $x\in \spc{L}$.
Note that since $\gamma$ is a line, we have 
\[\bus_+(x)+\bus_-(x)\ge0.\]

On the other hand, by \ref{comp-kappa}, 
$f(t)=\distfun[2]{x}{}{}\circ\gamma(t)$ 
is $2$-concave.
In particular, $f(t)\le t^2+at+b$ for some constants $a,b\in\RR$. 
Passing to the limit as $t\to\pm\infty$, we have \[\bus_+(x)+\bus_-(x)\le0.\]

Hence
\[
\bus_+(x)+\bus_-(x)= 0
\]
for any $x\in \spc{L}$.
In particular the functions $\bus_\pm$ are affine;
that is, they are convex and concave at the same time.

It follows that for any $x$,
\begin{align*}
|\nabla_x \bus_\pm|
&=\sup\set{\dd_x\bus_\pm(\xi)}{\xi\in\Sigma_x}=
\\
&=\sup\set{-\dd_x\bus_\mp(\xi)}{\xi\in\Sigma_x}\equiv
\\
&\equiv1.
\end{align*}
By Exercise~\ref{ex:grad-curve-condition}, a 
$1$-Lipschitz curve $\alpha$ such that $\bus_\pm(\alpha(t))=t+\Const$ is a $\bus_\pm$-gradient curve. 
In particular, $\alpha(t)$ is a $\bus_+$-gradient curve if and only if $\alpha(-t)$ is a $\bus_-$-gradient curve.
It follows that for any $t>0$, the $\bus_\pm$-gradient flows commute;
that is, 
\[\GF_{\bus_+}^t\circ\GF_{\bus_-}^t=\id_\spc{L}.\]
Setting
\[\GF^t=\left[\begin{matrix}
\GF_{\bus_+}^t&\hbox{if}\ t\ge0\\
\GF_{\bus_-}^t&\hbox{if}\ t<0
               \end{matrix}\right.\]
defines an $\RR$-action on $\spc{L}$.

Consider the level set $\spc{L}'=\bus_+^{-1}(0)=\bus_-^{-1}(0)$;
it is a closed convex subset of $\spc{L}$, and therefore forms an Alexandrov space.
Consider the map $h\:\spc{L}'\times \RR\to \spc{L}$ defined by $h\:(x,t)\mapsto \GF^t(x)$.
Note that $h$ is onto.
Applying Lemma \ref{lem:dist-estimate} for $\GF_{\bus_+}^t$ and $\GF_{\bus_-}^t$ shows that $h$ is short and non-contracting at the same time; that is, $h$ is an isometry.
\qeds



\section{Radial curves}\label{sec:Radial curves: definition}

The radial curves are specially reparametrized gradient curves for distance functions.
This parametrization makes them behave like unit-speed geodesics in a natural comparison sense (\ref{sec:Radial comparisons}).

\begin{thm}{Definition}\label{def:rad-curv}
Assume $\spc{L}$ is a complete length $\Alex{}$ space, 
$\kappa\in\RR$, 
and $p\in \spc{L}$.
A curve 
$$\sigma\:[s_{\min},s_{\max})\to \spc{L}$$  
is called a 
\emph{$(p,\kappa)$-radial curve} 
if
$$s_{\min}
\z=
\dist{p}{\sigma(s_{\min})}{}\in(0,\tfrac{\varpi\kappa}2)$$ 
and $\sigma$ satisfies the differential equation
\[\sigma^+(s)
\z=
\frac{\tg\kappa\dist[{{}}]{p}{\sigma(s)}{}}{\tg\kappa s}
\cdot
\nabla_{\sigma(s)}\distfun{p}{}{}
\eqlbl{eq:rad}\]
for any $s\in[s_{\min},s_{\max})$, where $\tg\kappa x\df\frac{\sn\kappa x}{\cs\kappa x}$.

If $x=\sigma(s_{\min})$, we say that $\sigma$ {}\emph{starts at}  $x$.
\end{thm}

Note that according to the definition, $s_{\max}\le\tfrac{\varpi\kappa}2$.

In the remainder of the chapter we will see that  $(p,\kappa)$-radial curves 
work best for $\Alex\kappa$ spaces.



\begin{thm}{Definition}\label{def:rad-geod}
Let $\spc{L}$ be a complete length $\Alex{}$ space
and $p\in\spc{L}$.
A unit-speed geodesic  $\gamma\:\II\to \spc{L}$  is called a
\index{radial geodesic}\emph{$p$-radial geodesic} if 
$\dist{p}{\gamma(s)}{}\equiv s$.
\end{thm}

The proofs of the following two propositions follow directly from the definitions. 

\begin{thm}{Proposition}\label{prop:rad-geod}
Let $\spc{L}$ be a complete length $\Alex{}$ space
and $p\in\spc{L}$.
Assume $\tfrac{\varpi\kappa}{2}
\ge 
s_{\max}$.
Then any $p$-radial geodesic 
$\gamma\:[s_{\min},s_{\max})
\to 
\spc{L}$ 
is a $(p,\kappa)$-radial curve.
\end{thm}

\begin{thm}{Proposition}\label{prop:dist<s}
Suppose $\spc{L}$ is a complete length $\Alex{}$ space, 
$p\in\spc{L}$ 
and $\sigma\:[s_{\min},s_{\max})\to \spc{L}$ is a $(p,\kappa)$-radial curve.
Then for any $s\in [s_{\min},s_{\max})$, 
we have $\dist{p}{\sigma(s)}{}\le s$.

Moreover, 
if for some $s_0$ we have $\dist{p}{\sigma(s_0)}{}= s_0$, 
then the restriction $\sigma|_{[s_{\min},s_0]}$ is a $p$-radial geodesic.
\end{thm}

\begin{thm}{Existence and uniqueness}\label{rad-curv-exist}
Let $\spc{L}$ be a complete length $\Alex{}$ space, 
$\kappa\in\RR$, 
$p\in\spc{L}$, 
and $x\in \spc{L}$.
Assume
$0
<
\dist{p}{x}{}
<
\tfrac{\varpi\kappa}2$.
Then there is a unique $(p,\kappa)$-radial curve $\sigma\:[\dist{p}{x}{},\tfrac{\varpi\kappa}2)\to \spc{L}$ 
that starts at $x$;
that is, $\sigma(\dist{p}{x}{})=x$.
\end{thm}


\parit{Proof; existence.}
Let \index{$\itg\kappa$} 
\[\itg\kappa\:[0,\tfrac{\varpi\kappa}2)\to\RR,
\quad 
\itg\kappa (t)=\int\limits_0^t\tg\kappa\under t\cdot\dd\under t.\]
Clearly $\itg\kappa$ is smooth and increasing.
From \ref{prop:conv-comp} it follows that the composition 
\[f=\itg\kappa\circ\distfun{p}{}{}\] 
is semiconcave in $\oBall(p,\tfrac{\varpi\kappa}2)$.

According to \ref{thm:exist-grad-curv}, there is an $f$-gradient curve $\alpha\:[0,t_{\max})\to \spc{L}$ defined on the maximal interval such that $\alpha(0)=x$.

Now consider a solution of the  differential equation for  $\tau(t)$, $\tau'\z=(\tg\kappa\tau)^2$, $\tau(0)=r$. 
Note that $\tau(t)$ is also a gradient curve  for the function $\itg\kappa$ defined on $[0,\tfrac{\varpi\kappa}2)$.
Direct calculations show that the composition $\alpha\circ\tau^{-1}$ 
is a $(p,\kappa)$-radial curve.

\parit{Uniqueness.} Assume $\sigma^1,\sigma^2$ are two $(p,\kappa)$-radial curves that start at $x$.
Then the compositions $\sigma^i\circ\tau$ both give $f$-gradient curves.
By Picard's theorem (\ref{thm:picard}), we have
$\sigma^1\circ\tau 
\equiv 
\sigma^2\circ\tau$.
Therefore $\sigma^1(s)=\sigma^2(s)$ 
for any $s\ge r$ such that both sides are defined.
\qeds

\section{Radial comparisons}\label{sec:Radial comparisons}

In this section we show that radial curves behave in a comparison sense
 like unit-speed geodesics.

\begin{thm}{Radial monotonicity}\label{rad-mon}
Let $\spc{L}$ be a complete length $\Alex{\kappa}$ space and
$p, q$ be distinct points in $\spc{L}$.
Let $\sigma\:  [s_{\min},\tfrac{\varpi\kappa}2)\to \spc{L}$
be a $(p,\kappa)$-radial curve.
Then the function 
\[s\mapsto 
\tangle\mc\kappa\{
\dist{q}{\sigma(s)}{};
\dist{p}{q}{},
s
\}\]
is nonincreasing in its entire domain of definition.
\end{thm}

Radial monotonicity implies the following by straightforward calculations.

\begin{thm}{Corollary}\label{cor:rad-comp}
Let $\kappa\le0$,
$\spc{L}$ be a complete $\Alex\kappa$ space,
and $p, q\in \spc{L}$.
Let $\sigma\:[s_{\min},\tfrac{\varpi\kappa}2)\to \spc{L}$ be a $(p,\kappa)$-radial curve.
Then for any $w\ge 1$, 
the function
\[
s\mapsto \tangle\mc\kappa\{\dist{q}{\sigma(s)}{};\dist{p}{q}{},w\cdot s\}
\]
is nonincreasing in its entire domain of definition.
\end{thm}


\begin{thm}{Radial comparison}\label{rad-comp}
Let $\spc{L}$ be a complete length $\Alex{\kappa}$ space 
and $p\in \spc{L}$.
Let $\rho\:  [r_{\min},\tfrac{\varpi\kappa}2)\to \spc{L}$
and    $\sigma\:[s_{\min},\tfrac{\varpi\kappa}2)\to \spc{L}$
be two $(p,\kappa)$-radial curves.
Let
\[\phi_{\min}=\angkk\kappa p{\rho(r_{\min})}{\sigma(s_{\min})}.
\]
Then for any $r\in[r_{\min},\tfrac{\varpi\kappa}2)$ and  $s\in[s_{\min},\tfrac{\varpi\kappa}2)$,
we have
\[
\tangle\mc\kappa\{\dist{\rho(r)}{\sigma(s)}{};r,s\}
\le \phi_{\min},
\]
or equivalently,
\[
\dist{\rho(r)}{\sigma(s)}{}
\le \side\kappa\{\phi_{\min};r,s\}.
\]

\end{thm}


We prove Theorems \ref{rad-mon} and \ref{rad-comp} simultaneously.
The proof is an application of \ref{lem:grad-lip} plus trigonometric manipulations.
We give a proof first in the simplest case $\kappa=0$, since that is easier to follow,
and then in the harder case $\kappa\ne 0$.
The arguments for both cases are nearly the same, 
but the case $\kappa\ne 0$ requires an extra idea.



\parit{Proof of \ref{rad-mon} and \ref{rad-comp} in case $\kappa=0$.}
Set
\begin{align*}
R=R(r)&=\dist{p}{\rho(r)}{},
\\
S=S(s)&=\dist{p}{\sigma(s)}{},
\\
\ell=\ell(r,s)&=\dist{\rho(r)}{\sigma(s)}{},
\\
\phi=\phi(r,s)&=\tangle\mc0\{\ell(r,s);r,s\}.
\end{align*}

\begin{figure}[!ht]
\vskip-0mm
\centering
\includegraphics{mppics/pic-1505}
\vskip0mm
\end{figure}


It will be sufficient to prove the following inequalities:
\[\tfrac{\partial^+}{\partial r}\phi(s_{\min},r)\le 0,\qquad
\tfrac{\partial^+}{\partial s}\phi(s,r_{\min})\le 0\leqno(*)\mc0_\phi\]
\[
s\cdot\tfrac{\partial^+}{\partial s}\phi
+
r\cdot\tfrac{\partial^+}{\partial r}\phi\le 0.
\leqno(**)\mc0_\phi
\]

The radial monotonicity follows from $(*)\mc0_\phi$.
The radial comparison follows from  $(*)\mc0_\phi$ and $(**)\mc0_\phi$.
Indeed, one can connect $(s_{\min},r_{\min})$ and $(s_0,r_0)$ in $[s_{\min},\infty)\times[r_{\min},\infty)$ 
by a join of a coordinate segment and a segment defined by $r/s=r_0/s_0$ as in the figure.
According to $(*)\mc0_\phi$ and $(**)\mc0_\phi$, $\phi$ does not increase while the  pair $(r,s)$ moves along this join with nondecreasing $r$ and $s$.
Thus $\phi(r_0,s_0)\le\phi(r_{\min},s_{\min})=\phi_{\min}$.

\begin{figure}[!ht]
\vskip-0mm
\centering
\includegraphics{mppics/pic-1510}
\vskip0mm
\end{figure}

It remains to show $(*)\mc0_\phi$ and $(**)\mc0_\phi$. 
First let us rewrite the inequalities $(*)\mc0_\phi$ and $(**)\mc0_\phi$ in an equivalent form:
\[
\begin{aligned}
\tfrac{\partial^+}{\partial s}\ell(s,r_{\min})
&\le 
\cos\tangle\mc0\{r_{\min};s,\ell\},
\\
\tfrac{\partial^+}{\partial r}\ell(s_{\min},r)
&\le 
\cos\tangle\mc0\{s_{\min};r,\ell\},
\end{aligned}
\leqno(*)\mc0_\ell
\]

\[
s\cdot\tfrac{\partial^+}{\partial s}\ell
+
r\cdot\tfrac{\partial^+}{\partial r}\ell\le 
 s\cdot\cos\tangle\mc0\{r;s,\ell\}
+
r\cdot\cos\tangle\mc0\{s;r,\ell\}=\ell.
\leqno(**)\mc0_\ell
\]

Let 
\[f=\tfrac{1}{2}\cdot\distfun[2]{p}{}{}.\leqno(A)\mc0\] 
Clearly $f$ is $1$-concave, and
\[\rho^+(r)=\tfrac{1}{r}\cdot\nabla_{\rho(r)} f\quad \text{and}\quad \sigma^+(s)=\tfrac{1}{s}\cdot\nabla_{\sigma(s)} f.\leqno(B)\mc0\]
Thus from \ref{lem:grad-lip}, we have
\[\tfrac{\partial^+}{\partial r}\ell
=
-\tfrac{1}{r}\cdot\<\nabla_{\rho(r)} f,\dir{\rho(r)}{\sigma(s)}\>
\le\frac{{\ell^2}+{R^2}-{S^2}}{2\cdot\ell\cdot r}.\leqno(C)\mc0\]
Since $R(r)\le r$ and $S(s_{\min})=s_{\min}$, then
\[
\begin{aligned}
\tfrac{\partial^+}{\partial r}\ell(r,s_{\min})
&\le
\frac{{\ell^2}+r^2-s_{\min}^2}{2\cdot\ell\cdot r}
=\\
&=
\cos\tangle\mc0\{s_{\min};r,\ell\},
\end{aligned}
\leqno(D)\mc0
\]
which is the first inequality in $(*)\mc0_\ell$.
By switching $\rho$ and $\sigma$ we obtain the second inequality in $(*)\mc0_\ell$.
Further, adding $(C)\mc0$ and its mirror-inequality for $\frac{\partial^+}{\partial s}\ell$, we have
\[r\cdot\tfrac{\partial^+}{\partial r}\ell
+
s\cdot\tfrac{\partial^+}{\partial s}\ell
\le 
\frac{{\ell^2}+{R^2}-{S^2}}{2\cdot\ell }+\frac{{\ell^2}+{S^2}-{R^2}}{2\cdot\ell }
= 
\ell,
\leqno(E)\mc0\]
namely $(**)\mc0_\ell$.
\qeds

\parit{Proof of \ref{rad-mon} and \ref{rad-comp} in case $\kappa\ne 0$.} 
As before, let
\begin{align*}
R=R(r)&=\dist{p}{\rho(r)}{},&\ell&=\ell(r,s)=\dist{\rho(r)}{\sigma(s)}{},
\\
S=S(s)&=\dist{p}{\sigma(s)}{},&\phi&=\phi(r,s)=\tangle\mc\kappa\{\ell(r,s);r,s\}.
\end{align*}
It suffices to prove the following three inequalities:
\[
\begin{aligned}
&\tfrac{\partial^+}{\partial r}\phi(s_{\min},r)\le 0, 
&
&\tfrac{\partial^+}{\partial s}\phi(s,r_{\min})\le 0,
\end{aligned}
\leqno(*)\mc\pm_\phi
\]
\[
\sn\kappa s\cdot\cs\kappa S\cdot\tfrac{\partial^+}{\partial s}\phi
+
\sn\kappa r\cdot\cs\kappa R\cdot\tfrac{\partial^+}{\partial r}\phi\le 0.
\leqno(**)\mc\pm_\phi.
\]

Then radial monotonicity follows from $(*)\mc\pm_\phi$.
The radial comparison follows from $(*)\mc0_\phi$ and $(**)\mc\pm_\phi$.
Indeed, the functions $s\mapsto \sn\kappa s\cdot\cs\kappa S$ and $r\mapsto \sn\kappa r\cdot\cs\kappa R$ are Lipschitz.
Thus there is a solution for the differential equation
\[(r',s')=(\sn\kappa s\cdot\cs\kappa S,\sn\kappa r\cdot\cs\kappa R)\] 
with any initial data $(r_0,s_0)\in[r_{\min},\tfrac{\varpi\kappa}2)\times[s_{\min},\tfrac{\varpi\kappa}2)$.
(Unlike the case $\kappa=0$, the solution cannot be written explicitly.)
Since $\sn\kappa s\cdot\cs\kappa S$, $\sn\kappa r\cdot\cs\kappa R>0$, this solution $(r(t),s(t))$ must meet one of the coordinate rays
$\{r_{\min}\}\times[s_{\min},\tfrac{\varpi\kappa}2)$ or $[r_{\min},\tfrac{\varpi\kappa}2)\times\{s_{\min}\}$.
That is, one can connect the pair $(s_{\min},r_{\min})$ to $(s_0,r_0)$ by a join of a coordinate segment and part of the solution $(r(t),s(t))$.
According to $(*)\mc\pm_\phi$ and $(**)\mc\pm_\phi$, the value of $\phi$ does not increase while the pair $(r,s)$ moves along this join in direction of increasing $r$ and $s$.
Thus $\phi(r_0,s_0)\le\phi(r_{\min},s_{\min})$.

As before, we rewrite the inequalities $(*)\mc\pm_\phi$ and $(**)\mc\pm_\phi$ in terms of $\ell$:
\[
\begin{aligned}
\tfrac{\partial^+}{\partial s}\ell(s,r_{\min})
&\le 
\cos\tangle\mc\kappa\{r_{\min};s,\ell\},
\\
\tfrac{\partial^+}{\partial r}\ell(s_{\min},r)
&\le 
\cos\tangle\mc\kappa\{s_{\min};r,\ell\},
\end{aligned}
\leqno(*)\mc\pm_\ell
\]

\[
\begin{aligned}
\sn\kappa s&\cdot\cs\kappa S\cdot\tfrac{\partial^+}{\partial s}\ell
+
\sn\kappa r\cdot\cs\kappa R\cdot\tfrac{\partial^+}{\partial r}\ell\le 
\\
&\le\sn\kappa s\cdot\cs\kappa S\cdot\cos\tangle\mc\kappa\{r;s,\ell\}
+
\sn\kappa r\cdot\cs\kappa R\cdot\cos\tangle\mc\kappa\{s;r,\ell\}.
\end{aligned}
\leqno(**)\mc\pm_\ell.
\]
Let
\[f=-\tfrac{1}{\kappa}\cdot\cs\kappa\circ\distfun{p}{}{}
=
\md\kappa\circ\distfun{p}{}{}-\tfrac{1}{\kappa}.\leqno(A)\mc\pm\]
Clearly $f''+\kappa\cdot  f\le 0$ and
\[
\begin{aligned}
\rho^+(r)&=\frac{1}{\tg\kappa r\cdot\cs\kappa R}\cdot\nabla_{\rho(r)} f,
\\
\sigma^+(s)&=\frac{1}{\tg\kappa s\cdot\cs\kappa S}\cdot\nabla_{\sigma(s)} f.
\end{aligned}
\leqno(B)\mc\pm\]
Thus from \ref{lem:grad-lip}, we have
\[\begin{aligned}
\tfrac{\partial^+}{\partial r}\ell
&=
-\frac{1}{\tg\kappa r\cdot\cs\kappa R}
\cdot
\<\nabla_{\rho(r)} f,\dir{\rho(r)}{\sigma(s)}\>
\le
\\
&\le
\frac
{1}
{\tg\kappa r\cdot\cs\kappa R}
\cdot
\frac
{\cs\kappa S-\cs\kappa R\cdot\cs\kappa\ell}
{\kappa\cdot\sn\kappa\ell}
=
\\
&=
\frac
{\frac{\cs\kappa S}{\cs\kappa R}-\cs\kappa\ell}
{\kappa\cdot\tg\kappa r\cdot\sn\kappa\ell}.
\end{aligned}
\leqno(C)\mc\pm\]
Note that for all $\kappa\ne 0$,
the function $x\mapsto\frac{1}{\kappa\cdot\cs\kappa x}$ is increasing.
Thus, since $R(r)\le r$ and $S(s_{\min})=s_{\min}$, we have 
\[\begin{aligned}
\tfrac{\partial^+}{\partial r}\ell(r,s_{\min})
&\le 
\frac
{\frac{\cs\kappa s_{\min}}{\cs\kappa r}-\cs\kappa\ell}
{\kappa\cdot\tg\kappa r\cdot\sn\kappa\ell}
=
\\
&=
\frac
{{\cs\kappa s_{\min}}-\cs\kappa\ell\cdot\cs\kappa r}
{\kappa\cdot\sn\kappa r\cdot\sn\kappa\ell}=
\\
&=\cos\tangle\mc\kappa\{s_{\min};r,\ell\},
  \end{aligned}\leqno(D)\mc\pm\]
which is the first inequality in $(*)\mc\pm_\ell$ for $\kappa\ne 0$.
By switching $\rho$ and $\sigma$ we obtain the second inequality in $(*)\mc\pm_\ell$.
Further, adding $(C)\mc\pm$ and its mirror-inequality for $\tfrac{\partial^+}{\partial s}\ell$, we have
\[\begin{aligned}
\sn\kappa r\cdot\cs\kappa R\cdot\tfrac{\partial^+}{\partial r}\ell
&+
\sn\kappa s\cdot\cs\kappa S\cdot\tfrac{\partial^+}{\partial s}\ell\le
\\
&\le
\frac
{{\cs\kappa S}\cdot\cs\kappa r-\cs\kappa\ell\cdot\cs\kappa R\cdot\cs\kappa r}
{\kappa\cdot\sn\kappa\ell}
+\\
&\quad +
\frac
{{\cs\kappa R}\cdot\cs\kappa s-\cs\kappa\ell\cdot\cs\kappa S\cdot\cs\kappa s}
{\kappa\cdot\sn\kappa\ell}=
\\
&=
\sn\kappa r\cdot\cs\kappa R\cdot
\frac
{\cs\kappa s-\cs\kappa\ell\cdot\cs\kappa r}
{\kappa\cdot\sn\kappa r\cdot\sn\kappa\ell}
+\\
&\quad +
\sn\kappa s\cdot\cs\kappa S\cdot
\frac
{\cs\kappa r-\cs\kappa\ell\cdot\cs\kappa s}
{\kappa\cdot\sn\kappa s\cdot\sn\kappa\ell}
=
\\
&=\sn\kappa r\cdot\cs\kappa R\cdot\cos\tangle\mc\kappa\{r;s,\ell\}
+\\
&\quad +\sn\kappa s\cdot\cs\kappa S\cdot\cos\tangle\mc\kappa\{s;r,\ell\},
\end{aligned}
\leqno(E)\mc\pm\]
which is $(**)\mc\pm_\ell$.\qeds


\begin{thm}{Exercise}\label{ex:geodesic}
Suppose $\spc{L}$ is a complete length $\Alex\kappa$ space 
and $x,y,z\in \spc{L}$.
Assume $\angk\kappa zxy=\pi$.
Show that there is a geodesic $[xy]$
that contains $z$.
In particular, $x$ can be connected to $y$ by a minimizing geodesic. (Compare to Exercise~\ref{ex:flat-in-CBB}.)
\end{thm}





\section{Gradient exponential map}\label{sec:gexp}

Let $\spc{L}$ be a complete length $\Alex{\kappa}$ space, 
$p\in \spc{L}$, 
and $\xi\in \Sigma_p$.
Consider a sequence of points $x_n\in \spc{L}$ such that $\dir p{x_n}\to \xi$.
Let $r_n=\dist{p}{x_n}{}$, and let
$\sigma_n\:[r_n,\tfrac{\varpi\kappa}2)\to \spc{L}$ be the $(p,\kappa)$-radial curve that starts at~$x_n$.

By the radial comparison (\ref{rad-comp}), 
the curves $\sigma_n\:[r_n,\tfrac{\varpi\kappa}2)\to \spc{L}$ 
converge to a curve $\sigma_\xi\:(0,\tfrac{\varpi\kappa}2)\to \spc{L}$, 
and this limit is independent of the choice of the sequence $x_n$.
Let $\sigma_\xi(0)=p$, and if $\kappa>0$ define \[\sigma_\xi(\tfrac{\varpi\kappa}2)
=
\lim_{t\to\frac{\varpi\kappa}2}\sigma_\xi(t).\]
The resulting curve $\sigma_\xi$ will be called the \index{radial curve}\emph{$(p,\kappa)$-radial curve in direction~$\xi$}.

Let us define the \index{gradient exponential map}\emph{gradient exponential map} as 
\[
\gexp\mc\kappa_p\:\cBall[\0,\tfrac{\varpi\kappa}2]\subset \T_p\to \spc{L} \: r\cdot\xi\mapsto\sigma_\xi(r).
\]
\index{$\gexp\mc\kappa_p$}
Here are properties of radial curves reformulated in terms of the gradient exponential map:

\begin{thm}{Theorem}\label{thm:prop-gexp}
Let $\spc{L}$ be a complete length $\Alex{\kappa}$ space. 
Then:
\begin{subthm}{}
If $p,q\in \spc{L}$ are points such that $\dist{p}{q}{}\le\tfrac{\varpi\kappa}2$, then for any geodesic $[pq]$ in $\spc{L}$ we have
\[\gexp\mc\kappa_p(\ddir p q)=q.\] 
\end{subthm}

\begin{subthm}{thm:prop-gexp:short} 
For any $v,w\in \cBall[\0,\tfrac{\varpi\kappa}2]\subset \T_p$,
\[\dist{\gexp\mc\kappa_p v}{\gexp\mc\kappa_p w}{}
\le
\side\kappa\hinge{\0}v w.\]
In other words, if we denote by $\mathcal{T}_{p}\mc\kappa$ the set $\cBall[\0,\tfrac{\varpi\kappa}2]\subset \T_p$ 
equipped with the metric $\dist{v}{w}{\mathcal{T}\mc\kappa_{p}}=\side\kappa\hinge{\0}v w$, 
then 
\[\gexp\mc\kappa_p:\mathcal{T}\mc\kappa_{p}\to \spc{L}\] 
is a short map.
\end{subthm}

\begin{subthm}{gexp-mono} 
Suppose
$p, q\in \spc{L}$ 
and $\dist{p}{q}{}\le \tfrac{\varpi\kappa}2$.
If $v\in\T_p$, $|v|\le 1$, and 
\[\sigma(t)=\gexp\mc\kappa_p(t\cdot v),\]
then the function
\[
s
\mapsto 
\tangle\mc\kappa(\sigma|_0^s,q)
\df
\tangle\mc\kappa\{\dist{q}{\sigma(s)}{};\dist{q}{\sigma(0)},s\}
\]
is nonincreasing in its entire domain of definition.
\end{subthm}
\end{thm}

\parit{Proof.}
Follows directly from the construction of $\gexp\mc\kappa_p$ and the radial comparison (\ref{rad-comp}).
\qeds

Applying the theorem above together with \ref{LinDim+-f},
we obtain the following.

{\sloppy 
 
\begin{thm}{Corollary}\label{cor:short-map-to-ball}
Let $\spc{L}$ be an $m$-dimensional complete length $\Alex\kappa$ space, $p\in\spc{L}$, and $0<R\le\tfrac{\varpi\kappa}2$.
Then there is a short map 
$f\:\cBall[R]_{\Lob{m}{\kappa}}\to \spc{L}$
such that $\Im f= \cBall[p,R]\subset \spc{L}$.
\end{thm}

}





\begin{thm}{Exercise}\label{ex:gexp} 
Let $\spc{L}\subset\EE^2$ be the Euclidean halfplane. 
Clearly $\spc{L}$ is a two-dimensional complete length $\Alex{0}$ space.
Given a point $x\in \EE^2$, denote by $\proj(x)$ the closest point to $x$ on $\spc{L}$. 

Apply the radial comparison (\ref{rad-comp}) to show that for any interior point $p\in \spc{L}$ and any $v\in\R^2$  we have 
\[\gexp_p v=\proj(p+v).\]
\end{thm}

\begin{thm}{Exercise}\label{ex:inv-gexp}
Suppose $x,p,$ and $q$ are points in a complete length $\Alex\kappa$ space, and $x\in [pq[$.
Show that there is a unique vector $v\in\T_p$ such that $\gexp_p v=x$.
\end{thm}


{\sloppy 

\begin{thm}{Exercise}\label{ex:bry-cover}
Let $\spc{L}$ be an $m$-dimensional complete length $\Alex\kappa$ space. Writing $\rad\spc{L}=R$, 
prove that there is a $(\sn\kappa R)$-Lipschitz map $\map\:\mathbb{S}^{m-1}\to\spc{L}$ such that $\Im\map\supset\partial\spc{L}$.
\end{thm}

}

\section{Remarks}

\subsection*{Gradient flow on Riemannian manifolds}
The gradient flow for general semiconcave functions 
on smooth Riemannian manifolds  can be introduced with much less effort.
To do this note that the distance estimates proved in the Section~\ref{sec:grad-curv:dist-est}
can be proved in the same way for gradient curves of smooth semiconcave subfunctions.
By the Greene--Wu lemma \cite{greene-wu}, 
given 
a $\lambda$-concave function $f$, 
a compact set $K\subset \Dom f$,
and $\eps>0$
there is a smooth $(\lambda-\eps)$-concave function that is 
$\eps$-close to $f$ on $K$.
Hence one can apply smoothing and pass to the limit as $\eps\to0$.
Note that by the second distance  estimate (\ref{lem:fg-dist-est}), the  limit curve obtained does not depend on the smoothing.

\subsection*{Gradient curves of a family of functions}

Gradient flow can be extended to a family of functions.
This type of flow was studied by Chanyoung Jun \cite{jun-thesis,jun:grad}, by Lucas Ferreira and Julio Valencia-Guevara \cite{ferreira-valencia}, and by Alexander Mielke, Riccarda Rossi, and Giuseppe Savar\'{e} \cite{mielke-rossi-savare}.
We will follow the simplified and generalized approach given by Alexander Lytchak and the third author \cite{lytchak-petrunin-2020}, where an application related to this type of flow is given.
The original motivation of Chanyoung Jun came from the study of pursuit-evasion problems.
Another application of this type of flow comes from the fact that
 the optimal transport plan, or equivalently geodesics in the Wasserstein metric, can be described as gradient flow for a family of semiconcave functions.
This observation was used by the third author to prove Alexandrov spaces with nonnegative curvature have nonnegative Ricci curvature in the sense of Lott--Villani--Sturm \cite{petrunin:optimal}.

Suppose that $\spc{Z}$ is either $\Alex{}$ or $\CAT{}$.
Let $f_t$ be a family of functions defined on open subsets $\Dom f_t$ of~$\spc{Z}$.
More precisely, we assume that the parameter $t$ lies in a real interval $\II$ and 
\[\Omega=\set{(x,t)\in\spc{Z}\times \II}{x\in\Dom f_t}\]
is an open subset in $\spc{Z}\times \II$.

A family of functions $f_t$ is called \index{Lipschitz family}\emph{Lipschitz} if 
the function $(x,t)\mapsto f_t(x)$ is 
%$\Lip$-Lipschitz for some constant $\Lip$.
$L$-Lipschitz for some constant $L$.

A family of functions $f_t$ will be called \index{semiconcave family of functions}\emph{semiconcave} if 
the function $x\mapsto f_t(x)$ is $\lambda$-concave for each $t$.
A family $f_t$ is called \emph{locally semiconcave} if for each $(p_0,t_0)\in \Omega$ there is a neighborhood $\Omega'$ and $\lambda\in\RR$ such that the restriction of $f_t$ to $\Omega'$ is semiconcave. 

One cannot expect that a direct generalization of Definition \ref{def:grad-curve}  holds for every family of functions $f_t$; that is, gradient curves of a family $f_t$ cannot be defined as curves satisfying the equation $\alpha^+=\nabla_{\alpha} f$.

For example, consider a $1$-Lipschitz curve $\alpha$ in the real line. 
It is reasonable to assume that $\alpha$ is an $f_t$-gradient curve for the family $f_t(x)\z=-|x-\alpha(t)|$.
(Indeed $\alpha$ can be realized as a limit of  gradient curves for a family of functions obtained by smoothing $f_t$.)
On the other hand, $\alpha^+(t)$ might be undefined,
and even if it is defined, in general $\alpha^+(t)\ne0$,  while $\nabla_{\alpha(t)} f_t\equiv0$.


Instead we define an {}\emph{$f_t$-gradient curve} as a Lipschitz curve $\alpha$ that satisfies the following inequality
for any point $p$, time $t$, and
small $\eps >0$:  
\[\distfun{p}{}{}\circ\alpha(t+\eps)\le \distfun{p}{}{}\circ\alpha(t)-\eps\cdot \dd_{\alpha(t)}f_t(\dir{\alpha(t)}p)+o(\eps).\eqlbl{def:tdflow}\]
If there is no geodesic $[\alpha(t)\,p]$ then we impose no condition.

If $\alpha^+(t)=\nabla_{\alpha(t)}f_t$ for all $t$, then \ref{def:tdflow} holds by the definition of gradient (\ref{def:grad}).
On the other hand, the example above shows that the converse does not hold;
that is, \ref{def:tdflow} generalizes Definition \ref{def:grad-curve}.
The defining  inequality \ref{def:tdflow} is closely related to the so-called \index{evolution variational inequality}\emph{evolution variational inequality} \cite[Theorem 4.0.4(iii)]{ambrosio-gigli-savare}.

\begin{thm}{Distance estimate}\label{Distance estimate}
Let $f_t$ and $h_t$ be two families of $\lambda$-concave functions on a complete length space $\spc{Z}$, and $s\ge 0$.
Suppose that $\spc{Z}$ is either $\Alex{}$ or $\CAT{}$.
Assume $f_t$ and $h_t$ have common domain $\Omega\subset {\spc{Z}}\times \RR$, and $|f_t(x)-h_t(x)|\le s$ for any $(x,t)\in \Omega$.
Assume $t\mapsto \alpha(t)$ and $t\mapsto \beta(t)$ are $f_t$- and $h_t$-gradient curves respectively defined on a common interval $t\in [a,b)$, and let $\ell(t)\z=\dist{\alpha(t)}{\beta(t)}{\spc{Z}}$.
If for all $t$, a minimizing geodesic $[\alpha(t)\,\beta(t)]$ lies in $\set{x\in {\spc{Z}}}{(x,t)\in \Omega}$, then
\[\ell'(t)\le \lambda\cdot\ell(t)+2\cdot s/\ell(t),\]
whenever the left-hand side is defined.
Moreover,
\[\ell(t)^2+\tfrac{2\cdot s}\lambda\le(\ell(a)^2+\tfrac{2\cdot s}\lambda)\cdot e^{2\cdot\lambda\cdot (t-a)}.\]

In particular, these inequalities hold for any $t\in\II$ if $\Omega\supset B(p,2\cdot r)\times \II$ and $\alpha(t),\beta(t)\z\in B(p, r)$ for any $t\in \II$.
\end{thm}

Note that if $f_t=h_t$ then $s=0$;
in this case the second inequality can be written as
\[\ell(t)\le \ell(a)\cdot e^{\lambda\cdot (t-a)}.\eqlbl{dist-est-s=0}\]
In particular, this inequality implies uniqueness of the future of gradient curves with given initial data.
This inequality also makes it possible to estimate the distance between two gradient curves for close functions.
In particular, it implies convergence for $f_t^n$-gradient curves if a sequence of $\Lip$-Lipschitz and $\lambda$-concave families $f^n_t$ converges uniformly as $n\to \infty$. 

\parit{Proof of \ref{Distance estimate}.}
Fix a time moment $t$ and set $f=f_t$ and $h=h_t$.
Let $p$ be the midpoint of the geodesic $[\alpha(t)\beta(t)]$.
Let $\gamma\:[0,\ell(t)]\to \spc{Z}$ be an arclength parametrization of $[\alpha(t)\beta(t)]$.  Note that $\dd_{\alpha(t)}f(\dir{\alpha(t)}{p})$ is the right derivative of $f\circ\gamma$ at $0$
and $-\dd_{\alpha(t)}h(\dir{\beta(t)}p)$ is the left derivative of $h\circ\gamma$ at $\ell(t)$.
Since $f$ and $h$ are $\lambda$-concave,
\begin{align*}
f\circ\beta(t)&\le f\circ\alpha(t)+\ell(t)\cdot \dd_{\alpha(t)}f(\dir{\alpha(t)}{p}) +\tfrac12\cdot\lambda\cdot\ell(t)^2,
\\
h\circ\alpha(t)&\le h\circ\beta(t)+\ell(t)\cdot \dd_{\alpha(t)}h(\dir{\beta(t)}p) +\tfrac12\cdot\lambda\cdot\ell(t)^2.
\end{align*}
Adding these inequalities and taking into account  $|f(x)-h(x)|<s$ for any $x$, we conclude that 
\[\dd_{\alpha(t)}f(\dir{\alpha(t)}{p})+\dd_{\alpha(t)}h(\dir{\beta(t)}p)\ge \lambda\cdot \ell(t)+2\cdot s/\ell(t).\]

Applying the triangle inequality and the definition of gradient curve at $p$, we obtain
\begin{align*}
\ell(t+\eps)&=\dist{\alpha(t+\eps)}{\beta(t+\eps)}{}\le
\\
&\le \dist{\alpha(t+\eps)}{p}{}+\dist{\beta(t+\eps)}{p}{}\le 
\\
&\le \dist{\alpha(t)}{p}{}-\eps\cdot \dd_{\alpha(t)}f(\dir{\alpha(t)}{p})+
\\
&\quad +\dist{\beta(t+\eps)}{p}{}-\eps\cdot \dd_{\beta(t)}h(\dir{\beta(t)}p)+o(\eps)=
\\
&=\ell(t)-\eps\cdot(\lambda\cdot \ell(t)+2\cdot s/\ell(t))+o(\eps)
\end{align*}
for $\eps>0$. The first inequality follows.

Since $\alpha$ and $\beta$ are Lipschitz, $t\mapsto \ell(t)$ is a Lipschitz function.
By Rademacher's theorem, its derivative $\ell'$ is defined almost everywhere and satisfies the fundamental theorem of calculus.
Therefore the first inequality implies the second.
\qeds

\begin{thm}{Proposition}\label{prop:def-time-dependent}
Suppose  $\spc{Z}$ is a complete length space that is either $\Alex{}$ or $\CAT{}$.
Let $f_t$ be a family of $\lambda$-concave functions for $t\in [a,b)$, where $\Dom f_t\supset B(z,2\cdot r)$ for some fixed $z\in\spc{Z}$, $r>0$ and any~$t$.

Let $\alpha\:[a,b)\to B(z,r)$ be Lipschitz.
Then $\alpha$ is an $f_t$-gradient curve if and only if 
\[\begin{aligned}
&\distfun{p}{}{}\circ\alpha(t+\eps)\le 
\\
&\quad\le \distfun{p}{}{}\circ\alpha(t)-\eps\cdot \left[\frac{f_t(p)-f_t\circ\alpha(t)}{\dist{p}{\alpha(t)}{}}-\tfrac\lambda2\cdot \dist{p}{\alpha(t)}{}\right]+o(\eps)
\end{aligned}
\eqlbl{def:tdflow-plus}\]
for any $t\in [a,b)$ and $p\in B(z,r)\setminus \{\alpha (t)\}$.
\end{thm}

\parit{Proof.}
Note that the geodesics $[\alpha(t)p]$ lie in $\Dom f_t$ for any $t$.

Since $f_t$ is $\lambda$-concave, we have 
\[\dd_{\alpha(t)}f_t(\dir{\alpha(t)}p)
\ge
\frac{f(p)-f\circ\alpha(t)}{\dist{p}{\alpha(t)}{}}-\tfrac\lambda2\cdot \dist{p}{\alpha(t)}{}.\]
Hence the only-if part follows.

Given  $p\in \spc{Z}$ and $t$,
consider a point $\bar p\in [\alpha(t)p]$.
Applying \ref{def:tdflow-plus} for $\bar p$, and the triangle inequality, we have
\[\distfun{p}{}{}\circ\alpha(t+\eps)
\le
\distfun{p}{}{}\circ\alpha(t)-\eps\cdot \left[\frac{f(\bar p)-f\circ\alpha(t)}{\dist{\bar p}{\alpha(t)}{}}-\tfrac\lambda2\cdot \dist{\bar p}{\alpha(t)}{}\right]+o(\eps).\]
By taking $\bar p$ close to $\alpha(t)$,
the value $\tfrac{f(\bar p)-f\circ\alpha(t)}{\dist{\bar p}{\alpha(t)}{}}-\tfrac\lambda2\cdot \dist{\bar p}{\alpha(t)}{}$ can be made arbitrarily close to $\dd_{\alpha(t)}f_t(\dir{\alpha(t)}p)$.
Therefore, given $\delta>0$, the inequality
\[\distfun{p}{}{}\circ\alpha(t+\eps)\le \distfun{p}{}{}\circ\alpha(t)-\eps\cdot \dd_{\alpha(t)}f_t(\dir{\alpha(t)}p)+\eps\cdot\delta\]
holds for all sufficiently small positive values $\eps$.
Therefore \ref{def:tdflow} holds.
\qeds


Now we are ready to formulate and prove global existence of gradient curves for time-dependent families --- an analog of \ref{thm:glob-exist-grad-curv}.

\begin{thm}{Theorem}\label{prop:time-dependent}
Suppose $\spc{Z}$ is a complete length space that is either $\Alex{}$ or $\CAT{}$.
Let
$\{f_t\}$ be a family of functions defined on an open set
\[\Omega=\set{(x,t)\in \spc{Z}\times\RR}{x\in \Dom f_t}.\]
Suppose that $f_t$ is Lipschitz and locally semiconcave.
Then for any time  $a$ and initial point $p\in \Dom f_a$, there is a unique $f_t$-gradient curve $t\mapsto\alpha(t)$ defined on a maximal semiopen interval $[a,b)$. 
Moreover, if $b<\infty$ then $(\alpha(t),t)$ escapes from any closed set $K\subset \Omega$.
\end{thm}

\parit{Proof.}
Let $L$ be a Lipschitz constant of $f_t$.
Fix $b>a$ sufficiently small  that $\Dom f_t\supset B(p,\eps\cdot L)$ for any $t\in[a,b)$.
Consider a sequence  $a=t_0<t_1\dots<t_n\z=b$, and a piecewise constant family of functions on $B(p,\eps\cdot L)$ defined by $\hat f_t=f_{t_i}$ if $t_i\le t<t_{i+1}$.

Note that $\hat f_t$ is time-independent on each interval $[t_i,t_{i+1})$.
By  \ref{thm:glob-exist-grad-curv} applied recursively on each interval $[t_i,t_{i+1})$,  the proposition holds for $\hat f_t$.
That is, there is a unique $\hat f_t$-gradient curve $\hat \alpha$ that starts at $p$ and is defined on the interval $[a,b)$.

The distance estimates (\ref{Distance estimate}) show that as the partition gets finer, the gradient curves $\hat\alpha$ form a Cauchy sequence; denote its limit by $\alpha$.
Then
\begin{align*}
\distfun{p}{}{}\circ\hat\alpha(t+\eps)
&\le 
\distfun{p}{}{}\circ\hat\alpha(t)-
\\
&\quad
-\eps\cdot \left[\frac{\hat f_t(p)-\hat f_t\circ\hat\alpha(t)}{\dist{p}{\alpha(t)}{}}-\tfrac\lambda2\cdot \dist{p}{\hat\alpha(t)}{}\right] 
+o(\eps)\le
\\
&\le 
\distfun{p}{}{}\circ\hat\alpha(t)-
\\
&\quad
-\eps\cdot \left[\frac{f_t(p)-f_t\circ\hat\alpha(t)-2\cdot s}{\dist{p}{\alpha(t)}{}}-\tfrac\lambda2\cdot \dist{p}{\hat\alpha(t)}{}\right]
+o(\eps),
\end{align*}
where 
\[s=\sup_{t,x} \{|f_t(x)-\hat f_t(x)|\}.\]
Since $s\to 0$ as $\hat\alpha\to \alpha$, then \ref{def:tdflow-plus} holds for $\alpha$;
that is, $\alpha$ is an $f_t$-gradient curve.

This proves short time existence.
Applying this argument recursively, we can find a gradient curve defined on a maximal interval $[a,b)$.
Uniqueness of this curve follows from the distance estimate \ref{dist-est-s=0}. 

Note that $\alpha$ is $L$-Lipschitz.
In particular, if $b<\infty$ then $\alpha(t)\to p'$ as $t\to b$.
If $(p',b)\in \Omega$ then we can repeat the procedure; otherwise $\alpha$ escapes from any closed set in $\Omega$. 
\qeds

\subsection*{Gradient curves for non-Lipschitz functions}\label{sec:non-lip}

In this book, we only consider gradient curves for locally Lipschitz semiconcave subfunctions;
this turns out to be sufficient for all our needs.
However, 
instead of Lipschitz semiconcave subfunctions,
it is more natural to consider upper semicontinuous semiconcave functions
with target in $[-\infty,\infty)$,
and to assume in addition that 
the functions take finite values at a dense set in the domain of definition.
Suppose that $\spc{Z}$ is a complete length space that is either $\Alex{}$ or $\CAT{}$.
The set of such subfunctions on $\spc{Z}$ will be denoted by 
$\LSCSC(\spc{Z})$ (for \textbf{l}ower semi-\textbf{c}ontinous and semi-\textbf{c}oncave).

In this section we describe the adjustments needed
to construct gradient curves for the subfunctions in $\LSCSC(\spc{Z})$.

This type of function appears in
entropy and some other closely related functionals on the Wasserstein space over a  $\Alex0$ space.
The gradient flow for these functions plays an important role in the theory of optimal transport, see \cite{villani} and references there in. 


\parbf{Differential.} 
First we need to extend the definition of differential (\ref{def:differential}) to $\LSCSC$ subfunctions.

Let $\spc{Z}$ be a complete length space and $f\in\LSCSC(\spc{Z})$.
Suppose that $\spc{Z}$ is either $\Alex{}$ or $\CAT{}$.
Given a point $p\in \Dom f$ and a geodesic direction $\xi=\dir pq$, 
let 
$\hat \dd_pf(\xi)=(f\circ\geod_{[pq]})^+(0)$.
Since $f$ is semiconcave, the value $\hat \dd_pf(\xi)$ is defined if $f\circ\geod_{[pq]}(t)$ is finite at all sufficiently small values $t>0$,
but $\hat \dd_pf(\xi)$ may take value $\infty$. 
Note that $\hat \dd_pf$ is defined on a dense subset of $\Sigma_p$.

Let 
\[\dd_pf(\zeta)=\limsup_{\xi\to\zeta}\hat\dd_pf(\xi),\]
and $\dd_pf(v)=|v|\cdot \dd_pf(\xi)$ if $v=|v|\cdot\xi$ for some $\xi\in\Sigma_p$.

In other words, we define differential as the smallest 
upper semi-continuous  positive-homogeneous function $\dd_pf\:\T_p\to\RR$
such that if $\hat\dd_pf(\xi)$ is defined, then $\dd_pf(\xi)\ge \hat \dd_pf(\xi)$.



\parbf{Existence and uniqueness of the gradient.}
Note that in the proof of \ref{thm:ex-grad}, 
we used the Lipschitz condition just once,
to show that 
\begin{align*}
s&=\sup\set{(\dd_p f)(\xi)}{\xi\in\Sigma_p}=
\\
&=\limsup_{x\to p}\frac{f(x)-f(p)}{\dist{x}{p}{}}<
\\
&<\infty.
\end{align*}


The value $s$ above will be denoted by $|\nabla|_pf$.
Note that 
if the gradient $\nabla_pf$ is defined then $|\nabla|_pf=|\nabla_pf|$,
and otherwise $|\nabla|_pf=\infty$.

Summarizing the discussion above, 
we have the following.

\parbf{\ref{thm:ex-grad}$'$ Existence and uniqueness of the gradient.}
\textit{Assume $\spc{Z}$ is a complete space and $f\in \LSCSC(\spc{Z})$. 
Suppose that $\spc{Z}$ is either $\Alex{}$ or $\CAT{}$.
Then for any point $p\in \Dom f$, either there is a unique gradient $\nabla_p f\in \T_p$ 
or $|\nabla|_pf=\infty$.}

\medskip

Further, in all the results of Section~\ref{sec:grad-calculus} 
we may assume only that both $f$ and the gradient of $f$ are defined at the points under consideration. The proofs are the same.

Sections \ref{sec:grad-semicont}--\ref{sec:grad-curv:exist}
require almost no changes.
Mainly, where appropriate
one needs to change $|\nabla_p f|$ 
to $|\nabla|_pf$ 
and/or assume that the gradient is defined at the points of interest.
Also  \ref{eq:thm:grad-like-2nd-def-1} in Theorem \ref{thm:grad-like-2nd-def}
is taken as the definition of gradient-like curve.
Then the theorem states that any  gradient-like curve $\alpha\:\II\to\spc{Z}$ satisfies Definition \ref{def:grad-like-curve} at $t\in \II$ if $\nabla_{\hat\alpha(s)} f$ is defined.
Further, Definition \ref{def:grad-curve}, should be changed to the following:

\medskip

\parbf{\ref{def:grad-curve}$'$. Definition.}
{\it Let $\spc{Z}$ be a complete length space
and $f\in\LSCSC(\spc{Z})$.
Suppose that $\spc{Z}$ is either $\Alex{}$ or $\CAT{}$.

A curve 
$\alpha\:[t_{\min},t_{\max})\to\Dom f$ will be called an  \index{gradient curve}\emph{$f$-gradient curve} if
\[\alpha^+(t)=\nabla_{\alpha(t)} f\]
iwhen $\nabla_{\alpha(t)} f$ is defined and 
\[(f\circ\alpha)^+(t)=\infty\]
otherwise.}

\medskip

In the proof of local existence (\ref{thm:exist-grad-curv}), condition (\ref{alm-grad})
should be changed to the following condition:
\begin{itemize}

\item[{(\ref{alm-grad})}$'$]
$f\circ\hat\alpha_n(\bar\varsigma_i)-f\circ\hat\alpha_n(\varsigma_i)
>
(\bar\varsigma_i-\varsigma_i)
\cdot
\max\{n,|\nabla|_{\hat\alpha_n(\varsigma_i)}f-\tfrac{1}{n})\}.$
\end{itemize}

Any gradient curve $\alpha[0,\ell)\to\spc{Z}$
for a subfunction
$f\in \LSCSC(\spc{Z})$
satisfies the equation
\[\alpha^+(t)=\nabla_{\alpha(t)} f\]
at all values $t$, with the possible exception of $t=0$.
In particular, the gradient of $f$ is defined at all points of any 
$f$-gradient curve, with the exception of the initial point.

\subsection*{Slower radial curves}
Let $\kappa\ge 0$. 
Assume that for some function $\psi$, the curves defined by the equation 
\[\sigma^+(s)=\psi(s,\dist{p}{\sigma(s)}{})\cdot\nabla_{\sigma(s)}\distfun{p}{}{}\]
satisfy radial comparison \ref{rad-comp}.
Then in fact the $\sigma(s)$ are radial curves; 
that is, 
\[\psi(s,\dist{p}{\sigma(s)}{})= \frac{\tg\kappa\dist[{{}}]{p}{\sigma(s)}{})}{\tg\kappa s},\]
see exercise \ref{ex:gexp}.

In case $\kappa<0$, such a function $\psi$ is not unique.
In particular, one can take curves defined by the simpler equation
\[\sigma^+(s)
=
\frac{\sn\kappa \dist[{{}}]{p}{\sigma(s)}{}}{\sn\kappa s}\cdot\nabla_{\sigma(s)}\distfun{p}{}{}
=
\frac{1}{\sn\kappa s}\cdot\nabla_{\sigma(s)}(\md\kappa\circ\distfun{p}{}{}).\]
Among all curves of that type, the radial curves for curvature $\kappa$ 
as defined in \ref{def:rad-curv} maximize the growth of $\dist{p}{\sigma(s)}{}$.

\subsection*{Radial curves for sets}

Here we generalize the constructions of radial curves and gradient exponent.
We show that one can use a distance function 
$\distfun{A}{}{}$ to any closed set $A$ instead of the distance function to one point.
We only give the corresponding definitions and state the results.
The proofs are straightforward generalizations of the corresponding one-point-set version. 

First we give a more general form of the definitions of radial curves (\ref{def:rad-curv}) and radial geodesics (\ref{def:rad-geod}):

\begin{thm}{Definition}
Let $\spc{L}$ be a complete length $\Alex{}$ space, 
$\kappa\in\RR$, 
and $A\subset \spc{L}$ be a closed subset of $\spc{L}$.
A curve $\sigma\:[s_{\min},s_{\max})\to \spc{L}$  is called an
\emph{$(A,\kappa)$-radial curve} 
if
$s_{\min}
\z=
\distfun{A}{\sigma(s_{\min})}{}\in(0,\tfrac{\varpi\kappa}2)$, 
and $\sigma$ satisfies the differential equation
\[\sigma^+(s)
\z=
\frac{\tg\kappa\dist[{{}}]{p}{\sigma(s)}{}}{\tg\kappa s}
\cdot
\nabla_{\sigma(s)}\distfun{A}{}{}\]
for any $s\in[s_{\min},s_{\max})$, where $\tg\kappa x=\frac{\sn\kappa x}{\cs\kappa x}$.

If $x=\sigma(s_{\min})$, we say that $\sigma$ {}\emph{starts at}  $x$.
\end{thm}

\begin{thm}{Definition}
Let $\spc{L}$ be a complete length $\Alex{}$ space
and $A\subset \spc{L}$ be a closed subset of $\spc{L}$.
A unit-speed geodesic  $\gamma\:\II\to \spc{L}$  is called an
\index{radial geodesic}\emph{$A$-radial geodesic} if 
$\distfun{A}{\gamma(s)}{}\equiv s$.
\end{thm}

The following propositions are analogous to  \ref{prop:rad-geod} and \ref{prop:dist<s}.
Their proofs follow directly from the definitions.

\begin{thm}{Proposition}
Let $\spc{L}$ be a complete length $\Alex{}$ space,
$A\subset\spc{L}$ be a closed subset of $\spc{L}$.
Assume that 
$\tfrac{\varpi\kappa}{2}
\ge 
s_{\max}$.
Then any $\distfun{A}{}{}$-radial geodesic 
$\gamma\:[s_{\min},s_{\max})
\to 
\spc{L}$ 
is an $(A,\kappa)$-radial curve.
\end{thm}

\begin{thm}{Proposition}
Let $\spc{L}$ be a complete length $\Alex{}$ space,
$A\subset\spc{L}$ be a closed subset of $\spc{L}$,
and $\sigma\:[s_{\min},s_{\max})\to \spc{L}$ be an $(A,\kappa)$-radial curve.
Then for any $s\in [s_{\min},s_{\max})$, 
we have $\distfun{A}{\sigma(s)}{}\le s$.

Moreover, if for some $s_0$ we have $\distfun{A}{\sigma(s_0)}{}= s_0$, 
then the restriction $\sigma|_{[s_{\min},s_0]}$ is an $A$-radial geodesic.
\end{thm}

Here is the corresponding generalization of existence and uniqueness 
for $(A,\kappa)$-radial curves;
it can be proved in the same way as \ref{rad-curv-exist}.

\begin{thm}{Existence and uniqueness}
Let $\spc{L}$ be a complete length $\Alex{}$ space, 
$\kappa\in\RR$, 
$A\subset\spc{L}$ be a closed subset of $\spc{L}$, 
and $x\in \spc{L}$.
Assume
$0
<
\distfun{A}{x}{}
\z<
\tfrac{\varpi\kappa}2$.
Then there is a unique $(A,\kappa)$-radial curve $\sigma\:[\distfun{A}{x}{},\tfrac{\varpi\kappa}2)\to \spc{L}$ 
that starts at $x$.
\end{thm}

Next we formulate radial monotonicity and radial comparison for $(A,\kappa)$-radial curves.
The proof of these two statements are almost exactly the same as the proofs of \ref{rad-mon} and \ref{rad-comp}.

\begin{thm}{Radial monotonicity}\label{gen-rad-mon}
Let $\spc{L}$ be a complete length $\Alex{\kappa}$ space,
$A\subset \spc{L}$ be a closed subset of $\spc{L}$,
and $q\in\spc{L}\backslash A$.
Assume $\sigma\:  [s_{\min},\tfrac{\varpi\kappa}2)\to \spc{L}$
is an $(A,\kappa)$-radial curve.
Then the function 
\[s\mapsto 
\tangle\mc\kappa\{
\dist{q}{\sigma(s)}{};
\distfun{A}{q}{},
s
\}\]
is nonincreasing in its entire domain of definition.
\end{thm}

To formulate a generalized radial comparison,
we  introduce a suitable notation.
Given a subset $A$ and two points $x$ and $y$ in a metric space, define
\[
\angkk\kappa A{x}{y}
\df
\tangle\mc\kappa\{
\dist{x}{y}{};
\distfun{A}{x}{},
\distfun{A}{y}{}
\}.
\]
Note that distances $\dist{x}{y}{}$, 
$\distfun{A}{x}{}$ and 
$\distfun{A}{y}{}$ might not satisfy the triangle inequality.
Therefore the model angle 
$\angkk\kappa A{x}{y}$ might be undefined even for $\kappa\le0$.

\begin{thm}{Radial comparison}\label{gen-rad-comp}
Let $\spc{L}$ be a complete length $\Alex{\kappa}$ space 
and $A\subset \spc{L}$ be a closed subset of $\spc{L}$.
Assume $\rho\:  [r_{\min},\tfrac{\varpi\kappa}2)\to \spc{L}$
and    $\sigma\:[s_{\min},\tfrac{\varpi\kappa}2)\to \spc{L}$
are two $\distfun{A}{}{}$-radial curves for curvature $\kappa$.
Assume further that 
\[\phi_{\min}
=
\angkk\kappa A{\rho(r_{\min})}{\sigma(s_{\min})}
\]
is defined.
Then for any $r\in[r_{\min},\tfrac{\varpi\kappa}2)$ and  $s\in[s_{\min},\tfrac{\varpi\kappa}2)$,
we have
\[
\dist{\rho(r)}{\sigma(s)}{}
\le \side\kappa\{\phi_{\min};r,s\}.
\]

\end{thm}

Finally, 
suppose $p$ is an isolated point of a closed subset $A$ of  a complete length $\Alex{}$ space $\spc{L}$.
Applying the same limiting procedure as in Section \ref{sec:gexp},
for any $\xi\in\Sigma_p$
one can construct an $(A,\kappa)$-radial curve $\sigma_\xi$
such that $\sigma_\xi(0)=p$ and $\sigma^+(0)=\xi$.
Thus we obtain a map $\gexp\mc\kappa_A\:\T_p\subto\spc{L}$:
$r\cdot\xi\mapsto\sigma_\xi(r)$.
For this map, the following analog of \ref{thm:prop-gexp} holds;
the proof is straightforward.

\begin{thm}{Theorem}
Let $\spc{L}$ be a complete length $\Alex{\kappa}$ space, and $A\subset\spc{L}$ be a closed subset of $A$ with an isolated point $p\in A$.
Then:
\begin{subthm}{}
Let $\distfun{A}{q}{}=\dist{p}{q}{}\le\tfrac{\varpi\kappa}2$.  
Let $[pq]$ be an $A$-radial geodesic. Then
\[\gexp\mc\kappa_A(\ddir p q)=q.\] 
\end{subthm}

\begin{subthm}{} 
For any $v,w\in \cBall[\0,\tfrac{\varpi\kappa}2]\subset \T_p$,
\[\dist{\gexp\mc\kappa_p v}{\gexp\mc\kappa_p w}{}
\le
\side\kappa\hinge{\0}v w.\]
In other words, if we denote by $\mathcal{T}_{p}\mc\kappa$ the set $\cBall[\0,\tfrac{\varpi\kappa}2]\subset \T_p$ 
equipped with the metric $\dist{v}{w}{\mathcal{T}\mc\kappa_{p}}=\side\kappa\hinge{\0}v w$, 
then 
\[\gexp\mc\kappa_p:\mathcal{T}\mc\kappa_{p}\to \spc{L}\] 
is a short map.
\end{subthm}

\begin{subthm}{gexp-mono-1} 
Suppose $p, q\in \spc{L}$ 
and $\dist{p}{q}{}\le \tfrac{\varpi\kappa}2$.
If $v\in\T_p$, $|v|\le 1$ and 
\[\sigma(t)=\gexp\mc\kappa_p(t\cdot v),\]
then the function
$
s\mapsto \tangle\mc\kappa(\sigma|_0^s,q)
$
is nonincreasing in its entire domain of definition.
\end{subthm}
\end{thm}



