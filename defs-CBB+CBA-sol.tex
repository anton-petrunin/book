%%!TEX root = the-defs-CBB+CBA-sol.tex
\parbf{\ref{exr-crofton}.}
Suppose $\alpha$ is a closed spherical curve. 
By Crofton's formula, the length of a curve $\alpha$ on the sphere is $\pi\cdot n_\alpha$, where $n_\alpha$ denotes the average number of crossings of $\alpha$ with equators.

Since $\alpha$ is closed, almost all equators cross it at an even number of points (we assume that $\infty$ is an even number).
If $\length \alpha<2\cdot\pi$ then $n_\alpha<2$.
Therefore there is an equator that does not cross $\alpha$ --- hence the result.


\parbf{\ref{ex:complete=>complete};}
\textit{(\ref{SHORT.ex:complete=>complete:complete}).}
Note that any Cauchy sequence $x_n$ in $(\spc{X},\yetdist{}{}{})$ is also Cauchy in $\spc{X}$.
Since $\spc{X}$ is complete, $x_n$ converges; denote its limit by $x_\infty$.

Passing to a subsequence, we may assume that $\yetdist{x_{n-1}}{x_n}{}<\tfrac1{2^n}$.
It follows that there is a 1-Lipschitz curve $\alpha\:[0,1]\to (\spc{X},\yetdist{}{}{})$ such that $x_n=\alpha(\tfrac1{2^n})$ and $x_\infty=\alpha(0)$.
In particular, $\yetdist{x_n}{x_\infty}{}\to0$ and $n\to\infty$.

\parit{(\ref{SHORT.ex:complete=>complete:compact}).}
Fix two points $x,y\in\spc{X}$ such that $\ell=\yetdist{x}{y}{}<\infty$.
Let $\alpha_n$ be a sequence of paths from $x$ to $y$ such that $\length(\alpha_n)\to\ell$ as $n\to \infty$.
Without loss of generality we may assume that each $\alpha_n$ is $(\ell+1)$-Lipschitz.

Since $\spc{X}$ is compact, there is a partial limit $\alpha_\infty$ of $\alpha_n$ as $n\to \infty$. By semicontinuity of length, $\length\alpha_\infty\le\ell$;
that is; $\alpha$ is a shortest path in $\spc{X}$.

\parbf{\ref{ex:no-geod}.}
The following example was suggested by Fedor Nazarov~\cite{nazarov}.

\medskip

Consider the unit ball $(B,\rho_0)$
in the space $c_0$ of all sequences converging to zero equipped with the sup-norm.

Consider another metric $\rho_1$ which is different from $\rho_0$ by the conformal factor
\[\phi(\bm{x})=2+\tfrac{1}2\cdot x_1+\tfrac{1}4\cdot x_2+\tfrac{1}8\cdot x_3+\dots,\]
where $\bm{x}=(x_1,x_2\,\dots)\in B$.
That is, if $\bm{x}(t)$, $t\in[0,\ell]$, is a curve parametrized by $\rho_0$-length 
then its $\rho_1$-length is 
\[\length_{\rho_1}\bm{x}=\int\limits_0^\ell\phi\circ\bm{x}.\]
Note that the metric $\rho_1$ is bi-Lipschitz equivalent  to~$\rho_0$.

Assume $\bm{x}(t)$ and $\bm{x}'(t)$ are two curves parametrized by $\rho_0$-length that differ only in the $m$-th coordinate, denoted by $x_m(t)$ and $x_m'(t)$ respectively.
Note that if $x'_m(t)\le x_m(t)$ for any $t$ and 
the function $x'_m(t)$ is locally $1$-Lipschitz at all $t$ such that $x'_m(t)< x_m(t)$, then 
\[\length_{\rho_1}\bm{x}'\le \length_{\rho_1}\bm{x}.\]
Moreover this inequality is strict if $x'_m(t)< x_m(t)$ for some~$t$.

Fix a curve $\bm{x}(t)$, $t\in[0,\ell]$, parametrized by  $\rho_0$-length.
We can choose $m$ large, so that $x_m(t)$ is sufficiently close to $0$ for any~$t$.
In particular, for some values $t$, we have $y_m(t)<x_m(t)$, where
\[y_m(t)=(1-\tfrac t\ell)\cdot x_m(0)
+\tfrac t\ell\cdot x_m(\ell)
-\tfrac 1{100}\cdot \min\{t,\ell-t\}.\]
Consider the curve $\bm{x}'(t)$ as above with
\[x'_m(t)=\min\{x_m(t),y_m(t)\}.\]
Note that $\bm{x}'(t)$ and $\bm{x}(t)$ have the same end points, and by the above
\[\length_{\rho_1}\bm{x}'<\length_{\rho_1}\bm{x}.\]
That is, for any curve $\bm{x}(t)$ in $(B,\rho_1)$, we can find a shorter curve $\bm{x}'(t)$ with the same end points.
In particular, $(B,\rho_1)$ has no geodesics.

\parbf{\ref{exercise from BH}.}
The following example is taken from~\cite{bridson-haefliger}.

\medskip

\begin{wrapfigure}{r}{20 mm}
\vskip-0mm
\centering
\includegraphics{mppics/pic-5}
\end{wrapfigure}

Consider the following subset of $\R^2$ equipped with the induced length metric
\[
\spc{X}
=
\bigl((0,1]\times\{0,1\}\bigr)
\cup
\bigl(\{1,\tfrac12,\tfrac13,\dots\}\times[0,1]\bigr).
\]
Note that $\spc{X}$ is locally compact and geodesic.

Its completion $\bar{\spc{X}}$ is isometric to the closure of $\spc{X}$ equipped with the induced length metric;
$\bar{\spc{X}}$ is obtained from $\spc{X}$ by adding two points $p=(0,0)$ and $q=(0,1)$.

The point $p$ admits no compact neighborhood in $\bar{\spc{X}}$ 
and there is no geodesic connecting $p$ to $q$ in~$\bar{\spc{X}}$. \qeds 

\parbf{\ref{ex:compact-in-lenght}}
Let $\spc{X}$ be a compact metric space.
Let us identify $\spc{X}$ with its image in $\Bnd(\spc{X},\RR)$ under the Kuratowsky embedding (see section~\ref{Kuratowsky embedding}). 
Denote by $\spc{K}$ the \emph{linear} convex hull of $\spc{X}$ in the space of bounded functions on $\spc{X}$; 
that is, $x\in \spc{K}$ if and only if $x$ cannot be separated from $\spc{X}$ by a hyperplane.

Since $\spc{X}$ is compact, so is $\spc{K}$.
It remains to observe that $\spc{K}$ is a length space since it is convex.

\parbf{\ref{ex:nonconvex-limit}.} Let $\spc{X}_n$ be the plane with the metric induced by the $\ell^n$-norm and let $f_n(x,y)=x$ for all $n$.
Observe that $\spc{X}_\o$ is the plane with the metric induced by the $\ell^\infty$-norm where the limit function $f_\o(x,y)=x$ is not 
%convex.
concave.

%*S. missing solution to Exercise 4.4.7??

\parbf{\ref{ex:adjacent-angles}.}
If $\mangle\hinge pxz+\mangle\hinge pyz< \pi$, then by the triangle inequality for angles (\ref{claim:angle-3angle-inq}) we have $\mangle\hinge pxy< \pi$.
The latter implies that the curve $[xy]$ fails to be a shortest geodesic near $p$.

\parbf{\ref{ex:tangent-vect=o(t)}.}
By the definition of a right derivative, there is a geodesic $\gamma$ such that both limits 
\[\limsup_{\eps\to0+}\frac{\dist{\alpha(\eps)}{\gamma(\eps)}{\spc{X}}}{\eps}
\quad\text{and}\quad
\limsup_{\eps\to0+}\frac{\dist{\beta(\eps)}{\gamma(\eps)}{\spc{X}}}{\eps}\]
are arbitrarily small.
Therefore 
\[\limsup_{\eps\to0+}\frac{\dist{\alpha(\eps)}{\beta(\eps)}{\spc{X}}}{\eps}=0.\]

\parbf{\ref{ex:both-sided-diff}.}
This follows directly from the definition.

\parbf{\ref{ex:diff}.}
Observe that
\[\speed_t\alpha=|\alpha^+(t)|=|\alpha^-(t)|.\]
Apply Theorem~\ref{thm:speed} to show that
\[\dist{\alpha^+(t)}{\alpha^-(t)}{\T_{\alpha(t)}}=2\cdot\speed_t\alpha.\]


\parbf{\ref{ex:schroeder-foetch}.}
Choose two non-Euclidean norms $\|{*}\|_{\spc{X}}$ and $\|{*}\|_{\spc{Y}}$ on $\RR^{10}$ such that the sum $\|{*}\|_{\spc{X}}+\|{*}\|_{\spc{Y}}$ is Euclidean.
See \cite{schroeder-foetch} for more details.

\parbf{\ref{ex:(3+1)-expanding}.} 
Assume $\dist{p}{x^i}{}=\dist{q}{y^i}{}$ for each $i$.
Observe and use that
\[\dist{x^i}{x^j}{}\le\dist{y^i}{y^j}{}
\quad\iff\quad
\angk\kappa p{x^i}{x^j}\le \angk\kappa q{y^i}{y^j}.\]

\parbf{\ref{ex:cbb-area}.} Follows from the overlap lemma (\ref{lem:extend-overlap}).




\parbf{\ref{ex:nongeod-cbb}.}
Modify the induced length metric on the unit sphere in a Hilbert space in small neighborhoods of a countable collection of points. To prove that the obtained space is $\Alex0$, you may need to use the technique from Halbeisen's example (\ref{Halbeisen's example}).

\parbf{\ref{ex:almost.geod}.} Mimic the proof of Theorem~\ref{thm:almost.geod}.

\parbf{\ref{ex:G-delta-not-thru}.}
On the plane, any nonnegatively curved metric having an everywhere dense set of singular points will do the job, where 
by singular point we mean a point having total angle around it strictly smaller than $2\cdot\pi$.

Indeed, if $x_i$ is a singular point, then there is $\eps_i>0$ such that no geodesic with ends outside of $\oBall(x_i,r)$ can meet the ball $\oBall(x_i,\eps_i\cdot r)$.
The set 
\[\Omega_n=\bigcup_i \oBall(x_i,\tfrac{\eps_i}n)\]
is open and everywhere dense.
Note that $\Omega_n$ may intersect a geodesic only at 
%$\tfrac1n$-end of it.
an endpoint.
The intersection of the $\Omega_n$ is a G-delta dense set that does not intersect the interior of any geodesic.

\parbf{\ref{mink+alex=euclid}.} 
Note that scaling does not change the space.
Therefore if the space is $\Alex\kappa$ then it is $\Alex{\lambda\cdot\kappa}$ for any $\lambda>0$.
Passing to the limit as $\lambda\to 0$, we may assume that the space is $\Alex0$.

The point-on-side comparison (\ref{point-on-side}) for $p=v$, $x=w$, $y=-w$ and $z=0$ implies that 
\[\|v+w\|^2+\|v-w\|^2\le 2\cdot\|v\|^2+2\cdot\|w\|^2.\]
Applying the comparison for 
$p=v+w$, $x=w-v$, $y=v-w$ and $z=0$ gives the opposite inequality.
That is, the parallelogram identity
\[\|v+w\|^2+\|v-w\|^2= 2\cdot\|v\|^2+2\cdot\|w\|^2\]
holds for any vectors $v$ and $w$.
Whence the statement follows.

\parbf{\ref{ex:cbb-geod-overlap}.}
Apply hinge comparison (\ref{angle}).

\parbf{\ref{ex:equality-alexlemma}.} Without loss of generality, we may assume that the points $x,v,w,y$ appear on the geodesic $[xy]$ in that order.
By point-on-side comparison (\ref{point-on-side}) we have
\begin{align*}
\angk\kappa xyp\le\angk\kappa xwp&\le \angk\kappa xvp,
\\
\angk\kappa ywp&\ge\angk\kappa yvp\ge\angk\kappa yxp.
\end{align*}
Therefore
\begin{align*}\angk\kappa xyp<\angk\kappa xwp
\quad&\Longrightarrow\quad
\angk\kappa xyp<\angk\kappa xvp,
\\
\angk\kappa yxp<\angk\kappa ywp
\quad&\Longleftarrow\quad
\angk\kappa yxp<\angk\kappa yvp.
\end{align*}

By Alexandrov's lemma (\ref{lem:alex}), we have
\begin{align*}
\angk\kappa xyp<\angk\kappa xvp
\quad&\Longleftrightarrow\quad
\angk\kappa yxp<\angk\kappa yvp,
\\
\angk\kappa xyp<\angk\kappa xwp
\quad&\Longleftrightarrow\quad
\angk\kappa yxp<\angk\kappa ywp.
\end{align*}
Hence the statement follows.


\parbf{\ref{ex:urysohn}.} See the construction of Urysohn's space \cite[3.11$\tfrac{3}{2}_+$]{gromov-MS}.

\parbf{\ref{ex:lebedeva-petrunin}.}
Read \cite{lebedeva-petrunin}.

\parbf{\ref{ex:fat-triangle}.} Apply angle-sidelength  monotonicity (\ref{cor:monoton}) twice. 

\parbf{\ref{ex:busemann}.} The first part follows from angle-sidelength  monotonicity (\ref{cor:monoton}).
An example for the second part can be found among metrics on $\RR^2$ induced by a norm. (Compare to Exercise~\ref{mink+alex=euclid}.)

\parbf{\ref{ex:busemann-CBB} and \ref{ex:busemann-CBA}.}
By the definition of Busemann function (see  \ref{prop:busemann}),
\begin{align*}
\exp(&\sqrt{-\kappa}\cdot\bus_\gamma) 
= \exp \left[\lim_{t\to \infty} \sqrt{-\kappa}\cdot(\distfun{{\gamma (t)}}{}{} - t)\right]=
\\
&= \lim_{t\to \infty} \biggl(\exp \left[\sqrt{-\kappa}\cdot(\distfun{\gamma (t)}{}{} -t)\right]
+\exp\left[\sqrt{-\kappa}\cdot(-\distfun{\gamma (t)}-t)\right]\biggr)=
\\
&=  \lim_{t\to \infty} \left(2\cdot \cosh \left[\sqrt{-\kappa}\cdot\distfun{\gamma (t)}{}{}\right]\cdot \exp\left[\sqrt{-\kappa}\cdot(-t)\right]\right).
\end{align*}

By the function comparison definitions of $\CAT\kappa$ space (\ref{function-comp}) or $\Alex{\kappa}$ space (\ref{comp-kappa}),  for any $p\in \spc{U}$ the function $f=\cosh \sqrt{-\kappa}\circ\distfun{p}{}{}$ satisfies $f''+\kappa \cdot f\ge 0$ (respectively  $f''+\kappa \cdot f\le 0$). The result follows.

\parbf{\ref{ex:noncomplete-globalization}.} Read \cite{petrunin:globalization}.

\parbf{\ref{ex:fixed-point}.} If $\diam(\spc{L}/G)>\tfrac\pi2$, then for some $x\in \spc{L}$ we have
\[\sup \set{\distfun{G\cdot x}(y)}{y\in \spc{L}}
>
\tfrac\pi2.\]
Use comparison to show that there is a unique point $y^{*}$ that lies at maximal distance from the orbit $G\cdot x$.
Observe that $y^{*}$ is a fixed point.

\parbf{\ref{ex:kleiner}.}
This exercise is based on the main idea in \cite{hsiang-kleiner}.

\medskip

Assume there are 4 such points $x_1,x_2,x_3,x_4$.
Since the space $\spc{L}$ is $\Alex{1}$  it is also $\Alex{0}$. By angle comparison the sum of the angles in any geodesic triangle in an $\Alex{0}$ space is $\ge \pi$.
Therefore the average of the $\mangle\hinge{x_i}{x_j}{x_k}$ is  larger than $\tfrac\pi3$.
On the other hand, since each $x_i$ has space of directions $\le\tfrac12\cdot\mathbb{S}^n$ and the perimeter of any triangle in $\tfrac12\cdot\mathbb{S}^n$ is at most $\pi$, the average of $\mangle\hinge{x_i}{x_j}{x_k}$ is at most $\tfrac\pi3$ --- a contradiction.


\parbf{\ref{ex:ccat-(3+1)}.}
Suppose that 
\[\angk\kappa {x^0}{x^1}{x^2}+\angk\kappa {x^0}{x^2}{x^3}<\angk\kappa {x^0}{x^1}{x^3}.\]
Show that
\[\angk\kappa {x^2}{x^0}{x^1}+\angk\kappa {x^2}{x^1}{x^3}+\angk\kappa {x^2}{x^3}{x^0}>2\cdot\pi.\]
Conclude that one can take $p=x^2$.

\parbf{\ref{ex:sba-2+2-short}.}
This is analogous to Exercise \ref{ex:(3+1)-expanding}.

\parbf{\ref{ex:berg-nikolaev}.} Read \cite{sato}. (The original proof \cite{berg-nikolaev} is much longer and is harder to follow.)

An example for the second part of the problem can be found among 4-point metric spaces.
It is sufficient to take a generic convex quadrangle and increase one of its diagonals slightly;
it will still satisfy the inequality for all relabeling, but will fail to meet \ref{def:2+2-reformulated}.

\parbf{\ref{ex:CAT-mnfld=>ext.geod}.}
Suppose that a geodesic $[px]$ is not extendable beyond $x$.
We may assume that $\dist{p}{x}{}<\varpi\kappa$;
otherwise move $p$ along the geodesic toward  $x$.

By the uniqueness of geodesics (\ref{thm:cat-unique}), any point $y$ in a neighborhood $\Omega\ni x$ is connected to $p$ by a unique geodesic path; denote it by $\gamma_y$.
Moreover, $h_t(y)=\gamma_y(t)$ defines a homotopy, called the  \index{geodesic homotopy}\emph{geodesic homotopy}, between the identity map of $\Omega$ and the constant map with value $p$.

Since $[px]$ is not extendable, $x\notin h_t(\Omega)$ for any $t<1$.
In particular, the local homology groups vanish at $x$ --- a contradiction.

\parbf{\ref{ex:complete-space-of-dir}.} Choose a sequence of 
directions $\xi_n$ at $p$
of both-side local geodesics; denote by $\gamma_n$ the corresponding geodesics.
Since the space $\spc{U}$ is locally compact, we may pass to a converging subsequence of $(\gamma_n)$; denote its limit by limit $\gamma_\infty$ and its direction by $\xi_\infty$.
By comparison, $\xi_\infty$ is a limit of $(\xi_n)$.

\parbf{\ref{mink+CAT=euclid}.} Follow the solution in the \ref{mink+alex=euclid}, reversing all the inequalities.

\parbf{\ref{ex:convexity-CAT0}.} 
It is sufficient to show that if $v$ and $y$ are midpoints of geodesics $[uw]$ and $[xz]$ in $\spc{U}$, then
\[\dist{v}{y}{}\le \tfrac12\cdot(\dist{u}{x}{}+\dist{w}{z}{}).\]

\begin{wrapfigure}{r}{45 mm}
\vskip-0mm
\centering
\includegraphics{mppics/pic-10}
\end{wrapfigure}

Denote by $p$ the midpoint of $[uz]$.
Applying angle-sidelength  monotonicity (\ref{cor:monoton-cba}) twice, we have
\[\dist{v}{p}{}\le \tfrac12\cdot\dist{w}{z}{}.\]
Similarly we have
\[\dist{y}{p}{}\le \tfrac12\cdot\dist{u}{x}{}.\]
It remains to add these two inequalities and apply the triangle inequality.

\parit{Comment.}
This inequality also follows directly from the majorization theorem (\ref{thm:major}).

\parbf{\ref{ex:equality-for-thin}.}
The only-if part is evident.

Use \ref{thm:defs_of_cat} to show that 
(\ref{SHORT.ex:equality-for-thin:side-side})$\Rightarrow$(\ref{SHORT.ex:equality-for-thin:vertex-base})$\Rightarrow$(\ref{SHORT.ex:equality-for-thin:angle}).

By \ref{thm:defs_of_cat}, condition (\ref{SHORT.ex:equality-for-thin:angle}) implies that the natural map is distance-preserving on the sides $[\tilde x\tilde y]$ and $[\tilde x\tilde z]$.
Applying it again, we have that condition (\ref{SHORT.ex:equality-for-thin:angle}) holds for all permutations of the labels $x,y,z$.
Whence the natural map is distance-preserving on all three sides.





\parit{Comment.} These conditions imply that the natural map can be extended to a distance-preserving map to the solid model triangle.
In fact the image of the line-of-sight map (\ref{def:sight}) is isometric to the model triangle.


\parbf{\ref{ex:busemann-CBA}.}
See the solution of Exercise~\ref{ex:busemann-CBB}.

\parbf{\ref{ex:closest-point-projection}.}
Let $p\mapsto\bar p$ denote the closest-point projection to $K$.
We need to show that $\dist{\bar p}{\bar q}{}\le\dist{ p}{ q}{}$ for any $p,q\in \spc{U}$.

Assume $p\ne \bar p\ne \bar q\ne q$.
Note that in this case $\mangle\hinge {\bar p}{p}{\bar q}\ge \tfrac\pi2$ and $\mangle\hinge {\bar q}{q}{\bar p}\ge \tfrac\pi2$.
Otherwise a point on the geodesic $[\bar p\bar q]$ would be closer to $p$ or to $q$ than $\bar p$ or $\bar q$ respectively.
The latter is impossible since $K$ is convex and therefore $[\bar p\bar q]\subset K$.

Applying the arm lemma (\ref{lem:arm}), we get the statement.

The cases $p= \bar p\ne \bar q\ne q$ and $p\ne \bar p\ne \bar q= q$ can be done similarly.
The rest of the cases are trivial.

\parbf{\ref{ex:short-retraction-CBA(1)}.}
A more transparent, but less elementary solution via gradient flow is given by Alexander Lytchak and the third author~\cite{lytchak-petrunin}.

\medskip

Without loss of generality, we may assume that $p\in K$.

If $\dist{K}{x}{}\ge\pi$, then set $\map[2](x)=p$.

Otherwise, if $\dist{K}{x}{}<\pi$, by the closest-point projection lemma~\ref{lem:closest point}, 
there is a unique point $x^*\in K$ that minimizes distance to $x$;
that is, $\dist{x^*}{x}{}=\dist{K}{x}{}$.
Let us define $\ell_x$, $\phi_x$ and $\psi_x$ using the following identities:
\begin{align*}
\ell_x&=\dist{p}{x^*}{},
\\
\phi_x&=\tfrac\pi2-\dist[{{}}]{x^*}{x}{},
\\
\sin\psi_x&=\sin\phi_x\cdot\sin\ell_x, 
\quad 0\le \psi_x\le \tfrac\pi2.
\intertext{Let}
\map[2](x)&=\geod_{[px^*]}(\psi_x).
\end{align*}

Note that $\map[2]$ is a retraction to $K$; 
that is,
$\map[2](x)\in K$ for any $x\in \spc{U}$
and 
$\map[2](a)=a$ for any $a\in K$.

Let us show that $\map[2]$ is short.
Given $x,y\in\oBall(K,\tfrac\pi2)$, let
\begin{align*}
x'&=\map[2](x)
&
y'&=\map[2](y)
\\
r&=\dist{x}{y}{}
&
r'&=\dist{x'}{y'}{}
\\
d&=\dist{x^*}{y^*}{}
&
\alpha&=\angk1{p}{x^*}{y^*}.
\end{align*}

Note that 
\[\cos r\le 
\cos\phi_x\cdot\cos\phi_y
-
\cos d\cdot\sin\phi_x\cdot\sin\phi_y.
\eqlbl{eq:cos(r)}\]

Indeed, if $x,y\notin K$,
then 
$\mangle\hinge{x^*}{x}{y*}, 
\mangle\hinge{y^*}{y}{x*}
\ge 
\tfrac\pi2$
and
the inequality~\ref{eq:cos(r)} follows from the Arm lemma (\ref{lem:arm}).
If $x\in K$ and $y\notin K$, we obtain \ref{eq:cos(r)} by angle comparison (\ref{cat-hinge}) 
since $\mangle\hinge{y^*}{y}{x*}\ge \tfrac\pi2$.
In the same way, \ref{eq:cos(r)} is proved 
if $x\notin K$ and $y\in K$.
Finally, if $x,y\in K$, then $\phi_x=\phi_y=\tfrac\pi2$ and $r=d$;
that is, the inequality trivially holds.

Further note that
\[\cos\alpha
=
\frac{\cos d-\cos \ell_x\cdot\cos\ell_y}{\sin\ell_x\cdot\sin\ell_y}.\]
Applying angle-sidelength  monotonicity (\ref{cor:monoton-cba}), we have
\begin{align*}
\cos r'&\ge
\cos\psi_x\cdot\cos\psi_y
-
\cos \alpha \cdot\sin\psi_x\cdot\sin\psi_y=
\\
&=
\cos\psi_x\cdot\cos\psi_y
-(\cos d-\cos \ell_x\cdot\cos\ell_y)\cdot\sin\phi_x\cdot\sin\phi_y\ge
\\
&\ge \cos\psi_x\cdot\cos\psi_y
-\cos d\cdot\sin\phi_x\cdot\sin\phi_y.
\end{align*}


Note that 
$\psi_x\le \phi_x$
and
$\psi_y\le \phi_y$;
in particular,
\[
\cos\phi_x\cdot\cos\phi_y\le \cos\psi_x\cdot\cos\psi_y.
\]
Hence 
\[\cos r'\ge \cos r;\]
that is, the restriction $\map[2]|_{\oBall(K,\tfrac\pi2)}$ is short.
Clearly $\map[2]$ is continuous. 
Since the complement of $\oBall(K,\tfrac\pi2)$ is mapped to $p$,
 $\map[2]$ is short; that is,
\[r'\le r \eqlbl{eq:cos=<cos}\]
for any $x,y\in\spc{U}$.

If we have equality in \ref{eq:cos=<cos}
then 
\[\cos\ell_x\cdot\cos\ell_y\cdot\sin\phi_x\cdot\sin\phi_y=0.\]
If $K\subset \oBall(p,\tfrac\pi2)$, then $\ell_x,\ell_y<\tfrac\pi2$, 
which implies that $x\in K$ or $y\in K$.
Without loss of generality we may assume that $x\in K$.

It remains to show that if $y\notin K$ 
then the inequality~\ref{eq:cos=<cos}
is strict.
If $\dist{K}{y}{}\ge\tfrac\pi2$, then \ref{eq:cos=<cos} holds since 
the left-hand side is $<\tfrac\pi2$
while the right hand side is $\ge \tfrac\pi2$.
If $\dist{K}{y}{}<\tfrac\pi2$, then $\phi_y>0$. Clearly $\psi_y<\phi_y$,
hence the inequality~\ref{eq:cos=<cos} is strict.
\qeds

Below you will find a geometric way to think about the given construction; 
in fact it is very close to the construction 
in the proof of Kirszbraun's theorem (\ref{thm:kirsz+}).

\parit{Geometric interpretation of the map $\map[2]$.}
Let $\mathring{\spc{U}}=\Cone \spc{U}$, and 
denote by $\mathring{K}$ the subcone of $\mathring{\spc{U}}$ spanned by $K$.
The space $\spc{U}$ can be naturally identified with the unit sphere in $\mathring{\spc{U}}$, 
that is, the set 
\[\set{z\in \mathring{\spc{U}}}{|z|=1}.\]

According to \ref{thm:warp-curv-bound:cat}, $\mathring{\spc{U}}$ is $\CAT0$.
Note that $\mathring{K}$ forms a convex closed subset of $\mathring{\spc{U}}$.
According to \ref{lem:closest point}, for any point $x$ there is a unique point $\hat x\in \mathring{K}$
that minimizes the distance to $x$,
that is, $\dist{\hat x}{x}{}=\dist{K}{x}{}$.
(If $|\hat x|\ne0$, then in the notation above we have
$x^*=\tfrac1{|\hat x|}\cdot\hat x$.)

Consider the half-line $t\mapsto t\cdot p$ in  $\mathring{\spc{U}}$.
By comparison, 
for given $s\in \mathring{\spc{U}}$
the geodesics $\geod_{[s\ t\cdot p]}$ converge as $t\to\infty$ to a half-line, 
say $\alpha_s\:[0,\infty)\to \mathring{\spc{U}}$.



Note that if $|x|=1$, then $|\hat x|\le 1$.
By assumption, for any $a\in K$ the function $t\mapsto |\alpha_a(t)|$ is monotonically increasing.
Therefore there is a unique value $t_x\ge 0$ such that
$|\alpha_{\hat x}(t_x)|=1$.
Define $\map[2]\:\spc{U}\to K$
 by 
\[\map[2](x)=\alpha_{\hat x}(t_x).\]

\parbf{\ref{ex:patchwork};} \ref{SHORT.ex:patchwork:proper}.
Suppose that $x_n\to x_\infty$, $y_n\to y_\infty$ as $n\to\infty$,
but $[x_ny_n]$ does not converge to $[x_\infty y_\infty]$.
Since the space is proper, we can pass to a subsequence such that $[x_ny_n]$ converges to another geodesic.
That is, we have at least two geodesics between $x_\infty$ and $y_\infty$.

\parit{\ref{SHORT.ex:patchwork:complete}.} The following example is taken from \cite[Chapter I, Exercise 3.14]{bridson-haefliger}.

Let $\Delta_n$ be a sequence of solid spherical triangles 
with angle $\tfrac\pi4$ and adjacent sides $\pi-\tfrac1n$.
Let us glue each $\Delta_n$ to $[0,\pi]$ along an isometry of one of the longer sides.
It remains to show that the obtained space $\spc{X}$ is a needed example.

\parbf{\ref{ex:two-rays}.}
%Consider the angle $A$ in the plane of measure $\pi-\alpha$.
Note that 
%$A$
$Q$ 
is $\CAT0$.
Therefore by the Reshetnyak gluing theorem (\ref{thm:gluing}),
by gluing 
%a side of $A$ 
$\gamma_1$ 
%to $\gamma_1$
to $\gamma_1'$ 
%in $\spc{U}$ 
we obtain a $\CAT0$ space, say $\spc{U}'$.

The above curve resulting from gluing  
$\gamma_1$ to $\gamma_1'$ 
%together with the other side of $A$ forms
forms together with $ \gamma_2$ 
a both-sides infinite geodesic, say $\gamma$, in $\spc{U}'$.
In particular, $\gamma$ is a convex set isometric to $\RR$.

Now glue 
%a half-plane along its boundary to $\gamma$.
$Q$ along its boundary to $\spc{U}$ along $\gamma$.
By the Reshetnyak gluing theorem, the resulting space is $\CAT0$.

It remains to note that this space can be obtained directly by gluing $\spc{U}$ to $Q$ along 
%$\gamma_1$ and $\gamma_2$.
corresponding half-lines.

\parbf{\ref{ex:reshetnyak-doubling}.}
Suppose that $A$ is not convex.
Then there is a geodesic $[xy]$ with ends in $A$ that does not lie in $A$ completely.
Note that $[xy]$ can be lifted to two different geodesics with the same ends  in the doubling, and apply uniqueness of geodesics (\ref{cor:cat-unique}).

\parbf{\ref{ex:glue-spherical-suspension}.}
Since $K$ is $\pi$-convex, it is $\CAT1$.
By \ref{thm:warp-curv-bound:cat}, the spherical suspension $\Susp K$ is $\CAT1$ as well.
Let us glue $\Susp K$ to $\spc{U}$  along $K$;
according to the Reshetnyak gluing theorem, the resulting space, say $\spc{U}'$, is $\CAT1$.

Consider the geodesic path $\gamma\:[0,1]$ from $p$ to a pole of the suspension in $\spc{U}'$.
Set $K_t=\spc{U}\cap\cBall[\gamma(t),\tfrac\pi2]$.
By \ref{cor:convex-balls}, $K_t$ is $\pi$-convex for any $t$,  and monotonicity of the family should be evident.

\parit{Remark.}
Note that by applying Sharfutdinov retraction to the family of convex sets provided by the exercise,
we get a short strong deformation retraction from $\cBall[p,\tfrac\pi2]$ to $K$;
that is, there is a family of maps $\phi_t\:\cBall[p,\tfrac\pi2]\to \cBall[p,\tfrac\pi2]$ such that 
the function $t\mapsto \dist{\phi_t(x)}{\phi_t(y)}{}$ is nonincreasing for any pair of points $x,y\in\cBall[p,\tfrac\pi2]$, $\phi_t(x)=x$ for any $x\in K$ and $\phi_1(\cBall[p,\tfrac\pi2])=K$. 
Moreover we can assume that there is a family of short maps 
%$\phi_t(\cBall[p,\tfrac\pi2])= K_t$ and $\phi_t(x)=x$ for any $t$ and $x\in K_t$.
$\phi_t\:\cBall[p,\tfrac\pi2])\to  K_t$ such that $\phi_t(x)=x$ for any $t$ and $x\in K_t$.
This leads to another solution of Exercise~\ref{ex:short-retraction-CBA(1)}.




\parbf{\ref{ex:isometric-majorization}.}
\textit{(Easier way.)} 
Let 
$(t,s)\mapsto \gamma_t(s)$ be the line-of-sight map 
for $\alpha$ with respect to $\alpha(0)$,
and 
$(t,s)\mapsto \tilde \gamma_t(s)$ be the line-of-sight map 
for $\tilde \alpha$ with respect to $\tilde \alpha(0)$.
Consider the map  $F\:\Conv\tilde \alpha\to \spc{U}$ such that 
$F\:\tilde \gamma_t(s)\mapsto \gamma_t(s)$.

Show that $F$ majorizes $\alpha$
and conclude that $F$ is distance-preserving.

\parit{(Harder way.)}
Prove and apply the following statement together with the Majorization theorem.
\begin{itemize}
\item Let $\alpha$ and $\beta$ be two convex curves in $\Lob2\kappa$.
Assume 
\[\length \alpha=\length\beta<2\cdot\varpi\kappa\]
and there is a short bijecction $f\:\alpha\to\beta$.
Then $f$ is an isometry.
\end{itemize}

\parbf{\ref{ex:bishop}.}
Suppose that points $p,x,q,y$ appear on the curve in that cyclic order.
Assume that the geodesics $[pq]$ and $[xy]$ do not intersect.
Use the argument in the proof of the majorization theorem (\ref{thm:major}) to show that in this case there are nonequivalent majorization maps.

Now we can assume that pairs of geodesics $[pq]$ and $[xy]$ intersect for all choices of points $p,x,q,y$ on the curve in that cyclic order.
Show that in this case the convex hull $K$ of the curve is isometric to a convex figure.

Note that the composition of a majorization map and closest point projection to $K$ is a majorization.
Show and use that the boundary of a convex figure in the plane admits a unique majorization up to equivalence.

\parit{Comment.}
A typical rectifiable closed curve in a $\CAT0$ space can be majorized by more than one convex figure.
There are two exceptions: (1) if the majorization map is distance-preserving, and (2) if the curve is  formed by three sides of a geodesic triangle.
Richard Bishop asked if there are no other exceptions.

\parbf{\ref{ex:square}.}
Look at the four triples in $x^1$, $x^2$, $x^3$, $x^4$.
By  hypothesis, for each triple there is a triangle ${\triangle}^i$ in $\EE^2$ whose  sidelengths are the three distances between pairs.
By the Arm Lemma, for each triple, $\spc{U}$ contains an isometric copy of the corresponding solid triangle.
Consider adjacent pairs of these solid triangles in $\spc{U}$, with common sides moving around the sides of the quadrangle in order.
The fourth solid triangle has a common side with the first.
Therefore the union of the four solid triangles contains a triangle $\triangle$ with vertex $x^1$ and its other two vertexes in the intersections of the solid  triangle opposite $x^1$ with the two solid triangles with vertex $x^1$ respectively, and where $\triangle$ has the same sidelengths as a Euclidean triangle.
Therefore $\spc{U}$ contains an isometric copy of the solid Euclidean triangle with those side lengths.
Moving around the adjacent pairs of solid triangles in $\spc{U}$ corresponding to the ${\triangle}^i$ shows that $\spc{U}$ contains an isometric copy of the solid Euclidean quadrangle, as required.

\parbf{\ref{ex:cover-branching-along-2-lines}.}
If $\ell$ and $m$ do not intersect, then the double cover $\spc{X}$ is not simply connected.
In particular, by the Hadamard--Cartan theorem, $\spc{X}$ is not $\CAT0$.

If $\ell$ and $m$ intersect then $\spc{X}$ is a cone over double cover $\Sigma$ of $\mathbb{S}^2$ branching at two pairs $(x,y)$ and $(v,w)$ of antipodal points.
Suppose $\dist{x}{v}{\mathbb{S}^2}=\ell<\tfrac\pi2$.
Note that the inverse image of $[xv]_{\mathbb{S}^2}$ is a closed geodesic of length $4\cdot\ell<2\cdot\pi$.
Therefore, by the generalized Hadamard--Cartan theorem, $\Sigma$ is not $\CAT1$. Hence $\spc{X}$ is not $\CAT0$ by Theorem \ref{thm:warp-curv-bound:cat}  on curvature of cones.

\parbf{\ref{ex:branching}.}
Let us do the second part first.
Assume $A$ has nonempty interior. 
Note that the space $\tilde{\spc{U}}$ is simply connected and locally isometric to the doubling $\spc{W}$ of $\spc{U}$ in $A$;
that is, any point in $\tilde{\spc{U}}$ has a neighborhood 
%\Omega$ 
that is isometric to a neighborhood of a point in $\spc{W}$.

By the Reshetnyak gluing theorem (\ref{thm:gluing}), $\spc{W}$ is $\CAT0$.
By the Hadamard--Cartan theorem (\ref{thm:hadamard-cartan}), $\tilde{\spc{U}}$ is $\CAT0$.

For the general case, apply the above argument to a closed $\eps$-neighborhood of $A$
and pass to a limit as $\eps\to 0$.

The first part of problem follows since a geodesic is a convex set.

\parbf{\ref{ex:cats-cradle}.} Prove that the angle comparison (\ref{cat-hinge}) holds.

\parbf{\ref{ex:Hadamard--Cartan}.} Mimic the proof of the Hadamard--Cartan theorem.

\parbf{\ref{ex:CBB+CBA}.}
Note that it is sufficient to show that any finite set of points $x^1,\dots,x^n\in\spc{X}$ lies in an isometric copy of a Euclidean polyhedron.

Observe that $\spc{X}$ is $\Alex0$ and $\CAT0$ at the same time.
Show that there is unique point $p$ that minimizes the sum $\dist{p}{x^1}{}+\dots+\dist{p}{x^n}{}$.
Note that the vectors $v^i=\lg_{[{p}{x^i}]}$ lie in a linear subspace of $\T_p$, where $\log_{x_0}$  is the gradient logarithm map $\log_{x_0}\:\spc{L}\to\T_{x_0}$ corresponding to the gradient exponential $\gexp_{x_0}$ (see Section \ref{sec:gexp}).
Moreover if $K$ is the convex hull of $v_i$, then the origin of $\T_p$ lies in the interior of $K$ relative to its affine hull.
Finally observe that the exponential map is defined on all of $K$ and is distance-preserving.
The statement follows since the exponential map sends $v^i\mapsto x^i$ for each $i$.

\parbf{\ref{ex:5-point-CBA=>CBB}.}
The answers are $s\le \sqrt3$ and $s\le 2$ respectively.

The upper bound $s\le \sqrt3$ follows from (2+2)-point comparison.

The Euclidean space works as an example if $s$ is smaller than the large diagonal of the double pyramid with unit side (that is, if $s\le 2\cdot\sqrt{2/3}$).
Otherwise take the product $\spc{K}\times \RR$ with a 2-dimensional cone for the $\CAT0$ case.
For the $\Alex0$ case, the needed space can be constructed by doubling  a proper convex set $K\subset\EE^3$ in its boundary.
We assume that the points correspond to vertices of a regular tetrahedron with 3 vertices on the boundary of $K$ and one  in its interior; this point corresponds to a pair of points in the doubling at distance $s$ from each other.

\parit{Remarks.}
Tetsu Toyoda \cite{toyoda} showed that any 5-point metric space that satisfies the (2+2)-point comparison admits a distance-preserving map into a $\CAT0$ space.
For spaces with more than 5 points the condition is unknown;
for 6-point metric spaces, a conjecture is formulated in \cite{lebedeva-petrunin-zolotov}.

\parbf{\ref{ex:cbb-wald}.}
Choose a quadruple of points $p,q,r,s$. 
Suppose that it admits a distance-preserving embedding into some $\Lob2{\Kappa}$ for some $\Kappa\ge \kappa$.
Then 
\[\angk\Kappa p{q}{r}
+\angk\Kappa p{r}{s}
+\angk\Kappa p{a}{q}\le 2\cdot\pi.\]
Applying monotonicity of the function $\kappa\mapsto\angk\kappa p{q}{r}$ (\ref{k-decrease}) shows  that
\[\angk\kappa p{q}{r}
+\angk\kappa p{r}{s}
+\angk\kappa p{s}{q}\le 2\cdot\pi.\]
Since the quadruple $p,q,r,s$ is arbitrary, the if part follows.

Now let us prove the only-if part.
Denote by $\sigma$ the exact upper bound on values $\Kappa\ge \kappa$ such that all model triangles with the vertices $p,q,r,s$ are defined.

Recall that $\angk{\Kappa+} p{q}{r}$ denotes extended angle (\ref{def:extended-angle}).
Observe that if 
\[\angk{\Kappa+} p{q}{r}
+\angk{\Kappa+} p{r}{s}
+\angk{\Kappa+} p{s}{q}= 2\cdot\pi\eqlbl{eq:Kappa3}\]
for some $\sigma\ge \Kappa\ge \kappa$, then the quadruple admits a distance-preserving embedding into $\Lob2\Kappa$.

Observe that the left-hand side of \ref{eq:Kappa3} is continuous in $\Kappa$.
Since $\spc{L}$ is $\Alex\kappa$, for $\Kappa=\kappa$ the left-hand side cannot exceed $2\cdot \pi$.
Therefore it remains smaller than $2\cdot\pi$ for all $\sigma\ge \Kappa\ge \kappa$;
moreover the same holds for all permutations of the labels $p,q,r,s$.

Note that we can assume the perimeter of the triple $q,r,s$ is $2\cdot\varpi{\sigma}$, and use this and the overlap lemma (\ref{lem:extend-overlap}) to arrive at a contradiction.

According to our definition, the real line is $\Alex\kappa$ for any $\kappa\in\RR$,
but it does not satisfy the property for $\kappa>0$. 
The condition $\kappa\le 0$ was used just once to ensure that the $\kappa$-model triangles with the vertices $p,q,r,s$ are defined.
One can assume instead that perimeters of all triangles in $\spc{L}$ are at most $2\cdot\varpi\kappa$.
This condition holds for most  complete length $\Alex\kappa$ spaces of dimension at least 2.

\parbf{\ref{ex:sturm}.}
Let $\tilde p,\tilde x_1,\dots,\tilde x_n$ be the array in $\EE^n$ provided by the (1+\textit{n})-point comparison (\ref{thm:pos-config}).
We may assume that $\tilde p$ is the origin of $\EE^n$.

Consider an $n{\times}n$-matrix $\tilde M$ with components 
\[\tilde m_{i,j}=\tfrac12\cdot(\dist[2]{\tilde x_i}{\tilde p}{}+\dist[2]{\tilde x_j}{\tilde p}{}-\dist[2]{\tilde x_i}{\tilde x_j}{}).\]
Note that $\tilde m_{i,j}=\langle\tilde x_i,\tilde x_j\rangle$.
Therefore $\tilde M=A\cdot A^\top$ for an $n{\times}n$-matrix $A$ that defines a linear transformation sending the standard basis to the array $\tilde x_1,\dots,\tilde x_n$.
Therefore
\[\bm{s}\cdot \tilde M\cdot \bm{s}^\top=|A^\top\cdot \bm{s}^\top|^2 \ge 0\]
for any vector $\bm{s}$.
Further show that
\[\bm{s}\cdot M\cdot \bm{s}^\top\ge \bm{s}\cdot \tilde M\cdot \bm{s}^\top\]
for any vector $\bm{s}=(s_1,\dots,s_n)$ with nonnegative components.

\parbf{\ref{ex:(3+1)-nonsufficient}.}
It is sufficient to construct a metric on the set of points $\{p$, $x^1$, $x^2$, $x^3$, $x^4\}$ that does not satisfy (4+1)-comparison but does satisfy all (3+1)-comparisons.
To do this, put the $x^i$ in the vertices of a regular tetrahedron in $\EE^3$. Suppose $p$ is its center and reduce the distances $\dist{p}{x^1}{}$ slightly.


\parbf{\ref{ex:strut+embedding}.}
By  (1+\textit{n})-point comparison (\ref{thm:pos-config}), there is a point array $\tilde p,\tilde a^0,\dots,\tilde a^m\in \Lob{m+1}\kappa$ such that
\[\dist{\tilde p}{\tilde a^i}{}=\dist{p}{a^i}{}\quad \text{and}\quad \dist{\tilde a^i}{\tilde a^j}{}\ge\dist{a^i}{a^j}{}\]
for all $i$ and $j$.

For each $i$, set 
$\tilde \xi^i=\dir{\tilde p}{\tilde a^i}\in\mathbb{S}^m=\Sigma_{\tilde p}(\Lob{m+1}\kappa)$.
Note that 
\[\dist{\tilde \xi^i}{\tilde \xi^j}{\mathbb{S}^m}\ge \angk\kappa{p}{ a^i}{ a^j}>\tfrac\pi2.\]

Consider two matrices $S$ and $\tilde S$ with components
$s_{i,j}=\langle\tilde \xi^i,\xi^j\rangle$
and
$\tilde s_{i,j}=\cos[\angk\kappa{p}{a^i}{a^j}]$.
By construction, $S\ge 0$.

Note that $s_{i,j}\le \tilde s_{i,j}\le 0$ if $i\ne j$ and
$s_{i,j}= \tilde s_{i,j}=1$ if $i=j$.
Therefore $\tilde S\ge0$.
The latter implies  we can assume 
\[\dist{\tilde \xi^i}{\tilde \xi^j}{\mathbb{S}^m}= \angk\kappa{p}{ a^i}{ a^j}\]
for each $i$ and $j$.
Whence the statement follows.

\parbf{\ref{ex:flat-in-CAT}.}
Set $\tilde Q=\Conv\{\tilde x^0,\tilde x^1,\dots,\tilde x^\kay\}$.
By Kirszbraun's theorem, the map $\tilde x^i\mapsto x^i$ can be extended to a short map $F\:\tilde Q\to\spc{L}$;
it remains to show that the map $F$ is distance-preserving.

Consider the gradient logarithm map $\log_{x_0}\:\spc{L}\to\T_{x_0}$ corresponding to the gradient exponential map $\gexp_{x_0}$ (see Section \ref{sec:gexp}). The map $\log_{x_0}$ also is short.
Observe that the composition $\log_{x_0}\circ F$ is distance-preserving.
Therefore $F$ is distance-preserving;
in particular we can take $Q=F(\tilde Q)$.

\parbf{\ref{ex:flat-in-CBB}.} Consider vectors $v^i=\lg_{x^0}{x^i}\in\T_{x^0}$.
Show that all the $v^i$ lie in a linear subspace of $\T_{x^0}$ and that $x^i\mapsto v^i$ is distance-preserving.
It follows that we can identify the convex hull $K$ of  the $v^i$ with the convex hull of  the $\tilde x^i$.

Note that the gradient exponential map $\gexp_{x_0}$ maps $v^i$ to $x^i$.
By assumption, 
\[\dist{v^i}{v^j}{}=\dist{x^i}{x^j}{}\eqlbl{eq:vv=xx}\]
for all $i$ and $j$.
By \ref{thm:prop-gexp}, $\gexp_{x_0}$ is a short map.
By \ref{eq:vv=xx}, $\gexp_{x_0}$ cannot be strictly short at a pair of points in $K$.
That is, $\gexp_{x_0}$ is distance-preserving on $K$.

\parbf{\ref{CBA-n-point}.}
Apply \ref{thm:kirsz} for each of the following maps
\begin{itemize}
\item $f_0\:\tilde x\mapsto x$, $\tilde p^1\mapsto p^1$, $\tilde q^1\mapsto q^1$;
\item $f_i\:\tilde p^i\mapsto p^i$, $\tilde p^{i+1}\mapsto p^{i+1}$, $\tilde q^i\mapsto q^i$, $\tilde q^{i+1}\mapsto q^{i+1}$ for $1\le i<n$;
\item $f_n\:\tilde y\mapsto y$, $\tilde p^n\mapsto p^n$, $\tilde q^n\mapsto q^n$.
\end{itemize}
Denote by $F_i$ the short extension of $f_i$.
Observe that $F_{i-1}(\tilde z_i)=F_{i}(\tilde z_i)$ for each $i$ and use it.

\parbf{\ref{ex:perunin-stadler}.} Consider the space $\spc{Y}^{\spc{X}}$ of all maps $\spc{X}\to \spc{Y}$ equipped with the product topology.

Denote by $\mathfrak{S}_F$ the set of maps $h\in \spc{Y}^\spc{X}$ such that the restriction $h|_F$  is short and agrees with $f$ in $F\cap A$.
Note that the sets $\mathfrak{S}_F\subset \spc{Y}^\spc{X}$ are closed and any finite interection of these sets is nonempty.

According to Tikhonov's theorem, $\spc{Y}^{\spc{X}}$ is compact.
By the finite intersection property, the intersection $\bigcap_F\mathfrak{S}_F$ for all finite sets $F\subset X$ is nonempty.
Hence the statement follows.

\parbf{\ref{ex:isbell}.}
The Kuratowsky embedding is a distance-preserving map of $\spc{X}$ into the space of bounded functions $\spc{X}$ equipped with the metric induced by the sup-norm (see Section~\ref{Kuratowsky embedding}).
It remains to show that the latter space is injective.

The second part of the exercise is a classical result of John Isbell \cite{isbell} which was rediscovered several times after him.

\parbf{\ref{ex:kirszbrun-source}.}
To prove the only-if part it is sufficient to consider only the case of two-point spaces $\spc{Z}$.

It remains to prove the if part.

By Zorn's lemma, we may assume that the short map $f\:Q\to \spc{Z}$ is defined on a maximal subset $Q\subset \spc{X}$;
that is, $f$ cannot be extended to a larger set as a short map.
Note that in this case $Q$ is a closed subset of $\spc{X}$.

Assume there is a point $x\in \spc{X}\backslash Q$.
Denote by $\bar x$ a point in $Q$ that lies at minimal distance from $x$.
By the ultratriangle inequality, $\dist{\bar x}{y}{}\le\dist{x}{y}{}$ for any $y\in Q$.
Assigning $f(x)=f(\bar x)$, we extend $f$ to $Q\cup\{x\}$ as a short map --- a contradiction.
Therefore $Q=\spc{X}$, hence the statement.


\parbf{\ref{ex:warp=<}.}
It is sufficient to show that the natural map $\spc{B}\warp{g}\spc{F}\to \spc{B}\warp{f}\spc{F}$ is short.
The latter follows from the fiber-independence theorem (\ref{thm:fiber-independence}).

\parbf{\ref{ex:convexity-in-cone}.}
Show and apply that any geodesic path in $\Cone^\kappa\spc{F}$ projects to a geodesic in $\spc{F}$ of length less than $\pi$.

\parbf{\ref{ex:spherical-join}.}
By \ref{thm:warp-curv-bound:cbb:a}, the space $\spc{U}$, $\spc{V}$, or $\spc{U}\star\spc{V}$ is $\Alex1$ if and only if $\Cone\spc{U}$, $\Cone\spc{V}$, or $\Cone(\spc{U}\star\spc{V})=\Cone\spc{U}\times\Cone\spc{V}$ is $\Alex0$ respectively.

By \ref{thm:warp-curv-bound:cbb:S}, the space $\spc{U}$, $\spc{V}$, or $\spc{U}\star\spc{V}$ is $\CAT1$ if and only if $\Cone\spc{U}$, $\Cone\spc{V}$, or $\Cone(\spc{U}\star\spc{V})=\Cone\spc{U}\times\Cone\spc{V}$ is $\CAT0$ respectively.

It remains to show that the product of two spaces is $\Alex0$ or $\CAT0$ if and only if each space is $\Alex0$ or $\CAT0$ respectively.

\parbf{\ref{ex:poly-unique-geodesic}.}
Assume $\spc{P}$ is not $\CAT0$.
Then by \ref{thm:PL-CAT}, a link $\Sigma$ of some simplex contains a closed local geodesic $\alpha$ with length $4\cdot\ell<2\cdot\pi$.
We can assume that $\Sigma$ has minimal possible dimension;
then by \ref{thm:PL-CAT}, $\Sigma$ is locally $\CAT1$.

Divide $\alpha$ into two equal arcs $\alpha_1$ and $\alpha_2$.

Assume $\alpha_1$ and $\alpha_2$ are length-minimizing, and 
parametrize them by $[-\ell,\ell]$.
Fix a small $\delta>0$ and 
consider the two curves in $\Cone\Sigma$ given in polar coordinates by 
\[\gamma_i(t)=(\alpha_i(\arctan \tfrac t\delta),\sqrt{\delta^2+t^2}).\]
Observe that the curves $\gamma_1$ and $\gamma_2$ are geodesics in $\Cone\Sigma$ having common endpoints.

Observe that a small neighborhood of the tip of $\Cone\Sigma$ admits a distance-preserving embedding into~$\spc{P}$.
Hence we can construct two geodesics $\gamma_1$ and $\gamma_2$ in $\spc{P}$ with common endpoints.

It remains to consider the case where $\alpha_1$ (and therefore $\alpha_2$) is not length-minimizing.

Pass to a maximal length-minimizing arc $\bar\alpha_1$ of $\alpha_1$.
Since $\Sigma$ is locally $\CAT1$, by the no-conjugate-point theorem (\ref{thm:no-conj-pt}) 
there is another geodesic $\bar\alpha_2$ in $\Sigma_p$ that shares endpoints with $\bar\alpha_1$.
It remains to repeat the above construction for the pair $\bar\alpha_1$, $\bar\alpha_2$.

\parit{Remark.}
By \ref{thm:cat-unique} the converse holds as well.

\parbf{\ref{ex:polyKk}.} Apply \ref{thm:tan-is}, \ref{thm:poly-CBB}, and \ref{thm:PL-CAT}.

\parbf{\ref{ex:barycenric-flag}.}
Observe and use that (1) in the barycentric subdivision every vertex corresponds to a simplex of the original triangulation,
and (2) a simplex of the subdivision corresponds to a decreasing sequence of simplexes in the original triangulation. 

\parbf{\ref{ex:obtuce-flag}.}
Use induction on the dimension  to prove that if in a spherical simplex $\triangle$ every edge is at least $\tfrac\pi2$, then 
all dihedral angles of $\triangle$ are at least~$\tfrac\pi2$.

The rest of the proof goes along the same lines as the proof of the flag condition (\ref{thm:flag}).
The only difference is that a geodesic may spend time {}\emph{at least} $\pi$ on each visit to $\Star_v$.

\parit{Remark.}
Note that it is not sufficient to assume only that all the dihedral angles of the simplexes are at least~$\tfrac\pi2$. 
Indeed, the two-dimensional sphere with the interior of a small rhombus removed is a spherical polyhedral space glued from four triangles with angles at least~$\tfrac\pi2$.
On the other hand, the boundary of the rhombus is a closed local geodesic in this space.
Therefore the space cannot be $\CAT1$.

\parbf{\ref{ex:short+commuting}.}
Observe that if we glue two copies of spaces along $A_i$, then the copies of $A_j$ for some $j\ne i$ form a convex subset in the glued space.
Use this and the Reshetnyak gluing theorem (\ref{thm:gluing}) $n$ times, one for each label of the edges.

\parbf{\ref{ex:space-of-trees}.}
The space $\spc{T}_n$ has a natural cone structure whose vertex is the  completely degenerate tree --- all its edges have zero length.

Note that the space $\Sigma$
over which the cone is taken comes naturally with a triangulation 
by right-angled spherical simplexes.
Each simplex corresponds to the combinatorics of a possibly degenerate tree.

Note that the link of any simplex of this triangulation satisfies the no-triangle condition.
Indeed, fix a simplex $\triangle$ of the complex;
suppose it is described by a possibly degenerate topological tree $t$.
A triangle in the link of  $\triangle$ can be described by three ways to resolve a degeneracy of $t$ by adding one edge, where 
(1) any pair of these resolutions can be done simultaneously, but (2) all three cannot be done simultaneously.
Direct inspection shows that this is impossible.

Therefore by Proposition~\ref{prop:no-trig} our complex is flag.
It remains to apply the flag condition (\ref{thm:flag}) and \ref{thm:warp-curv-bound:cbb:a}.

\parbf{\ref{ex:cubical-complex}.}
Apply the flag condition (\ref{thm:flag}) and Theorem~\ref{thm:warp-curv-bound:cbb:S}.

\parbf{\ref{ex:norays}.}
Consider a cube in the Hilbert space
%;that is the space 
of all sequences $x_1,x_2,\dots$ such that $|x_i|\le 1$ and $x_1^2+x_2^2+\dots<\infty$ with the metric inducted by $\ell^2$-norm.

\parbf{\ref{ex:d_q dist_p(v)=-<dri p q, v>-CAT} and \ref{ex:d_q dist_p(v)=-<dri p q, v>}.} Apply the strong angle lemmas
\ref{lem:strong-angle-cba}
and \ref{lem:strong-angle}.


\parbf{\ref{ex:df(v)=<grad f,v>}.}
Since $\alpha$ is Lipschitz, so is $f\circ\alpha$.
By the standard Rademacher theorem, the derivative $(f\circ\alpha)'$ is defined almost everywhere.
In particular, 
\[(\dd_{\alpha(t)}f)(\alpha^+(t))+(\dd_{\alpha(t)}f)(\alpha^-(t))\ae0.\]

Further, by the extended Rademacher theorem (more precisely its 1-dimensional case; see Proposition~\ref{prop:Rademacher-dim=1}),
we have 
\[\alpha^+(t)+\alpha^-(t)\ae0.\]
In particular,
\[\<\nabla_{\alpha(t)}f,\alpha^+(t)\>+\<\nabla_{\alpha(t)}f,\alpha^-(t)\>
\ae0.\]

Finally, by the definition of gradient, we have 
\[\<\nabla_{\alpha(t)}f,\alpha^\pm(t)\>\ge (\dd_{\alpha(t)}f)(\alpha^\pm(t)).\]
Hence the result follows.

\parbf{\ref{ex:d dist(grad)<0}.}
Without loss of generality we may assume that geodesics $[pa]$ and $[pb]$ are uniquely defined.
Applying \ref{ex:d_q dist_p(v)=-<dri p q, v>}, we have
\begin{align*}(\dd_p\distfun{a}{}{})(\nabla_p\distfun{b}{}{})&=-\langle\dir pa,\nabla_p\distfun{b}{}{} \rangle\le
\\
&\le -\dd_p\distfun{b}{}{}(\dir pa)=
\\
&=\langle\dir pb, \dir pa\rangle=
\\
&= \cos\mangle\hinge p a b \le 
\\
&\le \cos\angk\kappa p a b.
\end{align*}


\parbf{\ref{ex:compact-dimension-cbb}.}
Suppose that $\spc{L}$ is infinite dimensional.
Denote by $\Omega_m\subset \spc{L}$ the set of all points $p$ with $\rank_p\ge m$.
Evidently $\Omega_1\supset \Omega_2\supset\dots$ and $\Omega_m$ is open for each $m$.

By \ref{LinDim+}, each $\Omega_m$ is dense in $\spc{L}$.
Hence there is a G-delta dense set of points $p\in\spc{L}$ such that $\rank_p=\infty$.
It follows that $\Sigma_p$ is not compact.

\parbf{\ref{ex:sharafutdinov}.}
Choose a finite sequence $t_0<\dots<t_n$.
Denote by $\Phi_{t_i}$ the composition of projections to $K_{t_0},\dots, K_{t_i}$.
Pass to a limit of the $\Phi_{t_i}$ as the sequence becomes denser in the parameter interval. Observe that the limit $\phi_t$ does not depend on the choice of the sequences. The exercise follows. 


\parbf{\ref{ex:elf-contracting}.}
Let $\ell(t)=\dist{\alpha(t)}{\alpha(t_3)}{}$.
Note that 
\[\ell'(t)\le -\langle \nabla_{\alpha(t)}f,\dir{\alpha(t)}{\alpha(t_3)}\rangle.\]

Observe that the function $t\mapsto f\circ\alpha(t)$ is nondecreasing;
in particular, $f(\alpha(t_1))\le f(\alpha(t_2))\le f(\alpha(t_3))$.
Therefore 
\begin{align*}\langle \nabla_{\alpha(t)}f,\dir{\alpha(t)}{\alpha(t_3)}\rangle&\ge\dd_{\alpha(t)}f(\dir{\alpha(t)}{\alpha(t_3)})\ge 0
\end{align*}
for any $t\in[t_1,t_2]$.
Therefore $\ell'\le 0$ for any $t\in[t_1,t_2]$. Hence the statement.

\parbf{\ref{ex:grad-curve-condition}.} 
Without loss of generality, we may assume that $(f\circ\alpha)'(t)>0$ for any $t$.

Let $\hat\alpha$ be the arclength reparametrization of $\alpha$.
Note that 
\[(f\circ\hat\alpha)'(s)\ge |\nabla_{\hat\alpha(s)}f|\]
almost everywhere.
Therefore, by Theorem~\ref{thm:grad-like-2nd-def}, $\hat\alpha$ is a gradient-like curve.
It remains to apply Lemma~\ref{lem:grad--grad-like}.

\parbf{\ref{ex:grad-curve-analitic}.}
Use \ref{ex:grad-curve-condition} to prove the only-if part.

To prove the if part, set $h(z)=\tfrac12\cdot\dist[2]{x}{z}{}$.
If $\alpha$ is an $f$-gradient curve, then 
\begin{align*}
(h\circ\alpha)^+&\ge \dist{\alpha(t)}{x}{}\cdot\langle\dir{\alpha(t)}{x},\nabla_{\alpha(t)}f\rangle\ge
\\
&\ge \dist{\alpha(t)}{x}{}\cdot\dd_{\alpha(t)}f(\dir{\alpha(t)}{x})\ge 
\\
&\ge f(x)-f\circ\alpha(t).
\end{align*}
It remains to integrate the inequality and observe that $f\circ\alpha$ is nondecreasing.

\parbf{\ref{ex:geodesic}.}
Consider $(x,\kappa)$- and $(z,\kappa)$-radial curves that start at $y$
and observe that they form a geodesic from $x$ to $z$.

\parbf{\ref{ex:gexp}.} Set $q=p+v$ and $q'=\gexp_pv$. 
By radial comparison, $\dist{q'}{x}{}\le \dist{q}{x}{}$ for any $x\in \spc{L}$.
If $q\in \spc{L}$, this implies that $q=q'$.
Otherwise note that $q'$ lies on the boundary line of $\spc{L}$, and $\proj(q)$ is the only point on this line that satisfies the inequality.

\parbf{\ref{ex:inv-gexp}.}
By angle comparison,
$|\nabla_x\distfun p|\ge-\cos \angk\kappa xpq$.

Choose a $(p,\kappa)$-radial curve $\alpha$ that starts at $p$.
Observe that 
\[(\distfun p\circ \alpha)^+(t)\ge-|\alpha^+(t)|\cdot \cos \angk\kappa {\alpha(t)}p q\]
and
\[(\distfun q\circ\alpha)^+(t)\ge -|\alpha^+(t)|.\]
Therefore $t\mapsto\angk\kappa q{\alpha(t)}p$  is nondecreasing, hence the result.

\parbf{\ref{ex:bry-cover}.} 
Choose a regular point $p\in\spc{L}$.
Observe that $\gexp_p^\kappa(\cBall[\0,R]_{\T_p})\z=\spc{L}$.

Use \ref{ex:inv-gexp} to show that 
$\gexp_p^\kappa(\partial\cBall[\0,R]_{\T_p})\supset\partial\spc{L}$ and apply \ref{thm:prop-gexp:short}.

