%%!TEX root = arXiv.tex
\parbf{Exercise~\ref{ex:urysohn}.} See the construction of Urysohn's space \cite[3.11$\tfrac{3}{2}_+$]{gromov-MS}.

\parbf{Exercise~\ref{ex:lebedeva-petrunin}.}
Read \cite{lebedeva-petrunin}.

\parbf{Exercises \ref{ex:busemann-CBB} and \ref{ex:busemann-CBA}.}

By the definition of Busemann function,
\begin{align*}
\exp(\sqrt{-\kappa}\cdot\bus_\gamma) 
&= \exp \left[\lim_{t\to \infty} \sqrt{-\kappa}\cdot(\distfun{{\gamma (t)}}{}{} - t)\right] 
\\
&= \lim_{t\to \infty} \left(\exp \left[\sqrt{-\kappa}\cdot(\distfun{\gamma (t)}{}{} -t)\right]
+ \exp\left[\sqrt{-\kappa}\cdot(-d_{\gamma (t)}-t)\right]\right)\\
&=  \lim_{t\to \infty} \left(2\cdot \cosh \left[\sqrt{-\kappa}\cdot\distfun{\gamma (t)}{}{}\right]\cdot \exp\left[\sqrt{-\kappa}\cdot(-t)\right]\right).
\end{align*}

By the function comparison definitions of $\CAT\kappa$ space (\ref{function-comp}) or $\Alex{\kappa}$ space (\ref{comp-kappa}),  for any $p\in \spc{U}$ the function\\ $f=\cosh \sqrt{-\kappa}\circ\distfun{p}{}{}$ satisfies $f''+\kappa \cdot f\ge 0$ (respectively  $f''+\kappa \cdot f\le 0$). The result follows.

\parbf{Exercise~\ref{ex:short-retraction-CBA(1)}.}
Without loss of generality, we may assume that $p\in K$.

If $\dist{K}{x}{}\ge\pi$, then set $\map[2](x)=p$.

Otherwise, if $\dist{K}{x}{}<\pi$, by Closest-point projection lemma~\ref{lem:closest point}, 
there is unique point $x^*\in K$ that minimizes distance to $x$;
that is, $\dist{x^*}{x}{}=\dist{K}{x}{}$.
Let us define $\ell_x$, $\phi_x$ and $\psi_x$ using the floowing identities:
\begin{align*}
\ell_x&=\dist{p}{x^*}{},
\\
\phi_x&=\tfrac\pi2-\dist[{{}}]{x^*}{x}{},
\\
\sin\psi_x&=\sin\phi_x\cdot\sin\ell_x, 
\ \ 0\le \psi_x\le \tfrac\pi2.
\intertext{Set}
\map[2](x)&=\geod_{[px^*]}(\psi_x).
\end{align*}

Note that $\map[2]$ is a retraction to $K$; 
that is,
$\map[2](x)\in K$ for any $x\in \spc{U}$
and 
$\map[2](a)=a$ for any $a\in K$.

Let us show that $\map[2]$ is short.
Given $x,y\in\oBall(K,\tfrac\pi2)$, set
\begin{align*}
x'&=\map[2](x)
&
y'&=\map[2](y)
\\
r&=\dist{x}{y}{}
&
r'&=\dist{x'}{y'}{}
\\
d&=\dist{x^*}{y^*}{}
&
\alpha&=\angk1{p}{x^*}{y^*}
\end{align*}

Note that 
\[\cos r\le 
\cos\phi_x\cdot\cos\phi_y
-
\cos d\cdot\sin\phi_x\cdot\sin\phi_y.\eqlbl{eq:cos(r)}\]

Indeed, if $x,y\notin K$,
then 
$\mangle\hinge{x^*}{x}{y*}, 
\mangle\hinge{y^*}{y}{x*}
\ge 
\tfrac\pi2$
and
the inequality~\ref{eq:cos(r)} follows from the Arm lemma (\ref{lem:arm}).
If $x\in K$ and $y\notin K$, we get \ref{eq:cos(r)}, by angle comparison (\ref{cat-hinge}) 
since $\mangle\hinge{y^*}{y}{x*}\ge \tfrac\pi2$.
The same way \ref{eq:cos(r)} is proved 
in case $x\notin K$ and $y\in K$.
Finally, if $x,y\in K$, $\phi_x=\phi_y=\tfrac\pi2$ and $r=d$;
that is, the inequality trivially holds.

Further note that
\[\cos\alpha
=
\frac{\cos d-\cos \ell_x\cdot\cos\ell_y}{\sin\ell_x\cdot\sin\ell_y}.\]
Applying angle-sidelength  monotonicity (\ref{cor:monoton-cba}) we get
\begin{align*}
\cos r'&\ge
\cos\psi_x\cdot\cos\psi_y
-
\cos \alpha \cdot\sin\psi_x\cdot\sin\psi_y=
\\
&=
\cos\psi_x\cdot\cos\psi_y
-(\cos d-\cos \ell_x\cdot\cos\ell_y)\cdot\sin\phi_x\cdot\sin\phi_y\ge
\\
&\ge \cos\psi_x\cdot\cos\psi_y
-\cos d\cdot\sin\phi_x\cdot\sin\phi_y
\end{align*}


Note that 
$\psi_x\le \phi_x$
and
$\psi_y\le \phi_y$;
in particular,
\[
\cos\phi_x\cdot\cos\phi_y\le \cos\psi_x\cdot\cos\psi_y.
\]
Hence 
\[\cos r'\ge \cos r;\]
that is, the restriction $\map[2]|\oBall(K,\tfrac\pi2)$ is short.
Clearly $\map[2]$ is continuous,
since the complement of $\oBall(K,\tfrac\pi2)$ is mapped to $p$,
we get that $\map[2]$ is short; that is,
\[r'\le r \eqlbl{eq:cos=<cos}\]
for any $x,y\in\spc{U}$.

If we have equality in \ref{eq:cos=<cos}
then 
\[\cos\ell_x\cdot\cos\ell_y\cdot\sin\phi_x\cdot\sin\phi_y=0.\]
If $K\subset \oBall(p,\tfrac\pi2)$, then $\ell_x,\ell_y<\tfrac\pi2$;
which implies that $x\in K$ or $y\in K$.
Without loss of generality we may assume that $x\in K$.

It remains to show that if $y\notin K$ 
then the inequality~\ref{eq:cos=<cos}
is strict.
If $\dist{K}{y}{}\ge\tfrac\pi2$, then \ref{eq:cos=<cos} holds since 
the left hand side is $<\tfrac\pi2$,
while right hand side is $\ge \tfrac\pi2$.
If $\dist{K}{y}{}<\tfrac\pi2$, then $\phi_y>0$ and clearly $\psi_y<\phi_y$,
hence the inequality~\ref{eq:cos=<cos} is strict.
\qeds

We fail to find a transparent geometric proof of the statement above.
Below you will find a geometric way to think about the construction; 
%%%DOWN
compare to the construction 
in the proof of Kirszbraun's theorem (\ref{thm:kirsz+}).
%%%UP

\parit{Geometric interpretation of the map $\map[2]$.}
Set $\mathring{\spc{U}}=\Cone \spc{U}$;
denote by $\mathring{K}$ the subcone of $\mathring{\spc{U}}$ spanned by $K$.
The space $\spc{U}$ can be naturally identified with the unit sphere in $\mathring{\spc{U}}$;
that is, the set 
\[\set{z\in \mathring{\spc{U}}}{|z|=1}.\]

According to \ref{thm:warp-curv-bound:cat}, $\mathring{\spc{U}}$ is $\CAT0$.
Note that $\mathring{K}$ forms a convex closed subset of $\mathring{\spc{U}}$.
According to \ref{lem:closest point}, for any point $x$ there is unique point $\hat x\in \mathring{K}$
that minimize the distance to $x$;
that is, $\dist{\hat x}{x}{}=\dist{K}{x}{}$.
(If $|\hat x|\ne0$, then in the notations above we have
$x^*=\tfrac1{|\hat x|}\cdot\hat x$.)

Consider the ray $t\mapsto t\cdot p$ in  $\mathring{\spc{U}}$.
According to ???, %ASK Stephanie???
for given $s\in \mathring{\spc{U}}$
the geodesics $\geod_{[s\ t\cdot p]}$ converge as $t\to\infty$ to a ray, 
say $\alpha_s\:[0,\infty)\to \mathring{\spc{U}}$.



Note that if $|x|=1$, then $|\hat x|\le 1$.
By assumption for any $a\in K$ the function $t\mapsto |\alpha_a(t)|$ is monotonicity increasing.
Therefore there is unique value $t_x\ge 0$ such that
$|\alpha_{\hat x}(t_x)|=1$.
Consider the map $\map[2]\:\spc{U}\to K$
defined as 
\[\map[2](x)=\alpha_{\hat x}(t_x).\]

\parbf{Exercise~\ref{ex:two-rays}.}
Consider the angle $A$ in the plane of measure $\pi-\alpha$.
Note that $A$ is $\CAT0$.
Therefore by Reshetnyak gluing theorem \ref{thm:gluing},
by gluing a side of $A$ to $\gamma_1$ in $\spc{U}$ we obtain a $\CAT0$ space, say $\spc{U}'$.

Note that $\gamma_2$ together with the other side of $A$ forms a both sides infinite geodesic, say $\gamma$ in $\spc{U}'$.
In particular, $\gamma$ is a convex set isometric to $\RR$.

Glue a half-plane along its boundary to $\gamma$.
By Reshetnyak gluing theorem \ref{thm:gluing} the obtained space is $\CAT0$.

It remains to note that this space can be obtained directly by gluing $\spc{U}$ to with $Q$ along $\gamma_1$ and $\gamma_2$.

\parbf{Exercise~\ref{ex:glue-spherical-suspension}.}
Since $K$ is $\pi$-convex, it is $\CAT1$.
By \ref{thm:warp-curv-bound:cat}, the spherical suspension $\Susp K$ is $\CAT1$ as well.
Let us glue $\Susp K$ to $\spc{U}$ by along $K$;
according to Reshetnyak's gluing theorem, the obtained space, say $\spc{U}'$ is $\CAT1$.

Consider the geodesic path $\gamma\:[0,1]$ from $p$ to a pole of the suspension in $\spc{U}'$.
Set $K_t=\spc{U}\cap\cBall[\gamma(t),\tfrac\pi2]$.
By \ref{cor:convex-balls}, $K_t$ is $\pi$-convex of any $t$ and monotonicity of the family should be evident.

\parit{Remark.}
Note that if one applies Shrafutdinov's construction to the family of convex sets provided by the exercise we get a short strong deformation retraction from $\cBall[p,\tfrac\pi2]$ to $K$;
that is, there is a family of maps $\phi_t\:\cBall[p,\tfrac\pi2]\to \cBall[p,\tfrac\pi2]$ such that 
the function $t\mapsto \dist{\phi_t(x)}{\phi_t(y)}{}$ is nonincreasing for any pair of points $x,y\in\cBall[p,\tfrac\pi2]$, $\phi_t(x)=x$ for any $x\in K$ and $\phi_1(\cBall[p,\tfrac\pi2])=K$. 
Moreover we can assume that there is a family of short maps $\phi_t(\cBall[p,\tfrac\pi2])= K_t$ and $\phi_t(x)=x$ for any $t$ and $x\in K_t$.
It leads to another solution of Exercise~\ref{ex:short-retraction-CBA(1)}.

\parbf{Exercise~\ref{ex:isometric-majorization}.}
\textit{(Easier way.)} 
Let 
$(t,s)\mapsto \gamma_t(s)$ be the line-of-sight map 
for $\alpha$ from $\alpha(0)$,
and 
$(t,s)\mapsto \tilde \gamma_t(s)$ be the line-of-sight map 
for $\tilde \alpha$ from $\tilde \alpha(0)$.
Consider the map  $F\:\Conv\tilde \alpha\to \spc{U}$ such that 
$F\:\tilde \gamma_t(s)\mapsto \gamma_t(s)$.

Show that $F$ majorizes $\alpha$
and conclude that $F$ is distance-preserving.

\parit{(Harder way.)}
Prove and apply the following lemma together with the Majorization theorem.
\begin{thm}{Lemma}\label{lem:short+convex}
Let $\alpha$ and $\beta$ be two convex curves in $\Lob2\kappa$.
Assume 
\[\length \alpha=\length\beta<2\cdot\varpi\kappa\]
and there is a short bijecction $f\:\alpha\to\beta$.
Then $f$ is an isometry.
\end{thm}
