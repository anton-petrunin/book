%%!TEX root = all.tex
\chapter{Nonnegatively curved spaces}


\section{Soul theorem}

In Riemannian case, the following theorem was proved in \cite{cheeger-gromoll-soul}.
The last statement was proved in \cite{sharafutdinov}.

\begin{thm}{Soul theorem}
Let $\spc{L}$ be an $m$-dimensional complete length $\Alex0$ space. 
Then there is a convex compact subset $S\subset \spc{L}$ (which is called \emph{soul}\index{soul} of $\spc{L}$), such that  $S$ is a $\kay$-dimensional $\Alex{0}$ space with $\kay\le m$,
$S$ has no boundary,
$S$ is a strict short deformational retract of $\spc{L}$; that is, there one parameter of short maps $h_t\:\spc{L}\to \spc{L}$ such that $h_0=\id$, $h_1(\spc{L})=S$, $h_t(s)=s$ for any $s\in S$ and $t\in[0,1]$.

More over, $S$ is uniquely defined up to isometry.
\end{thm}

If $\spc{L}$ is a Riemannian manifold then in fact
$\spc{L}$ is diffeomorphic to the total space of the normal bundle of $\iota(S)$ in $\spc{L}$. 
For Alexandrov spaces a similar strengthening is false as the following example shows:

\parbf{Example.} Consider product space $\SS^2\times \EE^2$ equiped with isometric $\ZZ_2$-action that acts by cental simmetry on each $\SS^2$ and $\EE^2$; thus it has exactly two fixed points $v$ and $w$ on $\SS^2\times 0$.

According to ???,
the factor-space $\spc{L}=\SS^2\times \EE^2/\ZZ_2$ is a $4$-dimensional Alexandrov space with curvature $\ge 0$.
It has exactly two singular points, the orbits of $v$ and $w$, a neigborhood at each is homeomorphic to $\Cone\RP^3$.
The image of $\SS^2\times 0$ in $\spc{L}$ is homeomorphic to $\SS^2$ and it forms an Alexandrov space without boundary.

Thus, the soul of $\spc{L}$ is homeomorphic to $\SS^2$. 
Since $\spc{L}$ has exactly two topological singularities it can not be presented as a fiber bunndle over $\SS^2$. 

\smallskip

The proof of is based on ???,
the rest of the proof repeats the proof of classical soul theorem \cite{cheeger-gromoll-soul} and \cite{sharafutdinov}.

\parit{Proof.}???
\qeds







\section{Shioya's splitting}

For Riemannian case, corollary \ref{cor:shioya} was proved in \cite{mendonca:shioya}, it was conjectured in \cite{shioya} where a weaker statement had been proved.
The followng more general theorem was obtained in \cite{perelman:collapsing} independently.

Given  an $m$-dimensional complete length $\Alex{}$ space $\spc{L}$ and $p\in \spc{L}$, set\index{$\T_\infty$} $(\T_\infty,0)=\GHlim_{\eps\to0} \eps\blow(\spc{L},p)$.
Clearly $\T_\infty$ does not depends on the choice of $p$.
From ??? it is clear that $\T_\infty \spc{L}$ is a $\kay$-dimensional complete length $\Alex{0}$ space for $\kay\le m$.
The space $\T_\infty \spc{L}$ is called \emph{cone at infinity}\index{cone at infinity}, it has natural cone structure WHy???.
Thus, there is a $(\kay-1)$-dimensional $\Alex{1}$ space $\Sigma_\infty$\index{$\Sigma_\infty$} such that $\T_\infty \spc{L}\iso\Cone\Sigma_\infty \spc{L}$.
The space $\Sigma_\infty \spc{L}$ is called \emph{ideal boundary}\index{ideal boundary} of $\spc{L}$.

\begin{thm}{Theorem}\label{thm:shioya} 
Let $\spc{L}$  be an $m$-dimensional complete length $\Alex{0}$ space
and $\dim\T_\infty=\kay$. 
Assume $\T_\infty$ has no extremal subsets, then $\spc{L}$ can be isometrically splitted into product
\[\spc{L}=K\oplus \spc{L}'\]
where $K,\spc{L}'\subset \spc{L}$ are closed convex subsets, $\dim \spc{L}'=\kay$ and $K$ is copmact.
\end{thm}

\begin{thm}{Corollary}\label{cor:shioya}
Let $\spc{L}$  be an $m$-dimensional complete length $\Alex{0}$ space
and $\dim\Sigma_\infty \spc{L}=\kay-1$. 
Assume $\rad\Sigma_\infty \spc{L}>\tfrac\pi2$, then 
$\spc{L}$ can be isometrically splitted into product
\[\spc{L}=K\oplus \spc{L}'\]
where $K,\spc{L}'\subset \spc{L}$ are closed convex subsets, $\dim \spc{L}'=\kay$ and $K$ is copmact.
\end{thm}

Note that if $\rad\Sigma_\infty \spc{L}=\tfrac\pi2$, the conclusion of the corollary does not hold even in Riemannian category.
In fact by taking a product of flat open M\"obius band with $\RR$ we have a counterexample.



\section{Erd\H{o}s' problem rediscovered}

The following theorem was proved in ???.

\begin{thm}{Theorem}\label{thm:extr-point}
Let $\spc{L}$ is an $m$-dimensional complete length $\Alex0$ space.
Then it has at most $2^m$ one-point extremal sets.
\end{thm}


The proof is a translation of proof of classical problem in combinatoric geometry to Alexandrov's language.

\begin{thm}{Erd\H{o}s' problem}
Let $F$ be a set of points in $\EE^m$ such that any triangle formed by three distinct points in $F$ has no obtuse angles.
Then number of elements in $F$ can not exeed $2^m$.

Moreover, if $|F|=2^m$, then $F$ consists of vertexes of right paralelepiped.
\end{thm}

This problem was posted in \cite{erdos} and solved in \cite{danzer-gruenbaum}.
The question of classifing all $\spc{L}$ with the maximal number one-point extremal sets turns out to be more delicate see \cite{lebedeva}.
Let us describe a class of examples.
Let $\TT^m$ be standart torus 
and $\Gamma$ be a subgroup generated by all coordinte reflections.
Clearly, $\Gamma$ is isomorphic to $\ZZ_2^m$ and it has exactly $2^m$ fixed points in $\TT^m$.
Let $\Gamma'\subset\Gamma$ be a subgroup that has the same fixed point set as $\Gamma$ 
and $g$ be a flat $\Gamma'$-ivariant metric on $\TT^m$.
Then, $(\TT^m,g)/\Gamma'$ is an $m$-dimensional complete length $\Alex{}$ space and it has exactly $2^m$ one-point extremal subsets; it follows from ??? and ???.
The above construction does not describes all the spaces with $2^m$ one-point extremal subsets, but it is close to the correct answer.

The finding reasonable estimates for maximal number of extremal one-point subsets in an $m$-dimensional complete length $\Alex{1}$ space is completely open;
it analogous to the following problem: finding maximal number $n(m)$ such that there are points $p_1,p_2,\dots p_n$ in $\EE^m$ such that any triangle $\trig{p_i}{p_j}{p_\kay}$ is acute.
It is expected $n(m)\ll 2^m$, but so far it is only known that 
\begin{enumerate}
\item $n(m)\ge 2\cdot m$ --- the configuration s slight perturbation of vertexes of $m$-octahedra.
\item $n(m)\le ???$
\end{enumerate}


\parit{Proof of \ref{thm:extr-point}.}
Let $\{p_i\}$, $i\in\{1,2,\dots,N\}$ be the one-point extreaml sets.
For each $p_i$ consider its open Voronoi domain $V_i$; that is, 
\[V_i=\set{x\in \spc{L}}{\dist{p_i}{x}{}<\dist{p_j}{x}{}\ \t{for any}\ j\not=i}.\]
Clearly $V_i\cap V_j=\emptyset$ if $i\not=j$.
Note that $\vol_mV_i>\frac{1}{2^m}\vol_m \spc{L}$.

Indeed, fix $i$ and for given $\alpha\in(0,1)$, consider $\alpha$-homothety $\map_\alpha\:\spc{L}\to \spc{L}$ with center at $p_i$; 
that is, for each point $x\in \spc{L}$ choose a geodesic $[p_ix]$ and set
$\map_\alpha x=\geod_{[p_ix]}(\alpha\dist{p_i}{x}{})$.
From comparison we have that $\vol(\map_\alpha \spc{L})\ge\alpha^m\vol \spc{L}$.
For any $x\in \spc{L}$ and all $\alpha<\tfrac{1}{2}$ we have $\map_\alpha x\in V_i$.
Assume $x'=\map_\alpha x\notin V_i$,
then threre is $p_j$ such that $\dist{p_i}{x'}{}\ge\dist{p_j}{x'}{}$.
Then from comparison, we have $\angk0{p_j}{p_i}{x}>\tfrac\pi2$;
that is, $p_j$ does not form a one-point extremal set.???
\qeds

\section{Circle actions on 4-dimensional manifolds}



The following result was obtained by Hsiang and Kleiner in \cite{hsiang-kleiner}.
The original proof used Alexandrov geometry implicitly;
in fact at the moment 
Alexandrov spaces with lower curvature bound were defined only in dimension $2$.

\begin{thm}{Theorem}\label{thm:keliner}
Let $M$ be a connected complete Riemannian manifold 
and $\mathbb{S}^1\acts M$ be isometric effective circle action.
Then
\begin{subthm}{thm:keliner:nonneg}
If $M$ has nonnegative sectional curvature, then the action has at most $4$ isolated fixed points.
\end{subthm}

\begin{subthm}{thm:keliner:positive}
If $M$ has positive sectional curvature, then the action has at most $3$ isolated fixed points.
\end{subthm}

\end{thm}

\parit{Proof.}
Consider the quotient space $\spc{L}=M/\mathbb{S}^1$.

Assume $x$ is a fixed point of $\mathbb{S}^1$-action;
denote by $y$ the projection of $x$ in $\spc{L}$.
Note that

\begin{clm}{}\label{clm:Sigma=<sphere/2}
$\Sigma_y\le \tfrac12\cdot\SS^2$;
that is there is a noncontracting map $\Sigma_y\to \tfrac12\cdot\SS^2$.
In particular, 
\[\mangle\hinge y{z^1}{z^2}+\mangle\hinge y{z^2}{z^3}+\mangle\hinge y{z^3}{z^1}\le\pi\]
for any triple of points $z^1,z^2,z^3\in\spc{L}$.
\end{clm}

Indeed, $\Sigma_x\iso\SS^3$.
Therefore $\Sigma_y\iso\SS^3/\SS^1$, 
for an isometric $\SS^1$-action.
Note that we can identify $\SS^3$ as the unit sphere in $\CC^2$
in such a way that the $\SS^1$-action is given by diagonal matrices
$\left(\begin{smallmatrix}
z^p&0\\0&z^q
\end{smallmatrix}\right)$ for some realatively prime integers $p$, $q$
and $z\in\SS^1\subset\CC$.

Consider the torus action $\TT^2\acts\SS^3$ by diagonal matrices
$\left(\begin{smallmatrix}
v&0\\0&w
\end{smallmatrix}\right)$.
Note that $[0,\tfrac\pi2]\iso\SS^3/\TT^2$
and $\Sigma_w$ is isometric to the warped product $\SS^1\warp{f_{p,q}}[0,\tfrac\pi2]$ (compare with Exercise~\ref{ex:chohom-1=warped-product}).
Let us compute the warping function $f_{p,q}$. 
If $t\in [0,\tfrac\pi2]$
is the projection of $x\in \SS^3$
then
\begin{align*}
f_{p,q}(t)
&=\frac{\area [\TT^2\cdot x]}{\length[\SS^1\cdot x]}
\\
&=
\frac{\sin t\cdot\cos t}{\sqrt{(p\cdot\sin t)^2+(q\cdot\cos t)^2}}
\end{align*}


Note that 
$f_{1,1}(t)\ge f_{p,q}(t)$
for any pair $(p,q)$ of relatively prime inegers and any $t\in[0,\tfrac\pi2]$ and
\[\tfrac12\cdot\SS^2\iso \SS^1\warp{f_{1,1}}[0,\tfrac\pi2],\]
(In other words $\tfrac12\cdot\SS^2$ is isometric to the orbit space of Hopf fibration on $\SS^3$.)
Applying Exercise \ref{ex:warp=<}, we get \ref{clm:Sigma=<sphere/2}.
\claimqeds

\parit{(\ref{SHORT.thm:keliner:positive}).}
Assume the circle action has 4 isolated points.
Denote by $y_1,y_2,y_3,y_4$ their projections in $\spc{L}$.

According to ???, $\spc{L}$ is a complete length $\Alex\kappa$ space for some $\kappa>0$.
According to Claim~\ref{clm:Sigma=<sphere/2},
\[\mangle \hinge{y_i}{y_j}{y_k}\le\tfrac\pi2\]
for any distinct $i$, $j$ and $k$.
In particular non of 4 triangles $[y^1y^2y^3]$,
$[y^1y^2y^4]$,
$[y^1y^3y^4]$,
$[y^2y^3y^4]$
is degenerate.
Therefore 
The sum of the angles in each triuangle is strictly larger than $\pi$.
Denote by $\omega$ the sum of all 12 angles of these 4 triangles.
We get
\[\omega>4\cdot\pi.\eqlbl{eq:omega>4pi}\]

From Claim~\ref{clm:Sigma=<sphere/2}, 
the any 3 angles among these 12 that share one vertex have sum most $\pi$.
Therefore 
\[\omega\le 4\cdot\pi.\]
The later contradicts \ref{eq:omega>4pi}.

\parit{(\ref{SHORT.thm:keliner:nonneg}).}
Assume there are 5 fixed points of the circle action.
Denote by $y_1,y_2,y_3,y_4,y^5$ their projections in $\spc{L}$.

\qeds

The result above was used to classify positively and nonnegatively curved mainifolds that admit isometric circle action.
This classification was improved later by Grove and Wilking in \cite{grove-wilking}.
Namely they show the following.

\begin{thm}{Theorem}
A closed nonnegatively curved complete simply connected 4-dimensional manifold $M$
with an isometric circle action is diffeomorphic to
$\mathbb{S}^4$,
$\CP^2$,
$\mathbb{S}^2\times\mathbb{S}^2$
or one of
$\CP^2\#(\pm\CP^2)$
and the action extends
to a smooth torus
action.

In particular, if $M$ is positively curved then the circle action is equivariantly diffeomorphic to a linear action on 
$\mathbb{S}^4$,
$\RP^2$
or
$\CP^2$.
\end{thm}

\section{Remarks and open questions}

In \cite{perelman-soul}, it was proved that if $\spc{L}$ is non-negatively curved Riemannian manifold 
and $S\subset \spc{L}$ is its soul 
then Sharafundinov's returction $\Shar\:\spc{L}\to S$ is a submetry;
that is, for any $x\in \spc{L}$ and $y\in S$ there is $y'\in \Shar^{-1}(y)$ such that  $\dist{\Shar(x)}{y}{}=\dist{x}{y'}{}$. 

This statement is likely remains true for non-negatively curved Alexandrov space, 
but the proof can not be generalized directly. 

\section{Exercises}

\begin{thm}{Exercise}
Let $\spc{L}$ be a complete length $\Alex{0}$ space,
$b_1,\dots,b_\kay$ be an array of Busemann's functions 
and $\lambda_1,\dots,\lambda_\kay$ be an array of positive real numbers.
Assume 
\[\lambda_1\cdot b_1+\dots+\lambda_\kay\cdot b_k\]
admits minimum in $\spc{L}$.
Show that $\spc{L}$ admits splitting as $\spc{L}'\oplus H$,
where $H$ is isometric to a Hilbert space and for 
\end{thm}


