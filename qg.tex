%%!TEX root = all.tex
\chapter{Quasigeodesics %ready
}\label{chap:qg}



The class of quasigeodesics%
\footnote{It should be noted that the class of quasigeodesics described here 
has nothing to do with the Gromov's quasigeodesics in $\delta$-hyperbolic spaces.} 
generalizes the class of geodesics to nonsmooth metric spaces.
This class of curves was first considered for $2$-dimensional convex hypersurfaces in the Euclidean space
by Alexandrov in \cite{alexandrov:qg}. 
Alexandrov defined quasigeodesics as curves which ``turn'' right and left simultaneously.
This type of curves was studied further in \cite{pogorelov:qg}, \cite{alexandrov-burago},
\cite{milka:qg}.
Later, it was generalized to surfaces with bounded integral
curvature \cite{alexandrov:int-qg} and to multidimensional polyhedral spaces \cite{milka:poly1},
\cite{milka:poly2}.
A. D. Milka gave a characterization of quasigeodesics using notion of development (\ref{def:devel}).
This characterization made it possible to define quasigeodesics multi-dimensional Alexandrov's spaces;
this was first done in Petrunin's master thesis and developed further in \cite{perelman-petrunin:qg}.

In Alexandrov's spaces, quasigeodesics behave more naturally than geodesics, mainly: 
\begin{itemize}
\item There is a quasigeodesic starting in any direction from any point (\ref{thm:exist-qg}).
\item The limit of quasigeodesics is a quasigeodesic (\ref{cor:lim-length:qg}).
\end{itemize}

These two property make quasigeodesics to be a useful technical tool. 
Also due to the generalized Lieberman lemma (\ref{lib-lem}), 
they are useful in the study of length-metric of extremal subsets, 
in particular the boundary of Alexandrov's space. 
There is an overlap in the applications of quasigeodesics and gradient
exponent, defined in Section~\ref{sec:gexp}.
A good example is the proof of
Theorem~\ref{thm:dist-to-bry}, see footnote~\ref{qg-grad},
page~\pageref{qg-grad}.
In this type of argument, radial curves give a simpler tool,
which is also
superior since it works in infinitely dimensional Alexandrov's spaces.

\section{Convex curves and quasigeodesics}

In this secion we define \emph{$\kappa$-convex curves} and \emph{$\kappa$-quasigeodesics}.
Roughly, a curve is called $\kappa$-convex if it satisfy the same comparison as geodesic in $\CBB{}\kappa$-space.
If in addition it is a unit-speed curve, then it is called $\kappa$-quasigeodesic.

\begin{thm}{Definition}\label{def:k-convex.curve}
Let 
$\spc{L}\in\CBB m\kappa$
and $\beta\:\II\to \spc{L}$ be a curve.
We say that $\beta$ is \emph{$\kappa$-convex}\index{convex!$\kappa$-convex curve} if for any $p\in \spc{L}$ the function
\[y_p(t)=\md\kappa\dist[{{}}]{p}{\beta(t)}{}\] 
satisfies differential inequality $y_p''+\kappa\cdot  y_p\le 1$ for all $t\in\II$ such that $\dist{p}{\beta(t)}{}\z<\varpi\kappa$.

If in addition $\beta$ is a unit-speed curve  then it is called \emph{$\kappa$-quasigeodesic}\index{quasigeodesic!$\kappa$-quasigeodesic}.
\end{thm}


In Section~\ref{sec:qg-inv.def}, 
we give an invariant definition of quasigeodesics; 
once equivalence of these definitions is proved, the parameter $\kappa$ in the definition of $\kappa$-quasigeodesic can be omited. 
In Theorem~\ref{thm:defs.of.k-convex.curve}, 
we will give a number of reformulations of the above definition in terms of angle comparison and development.

Let us list some examples of $\kappa$-convex curves and $\kappa$-quasigeodesics:
\begin{itemize}
\item From function comparison \ref{thm:conc}, any local geodesic in $\CBB{m}{\kappa}$-space is a $\kappa$-\nospace quasigeodesic as well as $\kappa$-convex curve.
\item If $\diam \spc{L}\le \tfrac{\varpi\kappa}{2}$ then any constant curve in $\spc{L}$ is $\kappa$-convex.
In particular, if $\kappa\le0$ then any constant curve is $\kappa$-convex.
\item 
Let $\spc{L}$ be isometric to a convex figure in a Euclidean plane.
Then any billiard trajectory in  $\spc{L}$ as well as arc of its boundary equipped with a unit-speed parametrization forms a quasigeodesic in $\spc{L}$.
\item A joint of two rays in a 2-dimensional cone is a quasigeodesic if they cut form the cone two angles with angle measure $\le\pi$. 
\end{itemize}

\medskip

Now let us turn to differentiability of $\kappa$-convex curves, see Definition~\ref{def:diff-curv}.
Let us remind that two vectors $v$ and $w$ in the tangent space $\T_p$ are called \emph{opposite} (briefly $v+w=0$)
if $|v|=|w|$ and $\mangle(v,w)=\pi$, see Definition~\ref{def:opp+Lin}.
To formulate the next theorem, we need to define the following weak analog of opposite vectors. 

\begin{thm}{Definition}\label{def:polar}
Let $\spc{L}$ be a complete length $\Alex{}$ space, 
$p\in \spc{L}$ 
and $v,w\in \T_p$.
We say that $v$ is \emph{polar}\index{polar} to $w$ if 
\[\<x,v\>+\<x,w\>\ge0\] 
for any vector $x\in\T_p$.
\end{thm}

\begin{thm}{Theorem}\label{thm:k-conc=lip} 
Let $\spc{L}\in\CBB m\kappa$ 
and $\beta\:\II\to \spc{L}$ be a $\kappa$-convex curve,
then 

\begin{subthm}{thm:k-conc=lip-exist} $\beta$ is \emph{both-side differentiable}\index{both-side differentiable}; in particular, $\beta^+(t_0)$ and $\beta^-(t_0)$ are defined for any interior value $t_0$ of $\II$, 
\end{subthm}

\begin{subthm}{thm:k-conc=lip-diff}$\beta$ is differentiable at all but countable subset of values in $\II$; that is, for some countable set $A\subset \II$, we have $\beta^+(t_0)+\beta^-(t_0)=0$ 
for any $t\in \II\backslash A$.
\end{subthm}


\begin{subthm}{thm:k-conc=lip-polar} For each interior value $t_0$ of $\II$,
vector $\beta^+(t_0)$ is polar to $\beta^-(t_0)$. 
\end{subthm}

\begin{subthm}{thm:k-conc=lip-lip} $\beta$ is a $1$-Lipschitz curve.
\subitem In particular, $|\beta^\pm(t_0)|\le 1$.
\end{subthm}
\end{thm}

\parit{Proof.}
Set $p=\beta(t_0)$ and $y_q(t)=\md\kappa(\dist{q}{\beta(t)}{})$ for $q\in \spc{L}$.

\parit{(\ref{SHORT.thm:k-conc=lip-lip}).} If $t_0$ is an interior value of $\II$,
from the barrier inequality (\ref{barrier}), for any $|t-t_0|<\varpi\kappa$ we have 
$y_p(t)\le \md\kappa(t-t_0)$ or, equivalently $\dist{\beta(t_0)}{\beta(t)}{}\le |t-t_0|$.

\parit{(\ref{SHORT.thm:k-conc=lip-exist})+(\ref{SHORT.thm:k-conc=lip-diff}).} Since 
the function $y_q$ is semiconcave, it is both-side differentiable at any $t$ such that $0<\dist{q}{\beta(t)}{}<\varpi\kappa$.
Moreover, it is differentable at all but countable subset of such values $t$.
Clearly, the same remains true for the function $\dist{q}{}{}\circ\beta$.

It remains to apply \ref{lem:count-der} for a dense countable subset $Q$ of $\oBall(p,\varpi\kappa)$; the existance of such $Q$ follows from \ref{cor:dim>proper}.

\parit{(\ref{SHORT.thm:k-conc=lip-polar}).} It is sufficient to prove that for any geodesic direction $\xi$ we have
\[\<\xi,\beta^+\>+\<\xi,\beta^+\>\ge 0.\eqlbl{eq:geod-polar}\]

Choose $q\in \Str(p)$ so that $\xi=\dir p q$ is uniquely defined.
From \ref{thm:d_q dist_p(v)=-<dri p q, v>}, we have $(\d_p\dist{q}{}{})(x)\z=-\<\xi,x\>$.
Since $y_q$ is semiconcave, $y_q^+(t_0)+y_q^-(t_0)\le 0$.
Further,
\[y^\pm_q(t_0)=-\sn\kappa\dist[{{}}]{p}{q}{}\cdot\<\xi,\beta^\pm(t_0)\>\] 
and \ref{eq:geod-polar} follows.\qeds

\parbf{Arc-point model angle.}
Let us introduce a new shortcut notation for model angle.
Assume $\alpha\:\II\to \spc{X}$ is a $1$-Lipschitz curve $t_0, t\in \II$
and $p$ is a point in a metric space $\spc{X}$.
Consider a triangle in $\Lob{2}{\kappa}$ with sides 
$|t-t_0|$, $\dist{q}{\alpha(t_0)}{}$ and $\dist{q}{\alpha(t)}{}$
and denote by $\tangle\mc\kappa(\alpha|_{t_0}^{t},p)$\index{$\tangle\mc\kappa$!$\tangle\mc\kappa({*}\vert_{*}^{*},{*})$} the angle at the vertex corresponding to $\alpha(t_0)$.
It will be called \emph{arc-point model angle}\index{arc-point model angle}\index{model angle!arc-point model angle} for arc $\alpha|_{t_0}^{t}$ and point $p$.

In other words,
\[\tangle\mc\kappa(\alpha|_{t_0}^{t},p)
\df
\tangle\mc\kappa\{\dist{p}{\alpha(t)}{};\dist{p}{\alpha(t_0)}{},|t-t_0|\},\] 
where $\tangle\mc\kappa\{a;b,c\}$ is defined in Section~\ref{model}.
Note that 
\begin{itemize}
\item If the angle $\tangle\mc\kappa(\alpha|_{t_0}^{t_1},p)$ is defined then $\tangle\mc\kappa(\alpha|_{t_0}^{t},p)$ is defined for any $t$ between $t_0$ and $t_1$.
\item If one fixes $p$ and $t_0$ such that $\dist{p}{\gamma(t_0)}{}<\varpi\kappa$, then the angle $\tangle\mc\kappa(\alpha|_{t_0}^{t},p)$ is defined once $|t-t_0|>0$ is sufficiently small.
\item If $\phi=\lim_{t\to t_0+}\tangle\mc\kappa(\alpha|_{t_0}^{t},p)$ is defined then
$$\cos\phi=-(\dist{p}{}{}\circ\alpha)^+(t_0).$$
\end{itemize}


\begin{thm}{Theorem}\label{thm:defs.of.k-convex.curve}
Let $\spc{L}\in \CBB m \kappa$ 
and
$\beta\:\II\to \spc{L}$ be a $1$-Lipschitz curve.
Then the following conditions are equivalent:

\begin{subthm}{k-convex-main} $\beta$ is $\kappa$-convex. 
\end{subthm}


\begin{subthm}{k-convex-mono} For any $t_0\in\II$ and $p\not=\beta(t_0)$, 
the subfunction
\[\tau \mapsto\tangle\mc\kappa(\beta|_{t_0}^{t_0+\tau},p)\]
is non-increasing for $\tau>0$.
\end{subthm}

\begin{subthm}{k-convex-angle} $\beta$ is both-side differentiable, and vector $\beta^+(t_0)$ is polar to $\beta^-(t_0)$ at each interior value $t_0\in \II$.
Moreover, for any $t_0\in\II$ and $p\not=\beta(t_0)$, inequalities
\[\<\dir{\beta(t_0)}p,\beta^\pm(t_0)\>
\le 
\cos\tangle\mc\kappa(\beta|_{t_0}^{t_0\pm\tau}, p)\] 
hold for $\tau>0$; once the right hand side is defined.
\end{subthm}

\begin{subthm}{k-convex-devel} For any subinterval $[a,b]\subset\II$ and point $p\in \spc{L}$,
if the $\kappa$-developing of the restriction $\beta|[a,b]$
with respect to $p$ is defined then it is locally convex.
\end{subthm}

Moreover, if in addition $\beta$ is a unit-speed curve then each of the conditions  (\ref{SHORT.k-convex-main})--(\ref{SHORT.k-convex-devel}) imply that $\beta$ is a $\kappa$-quasigeodesic.
In this case the inequality in (\ref{SHORT.k-convex-angle}) can be also rewritten the following way:
\begin{itemize}
 \item[\textit{\ref{SHORT.k-convex-angle}$'$)}] $\mangle(\dir{\beta(t_0)}p,\beta^\pm(t_0))
\ge \tangle\mc\kappa(\beta|_{t_0}^{t_0\pm\tau}, p).$
\end{itemize}



\end{thm}

\parbf{Remark.}
The inequality (\ref{SHORT.k-convex-angle}$'$) states that in a ``triangle'', one side of which is a quasigeodesic and two other sides are geodesics, one has usual comparison for the angle adjacent to the quasigeodesic.
Note however, that it gives no controle on the angle between geodesic sides in this triangle.

\parit{Proof.} Equivalence {\it (\ref{SHORT.k-convex-main})$\Leftrightarrow$(\ref{SHORT.k-convex-mono})} follows from Alexandrov's lemma (\ref{lem:alex}) and description of solution $y''+\kappa\cdot  y\le 1$ using Jensen inequality (\ref{y''-mono}).

The equivalence {\it(\ref{SHORT.k-convex-main})$\Leftrightarrow$(\ref{SHORT.k-convex-angle})} is essentially description of solution $y''+\kappa\cdot  y\le 1$ trough barrier inequality (\ref{barrier}).

The implementations {\it(\ref{SHORT.k-convex-angle})$\Rightarrow$(\ref{SHORT.k-convex-devel})$\Rightarrow$(\ref{SHORT.k-convex-mono})} can be proved the same way as Development comparison (\ref{thm:devel}).

The last statement will follow from \ref{cor:|gamma'|=1}.%???REF TO SOMETHING LATER IN THE CHAPTER???
\qeds













\section{Passage to the limit}

The following proposition follows directly from the definition of $\kappa$-convex curve (\ref{def:k-convex.curve}) and Theorem~\ref{thm:k-conc=lip-lip}.

\begin{thm}{Proposition}\label{prop:lim-length}
Let 
$\spc{L}_n\in\CBB m{\kappa_n}$, 
$\spc{L}_n\GHto \spc{L}_\infty$
and $\kappa_n\to\kappa_\infty\in\RR$.

Assume $\beta_n\:\II\to \spc{L}_n$ be a sequence of $\kappa_n$-convex curves
and $\beta_n(t)\to \beta_\infty(t)$ for all $t\in\II$.
Then $\beta_\infty$ is a $\kappa_\infty$-convex curve in $\spc{L}_\infty$.
\end{thm}


Here is the main theorem in this section.

\begin{thm}{Theorem}\label{thm:lim-length}
Let 
$\spc{L}_n\in\CBB m{\kappa_n}$, 
$\spc{L}_n\GHto \spc{L}_\infty$ be noncollapsing (that is, $\dim\spc{L}_\infty=m$)
and $\kappa_n\to\kappa_\infty\in\RR$.

Assume $\beta_n\:\II\to \spc{L}_n$ be a sequence of $\kappa_n$-convex curves
and $\beta_n(t)\to \beta_\infty(t)$ for all $t\in\II$.
Then if $\beta_\infty\:\II\to\spc{L}_\infty$ is differentiable at an interior value $t_0\in\II$
then 
\[|\beta^+(t_0)|=|\beta^-(t_0)|=\lim_{n\to\infty} |\beta_n^\pm(t_0)|.\]

\end{thm}


Before giving a proof, let us give couple of corollaries. 

\begin{thm}{Corollaries}\label{cor:lim-length}
Let 
$\spc{L}_n\in\CBB m{\kappa_n}$, 
$\spc{L}_n\GHto \spc{L}_\infty$  be noncollapsing (that is, $\dim\spc{L}_\infty=m$)
and $\kappa_n\to\kappa_\infty\in\RR$.

Assume $\beta_n\:\II\to \spc{L}_n$ be a sequence of $\kappa_n$-convex curves
and $\beta_n(t)\to \beta_\infty(t)$ for all $t\in\II$.
Then
\begin{subthm}{}
\[|\beta_\infty^\pm(t)|
=
\lim_{n\to\infty}|\beta_n^\pm(t)|\]
for all but countable set of values $t\in\II$.
\end{subthm}

\begin{subthm}{}
$\length\beta_n\to\length\beta_\infty$ as $n\to\infty$.
\end{subthm}

\begin{subthm}{cor:lim-length:qg}
If in addition $\beta_n\:\II\to \spc{L}_n$ be a sequence of $\kappa_n$-quasigeodesics
then $\beta_\infty\:\II\to \spc{L}$ is a $\kappa_\infty$-quasigeodesic.
\end{subthm}
\end{thm}

\parit{Proof.}
Follows from Theorem~\ref{thm:lim-length},
Proposition \ref{prop:lim-length} 
and Theorem~\ref{thm:k-conc=lip-diff}.
\qeds

\begin{thm}{Corollary}\label{cor:one-side-cont}\label{cor:|gamma'|=1}
Let  
$\spc{L}\in \CBB m \kappa$
and $\beta\:\II\to \spc{L}$ be a $\kappa$-convex curve.
Then for any $t_0\in \II$, we have
\begin{align*}
|\beta^+(t_0)| &= \lim_{t\to t_0+}|\beta^\pm(t)|,\ \ &|\beta^-(t_0)| &= \lim_{t\to t_0-}|\beta^\pm(t)|.
\end{align*}

In particular, if $\beta\:\II\to \spc{L}$ is a $\kappa$-quasigeodesic then $|\beta^\pm(t)|\equiv1$ for all $t\in \II$ 
with the exception for the ends of $\II$, 
where one of derivatives is not defined.
\end{thm}

\parit{Proof.}
We will prove the first identity.
Without loss of generality we can assume $t_0=0$ and $\II=[0,1]$.
Set $p=\beta(0)$.

From \ref{thm:k-conc=lip-lip}, $|\beta^\pm(t)|\le 1$ for any $t$. 
Thus, it is sufficient to show that for any sequence $t_n\to0+$ such that $|\beta^\pm(t_n)|$ converges we have 
$$|\beta^+(0)|=\lim_{n\to\infty}|\beta^\pm(t_n)|.$$

Consider sequence of blowups $\spc{L}_n=\frac{1}{t_n}\blow \spc{L}$ with sequence of 
curves 
$$\beta_n\:[0,\tfrac1{t_n}]\to \spc{L}_n\:s\z\mapsto\tfrac{1}{t_n}\blow \beta(t_n\cdot s).$$
Note that $\spc{L}_n\in\CBB{m}{(t_n)^2\cdot\kappa}$ 
and $\beta_n(s)$ is $(t_n)^2\cdot\kappa$-convex curve. 

For the canonical convergence $\spc{L}_n\GHto \T_p$ defined in \ref{thm:approx4tan},
the sequence of curves $\beta_n\:[0,\tfrac1{t_n}]\to \spc{L}_n$ converges to 
$$\beta_\infty\:[0,\infty)\to\T_p\:s\mapsto
s\cdot \beta^+(0).
$$
Note that $|\beta_\infty^\pm(1)|=|\beta^+(0)|$ and $\beta_\infty$ is differentiable at $1$.
Thus applying Theorem~\ref{thm:lim-length}, 
we get 
\[|\beta^\pm(t_n)|
=
|\beta_n^\pm(1)|
\xto[n\to\infty]{}
|\beta^\pm_\infty(1)|
=
|\beta^+(0)|.
\]

Now if $\beta$ is a $\kappa$-quasigeodesic then it is a unit-speed curve.
From \ref{vel-length}, $|\beta^\pm(t)|\ae 1$.
Applying the proved identity, we get that the identity $|\beta^\pm(t)|\equiv1$ holds everywhere.
\qeds

In the proof of the theorem we will use the followng lemma.
%The proof is left to the reader.???In fact the depends on how volume part is written, so I can not write it yet???


\begin{thm}{Lemma}\label{lem:<x,v>=<x,w> => v=w}
Let $K=\Cone\Sigma\in\CBB{m}{0}$ and $v,w\in K$.

Assume that equality 
\[\<v,x\>=\<w,x\>\]
holds for $x$ in a subset $E\subset K$ of positive volume
then $v=w$.
\end{thm}

\parit{Proof.} To be written.%???To be written???
\qeds



\parit{Proof Theorem~\ref{thm:lim-length}.}
It is sufficient to show that 
$|\beta_n^\pm(t_0)|\to |\beta_\infty^\pm(t_0)|$ as $n\to\o$
for some selective utrafilter $\o$.

Set 
\begin{enumerate}[(a)]
\item $p_\infty=\beta_\infty(t_0)$, $p_n=\beta_n(t_0)$;
\item $(K_\o,0)=\lim_{n\to\o}(\T_{p_n},0)$;
\item\label{beta->w} $\beta_n^\pm(t_0)\to w_\pm\in K_\o$ as $n\to\o$.
\end{enumerate}

We will construct a map $\map\:\T_{p_\infty}\to K_\o$ which satisfies the following properties:
\begin{enumerate}[(i)]
\item\label{map is a conic map} It is a \emph{conic map}\index{conic map}; that is, $\map(r\cdot x)=r\cdot\map(x)$ for all $r\ge 0$ and $x\in \T_{p_\infty}$.
\item\label{map is a norm-preserving} It is norm-preserving; 
that is, $|\map(x)|=|x|$ for any $x\in \T_{p_\infty}$.
\item\label{map is a non-contracting} It is non-contracting; 
that is $\dist{\map(x)}{\map(y)}{}\ge\dist{x}{y}{}$ 
for any $x,y\in \T_{p_\infty}$.
\item\label{the claim for map} for any $z\in\map(\T_{p_\infty})$, we have 
\[\<z,w_\pm\>
=\<z,\map(\beta_\infty^\pm(t_0))\>.\]
\end{enumerate}

Once $\map$ is constructed,
Lemma~\ref{lem:<x,v>=<x,w> => v=w} 
and properties (\ref{map is a non-contracting}) and (\ref{the claim for map}) imply $$\map(\beta_\infty^\pm(t_0))=w_\pm.$$
By property (\ref{map is a norm-preserving}),
\[
|w_\pm|
=
|\beta_\infty^\pm(t_0)|.\]
Further, by (\ref{beta->w}),
\[|\beta_n^\pm(t_0)|
\xto[n\to\o]{}
|\beta_\infty^\pm(t_0)|;\]
that is, once the map $\map\:\T_{p_\infty}\to K_\o$ is constructed, 
the theorem follows.

\parit{Construction of $\map$.} 
Set 
$$\Sigma_\o=\lim_{n\to\o}\Sigma_{p_n};$$ 
clearly $K_\o=\Cone\Sigma_\o$.
According to Corollary~\ref{cor:S>Sigma}, $\Sigma_{p_n}\le \SS^{m-1}$ for each $n$;
thereofre $\Sigma_\o\le \SS^{m-1}$.
In particular $\Sigma_\o$ is compact.

Choose $q_\infty\in \Str (p_\infty)$,
so the geodesic $[p_\infty q_\infty]$ is uniquely defined.
Set $\xi_\infty=\dir {p_\infty}{q_\infty}$.
Further, choose a sequence of points $q_n\in \Str(p_n)$ so that $q_n\GHto q_\infty$ and set $\xi_n=\dir{p_n}{q_n}$.
Finally set
\[\map (\xi_\infty)=\lim_{n\to\o}\xi_n.\]

The value $\map(\xi_\infty)\in\Sigma_\o$ depends on choice of point $q_\infty$ and the choice of sequence $(q_n)$.
But we can fix such choices for each direction $\xi_\infty\in\Sigma'_p$;
recal that $\Sigma'_p$ denotes the set of geodesic directions in $\Sigma_p$.

Since $[p_\infty q_\infty]$ is unique, we have $[p_n q_n]\to[p_\infty q_\infty]$.
From angle semicontinuity (\ref{lem:ang.semicont-cbb}), 
the map $\map\:\Sigma'_{p_\infty}\to\Sigma_\o$ is non-contracting.

For a non-geodesic direction $\zeta\in\Sigma_{p_\infty}$, 
fix a sequence of geodesic directions $\zeta_n\to\zeta$ and set $\map(\zeta)$ to be a partial limit of $\map(\zeta_n)$ as $n\to\infty$.
The pratial limit exists since $\Sigma_\o$ is compact.
This way we extended the map $\map$ to $\Sigma_{p_\infty}$ keeping it noncontracting.

Finally, let us extend $\map$ to a conic map $\map\:\T_{p_\infty}\to K_\o$ by setting
$$\map(r\cdot\zeta)=r\cdot\map(\zeta)$$ 
for all $r\ge0$ and $\zeta\in\Sigma_{p_\infty}$.

\medskip

Now let us show that the constructed map $\map$ satisfies the conditions.
The conditions (\ref{map is a conic map})--(\ref{map is a non-contracting}) follow directly from the construction.
To prove (\ref{the claim for map}), 
it is sufficient to show that equality
\[\<\map(\xi_\infty),w_\pm\>=\<\map(\xi_\infty),\map(\beta_\infty^\pm(t_0))\>\]
holds for any geodesic directions $\xi_\infty$.

Let $q_\infty$, $(q_n)$ and $(\xi_n)$ be as in the construction of $\map$ above.
Set
\begin{align*}
f_\infty(t)&=\dist{q_\infty}{\beta_\infty(t)}{},
& f_n(t)&=\dist{q_n}{\beta_n(t)}{}.\\
\intertext{Clearly, $f_n\to f_\infty$.
Since $q_\infty\in \Str(p_\infty)$ and $q_n\in \Str_n(p_n)$, \ref{thm:d_q dist_p(v)=-<dri p q, v>} implies}
(\d_{p_\infty}\dist{q_\infty}{}{})(x)&\equiv-\<\xi_\infty,x\>,
&(\d_{p_n}\dist{q_n}{}{})(x)&\equiv-\<{\xi_n},x\>.
\intertext{Thus, the functions $f_\infty$ is differentiable at $t_0$ and}
f_\infty'(t_0)&=\mp\<\xi_\infty,\beta_\infty^\pm(t_0)\>,
& f^\pm_n(t_0)&=-\<\xi_n,\beta^\pm_n(t_0)\>.
\end{align*}

Set $y_\infty=\md\kappa\circ f_\infty$ and $y_n=\md\kappa\circ f_n$.
Since $\beta_\infty$ is $\kappa_\infty$-convex and $\beta_n$ are $\kappa_n$-convex,
we have
\begin{align*}
y_\infty''+\kappa_\infty\cdot  y_\infty&\le 1
&
y_n''+\kappa_n\cdot  y_n&\le 1
\end{align*}
Therefore, according to the lemma on equilibrium (\ref{lem:der-conv-lim}), $y_n^\pm(t_0)\to \pm y'(t_0)$ and hence
$f_n^\pm(t_0)\to \pm f_\infty'(t_0)$ as $n\to\o$.
Summarizing, all above we get
\begin{align*}
\<\map(\xi_\infty),w_\pm\>
&=\lim_{n\to\o}\<\xi_n,\beta_n^\pm(t_0)\>
=
\\
&=-\lim_{n\to\o} f^\pm_n(t_0)
=
\\
&=\mp f_\infty'(t_0)
=
\\
&=
\<\xi_\infty,\beta_\infty^\pm(t_0)\>.
\end{align*}

Therefore it remains to show that
\[\<\map(\xi_\infty),\map(\beta_\infty^\pm(t_0))\>
=\<\xi_\infty,\beta_\infty^\pm(t_0)\>.\]
Note that since $\map$ is non-contracting and norm-preserving, we have
\[
\begin{aligned}
\mangle\l({\map(\xi_\infty)},\map(\beta_\infty^\pm(t_0))\r)
&\ge
\mangle\l({\xi_\infty},{\beta_\infty^\pm(t_0)}\r),
\\
\mangle\l({\map(\beta_\infty^+(t_0))},{\map(\beta_\infty^-(t_0))}\r)
&\ge
\mangle\l({\beta_\infty^+(t_0)},{\beta_\infty^-(t_0)}\r).
\end{aligned}
\eqlbl{eq:ang-non-contr}
\]
Since ${\beta_\infty^+(t_0)}+{\beta_\infty^-(t_0)}=0$,
\[
\mangle({\xi_\infty},{\beta_\infty^+(t_0)})+\mangle({\xi_\infty},{\beta_\infty^-(t_0)})=\pi,
\ \ \ \ 
\mangle({\beta_\infty^+(t_0)},{\beta_\infty^-(t_0)})=\pi.\]
Therefore we have equalities in both inequalites of \ref{eq:ang-non-contr}. 
Hence
\[ \<\map(\xi_\infty),\map(\beta_\infty^\pm(t_0))\>
=|\beta_\infty^\pm(t_0)|\cdot\cos\mangle({\xi_\infty},{\beta_\infty^\pm(t_0)})
=\<\xi_\infty,\beta_\infty^\pm(t_0)\>.
\]
\qedsf
















\section{Invariant description of quasigeodesics}\label{sec:qg-inv.def}

Recall that a semiconcave subfunction $f\:\spc{L}\to\RR$
on a $\Alex{}$ space $\spc{L}$  is called \emph{native}
if at each point $p\in\Dom f$
there is a vector $w\in\T_p$
such that
\[d_pf(x)+\<w,x\>\le 0\]
for any $x\in\T_p$;
see Definition \ref{def:native}.
Note, that according to \ref{with-no-bry}, if $\partial \spc{L}=\emptyset$ then any semiconcave function is native.


\begin{thm}{Theorem}\label{thm:qg+concave}
Let 
$\spc{L}\in\CBB m\kappa$ 
and $\gamma\:\II\to \spc{L}$ be a curve.
Then the following conditions are equivalent:
\begin{subthm}{qg+concave:qg} $\gamma$ is a $\kappa$-quasigeodesic;
\end{subthm}
\begin{subthm}{qg+concave:strong} for any $\lambda\in\RR$ and any native $\lambda$-concave locally Lipschitz subfunction $f\:\spc{L}\subto\RR$ the composition $f\circ\gamma$ is $\lambda$-concave.
\end{subthm}
\end{thm}

Before going ito proof, let us show that the word \emph{native} is necessury in the formulation of this theorem.

\parbf{Example.}
The unit disc $\DD$ in $\EE^2$ forms a $\CBB{2}{0}$ space.
Let $\gamma$ be an arc of boundary of $\DD$ equipped with unit-speed parametrization.
Note that $\gamma$ is a quasigeodesic in $\DD$.

On the other hand any linear function $f\:\EE^2\to\RR$ is also a concave function on $\DD$
and
it is easy to choose $f$ so that the composition $f\circ\gamma$ is not concave.

\medskip

The theorem implies that we can talk about quasigeodesics in $\Alex{}$ spaces without ambiguity; more presicely:

\begin{thm}{Corollary}\label{qg=k-qg} 
Let $\Kappa>\kappa$ 
and $\spc{L}\in \CBB m\Kappa$.
Then the classes of $\Kappa$-\nospace quasigeodesics and $\kappa$-\nospace quasigeodesics in $\spc{L}$ coincide.
\end{thm}


\parit{Proof of \ref{thm:qg+concave},
(\ref{SHORT.qg+concave:qg}) $\Leftarrow$ (\ref{SHORT.qg+concave:strong}).} 
First, let us define class of \emph{probation curves} which includes all curves satisfying condition (\ref{SHORT.qg+concave:strong}) for the functions with controlled concavity.
Thus, it will be sufficient to show that any probation curve $\gamma$ is a $\kappa$-quasigeodesic.

\begin{thm}{Definition}
A curve $\gamma$ in $\spc{L}\in\CBB m\kappa$ is called \emph{probation curve}\index{probation curve} if for any $\lambda\in \RR$ and any subfunction $f\:\spc{L}\subto\RR$ of controlled concavity type $(\lambda,\kappa)$, we have that $f\circ\gamma$ is $\lambda$-concave.
\end{thm}

First, let us briefly discuss properties of probation curves.
Note that for any $p,q\in \spc{L}$,
function $f_p=\md\kappa\circ\dist{p}{}{}$ 
has controlled concavity type $(1-\kappa\cdot  f_p(q),\kappa)$ at $q$.
Thus, any probation curve is $\kappa$-convex. 

Since lifting of functions with given controlled concavity type has the same controlled concavity type (see \ref{lem:lifting}), 
we have that limit of probation curves is a probation curve.
More precisely:

\begin{clm}{}\label{clm:lim-prob}
Let 
$\kappa_n\to\kappa$, 
$\spc{L}_n\in\CBB m{\kappa_n}$, 
$\spc{L}_n\xGHto{\GH} \spc{L}$ 
and $\spc{L}\in\CBB m{\kappa}$ (that is, $\spc{L}_n$ is not a collapsing).

Assume $\gamma_n\:\II\to \spc{L}$ is a sequence of probation curves 
and $\gamma_n(t)\xto\tau\gamma(t)$ for any $t\in\II$.
Then $\gamma$ is a probation curve.
\end{clm}

Now, let us continue the proof;
assume $\gamma\:\II\to\spc{L}$ is a probation curve.
Since $\gamma$ is $\kappa$-convex, it is $1$-Lipschitz (see \ref{thm:k-conc=lip}).
Therefore, according to \ref{lem:count-der}, it is sufficient to show that
\[\gamma^+(t_0)+\gamma^-(t_0)=0
\ \ 
\Rightarrow
\ \ 
|\gamma^\pm(t_0)|\ge1.
\eqlbl{eq:|gamma+|>=1}\]
for any $t_0\in\II$.

Without loss of generality, we can assume that $t_0=0$.
Set $p=\gamma(0)$.
Consider curves $\gamma_n(t)\df n*\gamma(t/n)$ in the blowup $n*\spc{L}$. 
For the canonical convergence $n*\spc{L}\xGHto{\tau_p} \T_p$ defined in \ref{thm:approx4tan}, 
we have
\[\gamma_n(t)\xto{\tau_p} \gamma_\infty(t)=\pm t\cdot\gamma^\pm(0).\]
Since $\gamma^+(0)+\gamma^-(0)=0$ we can multiply $\gamma^\pm(0)$ by all real numbers --- thus the last expression has sence.
From Claim~\ref{clm:lim-prob},
we get that $\gamma_\infty$ is a probation curve in $\T_p$.

Take $f\:\T_p\to\RR$ as in Lemma~\ref{lem:exist-CCT}.
Then $f\circ\gamma_\infty(s)$ is $(-1)$-concave.
On the other hand $\gamma^\pm(0)\in\Lin_p$, 
therefore 
\begin{align*}
f\circ\gamma_\infty(t)
&=
f\l(t\cdot\gamma^+(0)\r)
=
\\
&=
-\tfrac{1}{2}{|\gamma^\pm(0)|^2}\cdot t^2+o(t^2).
\end{align*}
Thus, \ref{eq:|gamma+|>=1} follows.

\parit{(\ref{SHORT.qg+concave:qg}) $\Rightarrow$ (\ref{SHORT.qg+concave:strong}).} 
We rewrite the proof given in \cite[6.1]{perelman-petrunin:qg},
it is based on the same idea as \ref{thm:cont=>lip} and \ref{thm:dist-to-bry}.

In the proof we will use so called \emph{Hamilton--Jacobi shifts}\index{Hamilton--Jacobi shifts} of $f$.
It is a function analog of equidistant for hypersurface.
It is named this way since the family of functions $f_s$ can be also regarded as a generalized solution for the Hamilton--Jacobi equations 
$\tfrac{\partial f_s}{\partial s}+\tfrac{1}{2}\cdot|\nabla f_s|^2=0$.

\begin{clm}{Hamilton--Jacobi shifts.}
Let $\spc{L}$ be a metric space 
and $f\:\spc{L}\subto\RR$ be a subfunction.
A Hamilton--Jacobi shift of $f$ for time $s>0$ is a subfunction $f_s\:\spc{L}\subto\RR$ defined as
\[f_s(y)
\df
\min_{x\in\Dom f} f(x)+\tfrac{1}{2\cdot s}\cdot\dist[2]{x}{y}{};\]
with $\Dom f_s$ to be the maximal open set such that the minimum is defined.
\end{clm}

Let us list its basic properties.
Note first that $f_{s_1+s_2}=(f_{s_1})_{s_2}$; that is, using a more fancy language, $f\mapsto f_s$ is an \emph{action of semigroup} $(\RR_{\ge0},+)$.

Assume that $f$ is $\Lip$-Lipschitz.
Then for any $x\in\Dom f$, we have $f_s(x)$ is defined for  all sufficiently small $s>0$ and moreover, 
\[0\le f(x)-f_s(x)\le \tfrac{1}{2}\cdot\Lip^2\cdot s.\]
In particular, $f_s(x)\to f(x)$ as $s\to0+$.

By the definition, for any $y\in\Dom f_s$ 
there is \emph{foot point} $x\in\Dom f$ such that 
\[f_s(y)
=f(x)+\tfrac{1}{2\cdot s}\cdot\dist[2]{x}{y}{};\]
that is,
the function $f+\tfrac{1}{2\cdot s}\cdot\dist[2]{y}{}{}$ has minimum at $x$.
Clearly 
\[\dist{x}{y}{}\le \Lip\cdot s.
\eqlbl{eq:lamb-conc-4}\]

If we assume in addition that $\spc{L}$ is an $m$-dimensional complete length $\Alex{}$ space, 
then $f_s$ is defined as a minimum of semiconcave functions. 
Thus, $f_s$  is semiconcave for all $s>0$.
If in addition 
if $f$ is native, 
then subfunction $f+\tfrac{1}{2\cdot s}\cdot\dist[2]{x}{}{}$ is native, see \ref{thm:native-operations}.
Thus, if $x$ is a foot point of $y\in\Dom f_s$
then, from the existence of supporting vector, we have 
$\d_y f+\tfrac{1}{2\cdot s}\cdot\d_y\dist[2]{x}{}{}=0$.
It follows that the geodesic $[x y]$ is uniquely defined,
$\ddir x y\in \Lin_y$ and 
$(\d_x f)(v)
=
\tfrac{1}{s}\cdot\<\ddir x y,v\>$ for any $v\in\T_x$.

\smallskip

Now, we ready for the proof.
First note that without loss of generality, we can assume that $f$ is $\Lip$-Lipschitz for some $\Lip\in\RR$;
otherwise restrict $f$ to a smaller domain.
It is sufficient to show that for any interior value $t_0 \in \II$, $s>0$ and all sufficiently small $\tau$, we have
\[f_s\circ\gamma(t_0+\tau)
\le
f_s\circ\gamma(t_0)
+ \Const\cdot\tau+\tfrac{1}{2}\cdot\l[\lambda-O(s)\r]\cdot\tau^2.
\eqlbl{eq:lamb-conc-1}\]

\begin{wrapfigure}{r}{35mm}
\begin{lpic}[t(-7mm),b(0mm),r(0mm),l(0mm)]{pics/foot-point(0.37)}
\lbl[l]{5,90;$\gamma$}
\lbl[rt]{11,27;$y$}
\lbl[rt]{5,54;$y_\tau$}
\lbl[t]{90,24;$x$}
\lbl[b]{68,56;$x_\tau$}
\lbl[lb]{67,86;$\xi_\tau$}
\lbl[lb]{17,34;{\small $\alpha$}}
\lbl[r]{74,30;{\tiny $\beta_\tau$}}
\end{lpic}
\end{wrapfigure}

Set $y=\gamma(t_0)$ and $y_\tau=\gamma(t_0+\tau)$.
Let $x$ be a foot point of $y$; 
that is, the function $f + \tfrac{1}{2\cdot s}\cdot\dist[2]{y}{}{}$ has a minimum at $x$
(we can assume $x\not=y$, otherwise \ref{eq:lamb-conc-1} follows since $f_s\le f$, $\d_x f\equiv0$ and $\dist[2]{y}{y_\tau}{}\le\tau$).

Set $\alpha=\mangle(\ddir y x,\gamma^+(0))$ and $\beta_\tau=\mangle\hinge x y {y_\tau}$.
Since $\T_p$ splits in the direction of $\ddir x y$,
there is direction $\xi_\tau\in\Sigma_x$ such that 
\[\mangle(\xi_\tau,\dir x y)
=\mangle(\xi_\tau,\dir x {y_\tau})+\beta_\tau
=\pi-\alpha.\]
Define $x_\tau=\gexp\mc0_x(\tau\cdot\xi_\tau)$.

First note that 
\[\dist[2]{x_\tau}{y_\tau}{}
\le \dist[2]{x}{y}{} +[1+O(s^2)]\cdot\tau^2.
\eqlbl{eq:lamb-conc-7}\]
Indeed, for any $p\in \spc{L}$, the function $\tfrac{1}{2}\cdot\dist[2]{p}{}{}$ is $\l[1+O(r^2)\r]$-concave in $\oBall(p,r)$.
Thus, from \ref{eq:lamb-conc-4} and \ref{thm:prop-gexp:func}, for all small $\tau$,  we have
\[\begin{aligned}
\dist[2]{x_\tau}{y_\tau}{}
&\le \dist[2]{x}{y_\tau}{}+2\cdot\tau\cdot\dist[{{}}]{x}{y_\tau}{}\cdot\cos(\alpha+\beta_\tau)+\l[1+O(s^2)\r]\cdot\tau^2,
\\
\dist[2]{x}{y_\tau}{}
&\le \dist[2]{x}{y}{}-2\cdot\tau\cdot\dist[{{}}]{x}{y}{}\cdot\cos\alpha+\l[1+O(s^2)\r]\cdot\tau^2
\end{aligned}
\eqlbl{eq:lamb-conc-5}\]
Note also that $\dist{x}{y_\tau}{}
=\dist{x}{y}{}-\tau\cdot\cos\alpha+o(\tau)$ 
and $\dist{y}{y_\tau}{}=\tau+o(\tau)$;
therefore 
\[\beta_\tau
>
\frac{\tau\cdot\sin\alpha}{\sn\kappa\dist[{{}}]{x}{y}{}}+o(t).
\eqlbl{eq:lamb-conc-6}\]
From \ref{eq:lamb-conc-5} and \ref{eq:lamb-conc-6}, we get 
\ref{eq:lamb-conc-7}.

Further, from \ref{thm:prop-gexp:func},
\[\begin{aligned}
f(x_\tau)
&\le
f(x)+(\d_x f)(\xi_\tau)\cdot \tau+\tfrac\lambda2\cdot\tau^2
=
\\
&=
f(x)-\tfrac{\dist{x}{y}{}}{s}\cdot(\cos\alpha)\cdot\tau +\tfrac\lambda2\cdot\tau^2.
\end{aligned}
\eqlbl{eq:lamb-conc-3}\]
Since,
\[f_s(y_\tau)
\le 
f(x_\tau)+\tfrac{1}{2\cdot s}\cdot\dist[2]{x_\tau}{y_\tau}{},\]
\ref{eq:lamb-conc-4}, 
\ref{eq:lamb-conc-7} 
and \ref{eq:lamb-conc-3} 
imply \ref{eq:lamb-conc-1}.\qeds









\section{Pre-quasigeodesics}
\label{sec:pqg}

Pre-quasigeodesic is an  intermediate class of curves --- it only used to construct quasigeodesics.
It was introduced by Perelman, 
he first proved existence of pre-quasigeodesics 
and only few year later existence of quasigeodesics.

The main result of this section is existence of pre-quasigeodesics --- theorem \ref{exist-pre-qg}.
We also prove a number of properties of pre-quasigeodesics which will be needed the construction of quasigeodesics in the next section.

\begin{thm}{Definition}\label{def:pqg}
Let  $\spc{L}\in\CBB m \kappa$.
A $\kappa$-convex curve $\gamma\:\II \to \spc{L}$ is called a \emph{pre-quasigeodesic}\index{pre-quasigeodesic} if for
any $a\in \II$ such that ${|\gamma^+(a)|}>0$, the curve $\gamma^a$ defined by
\[\gamma^a(t)
\df
\gamma\left(a+\tfrac{t}{|\gamma^+(a)|}\right)\]
is $\kappa$-convex for $t\ge0$, and if ${|\gamma^+(a)|}=0$ then $\gamma(t)=\gamma(a)$ for
all $t\ge a$.
\end{thm}

Formally speaking, this definition of depends on $\kappa$, 
but, in this and the next section we fix some $\kappa\le 0$ such that $\spc{L}\in\CBB{}\kappa$, thus it will be no ambiguity.


\begin{thm}{Existence of pre-quasigeodesics}\label{exist-pre-qg}
Let $\kappa\le 0$, 
$\spc{L}\in\CBB m \kappa$, 
$x\in \spc{L}$,
$v\in \T_x$ and $|v|\le 1$. 
Then there is a pre-quasigeodesic $\gamma\:[0,\infty)\to \spc{L}$ such that
 $\gamma^+(0)=v$.
\end{thm}

Pre-quasigeodesics will be constructed in two steps. 
First we construct $\kappa$-convex curves out of radial curves, which are \emph{$\kappa$-monotonic} in the sense of the following definition.
Second we construct pre-quasigeodesics out of $\kappa$-convex.

\begin{thm}{Definition}\label{def:mono}
Let $\spc{L}\in\CBB m\kappa$.
A $1$-Lipschitz curve $\alpha\:\II\to \spc{L}$ is called \emph{$\kappa$-monotonic}\index{$\kappa$-monotonic curve} with
respect to $t_0\in\II$ if for any $p\in \spc{L}$ the function $t\mapsto\tangle\mc\kappa(\alpha|_{t_0}^t,p)$
is non-increasing for $t>t_0$ in the domain of definition. 
\end{thm}


Here is a construction which makes possible to join $\kappa$-monotonic curves as well as $\kappa$-convex curves. 








\begin{thm}{Extention procedure I}\label{ext-mono}
Let $\kappa\le 0$, 
$\spc{L}\in\CBB m \kappa$ 
and
$\alpha_1\:[a,\infty)\to \spc{L}$,  $\alpha_2\:[b,\infty)\to \spc{L}$ be two
$\kappa$-monotonic curves with respect to $a$ and $b$ respectively. 
Assume 
\[a\le b,\ \ \alpha_1(b)=\alpha_2(b)\ \ \text{and}\ \
\alpha^+_1(b)=\alpha^+_2(b).\] Then its joint
\[\beta\:[a,\infty)\to \spc{L},\ \ \beta(t)=\l[
\begin{matrix}
\alpha_1(t)&\text{if}&t< b\\
\alpha_2(t)&\text{if}&t\ge b
\end{matrix}\right.\]
is $\kappa$-monotonic with respect to $a$ and $b$.

Moreover, if in addition, $\alpha_1$ and $\alpha_2$ are $\kappa$-convex curves then so is $\beta$.
\end{thm}

\parit{Proof.}
It is sufficient to show that for any $p\in \spc{L}$, 
\[\t{the function}\ \ t\mapsto\tangle\mc\kappa(\beta|^t_a,p)\ \ \t{is non-increasing for}\ \  t\ge b.\eqlbl{eq:ext-mono-1}\]
\begin{wrapfigure}[9]{l}{46mm}
\begin{lpic}[t(3mm),b(0mm),r(0mm),l(0mm)]{pics/mono-alex(0.35)}
\lbl[lt]{98,2;$p$}
\lbl[rb]{26,35;$q$}
\lbl[tl]{-1,1;$\alpha_1(a)$}
\lbl[b]{69,47;$\alpha_2(t_1)$}
\lbl[bl]{105,27;$\alpha_2(t_2)$}
\end{lpic}
\end{wrapfigure}
Set $q=\beta(b)=\alpha_1(b)=\alpha_2(b)$.
Applying Alexandrov's lemma twice, 
the statement \ref{eq:ext-mono-1} can be reduced to the following two inequalities for $b<t_1<t_2$:
\[\tangle\mc\kappa(\alpha_2|_{t_1}^b,p)
+
\tangle\mc\kappa(\alpha_2|_{t_1}^{t_2},p)
\le\pi,\eqlbl{eq:ext-mono-2}
\]
\[
\tangle\mc\kappa(\alpha_1|_b^a,p)
+
\tangle\mc\kappa(\alpha_2|_b^{t_1},p)
\le\pi. \eqlbl{eq:ext-mono-3}\]

Inequality \ref{eq:ext-mono-2} follows directly from $\kappa$-monotonicity of $\alpha_2$ ---
again Alexandrov's lemma.

To prove \ref{eq:ext-mono-3}, set
$\phi=\sup\set{\tangle\mc\kappa(\alpha_2|_b^t,p)}{t>b}$.
Since $\alpha_2$ is $\kappa$-monotonic w.r.t. $b$, 
we have $\cos\phi=-(\d_q\dist{p}{}{})(v)$,
where $v=\alpha_1^+(b)=\alpha_2^+(b)$.
In particular,
\[\phi=\lim_{t\to b+}\tangle\mc\kappa(\alpha_1|^t_b,p).\]
Yet applying Alexandrov's lemma,
from $\kappa$-monotonicity of $\alpha_1$ w.r.t. $a$, we get
\[\tangle\mc\kappa(\alpha_1|^a_b,p)+\phi\le \pi\]
and \ref{eq:ext-mono-3} follows.

The last part follows from \ref{k-convex-mono}.\qeds

\begin{thm}{Lemma}\label{lem:mono-lim}
Let $\spc{L}\in\CBB m\kappa$.
Assume  $\alpha_n\:[0,a]\to \spc{L}$ be a sequence of $\kappa$-monotonic curves w.r.t. $0$ 
and
$\alpha_n(t)\to\alpha(t)$ for each $t\in [0,a]$.
Then $\alpha$ is $\kappa$-monotonic.

Moreover, if for some $x\in \spc{L}$  we have $\alpha_n(0)=x$ and $|\alpha^+(0)|=1$ then $\alpha_n^+(0)\to\alpha^+(0)$.
\end{thm}


\parit{Proof of \ref{lem:mono-lim}.} 
The first part follows directly from Definition~\ref{def:mono}.
Note that for any $p$, 
\[(\dist{p}{}{}\circ\alpha)^+(0)=-\cos\l(\lim_{t\to0+}\tangle\mc\kappa(\alpha|_0^t,p)\r).\]
From monotonicity, the limit is defined.
Hence $(\dist{p}{}{}\circ\alpha)^+(0)$ and, from \ref{lem:count-der:rl}, also $\alpha^+(0)$ are defined.

Set $\xi_n=\alpha^+(0)$. 
Passing to a subsequence if nesesurily, we can assume that $\xi_n\to\xi\in\Sigma_x$.
For any geodesic $[x p]$, we have 
\[\limsup_{t\to t_0+}\tangle\mc\kappa(\alpha_n|_0^t,p)
\le
\mangle(\xi_n,\dir x p).\]
Since, $\alpha_n$ is $\kappa$-monotonic, we have that
$\tangle\mc\kappa(\alpha_n|_0^t,p)\le \mangle(\xi_n,\dir x p)$
 and hence $\tangle\mc\kappa(\alpha|_0^t,p)\le \mangle(\xi,\dir x p)$,
 for any $t>0$.
In other words, 
\[\<\alpha^+(0),\dir x p\>\ge \cos\mangle(\xi,\dir x p).\eqlbl{lem:mono-lim*}\]
Note that $|\alpha^+(0)|\le 1$ --- monotonic curves are  by definition $1$-Lipschitz.
Thus, applying \ref{lem:mono-lim*} for $\dir x p\approx \xi$, 
we get $\alpha^+(0)=\xi$.
\qeds


\begin{thm}{Existence of convex curve}\label{exist-convex}
Let $\kappa\le 0$, 
$\spc{L}\in\CBB m \kappa$, 
$x\in \spc{L}$ 
and $v\in \T_x$ be such that $|v|\le 1$. 
Then there is a $\kappa$-convex curve $\beta_v\:[0,\infty)\to \spc{L}$ with
 $\beta_v^+(0)=v$.
\end{thm}

\parit{Proof.}
For $z\in \spc{L}$ and $w\in \T_z$ such that $|w|\le 1$,
consider the curve $\alpha_w\:[0,\infty)\to \spc{L}$ defined as
\[\alpha_w(t)=\gexp\mc\kappa_z t\cdot w.\]
Directly from the construction of gradient exponential map (section~\ref{sec:gexp}), 
we have $\alpha_w^+(t)$ is defined for all $t\ge 0$ and
$\alpha_w$ is $1$-Lipschitz.
According to Theorem~\ref{gexp-mono}, $\alpha_w$ is $\kappa$-monotonic. 

Fix $\eps>0$. 
Given $\xi\in\Sigma_x$, consider the following recursively
defined sequence of curves $\alpha_{v_n}(t)$ such that $v_0=\xi$ and
$v_{n}=\alpha^+_{v_{n-1}}(\eps)$.
Then joint all $\alpha_{v_n}|[0,\eps]$ into a new curve
\[\beta_{\xi,\eps}(t)=\alpha_{v_n}(t-n\cdot\eps),\ \ \t{where}\ \ n={\lfloor t/\eps\rfloor}.\]
Applying extension procedure I (\ref{ext-mono}), 
we get that $\beta_{\xi,\eps}\:[0,\infty)\to \spc{L}$ is $\kappa$-\nospace monotonic with respect to every parameter value of the form $n\eps$, $n\in\ZZ_{\ge0}$.
Passing to a partial limit, $\beta_{\xi,\eps}\to\beta_{\xi}$ as $\eps\to 0$ we get a curve $\beta_{\xi}$ which is $\kappa$-monotonic w.r.t. any value of parameter.
Thus, according to \ref{k-convex-mono}, $\beta_{\xi}$ is $\kappa$-convex and from \ref {lem:mono-lim}, $\beta_{\xi}^+(0)=\xi$.

Finally, for $v=r\cdot\xi$, set
$\beta_v(t)=\beta_\xi(r\cdot t)$.
If $|v|\le 1$, the curve $\beta_v$ is $\kappa$-convex;
 that follows directly from definition \ref{def:k-convex.curve} since $\kappa\le 0$.\qeds


\parit{Proof of \ref{exist-pre-qg}.} 
For each point $z\in \spc{L}$ and  $w\in \T_z$ such that $|w|\le 1$, choose a $\kappa$-convex
curve $\beta_w\:[0,\infty)\to \spc{L}$ with
$\beta_w^+(0)=w$;
it exists due to \ref{exist-convex}. 

Let us construct a $\kappa$-convex curve 
$\gamma_\eps \: [0,\infty)\rightarrow \spc{L}$ 
with $\gamma^+_\eps(0)=v$,
such that there is a representation of $[0,\infty)$ as  a countable union of disjoint
half-open intervals $[a_i,b_i)$, 
such that $|a_i-b_i|\le \eps$ and for
any $t\in [a_i,b_i)$ we have
\[|\gamma_\eps^+(a_i)|\ge |\gamma_\eps^+(t)| \ge
(1-\eps)|\gamma_\eps^+(a_i)|;\eqlbl{eq:exist-pre-qg-1}\]
moreover, for each $i$, the curve
\[\gamma_\eps^{a_i}(t)
=
\gamma_\eps\left(a_i+ \tfrac{t}{|\gamma_\eps^+(a_i)|}\right)\]
is also $\kappa$-convex for $t\ge 0$.

\parit{Construction of $\gamma_\eps$.}\label{transfinite-extension-page}
Assume we already can construct $\gamma_\eps$ in the interval $[0,a)$, and cannot extend it any further. 
Since $\gamma_\eps$ is $1$-Lipschitz, we can extend it continuously to
$[0,a]$.
Use \ref{cor:polar} to
construct a vector $w$ polar to $\gamma^-_\eps(a)$ with $|w|\le
|\gamma^-_\eps(a)|$ (in case $a=0$ take $\gamma(0)=x$ and $w=v$).
Joint $\gamma_\eps$ with a short half-open segment of $\beta_w$.
According to the extension procedure I (\ref{ext-mono}) the new curve is convex. 
From \ref{cor:one-side-cont} we get that the added segment satisfies \ref{eq:exist-pre-qg-1}, a contradiction.

\smallskip

Choose a sequence $\eps_n\to0+$ such that $\gamma_{\eps_n}$ converges to a curve $\gamma$.
From Corollary~\ref{cor:lim-length}???, $|\gamma^+(t)|\ae\lim|\gamma_{\eps_n}^+(t)|$.
Combining this with inequality \ref{eq:exist-pre-qg-1} and \ref{cor:one-side-cont}, we get that
$\gamma^{a}$
is $\kappa$-convex for any $a\ge0$.
Applying \ref{lem:mono-lim} for $\gamma_n^0\to \gamma^0$, we get $\gamma^+(0)=v$.
\qeds



\parbf{Entropy of pre-quasigeodesics.}
Let us  define \emph{entropy} of pre-\nospace quasigeodesic;
it measures ``how far''
a pre-quasigeodesic is from being a quasigeodesic.

\begin{thm}{Definition}\label{def:entropy}
Let $\gamma$ be a pre-quasigeodesic defined on real interval $\II$.
The \emph{entropy} of $\gamma$,  is a measure $\mu_\gamma$ defined  on the interior of $\II$
such that
\[ \mu_\gamma (a,b)=\ln |\gamma^+(a)|-\ln |\gamma^-(b)|.\]
for any open subinterval $(a,b)\subset\II$.
\end{thm}

Note that since $\gamma^a$ is $\kappa$-convex, we have that for any $b>a$, $|\gamma^+(a)|\ge|\gamma^\pm(b)|$.
In particular entropy is non-negative (that is, it is indeed a measure).
From \ref{cor:one-side-cont}, it follows that for any interior value $a\in\II$, we have
\[\mu_\gamma\{a\}=\ln|\gamma^-(a)|-\ln|\gamma^+(a)|.\]
Directly from the definition we have that if $|\gamma^+(a)|=1$ and $\mu_\gamma(a,b)=0$ then $\gamma|[a,b]$ is a quasigeodesic.


The following proposition follows directly from 
Corollary~\ref{thm:lim-length}???, 
Lemma~\ref{lem:mono-lim} 
and above discussion.


\begin{thm}{Proposition}\label{lim-entropy}
Let $\kappa\le 0$, 
$\spc{L}\in\CBB m\kappa$
and $\gamma_n\:\II\to \spc{L}$ is a sequence of pre-quasigeodesics.
Assume $\gamma_n(t)\to\gamma(t)$ for each $t\in \II$,
then $\gamma$ is a pre-quasigeodesic.
Moreover, $\mu_{\gamma_n}\rightharpoonup\mu_\gamma$; 
that is, the entropies of $\gamma_n$ weakly converge to the entropy of $\gamma$.

In particular, 
if $\mu_{\gamma_n}(a,b)\to0$ and $|\gamma_n^+(a)|=1$ then the restriction $\gamma|[a,b]$ is a quasigeodesic.
\end{thm}



The next construction  is similar to the extension procedure I (\ref{ext-mono}) and it can be proved the same way.

\begin{thm}{Extention procedure II}\label{ext-pqg}
Let $\spc{L}\in\CBB m \kappa$, $\gamma_1\:[a,\infty)\to \spc{L}$ and $\gamma_2\:[b,\infty)\to \spc{L}$ be
two pre-quasigeodesics. 
Assume 
\[a\le b,\ \ \gamma_1(b)=\gamma_2(b),\ \  \gamma^-_1(b)\ \
\text{is polar to}\ \ \gamma^+_2(b)\ \  \text{and}\ \ |\gamma^-_1(b)|\ge|\gamma^+_2(b)|\] 
then its joint
\[\gamma\:[a,\infty)\to \spc{L},\ \ \gamma(t)=\l[
\begin{matrix}
\gamma_1(t)&\text{if}&t\le b\\
\gamma_2(t)&\text{if}&t\ge b
\end{matrix}\right.\]
is a pre-quasigeodesic.
Moreover, its entropy is defined by \[\mu_\gamma|(a,b)=\mu_{\gamma_1},\ \
\mu_\gamma|(b,\infty)=\mu_{\gamma_2}\ \ \text{and}\ \
\mu_\gamma(\{b\})=\ln|\gamma^-(b)|-\ln|\gamma^+(b)|.\]

\end{thm}

\claim{Chopping procedure}\label{chopping} 
Let $\kappa\le 0$, 
$\spc{L}\in\CBB m\kappa$, 
$\gamma\:[0,\infty)\to \spc{L}$ be a pre-\nospace quasigeodesic. 
Assume for $a\ge 0$ we have $|\gamma^+(a)|>0$, 
then for any $\eps>0$ there is $b>a$ 
such that 
\[\mu_\gamma(a,b) <\eps[\mangle(\xi,\nu)+b-a],\ \ b-a<\eps,\ \ \t{and}\ \ \mangle(\xi,\nu)<\eps,\]
where 
$\xi=\tfrac{\gamma^+(a)}{|\gamma^+(a)|}$, 
$\nu=\dir x y$,
$x=\gamma(a)$ and $y=\gamma(b)$.
\endclaim\rm

\begin{wrapfigure}{r}{25mm}
\begin{lpic}[t(-15mm),b(0mm),r(0mm),l(3mm)]{pics/rotation-B(0.3)}
\lbl[l]{25,14;$x=\gamma(a)$}
\lbl[l]{25,103;$y=\gamma(b)$}
\lbl[r]{1,31;$\xi$}
\lbl[l]{24,39;$\nu$}
\end{lpic}
\end{wrapfigure}

\parit{Proof.} 
Set $s=b-a$.
Clearly, if  $s>0$ is sufficiently small, we have 
$\mangle(\xi,\nu)<\eps$.
Further, from $\kappa$-convexity of $\gamma^a$ it follows that 
\[\mu_\gamma(a,a+\tfrac{s}{3}) <\Const\cdot\l[\mangle(\xi,\nu)\r]^2.\]
The following exercise completes the proof.\qeds

\claim{Exercise}
Let $\phi,\psi\:\RR_>\to \RR_>$ be two functions such that 
\begin{subthm}{}$\phi(s)\to 0$ as $s\to0+$;
\end{subthm}

\begin{subthm}{}for any sufficiently small $s>0$ we have
\[\psi(\tfrac{s}{3})\le (\phi(s))^2.\]
\end{subthm}
Show that for any $\eps>0$ there is  $s>0$ such that
\[\psi(s)< 10 (\phi(s)+s)^2\ \ \ \t{and}\ \ \ \phi(s)+s\le\eps.\]

\endclaim\rm






\section{Existence of quasigeodesics}\label{step3-1}

Here is the main result of this section.
It was proved by Perelman around 1992.


\begin{thm}{Existence of quasigeodesics}\label{thm:exist-qg}
Let $\spc{L}\in\CBB m{}$. 
Then for any point $x\in \spc{L}$, and any direction $\xi\in \Sigma_x$
there is a quasigeodesic $\gamma\:\RR\to \spc{L}$ such that 
$\gamma^+(0)=\xi$.
\end{thm}

The prove relies heavily on the results in the previous section.

\parit{Proof.} 
Fix $\kappa\le0$ such that $\spc{L}\in\CBB m\kappa$.

Note that base case $m=1$ is trivial.
Thus, making an induction argument, we can assume that \ref{thm:exist-qg} holds in all dimensions smaller than $m$.
This assumption is used just once, to prove the following lemma;
the proof is taken from \cite{milka:poly1}.

\begin{thm}{Milka's lemma}\label{lem-milka}
\label{lem:milka}
Let $m\ge 2$, 
$\spc{L}\in\CBB m\kappa$, 
$x\in \spc{L}$ 
and $\xi\in \Sigma_x$.
Then there is a direction $\xi^*\in\Sigma_x$ polar to $\xi$; 
that is, such that
\[\<\xi,v\>+\<\xi^*,v\>\ge 0\]
for any $v\in \T_x$.
\end{thm}


\parit{Proof.} Note that $\Sigma_x\in\CBB{m-1}1$ (see \ref{thm:tan4finite}). 
Applying  existence  of quasigeodesics (\ref{thm:exist-qg}) in dimension $m-1$,
for any 
$\xi\in \Sigma_x$ there is a quasigeodesic%
\footnote{Note that according to \ref{thm:qg+concave}, it is also $1$-quasigeodesic.}
 $\gamma$ in $\Sigma_x$ with $\gamma(0)=\xi$.
Set $\xi^*=\gamma(\pi)$;
the quasigeodesic comparison (\ref{k-convex-angle}$'$) implies that
\[\dist{\xi}{\eta}{}+\dist{\eta}{\xi^*}{}
=
\mangle(\xi,\eta)+\mangle(\eta,\xi^*)\leq\pi\] 
for any $\eta\in \Sigma_x$.
Thus, $\xi$ is polar to $\xi^*$.
\qeds




Let $\Theta\subset \spc{L}$ be a maximal open set which satisfies the following property:
\textit{for any $x\in \Theta$ and any $\xi\in\Sigma_x$ there is a quasigeodesic $\gamma\:[0,r]\to \spc{L}$ with $\gamma^+(0)=\xi$ and $r>0$.}

\begin{clm}{Definition.}
Let $\Omega\subset \spc{L}$ be an open subset. 
A pre-quasigeodesic $\gamma$ in $\spc{L}$ is called \emph{$\Omega$-pre-quasigeodesic}\index{pre-quasigeodesic!$\Omega$-pre-quasigeodesic} if  its
entropy vanishes on $\Omega$; 
that is, $\mu_\gamma(\gamma^{-1}(\Omega))=0$.
\end{clm}

\begin{clm}{}\label{clm:loc-exist} For any $x\in \spc{L}$  and any $v\in \T_x$ such that $|v|\le 1$ there is an $\Theta$-pre-quasigeodesic $\gamma\:[0,\infty)\to \spc{L}$ with $\gamma^+(0)=v$.
\end{clm}

\begin{clm}{}\label{clm:suff-qg-exist} In order to prove \ref{thm:exist-qg}, it is sufficient to show that $\spc{L}=\Theta$
\end{clm}

\parit{Proof of claims \ref{clm:loc-exist} and \ref{clm:suff-qg-exist}.} 
Fix $\eps>0$ and set
$\Theta_\eps
=
\set{x\in\Theta}{\cBall[x,\eps]\subset\Theta}$.
Clearly, $\Theta_\eps$ is an open set.

Let us construct a $\Theta_\eps$-pre-quasigeodesic $\gamma_\eps\:[0,\infty)\to \spc{L}$ with $\gamma_\eps^+(0)=v$.
Assume we have a $\Theta_\eps$-pre-quasigeodesic $\gamma_\eps\:[0,a)\to \spc{L}$  with $\gamma_\eps^+(0)=\xi$.
We can extend $\gamma_\eps$ to $[0,a]$ by setting $\gamma_\eps(a)=\lim_{t\to a-}\gamma_\eps(t)$;
this limit exists since $\kappa$-convex curves are $1$-Lipschitz, see \ref{thm:k-conc=lip-lip}
(if $a=0$ set $\gamma_\eps(0)=x$).

For $\gamma_\eps\:[0,a]\to \spc{L}$, applying Milka's lemma, we can find a vector $w$ which is polar to $\gamma_\eps^-(a)$ and such that $|w|=|\gamma_\eps^-(a)|$ (if $a=0$, set $w=v$). 
Further:
\begin{itemize}
\item If $|w|=0$, set $\gamma_\eps(t)=\gamma_\eps(a)$ for all $t\ge a$. 
\end{itemize}
Otherwise:
\begin{itemize}
\item If $\gamma_\eps(a)\in\Theta$ then, from the assumption of the claim, there is a quasigeodesic $\bar\gamma\:[0,r]\to \spc{L}$ with $\bar\gamma^+(0)=w/|w|$.
Let us extend $\gamma_\eps$ behind $a$ as $\gamma_\eps(a+t)=\bar\gamma(|w|\cdot t)$.
\item If $\gamma_\eps(a)\notin\Theta$, according to \ref{exist-pre-qg},
we can find a pre-quasigeodesic $\bar\gamma\:[0,\eps]\to \spc{L}$ with $\bar\gamma^+(0)=w$.
Then extend $\gamma_\eps$ behind $a$ as $\gamma_\eps(a+t)=\bar\gamma(t)$, clearly $\gamma_\eps([a,a+\eps])\subset \spc{L}\backslash\Theta_\eps$.
\end{itemize}
According to extension procedure II (\ref{ext-pqg}), in all above cases, the extended $\gamma_\eps$ is still $\Theta_\eps$-pre-\nospace quasigeodesic.
Thus, the maximal interval of definition of $\gamma_\eps$, must be open and closed in $[0,\infty)$;
hence the desired $\gamma_\eps$ exists.

Now take a converging sequence $\gamma_{\eps_n}\to\gamma$ as $\eps_n\to0+$.
Appling Lemma~\ref{lem:mono-lim} for $\gamma_n^0\to\gamma^0$, we get $\gamma^+(0)=v$.
From \ref{lim-entropy}, $\gamma$ is $\Theta$-pre-quasigeodesic.

In case $\Theta=\spc{L}$ and $|v|=1$, the constructed pre-quasigeodesic $\gamma\:[0,\infty)\to \spc{L}$
is also a quasigeodesic.
It only remains to extend $\gamma$ to negative part of real line.
To do this we can apply open-close argument as above in a backward direction.
\claimqeds

According to claim \ref{clm:suff-qg-exist}, the following claim implies the theorem.

\begin{clm}{Key claim.}\label{clm:qg-main}
If $\Theta\not=\spc{L}$ then there is $p\in \spc{L}\backslash \Theta$ and $r>0$ such that for any $x\in\oBall(p,\tfrac{r}{6})\backslash \Theta$ and $\xi\in \Sigma_x$ there is a quasigeodesic $\gamma\:[0,\tfrac{r}{6}]\to \spc{L}$ with  $\gamma^+(0)=\xi$.
\end{clm}


First let us list the properties of $p\in \spc{L}$ and $r>0$ which are sufficient for the key claim to hold.

Set $\mathfrak C=\spc{L}\backslash \Theta$. 
The following claim is an extract from Lemma~\ref{lem:amost=Sigma}.

\begin{clm}{}\label{inq:di-inq}
Given a nonempty closed subset $\mathfrak C\subset \spc{L}$,
there is a point $p\in \mathfrak C$, 
a point array $q_1,q_2,\dots,q_\kay$ in  $\spc{L}$ 
and sufficiently small $r>0$ (say $r<\tfrac{1}{100\cdot(1+|\kappa|)}$ and $r<\tfrac{1}{100}\cdot\dist[{{}}]{p}{q_i}{}$ for each $i$)
such that 

\begin{enumerate}[a)]
\item \label{inq:di-inq:net}
For each $x\in\mathfrak C\cap \oBall(p,r)$, the set $\{\dir x {q_i}\}$ forms an $\tfrac\pi{10}$-net of $\Sigma_x$.
\item \label{inq:di-inq:main} 
For each $x\in\mathfrak C\cap \oBall(p,r)$, if
$\xi\in \Sigma_x\mathfrak C$, 
$\nu\in\Sigma_x$ 
and $\mangle(\xi,\nu)<\tfrac\pi{10}$,
then for some $i$ there is a choice of geodesic $[x q_i]$ such that
\begin{align*}
 \mangle(\xi,\nu)
&\le 
2[\mangle(\xi,\dir x {q_i})-\mangle(\nu,\dir x {q_i})],
&
\tfrac{3}{5}\cdot\pi&<\mangle(\xi,\dir x {q_i})<\tfrac{4}{5}\cdot \pi.
\end{align*}

\end{enumerate}

\end{clm}




\parit{Proof of claim \ref{inq:di-inq}.} 
Fix $\eps>0$ and choose $p\in\mathfrak C$ and $r>0$ as in Lemma~\ref{lem:amost=Sigma}.

Choose a point array $q_1,q_2,\dots,q_\kay$ such that directions $\dir p{q_i}$ form a $\eps$-net in $\Sigma_p$.
We can assume that for any $x\in\oBall(p,r)$, the values $\angk\kappa x{q_i}{q_j}$, $\angk\kappa p{q_i}{q_j}$ and $\mangle\hinge p{q_i}{q_j}$ are $\eps$-close to each-other
(otherwise move $q_i$ to $p$ along geodesics and pass to smaller $r>0$).
Since $\mangle\hinge x{q_i}{q_j}\ge\angk\kappa x{q_i}{q_j}$ 
and $\GHdist(\Sigma_x,\Sigma_p)<\eps$, 
the set of directions $\dir x {q_i}$ forms a $10\cdot\eps$-net in $\Sigma_x$; 
hence (\ref{SHORT.inq:di-inq:net}).

For any $x\in\oBall(p,r)\cap\mathfrak C$ and $\xi\in \Sigma_x\mathfrak C$ we have $\GHdist(\Sigma_\xi\T_x,\Sigma_p)<\eps$.
Thus, if $\eps>0$ is sufficiently small,  one can choose a point $q_i$ such that
\vskip-.7em
\[\tfrac{3}{5}\cdot\pi
<
\mangle(\xi,\dir x {q_i})
<\tfrac{4}{5}\cdot\pi
\ \ \t{and}\ \ 
\mangle\hinge\xi{\dir x {q_i}}{\nu}
<\tfrac\pi{10}.\]
Hence (\ref{SHORT.inq:di-inq:main}).  
\claimqeds

\begin{wrapfigure}{r}{30mm}
\begin{lpic}[t(-5mm),b(0mm),r(0mm),l(3mm)]{pics/rotation-B(0.4)}
\lbl[l]{25,14;$x_i=\gamma(a_i)$}
\lbl[l]{25,103;$y_i=\gamma(b_i)$}
\lbl[r]{1,31;$\xi_i$}
\lbl[l]{24,39;$\nu_i$}
\end{lpic}
\end{wrapfigure}

\parit{Proof of key claim \ref{clm:qg-main}.}
Given $\eps>0$, we first construct a $\Theta$-pre-quasigeodesic $\gamma_\eps\:[0,\tfrac{r}{6}]\to \spc{L}$ with $\gamma_\eps^+(0)=\xi$ and an open set $\Upsilon_\eps\subset (0,\tfrac{r}{6})$ formed by countable disjoined union of intervals $(a_i,b_i)$
such that the enthropy $\mu_{\gamma_\eps}$ vanish on $(0,\tfrac{r}{6})\backslash \Upsilon_\eps$ and for each $i$ we have

\begin{enumerate}[(i)]
\item 
$\xi_i=\tfrac{\gamma^+(a_i)}{|\gamma^+(a_i)|}
\in \Sigma_{\gamma(a_i)}\mathfrak C$;

\item\label{entr-estim} 
If $x_i=\gamma(a_i)$, $y_i=\gamma(b_i)$ and $\nu_i=\dir{x_i}{y_i}$ then $\mangle(\xi_i,\nu_i)<\eps$ and
\[\mu_\gamma(a_i,b_i)
\le \eps\cdot\l[\mangle(\xi_i,\nu_i)+b_i-a_i\r].\]
\end{enumerate}
After that we will show that $\mu_{\gamma_\eps}(0,\tfrac{r}{6})\to 0$ as $\eps\to0$.

Finally, we consider $\gamma$, a partial limit of $\gamma_\eps$ as $\eps\to0$.
From \ref{lim-entropy}, $\gamma$ is a quasigeodesic 
and according to \ref{lem:mono-lim}, $\gamma^+(0)=\xi$, which proves the claim.

\parit{The construction of $\gamma_\eps$.} \label{transfinite-extension-2-page} 
It almost repeats the construction in the proof of Claim~\ref{clm:loc-exist}.
Assume that we have constructed
$\gamma_\eps$ on an interval $[0,a]$, $a<\tfrac{r}{6}$ and cannot extend it
any further. 
Applying Milka's lemma (\ref{lem:milka}), we can find a vector $w\in\T_p$ which is polar to $\gamma^-_\eps(a)$ and such that $|w|=|\gamma^-_\eps(a)|$.
Then extend $\gamma$ by a $\Theta$-pre-quasigeodesic
with $\gamma^+(a)=w$; it exists according to claim \ref{clm:loc-exist}.
Clearly, \[\mu_{\gamma_\eps}\{a\}=\ln|\gamma^-(a)|-\ln|\gamma^+(a)|=0.\]
Further:
\begin{itemize}
\item If $\tfrac{\gamma^+(a)}{|\gamma^+(a)|}\in\Sigma_{\gamma(a_i)}\mathfrak C$, apply
chopping procedure (\ref{chopping}) starting
from $a$. In this case include the chopped interval $(a,b)$ in $\Upsilon_\eps$.

\item If $\tfrac{\gamma^+(a)}{|\gamma^+(a)|}\notin\Sigma_{\gamma(a_i)}\mathfrak C$, then  for on some $b>a$ the curve interval  $\gamma|(a,b)$ lies in $\Theta$, therefore its entropy vanish.
\end{itemize}
In both cases, 
we get a longer curve $\gamma\:[0,b]\to \spc{L}$ with the desired property, a contradiction

\parit{Vanishing of entropy:} 
From the condition~(\ref{entr-estim}) above, we have that 
\[
\mu_{\gamma_\eps}(0,\tfrac{r}{6})
<
\eps\cdot\l[\tfrac{r}{6}+\sum_i\mangle(\xi_i,\nu_i)\r].\]
Therefore, to show that $\mu_{\gamma_\eps}(0,\tfrac{r}{6})\to 0$, it only remains to show
that $\sum_i\mangle(\xi_i,\nu_i)$ is bounded above by a constant independent
of $\eps$.

Since $\xi_i\in\Sigma_{x_i}\mathfrak C$, according to claim \ref{inq:di-inq}, for each $i$ there is $q_j$ such that
\[
\mangle(\xi_i,\nu_i)
<
2\cdot\l[\mangle(\dir{x_i}{q_j},\xi_i)-\mangle(\dir{x_i}{q_j},\nu_i)\r]
\eqlbl{eq:inq:di-inq-1}\]
and moreover,
\[
\tfrac{3}{5}\cdot\pi
<\mangle(\dir{x_i}{q_j},\xi_i)
<\tfrac{4}{5}\cdot\pi
\ \ \t{and}\ \ 
\mangle(\xi_i,\nu_i)
<
\eps<\tfrac\pi{10}.\eqlbl{eq:inq:di-inq-2}\]
In particular, $\mangle(\dir{x_i}{q_j},\nu_i)>\tfrac\pi2$.
Note that, $\gamma\subset\oBall(p,\tfrac{r}3 )$ and thus, any geodesic $[x_i y_i]$ lies in $\oBall(p,r)$.
Set $\lambda=1/r$, since $r$ is small, we have
\begin{align*}
\dist{q_j}{y_i}{}-\dist{q_j}{x_i}{}
&\le-\dist[{{}}]{x_i}{y_i}{}\cdot\cos\mangle(\dir{x_i}{q_j},\nu_i)+\tfrac\lambda2\cdot\dist[2]{x_i}{y_i}{}
\le
\\
&\le
-(b_i-a_i)\cdot\cos\mangle(\dir{x_i}{q_j},\nu_i)+\tfrac\lambda2\cdot (b_i-a_i)^2.
\\
\intertext{Since $\gamma$ is $\kappa$-convex,}
\dist{q_j}{x_i}{}-\dist{q_j}{y_i}{}
&\le
-(b_i-a_i)\cdot\<\dir{y_i}{q_j},\gamma^-(b_i)\>
+\tfrac\lambda2\cdot(b_i-a_i)^2.
\end{align*}
Therefore,
\[
\cos\mangle(\dir{x_i}{q_j},\nu_i)
\le-\<\dir{y_i}{q_j},\gamma^-(b_i)\>
+\lambda\cdot(b_i-a_i).
\eqlbl{eq:inq:di-inq-3}\]
Since $\gamma^+(a_i)$ is polar to $\gamma^-(a_i)$, we have
\[\begin{aligned}
\cos\mangle(\dir{x_i}{q_j},\xi_i)
&=\tfrac{1}{|\gamma^+(a_i)|}\cdot\<\dir{x_i}{q_j},\gamma^+(a_i)\>\ge
\\
&\ge -\tfrac{1}{|\gamma^+(a_i)|}\cdot\<\dir{x_i}{q_j},\gamma^-(a_i)\>
\end{aligned}
\eqlbl{eq:inq:di-inq-4}\]
Note that $\gamma^+(a_i)\ge \tfrac{1}{2}$; it follows from $\kappa$-convexity of $\gamma$ since there is $j$ such that $\mangle(\xi,\dir x {q_j})<\tfrac\pi{10}$, see property  \ref{inq:di-inq:net}.
Set
\[
\theta_{q_j}[0,b)
=1-\<\dir{y_i}{q_j},\gamma^-(b)\>+\lambda\cdot b.\]
Clearly, $b\mapsto-\<\dir{y_i}{q_j},\gamma^-(b)\>+\lambda\cdot b$ is non-decreasing;
thus
$\theta_{q_j}$ defines a positive measure on $[0,\tfrac{r}{6})$.
Using \ref{eq:inq:di-inq-2}, \ref{eq:inq:di-inq-3} and \ref{eq:inq:di-inq-4}, we can continue \ref{eq:inq:di-inq-1}:
\begin{align*}
\mangle(\xi_i,\nu_i)
&<
2\cdot\l[\mangle(\dir{x_i}{q_j},\xi_i)-\mangle(\dir{x_i}{q_j},\nu_i)\r]
\le
\\
&\le 10\cdot\l[\cos\mangle(\dir{x_i}{q_j},\nu_i)-\cos\mangle(\dir{x_i}{q_j},\xi_i)\r]
\le
\\
&\le 20\cdot\l[\<\dir{x_i}{q_j},\gamma^-(a_i)\>-\<\dir{y_i}{q_j},\gamma^-(b_i)\>+\lambda (b_i-a_i)\r]
=
\\
&=20\cdot\theta_{q_j}[a_i,b_i)
\end{align*}
In the exceptional case $a_i=0$, although $\gamma^-(0)$ might be undefined, the last inequality holds.
Since the sistance function $\dist{q_j}{}{}$ is 1-Lipschitz, we get $\theta_{q_j}[0,\tfrac{r}{6})\le 2+\lambda r$.
Therefore,
\[\sum_i\mangle(\xi_i,\nu_i)
\le 
20\cdot\sum_{j}\theta_{q_j}[0,\tfrac{r}{6})
\le 20\cdot \kay\cdot(2+\lambda\cdot r).\]
\qedqedsf

Informally, the following lemma says that if in any closed set $\mathfrak C$ of $\Alex{}$ space, one can find a special point $p$ such that any near by point $x\in\mathfrak C$ has \emph{almost the same} tangent space, and moreover, $\T_x$ \emph{almost splits} in direction of any $\xi\in\Sigma_x \mathfrak C$.

\begin{thm}{Lemma} \label{lem:amost=Sigma}
Let $\spc{L}\in\CBB m \kappa$ and $\mathfrak C\subset \spc{L}$ be a closed subset.
Then, given $\eps>0$,
there is a point $p\in \mathfrak C$, and $r>0$ 
such that 
\begin{subthm}{}for any $x\in\mathfrak C\cap \oBall(p,r)$ we have
$\GHdist(\Sigma_x,\Sigma_p)<\eps$ and moreover,
\end{subthm}
\begin{subthm}{} for any $\xi\in\Sigma_x\mathfrak C$, $\GHdist(\Sigma_\xi\T_x,\Sigma_p)<\eps$.
\end{subthm}
\end{thm}

\parit{Proof.}
Note that as follows from volume continuity (\ref{vol-cont}),
for any $\eps>0$ there is $\delta=\delta(m,\eps)>0$ such that if $\Sigma_x+\delta\ge\Sigma_p$ and $\vol\Sigma_x-\delta\le\vol\Sigma_p$ then
$\GHdist(\Sigma_p,\Sigma_x)<\eps$.

Let us choose $p$ with $\delta$-maximal volume;
that is, such that $\vol\Sigma_x-\delta<\vol\Sigma_p$ for any $x\in \mathfrak C$.

Now, let us apply semicontinuity of space of directions (\ref{cor:simicont-Sigma}) several times:
\begin{itemize}
\item First, there is $r>0$ such that $\Sigma_x+\delta\ge\Sigma_p$ for any $x\in\oBall(p,r)$.
Hence $\GHdist(\Sigma_x,\Sigma_p)<\eps$.
\item Second, $\Sigma_\xi\T_x\ge\Sigma_x$ (here we apply \ref{cor:simicont-Sigma} for $\T_x$).
\item Third, if $\Sigma$ is a partial limit of $\Sigma_{x_i}$ for $x_i\in\mathfrak C\cap\oBall(p,r)$, then $\Sigma_\xi\T_p\le \Sigma$.
Thus, $\GHdist(\Sigma_\xi\T_x,\Sigma_p)<\eps$.\qeds
\end{itemize}














\section{Quasigeodesics in extremal sets.}\label{qg-extrim}

Here we give an extended version of Theorem~\ref{thm:exist-qg}.
The proof repeats the proof of existence of quasigeodesics (\ref{thm:exist-qg-ext});
it only require a stronger analog of Milka's lemma (\ref{lem:milka}).

\begin{thm}{Quasigeodesic in extremal set}\label{thm:exist-qg-ext}
Let $\spc{L}\in\CBB m \kappa$, 
$E\subset \spc{L}$ be an extremal set.
Then for any point $x\in E$, and any direction $\xi\in \Sigma_x E$
there is a quasigeodesic $\gamma\:\RR\to E$ such that $\gamma(0)=x$ and
$\gamma^+(0)=\xi$.
\end{thm}

\parit{Sketch of proof.} 
Note that according to ??? as a result of construction in \ref{exist-convex} we get a $\kappa$-convex curve in $E$.
Further, construction in \ref{exist-pre-qg} gives a pre-quasigeodesic in $E$.
According to ???, for any $p\in E$, the tangent space $\T_p E$ is defined and it is an extremal set of $\T_p$.
Thus if one use extended Milka's lemma \ref{lem:milka-axtemal} instead of \ref{lem:milka}, one can keep in $E$ all the constructions in the proof of \ref{thm:exist-qg}. \qeds

\begin{thm}{Extremal Milka's lemma}\label{lem:milka-axtemal}
Let $\spc{L}\in\CBB m\kappa$, 
$p\in \spc{L}$ 
and $E\subset \T_p$ be an extremal set of a tangent cone.
Then for any vector $v\in E$ there is a polar vector $v^*\in E$
such that $|v|=|v^*|$.
\end{thm}

\parit{Proof.} Set $X=E\cap \Sigma_p$. If $\Sigma_\xi X\not=\emptyset$ then the proof is the
same as for the standard Milka's lemma; it is sufficient to choose a direction in
$\Sigma_\xi X$ and shoot a quasigeodesic $\gamma$ in $X$ of length $\pi$ in this direction such that $\gamma\subset X$;
such $\gamma$ exists from the induction hypothesis. 

If $X=\{\xi\}$ then from ???, we have $\cBall[\xi,{\tfrac\pi2}]=\Sigma_p$. 
Thus, $\xi$ is polar to itself.

Otherwise, if $\Sigma_\xi X=\emptyset$ and $X$ contains at least two points, choose $\xi^*$ to be closest point in $X\backslash\xi$ from $\xi$. 
Since $X\subset \Sigma_p$ is extremal (???), for any $\eta\in \Sigma_p$ we have $\mangle\hinge{\xi^*}\eta\xi\le\tfrac\pi2$ and since $\Sigma_\xi X=\emptyset$ we have $\mangle\hinge\xi\eta{\xi^*}\le\tfrac\pi2$.
Therefore, in the model spherical triangle 
$\trig{\~\xi}{\~\eta}{\~\xi^*}
=
\modtrig1(\xi\eta\xi^*)$, 
we have $\mangle\~\xi\le \tfrac\pi2$ 
and $\mangle\~\xi^*\le \tfrac\pi2$.
Hence 
\begin{align*}
\mangle(\xi,\eta)+\mangle(\eta,\xi^*)
&=\dist{\~\xi}{\~\eta}{}+\dist{\~\eta}{\~\xi^*}{}
\le\pi.
\end{align*}
\qedsf







\section{Remarks and open problems}
Is there an analog of the Liouville's theorem for ``quasigeodesic flow''?
Namely:
\begin{thm}{Question}
Let $\spc{L}\in\CBB m{}$ 
and $\Gamma$ be the space of all quasigeodesics $\gamma\:\RR\to \spc{L}$.
Is it possible to define measure $\mu$ on $\Gamma$ such that 
\begin{subthm}{}
$\mu$ is invariant with respect to natural group action on $\Gamma$ generated by $\gamma\mapsto\gamma\circ\sigma$, where $\sigma(t)=t+a$ or $\sigma(t)=-t$
\end{subthm}

\begin{subthm}{}
if $e_t$ denotes evaluation map $e_t\:\gamma\mapsto\gamma(t)$ then 
\[e_t\#\mu=\vol_m\]
where $\#$ stays for push-forward.
%???MAyBE DEFINE AS $e_t\:\gamma\mapsto\gamma^+(t)$???
\end{subthm}
\end{thm}

If the answer to the following question is positive, it could give an alternative proof of existence of quasigeodesics (\ref{thm:exist-qg}).

\begin{thm}{Question}
Is it true that in a $\Alex{}$ space without boundary there is an infinitely long geodesic?
\end{thm}


As it was noted by A.~Lytchak, the these two questions can be reduced to the following:
Assume $\spc{L}\in\CBB m{}$ and $\partial \spc{L}=\emptyset$. 
Set
\[V(r)=\int\limits_\spc{L}\vol_m\oBall(x,r)\cdot\d_x\vol_m,\]
then \[V(r)=\vol_m\spc{L}\cdot\Const_m\cdot r^m+o(r^{m+1}).\]
The technique of \emph{tight maps} makes it possible to prove only that
\[V(r)=\vol_m\spc{L}\cdot\Const_m\cdot r^m+O(r^{m+1}).\]
Note that if $\spc{L}$ is a Riemannian manifold with boundary then 
\[V(r)
=
r^m\cdot\Const_m\cdot\vol_m \spc{L} +r^{m+1}\cdot\Const'_m\cdot\vol_{m-1}\partial \spc{L}
+o(r^{m+1}).\]

\smallskip

\emph{Variation of turn} of a broken geodesic $a_0 a_2\dots a_n$ can be defined as sum of angle excesses
\[\sum_{i=1}^{n-1}\l(\pi-\mangle\hinge{a_i}{a_{i+1}}{a_{i-1}}\r).\]
Further, variation of turn for a curve can be defined as lower limit of turns of broken geodesics which converge to a given curve (you can choose in which sense).

\begin{thm}{Question}
Let $\spc{L}\in\CBB m{}$.
Is it true that any quasigeodesic in $\spc{L}$ defined on a bounded interval has bounded variation of turn.
\end{thm}

If the all tangent spaces of $\spc{L}$ are sufficiently close to $\EE^m$ then the answer is ``yes''.
In this case variation of turn can be estimated though \emph{total turn to a point}\index{turn to the point $p$}, which defined as follows:
\[\sum_{i=1}^{n-1}\max\l\{0,\pi-\angk\kappa{a_i}{a_{i-1}}{p}-\angk\kappa{a_i}{a_{i+1}}{p}\r\}.\]






One could look at the following class of curves in general metric space.
This class of curves should be interesting in the metric spaces, which has ``a lot'' of semiconcave functions.

\begin{thm}{Definition}
Let $\spc{X}$ be a metric space.
A curve $\gamma\:\II\to \spc{X}$ is called \emph{funny quasigeodesic}\index{funny quasigeodesic} if for any $\lambda\in\RR$ and any $\lambda$-concave subfunction $f\:\spc{X}\subto \RR$ the composition $f\circ\gamma$ is $\lambda$-concave.
\end{thm}

An analog of existence theorem (???) does not hold for this type of curves,
once funny quasigeodesic comes to the boundary it terminates.
On the other hand it has a nice property: if $\spc{X}$ is a convex closed subset of $\spc{L}\in\CBB m{}$ then any funny quasigeodesic in $\spc{X}$ is a funny quasigeodesic in $\spc{L}$.
Also, it is easy to see that in $\cCat{}{}$ spaces, 
the class of funny quasigeodesics coinsides with class of geodesics.








\section{Exercises}

\begin{thm}{Exercise} Show that if $\spc{L}\in\CBB m1$ and $\partial \spc{L}\not=\emptyset$ then $\rad \spc{L}\le\tfrac\pi2$.
\end{thm}


\begin{thm}{Exercise}
Let $\spc{L}\in\CBB m1$, and $p_1,p_2,\cdots,p_m\in \spc{L}$. 
Show that there is $x\in \spc{L}$ such that $\dist{p_n}{x}{}\le\frac\pi2$ for each $n$.
\end{thm}

\begin{thm}{Exercise}
Let $\spc{L}\in\CBB m1$.
Assume $\rad\spc{L}>\tfrac\pi2$.
Show that 
\[\rad\Sigma_p>\tfrac\pi2\]
for any $p\in \spc{L}$.
\end{thm}

\begin{thm}{Exercise} Construct $\spc{L}\in\CBB 2 0$ with $p\in \spc{L}$ and $\xi\in \Sigma_p$ such that there are two quasigeodesics $\gamma_1,\gamma_2\:[0,\infty)\to \spc{L}$ with initial data $\gamma_1^+(0)=\gamma_2^+(0)=\xi$ such that $\gamma_1(t)\not=\gamma_2(t)$ for any $t>0$.
\end{thm}

\begin{thm}{Exercise} Construct $\spc{L}\in \CBB 3 0$, 
$\partial \spc{L}\not=\emptyset$ and a quasigeodesic $\gamma$ in $\spc{L}$ which is not a projection of any quasigeodesic in doubling of $\spc{L}$.

Show that there is no such example in $\CBB20$.
\end{thm}
