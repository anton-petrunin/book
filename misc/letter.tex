\documentclass[11pt]{amsart}

\usepackage{enumerate,}
\begin{document}
First, we would like to thank the referees for their report and the suggestions.

Let us address the specific comments and suggestions by the second referee, numbered as in the report.

\medskip

{\bf 3a.} 
We have eliminated some but not all forward referencing in the book.
Having some forward referencing was a deliberate decision during the writing of the book. This was done to make the presentation more natural. Most people reading the book will have some familiarity with the subject and will not find forward referencing that we do have
too distracting.
Also, whenever we do use it we provide explicit references to the places in the book where relevant material is introduced.
This is not something we are willing to change.

However, we did reorder Chapters 3, 4, 5 to reduce forward referencing in early chapters.
This in particular addresses referee's complain about the use of sub function notation before it was introduced.

\medskip

{\bf 3b.} 
We removed the ``very limited'' remark about our use of ultralimits. We also expanded the first paragraph of Chapter 4 motivating our use of ultralimits. 

\medskip

{\bf 3c.}
The description of our series says ``carefully written as teaching aids ... often used as textbooks''.  Although it can be used as such our book is not primarily meant as a textbook but as a teaching and reference aid for a course or self-study.

Regarding the lack of examples, we do have a large number examples in the book that are contained in exercises and are meant to be both illustrative and instructive.
We added  more examples in the revision.
In particular, we added subsections on examples to both the Chapter on CBB spaces and the chapter on CBA spaces.

\medskip

{\bf 3d. }
Our index includes notations and is searchable.
In particular, it includes $\varpi^\kappa$ that the referee mentions. It is defined on page 3 and this is mentioned in the index.
The use of $\bar k$ was a mistake.
We removed it and we thank the referee for pointing it out.
We added a few  items to the index that were missing. 

\medskip

{\bf 12.}
Following the suggestion by the referee we added a clarifying remark on the Hsiang--Kleiner work.
We mention the Grove--Wilking paper because of an exceptionally interesting proof. As this is only a brief mention and we are not stating the actual results in that paper we don't want to go into the full mathematical history of the results in question.

\medskip

{\bf 15.} Concerning the use of first names in references to various people, our convention is as follows.
Most of the time we use (first-name)+(last-name) for every reference.
We use just last names  in the names of theorems like Grove--Petersen comparison or Yamaguchi fibration theorem,  and sometimes when names reappear in the same context (Alexandrov and Gromov often come without their first names).
We believe this usage is consistent throughout the book.

\medskip

We fixed a number of typos in the revision including the ones the referee mentioned.

\medskip

We eliminated the footnotes as the referee suggested.

\medskip

We use ``$\omega$-limit'' when a choice of $\omega$ is implicit, and ``ultralimit'' when  discussing ultralimits in general.

\medskip

The word ``strut'' has the suggestive meaning ``framework that resists compression''.
The word ``strainer'' in English is unrelated to this context.


 \end{document}
