%%!TEX root = all.tex
\chapter{Non-positively curved spaces}

\section{Line strip theorem}

\begin{thm}{Line strip theorem}\label{split=<0}
Let $\spc{U}$ be a complete length $\CAT0$ space and $\ell???$ be a line in $\spc{U}$.
 Let $\spc{U}_\ell$ denotes the subset of all points in $\spc{U}$ for which sum of two Busemann functions ??? is equlal $0$.
Then $\spc{U}_\ell$ is a closed convex subset of $\spc{U}$ and it admits an isometric splitting $\spc{U}_\ell=\ell^\perp\oplus\ell$
\end{thm}

\section{Strip spaces}
5.1 Metrics on strip spaces. In [A1A1] wd approxhmasdd B x f R eor
K = 0 ax nonporhshvdlx curvdd rsrhp rpacdr+ crdasdd erom rsrhpr hn B x R
whsh shd producs mdsrhc- Hdrd wd a rhdflx rdvhdw shd conrsrucshon+ and
rhow shas+ whdn K = 1+ wd can rdsahn shd ddrhrdd curvasurd aound on
shd rs rhp rpacd ax plachnf a ruhsaald hxpdraolhcallx warpdd mdsrhc B arkx R
(= R xark B whsh shd orddr oee acsorr rdvdrrdd) on shd rs rhpr-

Takd hr omdsrhc cophdr (hnddxdd ax i à Z) oe a strip ff( h n B x R:

W^( = {(p, u) :   ef(p) <u< ef(p)} ¶                                                                                                                                                                                                                                                   (5-1)

Thd strip space We hr oasahndd ax f luhnf s hd rs rhpr We rdqudnshallx alonf
shdhr aoundarhdr- Thdrd hr a nasural homdomorphhrm cpe : B x f R ó> We+
fhvdn ax ddcomporhnf Bx^R hnso horhzonsal rsrhpr {(p,y) : (1i l)e <
V < (1i + l)e}- Thd rdrsrhcshon tp\ oe tpe so shhr horhzonsal rsrhp mapr hs
onso M/jr   ax lhndar rdparamdsnzashon oes hd fiadrr:

iff( : (p,y) h^ p,f(p)[y   1ie]) .                                                                                                                                                                                                                                                                                                         (5-1)

Takd s hd mdsrhc oe We so ad shas h nducdd erom B arkx R+ whdrd k whll
ad chordn so consrol shd curvasurd oe We hn shd lhmhs- In shd BAA card+
shhr curvasurd consrol ddpdndr on Pdrdlman'r doualhnf s hdordm [P]+ rdd
alro Pdsrunhn'r mord fd ndral fluhnf s hdordm [Pds]; shdrd s hdordmr concdrn
fluhnf BAA r pacdr alonf hromdsrhc aoundarhdr- In shd BA card+ wd urd
shd dual conrsrucshon+ namdlx+ Rdrhdsnxak'r f luhnf s hdordm [R] eor fluhnf

Unk-03+1//3                                                                                                                                                             BTPU:STPD ANTLCR ENP V:PODC OPNCTBSR                                                                                                                                                             0046

along isometric totally convex sets. This construction was used previously
in [A1B1], and a similar construction was used by Burago, Ferleger and
Kononenko in [BuFK] to solve long-standing billiards problems. We de-
compose copies of B ar ix P into three regions, Lv  U We   U L\ , where

U^( = |(p, u) : ef(p) < w} .  Ly( = {(p,u) : u <   e/(p)} ï        (5-3)
Construct a space W* by sequentially identifying the isometric sets Lv

l   T-f^ )0 (    mi        rrr* ï                                                                                                                        ±  ï                                                                                                                                                   Trr                                                                                                                                                                ±    l l        ^     l ï        ai       55

and Le      . Ihus Vre is our strip space We augmented by attaching  ffns

rr(i( r^j  T f^ )0 (    mi                                                                                                                                                                                                                                      ^-                                                                           ^  ï                                                                                                                                                tit                                                                                                              -^    if 1                                                                                                                                                 -^-     1

Ut = Le . 1 he approximating strip space We may itselt have positively
inffnite curvature, but is embedded in W* in such a way that we can bound
its curvature in the limit.

Assume K ? {  1. 0}. If K = 0, set k = 0. If K =    1, set k = k(e) =
1 + e in case (CBA), and f =   1 in case (CBB).

Lemma 5.1. (BAA) I dt B ad a compldtd, finhtd-dhmdnrhonal rpacd of BAA
ax K whth dmptx aoundarx, and f : B ó> P>/ ad a porhthvd J-K-concavd
functhon. Thdn thd r trhp rpacd We, formdd ax rd qudnthallx f luhnf r trhpr
We    C Bar I x P, har BAA ax K.

(CBA) I dt B ad a CAT(K) rpacd and f : B ó> P>/ ad a Ihprchhtz
J-K-convdx functhon. Thdn for e rufichdntlx r mall, thd aufmdntdd rtrhp
rpacd W*, formdd ax attachhnf "finr" to We, hr CAT(fc) for k = k(e) ar
aaovd.

Proof. First note that B ar ix P has CBB by k in the case (CBB), and BA
by k in the case (BA). Such standard coning constructions are discussed,
for example, in [BuBI]. Now by the doubling and gluing theorems, respec-

^-       1                                                                                                                                                                                  111                                                                              ^1      ^   ï       ^1                                                   r-1T->T->                                                                                                                    ^1                                                                ±    ï       TTr(^(     1   f                                                     11

tively, we need only show that in the CBB case, the strip W^   defined by

/r   -I \    ï                                                                                                                                                                                                                                 n                                                                                                                                                                                                                  1*^1                                                                                                                   A                                                                                                                             ^1        f       T j(i(     1   f                                                     11

(5.1) is convex in B ar ix P, and in tfie BA case, tfie fn U^ defned by
(5.3) is convex in B ar ix P when e is sufficiently small.

Since minimizers in B ar ix P project to minimizers in B (Lemma 2.3.a),

ï±        r ^1        ^  ï      Trr(i)                                                                                                                                                  rr(i(   ï                                                                                                       ï     l     ^   ^                                                                                                                                                                   -^      ï     ^l

convexity ol tfie strip We or ffn U^ is equivalent to convexity in tfie
cylinder above a minimizer in B. If k = 0, this is immediate from the
concavity (CBB) or convexity (BA) of / along geodesies of B. In the
remaining cases, the problem reduces to the hyperbolic plane P ari x P.
We must verify the convexity of {(?, u) : u < e/(t)}, where f"(t) f(t) < 0
in the barrier sense, in the case (CBB); and the same statement with the
inequalities reversed in (BA). By deffnition, the di?erential inequalities
imply that the graph of / lies above (CBB) or below (BA) sufficiently
ffne inscriptions by broken T(   l)-affine graphs, and the one-sided tangent

0047

R-A-:IDW:LCDP :LC P-I-AHRGNO

F:E:

udbsnqr nesgdrd hmrbqhoshnmr qns]sd cnvmv]qc íBAA( nq tov]qc íBA=( ]s
sgd aqd]jr- Sgdqdenqd hs rtfl
] fhudm Khorbghsy bnmrs]ms-

the breaks. Therefore it suffices for this calculation to take / ? J-{   1) with

__

Writing the metric as ds   = cosh (y   ku)dt + du , we calculate the

curvature vector of the curve u = ef. Specifically, if V = dt + f'(t)du is
the velocity vector and N is the unit normal, then the 9Ñ-component of the
curvature vector kN is

__                                                                                                                                                                                                      __                                                                                                                         __                                                                                                                                                             __

\v\    [cosh V kef ef"    \J  fcoshy7  fce/sinhy7  fee/)

2e (/') v   fcoshy7  fce/sinhy7  fee/].   (5.4)

(CBb) Ihe condition tor convexity of Vre   is that Kiv point downward,

hence that the expression in (5.4) be nonpositive. After substituting /" = /

and multiplying by \v\ /e/cosh y7  fee/, we find that concavity is detected

by the following inequality:

__       ^      __        __                                                                                                                                                                                                                                                                                                                              __       __

(  k) [ cosliV  kef) sinliV  kef)/V  kef + 2e (f ) (tanliV  kef)/V  kef\ >1.

(5-5)
Clearly this inequality holds if k =    1, since then it holds for the first
term and the second term is positive .

(bA) .tor convexity of ue¶ , we want the curvature vector to point
upward, hence the inequality in (5.5) to be reversed. Since /' and / are
bounded, the left-hand expression is on the order of fc[l-\-Ce ] for a uniform
constant C. Thus the reversed inequality (5.5) is satisfied for e sufficiently
small if k = k(e) =   1 + e.



