%%!TEX root = all.tex
\chapter{Spaces of curvature bounded above}

\section{Line strip theorem}

\begin{thm}{Line strip theorem}\label{split=<0}
Let $\spc{U}$ be a complete length $\CAT0$ space and $\ell???$ be a line in $\spc{U}$.
 Let $\spc{U}_\ell$ denotes the subset of all points in $\spc{U}$ for which sum of two Busemann functions ??? is equlal $0$.
Then $\spc{U}_\ell$ is a closed convex subset of $\spc{U}$ and it admits an isometric splitting $\spc{U}_\ell=\ell^\perp\oplus\ell$
\end{thm}








}

%\begin{thm}{Exercise} 
%(Gluing with short maps)
% Let $X,X' \in\Cat{}{0}$, $K \subset X, K' \subset X'$ be closed convex subsets, 
%and suppose the length-distance between any two points
%on $\partial_X K$ and $\Fr_{X'} K'$ are finite.
%Suppose $\phi\: \partial_X K \to \Fr_{X'} K'$ is length-preserving and 
%short with respect to the inherited
%metrics. 
%To $X'\setminus\Int K'$ in its length metric, 
%glue $K$ along $\phi$, to obtain the space $Y$. 
%Show $Y \in \Cat{}{0}$.

%For example, take $K= \cBall[x,\sinh R]  \subset \Lob{m}0$ and $K' = \cBall(x',R)  \subset \Lob{m}{-1}$.
%\end{thm}


\section{Strip spaces}
%5.1 Metrics on strip spaces. In [A1A1] wd approxhmasdd B x f R eor
%K = 0 ax nonporhshvdlx curvdd rsrhp rpacdr+ crdasdd erom rsrhpr hn B x R
%whsh shd producs mdsrhc- Hdrd wd a rhdflx rdvhdw shd conrsrucshon+ and
%rhow shas+ whdn K = 1+ wd can rdsahn shd ddrhrdd curvasurd aound on
%shd rs rhp rpacd ax plachnf a ruhsaald hxpdraolhcallx warpdd mdsrhc B %arkx R
%(= R xark B whsh shd orddr oee acsorr rdvdrrdd) on shd rs rhpr-

A \emph{strip space} is an important way to construct  CBA and CBB spaces. This construction is used to characterise warped products with a given curvature bound.

Consider  isometric copies (indexed by $i\in Z$) of a \emph{strip} $W^{(i)}$ in $ B \times R$:
\begin{eqnarray}
W_{\epsilon}^{(i)} = \{(p, u) :   -\epsilon f(p) \le u\le  \epsilon f(p)\}.                                                                                                                                                                                                                                                \end{eqnarray}
The strip space $W_\epsilon$ is obtained by gluing the strips  $W_{\epsilon}^{(i)}$ 
sequentially  along 
their boundaries. There is a natural homeomorphism $\phi : B \times_ f R \to W_\epsilon$ 
given by decomposing $B \times_ f R$ into horizontal strips 
  $\{(p, y) :   (2i-1)\epsilon\le y\le (2i+1) \epsilon \}$. 
The restriction $\phi_\epsilon^{(i)}$ of $\phi_\epsilon$ to this horizontal strip maps it onto $W_{\epsilon}^{(i)}$
by linear reparametrization of the fibers:
\begin{eqnarray}
\phi_\epsilon^{(i)}: (p,y) \to ( p,f(p)[y-2i\epsilon) .                                                                                                                                                                                                                                                                                                         
\end{eqnarray}
Take the metric of $W_\epsilon$ to be that induced from $B_{\cs_k}}\times R$,
where $k$ will be chosen to control the curvature of $W_\epsilon$ in the limit.
%ad chordn so consrol shd curvasurd oe We hn shd lhmhs- In shd BAA card+
%shhr curvasurd consrol ddpdndr on Pdrdlman'r doualhnf s hdordm [P]+ rdd
%alro Pdsrunhn'r mord fd ndral fluhnf s hdordm [Pds]; shdrd s hdordmr concdrn
%fluhnf BAA r pacdr alonf hromdsrhc aoundarhdr- In shd BA card+ wd urd
%shd dual conrsrucshon+ namdlx+ Rdrhdsnxak'r f luhnf s hdordm [R] eor fluhnf

%Unk-03+1//3                                                                                                                                                             BTPU:STPD ANTLCR ENP V:PODC OPNCTBSR                                                                                                                                                             0046

%along isometric totally convex sets. This construction was used previously
%in [A1B1], and a similar construction was used by Burago, Ferleger and
%Kononenko in [BuFK] to solve long-standing billiards problems. We de-
%compose copies of B ar ix P into three regions, Lv  U We   U L\ , where

%U^( = |(p, u) : ef(p) < w} .  Ly( = {(p,u) : u <   e/(p)} ï        (5-3)
%Construct a space W* by sequentially identifying the isometric sets Lv

%l   T-f^ )0 (    mi        rrr* ï                                                                                                                        %±  ï                                                                                                                                                   %Trr                                                                                                                                                                %±    l l        ^     l ï        ai       55

%and Le      . Ihus Vre is our strip space We augmented by attaching  ffns

%rr(i( r^j  T f^ )0 (    mi                                                                                                                                                                                                                                      %^-                                                                           ^  ï                                                                                                                                                %tit                                                                                                              -^    if 1                                                                                                                                                 %-^-     1

%Ut = Le . 1 he approximating strip space We may itselt have positively
%inffnite curvature, but is embedded in W* in such a way that we can bound
%its curvature in the limit.

Specifically,  take $K\in \{-1,0\}$.  If $K = 0$, let $ k = 0$.  If $ K =  -  1$, let $k = k(\epsilon) = -1+\epsilon$ in case  (CBA), and $k=-1$ in case (CBB).

(CBB)Let $B$ be a complete, finite-dimensional space of CBB by $K$
with empty boundary, and $f : B \to \R_{ > 0}$ be a positive K-concave 
function. Then the strip space $W_\epsilon$, formed by sequentially gluing strips $W_\epsilon^{(n)} \subset $ {B_{\cs_K}} \times \R$,
has CBB by $K$.

(CBA) Let $B$ be a CAT(K) space  and  $f : B \to \R_{\ge 0}$
be a Lipschitz 
$K$-convex function. Then for $\epsilon$ sufficiently small, the augmented strip
space $W^*_\epsilon$, formed by attaching "fins" to $W_\epsilon$, is CAT($K$) for $k = k(\epsilon)$ as
above.

Proof. First note that ${B_{\cs_k}}\times\R$  has CBB by k in the case (CBB), and CBA
by $k$ in the case (CBA). Such standard coning constructions are discussed,
for example, in [BuBI]. Now by the doubling and gluing theorems, 
respectively, we need only show that in the CBB case, the strip $W_\epsilon^{(i)}$   defined as above is convex in ${B_{\cs_k}}\times \R$, and in the CBA case, the  fin $U_\epsilon}^{(i)}$ defined as above is convex in $ {B_{\cs_k}}\times \R$   
when $\epsilon$ is sufficiently small.

Since minimizers in ${B_{\cs_k}}\times\R$ project t\o minimizers in $B$ (Lemma ),                                                                                                                                                                                                                                                                                                                                                                                             
convexity ol tfie strip $W_\epsilon^{(i)}$ or fin $U_\epsilon^{(i)}$is equivalent to convexity in the
cylinder above a minimizer in $B$. If $k = 0$, this is immediate from the
concavity (CBB) or convexity (BA) of $f$ along geodesies of $B$. In the
remaining cases, the problem reduces to the hyperbolic plane P ari x P.
We must verify the convexity of $\{(t, u) : u \le \epsilon f(t)\}$, where $f"(t)- f(t) \le 0$
in the barrier sense, in the case (CBB); and the same statement with the
inequalities reversed in (CBA). By defnition, the dfferential inequalities
imply that the graph of $f$ lies above (CBB) or below (CBA) sufficiently
fine inscriptions by broken $\mathcal{F}(-1)$-affine graphs, and the one-sided tangen:
vectors of these inscriptions rotate downward (CBB) or upward (CBA) at thee breaks. Therefore it suffices for this calculation to take  $f\in \mathcal{F}(-1)$
with a given Lipschitz constant.

Writing the metric as ds^2$   = \cosh^2 (\sqrt{-k}u)dt^2 + du^2$ , we calculate the
curvature vector of the curve $u = \epsilon f$. Specifically, if $V = \partial_t + f'(t)\partial_u$ is
the velocity vector and $N$ is the unit normal, then the $\partial_u$-component of the
curvature vector $\kappa N$ is
\begin{eqnarray}
|v|^{-4}    [\cosh^2 \sqrt{-k}\epsilon f (\epsilon f"- \sqrt{-k}\cosh    \sqrt{-k}\epsilon f \sinh\sqrt{-k} \epsilon f) \nonumber \\
- 2\epsilon^2(f')^2\cosh    \sqrt{-k}\epsilon f \sinh\sqrt{-k} \epsilon f) 
\end{eqnarray


(CBB) The condition tor convexity of $W_i^{(\epsilon)}$   is that $\kappa N$ point downward,
hence that this expression  be nonpositive. After substituting $f''=f$ 
and multiplying by $|v|^4/\epsilon f \cosh^2 \sqrt{-k}$ \epsilon f/\sqrt{-k}$+2\epsilon^2$, we find that concavity is detected
by the following inequality:

__       ^      __        __                                                                                                                                                                                                                                                                                                                              __       __

\begin{equation}
 (-k)[(\cosh\sqrt{-k}\epsilon f)\sinh\sqrt{-k}\epsilon f/\sqrt{-k}\epsilon f+2\epsilon^2 (f')^2(\tanh\sqrt{-k}\epsilon f)/\sqrt{-k}\epsilon f]\ge 1.  
\end{equation}

Clearly this inequality holds if $k =1$, since then it holds for the first
term and the second term is positive .

(CBA) For convexity of $U^{(i)}_\epsilon$, we want the curvature vector to point
upward, hence the inequality in (5.5) to be reversed. Since $f'$and $f$are
bounded, the left-hand expression is on the order of $k[1+C\epsilon^2]$ for a uniform
constant C. Thus the reversed inequality (5.5) is satisfied for e sufficiently
small if $k = k(\epsilon) = -1 + \epsilon$$.

\section{Jacobi lengths}
\label{sec:jacobi-length}

This section gives a characterization of  locally $\CAT\kappa$ spaces that captures the notion in a Riemannian manifold of lengths of normal and tangential Jacobi fields.
Tangential lengths, which are linear, must be discarded in order to identify specific curvature bounds.

This material is from \cite{???} and \cite{???}.  
It is used in \cite{???} to prove a sharp Alexandrov curvature bound above for Riemannian manifolds-with-boundary (\ref{thm:example-mnflds-with-bry:CBA}), and in \cite{} to obtain sharp curvature bounds for subspaces of spaces with curvature $\le\kappa$ (\ref{sec:gauss-equation}).

\begin{thm}{Definition}
Let $[pq]$ 
be a geodesic in an intrinsic space $\spc{X}$
and $\ell=\dist{p}{q}{}$.
A function $f\:[0,\ell)\to\RR_{\ge0}$ is called %\emph{normal Jacobi length} along $[pq]$
if there is a sequence of geodesics $[pq_n]$ in $\spc{X}$
such that $q_n\to q$ as $n\to\infty$ and
\[f(t)
=
\lim_{n\to\infty}\frac{\dist{\geod_{[pq_n]}(t)}{\geod_{[pq]}(t)}{}}{\dist{q_n}{q}{}}.\]

The geodesic $[pq] 
%is said to satisfy \emph{Jacobi splitting} if 
%for any normal Jacobi length $f$ along $[pq]$, 
%%the sequence of geodesics $[pq_n]$ as above we have
\[\dist[2]{\gamma(t)}{\gamma_n(t_n)}{}=
f_n(t)^2\cdot\dist[2]{q}{q_n}{}
+
|t-t_n|^2
+
o(\dist[2]{q}{q_n}{}+|t-t_n|^2).\]
\end{thm}


\begin{thm}{Theorem}\label{thm:jacobi-length}
Let $\spc{U}$ be a locally geodesic metric space. 
Then $\spc{U}$ is locally $\CAT\kappa$ if and only if for any $p\in\spc{U}$
there is a neighborhood $N\ni p$
such that for any geodesic $[pq]$ in $N$ the following two conditions hold

\begin{subthm}{jacobi-split} 
The geodesic $[pq]$ satisfies Jacobi splitting;
\end{subthm}

\begin{subthm}{jacobi-convex}
Any normal Jacobi length $f$ along $[pq]$ 
satisfies the following differential inequality
\[f''+\kappa\cdot f\ge 0\]
in the sense deascribed in ???
 \end{subthm}
\end {thm}

\begin{thm}{Lemma}\label{lem:model-jacobi}
Normal Jacobi length $f$ for any geodesic with lenght $<\varpi\kappa$ in the model space $\Lob2\kappa$
satisfies
$$f'' +\kappa\cdot f=0.$$
  
\end{thm}

\begin{thm}{Lemma}
Let $\gamma_0,\gamma_1\:[a,b]\to\Lob{}\kappa$, 
be two geodesics such that
\[
\dist{\gamma_0(a)}{\gamma_0(b)}{}+ \dist{\gamma_0(b)}{\gamma_1(b)}{}+ \dist{\gamma_1(b)}{\gamma_1(a)}{} +  \dist{\gamma_1(a)}{\gamma_0(a)}{} <2 \varpi\kappa.
\]
Set $f(t)=\dist{\gamma_0(t)}{\gamma_1(t)}{}$.
Then 
\[f''+K\cdot f\ge 0\]
where
\[
K=\min\{0,-\kappa\cdot\bigl(\varpi\kappa/(b-a)\bigr)^2\}.
\]
 \end{thm} 

%\begin{thm}{Definition (Jacobi
%length)}\label{def:jac}
%Let $\spc{X}$ be an intrinsic metric space, and $\gamma:\II\to\spc{X}$ 
%be a geodesic, where $\II$ is an interval. Then $f:\II\to [0,\infty)$  is 
%called a \emph{Jacobi  length along $\gamma$} if there is a sequence of 
%geodesics $\gamma^i:\II\to\spc{X}$ and a sequence of positive numbers $u_i\to 0$ such that for all $s\in \II$,
%\[
%f(s) = \lim_{i\to\infty} \  \dist{\gamma(s)}{\gamma^i(s)}{}/u_i.
%\eqlbl{eq:jac-length}
%\]
%If moreover
%\[
%\dist{\gamma^i(t)}{\gamma(t)}{}= \dist{\gamma^i(t)}%{\gamma(\II)}{} + o(u^i),\]
%then $f$ is called a \emph{normal Jacobi  length %along $\gamma$}.
%\end{thm}

Given  a point  $p$ and a curve $\alpha\:[0,1]\to \spc{X}$ of constant speed $A$, recall that a map
\[
[0,1]\times[0,1]\to\spc{X}\:(t,s)\mapsto\gamma_t(s)
\] 
is a \emph{line-of-sight map from $p$ to $\alpha$} 
if each curve $\gamma_t$ is a geodesic from $p$ to  $\alpha(t)$ (\ref{def:sight}). 

\parbf{Example.}  
A sinusoidally $K$-convex function $f:[a,b]\to [0,\infty)$ must be continuous on $(a,b)$.
However, a sinusoidally $\kappa$-convex Jacobi length $f:[a,b]\to [0,\infty)$  in a $\CAT\kappa$ space may be discontinuous at the endpoints. 
Consider the complement $\spc{U}$  in  $ \EE^2$ of an open disk, and  a 
geodesic $\gamma\: [0,1]\to\spc{U}$  which lies on the bounding circle.  
A line-of-sight  map is defined by lifting the endpoint $\gamma(1)$ along an ``evolute''  curve of speed $1$ running at constant \,$\spc{U}$-distance from $\gamma(0)$. 
There is a  corresponding Jacobi length $f$ along $\gamma$, where   $f(s)=0$ for $0\le s <1$ \,and\,  $f(1)=1$.

%\parit{Proof of Lemma \ref{lem:jac-exist}.} 
%By assumption, 
%$$f_i(s)\,\le\,(\,\dist{\alpha(t)}{\alpha(t_i)}{}   /\sn{K}1\,)\,\cdot\, \sn{K}s\,+\, o(t-t_i).$$
% Therefore for fixed $s$, any limiting value
%of $f_i(s)/(t-t_i)$  as $i\to\infty$ is  at most equal to\, 
%$ A\cdot(\sn{K}s) /\sn{K}1).$
%Now that infinite limits are ruled
%out, we can use a diagonal process to pass to a subsequence of the $\gamma_i$ such that $f(s)<\infty$ is defined on a countable dense subset $S$ of $[0,1]$.

%To show that $f$ exists and is continuous on $[0, 1)$, it suffices to show, for
%fixed $s_0\in [0, 1)$, that any limiting value \,$C$\, of $f_i(s_0)/(t-t_i)$ is equal to any limiting value $D$ of $f(s_i)$, as $s_i\in S$ converges  monotonically to $s_0$.

%Since the $K$-sinusoid  with values
%$f(s_1)$ and $C$ at $s_1$ and $s_0$, respectively, dominates $f(s_i)$ for all $i$, then $D\le C$. 

%When $s_0=0$, we have $D=C$ since  $C=0$.  
%Suppose $s_0\in (0,1)$.  Take $ \bar s\in S$ on the opposite side of $s_0$ from the monotonically converging $s_i$. The $K$-sinusoid ${\bar f}_i$ with values $f(s_i)$ and $f(\bar s)$ at $s_i$ and $\bar s$, respectively, satisfies ${\bar f}_i(s_0) \ge C$ for all $i$.
%Hence $D \ge C$.\qeds

%\begin{thm}{Definition}
%\label{def:line-of-sight}
%Let $\spc{X}$ be an intrinsic metric space.
%A line-of-sight map 
%is said to \,\emph{satisfy  Jacobi %splitting}\,  if
%the following holds:

%For each $t\in[0,1)$ there is a %sequence  $t_i\to t^+$, and   a  %Jacobi length $f$  along $\gamma$ %given by \ref{eq:jac-length}, 
%%and continuous on $[0,1)$, 
%where $\gamma=\gamma_t$ , %$\gamma_i=\gamma_{t_i}$ and %$u_i=t_i-t$. In addition, setting %\,$\ell=\length\gamma$\, and\, %$\ell_i=\length\gamma_i$, for $s\in %(0,1)$  set
\[
g_i(s)= 
\dist{\gamma(s)}{\gamma_i(
\ell\cdot
{\ell_i}^{-1}
\cdot s)}{}.
\]
%\Then, passing to a subsequence, 
%the \emph{normal Jacobi length} %$f_N$ and \emph{tangential Jacobi %length} $f_T$  defined on $(0,1)$ by 
\[
f_N(s)\, = \,\lim_{i\to \infty} \bigl( g_i(s)/(t-t_i)\bigr),\ \ \ 
f_T(s) \,= \, \lim_{i\to\infty}\bigl(|\ell - \ell_i|\cdot s\,/(t-t_i)\bigr)
\eqlbl{eq:normal-tangent-jacobi-length}
\]
%exist and satisfy
\[
f^2 = {f_N}^2 + {f_T}^2.\\
\eqlbl{eq:jacobi-split}
\]%If  there is no sequence $t_i\to t^+$ %such that $\ell_i\ge\ell$, then %$g_i(1)$ is undefined, and we %define $f_N(1)$ by \,\ref{eq:jacobi-split}; that is, $f_N(1)^2=f(1)^2-f_T(1)^2$.
%\end{thm}

Recall that $\geodpath_{[xy]}$ denote a geodesic path from $x$ to $y$.



%\begin{thm}{Lemma}\label{lem:model-jacobi}
%For any $p\in\Lob{}\kappa$ and any triangle $\trig {\tilde x}{\tilde y}{\tilde z}$ in $\oBall(p,\tfrac{\varpi\kappa}{2})$, the line-of-sight map from $\tilde x$ to $\geodpath_{[\tilde y\tilde z]}$ satisfies Jacobi splitting.  Moreover,   for each $t\in [0,1)$ the corresponding  normal Jacobi length  is a $(\kappa\cdot\ell^2)$-sinusoid.
%let $\gamma$ and $\sigma$ be geodesics of length $\le\varpi\kappa$ and  speed $\le 1$.  Then $\dist{\gamma(s)}{\sigma_i(s)}{}$ is sinusoidally $\hat\kappa$-convex where  $\hat\kappa=\max\{0,\kappa\}$.


\begin{thm}{Theorem}\label{thm:jacobi-length}
Let $\spc{U}$ be a locally geodesic metric space.
Then $\spc{U}$ is locally $\CAT\kappa$ if and only if for any $p\in\spc{U}$ and any triangle $\trig {x}{y}{z}$ in some open ball about $p$:

%\begin{subthm}{jacobi-lip}
%line-of-sight maps are Lipschitz; 
%\end{subthm}

%\begin{subthm}{first-var}
%first variation equation holds from $x$ to $\geodpath_{[yz]}$;\end{subthm}

\begin{subthm}{jacobi-split} 
the line-of-sight map from $x$ to $\geodpath_{[yz]}$
%is Lipschitz and 
satisfies   Jacobi splitting;
\end{subthm}

\begin{subthm}{jacobi-convex}
each corresponding normal Jacobi length is  sinusoidally $(\kappa\cdot\ell^2)$-convex,
where $\ell$ denotes ???.
 \end{subthm}

\end {thm}

\parit{Proof of ``only if''.} 
%\parit{($\Rightarrow$)} 
Suppose $\spc{U}$ is locally $\CAT\kappa$. 
For any $p\in\spc{U}$, there is a convex ball 
$\oBall(p,R)$ which is $\CAT\kappa$ and $R<\tfrac{\varpi\kappa}{2}$ (\ref{cor:convex-balls}). 
Choose the triangle $\trig {x}{y}{z}$ to lie in $\oBall(p,R)$.

%\parit{(\ref{SHORT.first-var}).}  (\ref{SHORT.first-var}) holds  by \ref{thm:1st-var=cba}. 
 
\parit{(\ref{SHORT.jacobi-split}).}
We use the notation of Definition  \ref{def:line-of-sight}. The line-of-sight map from $x$ to $\geodpath_{[yz]}$ is Lipschitz by  triangle comparison. Therefore for each $t\in[0,1]$ there is a sequence  $t_i\to t^+$, and   a  Jacobi length $f$  along $\gamma$ given by \ref{eq:jac-length}.

\begin{clm}{}\label{right-angles}
For $s\in (0,1)$
% and also for $s=1$ in case we may choose $t_i\to t^+$ such that $\ell_i\ge\ell$
and $i$ sufficiently large, the angles
%set 
%\[
% \mangle^\gamma_i(s) =\mangle\hinge {\gamma( s) }{p}{ \gamma_i(\ell\cdot
%{\ell_i}^{-1}\cdot s)}\,, \ \ \ 
% \mangle_i(s) =\mangle\hinge {\gamma_i ( \ell\cdot
%{\ell_i}^{-1}\cdot s)}{p}{ \gamma(s)}.
%%\eqlbl{eq:jacfd2}
%\]
\[
\mangle\hinge {\gamma( s) }{\gamma(1)}{ \gamma_i(\ell\cdot
{\ell_i}^{-1}\cdot s)}\,,  \ \ \ 
 \mangle\hinge {\gamma_i ( \ell\cdot
{\ell_i}^{-1}\cdot s)}{\gamma_i(1)}{ \gamma(s)}.
\eqlbl{eq:jacfd}
\]
converge to $\pi/2$, as do the angles 
\[
  \mangle\hinge {\gamma( s) }{\gamma(1)}{ \gamma_i(\ell\cdot
{\ell_i}^{-1}\cdot s)}\,,  \ \ \ 
 \mangle\hinge {\gamma_i ( \ell\cdot
{\ell_i}^{-1}\cdot s)}{\gamma_i(1)}{ \gamma(s)}.
\eqlbl{eq:jacfd1}
\]
%
%\[
%\mangle\hinge {\gamma( s) }{p}{ \gamma_i(\ell\cdot
%{\ell_i}^{-1}\cdot s)}\,, \ \ \ 
% \mangle\hinge {\gamma_i ( \ell\cdot
%{\ell_i}^{-1}\cdot s)}{p}{ \gamma(s)}.
%\eqlbl{eq:jacfd2}
%\]
(If $g_i$ vanishes on an
initial interval, we set the undefined angles to be $\pi/2$.)

\end{clm}

By majorization (\ref{thm:major}), since $g_i$ is  sinusoidally $K$-convex for some $K$, hence  semiconvex.  Therefore $g_i$  has derivatives almost everywhere and one-sided derivatives everywhere.   Since $g_i\to 0$, it follows that $g_i^{\pm}\to 0$ 
%converge uniformly to $0$ on $[0,1-\epsilon)$ 
(Corollary \ref{lem:der-conv-lim}).

 By the first variation equation \ref{cor:both-end-first-var-cba}, 
 \[g_i^-
= -\cos \mangle^{\gamma}_i - \cos \mangle_i.
\]
By convexity of balls, the angles  \ref{eq:jacfd} are $\le\pi/2$.  
Hence $\mangle^{\gamma}_i (s)$ and $\mangle_i(s)$ approach $\pi/2$ as $i\to\infty$. 

Similarly, since
the angles \ref{eq:jacfd1} are $
\ge \pi/2$ by  the triangle inequality for angles, both angles
approach
$\pi/2$. Hence claim \ref{right-angles}.

\begin{clm}{}\label{curv-jacobi-split}
The line-of-sight map from $x$ to  $\geodpath_{[yz]}$ 
%is Lipschitz and 
satisfies Jacobi splitting.\end{clm}
  
We continue in
 the notation of Definition  \ref{def:line-of-sight}. 
% Passing to a subsequence, we obtain a    Jacobi length $f_N$ along $\gamma|[0,1)$  given by \ref{eq:normal-tangent-jacobi-length}.
Let $s\in (0,1)$.
 
We must prove
\[
f(s)^2 = f_N(s)^2 + s^2\cdot \lim_{i\to\infty} \bigl((\ell -\ell_i)/(t-t_i)\bigr)^2,
%= F(s)^2 + s^2\cos^2\theta,
\]
or equivalently, 
\[
g_i(s)^2 - f_i(s)^2 + s^2\cdot(\ell -\ell_i)^2 = o((t-t_i)^2).
\]

Let $\triangle_i=[{\gamma(s)}\ {\gamma_i( \ell\cdot{\ell_i}^{-1}\cdot s)} \ \gamma_i(s)]$. Then  $\triangle_i$ has sidelengths $g_i(s)$,
$s\cdot (\ell_i -\ell)$,  $f_i(s)$. 
Let $\triangle_i^0=\modtrig 0({\gamma(s)}\,{\gamma_i( \ell\cdot{\ell_i}^{-1}\cdot s)}\,{\gamma_i(s)})$ be the Euclidean model triangle for $\triangle_i$. Let $\mangle_i(s)$ be the angle of $\triangle_i$ at the vertex $\gamma_i( \ell\cdot{\ell_i}^{-1}\cdot s)$ and  $\mangle_i^0(s)$ be the corresponding angle of ${\triangle}_i^0$. Let $\mangle_i^\kappa(s)$ be the corresponding comparison angle in $\Lob{}\kappa$.  
By triangle comparison,
${\mangle}_i^\kappa(s)\ge \mangle_i(s)$. 

By the law of cosines,
\[
g_i(s)^2 - f_i(s)^2 + s^2\cdot(\ell -\ell_i)^2=
g_i(s)\cdot s \cdot (\ell -\ell_i)\cdot\cos\mangle_i^0(s).
\eqlbl{eq:jacfd3}
\]
  

Consider  the quadrilateral $\Box_i= [{\gamma(s)}\,{\gamma_i( \ell\cdot{\ell_i}^{-1}\cdot s)}\,{\gamma_i(s)} \gamma(\ell_i\cdot{\ell}^{-1}\cdot s))]$. 
By majorization (\ref{thm:major}), $\Box_i$ is majorized by a quadrilateral ${\Box}_i^\kappa$ in $\Lob{}\kappa$.
${\Box}_i^\kappa$ has two
adjacent sidelengths agreeing with those of ${\triangle}_i^\kappa$,
and the diagonal joining their endpoints  is $\ge f_i(s)$, which is the third sidelength of
${\triangle}_i^\kappa$. Therefore the included angle $\mangle_i^{\Box^\kappa}$ satisfies 
\[
\mangle_i^{\Box^\kappa}\ge \mangle_i^\kappa(s) \ge \mangle_i(s).
\eqlbl{eq:jacfd4}
\]

All four angles of $\Box_i^\kappa$
majorize angles that become, by claim \ref{right-angles}, arbitrarily close to $\pi/2$ as $i\to\infty$.   Therefore all
angles of $\Box_i^\kappa$, including $\mangle_i^{\Box^\kappa}$, become
arbitrarily close to $\pi/2$ as $i\to\infty$.  By \ref{eq:jacfd4}, the same is true of  $\mangle_i^\kappa(s)$.  

Thus the expression \ref{eq:jacfd3}, when divided by $(t-t_i)^2$, approaches $0$, since  
\[
 \lim_{i\to\infty} \cos\mangle_i^0(s)=\lim_{i\to\infty} \cos\mangle_i^\kappa(s)=0,
\] 
$\lim_{i\to \infty}\bigl( g_i(s)/(t-t_i)\bigr)=f_N(s)\,$ and \,$\lim_{i\to \infty}\bigl((\ell_i - \ell)\,/(t-t_i)\bigr)=\cos\mangle\hinge {\alpha(t)} {\alpha(1)} {p}$.  This completes the verification of
claim \ref{curv-jacobi-split} and hence of (\ref{SHORT.jacobi-split}).

\parit{(\ref{SHORT.jacobi-convex}).}
As in the proof of (\ref{SHORT.jacobi-split}), the angles of the 
quadrilateral ${\Box}_i^\kappa$ in $\Lob{}\kappa$ become arbitrarily close to $\pi/2$ as $i\to\infty$.   
By Reshetnyak majorization and Lemma \ref{lem:model-jacobi}, it follows that
$f_N|[0,1)$ is sinusoidally $(\kappa\cdot\ell^2)$-convex.  
\qeds

\parit{Proof of ``if''.} 
Suppose
% (\ref{SHORT.first-var}), 
 (\ref{SHORT.jacobi-split}) and (\ref{SHORT.jacobi-convex}) hold for any triangle $\trig {x}{y}{z}$ in some open ball about each point of $\spc{U}$.
We must show that $\spc{U}$ is locally $\CAT\kappa$.

Let  $\tilde \alpha$ be  the $\kappa$-development based at $\tilde x$ in $\Lob2\kappa$ of $\geodpath_{[yz]}$ with respect to $x$.  We thus obtain a comparison figure in $\Lob2\kappa$, two of whose sides, $[\tilde x\tilde y]$ and $[\bar x \bar z]$, are geodesics of the same length as $[xy]$ and $[xz]$ respectively, and whose third side is $\tilde \alpha$. Since a model triangle for $\trig {x}{y}{z}$  may be obtained by replacing  $\tilde \alpha$ with a geodesic of the same length, thereby increasing the angle at $\tilde x$,  it suffices to show \,$\mangle\hinge {x}{y}
{z}\le\mangle\hinge {\tilde x}{\tilde y}{\tilde z}$.

Let $y^\prime$ and $z^\prime$ be two points  different from $x$ and lying on $[xy]$  and $[ x z]$  respectively.  Choose $y^\prime$ and $z^\prime$  sufficiently close to $x$ that a geodesic $\tilde \beta$ joining the corresponding points $\tilde  y^\prime$ and $\tilde  z^\prime$ on $[\tilde x\tilde y]$ and $[\bar x \bar z]$  respectively remains within the subgraph of $\tilde \alpha$. 

Consider a Lipschitz line-of-sight map from $x$ to $\geodpath_{[yz]}$ as in (\ref{SHORT.jacobi-split}) and  (\ref{SHORT.jacobi-convex}). By  definition of development, there is a corresponding line-of-sight map from $\tilde x$ to $\tilde \alpha$, whose varying sight-geodesics we denote by $\tilde \gamma_t$.    Then $\tilde \beta$  intersects transversely each $\tilde \gamma_t$ at a point $\tilde \gamma_t(s)$, and $s=s(t)$ is $C^1$. It follows that the corresponding curve  $\beta(t)=\gamma_t(s(t))$   in $\spc{U}$ is Lipschitz. Therefore $\beta$ has speed almost everywhere  and its length is obtained by integrating its speed. 
%Now fix a value of $t$ at which $\beta$ is differentiable. 

By (\ref{SHORT.jacobi-split}), there is a Jacobi length $f$ along \,$\gamma_t$\, 
that arises from varying $t$ in the positive direction, and satisfies Jacobi splitting $f^2 = {f_N}^2 + {f_T}^2$.  Let $\tilde f$ be the corresponding Jacobi length.  By Lemma \ref{lem:model-jacobi}, \,$\tilde f^2 = {\tilde f_N}^2 + {\tilde f_T}^2$, where $\tilde f_N$ is a  $(\kappa\cdot\ell^2)$-sinusoid. By (\ref{SHORT.jacobi-convex}), $f_N$ is sinusoidally $(\kappa\cdot\ell^2)$-convex. By definition of development, it follows that $f_N\le\tilde f_N$, and moreover $f_T=\tilde f_T$.  Thus the speed of $\beta$ at $t$  is no greater than the corresponding speed of $\tilde \beta$.   Therefore $\length \beta\le\length\tilde \beta$, hence $\dist{y^\prime}{z^\prime}{}\le \dist{\tilde y^\prime}{\tilde z^\prime}{}$.  By definition of angle, $\mangle\hinge {x}{y}
{z}\le\mangle\hinge {\tilde x}{\tilde y}{\tilde z}$, as required.\qeds

\begin{thm}{Corollary}
Suppose  $\spc{X}$ is locally geodesic locally $\CAT{\hat\kappa}$ space for some $\hat\kappa$.
If in some open ball about each point, normal Jacobi lengths satisfy $f''+\kappa f\ge 0$, then $\spc{X}$ is locally $\CAT\kappa$.
\end{thm}

