%%!TEX root = all.tex
%array^
\chapter{Distance maps %ready
}\label{chap:dist-maps}

This chapter provides analogous results of Chapter~\ref{chap:web+bary} for $\CBB{}{}$ spaces.
The material is slightly harder mostly due to nonexistence of strongly convex functions in infinite dimensional $\CBB{}{}$ spaces.

In Section~\ref{sec:struts+rank},
we define $\kappa$-strutting point arrays 
and rank of a point.
In Section~\ref{sec:right-inverse-1},
we consider distance maps for a strutting point arrays.
For these maps we give a proof of a right-inverse theorem.
This is an important technical result which is used in the very base of theory, in particular in the proof of some basic properties of dimension.

After that we turn our attention to the finite dimensional case.
In the Section~\ref{sec:dist-chart}, we define distance chart and prove its basic properties.
In Section~\ref{sec:perelman-lemma} we prove Perelman's lemma which provides the key observation for further study of regularity questions in distance charts.
In Section~\ref{sec:dist-embedding} we show that finite dimensional Alexandrov spaces are locally embeddable into Euclidean space of some big dimension.
%???



\section{Struts and rank}\label{sec:struts+rank}

Our definitions of strut 
and distant chart 
(\ref{def:strut-I} and \ref{def:dist-chart}) 
%is the same as it was originally proposed by Burago but it CHECK???
differ from the one 
given by 
Burago, 
Gromov 
and Perelman in \cite{BGP};
it is closer to Perelman's definitons in \cite{perelman:spaces2} %CHECK??? 
and \cite{perelman:morse}%CHECK???
.

The term ``strut'' seems to have the closest meaning to the original Russian term ``rasporka''%???
 used in \cite{BGP}.
In the so called professional transaltion,
it appears under name ``burst'' 
and in the non-official authors translation it was ``strainer''.
It seems that both are not good, 
so we decided to switch to ``strut''.

\begin{thm}{Definition of struts}\label{def:strut-I}
Let $\spc{L}\in\CBB{}{}$.
We say that a point array $(a^0,a^1,\dots,a^\kay)\in\spc{L}^{{\times}(\kay+1)}$
 is \emph{$\kappa$-strutting}\index{strut} for a point $p\in\spc{L}$ if $\angk\kappa p {a^i}{a^j}>\tfrac\pi2$ for all $i\not=j$.
\end{thm} 



\begin{thm}{Definition}\label{def:rank}
Let $\spc{L}\in\CBB{}{}$ 
and $p\in \spc{L}$.
Let us define rank of $\spc{L}$ at $p$ as 
\[\rank_p=\rank_p\spc{L}
\df\pack_{\pi/2}\Sigma_p-1.\]

\end{thm}




Thus $\rank$ takes values in $\ZZ_{\ge0}\cup\{\infty\}$.


\begin{thm}{Proposition}\label{prop:stutt}
Let $\spc{L}\in\CBB{}{\kappa}$ 
and $p\in \spc{L}$.
Then the following conditions are equivalent:

\begin{subthm}{lem:stut<=>rank:rank}$\rank_p\ge\kay$
\end{subthm}

\begin{subthm}{lem:stut<=>rank:strut}
There is a point array $(a^0,a^1,\dots,a^\kay)\in\spc{L}^{{\times}(\kay+1)}$ which is $\kappa$-strutting for $p$.
\end{subthm}
\end{thm}

\parit{Proof of \ref{prop:stutt}, (\ref{SHORT.lem:stut<=>rank:strut})$\Rightarrow$(\ref{SHORT.lem:stut<=>rank:rank}).}
For each $i$,
choose a point $\acute a^i\in\Str(p)$ sufficiently close to $a^i$ (so $[p\acute a^i]$ exists for each $i$).
One can choose $\acute a^i$ on such a way that we still have
$\angk\kappa{p}{\acute a^i}{\acute a^j}>\tfrac\pi2$ for all $i\not=j$.

From hinge comparison (\ref{angle}) we have 
\[\mangle(\dir{p}{\acute a^j},\dir{p}{\acute a^j})
\ge
\angk\kappa{p}{\acute a^i}{\acute a^j}
>
\tfrac\pi2\]
for all $i\not=j$.
In particular $\pack_{\pi/2}\Sigma_p\ge \kay+1$.

\parit{(\ref{SHORT.lem:stut<=>rank:rank})$\Rightarrow$(\ref{SHORT.lem:stut<=>rank:strut}).} 
Assume $\xi^0,\xi^1,\dots,\xi^\kay$ is an array of directions in $\Sigma_p$, such that $\mangle(\xi^i,\xi^j)>\tfrac\pi2$ if $i\not=j$.

Without loss of generality, 
we may assume that each direction $\xi^i$ is geodesic;
i.e. for each $i$ there is a geodesic $\gamma^i$ in $\spc{L}$ such that $\gamma^i(0)=p$ and $\xi^i=(\gamma^i)^+(0)$.
Then the array $a^i=\gamma^i(\eps)$ for small enough $\eps>0$
satisfies (\ref{SHORT.lem:stut<=>rank:strut}).
\qeds

\begin{thm}{Corollary}\label{cor:rank>=k-open}
Let $\spc{L}\in\CBB{}{}$ and $\kay\in\ZZ_{\ge0}$.
Then the set of all points in $\spc{L}$ 
with rank $\ge \kay$ is open.
\end{thm}

\parit{Proof.} Given an array of points $\bm{a}=(a^0,a^1,\dots,a^\kay)\in \spc{L}^{{\times}(\kay+1)}$, consider 
the set $\Omega_{\bm{a}}$ of all points $p\in \spc{L}$ such that array $\bm{a}$
 is $\kappa$-strutting for a point $p$.
Cleary $\Omega_{\bm{a}}$ is open.

According to Proposition~\ref{prop:stutt}, the set of points in $\spc{L}$ 
with rank $\ge \kay$ can be presented as
\[\bigcup_{\bm{a}\in \spc{L}^{{\times}(\kay+1)}}\Omega_{\bm{a}}.\]
Hence the result.
\qeds





\section{Right-inverse mapping theorem}\label{sec:right-inverse-1}

%\parbf{Remark of A.} We might want to call it semiinverse function theorem???
% should we change b --> a^0???

\begin{thm}{Right-inverse mapping theorem}
\label{thm:right-inverse-function}
Let $\spc{L}\in\CBB{}{\kappa}$,
$p,b\in\spc{L}$ 
and $\bm{a}=(a^1,a^2,\dots,a^\kay)\in\spc{L}^{{\times}\kay}$ be a point array.

Assume that $(b,a^1,a^2,\dots,a^\kay)$ is $\kappa$-strutting for $p$.
Then the distance map $\dist{\bm{a}}{}{}\:\spc{L}\to\RR^\kay$  has a right inverse defined in a neighborhood of $\dist{\bm{a}}{p}{}\in\RR^\kay$;
i.e. there is a submap $\map\:\RR^\kay\subto\spc{L}$ such that $\Dom \map\ni \dist{\bm{a}}{p}{}$ and 
$\dist{\bm{a}}{\map(\bm{x})}{}=\bm{x}$ for any $\bm{x}\in\Dom \map$.
Moreover

\begin{subthm}{thm:right-inverse-function:Hoelder}
The map $\map$ can be chosen to be $C^{\frac{1}{2}}$-continuous (i.e. H\"older continuous with exponent $\tfrac{1}{2}$) and such that 
$\map(\dist{\bm{a}}{p}{})=p$.
\end{subthm}

\begin{subthm}{thm:right-inverse-function:open-map}
The distance map $\dist{\bm{a}}{}{}\:\spc{L}\to\RR^\kay$ is locally co-Lipschitz (in particular, open) in a neighborhood of $p$.
\end{subthm}

\end{thm}

Part \ref{SHORT.thm:right-inverse-function:open-map} of the theorem 
is closely related to \cite[Theorem 5.4]{BGP} by Burago, Gromov and Perelman, 
but the proof presented here is different; 
yet alternative proof can be build on \cite[Proposition~4.3]{lytchak:open-map} by Lytchak.

%\parbf{Remark for AKP.} It seems that the set $\Im \map$ is $\kay$-rectifiable;i.e. it lies in an image of a Lipschitz map from bounded domain of $\RR^\kay$ to $\spc{L}$. It seems I have a proof, but it is not yet clear we need it.???


\parit{Proof.} 
Fix some $\eps,r,\lambda>0$ such that the following conditions hold: 
\begin{enumerate}[(i)]
\item Each distance function $\dist{a^i}{}{}$ and $\dist{b}{}{}$ is $\tfrac\lambda2$-concave in $\oBall(p,r)$.
\item For any $q\in \oBall(p,r)$, we have $\angk{\kappa}{q}{a^i}{a^j}>\tfrac\pi2+\eps$ for all $i\not=j$ and $\angk{\kappa}{q}{b}{a^i}\z>\tfrac\pi2+\eps$ for all $i$.
In addition $\eps<\tfrac{1}{10}$.
\end{enumerate}


Given $\bm{x}=(x^1,x^2,\dots,x^\kay)\in \RR^\kay$
consider the function 
$f_{\bm{x}}\:\spc{L}\to \RR$ defined as
\[f_{\bm{x}}=\min_{i}\{h_{\bm{x}}^i\}+\eps\cdot\dist[{{}}]{b}{}{},\]
where $h_{\bm{x}}^i(q)=\min\{0,\dist{a^i}{q}{}-x^i\}$.
Note that for any $\bm{x}\in\RR^\kay$, the function $f_{\bm{x}}$ is $(1+\eps)$-Lipschitz and $\lambda$-concave  in $\oBall(p,r)$.
Denote by $\alpha_{\bm{x}}(t)$ the $f_{\bm{x}}$-gradient curve which starts at $p$.

\begin{clm}{}\label{clm:|a alpha_x|=x}
If for some $\bm{x}\in\RR^\kay$ and $t_0\le\tfrac{r}{2}$, we have
$|\dist{\bm{a}}{p}{}-\bm{x}|
\le
\tfrac{\eps^2}{10}\cdot t_0$
then 
$
\dist{\bm{a}}{\alpha_{\bm{x}}(t_0)}{}
= 
\bm{x}$.

\end{clm}

First note that Claim \ref{clm:|a alpha_x|=x} follows if for any $q\in \oBall(p,r)$, we have
\begin{enumerate}[(i)]
\item\label{111} $(\d_q\dist{a^i}{}{})(\nabla_q f_{\bm{x}})<-\tfrac{1}{10}\cdot\eps^2$ if $\dist{a^i}{q}{}>x^i$ and
\item\label{222} $(\d_q\dist{a^i}{}{})(\nabla_q f_{\bm{x}})>\tfrac{1}{10}\cdot\eps^2$ if $\dist{a^i}{q}{}-x^i=\min_j\{\dist{a^j}{q}{}-x^j\}<0$.
\end{enumerate}
Indeed, since $t_0\le\tfrac{r}2$, $\alpha_{\bm{x}}(t)\in\oBall(p,r)$ for all $t\in[0,t_0]$.
Consider the following real-to-real functions:
\[\begin{aligned}
\phi(t)
&\df
\max_{i}\{\dist{a^i}{\alpha_{\bm{x}}(t)}{}-x^i\},
&
\psi(t)
&\df
\min_{i}\{\dist{a^i}{\alpha_{\bm{x}}(t)}{}-x^i\}.
\end{aligned}\eqlbl{eq:xy-def}\]
Then from (\ref{111}), 
we have $\phi^+<-\tfrac{1}{10}\cdot\eps^2$
if $\phi>0$ and $t\in[0,t_0]$.
The same way, 
from (\ref{222}), 
we have $\psi^+>\tfrac{1}{10}\cdot\eps^2$
if $\psi<0$ and $t\in[0,t_0]$.
Since $|\dist{\bm{a}}{p}{}-\bm{x}|
\le
\tfrac{\eps^2}{10}\cdot t_0$, it follows that $\phi(0)\le \tfrac{\eps^2}{10}\cdot t_0$ and $\psi(0)\ge -\tfrac{\eps^2}{10}\cdot t_0$.
Thus $\phi(t_0)\le 0$ and $\psi(t_0)\ge 0$.
On the other hand, from \ref{eq:xy-def} we have $\phi(t_0)\ge \psi(t_0)$.
I.e., $\phi(t_0)=\psi(t_0)=0$; hence Claim~\ref{clm:|a alpha_x|=x} follows.

Thus, to prove Calim~\ref{clm:|a alpha_x|=x}, it remains to prove (\ref{111}) and (\ref{222}).
First let us prove it assuming that $\spc{L}$ is geodesic.

Note that 
\[(\d_q\dist{b}{}{})(\dir{q}{a^i})
\le\cos\angk{\kappa}{q}{b}{a^j}<-\tfrac\eps2,
\eqlbl{inq-b}\]
for all $i$ and
\[(\d_q\dist{a^j}{}{})(\dir{q}{a^i})
\le
\cos\angk{\kappa}{q}{a^i}{a^j}
<
-\tfrac\eps2\eqlbl{inq-a^j}\]
for all $j\not=i$. 
Further, \ref{inq-a^j} implies
\[(\d_q h_{\bm{x}}^j)(\dir{q}{a^i})\le 0.\eqlbl{inq-h}\]
for all $i\not=j$.
The assumption in (\ref{111}), implies
\[\d_q f_{\bm{x}}
=
\min_{j\not=i} \{\d_q h_{\bm{x}}^j\}+\eps\cdot(\d_q\dist{b}{}{}).\]
Thus,
\begin{align*}
-(\d_q\dist{a^i}{}{})(\nabla_q f_{\bm{x}})
&\ge
\<\dir q{a^i},\nabla_q f_{\bm{x}}\>
\ge
\\
&\ge
(\d_qf_{\bm{x}})(\dir q{a^i})
=
\\
&=
\min_{i\not=j}\{(\d_qh_{\bm{x}}^i)(\dir q{a^i})\}+\eps\cdot(\d_q\dist{b}{}{})(\dir q{a^i}).
\end{align*}
Therefore (\ref{111}) follows from \ref{inq-b} and \ref{inq-h}.

The assumption in (\ref{222}) implies that $f_{\bm{x}}(q)
=
h_{\bm{x}}^i(q)+\eps\cdot\dist[{{}}]{b}{}{}$ and 
\[\d_q f_{\bm{x}}\le \d_q \dist{a^i}{}{}+\eps\cdot(\d_p\dist{b}{}{}).\] 
Therefore,
\begin{align*}
(\d_q \dist{a^i}{}{})(\nabla_q f_{\bm{x}})
&\ge 
\d_qf_{\bm{x}}(\nabla_q f_{\bm{x}})
\ge 
\\
&\ge
\l[(\d_qf_{\bm{x}})(\dir qb)\r]^2
\ge
\\
&\ge
\l[\min_i\{\cos\angk\kappa qb{a^i}\}-\eps^2\r]^2.
\end{align*}
Thus, (\ref{222}) follows from \ref{inq-b} since $\eps<\tfrac{1}{10}$. 

That finishes the proof of \ref{clm:|a alpha_x|=x} in case if $\spc{L}$ is geodesic.
If $\spc{L}$ is not geodesic,
perform the above estimate in $\spc{L}^\o$, the ultrapower  of $\spc{L}$. 
(Recall that according to \ref{cor:ulara-geod}, $\spc{L}^\o$ is geodesic.)
\claimqeds

Set $t_0(\bm{x})=\tfrac{10}{\eps^2}\cdot|\dist{\bm{a}}{p}{}-\bm{x}|$; 
so we have equality in \ref{clm:|a alpha_x|=x}.
Define the submap
\[\map\:{\bm{x}}\mapsto \alpha_{\bm{x}}\circ t_0(\bm{x}),\ \ 
\Dom\map=\oBall(\dist{\bm{a}}{p}{},\tfrac{\eps^2\cdot r}{20} )\subset\RR^\kay.\]
It follows from Claim~\ref{clm:|a alpha_x|=x}, that
$\dist{\bm{a}}{\map(\bm{x})}{}=\bm{x}$ for any $\bm{x}\in\Dom\map$.

Clearly $t_0(p)=0$; thus $\map(\dist{\bm{a}}{p}{})=p$.
Further, by construction of $f_{\bm{x}}$, 
\[|f_{\bm{x}}(q)-f_{\bm{y}}(q)|\le |\bm{x}-\bm{y}|,\]
for any $q\in \spc{L}$.
Therefore, according to Lemma~\ref{lem:fg-dist-est}, $\map$ is $C^{\frac{1}{2}}$-continuous.
Thus, we  proved (\ref{SHORT.thm:right-inverse-function:Hoelder}).

Further, note that 
\[\dist{p}{\map(\bm{x})}{}
\le (1+\eps)\cdot t_0(\bm{x})
\le\tfrac{11}{\eps^2}\cdot|\dist{\bm{a}}{p}{}-\bm{x}|.\eqlbl{co-lip}\]

Note that one can repeat the above construction for any $p'\in \oBall(p,\tfrac{r}{4})$, $\eps'=\eps$ and $r'=\tfrac{r}{2}$.
The inequlality \ref{co-lip}, for the obtained map $\map'$, implies that for any $p',q \in \oBall(p,\tfrac{r}{4})$
there is $q'\in\spc{L}$ such that $\map'(q)=\map'(q')$ and  
\[\dist{p'}{q'}{}\le \tfrac{11}{\eps^2}\cdot|\dist{\bm{a}}{p'}{}-\bm{x}|.\]
I.e. the distance map $\dist{\bm{a}}{}{}$ is locally $\tfrac{11}{\eps^2}$-co-Lipschitz in $\oBall(p,\tfrac{r}{4})$.
\qeds

\section{Distance chart}\label{sec:dist-chart}

Recall that given an array 
$\bm{a}=(a^0,a^1,\dots,a^\kay)$ 
we denote by $\bm{a}^{\ne 0}$ the subarray of $\bm{a}$ with $a^0$ removed;
i.e.,
\[\bm{a}^{\ne 0}=(a^1,a^2,\dots,a^\kay).\]

\begin{thm}{Inverse function theorem}\label{thm:inverse-function}
Let $\spc{L}\in\CBB{m}{\kappa}$ 
and $p,a^0,a^1,a^2,\dots,a^m\in\spc{L}$.

Assume that the point array $\bm{a}=(a^0,a^1,\dots,a^m)$ is $\kappa$-strutting for $p$.
Then
there are $R>0$ and $\eps>0$ such that

\begin{subthm}{thm:inverse-function:strut}
For all $i\not=j$ and any $q\in\oBall(p,R)$ we have
\[\angk\kappa{q}{a^i}{a^j}>\tfrac{\pi}{2}+\eps.\]

\end{subthm}

\begin{subthm}{thm:inverse-function:chart}
The restriction of the distance map 
$\dist{\bm{a}^{\without 0}}{}{}$ to the ball $\oBall(p,R)$
is an open $[\eps,\sqrt{m}]$-bi-Lipschitz embedding $\oBall(p,R)\hookrightarrow\RR^m$.
\end{subthm}

\begin{subthm}{thm:inverse-function:R}
The value $R$ depend only on the following data:  
\[\kappa,\  \dist{p}{a^i}{},\ \dist{a^i}{a^j}{}\]
for all $i$ and $j$.
\end{subthm}

\end{thm}

\begin{thm}{Definition}\label{def:dist-chart}
Let $\spc{L}\in\CBB{m}{\kappa}$.
If a point array $a^0,a^1,a^2,\dots,a^m$ 
and the value $R$ satisfy the conditions in Theorem~\ref{thm:inverse-function}, 
then the restriction 
$\bm{x}=\dist{\bm{a}^{\without 0}}{}{}|\oBall(p,R)$
is called a \emph{distant chart}\index{distant chart},
the restrictions $x^i=\dist{a^i}{}{}|\oBall(p,R)$ are called \emph{coordinates}\index{coordinate of the distant chart}
and the restriction $y=\dist{a^0}{}{}|\oBall(p,R)$ is called \emph{strut}\index{strut of the distant chart} of the distant chart.
\end{thm}

The proof of Theorem~\ref{thm:inverse-function} will reqire the following lemma.

\begin{thm}{Lemma}\label{lem:pack(S^m)+}
Let $\spc{L}\in\CBB{m}{\kappa}$ and $p\in\spc{L}$.
Assume for the directions $\xi,\zeta^1,\zeta^2,\dots,\zeta^\kay\in\Sigma_p$ the following conditions hold 

\begin{subthm}{}
$\mangle(\xi,\zeta^i)>\tfrac\pi2-\eps$ for all $i$,
\end{subthm}

\begin{subthm}{}
$\mangle(\zeta^i,\zeta^j)>\tfrac\pi2+\eps$ for all $i\not=j$.
\end{subthm}
Then $\kay\le m$.
\end{thm}

\parit{Proof.}
Without loss of generality, we can assume that all $\xi,\zeta^1,\zeta^2,\dots,\zeta^\kay$ are geodesic directions;
let $\xi=\dir{p}{x}$ and $\zeta^i=\dir{p}{z^i}$ for all $i$.
Fix small $r>0$, 
let $\bar x\in \l]px\r]$ and $\bar z^i\in\l]p z^i\r]$ be points 
such that $\dist{p}{\bar x}{}=\dist{p}{\bar z^1}{}=\dots=\dist{p}{\bar z^\kay}{}=r$.
From the definition of angle,
if $r$ is small enough, we have
\begin{itemize}
\item $\angk\kappa{p}{\bar x}{\bar z^i}>\tfrac\pi2-\eps$ for all $i$
and $\angk\kappa{p}{\bar z^i}{\bar z^j}>\tfrac\pi2+\eps$ for all $i\not=j$.
\end{itemize}
Choose a point $p'\in\Str(\bar x,\bar z^1,\bar z^2,\dots,\bar z^\kay)$ sufficiently close to $p$,
so that for $p'$, the above conditions still hold; i.e.,
\begin{clm}{}
 $\angk\kappa{p'}{\bar x}{\bar z^i}>\tfrac\pi2-\eps$ for all $i$ and $\angk\kappa{p'}{\bar z^i}{\bar z^j}>\tfrac\pi2+\eps$ for all $i\not=j$.
\end{clm}
Set $\acute\xi=\dir{p'}{\bar x}$ and $\acute\zeta^i=\dir{p'}{\bar z^i}$ for each $i$.
By hinge comparison (\ref{angle}), 
\begin{clm}{}
$\mangle(\acute\xi,\acute\zeta^i)>\tfrac\pi2-\eps$ for all $i$ and $\mangle(\acute\zeta^i,\acute\zeta^j)>\tfrac\pi2+\eps$ for all $i\not=j$.
\end{clm}

According to Corollary~\ref{cor:euclid-subcone}, all directions $\acute\xi,\acute\zeta^1,\acute\zeta^2,\dots,\acute\zeta^\kay$ lie in an isometric copy of standard $n$-sphere in $\Sigma_{p'}$ and clearly $n\le m-1$.
Thus, it  remains to prove the following claim, which is a partial case of the lemma.

\begin{clm}{}
If $\xi,\zeta^1,\zeta^2,\dots,\zeta^\kay\in\SS^{m-1}$,
$\dist{\xi}{\zeta^i}{}>\tfrac\pi2-\eps$ for all $i$ and
$\dist{\zeta^i}{\zeta^j}{}>\tfrac\pi2+\eps$ for all $i\not=j$,
then $\kay\le m$.
\end{clm}

For each $i$, 
set 
$\bar\zeta^i$
to be the closest point to $\zeta^i$
in
$\Xi=\SS^{m-1}\backslash \oBall(\xi,\tfrac\pi2)
\iso
\SS^{m-1}_+$ 
(if $\zeta\in\Xi$ then $\bar\zeta^i=\zeta^i$).
By straightforward calculations, we get
\[\dist{\bar\zeta^i}{\bar\zeta^j}{}\ge \dist{\zeta^i}{\zeta^j}{}-\eps>\tfrac\pi2.\]
Thus, it is enough to show the following claim.

\begin{clm}{}
$\pack_{\frac\pi2}\SS^{m-1}_+= m.$
\end{clm}

Clearly, $\pack_{\frac\pi2}\SS^{m-1}_+\ge m$.

The opposite inequality is proved by  induction on $m$.
The base case $m=1$ is obvious. 
Assume $\bar\zeta^1,\bar\zeta^2,\dots,\bar\zeta^{\kay}$ is an array of points in $\SS^{m-1}_+$ with $\dist{\bar\zeta^i}{\bar\zeta^j}{}>\tfrac\pi2$.
Without loss of generality we can also assume that $\bar\zeta^\kay\in\partial \SS^{m-1}_+$.
For each $i<\kay$, 
set $\check\zeta^i
=\dir{\bar\zeta^\kay}{\bar\zeta^i}\in\Sigma_{\bar\zeta^\kay}\SS^{m-1}_+\iso\SS^{m-2}_+$.
 By hinge comparison (\ref{angle}), $\mangle(\check\zeta^i,\check\zeta^j)>\tfrac\pi2$ 
for all $i<j<\kay$.
Thus, from the induction hypothesis we have that $\kay-1\le {m-1}$.
\qeds

\parit{Proof of \ref{thm:inverse-function}; (\ref{SHORT.thm:inverse-function:strut}).}  
Fix $\eps>0$ such that 
$\angk\kappa p{a^i}{a^j}>\tfrac\pi2+\eps$ for all $i\not=j$.
Choose sufficiently small $R>0$ so that 
$\angk\kappa q{a^i}{a^j}>\tfrac\pi2+\eps$ for all $i\not=j$ and any $q\in\oBall(p,R)$.
Clearly, (\ref{SHORT.thm:inverse-function:strut}) holds for $\oBall(p,R)$.

\parit{(\ref{SHORT.thm:inverse-function:chart}).}
Note that distance map $\dist{\bm{a}^{\without 0}}{}{}$ is Lipschitz
and its restriction $\dist{\bm{a}^{\without 0}}{}{}|\oBall(p,R)$ is open;
the later follows from the open map theorem (\ref{thm:right-inverse-function:open-map}).
Thus, to prove (\ref{SHORT.thm:inverse-function:chart}), it is sufficient to show that
\[
\max_{i>0}\{\,|\dist{a^i}{x}{}-\dist{a^i}{y}{}|\,\}
>
\tfrac\eps2\cdot\dist[{{}}]{x}{y}{}
\eqlbl{expend}
\]
for any $x,y\in {\oBall(p,R)}$.

According to Lemma~\ref{lem:pack(S^m)+}, 
\[
\mangle\hinge{x}y{a^i}
\le
\tfrac\pi2-\eps
\ \ \t{for some}\ \ i.
\]
Since $\dist{x}{y}{}<2\cdot R$ and $R$ is small, 
the hinge comparison (\ref{angle}) implies 
\[
\dist{a^i}{x}{}-\dist{a^i}{y}{}>\tfrac\eps2\cdot\dist[{{}}]{x}{y}{}
\ \ \t{for some}\ \ i.
\eqlbl{eq:y-x}\]
Switching $x$ and $y$, we get
\[
\dist{a^j}{y}{}-\dist{a^j}{x}{}
>
\tfrac\eps2\cdot\dist[{{}}]{x}{y}{}
\ \ \t{for some}\ \ j.
\eqlbl{eq:x-y}\] 
Clearly $i\not=j$,
therefore \ref{eq:y-x} and \ref{eq:x-y} implies \ref{expend}.

Finally, part (\ref{SHORT.thm:inverse-function:R})
follows since the angle $\angk\kappa q{a^i}{a^j}$ 
is a continuous function from the following 4 numbers $\kappa$, $\dist{q}{a^i}{}$, $\dist{q}{a^j}{}$ and $\dist{a^i}{a^j}{}$.
\qeds

The next proposition follows directly from Proposition~\ref{prop:stutt}.

\begin{thm}{Proposition} 
Let $\spc{L}\in\CBB{m}{}$.
A point $p\in \spc{L}$ belongs to some distance chart if and only if $\rank_p=m$.
\end{thm}

The following proposition follows from the definition of differential (\ref{def:differential}), Theorem~\ref{thm:tan4finite}, Theorem~\ref{thm:differential-of-dist} and Theorem~\ref{thm:inverse-function}.%???Does it really follows???

\begin{thm}{Proposition}\label{thm:inverse-function:differential}
Let $\spc{L}\in\CBB{m}{}$ and $\bm{x}\:\spc{L}\to\RR^m$ be a distance chart.
Then for each $q\in\Dom\bm{x}$, 
the differential $\d_q\bm{x}\:\T_q\to\RR^m$ is a bi-Lipschitz homeomorphism.
\end{thm}

\parbf{Liftings.}
Let $\spc{L}\in \CBB{m}{\kappa}$
and $\bm{x}\:\spc{L}\subto\RR^m$ 
be a distance chart with strut $y\:\spc{L}\subto\RR$.
Assume $\bm{x}=\dist{\bm{a}^{\without 0}}{}{}|\oBall(p,R)$
and $y\z=\dist{a^0}{}{}|\oBall(p,R)$ 
and a point array $\bm{a}=(a^0,a^1,\dots,a^m)$ in $\spc{L}$.

Assume we have a convergence $\spc{L}_n\GHto\spc{L}$ with
$\spc{L}_n\in \CBB{m}{\kappa}$.
Choose arbitrary liftings $a^i_n$ and $p_n$ in $\spc{L}_n$ of $a^i$ and $p$. 
If $R>0$ as in Inverse function theorem (\ref{thm:inverse-function}) then for each large enough $n$ the restriction
 $\bm{x}_n=\dist{\bm{a}^{\without 0}_n}{}{}|\oBall(p_n,R)$
is a distance chart 
with strut $y_n=\dist{a^0_n}{}{}|\oBall(p_n,R)$.


The constructed charts $\bm{x}_n\:\spc{L}_n\subto\RR^m$ with struts $y_n\:\spc{L}_n\subto\RR$ 
will be called 
\emph{liftings of the distance chart}%
\index{liftings of the distance chart}  
$\bm{x}\:\spc{L}\subto\RR^m$ with strutt $y\:\spc{L}\subto\RR$.

Note that a lifting forms a continuous submap 
$\bm{x}\:\bm{\spc{L}}\subto\RR^m$ and $\bm{y}\:\bm{\spc{L}}\to\RR$, where the space
$$\bm{\spc{L}}=\spc{L}\sqcup\l(\bigsqcup_{i\in\NN}\spc{L}_i\r),$$
is equipped with the topology of the convergence $\spc{L}_n\GHto\spc{L}$.

%%%%%%%%%%%%%%%%%%%%%%%%%%%%%%%%%%%%%%%%%%%%%%%%%%%%%%%%%%%%%%%%%%%%%

\section{Perelman's lemma}\label{sec:perelman-lemma}
%Should we name it Kleiner--Perelman???

The following lemma is analogous to Theorem~\ref{thm:web:Up-convex}.

\begin{thm}{Perelman's lemma}\label{thm:inverse-function:concave}
Let $\spc{L}\in\CBB{m}{\kappa}$,
$\bm{x}=(x^1,x^2,\dots,x^m)\:\spc{L}\subto \RR^m$ be a distance  chart with strut $y\:\spc{L}\subto \RR$.
Then the subfunction 
\[f=y\circ\bm{x}^{-1}\:\RR^m\subto \RR\] 
is a semiconcave and locally Lipschitz.
\end{thm}

\parit{Proof of \ref{thm:inverse-function:concave}.}
By the inverse fuction theorem (\ref{thm:inverse-function})
(which is practically the definition of distance chart),
it is sufficient to show, that given $z\in\Dom\bm{x}$ there is $r>0$ and $\Lambda\in\RR$,
such that for any $p$, $q\in\oBall(z,r)$ and $t\in[0,1]$, we have
\[f\circ\bm{\alpha}(t)
\ge 
t\cdot y(p)
+(1-t)\cdot y(q)
-\Lambda\cdot\tfrac{t\cdot(1-t)}{2}\cdot\ell^2,
\eqlbl{eq:f_is_concave}\]
where  
\begin{itemize}
\item $\ell=|\bm{x}(p)-\bm{x}(q)|$,
\item $\alpha^i(t)=(1-t)\cdot x^i(p)+t\cdot x^i(q)$ and 
\item $\bm{\alpha}(t)=(\alpha^1(t),\alpha^2(t),\dots,\alpha^m(t))\in \RR^m$.
\end{itemize}
By passing to a restriction of $\bm{x}$, if nesessury, 
we can assume that 
\begin{itemize}
\item $\Im\bm{x}$ is convex,
\item for some fixed $\lambda\ge 0$, the function $y$ 
and each $x^i$ are $\lambda$-concave in $\Dom\bm{x}$.
\item there is $\eps>0$ such that 
$\bm{x}$ is an open $[\eps,m]$-bi-Lipshitz embedding (this follows from \ref{thm:inverse-function:chart}). 
Moreover $\d_p\bm{x}\:\T_p\to\RR^m$ is $[\eps,m]$-bi-Lipshitz homeomorphism for any $p\in \Dom\bm{x}$ (this follows from \ref{thm:inverse-function:differential}).
\end{itemize}
Fix $r>0$ so that $\oBall(z,2\cdot r)\subset\Dom\bm{x}$.
Since $p$, $q\in\oBall(z,r)$, 
we have $[pq]\subset\Dom\bm{x}$.
Set 
\[\bm{\beta}(t)
=
(\beta^1(t),\beta^2(t),\dots,\beta^m(t))
=
\bm{x}\circ\geodpath_{[pq]}(t).\]
Then
\[\begin{aligned}
\beta^i(t)&= 
x^i\circ\geodpath_{[pq]}(t)
\ge
\\
&\ge
t\cdot x^i(p)
+
(1-t)\cdot x^i(q)
+
\lambda\cdot\tfrac{t\cdot(1-t)}2\cdot\dist[2]{p}{q}{}
\ge
\\
&
\ge\alpha^i(t)
+
\tfrac\lambda{\eps^2}\cdot\tfrac{t\cdot(1-t)}2\cdot\ell^2
\end{aligned}
\eqlbl{eq:beta>=alpha}\]
for each $i$.
Anaogousely,
\[\begin{aligned}
f\circ\bm{\beta}(t)&=y\circ\geodpath_{[pq]}(t)
\ge
\\
&\ge
t\cdot y(p)
+(1-t)\cdot y(q)
+\tfrac\lambda{\eps^2}\cdot\tfrac{t\cdot(1-t)}2\cdot\ell^2.
\end{aligned}\eqlbl{eq:f(beta)>=}\]

Using the partial order $\succcurlyeq$ on $\RR^m$ which is defined in \ref{def:ordung}, 
all inequalities in \ref{eq:beta>=alpha} can be rewritten as 
\[\bm{\beta}(t)
-
\tfrac\lambda{\eps^2}\cdot\tfrac{t\cdot(1-t)}2\cdot\ell^2\cdot\bm{1}
\succcurlyeq 
\bm{\alpha}(t).
\eqlbl{beta>=alpha}\]
where $\bm{1}\df(1,1,\dots,1)\in\RR^m$.

Now let us prove the following claim.

\begin{clm}{}\label{partial f} The function $f$ is decreasing in each argument; 
i.e.
\[\bm{v}\succcurlyeq\bm{w}\ \ \Rightarrow\ \ f(\bm{v})\le f(\bm{w})\]
for any $\bm{v},\bm{w}\in \Im\bm{x}$.
\end{clm}

Let $b,a^1,a^2,\dots,a^m\in\spc{L}$, be as in the definition of distant chart (\ref {def:dist-chart}).
Clearly,
\begin{itemize}
\item $\d_p y(u)\le-\<\dir{p}{b},u\>$ and $\d_p x^i(u)\le-\<\dir{p}{a^i},u\>$ for all $i$ and
\item $\mangle\hinge{p}{a^i}{a^j}$, $\mangle\hinge{p}{b}{a^j}>\tfrac{\pi}{2}$ for all $i,j$.
\end{itemize}
It follows from Lemma~\ref{lem:pack(S^m)+},
that
\[\t{if}\ \ \d_p x^i(u)\ge 0\ \ \t{for each}\ \ i\  \ \t{then}\ \ \d_p y(u)\le 0.
\eqlbl{1-st order}\]
for any $p\in\Dom\bm{x}$ and $u\in\T_p$.
Consider
\begin{itemize}
\item function $\phi\:[0,1]\to\RR$ defined as $\phi(t)=f[(1-t)\cdot\bm{v}+t\cdot\bm{w}]$ and 
\item path $\sigma\:[0,1]\to\Dom\bm{x}$ defined as $\sigma(t)=\bm{x}^{-1}[(1-t)\cdot\bm{v}+t\cdot\bm{w}]$
(it is defined since $\Im\bm{x}$ is convex).
\end{itemize}
Clearly,
$\sigma^+(t)=(\d_{\sigma(t)}\bm{x})^{-1}(\bm{w}-\bm{v})$;
thus
 \[(\d_{\sigma(t)}x^i)(\sigma^+(t))=w^i-v^i\ge 0.\]
From \ref{1-st order}, we have $(\d_{\sigma(t)}y)(\sigma^+(t))\le 0$.
Therefore 
\begin{align*}
f(\bm{w})-f(\bm{v})
&=\int\limits_0^1\phi^+(t)\cdot\d t=
\\
&=\int\limits_0^1(\d_{\phi(t)}y)(\sigma^+(t))\cdot\d t
\le 0.
\end{align*}
\claimqedsf

Note that $y$ is 1-Lipschitz.
As it is stated above, 
$\bm{x}$ is $[\eps,m]$-bi-Lipschitz.
Thus, $f=y\circ\bm{x}^{-1}$ is $\tfrac1\eps$-Lipschitz.
Applying this 
together with \ref{beta>=alpha}, 
\ref{eq:f(beta)>=}
and Claim~\ref{partial f},
we get 
\begin{align*}
f\circ\bm{\alpha}(t)
&\ge 
f
\l[
\bm{\beta}(t)
-
\tfrac\lambda{\eps^2}\cdot\tfrac{t\cdot(1-t)}2\cdot\ell^2\cdot\bm{1}
\r]
\ge
\\
&
\ge
f\circ\bm{\beta}(t)
+
\tfrac{\lambda
\cdot 
\sqrt{m}}{\eps^3}
\cdot
\tfrac{t\cdot(1-t)}2
\cdot
\ell^2
\ge
\\
&
\ge
t\cdot y(p)
+(1-t)\cdot y(q)+
\lambda\cdot\l(\tfrac{\sqrt{m}}{\eps^3}+\tfrac1{\eps^2}\r)
\cdot 
\tfrac{t\cdot(1-t)}2
\cdot
\ell^2.
\end{align*}
I.e. \ref{eq:f_is_concave} holds for $\Lambda=\lambda\cdot\l(\tfrac{\sqrt{m}}{\eps^3}+\tfrac1{\eps^2}\r)$.
\qeds

\section{Distance embedding}\label{sec:dist-embedding}


\begin{thm}{Theorem}\label{thm:dist-emb}
Let $\spc{L}\in\CBB{m}{\kappa}$, $z\in\spc{L}$ and $R>0$.
Then there is $\eps>0$, 
$\kay\in\NN$ 
and a point array $\bm{a}=(a^1,a^2,\dots,a^\kay)$ in $\spc{L}$
such that the distance map $\dist{\bm{a}}{}{}\:\spc{L}\to\RR^\kay$ gives an embedding $\oBall(z,R)\hookrightarrow\RR^\kay$, 
which is $[\eps,\kay]$-bi-Lipschitz.

Moreover, the values $\eps$, $\kay$ and the point array $\bm{a}$ can be chosen so that:

\begin{subthm}{thm:dist-emb:close.to.z}
All points $a^i$ lie arbitrary close to $z$;
say, given $r>0$ we can assume that $a^i\in\oBall(r,z)$ for each $i$.
\end{subthm}

\begin{subthm}{thm:dist-emb:L_n-->L}
Given a Gromov--Hausdorff convegence
$\spc{L}_n\GHto\spc{L}$ with $\spc{L}_n\in\CBB{m}{\kappa}$,
for all large enough $n$
the liftings $\dist{\bm{a}_n}{}{}\:\spc{L}_n\to\RR$ of the distance map $\dist{\bm{a}}{}{}$ 
give an $[\eps,\kay]$-bi-Lipschitz embedding of $\oBall(z_n,R)\subset \spc{L}_n$ for lifting $z_n\in\spc{L}_n$ of $z$.
\end{subthm}
\end{thm}

As a corollary, we obtain the following usefull technical statememnt.
The original proof of this lemma was build on volume comparison %???
 for a map $p\mapsto \ddir{x}{p}$ with arbitrary choice of geodesic $p\mapsto [xp]$;
compare \cite[Lemma~1.3]{grove-petersen:finiteness}.

\begin{thm}{Obtuse-angle lemma} \label{lem:tuda-suda} Given $v>0$, $r>0$,
$\kappa\in\RR$ and $m\in\NN$, there is $\eps=\eps(v,r,\kappa,m)>0$ such that if
$\spc{L}\in\CBB{m}{\kappa}$, $p\in \spc{L}$, $\vol_m \oBall(p,r)>v$, then for any two points
$x,y\in \oBall(p,r)$, $\dist{x}{y}{}<\eps$ there is point $z\in \oBall(p,r)$ such that 
\[\mangle\hinge x z y>\tfrac\pi2+\eps\ \ \text{or}\ \  \mangle\hinge y z x>\tfrac\pi2+\eps.\]
\end{thm}

\parit{Proof of \ref{thm:dist-emb}.}
Without loss of generality, we may assume that $\kappa=-1$;
i.e. $\spc{L}\in\CBB{m}{-1}$.

\begin{wrapfigure}[8]{r}{30mm}
\begin{lpic}[t(-5mm),b(0mm),r(0mm),l(0mm)]{pics/gradient-bisector-triangle(0.4)}
\lbl[br]{21,31;$x$}
\lbl[tl]{43,22;$y$}
\lbl[rb]{44,63,67;{\small $\GradBis(\ $}}
\lbl[b]{44,63,53;{\small $R$}}
\lbl[lb]{44,63,27;{\small $\ ;x,y)$}}
\lbl[]{55,45;$\spc{L}$}
\end{lpic}
\end{wrapfigure}

Assume 
$x,y\in \spc{L}$ be a pair of distinct points.
Let us construct a subset $\RB{}(R;x,y)\subset\spc{L}$, 
which will be called  \emph{radial bisector of radius $R$ between $x$ and $y$}.
To do this, first consider the function $f\z=\min\{\dist{x}{}{},\dist{y}{}{}\}$.
We say that $p\in\RB(R;x,y)$ if for some $0<s_0\le s_1\le R$ 
there is a point $q\in \spc{L}$ such that $s_0=\dist{x}{q}{}=\dist{y}{q}{}$ 
and $p=\sigma(s_1)$ where $\sigma$ is an $f$-radial curve such that $\sigma(s_0)=q$.



\parbf{Remark.}
Clearly $\RB(R;x,y)$ contains all point $p$ such that $\dist{p}{x}{}\z=\dist{p}{y}{}\le R$,
but in general this set is bigger.
For example, assume that space $\spc{L}$ is isometric to a solid plane triangle, then,
the set $\RB(R;x,y)$ for some $x,y\in\spc{L}$ might contain segments on the boundary of $\spc{L}$ (as on the figure).


\begin{clm}{}\label{clm:GB}
Given a pair of distinct points $x,y\in \spc{L}$, 
there is a $(\sinh R)$-Lipschitz map $\map\:\SS^{m-1}\to\spc{L}$ such that 
$\Im\map_R\supset\RB(R;x,y)$.
\end{clm}

\parit{Proof of the claim.}
Set $f=\min\{\dist{x}{}{},\dist{y}{}{}\}$ as above.

Let us denote by $\sigma_w$ the $f$-radial curve for curvature $-1$ initiated at $w\in\spc{L}\backslash\{x,y\}$.
Given $\xi\in\Sigma_x$ consider sequence of points $w_n\to x$ such that $\dir{x}{w_n}\to\xi$.
Applying \ref{gen-rad-comp} for $A=\{x,y\}$, we get that $\sigma_{w_n}$ convege pointwise to a radial curve $\sigma_\xi$ such that $\sigma_\xi(0)=x$ and $\sigma^+_\xi(0)=\xi$.
Moreover, it follows that if $\xi^1,\xi^2\in\Sigma_x$ and $\mangle(\xi^1,\xi^2)=\phi$ then $\dist{\sigma_{\xi^1}(t_1)}{\sigma_{\xi^2}(t_2)}{}\le \side{-1} \{\phi,t_1,t_2\}$.
In particular, the map $\map_R\:\xi\mapsto\sigma_\xi(R)$ is a $(\sinh R)$-Lipschitz map from $\Sigma_x$ to $\spc{L}$.

Let us show that $\Im\map_R\supset \RB(R;x,y)$.

First note that
if $\dist{x}{q}{}=\dist{y}{q}{}=s_0$ then $q\in\Im\map_{s}$ for any $s\ge s_0$.
Indeed,  $\Im\map_s$ separates %???WHY
 $x$ and $y$ in $\spc{L}$.
Thus, for any choice of geodesics $[q x]$ and $[q y]$, 
the set $\Im\map_s$ contains at least one point on $[x q]\cup[q y]$.
However, it can not contain any point except $q$.
Indeed, assume $\map_s(\xi)= w\in]xq[$
then according to \ref{prop:grad-like-unique-past}, $\sigma_\xi\l([0,s]\r)=[xw]$; 
consequently $s<s_0$, a contradiction.


Now assume $p\in \RB(R;x,y)$.
Let $s_0\le s_1\le R$ and $q\in\spc{L}$ be as in the construction of radial bisector.
From above, we have that for any $s\ge s_0$ there is $\xi(s)\in\Sigma_x$ such that $q=\sigma_{\xi(s)}(s)$.
Note that the $\sigma_{\xi(s)}\l([s,\infty)\r)=\sigma_q\l([s_0,\infty)\r)$ 
for any $s\ge s_0$.
Further note, that the map $\tau\:s\mapsto \sigma_{\xi(s)}(R)$ is continuous%
\footnote{Note however that the map $s\mapsto\xi(s)$ might be non-continuous.}
for $s\in[s_0,R]$.
Clearly, $\tau(R)=q$ and $\tau(s_0)=\sigma(R)$.
Thus, the curve $\tau|[s_0,R]$ runs along $\sigma|_{[s_0,R]}$ in opposite direction.
In particular, for some $s\in[s_0,R]$, we have
$p=\sigma_{\xi(s)}(R)$ or $p\in\Im\map_R$.

This finishes the proof of the claim in case $x$ is regular; i.e. $\Sigma_x\iso\SS^{m-1}$.

To prove the general case,
choose a sequence of regular points $x_n\to x$ (the existence of this sequence follows from \ref{LinDim+-f}).
Applying the above argument, we get the existence of a sequence of $(\sinh R)$-Lipschitz maps $\map_{R;n}\:\SS^{m-1}\to\spc{L}$ such that 
$\Im\map_{R;n}
\supset
\RB(R;x_n,y)$.
The sets $\RB(R;x_n,y)$ converges 
in the Hausdorff sense to $\RB(R;x,y)$. %???WHY???
Thus one can take $\map_R$ to be a partial limit of $\map_{R;n}$ as $n\to\infty$.
\claimqeds


Note that for any choice of $\bm{a}\in\spc{L}^{{\times}\kay}$,
the distance map $\dist{\bm{a}}{}{}\:\spc{L}\to\RR^\kay$ is $\kay$-Lipschitz.
Thus it remains to show that one can choose $\kay$, $\bm{a}$ and $\eps>0$ on such a way that 
\[|\dist{a^i}{x}{}-\dist{a^i}{y}{}|>\eps\cdot\dist[{{}}]{x}{y}{}
\ \ \t{for at least one}\ \ i
\eqlbl{eq:bilip}\]
and any $x,y\in\oBall(z,R)$.

\begin{wrapfigure}{l}{22mm}
\begin{lpic}[t(0mm),b(0mm),r(0mm),l(0mm)]{pics/pp-1(0.45)}
\lbl[rt]{10,3;$x$}
\lbl[lt]{33,3;$y$}
\lbl[l]{42,51;$p$}
\lbl[r]{25,60;$\check p$}
\lbl[b]{20,46,90;{\small $\sigma(s)$}}
\lbl[t]{21,31;$q$}
\lbl[l]{4,30;$\spc{L}$}
\end{lpic}
\end{wrapfigure}

Assume we want to choose all $a^i$ in $\oBall(z,r)$ for some $0<r\le R$;
that is only needed to prove (\ref{SHORT.thm:dist-emb:close.to.z}).

For small enough $\delta>0$,
choose $\bm{a}$ so that the set $\{a^1,a_2,\dots,a^\kay\}$ 
forms a \emph{maximal $\delta$-packing}\index{maximal $\eps$-packing} in $\oBall(z,r)$;
i.e. $\kay=\pack_\delta\oBall(z,r)$ and $\dist{a^i}{a^j}{}>\delta$ for all $i\not=j$.
Note that according to \ref{pack}, we have $\kay>\Const/\delta^m$ for some $\Const>0$.
Thus, we can assume that
\[\kay>\pack_{\delta_1}\SS^{m-1},
\ \ \t{where}\ \ 
\delta_1=\frac{\delta}{2\cdot\sinh (2\cdot R)}.
\eqlbl{delta-def}\]


For fixed $x,y\in\spc{L}$ and arbitrary point $p\in\spc{L}$, 
let us describe a construction of a new point $\check p$ on gradient bisector of $x$ and $y$.
Assume $\dist{x}{p}{}\ge\dist{y}{p}{}$; 
if not --- switch $x$ and $y$ in all the following construction.
Choose a point $q\in [x p]$ such that $\dist{x}{q}{}=\dist{y}{q}{}$.
Set $s_0=\dist{x}{q}{}=\dist{y}{q}{}$ and $s_1=\dist{x}{p}{}$.
As above, set $f=\min\{\dist{x}{}{},\dist{y}{}{}\}$.
Let $\sigma\:[s_0,\infty)\to\spc{L}$ be an $f$-radial 
curve which starts at $q$.
Then set $\check p=\sigma(s_1)$.
Clearly, $s_1\le 2\cdot R$; hence $\check p\in \RB(2\cdot R;x,y)$.

\begin{wrapfigure}{r}{22mm}
\begin{lpic}[t(0mm),b(0mm),r(0mm),l(0mm)]{pics/pp-1-model(0.45)}
\lbl[rt]{10,5;$\~x$}
\lbl[lt]{33,5;$\~y$}
\lbl[rb]{42,52;$\~p$}
\lbl[lb]{3,52;$\~{\check p}$}
\lbl[l]{24,19;$\~q$}
\lbl[l]{40,17;$\Lob2{-1}$}
\end{lpic}
\end{wrapfigure}

%???\~{\check p} is not a good thing!!!

Now let us construct corresponding model configuration.
Consider a model triangle $\trig{\~x}{\~y}{\~p}=\modtrig{-1}(x y p)$ in $\Lob2{-1}$.
Let $\~{\check p}$ be the reflection of $\~p$ in the bisecting perpendicular to $[\~x\~y]$
and $\~q$ lie on the intersection of $[\~x\~p]$ and $[\~y\~{\check p}]$.
From ??? comparison, 
$\dist{\~y}{\~q}{}\le s_0$
and clearly $\dist{\~y}{\~q}{}+\dist{\~q}{\~p}{}=s_1=\dist{y}{q}{}+\dist{q}{p}{}$.
It follows that $\mangle\hinge{\~y}{\~p}{\~q}\le\angk{-1}{y}{p}{q}$.
Thus, from radial comparison (\ref{gen-rad-comp}) 
we have 
\[\dist{p}{\check p}{}\le\dist{\~p}{\~{\check p}}{}.
\eqlbl{eq:|p p'|}\]

Assume 
$x,y\in\Omega$, 
$p\in\oBall(z,r)$,
and $|\dist{p}{x}{}-\dist{p}{y}{}|\le\eps\cdot\dist[{{}}]{x}{y}{}$
for small enough $\eps>0$.
Clearly $\dist{x}{y}{}$, $\dist{x}{p}{}$, $\dist{y}{p}{}\le 2\cdot R$;
thus $\check p\in\RB(2\cdot R;x,y)$.
Direct calculations show that 
$\dist{\~p}{\~{\check p}}{}<\sqrt{\eps}$.
Inequality \ref{eq:|p p'|} implies that we can fix $\eps=\eps(\delta,\RR)>0$ so small that 
\[\dist{p}{\check p}{}<\sqrt{\eps}<\tfrac\delta 4.
\eqlbl{eps-def}\]

Now assume \ref{eq:bilip} does not hold; i.e.
\[|\dist{a^i}{x}{}-\dist{a^i}{y}{}|\le\eps\cdot\dist[{{}}]{x}{y}{}
\ \ \t{for each}\ \ i.
\]
Then for each $i$ we have $\dist{a^i}{\check a^i}{}<\sqrt{\eps}$.
Therefore $\dist{\check a^i}{\check a^j}{}>\delta-2\cdot\sqrt{\eps}>\tfrac\delta2$.
Applying Claim~\ref{clm:GB}, we get a contradiction with \ref{delta-def}.

To prove (\ref{SHORT.thm:dist-emb:L_n-->L}),
we should notice that all the constructions above  survive under noncollapsing Gromov--Hausdorff convergence. %???
\qeds

\section{Exercises}

\begin{thm}{Exercise}\label{ex:lip+dist}
Let $\spc{L}\in\CBB{m}{}$.
Then any Lipschitz 
%(???DC, DMD???) 
function $f$ with compact support in  $\spc{L}$ can be presented as a composition
\[f=\phi\circ\dist{\bm{a}}{}{},\]
where $\bm{a}=(a^1,a^2,\dots,a^\kay)$ is an array of points in $\spc{L}$ and $\phi$ be a Lipschitz function with compact support in $\RR^\kay$.
\end{thm}

