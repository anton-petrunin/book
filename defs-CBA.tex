%%%!TEX root = the-defs-CBA.tex
%%%%arXiv

\chapter{Definitions of curvature bounded above}
Section~\ref{sec:cba-def} defines $\CAT\kappa$ spaces.  
Section~\ref{sec:cat-angles} gives definitions in terms of angle and distance comparisons for triangles.  
Section~\ref{sec:thin-triangle} looks at thin triangles and their inheritance lemma. 
Section~\ref{sec:func-comp} gives a useful definition in terms of convexity of distance functions.

Section~\ref{sec:convex-CBA} discusses convexity in  $\CAT{\kappa}$ spaces.
In Section~\ref{sec:development-CBA}, we discuss a definition via developments of  curves.

We give three types of globalization theorems:
Patchwork globalization in \ref{sec:patchwork},
Lifting globalization in \ref{sec:cat-globalize}
and the Hadamard--Cartan theorem in \ref{sec:Hadamard--Cartan}.

The necessary techniques to prove these theorems include 
Reshetnyak's gluing theorem discussed in \ref{sec:cba-gluing},
the structure of the space of geodesics discussed in \ref{sec:geod-space}, 
and Reshetnyak's majorization theorem in \ref{sec:resh-kirz}.

Section \ref{sec:angles-cba} discusses angles and the first variation formula.   


%%%%%%%%%%%%%%%%%%%%%%%%%%%%%%%%%%%%%%%%%%%%%%%%%%%%%%%%%%%%%%%%%%%%%%%%%%%%%%%%%%%%%%%%%%%%
\section{Four-point comparison.} \label{sec:cba-def}

      
\begin{thm}{Four-point comparison}\label{def:2+2}
A quadruple of points $p^1,p^2,x^1,x^2$ in a metric space 
satisfies
\emph{$\CAT\kappa$ comparison} 
\index{$\CAT\kappa$ comparison}
if
  
\begin{subthm}{}
$\angk{\kappa}{p^1}{x^1}{x^2} 
\le 
\angk{\kappa}{p^1}{p^2}{x^1}+\angk{\kappa}{p^1}{p^2}{x^2}$, or
\end{subthm}

\begin{subthm}{}
$\angk{\kappa} {p^2}{x^1}{x^2}\le \angk{\kappa} {p^2}{p^1}{x^1} + \angk{\kappa} {p^2}{p^1}{x^2}$, or
\end{subthm}

\begin{subthm}{}
one of the six model angles 
\[\angk{\kappa}{p^1}{x^1}{x^2},\ \angk{\kappa}{p^1}{p^2}{x^1},\ \angk{\kappa}{p^1}{p^2}{x^2},\ \angk{\kappa}{p^2}{x^1}{x^2},\ \angk{\kappa}{p^2}{p^1}{x^1},\ \angk{\kappa}{p^2}{p^1}{x^2}\]
is undefined.
\end{subthm}
\end{thm}

\begin{wrapfigure}{r}{34mm}
\begin{lpic}[t(-0mm),b(0mm),r(0mm),l(3mm)]{pics/22(1)}
\lbl[lt]{13,1;$p^1$}
\lbl[rb]{20,24;$p^2$}
\lbl[r]{2,15;$x^1$}
\lbl[l]{31,12;$x^2$}
\lbl[rb]{16,12;$z$}
\end{lpic}
\end{wrapfigure}

Here is more intuitive reformulation.

\begin{thm}{Reformulation}
Let $\spc{X}$ be a metric space.
A quadruple $p^1,p^2,x^1,x^2\in \spc{X}$ satisfies $\CAT\kappa$ point comparison if one of the following holds:
\begin{subthm}{}
One of the triples 
$(p^1,p^2,x^1)$ 
or 
$(p^1, p^2, x^2)$ 
has perimeter $>2\cdot\varpi\kappa$.
\end{subthm}

\begin{subthm}{}
If $\trig{\tilde p^1}{\tilde p^2}{\tilde x^1}
=
\modtrig\kappa(p^1 p^2 x^1)$ 
and
$\trig{\tilde p}{\tilde p^2}{\tilde x^2}
=
\modtrig\kappa p^1 p^2 x^2$, then
\[\dist{\tilde x^1}{\tilde z}{}+\dist{\tilde z}{\tilde x^2}{}\ge \dist{x^1}{x^2}{},\]
for any $\tilde z\in[\tilde p^1\tilde p^2]$.

\end{subthm}

\end{thm}


\begin{thm}{Definition}
\label{def:ccat}
A metric space $\spc{U}$ 
is called $\CAT{\kappa}$ if every quadruple $p^1,p^2,x^1,x^2$ satisfies the $\CAT\kappa$ comparison (\ref{def:2+2}).  

We say that $\spc{U}$ is $\CAT{}$ if it is $\CAT{\kappa}$ for some $\kappa\in\RR$.  
\end{thm}



The condition $\spc{U}$ is $\CAT\kappa$ should be understood as ``$\spc{U}$ has global curvature $\le\kappa$''.
In Proposition~\ref{prop:inherit-bound}, it will be shown that this formulation makes sense; 
in particular, if $\kappa\le\Kappa$, then any $\CAT\kappa$ space is $\CAT\Kappa$.


This terminology was introduced by Gromov;  
$\CAT{}$ stands for \'Elie Cartan, Alexandr Alexandrov and Victor Toponogov.
Originally these spaces were called \emph{$\mathfrak{R}_\kappa$ domains};
this is Alexandrov's terminology and is still in use.


\begin{thm}{Exercise}
\label{ex:ccat-(3+1)}
Let $\spc{U}$ be a metric space.
Show that $\spc{U}$ is $\CAT\kappa$
if and only if every quadruple of points in $\spc{U}$ admits one labeling by $(p,x^1,x^2,x^3)$ so that the three angles 
$\angk\kappa p{x^1}{x^2}$,
$\angk\kappa p{x^2}{x^3}$ and
$\angk\kappa p{x^1}{x^3}$
satisfy all three triangle inequalities or one of these angels is undefined.
\end{thm}

\begin{thm}{Exercise}
Show that $\spc{U}$ is $\CAT\kappa$
if and only if for any quadruple of points 
$p^1,p^2,x^1,x^2$ in $\spc{U}$ such that
$\dist{p^1}{p^2}{},\dist{x^1}{x^2}{}\le \varpi\kappa$
there is a quadruple $q^1,q^2,y^1,y^2$ in $\Lob m\kappa$
such that 
\begin{align*}
\dist{q^1}{q^2}{}&=\dist{p^1}{p^2}{},
&
\dist{y^1}{y^2}{}&=\dist{x^1}{x^2}{},
&
\dist{q^i}{y^j}{}&\le \dist{p^i}{x^j}{}
\end{align*}
for any $i$ and $j$.
\end{thm}

The next proposition follows directly from Definition \ref{def:ccat} and the definitions of ultralimit and ultrapower;
see Section~\ref{ultralimits} for the related definitions.
Recall that $\o$ denotes a selective ultrafilter on $\NN$ which is fixed once for all.


\begin{thm}{Proposition}
\label{prop:CAT^omega}
Let $\spc{U}_n$ be a $\CAT{\kappa_n}$ space for each $n\in\NN$.
Assume $\spc{U}_n\to \spc{U}_\o$ and $\kappa_n\to\kappa_\o$ as $n\to\o$.
Then $\spc{U}_\o$ is $\CAT{\kappa_\o}$.

Moreover, a metric space $\spc{U}$ is $\CAT\kappa$ if and only if so is its ultrapower $\spc{U}^\o$.

\end{thm} 

\section{Geodesics}

\begin{thm}{Uniqueness of geodesics}\label{thm:cat-unique}\label{thm:cat-complete} 
In a complete length $\CAT\kappa$ space, pairs of points at distance $<\varpi\kappa$ are joined by unique geodesics, and these geodesics depend continuously on their endpoint pairs.
\end{thm}

\parit{Proof.} 
Fix a complete length $\CAT\kappa$ space $\spc{U}$.
Fix two points $p^1,p^2\in \spc{U}$  such that 
\[\dist{p^1}{p^2}{\spc{U}}<\varpi\kappa.\]

Choose a sequence of approximate midpoints $z_n$ for $p^1$ and $p^2$;
that is,  
\[\dist{p^1}{z_n}{},\dist{p^2}{z_n}{}
\to\tfrac12\cdot\dist[{{}}]{p^1}{p^2}{}
\ \text{as}\ n\to\infty.
\eqlbl{eq:to|p1p2|/2}\]

By the law of cosines, $\angk{\kappa} {p^1}{z_n}{p^2}$ and $\angk{\kappa} {p^2}{z_n}{p^1}$ are arbitrarily small when $n$ is sufficiently large.

Let us apply $\CAT\kappa$  comparison (\ref{def:2+2}) to the quadruple $p^1$, $p^2$, $z_n$, $z_\kay$ with large $n$ and $\kay$.
We conclude that  $\angk{\kappa} {p}{z_n}{z_\kay}$ is arbitrarily small when $n,\kay$ are sufficiently large and $p$ is either $p^1$ or $p^2$.  
By \ref{eq:to|p1p2|/2} and the law of cosines, $(z_n)$ converges.  

Since $\spc{U}$ is complete, the sequence $(z_n)$ converges to a midpoint of $p^1$ and $p^2$. 
By Lemma~\ref{lem:mid>geod} we obtain  the existence of a geodesic $[p^1p^2]$.

Now suppose $p^1_n\to p^1$, $p^2_n\to p^2$ as $n\to\infty$.
Let $z_n$ be the midpoint of a geodesic $[p^1_n p^2_n]$ and $z$ be the midpoint of a geodesic $[p^1p^2]$.  

It suffices to show that 
\[\dist{z_n}{z}{}\to0
\ \ \t{as}\ \ 
n\to\infty.
\eqlbl{eq:z_n->z}\]

By the triangle inequality, the $z_n$ are approximate midpoints of $p^1$ and $p^2$. 
Apply $\CAT\kappa$  comparison (\ref{def:2+2}) to the quadruple $p^1,p^2,z_n,z$. 
For $p=p^1$ or $p=p^2$, we see that $\angk{\kappa} {p}{z_n}{z}$ is arbitrarily small when $n$ is sufficiently large.  
By the law of cosines, \ref{eq:z_n->z} follows.
\qeds

\begin{thm}{Exercise}\label{ex:CAT-mnfld=>ext.geod}
Let $\spc{U}$ be a complete length $\CAT{}$ space.
Assume $\spc{U}$ is a topological manifold.
Show that any geodesic in $\spc{U}$ can be extended 
as a two-side infinite local geodesic.

Moreover the same holds for any locally geodesic locally $\CAT{}$ space $\spc{U}$ with nontrivial local homology groups at any point;
the latter holds in particular, if $\spc{U}$ is a homological manifold.
\end{thm}

\begin{thm}{Exercise}
Assume $\spc{U}$ is a locally compact geodesic $\CAT{}$ space with extandable geodesics;
that is, any geodesic in $\spc{U}$ can be extended to a both-side infinite local geodesic.

Show that the space of geodesic directions $\Sigma_p'$ is complete for any $p\in \spc{U}$.
\end{thm}

By Uniqueness of geodesics~\ref{thm:cat-unique},
we have the following.

\begin{thm}{Corollary}\label{cor:cat-ccat}
Any  complete length $\CAT\kappa$ space is $\varpi\kappa$-geodesic.
\end{thm}

\begin{thm}{Proposition}\label{cor:cat-completion} 
The completion of any geodesic $\CAT{\kappa}$ space is a complete length $\CAT\kappa$ space.

Moreover, $\spc{U}$ is a geodesic $\CAT\kappa$ space
if and only if there is a complete length $\CAT\kappa$ space $\bar{\spc{U}}$ that contains a $\varpi\kappa$-convex dense set isometric to $\spc{U}$.
\end{thm}

\parit{Proof.} 
By Theorem \ref{thm:cat-complete},
in order to show that the completion $\bar{\spc{U}}$ of any geodesic $\CAT\kappa$ space $\spc{U}$ is $\CAT{\kappa}$,
it is sufficient to verify that the completion of a length space is a length space; 
this is straightforward.

For the second part of the proposition, note that the completion $\bar{\spc{U}}$
contains the original space $\spc{U}$ as a dense $\varpi\kappa$-convex subset, and the metric on $\spc{U}$ coincides with the induced length metric from $\bar{\spc{U}}$.
\qeds

Here is a corollary from Proposition~\ref{cor:cat-completion}
and Theorem~\ref{thm:cat-unique}.

\begin{thm}{Corollary}\label{cor:cat-unique}
Let $\spc{U}$ be a  $\varpi\kappa$-geodesic $\CAT\kappa$ space.
Then pairs of points in $\spc{U}$ at distance less than $\varpi\kappa$ are joined by unique geodesics, and these geodesics depend continuously on their endpoint pairs.

Moreover for any pair of points $p,q\in \spc{U}$ in and any 
\[\Lip>\sup\set{\frac{\sn\kappa r}{\sn\kappa \dist{p}{q}{}}}{0\le r\le \dist{p}{q}{}}\]
there are neighborhoods $\Omega_p\ni p$ and $\Omega_q\ni q$ such that the map
\[(x,y,t)\mapsto \geodpath_{[xy]}(t)\]
is $\Lip$-Lipschitz in $\Omega_p\times \Omega_q\times[0,1]$.
\end{thm}

\parit{Proof.}
By Proposition~\ref{cor:cat-completion} any geodesic $\CAT{\kappa}$ space is isometric to a convex dense subset of a complete length $\CAT\kappa$ space.
It remains to apply  Theorem~\ref{thm:cat-unique}.
\qeds


%%%%%%%%%%%%%%%%%%%%%%%%%%%%%%%%%%%%%%%%%%%%%%%%%%%%%%%%%%%%%%%%%%%%%%%%%%%%%%%%%%%%%%%%%

\section{More comparisons}\label{sec:cat-angles}

Here we give a few reformulations of Definition~\ref{def:ccat}.

\begin{wrapfigure}[6]{r}{28mm}
\begin{lpic}[t(-0mm),b(6mm),r(0mm),l(0mm)]{pics/lem_alex1(1)}
\lbl[lb]{10,23;$x$}
\lbl[rt]{1.5,.5;$p$}
\lbl[bl]{25,7.5;$y$}
\lbl[lb]{17,15;$z$}
\end{lpic}
\end{wrapfigure}


\begin{thm}{Theorem}
\label{thm:defs_of_cat} 
If $\spc{U}$ is a $\CAT\kappa$ space, then 
the following conditions hold for all triples $p,x,y\in \spc{U}$ of perimeter $<2\cdot\varpi\kappa$:

\begin{subthm}{cat-2-sum} (adjacent-angles comparison\index{comparison!adjacent-angles comparison}) for any geodesic $[x y]$ and $z\in \mathopen{]}x y\mathclose{[}$, we have
\[\angk\kappa z p x
+\angk\kappa z p y\ge \pi.\]
\end{subthm}

\begin{subthm}{cat-monoton}
(point-on-side comparison\index{comparison!point-on-side comparison}) 
for any geodesic $[x y]$ and $z\in \mathopen{]}x y\mathclose{[}$, we have
\[\angk\kappa x p y\ge\angk\kappa x p z,\]
or equivalently, 
\[\dist{\tilde p}{\tilde z}{}\ge \dist{p}{z}{},\]
where $\trig{\tilde p}{\tilde x}{\tilde y}=\modtrig\kappa(p x y)$, $\tilde z\in\mathopen{]} \tilde x\tilde y\mathclose{[}$, $\dist{\tilde x}{\tilde z}{}=\dist{x}{z}{}$.
\end{subthm}

\begin{subthm}{cat-hinge}(%hinge 
angle comparison\index{% hinge comparison})
for any hinge $\hinge x p y$, the angle 
$\mangle\hinge x p y$ exists and
\[\mangle\hinge x p y\le\angk\kappa x p y,\]
or equivalently,
\[\side\kappa \hinge x p y\le\dist{p}{y}{}.\]
\end{subthm}
%SBA:  a ``hinge'' in English is a ``movable mechanism''.  It must MOVE. BBI uses it correctly:  if you look up ``hinge'' in BBI index, it sends you to monotonicity.

Moreover, if $\spc{U}$ is  $\varpi\kappa$-geodesic, then the converse holds in each case.  

\end{thm}


\parbf{Remark.}
\label{22remark}
In the following proof, the part (\ref{SHORT.cat-hinge})$\Rightarrow$(\ref{SHORT.cat-2-sum})
only requires that the $\CAT\kappa$ comparison (\ref{def:2+2}) hold for any quadruple, and does not require the existence of geodesics at distance $<\varpi\kappa$. 
The same is true of the parts (\ref{SHORT.cat-2-sum})$\Leftrightarrow$(\ref{SHORT.cat-monoton}) and
(\ref{SHORT.cat-monoton})$\Rightarrow$(\ref{SHORT.cat-hinge}).  
Thus the conditions (\ref{SHORT.cat-2-sum}), (\ref{SHORT.cat-monoton}) and (\ref{SHORT.cat-hinge}) are valid for any metric space (not necessarily length space) that satisfies $\CAT\kappa$ comparison (\ref{def:2+2}). 
The converse does not hold; for example, all these conditions are 
vacuously true in a 
totally disconnected space, while 
$\CAT\kappa$ comparison is not.

\parit{Proof. (\ref{SHORT.cat-2-sum})}. 
Since the perimeter of $p,x,y$ is $<2\cdot \varpi\kappa$, so is the perimeter of any subtriple of $p,z,x,y$ by the triangle inequality. 
By Alexandrov's lemma (\ref{lem:alex}), 
\[\angk\kappa p z x +\angk\kappa p z y  < \angk{\kappa} p x y \ \text{or}\  \angk\kappa z p x  +\angk\kappa z p y  =\pi.\]
In the former case, $\CAT\kappa$ comparison (\ref{def:2+2}) applied to the quadruple $p, z, x, y$ implies
\[\angk\kappa z p x  +\angk\kappa z p y  \ge \angk{\kappa} z x y =\pi.\]

\parit{(\ref{SHORT.cat-2-sum}) $\Leftrightarrow$ (\ref{SHORT.cat-monoton})}. Follows from  Alexandrov's lemma (\ref{lem:alex}).

\parit{(\ref{SHORT.cat-monoton}) $\Rightarrow$ (\ref{SHORT.cat-hinge}).} 
By (\ref{SHORT.cat-monoton}), for $\bar p\in\mathopen{]}x p]$ and $\bar y\in\mathopen{]}x y]$ the function $(\dist{x}{\bar p}{},\dist{x}{\bar y}{})\mapsto\angk\kappa x{\bar p}{\bar y}$ is nondecreasing in each argument.
In particular, 
$\mangle\hinge x p y\z=\inf\angk\kappa x{\bar p}{\bar y}$.
Thus $\mangle\hinge x p y$ exists and is
at most $\angk\kappa x p y$. 

\parit{Converse.} Assume $\spc{U}$ is $\varpi\kappa$-geodesic. 
Let us first show that in this case (\ref{SHORT.cat-hinge}) $\Rightarrow$ (\ref{SHORT.cat-2-sum}).

Indeed, by (\ref{SHORT.cat-hinge}) and the triangle inequality for angles (\ref{claim:angle-3angle-inq}),
\[\angk\kappa z p x
+\angk\kappa z p y \ge \mangle\hinge z p x
+\mangle\hinge z p y \ge \pi.\]

\begin{wrapfigure}{r}{25mm}
\begin{lpic}[t(0mm),b(0mm),r(0mm),l(0mm)]{pics/cat-monoton-ineq(1)}
\lbl[rb]{13.5,20;$\tilde p^1$}
\lbl[lt]{11,1;$\tilde p^2$}
\lbl[rb]{3,13.5;$\tilde x^1$}
\lbl[lb]{23,13.5;$\tilde x^2$}
\lbl[tl]{13,11;$\tilde q$}
\end{lpic}
\end{wrapfigure}

Therefore, it is sufficient to prove the converse for \textit{(\ref{SHORT.cat-monoton})}.

Given a quadruple  $p^1,p^2,x^1,x^2$ whose subtriples have perimeter $<2\cdot\varpi\kappa$ , we must verify $\CAT\kappa$ comparison (\ref{def:2+2}).
In $\Lob2\kappa$, construct the model triangles  $\trig{\tilde p^1}{\tilde p^2}{\tilde x^1} = \modtrig\kappa(p^1 p^2 x^1 )$ 
and $\trig{\tilde p^1}{\tilde p^2}{\tilde x^2}= \modtrig\kappa(p^1 p^2 x^2)$, lying on either side of a common segment $[\tilde p^1 \tilde p^2]$.
We may suppose 
\[\angk{\kappa} {p^1}{p^2}{x^1}+\angk{\kappa} {p^1}{p^2}{x^2}
\le
\pi
\ \ \t{and}\ \ 
\angk{\kappa}{p^2}{p^1}{x^1}+\angk{\kappa} {p^2}{p^1}{x^2}
\le 
\pi,\] 
since otherwise $\CAT\kappa$ comparison holds trivially.  
Then $[\tilde p^1 \tilde p^2]$ and $[\tilde x^1 \tilde x^2]$ intersect, say at $\tilde q$.  

By assumption, there is a geodesic $[p^1 p^2]$.
Choose $q\in[p^1 p^2]$ corresponding to $\tilde q$; 
that is, $\dist{p^1}{q}{}=\dist{\tilde p^1}{\tilde q}{}$.
Then 
\[\dist{x^1}{x^2}{} \le \dist{x^1}{q}{} + \dist{q}{x^2}{} \le \dist{\tilde x^1}{\tilde q}{} + \dist{\tilde q}{\tilde x^2}{} = \dist{\tilde x^1}{\tilde x^2}{},\]
where the second inequality follows from (\ref{SHORT.cat-monoton}). 
Therefore by monotonicity of the function $a\mapsto\tangle\mc\kappa\{a;b,c\}$ (\ref{increase}),
\begin{align*}
\angk{\kappa} {p^1}{x^1}{x^2} \le  \mangle\hinge{ \tilde p^1}{ \tilde x^1}{ \tilde x^2}
= \angk{\kappa} {p^1}{p^2}{x^1} + \angk{\kappa} {p^1}{p^2}{x^2}.
\end{align*}
\qedsf

As a corollary, we display important information from the proof of \ref{thm:defs_of_cat},
namely, monotonicity of the model angle with respect to adjacent sidelengths. 

\begin{thm}{Angle-sidelength  monotonicity}\label{cor:monoton-cba} 
Suppose $\spc{U}$ is a $\varpi\kappa$-geodesic $\CAT\kappa$ space, and 
$p,x,y\in \spc{U}$ have  perimeter $<2\cdot \varpi\kappa$.
Then for $\bar y\in\mathopen{]}x y]$, the function 
\[\dist{x}{\bar y}{}\mapsto \angk\kappa x p{\bar y}\] 
is nondecreasing.

In particular, if $\bar p\in \mathopen{]}x p]$, then
\begin{subthm}{two-mono-cba}the function 
\[(\dist{x}{\bar y}{},\dist{x}{\bar p}{})\mapsto \angk\kappa x {\bar p}{\bar y}\] is nondecreasing in each argument,
\end{subthm}
 
\begin{subthm}{cor:monoton-cba:angle=inf}the angle $\mangle\hinge{x}{p}{y}$ exists and 
\[\mangle\hinge{x}{p}{y}
=
\inf\set{\angk\kappa x {\bar p}{\bar y}}{
\bar p\in\mathopen{]}x p],\ 
\bar y\in\mathopen{]}x y]}.\]
\end{subthm}
\end{thm}

\begin{thm}{Exercise}\label{ex:convexity-CAT0}
Assume $\spc{U}$ is a geodesic $\CAT0$ space.
Show that for any two geodesic paths 
$\gamma,\sigma\:[0,1]\to \spc{U}$
the function 
\[t\mapsto\dist{\gamma(t)}{\sigma(t)}{}\] 
is convex.
\end{thm}



\begin{thm}{Proposition}
\label{prop:inherit-bound}
Assume $\kappa<\Kappa$.
Then any complete length $\CAT\kappa$ space is $\CAT\Kappa$.

Moreover a space $\spc{U}$ is $\CAT\kappa$ if  $\spc{U}$ is $\CAT\Kappa$ for all $\Kappa>\kappa$.
\end{thm}

\parit{Proof.} The first statement follows from Corollary \ref{cor:cat-ccat}, the adjacent-angles comparison (\ref{cat-2-sum}) and the monotonicity of the function $\kappa\mapsto\angk\kappa x y z$ (\ref{k-decrease}).

The second statement follows since the function $\kappa\mapsto\angk\kappa x y z$ is continuous.
\qeds

%%%%%%%%%%%%%%%%%%%%%%%%%%%%%%%%%%%%%%%%%%%%%%%%%%%%%%%%%%%%%%%%%%%%%%%%%%%%%%%%%%%%%%%%%%%%

\section{Thin triangles} \label{sec:thin-triangle}

In this section we define thin triangles
and use them to characterize $\CAT{}$ spaces.
Inheritance for thin triangles with respect to decomposition
is the main result of this section.
It will lead to two fundamental constructions:  
Alexandrov's patchwork globalization  (\ref{thm:alex-patch}) 
and Reshetnyak gluing (\ref{thm:gluing}).
 
\begin{thm}{Definition of $\bm\kappa$-thin triangles}\label{def:k-thin}
Let $\trig{x^1}{x^2}{x^3}$ be a triangle of perimeter $<2\cdot \varpi\kappa$ in a metric space.
Consider its model triangle
$\trig{\tilde x^1}{\tilde x^2}{\tilde x^3}=\modtrig\kappa({x^1}{x^2}{x^3})$ 
and the  \emph{natural map}\index{natural map} $\trig{\tilde x^1}{\tilde x^2}{\tilde x^3}\to \trig{x^1}{x^2}{x^3}$ 
that sends a point $\tilde z\in[\tilde x^i\tilde x^j]$ to the corresponding point $z\in[x^ix^j]$
(that is, such that $\dist{\tilde x^i}{\tilde z}{}=\dist{x^i}{z}{}$ and therefore $\dist{\tilde x^j}{\tilde z}{}=\dist{x^j}{z}{}$).

We say the triangle $\trig{x^1}{x^2}{x^3}$ is \emph{$\kappa$-thin}\index{$\kappa$-thin} if the natural map $\trig{\tilde x^1}{\tilde x^2}{\tilde x^3}\to \trig{x^1}{x^2}{x^3}$ is short.
\end{thm}

\begin{thm}{Exercise}
Let $\spc{U}$ be a $\varpi\kappa$-geodesic $\CAT\kappa$ space,
$\trig xyz$ be a triangle in $\spc{U}$
and $\trig{\tilde x}{\tilde y}{\tilde z}$ its model triangle in $\Lob{2}{\kappa}$.
Prove that the natural map $f\:\trig{\tilde x}{\tilde y}{\tilde z}\to \trig xyz$ 
 is distance-preserving if and ony if one of the following conditions hold:

\begin{subthm}{}
$\mangle\hinge x y z= \angk\kappa x y z$,
\end{subthm}

\begin{subthm}{}
$\dist{x}{w}{}=\dist{\tilde x}{\tilde w}{}$ for some  $\tilde w\in]\tilde y\tilde z[$ and
$w= f(\tilde w)$,   
\end{subthm}

\begin{subthm}{} 
$\dist{v}{w}{}=\dist{\tilde v}{\tilde w}{}$ for some  
$\tilde v\in \mathopen{]}\tilde x \tilde y\mathclose{[}$,  $\tilde w\in\mathopen{]}\tilde x \tilde z\mathclose{[}$
and $v=f(\tilde v)$, $w=f(\tilde w)$.
\end{subthm} 

\end{thm}



\begin{thm}{Proposition}\label{prop:k-thin}
Let $\spc{U}$ be a $\varpi\kappa$-geodesic space. 
Then $\spc{U}$ is  $\CAT\kappa$
if and only if every triangle of perimeter $<2\cdot \varpi\kappa$ in $\spc{U}$  is $\kappa$-thin.
\end{thm}

\parit{Proof.}
``If'' is immediate from point-on-side comparison \ref{cat-monoton}.  
``Only if'' follows from the angle-sidelength  monotonicity \ref{two-mono-cba}.
\qeds


\begin{thm}{Corollary}\label{cor:loc-geod-are-min}
Suppose $\spc{U}$ is a $\varpi\kappa$-geodesic $\CAT\kappa$ space.  
Then any local geodesic in $\spc{U}$ of length $<\varpi\kappa$ is length-minimizing.
\end{thm}

\parit{Proof.}
Suppose $\gamma\:[0,\ell]\to\spc{U}$ is a local geodesic  that is not minimizing, with $\ell<\varpi\kappa$.
Choose $a$ to be the maximal value 
such that $\gamma$ is minimizing on $[0,a]$.
Further choose $b>a$ so that $\gamma$ is minimizing on~$[a,b]$.

Since triangle $\trig{\gamma(0)}{\gamma(a)}{\gamma(b)}$ is $\kappa$-thin, we have
\[\dist{\gamma(a-\eps)}{\gamma(a+\eps)}{}<2\cdot\eps\]
for all small $\eps>0$,
a contradiction.
\qeds


Now let us formulate the main result of this section.
The inheritance lemma states that  in any metric space, a triangle is $\kappa$-thin if it decomposes into $\kappa$-thin triangles. 
In contrast, $\Alex\kappa$ comparisons are not inherited in this way.

\begin{wrapfigure}[7]{r}{28mm}
\begin{lpic}[t(-0mm),b(6mm),r(0mm),l(0mm)]{pics/lem_alex1(1)}
\lbl[lb]{10,23;$x$}
\lbl[rt]{1.5,.5;$p$}
\lbl[bl]{25,7.5;$y$}
\lbl[lb]{17,15;$z$}
\end{lpic}
\end{wrapfigure}

\begin{thm}{Inheritance lemma}
\label{lem:inherit-angle} 
In a metric space, consider a triangle $\trig p x y$ that \emph{decomposes}\index{decomposed triangle} 
into two triangles $\trig p x z$ and $\trig p y z$;
that is, $\trig p x z$ and $\trig p y z$ have common side $[p z]$, and the sides $[x z]$ and $[z y]$ together form the side $[x y]$ of $\trig p x y$.

If the triangle $\trig p x y$ has perimeter $<2\cdot\varpi\kappa$
and both triangles $\trig p x z$ and $\trig p y z$ are $\kappa$-thin, then triangle $\trig p x y$ is  $\kappa$-thin.
\end{thm} 

We shall need the following model-space lemma, 
which is part of the proof of \cite[Lemma 2]{reshetnyak:major}.


\begin{thm}{Lemma}\label{lem:quadrangle}
Let $\trig{\tilde p}{\tilde x}{\tilde y}$ be a triangle in $\Lob2{\kappa}$ and $\tilde z\in[\tilde x\tilde y]$.
Consider the solid triangle $\tilde D=\Conv\trig{\tilde p}{\tilde x}{\tilde y}$.  
Construct  points $\dot p, \dot x, \dot z, \dot y\in \Lob2{\kappa}$ such that 
\begin{align*}
\dist{\dot p}{\dot x}{}&=\dist{\tilde p}{\tilde x}{},
&
\dist{\dot p}{\dot y}{}&=\dist{\tilde p}{\tilde y}{},
&
\dist{\dot p}{\dot z}{}&\le \dist{\tilde p}{\tilde z}{},
\\
\dist{\dot x}{\dot z}{}&=\dist{\tilde x}{\tilde z}{},
&
\dist{\dot y}{\dot z}{}&=\dist{\tilde y}{\tilde z}{},
\end{align*}
where points $\dot x$ and $\dot y$ lie on either side of $[\dot p\dot z]$.
Set 
\[\dot D=\Conv\trig {\dot p}{\dot x}{\dot z}\cup \Conv\trig {\dot p} {\dot y} {\dot z}.\]

Then there is a short map $F\:\tilde D\to \dot D$ that maps $\tilde p$, $\tilde x$, $\tilde y$ and $\tilde z$ to $\dot p$, $\dot x$, $\dot y$ and $\dot z$ respectively.
\end{thm}

\begin{wrapfigure}{r}{30mm}
\begin{lpic}[t(-5mm),b(0mm),r(0mm),l(0mm)]{pics/resh(1)}
\lbl[t]{17,-.5;$\tilde p$}
\lbl[b]{29,23.5;$\tilde x$}
\lbl[rb]{2,19;$\tilde y$}
\lbl[b]{15,21.5;$\tilde z$}
\lbl[tr]{13.5,13;$\tilde z_x$}
\lbl[tl]{18.3,14;$\tilde z_y$}
\end{lpic}
\end{wrapfigure}

\parit{Proof.} 
By Alexandrov's lemma (\ref{lem:alex}), 
there are nonoverlapping triangles 
$\trig{\tilde p}{\tilde x}{\tilde z_y}\iso\trig {\dot p}{\dot x}{\dot z}$ 
and 
$\trig{\tilde p}{\tilde y}{\tilde z_x}\iso\trig {\dot p}{\dot y}{\dot z}$
 inside triangle $\trig{\tilde p}{\tilde x}{\tilde y}$.

Connect points in each pair
$(\tilde z,\tilde z_x)$, 
$(\tilde z_x,\tilde z_y)$ 
and $(\tilde z_y,\tilde z)$ 
with arcs of circles centered at 
$\tilde y$, $\tilde p$, and $\tilde x$ respectively. 
Define $F$ as follows.
\begin{itemize}
\item Map  $\Conv\trig{\tilde p}{\tilde x}{\tilde z_y}$ isometrically onto  $\Conv\trig {\dot p}{\dot x}{\dot y}$;
similarly map $\Conv \trig{\tilde p}{\tilde y}{\tilde z_x}$ onto $\Conv \trig {\dot p}{\dot y}{\dot z}$.

\item If $w$ is in one of the three circular sectors, say at distance $r$ from center of the circle, let $F(w)$ be the point on the corresponding segment 
$[\dot p \dot z]$, 
$[\dot x \dot z]$ 
or $[\dot y \dot z]$ whose distance from the lefthand endpoint of the segment is $r$.
\item Finally, if $w$ lies in the remaining curvilinear triangle $\tilde z \tilde z_x \tilde z_y$, 
set $F(w) = \dot z$. 
\end{itemize}
By construction, $F$ satisfies the conditions of the lemma. 
\qeds


\parit{Proof of Inheritance lemma \ref{lem:inherit-angle}.}
Construct model triangles $\trig{\dot p}{\dot x}{\dot z}\z=\modtrig\kappa(p x z)$ 
and $\trig {\dot p} {\dot y} {\dot z}=\modtrig\kappa(p y z)$ so that $\dot x$ and $\dot y$ lie on opposite sides of $[\dot p\dot z]$.

\begin{wrapfigure}{r}{25mm}
\begin{lpic}[t(0mm),b(0mm),r(0mm),l(0mm)]{pics/cat-monoton-ineq(1)}
\lbl[rb]{14,21;$\dot z$}
\lbl[lt]{11,1;$\dot p$}
\lbl[r]{1,12.5;$\dot x$}
\lbl[l]{24,13;$\dot y$}
\lbl[tl]{12.5,11.5;$\dot w$}
\end{lpic}
\end{wrapfigure}

Suppose
\[\angk\kappa{z}{p}{x}+\angk\kappa{z}{p}{y}
<
\pi.\]
Then for some point $\dot w\in[\dot p\dot z]$, we have \[\dist{\dot x}{\dot w}{}+\dist{\dot w}{\dot y}{}
<
\dist{\dot x}{\dot z}{}+\dist{\dot z}{\dot y}{}=\dist{x}{y}{}.\]
Let $w\in[p z]$ correspond to $\dot w$; that is, $\dist{z}{w}{}=\dist{\dot z}{\dot w}{}$. 
Since $\trig p x z$ and $\trig p y z$ are $\kappa$-thin, we have 
\[\dist{x}{w}{}+\dist{w}{y}{}<\dist{x}{y}{},\]
contradicting the triangle inequality. 

Thus 
\[\angk\kappa{z}{p}{x}+\angk\kappa{z}{p}{y}
\ge
\pi.\]
By Alexandrov's lemma (\ref{lem:alex}), this is equivalent to 
\[\angk\kappa x p z\le\angk\kappa x p y.
\eqlbl{eq:for|pz|}\]

Let $\trig{\tilde  p}{\tilde  x}{\tilde  y}=\modtrig\kappa (p x y)$ 
and $\tilde  z\in[\tilde  x\tilde  y]$ correspond to $z$; that is, $\dist{x}{z}{}=\dist{\tilde  x}{\tilde  z}{}$.
Inequality~\ref{eq:for|pz|} is equivalent to $\dist{ p}{ z}{}\le \dist{\tilde  p}{\tilde  z}{}$.
Hence  Lemma~\ref{lem:quadrangle} applies.  Therefore 
there is a short map $F$ that  sends 
$\trig{\tilde  p}{\tilde  x}{\tilde  y}$ to $\dot D=\Conv\trig {\dot p}{\dot x}{\dot z}\cup \Conv\trig {\dot p} {\dot y} {\dot z}$ 
in such a way that 
$\tilde p\mapsto \dot p$,
$\tilde x\mapsto \dot x$,
$\tilde z\mapsto \dot z$
and
$\tilde y\mapsto \dot y$.

By assumption, the natural maps $\trig {\dot p} {\dot x} {\dot z}\to\trig p x z$ and $\trig {\dot p} {\dot y} {\dot z}\to\trig p y z$ are short.  
By composition,  the natural map from $\trig{\tilde  p}{\tilde  x}{\tilde  y}$ to $\trig p y z$ is short, as claimed.
\qeds


%%%%%%%%%%%%%%%%%%%%%%%%%%%%%%%%%%%%%%%%%%%%%%%%%%%%%%%%%%%%%%%%%%%%%%%%%%%%%%%%%%%%%%%%%%%


\section{Function comparison} \label{sec:func-comp}


In this section, we give analytic and geometric ways of viewing the point-on-side comparison (\ref{cat-monoton}) as a convexity condition.

First we obtain a corresponding differential  inequality for the distance function in $\spc{U}$.
In particular, a geodesic space $\spc{U}$ is $\CAT{0}$ if and only if  for any $p\in \spc{U}$, the function $\distfun[2]{p}{}{}\:\spc{U}\to\RR$ is $2$-convex;
see Section~\ref{sec:conv-fun} for the definition.
 
\begin{thm}{Theorem}\label{thm:function-comp} 
Suppose $\spc{U}$ is a $\varpi\kappa$-geodesic space. 
Then the following are equivalent:
\begin{subthm}{function-comp-cat} 
$\spc{U}$ is $\CAT\kappa$,
\end{subthm}
\begin{subthm}{function-comp}
for any $p\in \spc{U}$, the function $f=\md\kappa\circ\distfun{p}{}{}$ satisfies 
\[f''+\kappa \cdot f\ge 1\] 
in $\oBall(p,\varpi\kappa)$.
\end{subthm}\end{thm}

\parit{Proof.} Fix a sufficiently short geodesic $[x y]$ in $\oBall(p,\varpi\kappa)$.
We can assume that the model triangle $\trig{\tilde p}{\tilde x}{\tilde y}\z=\modtrig\kappa(p x y)$ is defined. 
Set \begin{align*} 
\tilde r(t)&=\dist{\tilde p}{\geod_{[\tilde x\tilde y]}(t)}{},
& 
r(t)&=\dist{p}{\geod_{[xy]}(t)}{}.                           \end{align*}
Set $\tilde f=\md\kappa\circ\tilde r$ and $f=\md\kappa\circ r$.
By Property~\ref{md-diff-eq}, we have $\tilde f''=1-\kappa\cdot  \tilde f$. Clearly $\tilde f(t)$ and $f(t)$ agree at $t=0$ and $t=\dist{x}{y}{}$. 
The point-on-side comparison (\ref{cat-monoton}) is the condition $r(t)\le\tilde r(t)$  for all $t\in[0,\dist{x}{y}{}]$.  Since $\md\kappa$ is increasing on $[0,\varpi\kappa)$, $r\le \tilde r$ and $f\le \tilde f$ are equivalent.  Thus the claim follows by Jensen's inequality 
(\ref{y''-mono}). \qeds

Recall that Busemann functions defined in Proposition \ref{prop:busemann}.
The following exercise is analogous to Exercise~\ref{ex:busemann-CBB}.

\begin{thm}{Exercise}\label{ex:busemann-CBA}
Let $\spc{U}$ be a complete length $\CAT\kappa$ space
and $\bus_\gamma\:\spc{U}\to \RR$ be the Busemann function for a ray $\gamma\:[0,\infty)\to \spc{L}$.

\begin{subthm}{}
If $\kappa=0$, then the Busemann function $\bus_\gamma$ is  convex.
\end{subthm}

\begin{subthm}{}
If $\kappa=-1$, then the function 
\[f=\exp\circ\bus_\gamma\] 
satisfies
\[f''- f\ge 0.\]
\end{subthm}

\end{thm}

\section{Convex sets}\label{sec:convex-CBA}

The following is a corollary of Theorem \ref{thm:function-comp}.

\begin{thm}{Corollary}\label{cor:convex-balls}
Suppose $\spc{U}$ is a $\varpi\kappa$-geodesic $\CAT\kappa$ space.
Then any ball (close or open) of radius $R<\tfrac{\varpi\kappa}2$ in $\spc{U}$ is convex.

Moreover, any open ball of radius $\tfrac{\varpi\kappa}2$ is convex
and any closed ball of radius $\tfrac{\varpi\kappa}2$ is $\varpi\kappa$-convex.
\end{thm}

\parit{Proof.}
Suppose $p\in\spc{U}$, $ R\le\varpi\kappa/2$,  and two points 
$x$ and $y$ lie in $\cBall[p, R]$ or $\oBall(p, R)$.
By the triangle inequality, if $\dist{x}{y}{}<\varpi\kappa$, then any
 geodesic $[x y]$ lies in $\oBall(p, \varpi\kappa)$.
 
By function comparison (\ref{thm:function-comp}), 
the geodesic $[x y]$ lies in $\cBall[p,R]$ or $\oBall(p,R)$ respectively.

Thus any ball (close or open) of radius $R<\tfrac{\varpi\kappa}2$ is $\varpi\kappa$-convex.
This implies convexity unless there is a pair of points in the ball at distance at least $\varpi\kappa$.
By the  triangle inequality, the latter is possible only for the closed ball of radius $\tfrac{\varpi\kappa}2$.
\qeds

\begin{thm}{Proposition}
Any weakly $\varpi\kappa$-convex set 
in a complete length $\CAT\kappa$ space is $\varpi\kappa$-convex.
\end{thm}

\parit{Proof.}
Follows from the uniqueness of geodesics in $\CAT\kappa$ spaces (\ref{thm:cat-unique}).
\qeds


\begin{thm}{Closest-point projection lemma}\label{lem:closest point}
Let $\spc{U}$ be a complete length $\CAT\kappa$ space and $K\subset \spc{U}$ be a closed $\varpi\kappa$-convex set. 
Assume that $\dist{K}{p}{}<\tfrac{\varpi\kappa}2$ for some point $p\in \spc{U}$.
Then  
there is unique point $p^*\in K$ that minimizes the distance to $p$;
that is, $\dist{p^*}{p}{}=\dist{K}{p}{}$. 



\end{thm}

\parit{Proof.} 
Fix $r$ properly between $\dist{K}{p}{}$ and $\tfrac{\varpi\kappa}2$.
By function comparison (\ref{thm:function-comp}) 
the function $f=\md\kappa\circ\distfun{p}{}{}$ is strongly convex in $\cBall[p,r]$.

The lemma follows from Lemma~\ref{lem:argmin(convex)} applied to the subspace $K'=K\cap\cBall[p,r]$ 
and the restriction $f|K'$. 
\qeds

\begin{thm}{Exercise}
Let  $\spc{U}$ be a complete length $\CAT0$ space and $K\subset \spc{U}$ be a closed convex set.
Show that the closest-point projection $\spc{U}\to K$ is short. 
\end{thm}

\begin{thm}{Advanced exercise}\label{ex:short-retraction-CBA(1)}
Let  $\spc{U}$ be a complete length $\CAT1$ space and $K\subset \spc{U}$ be a closed $\pi$-convex set.
Assume $K\subset \cBall[p,\tfrac\pi2]$ for some point $p\in \spc{U}$.
Show that there is a short retractionof $\spc{U}$ to $K$. 
\end{thm}



\begin{thm}{Lemma}\label{lem:dist-to-convex}
Let $\spc{U}$  be a $\varpi\kappa$-geodesic $\CAT\kappa$ space
and $K\subset \spc{U}$ be a closed $\varpi\kappa$-convex set.
Set
\[f=\sn\kappa\circ\distfun{K}{}{}.\]
Then
\[f''+\kappa \cdot f\ge 0\]
holds in $\oBall(K,\tfrac{\varpi\kappa}2)$.
\end{thm}

\parit{Proof.}
It is sufficient to show that Jensen's inequality (\ref{y''-mono})
holds on a sufficiently short 
geodsic $[pq]$ in $\oBall(K,\tfrac{\varpi\kappa}2)$.
Since 
\[\dist{K}{p}{},\dist{K}{q}{}<\tfrac{\varpi\kappa}{2},\] 
we may assume that 
\[\dist{p}{q}{}+\dist{K}{p}{}+\dist{K}{q}{}<\varpi\kappa.\eqlbl{eq:sum=<varpi}\]

Namely, for each $x\in[pq]$,
we need to find a value $h(x)\in \RR$
such that for any $x$ we have
\begin{align*}
h(x)&\le f(x),&
h(p)&=f(p),& 
h(q)&=f(q),
\end{align*}
and
\[h''+\kappa\cdot h\ge 0\eqlbl{h''+kh=<0}\]
along $[pq]$.

Denote by $p^{*}$ and $q^{*}$ the closest-point projections of $p$ and $q$ on $K$; 
they are provided by lemma \ref{lem:closest point}.
From \ref{eq:sum=<varpi} and triangle inequality,
we get 
\[\dist{p^*}{q^*}{}<\varpi\kappa.\]
Since $K$ is $\varpi\kappa$-convex, $K\supset[pq]$;
in particular
\[\dist{K}{x}{}\le \dist{[p^*q^*]}{x}{}\]
for any $x\in \spc{U}$.

Consider a majorization $F$ for quadraliteral $[pp^{*}q^{*}q]$.
By Majoriztion theorem \ref{thm:major} and Proposition \ref{prop:majorize-geodesic}, 
the figure $F$ is a solid convex model quadraliteral $[\tilde p\tilde p^*\tilde q^*\tilde q]$ in $\Lob2\kappa$ such that 
\begin{align*}
\dist{\tilde p}{\tilde p^*}{\Lob2\kappa}&=\dist{p}{p^*}{\spc{U}}
&
\dist{\tilde p}{\tilde q}{\Lob2\kappa}&=\dist{p}{q}{\spc{U}}
\\
\dist{\tilde q}{\tilde q^*}{\Lob2\kappa}&=\dist{q}{q^*}{\spc{U}}
&
\dist{\tilde p^*}{\tilde q^*}{\Lob2\kappa}&=\dist{p^*}{q^*}{\spc{U}}
\end{align*}
Given $x\in [pq]$, denote by $\tilde x$ the corresponding point on $[\tilde p\tilde q]$.
According to Majorization theorem, 
\[\dist{[pq]}{x}{\spc{U}}
\le
\dist{[\tilde p\tilde q]}{\tilde x}{\Lob2\kappa}\]
for any $x$.
Set 
\[h(x)
=
\sn\kappa
\dist{[\tilde p\tilde q]}{\tilde x}{\Lob2\kappa}.\]
By straghtforward calculations \ref{h''+kh=<0} hods
and hence the staement follows.
\qeds

\begin{thm}{Corollary}\label{cor::dist-to-convex}
Let $\spc{U}$  be a complete length $\CAT\kappa$ space
and $K\subset \spc{U}$ be a closed  locally convex set.
Then there is an open set $\Omega\supset K$
such that the function 
\[f=\sn\kappa\circ\distfun{K}{}{}\]
satisfies 
\[f''+\kappa\cdot f\ge 0\]
in $\Omega$.
\end{thm}

\parit{Proof.}
Fix $p\in K$.
By Corollary~\ref{cor:convex-balls},
all $\cBall[p,r]$ is convex for all small $r>0$.

Since $K$ is locally convex there is $r_p>0$ such that 
the intesection 
$K'=K\cap \oBall(p,r_p)$ is convex. 

Note that 
\[\dist{K}{x}{}=\dist{K'}{x}{}\]
for any $x\in \oBall(p,\tfrac{r_p}2)$.
Therefore the statement holds for 
\[\Omega=\bigcup_{p\in K}\oBall(p,\tfrac{r_p}2).\]
\qedsf



\begin{thm}{Theorem}\label{thm:local-global-convexity}
Assume $\spc{U}$ is a complete length $\CAT\kappa$ space and $K\subset \spc{U}$ is a closed connected locally convex set.
Assume $\dist{x}{y}{}<\varpi\kappa$ for any $x,y\in K$.
Then $K$ is convex.

In particular, if $\kappa\le 0$, then any closed connected locally convex set in a $\spc{U}$ is convex.
\end{thm}

The following proof is inspired by the answer of Sergei Ivanov to the question of Nathan Reading \cite{ivanov:local-global-convexity}.

\parit{Proof.}
Since $K$ is locally convex,
it is locally path connected.
Since $K$ is connected and locally path connected it is path connected.

Fix two points $x,y\in K$. 
Let us connect $x$ to $y$ by a path $\alpha\:[0,1]\to K$.
Since $\dist{x}{\alpha(s)}{}<\varpi\kappa$ for any $s$,
Theorem~\ref{thm:cat-unique} implies that the geodesic $[x\alpha(s)]$ 
uniquely defined and depends continuously on $s$.

Let $\Omega\supset K$ be the open set proveded by Corollary~\ref{cor::dist-to-convex}.
If $[xy]=[x\alpha(1)]$ does not completely lie in $K$, then 
there is a value $s\in [0,1]$ such that $[x\alpha(s)]$ 
lies in $\Omega$,
but does not completely lie in $K$.
By Corollary~\ref{cor::dist-to-convex},
the function $f=\sn\kappa\circ\distfun{K}{}{\spc{U}}$ 
satisfies differential inequality
\[f''+\kappa\cdot f\ge 0\eqlbl{f''+kappa f=<0}\]
along $[x\alpha(s)]$.

Since 
\begin{align*}
\dist{x}{\alpha(s)}{}&<\varpi\kappa,
&
f(x)&=f(\alpha(s))=0,
\end{align*}
the barrier inequality (\ref{barrier}) 
implies that $f(z)\le 0$ for $z\in [x\alpha(s)]$;
that is $[x\alpha(s)]\subset K$, a contradiction.
\qeds

\section{Development}\label{sec:development-CBA}
 
Geometrically,   the development construction (\ref{def:devel}) translates distance comparison into a local convexity statement for subsets of $\Lob2\kappa$.  Recall that a curve in $\Lob2\kappa$ is \emph{(locally) concave with respect to $p$} if (locally) its supergraph with respect to $p$ is a convex subset of $\Lob2\kappa$; see Definition~\ref{def:convex-devel}.

\begin{thm}{Development criterion\index{comparison!development comparison}}\label{thm:concave-devel} 
For a $\varpi\kappa$-geodesic space $\spc{U}$,
the following statements hold:

\begin{subthm}{locally-concave-dev}
For any $p\in \spc{U}$ and any geodesic $\gamma\:[0,T]\to\oBall(p,\varpi\kappa)$, suppose the $\kappa$-development $\tilde \gamma$ in $\Lob2\kappa$ of $\gamma$ with respect to $p$ is locally concave. 
Then $\spc{U}$ is $\CAT\kappa$.
\end{subthm}

\begin{subthm}{concave-dev} 
If $\spc{U}$ is $\CAT\kappa$, then for any $p\in \spc{U}$ and any geodesic $\gamma\:[0,T]\to\spc{U}$ 
such that the triangle $\trig{p}{\gamma(0)}{\gamma(T)}$ has perimeter $<2\cdot\varpi\kappa$,
the $\kappa$-development $\tilde \gamma$ in $\Lob2\kappa$ of $\gamma$ with respect to $p$ is concave. 
\end{subthm}

\end{thm}


\parit{Proof. (\ref{SHORT.locally-concave-dev})}.  
Set  $\gamma=\geod_{[x y]}$ and $T=\dist{x}{y}{}$. 
Let $\tilde \gamma\:[0,T]\to\Lob2\kappa$ be the concave $\kappa$-development based at $\tilde p$ of $\gamma$ with respect to $p$. 
Let us show that the function  
\[t\mapsto \angk\kappa x p{\gamma(t)}
\eqlbl{eq:ang-nondecreasing}\]   
is nondecreasing. 

For a partition $0=t^0<t^1<\dots<t^n=T$
, set 
\[\tilde y^i=\tilde \gamma(t^i)\ \ \text{and}\ \ \tau^i=\dist{\tilde y^0}{\tilde y^1}{}+\dist{\tilde y^1}{\tilde y^2}{}+\dots+\dist{\tilde y^{i-1}}{\tilde y^i}{}.\]  
Since $\tilde \gamma$ is locally concave, 
for a sufficiently fine partition the broken geodesic $\tilde y^0\tilde y^1\dots\tilde y^n$ is  locally convex  with respect to $\tilde p$. 
Alexandrov's lemma (\ref{lem:alex}), applied inductively to pairs of triangles  $\modtrig\kappa \{\tau^{i-1},\dist{p}{\tilde y^0}{},\dist{p}{\tilde y^{i-1}}{}\}$ and  $\modtrig\kappa\{\dist{\tilde y^{i-1}}{\tilde y^i}{}, \dist{p}{\tilde y^{i-1}}{},\dist{p}{\tilde y^{i}}{}\}$, shows that the sequence  $\tilde \mangle\mc\kappa\{\dist{p}{\tilde y^{i}}{};\dist{p}{\tilde y^0}{},\tau^i\}$ is nondecreasing.

Taking finer partitions and passing to the limit,   
\[\max\nolimits_i\{|\tau^i-t^i|\}\to0,\] 
we get \ref{eq:ang-nondecreasing} and 
the point-on-side comparison (\ref{cat-monoton}) follows. 



\parit{(\ref{SHORT.concave-dev}).}  
Consider a partition $0=t^0<t^1<\dots<t^n=T$, and set $x^i=\gamma(t^i)$. Construct a chain of model triangles  $\trig{\tilde p}{\tilde x^{i-1}}{\tilde x^i}=\modtrig\kappa(p x^{i-1}x^i)$ with the direction of $[\tilde p\tilde x^i]$ turning counterclockwise as $i$ grows. 
By angle comparison (\ref{cat-hinge}),
\[\mangle\hinge{\tilde x^i}{\tilde x^{i-1}}{\tilde p}+\mangle\hinge{\tilde x^i}{\tilde x^{i+1}}{\tilde p}\ge\pi.\eqlbl{eq1:concave-devel*}
\] 
Since $\gamma$ is a geodesic, 
 \[\length \gamma = \sum_{i=1}^n\dist{x^{i-1}}{x^i}{}\le \dist{p}{x^0}{}+\dist{p}{x^n}{}.
\eqlbl{eq2:concave-devel*}
\]  
By repeated application of Alexandrov's Lemma (\ref{lem:alex}), and inequality~\ref{eq2:concave-devel*}, 
\[\sum_{i=1}^n\mangle\hinge{\tilde p}{\tilde x^{i-1}}{\tilde x^i}
\le
\angk\kappa p{x^0}{x^n}\le\pi.\] 
Then by \ref{eq1:concave-devel*},  the broken geodesic $\tilde p\tilde x^0\tilde x^1\dots \tilde x^n$  is concave with respect to  $\tilde p$.

By Lemma~\ref{lem:lenght-of-convex-curve} applied to the polygons $[\tilde x^0\tilde x^1\dots \tilde x^n]$, the broken geodesics $\tilde x^0\tilde x^1\dots \tilde x^n$ approach the development of $\gamma$ with respect to $p$.  
Hence the result. 
\qeds


%%%%%%%%%%%%%%%%%%%%%%%%%%%%%%%%%%%%%%%%%%%%%%%%%%%%%%%%%%%%%%%%%%

\section{Patchwork globalization}\label{sec:patchwork}

The following definition gives an infinitesimal 
 notion of upper curvature bound at a point of a space.
 


If $\spc{U}$ is $\CAT\kappa$ space, then it is locally $\CAT\kappa$.
The converse, does not hold even for complete length space.
For example, $\mathbb{S}^1$ is locally isometric to $\RR$, and so
is locally $\CAT0$, but it is easy to find a quadruple of points in $\mathbb{S}^1$ that violates $\CAT0$ comparison.  

The following theorem was essentially proved by Alexandrov \cite[Satz 9]{alexandrov:devel}, 
gives a global condition on geodesics that is  necessary and sufficient for a locally $\CAT\kappa$ space to be globally $\CAT\kappa$. 
The proof use a thin-triangle decompositions, 
and the inheritance lemma (\ref{lem:inherit-angle}). 

\begin{thm}{Patchwork globalization theorem}\label{thm:alex-patch}
For a complete length space $\spc{U}$, the following two statements are equivalent:

\begin{subthm}{thm:alex-patch:ccat}
$\spc{U}$ is $\CAT\kappa$.
\end{subthm}
 
\begin{subthm}{thm:alex-patch:geo-uni}
$\spc{U}$ is locally $\CAT\kappa$; moreover,  pairs of points in $\spc{U}$ at distance $<\varpi\kappa$ are joined by unique geodesics, and these geodesics depend continuously on their endpoint pairs.
\end{subthm}

\end{thm}

Note that the implication (\ref{SHORT.thm:alex-patch:ccat})$\Rightarrow$(\ref{SHORT.thm:alex-patch:geo-uni})
is immediate, being the statement of Theorem~\ref{thm:cat-unique}.

\begin{thm}{Corollary}\label{cor:k-for-k}
Let $\spc{U}$ be a complete length  space 
and $\Omega\subset\spc{U}$ be an open locally $\CAT\kappa$ subset. 
Then for any point $p\in \Omega$ there is $R>0$ such that $\cBall[p,R]$ is a convex subset of $\spc{U}$ 
and $\cBall[p,R]$ is $\CAT\kappa$.
\end{thm}

\parit{Proof.}
Fix $R>0$ such that $\CAT\kappa$ comparison holds in $\oBall(p,R)$.

We may assume that $\oBall(p,R)\subset\Omega$ and $R<\varpi\Kappa$.
The same argument as in the proof of theorem on uniqueness of geodesics (\ref{thm:cat-unique}) 
shows that any two points in $\cBall[p,\tfrac R2]$ can be joint by a unique geodesic that depends continuously on the end points.

The same argument as in the proof of Corollary \ref{cor:convex-balls} shows that $\cBall[p,\tfrac R2]$ is a convex set.
Then (\ref{SHORT.thm:alex-patch:geo-uni})$\Rightarrow$(\ref{SHORT.thm:alex-patch:ccat})  of Patchwork globalization implies that $\cBall[p,\tfrac R2]$ is $\CAT\kappa$.
\qeds

The proof of Patchwork globalization uses the following construction:

\begin{thm}{Definition (Line-of-sight map)} \label{def:sight}
%Let $\spc{U}$ be a metric space in which pairs of points at distance $<\varpi\kappa$ are joined by unique geodesics and these geodesics depend continuously on their endpoint pairs. 
Let  $p$ be a point and $\alpha$ be a curve of finite length in  a length space $\spc{X}$. 
Let $\bar\alpha:[0,1]\to\spc{U}$ be the constant-speed parameterization of $\alpha$.  If   $\gamma_t\:[0,1]\to\spc{U}$ is a geodesic from $p$ to $\bar\alpha(t)$, we say 
\[[
0,1]\times[0,1]\to\spc{U}\:(t,s)\mapsto\gamma_t(s)%\eqlbl{eq:line-of-sight}
\]
is a \emph{line-of-sight map from $p$ to $\alpha$} .  
\end{thm}

\parit{Proof of Patchwork globalization theorem (\ref{thm:alex-patch}).} As was  already noted, it only remains to prove (\ref{SHORT.thm:alex-patch:geo-uni})$\Rightarrow$(\ref{SHORT.thm:alex-patch:ccat}). %!!!


\begin{wrapfigure}{r}{61mm}
\begin{lpic}[t(0mm),b(0mm),r(0mm),l(0mm)]{pics/cba-globalization-net(1)}
\lbl[t]{2,0;$x$}
\lbl[lt]{12,0;$x^{0,1}$}
\lbl[lt]{30,0;$\dots$}
\lbl[t]{54,0;$x^{0,N}=y$}
\lbl[l]{49,12;$x^{1,N}$}
\lbl[bl]{36,31,-52;$\dots$}
\lbl[b]{24,48;$x^{N,N}=p$}
\end{lpic}
\end{wrapfigure}

Let $[p x y]$ be a triangle of perimeter $<2\cdot\varpi\kappa$  in $\spc{U}$. 
According to propositions \ref{prop:k-thin} and \ref{prop:inherit-bound}, it is sufficient to show the triangle $\trig p x y$ is $\kappa$-thin.

Since pairs of points at distance $<\varpi\kappa$ are joined by unique geodesics and these geodesics depend continuously on their endpoint pairs, there is a unique and continuous line-of-sight map (\ref{def:sight}) for  $[x y]$ from $p$.    

For a partition \[0\z=t^0\z<t^1\z<\z\dots\z<t^N=1,\] 
set $x^{i,j}=\gamma_{t^i}(t^j)$. 
Since the line-of-sight map is continuous, we may assume each triangle $\trig{x^{i,j}}{x^{i,j+1}}{x^{i+1,j+1}}$ and $\trig{x^{i,j}}{x^{i+1,j}}{x^{i+1,j+1}}$ is $\kappa$-thin 
(see Proposition~\ref{prop:k-thin}).

Now we show that the $\kappa$-thin property propagates to $\trig p x y$ by repeated application of the inheritance lemma (\ref{lem:inherit-angle}):
\begin{itemize}
\item 
First, for fixed $i$, 
sequentially applying the lemma shows  that the triangles 
$\trig{x}{x^{i,1}}{x^{i+1,2}}$, 
$\trig{x}{x^{i,2}}{x^{i+1,2}}$, 
$\trig{x}{x^{i,2}}{x^{i+1,3}}$,
and so on are $\kappa$-thin. 
\end{itemize}
In particular, for each $i$, the long triangle $\trig{x}{x^{i,N}}{x^{i+1,N}}$ is $\kappa$-thin.
\begin{itemize} 
\item 
Applying the lemma again shows that the  triangles $\trig{x}{x^{0,N}}{x^{2,N}}$, $\trig{x}{x^{0,N}}{x^{3,N}}$, and so on are $\kappa$-thin. 
\end{itemize}
In particular, $\trig p x y=\trig{p}{x^{0,N}}{x^{N,N}}$ is $\kappa$-thin.
\qeds

\section{Reshetnyak's gluing theorem}\label{sec:cba-gluing}

The following theorem was proved by Yurii Reshetnyak \cite{reshetnyak:major}, assuming $\spc{U}_1$, $\spc{U}_2$ are proper and complete.  In the following form the theorem appears in \cite{BH}.

\begin{thm}{Reshetnyak's gluing theorem}\label{thm:gluing}
Suppose 
$\spc{U}_1$ and $\spc{U}_2$ are %complete 
$\varpi\kappa$-geodesic spaces 
with isometric complete $\varpi\kappa$-convex sets $A_i\subset\spc{U}_i$.  Let $\iota\:A_1\to A_2$ be an isometry.
Let $\spc{W}=\spc{U}_1\sqcup_{\iota}\spc{U}_2$;
that is, $\spc{W}$ is the gluing of $\spc{U}_1$ and  $\spc{U}_2$ along $\iota$ (see Section~\ref{sec:quotient}).

Then: 
\begin{subthm}{gluing0}
Both canonical mappings $\jmath_i\:\spc{U}_i\to\spc{W}$ are isometric 
and the images $\jmath_i(\spc{U}_i)$ are $\varpi\kappa$-convex subsets in $\spc{W}$.
\end{subthm}

\begin{subthm}{gluing2}
If $\spc{U}_1, \spc{U}_2$ are $\CAT\kappa$ spaces,
then $\spc{W}$ is a $\CAT\kappa$ space.
\end{subthm} 
\end{thm}

\parit{Proof.} 
Part (\ref{SHORT.gluing0})
follows directly from $\varpi\kappa$-convexity of $A_i$.

\parit{(\ref{SHORT.gluing2}).} 
According to (\ref{SHORT.gluing0}),
we can identify $\spc{U}_i$ with its image $\jmath_i(\spc{U}_i)$ in $\spc{W}$;
in this way, the subsets $A_i\subset \spc{U}_i$ will be identified and denoted further by $A$.
Thus   $A=\spc{U}_1\cap \spc{U}_2\subset \spc{W}$,
and $A$ is $\varpi\kappa$-convex in $\spc{W}$.

Part (\ref{SHORT.gluing2}) can be reformulated as follows:

\begin{thm}{Reformulation of \ref{gluing2}}
%Let $\spc{W}$ be a 
%complete 
%length space having two 
%closed 
%$\varpi\kappa$-convex sets $\spc{U}_1,\spc{U}_2\subset\spc{W}$ such that
%$\spc{U}_1\cup\spc{U}_2=\spc{W}$ and $\spc{U}_1$, $\spc{U}_2$ are $\CAT\kappa$ spaces.
%Then $\spc{W}$ is a $\CAT\kappa$ space.
Let $\spc{W}$ be a 
length space having two 
$\varpi\kappa$-convex subsets $\spc{U}_1,\spc{U}_2\subset\spc{W}$ such that
$\spc{W}=\spc{U}_1\cup\spc{U}_2$, $A=\spc{U}_1\cap \spc{U}_2$ is complete and  $\varpi\kappa$-convex in $\spc{W}$, and $\spc{U}_1$, $\spc{U}_2$ are $\CAT\kappa$ spaces.
Then $\spc{W}$ is a $\CAT\kappa$ space.
\end{thm}

\begin{clm}{}\label{clm:geod-gluing}
If $\spc{W}$ is $\varpi\kappa$-geodesic, then $\spc{W}$ is $\CAT\kappa$.
\end{clm}



\begin{wrapfigure}{r}{20mm}
\begin{lpic}[t(0mm),b(0mm),r(0mm),l(0mm)]
{pics/resh-glue(1)}
\lbl[b]{6.5,22.5;$x^0$}
\lbl[t]{2,0;$x^1$}
\lbl[tl]{16,0;$x^2$}
\lbl[r]{3,12;$z^1$}
\lbl[l]{14,10;$z^2$}
\end{lpic}
\end{wrapfigure}

Indeed, 
according to \ref{prop:k-thin},
it is sufficient to show that any triangle $\trig {x^0}{x^1}{x^2}$ of perimeter $<2\cdot \varpi\kappa$ 
in $\spc{W}$ is $\kappa$-thin.
This is obviously true if all three points $x^0$, $x^1$, $x^2$ lie in a single $\spc{U}_i$.
Thus, without loss of generality, we may assume that $x^0\in\spc{U}_1$ and $x^1,x^2\in\spc{U}_2$.

Choose points $z^1,z^2\in A=\spc{U}_1\cap\spc{U}_2$ 
lying respectively on the sides $[x^0x^1], [x^0x^2]$.
Note that all distances between any pair of points from $x^0$, $x^1$, $x^2$, $z^1$, $z^2$ are less than $\varpi\kappa$.
Therefore
\begin{itemize}
\item triangle $\trig{x^0}{z^1}{z^2}$ lies in $\spc{U}_1$,
\item both triangles $\trig{x^1}{z^1}{z^2}$ and $\trig{x^1}{z^2}{x^2}$ lie in $\spc{U}_2$.
\end{itemize}
In particular each triangle $\trig{x^0}{z^1}{z^2}$,
$\trig{x^1}{z^1}{z^2}$, $\trig{x^1}{z^2}{x^2}$ is $\kappa$-thin.

Applying the inheritance lemma for thin triangles (\ref{lem:inherit-angle}) twice, 
we get that $\trig {x^0}{x^1}{z^2}$ 
and consequently $\trig {x^0}{x^1}{x^2}$ is $\kappa$-thin.
\claimqeds

\begin{clm}{}\label{clm:geod-gluing0 }
$\spc{W}$ is geodesic if $\kappa\le0$.
\end{clm}
For $p_1\in \spc{U}_1$, $p_2\in \spc{U}_2$, we may choose a sequence $z^i\in A$ such that $|p_1z^i|+|p_2z^i|$
 converges to $|p_1p_2|$, and $|p_1z^i|$ and $|p_2z^i|$ converge.  
 Since $A$ is complete, it suffices to show $z^i$ is a Cauchy sequence.  
 In that case, the limit point $z$ of $z^i$ satisfies $|p_1z|+|p_2z|=|p_1p_2|$, so the geodesics $[p_1z]$ in $\spc{U}_1$ and $[p_2z]$ in $\spc{U}_2$ together give a geodesic $[p_1p_2]$ in $\spc{U}$.  
 
 Suppose $z^i$ is not a Cauchy sequence.  Then there are subsequences  $x^i$ and $y^i$ of $z^i$ satisfying  $\lim|x^iy^i|>0$. Let $m^i$ be the midpoint of $[x^iy^i]$. Since $|p_1m^i|+|p_2m^i| \ge |p_1p_2|$, and  $|p_1x^i|+|p_2x^i|$ and  $|p_1y^i|+|p_2y^i|$
 converge to $|p_1p_2|$, then for  any $\epsilon >0$, we may assume (taking subsequences and possibly relabeling $p_1$ and $p_2$)
 \[
 |p_1m^i\ge  |p_1x^i|-\epsilon,  |p_1m^i\ge  |p_1y^i|-\epsilon.
 \]
 
 Since triangle $\trig{p^1}{x^i}{y^i}$ is thin, the analogous inequalities hold for the Euclidean model triangle  $\trig{\tilde p^1}{\tilde x^i}{\tilde y^i}$.  
 Then there is a nondegenerate limit triangle $\trig{p}{x}{y}$ in the Euclidean plane satisfying $|px|=|py|\le|pm|$ where $m$ is the midpoint of $[xy]$.  This  contradiction proves the claim.
 
\claimqeds

Finally suppose $\kappa>0$; by  scaling, take $\kappa=1$. Consider the Euclidean cones $\Cone\spc{U}_i$ (see Section \ref{sec:Euclid-constructions}).
By Theorem \ref{thm:warp-curv-bound:cbb:S}, $\Cone\spc{U}_i$ is a $\CAT0$ space for $i=1,2$. Geodesics of $\Cone\spc{U}_i$ that do not pass through the vertex of the cone correspond, under projection from the vertex, to geodesics of $\spc{U}_i$ of length $<\pi$. 
  It follows that $\Cone A$ is convex in $\Cone\spc{U}_1$ and $\Cone\spc{U}_2$. By the cone distance formula  , $\Cone A$ is complete since $A$ is complete.
    Gluing on $\Cone A$ and applying \ref{clm:geod-gluing} and \ref{clm:geod-gluing0 } for $\kappa=0$, we find that 
$\Cone\spc{W}$ is a $\CAT0$ space.  By Theorem \ref{thm:warp-curv-bound:cbb:S}, $\spc{W}$ is a $\CAT1$ space.
\qeds
%
%Now we come back to the general case;
%that is, $\spc{W}$ is not necessury $\varpi\kappa$-geodesic.
%According to Corollary \ref{cor:ulara-geod},
%the ultrapower $\spc{W}^\o$ is geodesic.
%It is easy to see that 
%\begin{itemize}
%\item $\spc{U}_i^\o$ is weakly $\varpi\kappa$-convex subset of $\spc{W}^\o$,
%\item $A^\o=\spc{U}_1^\o\cap\spc{U}_2^\o$ is also a weakly $\varpi\kappa$-convex set.
%\end{itemize}
%%???IT SEEMS THAT ULTRAPRODUCT FOR SUBSETS ARE NOT REALLY DEFINED YET
%From Proposition~\ref{prop:CAT^omega}, $\spc{U}_i^\o$ is a complete length $\CAT\kappa$ space.
%By uniqueness of geodesics (\ref{thm:cat-unique})
% $A^\o$ is $\varpi\kappa$-convex in both $\spc{U}_i$ and
%consequently both $\spc{U}_i^\o$ are $\varpi\kappa$-convex subsets in $\spc{W}^\o$.
%
%From Claim~\ref{clm:geod-gluing}, 
%we get that $\spc{W}^\o$ is a complete length $\CAT\kappa$ space.
%Applying the Proposition~\ref{prop:CAT^omega} once more, 
%we get $\spc{W}$ is a complete length $\CAT\kappa$ space.
%\qeds 

\begin{thm}{Exercise}\label{ex:two-rays}
Let $Q$ be the nonconvex subset of the plane 
bounded by two rays $\gamma_1$ and $\gamma_2$
with common starting point and angle $\alpha$ between them.
Assume $\spc{U}$ is a complete length $\CAT0$ space
and $\gamma_1',\gamma_2'$ are two rays in $\spc{U}$ with common
starting point and angle $\alpha$ between them.
Show that the space glued from $Q$ and $\spc{U}$ along the corresponding rays is a $\CAT{0}$ space.
\end{thm}

\begin{thm}{Exercise}\label{ex:reshetnyak-doubling}
Assume $\spc{U}$ is complete length $\CAT0$ space and $A\subset \spc{U}$ is a closed subset.
Assume that the doubling of $\spc{U}$ in $A$ is $\CAT0$. 
Show that $A$ is a convex set of $\spc{U}$.
\end{thm}

\begin{thm}{Exercise} Let $\spc{U}$ be a complete length $\CAT0$ space.
Assume $\tilde{\spc{U}}\to \spc{U}$ is a metric covering branching along a geodesic.
Show that $\tilde{\spc{U}}$ is $\CAT0$.

More generally, assume $A\subset \spc{U}$ is a closed convex subset and $f\:\spc{X}\to \spc{U}\backslash A$ is a metric cover.
Denote by $\bar{\spc{X}}$ the completion of $\spc{X}$, and 
$\bar f\:\bar{\spc{X}}\to \spc{U}$ the continuous extension of $f$.
Let $\tilde{\spc{U}}$ be the space glued from $\bar{\spc{X}}$ and $A$ by identifying $x$ and $\bar f(x)$ if $\bar f(x)\in A$.
Show that $\tilde{\spc{U}}$ is $\CAT0$.
\end{thm}



%%%%%%%%%%%%%%%%%%%%%%%%%%%%%%%%%%%%%%%%%%%%%%%%%%%%%%%%%%%%%%%%%%%%%%

\section{Space of geodesics}\label{sec:geod-space}

In this section we prove a  ``no-conjugate-point'' theorem for spaces with upper curvature bounds, and derive from it a number of statements about
local geodesics.  These statements will be used in the proof of the Hadamard--Cartan theorem (\ref{thm:hadamard-cartan}) and Lifting
 globalization theorem (\ref{thm:globalization-lift}), in much the same way as  the exponential map is used in Riemannian geometry.

\begin{thm}{Proposition}\label{prop:geo-complete}
Let $\spc{U}$ be a is locally $\CAT\kappa$ space.
 Let $\gamma_n\:[0,1]\to\spc{U}$ be a sequence of local geodesic paths converging to a path $\gamma_\infty\:[0,1]\to\spc{U}$.
Then $\gamma_\infty$ is a local geodesic path.
Moreover 
\[\length\gamma_n\to\length\gamma_\infty\]
as $n\to\infty$.
\end{thm}

\parit{Proof.} 
Fix $t\in[0,1]$.
By Corollary~\ref{cor:k-for-k}, we may choose $R$ satisfying $0<R<\varpi\Kappa$,
and such that
the ball $\spc{B}=\oBall(\gamma_\infty(t),R)$ is a convex subset of $\spc{U}$ and forms a $\CAT\kappa$ space.

% ??? I replaced a sentence that seemed circular, because you need length $<\varpi\kappa$ in order to apply Jensen's inequality in order to prove length $<\varpi\kappa$. S.

%Take $R<\varpi\kappa/2$, then the function comparison (\ref{thm:function-comp}) implies that the length of any local geodesic is at most $2\cdot R<\varpi\kappa$. Therefore, by Corollary~\ref{cor:loc-geod-are-min}, any local geodesic in $\spc{B}$ is minimizing.

A local geodesic segment  with length less than $R/2$ that intersects $\oBall(\gamma_\infty(t),R/2)$, cannot leave $\spc{B}$ and hence  is  minimizing by Corollary~\ref{cor:loc-geod-are-min}.  In particular, for all sufficiently large $n$, any subsegment  of $\gamma_n$ through $\gamma_n(t)$  with length $<R/2$ is a geodesic.


Since $\spc{B}$ is $\CAT\kappa$, geodesic segments in $\spc{B}$ depend uniquely and continuously on their endpoint pairs by Theorem~\ref{thm:cat-unique}.  
Thus there is a subinterval $\II$ of $[0,1]$,
that contains a neighborhood of $t$ in $[0,1]$
and such that $\gamma_n|\II$ is minimizing for all large $n$.
It follows that the restriction $\gamma_\infty|\II$ is a geodesic,
and therefore $\gamma_\infty$ is a local geodesic.
%It follows that  $[0,1]$ is covered by finitely many intervals $\II_i$,  that are open in $[0,1]$, where $\gamma_\infty|\II_i$ is  a geodesic that is the limit of geodesics $\gamma_n|\II_i$. Therefore   $\gamma_\infty$ is a local geodesic path with $\length\gamma_n\to\length\gamma_\infty$.
%!!! I change it a bit in the beginning and at the end.
\qeds


The following theorem was proved by the first author and Richard Bishop \cite{a-b:h-c}.
In analogy with Riemannian geometry, the main statement of the following theorem could be restated as: 
\emph{In a space of curvature $\le\kappa$, two points cannot be conjugate along a local geodesic of length $<\varpi\kappa$.}


\begin{thm}{No-conjugate-point theorem}
\label{thm:no-conj-pt}
Suppose $\spc{U}$ is a locally complete locally $\CAT\kappa$ space.
Let $\gamma\:[0,1]\to\spc{U}$ be a local geodesic path with length $<\varpi\kappa$.
Then for some neighborhoods $\Omega^0\ni \gamma(0)$ and $\Omega^1\ni\gamma(1)$ 
there is a unique continuous map $\Omega^0\times \Omega^1\times[0,1]\to\spc{U}$
\[(x,y,t)\mapsto\gamma_{x y}(t)\]
such that 
$\gamma_{x y}\:[0,1]\to\spc{U}$ is a local geodesic path with 
$\gamma_{x y}(0)=x$ and 
$\gamma_{x y}(1)=y$ for each $(x,y)\in\Omega^0\times\Omega^1$,
and the family $\gamma_{x y}$ contains $\gamma$.  
Moreover, we can assume that the map 
\[(x,y,t)\mapsto\gamma_{x y}(t)\:\Omega^0\times\Omega^1\times[0,1]\to\spc{U}\] 
is $\Lip$-Lipschitz
for any
$\Lip>\max\set{\tfrac{\sn\kappa r}{\sn\kappa \ell}}{0\le r\le \ell}$.
\end{thm}

The following lemma was suggested to us by Alexander Lytchak.
The proof proceeds by piecing together $\CAT{\kappa}$  neighborhoods of points on a curve to construct a new $\CAT{\kappa}$ space.  
Exercise \ref{ex:cats-cradle} is inspired by the original idea of the proof of No-conjugate-point theorem (\ref{thm:no-conj-pt}) given in \cite{a-b:h-c}.

\begin{thm}{Patchwork along a curve}
\label{lem:patch}
Let $\spc{U}$ be a locally complete locally $\CAT\kappa$ space, 
and $\alpha\:[a,b]\to\spc{U}$ be a curve.

Then there is a complete length $\CAT\kappa$ space $\spc{N}$
with an open set $\hat\Omega\subset \spc{N}$,
a curve $\hat\alpha\:[0,1]\to\hat\Omega$,
and an open locally isometric immersion 
 $\map\:\hat\Omega\looparrowright\spc{U}$ such that
$\map\circ\hat\alpha=\alpha$.

Moreover if $\alpha$ is simple, then one can assume in addition that $\map$ is an open embedding;
thus $\hat\Omega$ is locally isometric to a neighborhood of $\Omega=\map(\hat\Omega)$ of $\alpha$.
\end{thm}



\parit{Proof.} 
According to Corollary~\ref{cor:k-for-k},
for any $t\in[a,b]$ there is $r(t)>0$ 
such that the closed ball
$\cBall[\alpha(t),r(t)]$ is a  convex set  that forms a complete length $\CAT\kappa$ space.

Choose balls $\spc{B}^i=\cBall[\alpha(t^i),r(t^i)]$
for some partition $a\z=t^0<t^1<\dots<t^n\z=b$
in such a way that 
$\Int\spc{B}^i\supset \alpha([t^{i-1},t^i])$ for all $i>0$.

Consider the disjoint union $\bigsqcup_i\spc{B}^i=\set{(i,x)}{x\in\spc{B}^i}$ with the minimal equivalence relation $\sim$ such that $(i,x)\sim(i-1,x)$ for all $i>0$.
Let  $\spc{N}$ be the space obtained by gluing the $\spc{B}^i$ along $\sim$.
Note that $A^i=\spc{B}^i\cap\spc{B}^{i-1}$ is convex in $\spc{B}^i$ and in $\spc{B}^{i-1}$.
Applying the Reshetnyak gluing theorem (\ref{thm:gluing}) $n$ times, 
we conclude that $\spc{N}$ is a complete length $\CAT\kappa$ space.
%???WE MAY NEED TO STATE THIS CONSTRUCTION AS A COROLLARY FROM Reshetnyak gluing theorem --- I'M SURE WE WILL NEED IT ONCE MORE...
%OK LET'S MOVE IT TO GLUING  WHEN WE NEED IT AGAIN

\begin{center}
\begin{lpic}[t(0mm),b(0mm),r(0mm),l(0mm)]{pics/patching-balls(1)}
\lbl{7,17;$\spc{B}^0$}
\lbl{25,4;{$\spc{B}^1$}}
\lbl{47,2;$\dots$}
\lbl{58,2.5,4;$\dots$}
\lbl{86,17;$\spc{B}^n$}
\end{lpic}
\end{center}

For $t\in[t^{i-1},t^i]$, let $\hat\alpha(t)$  be the equivalence class of $(i,\alpha(t))$ in $\spc{N}$.
Let $\hat\Omega$ be the $\eps$-neighborhood of $\hat\alpha$ in $\spc{N}$, where $\eps>0$ is chosen so that $\oBall(\alpha(t),\eps)\subset\spc{B}^i$ for all $t\in[t^{i-1},t^i]$.

Define $\map\:\hat\Omega\to\spc{U}$
by sending the equivalence class of $(i,x)$ to $x$.
It is straightforward to check that $\map\:\spc{N}\to\spc{U}$, $\hat\alpha\:[0,1]\to\spc{N}$ and $\hat\Omega\subset\spc{N}$ satisfy the conclusion of the main part of the lemma.

To prove the final statement in the lemma,
we only have to choose $\eps>0$ so that in addition $\dist{\alpha(\tau)}{\alpha(\tau')}{}>2\cdot\eps$ if $\tau\le t^{i-1}$ and $t^i\le\tau'$ for some $i$.
\qeds


\parit{Proof of No-conjugate-point theorem \ref{thm:no-conj-pt}.}
Apply patchwork along $\gamma$ (\ref{lem:patch}). 
\qeds



The No-conjugate-point theorem (\ref{thm:no-conj-pt}) allows us to move a local geodesic path 
so that its endpoints follow given trajectories.
The following corollary describe how this process might terminate. 

\begin{thm}{Corollary}\label{cor:geo-hom}
Let $\spc{U}$ be a locally complete locally $\CAT\kappa$ space.
Suppose $\gamma\:[0,1]\to\spc{U}$ is a local geodesic path with length $< \varpi\kappa$.  Let $\alpha^i\:[0,1]\to \spc{U}$, for $i=0,1$, be paths starting at $\gamma(0)$ and $\gamma(1)$ respectively.  

Then there is a uniquely determined pair consisting of an interval $\II $ satisfying $0\in \II\subset[0,1]$, and a continuous family of local geodesic paths $\gamma_t\:[0,1]\to \spc{U}$ for  $t\in \II$, such that  

\begin{subthm}{cor:geo-hom-length}
$\gamma_0=\gamma$, $\gamma_t(0)=\alpha^0(t)$, $\gamma_t(1)=\alpha^1(t)$, and $\gamma_t$ has length $< \varpi\kappa$,
\end{subthm} 

\begin{subthm}{cor:geo-hom-cauchy}
if $\II\ne [0,1]$, then $\II=[0, a)$, where either $\gamma_t$ converges uniformly to a local geodesic $\gamma_a$ of length $\varpi\kappa$, or 
for some fixed $s\in [0,1]$ the curve $\gamma_t(s):[0,a)\to\spc{U}$ is a Lipschitz curve with no limit as $t\to a-$.
%???DO WE REALLY NEED SUCH GENERALITY
\end{subthm}

\end{thm}


\parit{Proof.} Uniqueness follows from  Theorem \ref{thm:no-conj-pt}.

Let $\II$ be the maximal interval for which there is a family $\gamma_t$ satisfying condition (\ref{SHORT.cor:geo-hom-length}). 
By Theorem~\ref{thm:no-conj-pt}, such an interval exists and is open in $[0,1]$.  Suppose $\II\ne[0,1]$. Then  $\II=[0,a)$ for some $0<a\le 1$.  It suffices to show that $\II$ satisfies condition (\ref{SHORT.cor:geo-hom-cauchy}).

For each fixed $s\in [0,1]$, define the curve $\alpha_s:[0,a)\to\spc{U}$ by $\alpha_s(t)=\gamma_t(s)$. 
By Theorem~\ref{thm:no-conj-pt}, 
$\alpha_s$ is $ \Lip$-Lipschitz for some $\Lip$.  

If $\alpha_s$  for some value of $s$ does not converge as $t\to a-$, then condiion (\ref{SHORT.cor:geo-hom-cauchy}) holds.  If each $\alpha_s$  converges as $t\to a-$,  
 then $ \gamma_t$ converges as $t\to a-$, say to $\gamma_a$.
By  Proposition~\ref{prop:geo-complete}, $\gamma_a$ is a local geodesic path and\[\length\gamma_t\to\length\gamma_a\le \varpi\kappa.\]
By maximality of $\II$, $\length\gamma_a=\varpi\kappa$ and so condition (\ref{SHORT.cor:geo-hom-cauchy}) again holds.
\qeds

\begin{thm}{Corollary}\label{cor:homotopy-from-p}
Suppose $\spc{U}$ is a complete length  locally $\CAT\kappa$ space 
and $\alpha\:[0,1]\to \spc{U}$ is a path of length $\ell<\varpi\kappa$.
Then  for all $t\in[0,1]$,

\begin{subthm}{cor:homotopy-from-p-exist}
there is unique homotopy of local geodesic paths $\gamma_t\:[0,1]\to \spc{U}$
such that $\gamma_0(t)=\gamma_t(0)=\alpha(0)$ and $\gamma_t(1)=\alpha(t)$,
\end{subthm}

\begin{subthm}{cor:homotopy-from-p-length}
\[\length\gamma_t\le\length(\alpha|[0,t]),\]
and  equality holds for given $t$ if and only if the restriction $\alpha|[0,t]$ is a reparametrization of $\gamma_t$.
\end{subthm}
\end{thm}

\parit{Proof.}
%Fix $\bar r<r$ so that 
%\[\dist{p}{\alpha(0)}{}+\dist{p}{\alpha(1)}{}+\ell\le 2\cdot \bar r.\]
Apply Corollary \ref{cor:geo-hom}, setting $\alpha^0(t)=\alpha(0)$ for all $t\in [0,1],$ and $\alpha^1=\alpha$. Since $\spc{U}$ is complete,  there is an interval $\II$ such that statement (\ref{SHORT.cor:homotopy-from-p-exist}) holds for all $t\in\II$, and either $\II=[0,1]$ or $\II=[0,a)$ where $\gamma_t$ converges uniformly to a local geodesic $\gamma_a$ of length $\varpi\kappa$.
%for $t$ in a maximal subinterval $\II$ of $[0,1]$.
%From Theorem~\ref{thm:no-conj-pt}, $\II$ has to be open in $[0,1]$.

By the patchwork along a curve (\ref{lem:patch}), the values of $t$ for which condition (\ref{SHORT.cor:homotopy-from-p-length}) holds form an open subset of $\II$ containing $0$; clearly this subset is also closed in $\II$. Therefore (\ref{SHORT.cor:homotopy-from-p-length}) holds on all of $\II$. 
 
Sinse $\ell<\varpi\kappa$, 
$\II=[0,1]$ and the result follows. 
\qeds

The no-conjugate-point theorem  (\ref{thm:no-conj-pt}) immediately gives an extension of the first variation formula (\ref{thm:1st-var-cba}) from geodesics in $\CAT\kappa$ spaces to local geodesics in locally $\CAT\kappa$ spaces.

  \begin{thm}{Corollary}\label{cor:1st-var++cba}
Suppose $\spc{U}$ is a locally complete locally $\CAT\kappa$ space. 
For any geodesic $[px]$ in $\spc{U}$ and local geodesic $\gamma\:[0,1]\to \spc{U}$ from $p$ to $q$ of  length $<\varpi\kappa$, let  $\gamma_t\:[0,1]\to \spc{U}$ be a continuous family of local geodesic paths with $\gamma_0=\gamma$, $\gamma_t(0)=\geod_{[p x]}(t)$ and $\gamma_t(1)=q$.  Then
\[\length(\gamma_t)
=
\length(\gamma)-t\cdot\cos\mangle(\gamma,[px])+o(t).
\]

\end{thm}




%%%%%%%%%%%%%%%%%%%%%%%%%%%%%%%%%%%%%%%%%%%%%%%%%%%%%%%%%%%%%%%%%%%%%%%%%%%%%

\section{Lifting globalization}\label{sec:cat-globalize}

The Hadamard--Cartan theorem (\ref{thm:hadamard-cartan}) states that 
the universal metric cover of a complete space locally $\CAT0$ space is $\CAT0$.
The following Lifting globalization theorem gives an appropriate generalization of the above statement to arbitrary curvature bounds.



\begin{thm}{Lifting globalization theorem}
\label{thm:globalization-lift}
Suppose $\spc{U}$ is a complete length locally $\CAT\kappa$ space and  $p\in\spc{U}$.
Then there is a complete length $\CAT\kappa$ space $\spc{B}$, 
with a point $\hat p$ such that $\dist{\hat p}{\hat x}{\spc{B}}\le \tfrac{\varpi\kappa}2$ for any $\hat x\in \spc{B}$ and 
there is a locally isometric map $\map\:\spc{B}\to\spc{U}$
such that $\map(\hat p)=p$ and the following lifting property holds: 
for any path $\alpha\:[0,1]\to\spc{U}$ with $\alpha(0)=p$ and $\length\alpha<\varpi\kappa/2$, 
%???Maybe we want $\length\alpha\le\varpi\kappa/2$???
there is a unique path $\hat\alpha \:[0,1]\to \spc{B}$ such that $\hat\alpha(0)=\hat p$ 
and $\map\circ\hat\alpha\equiv\alpha$.
\end{thm}

Before proving the theorem we state and proof its corollary.

\begin{thm}{Corollary}\label{cor:loc-CAT(k)}
Suppose $\spc{U}$ is a complete length locally $\CAT\kappa$ space.
Then for any $p\in\spc{U}$ there is $\rho_p>0$
such that $\cBall[p,\rho_p]$ is a complete length $\CAT\kappa$ space.

Moreover, we can assume that $\rho_p<\tfrac{\varpi\kappa}2$
for any $p$ and the function $p\mapsto\rho_p$ is 1-Lipschitz.
\end{thm}

\parit{Proof.} 
Assume $\map\:\spc{B}\to \spc{U}$ 
and $\hat p\in \spc{B}$
are provided by Lifting globalization theorem
(\ref{thm:globalization-lift}).

Since $\map$ is local isometry,
we can choose $r>0$ so that the restriction of $\map$ to $\cBall[\hat p,r]$ is distance preserving.
By Lifting globalization, the image  $\Phi(\cBall[\hat p,r])$ coincides with the ball
$\cBall[p,r]$.
This proves the first part of theorem.

To prove the second part, lat us choose $\rho_p$ to be maximal value $\le\tfrac{\varpi\kappa}2$ such that $\cBall[p,\rho_p]$ is a complete length $\CAT\kappa$ space.
By Corollary~\ref{cor:convex-balls}, the ball
\[\cBall[q,\rho_p-\dist{p}{q}{}]\] 
is convex in $\cBall[p,\rho_p]$.
Therefore  
\[\cBall[q,\rho_p-\dist{p}{q}{}]\] is a complete length $\CAT\kappa$ space
for any $q\in \oBall(p,\rho_p)$.
In particular $\rho_q\ge \rho_p-\dist{p}{q}{}$ for any $p,q\in\spc{U}$.
Hence the second statement follows.
\qeds




The proof of the lifting globalization theorem relies heavily on the properties of the space of local geodesic paths discussed in Section~\ref{sec:geod-space}.
The following lemma proved by the first author and Richard Bishop \cite{a-b:cbc} 
is a key step in the proof.

\begin{thm}{Radial lemma}\label{lem:radial-glob}
Let $\spc{U}$ be a length locally $\CAT\kappa$ space,
and suppose $p\in\spc{U}$, $R\le\varpi\kappa/2$.  Assume the ball  $\cBall[p,\bar{R}]$ is complete
for all $\bar{R}<R$, and  there is a unique geodesic path, $\geodpath_{[p x]}$, from $p$ to any point $x\in\oBall(p,R)$, 
that depends continuously on $x$.
Then $\oBall(p,R)$ is a $\varpi\kappa$-geodesic $\CAT\kappa$ space.
\end{thm}
 
\parit{Proof.}
Without loss of generality, we may assume  $\spc{U}=\oBall(p,\varpi\kappa)$.

Set $f=\md\kappa\circ\distfun{p}{}{}$.  Let us show that
\[f''+\kappa\cdot f\ge 1.
\eqlbl{eq:rad-conv}\]



Fix $z\in \spc{U}$.
We will apply Theorem \ref{thm:no-conj-pt} for the unique geodesic path $\gamma$
from $p$ to $z$.  
The  notations $\Omega^i$, 
%$\iota_i$, 
$\gamma_{x y}$, $\spc{N}$, $\hat{x}$, $\hat{y}$ will be  as in the theorem.
In particular, $z\in\Omega^1$.

By assumption,
$\gamma_{p y}=\geodpath_{[p y]}$ for any $y\in\Omega^1$. 
Consequently,
 $f(y)\z=\md\kappa\dist[{{}}]{\hat{p}}{\hat{y}}{\spc{N}}$.
From the function comparison (\ref{thm:function-comp}) applied in $\spc{N}$,
we have $f''+\kappa\cdot f\ge 1$ in $\Omega^1$.

Fix $\bar R<R$, and consider  the complete closed ball $\cBall[ p,\bar R]\subset\spc{U}$.
The proof of the following claim takes most of the rest of proof of the theorem.

\begin{clm}{}\label{clm:B-is-convex}
$\cBall[ p,\bar R]$ is a convex set in $\spc{U}$.
\end{clm}

Choose arbitrary $x,z\in \cBall[ p,\bar R]$.
First note that \ref{eq:rad-conv} implies:

\begin{clm}{}\label{clm:B-is-almost-convex}
If $ \gamma\:[0,1]\to\spc{U}$ 
is a local geodesic path from $x$ to $z$ and  
$\length \gamma<\varpi\kappa$,  
then $\length \gamma < 2\cdot \bar R$ 
and $ \gamma$ lies completely in $\cBall[ p,\bar R]$.
\end{clm}

Note that  $\dist{x}{z}{}<\varpi\kappa$.
Thus, to prove Claim~\ref{clm:B-is-convex}, it is sufficient to show that there is a geodesic path from $x$ to $z$.
By Corollary~\ref{cor:homotopy-from-p},

\begin{clm}{}\label{clm:loc-geod<path}
Given a path $\alpha\:[0,1]\to\spc{U}$ from $x$ to $z$ with $\length\alpha<\varpi\kappa$,
there is a local geodesic path $\gamma$ from $x$ to $z$ such that
\[\length\gamma\le\length\alpha.\]

\end{clm}

Further, let us prove the following.

\begin{clm}{}\label{clm:unique-loc-geod}
There is unique local geodesic path $\gamma_{x z}$ in $\cBall[ p,\bar R]$ from $x$ to $z$.
\end{clm}

Denote by $\Delta_{x z}$ the set of all local geodesic paths of $\cBall[ p,\bar R]$ from $x$ to $z$.
 %Applyinglary \ref{subcor:hom-straight} twice in the subspace $\cBall[ p,r]$,first for a path from $x$ to $ p$ 
%and second for a path from $z$ to $p$ we get bijections $\Delta_{x z}\to\Delta_{p z}\to\Delta_{ p p}$.
%According to \ref{eq:rad-conv}, 
%$\Delta_{ p p}$ contains only the constant path; hence Claim~\ref{clm:unique-loc-geod} follows.
By Corollary \ref{cor:geo-hom}, there is a  bijection $\Delta_{x z}\to\Delta_{ p p}$.
According to \ref{eq:rad-conv}, 
$\Delta_{ p p}$ contains only the constant path. Claim~\ref{clm:unique-loc-geod} follows.


Note that 
claims~\ref{clm:B-is-almost-convex}, 
\ref{clm:loc-geod<path} 
and \ref{clm:unique-loc-geod}
imply that $\gamma_{x z}$ is minimizing; hence Claim~\ref{clm:B-is-convex}.

Further, Claim~\ref{clm:B-is-almost-convex} and the no-conjugate-point theorem (\ref{thm:no-conj-pt}) together 
imply that the map $(x,z)\mapsto\gamma_{x z}$ is continuous.

Therefore by the patchwork globalization theorem (\ref{thm:alex-patch}), 
$\cBall[ p,\bar R]$ is a $\varpi\kappa$-geodesic $\CAT\kappa$ space.

Since
\[\oBall( p,R)
=
\bigcup_{\bar R < R}\cBall[ p,\bar R],\] 
then $\oBall( p,R)$ is convex in $\spc{U}$ and 
$\CAT\kappa$ comparison holds  for any quadruple in $\oBall( p,R)$.
Therefore $\oBall( p,\varpi\kappa/2)$ is $\CAT\kappa$.
\qeds


In the following proof, we construct a space $\mathfrak{G}_p$ of  local geodesic paths that start at $p$.
The space $\mathfrak{G}_p$ comes with 
a marked point $\hat p$ 
and the right-hand-endpoint map $\map\:\mathfrak{G}_p\to\spc{U}$.
In the Riemannian case, one can think of
the map $\map$  as $\exp_p$, and
the space $\mathfrak{G}_p$ as the ball of radius $\varpi\kappa$ in the tangent space at $p$, equipped with the metric pulled back by $\exp_p$.

We are going to set $\spc{B}=\oBall[\hat p,\varpi\kappa/2]\subset \mathfrak{G}_p$,
and use the radial lemma (\ref{lem:radial-glob}) to prove that $\spc{B}$ is a $\varpi\kappa$-geodesic $\CAT\kappa$ space.

\parit{Proof of Lifting globalization theorem \ref{thm:globalization-lift}.}
Suppose $\hat\gamma$ is a homotopy of local geodesic paths that start at $p$.  Thus the map 
\[\hat\gamma\:(t,\tau)\mapsto\hat\gamma_t(\tau)\:[0,1]\times[0,1]\to\spc{U}\] 
is continuous,
and the following holds for each $t$:
\begin{itemize}
\item $\hat\gamma_t(0)=p$,
\item $\hat\gamma_t\:[0,1]\to\spc{U}$ is a local geodesic path in $\spc{U}$.
\end{itemize}

Let $\theta(\hat\gamma)$ denote the length of the path $t\mapsto\hat\gamma_t(1)$.

Let $\mathfrak{G}_p$ be the set of all local geodesic paths 
with length $<\varpi\kappa$ in $\spc{U}$ that start at $p$.
Denote by $\hat p\in \mathfrak{G}_p$ the constant path $\hat p(t)\equiv p$.
Given $\alpha,\beta\in \mathfrak{G}_p$ define
\[
\dist{\alpha}{\beta}{\mathfrak{G}_p}
=
\inf_{\hat\gamma} \{\theta(\hat\gamma)\},\]
with infimum  taken along all homotopies 
$\hat\gamma\:[0,1]\times[0,1]\to\spc{U}$ 
such that 
$\hat\gamma_0=\alpha$, 
$\hat\gamma_1=\beta$ 
and $\hat\gamma_t\in \mathfrak{G}_p$ for all $t\in[0,1]$.

From Theorem~\ref{thm:no-conj-pt}, we have $\dist{\alpha}{\beta}{\mathfrak{G}_p}>0$ for distinct $\alpha$ and $\beta$;
that is,

\begin{clm}{}
$\dist{{*}}{{*}}{\mathfrak{G}_p}$ is a metric on $\mathfrak{G}_p$.
\end{clm}

Further, again from Theorem~\ref{thm:no-conj-pt}, we have

\begin{clm}{}\label{clm:loc-iso}
The map
\[\map\:\xi\mapsto\xi(1)\:\mathfrak{G}_p\to\spc{U}\]
is a local isometry.
In particular, $\mathfrak{G}_p$ is locally $\CAT\kappa$.
\end{clm}

Let $\alpha\:[0,1]\to\spc{U}$ be a path with $\length\alpha<\varpi\kappa$ and $\alpha(0)=p$.
The homotopy constructed in Corollary~\ref{cor:homotopy-from-p} 
can be regarded as a path in $\mathfrak{G}_p$, say $\hat\alpha\:[0,1]\to \mathfrak{G}_p$,
such that $\hat\alpha(0)=\hat p$ and $\map\circ\hat\alpha=\alpha$; in particular $\hat\alpha(1)=\alpha$. 
By \ref{clm:loc-iso}, 
\[\length\hat\alpha=\length\alpha.\]
Moreover, it follows that $\alpha$ is a local geodesic path of $\spc{U}$  if and only if $\hat\alpha$ is a local geodesic path of $\mathfrak{G}_p$.

Further, from Corollary~\ref{cor:homotopy-from-p},
for any $\xi\in \mathfrak{G}_p$ and  path $\hat\alpha\:[0,1]\to   \mathfrak{G}_p$ from $\hat p$ to $\xi$,
we have 
\begin{align*}
\length\hat\alpha
&=\length\map\circ\hat\alpha
\ge
\\
&\ge
\length\xi
=\length\hat\xi
\end{align*}
where equality holds only if $\hat\alpha$ is a reparametrization of $\hat\xi$.
In particular, 
\[\dist{\hat p}{\xi}{\mathfrak{G}_p}=\length\xi
\eqlbl{eq:dist=length}\] 
and
$\hat\xi\:[0,1]\to \mathfrak{G}_p$ is the unique geodesic path from $\hat p$ to $\xi$.
Clearly, the map $\xi\mapsto\hat\xi$ is continuous.

From \ref{eq:dist=length} and Proposition~\ref{prop:geo-complete}, 
it follows

\begin{clm}{}\label{clm:complete-B} 
For any $\bar R<\varpi\kappa/2$, the closed ball
$\cBall[\hat p,\bar R]$ in $\mathfrak{G}_p$ is complete.
\end{clm}

Take $\spc{B}=\oBall(\hat p,\varpi\kappa/2)$ and $\map$ constructed above.
According to the Radial lemma (\ref{lem:radial-glob}), $\spc{B}$ is a $\varpi\kappa$-geodesic $\CAT\kappa$ space.
The remaining statements are already proved.
\qeds

\section{Reshetnyak majorization}\label{sec:resh-kirz}

\begin{thm}{Definition}\label{def:majorize}
Let $\spc{X}$ be a metric space,
$\tilde \alpha$ be a simple closed curve of finite length  in $\Lob2{\kappa}$,
and $D\subset\Lob2{\kappa}$ be a closed region bounded by $\tilde \alpha$.
A length-nonincreasing map $F\:D\to\spc{X}$ is called \emph{majorizing} if it is length-preserving on $\tilde \alpha$.

In this case, we say that $D$ \emph{majorizes} the curve $\alpha=F\circ\tilde \alpha$ under the map $F$.
\end{thm}

The following proposition is an immediate consequence of the definition.

\begin{thm}{Proposition}
\label{prop:majorize-geodesic} 
Let  $\alpha$  be a closed curve in a metric space $\spc{X}$.
Suppose $D\subset\Lob2{\kappa}$ majorizes $\alpha$ under $F\: D \to \spc{X}$.  
Then any geodesic subarc of $\alpha$ is the image under $F$ of a subarc of $\partial_{\Lob2{\kappa}} D$ that is geodesic in the length-metric of $D$.

In particular, if $D$ is convex, then the corresponding subarc is a geodesic in $\Lob2{\kappa}$.
\end{thm}

\parit{Proof.} For a geodesic subarc $\gamma\:[a,b]\to\spc{X}$ of $\alpha=F\circ\tilde \alpha$, set
\begin{align*}
\tilde r&=\dist{\tilde \gamma(a)}{\tilde \gamma(b)}{D},
&
\tilde \gamma &= (F|{\Fr D})^{-1}\circ\gamma,
\\
s&=\length \gamma,
&
\tilde s&= \length \tilde \gamma.
\end{align*}
Then
\[\tilde r\ge r = s =\tilde s\ge\tilde r.\]
Therefore $\tilde s=\tilde r$.
\qeds

\begin{thm}{Corollary}\label{cor:maj-triangle}
Let $\trig p x y$ be a triangle of perimeter $<2\cdot\varpi\kappa$ in a metric space $\spc{X}$. Assume a convex region $D\subset \Lob2\kappa$ majorizes $\trig p x y$.
Then $D=\Conv\trig{\tilde p}{\tilde x}{\tilde y}$ for a model triangle $\trig{\tilde p}{\tilde x}{\tilde y}=\modtrig\kappa(p x y)$, and the majorizing map sends  $\tilde p$, $\tilde x$ and $\tilde y$ respectively to $p$, $x$ and $y$.
\end{thm}

Now we come to the main theorem of this section.

\begin{thm}{Majorization theorem}
\label{thm:major}
Any closed curve $\alpha$ with length smaller than $2\cdot \varpi\kappa$ in  a $\varpi\kappa$-geodesic $\CAT\kappa$ space is majorized by a convex region in $\Lob2\kappa$. \end{thm}

This theorem is proved by Yurii Reshetnyak in \cite{reshetnyak:major};
our proof uses a trick which we learned from from the lectures of Werner Ballmann \cite{ballmann:lectures}.
An other proof can be build on Kirszbraun's theorem (\ref{thm:kirsz+}), but it works only for complete spaces.

The case when when $\alpha$ is a triangle, say $\trig p x y$, is leading and nontrivial.
In this case, by Proposition~\ref{cor:maj-triangle}, the majorizing convex region has to be isometric to $\Conv\trig{\tilde p}{\tilde x}{\tilde y}$, where $\trig{\tilde p}{\tilde x}{\tilde y}=\modtrig\kappa(p x y)$.  
There is a majorizing map for $\trig p x y$ which image $W$ is the image of the line-of-sight map (definition \ref{def:sight}) for $[x y]$ from  $p$,
but as one can see from the following example, the line-of-sight map is not majorizing in general.

\begin{wrapfigure}{r}{21mm}
\begin{lpic}[t(-4mm),b(-0mm),r(0mm),l(0mm)]{pics/example-for-lytchak(1)}
\lbl[r]{1,9;$p$}
\lbl[l]{20.5,2;$x$}
\lbl[l]{20,17;$y$}
\lbl[l]{16,9;$z$}
\lbl{10,10;$\spc{Q}$}
\end{lpic}
\end{wrapfigure}

\parbf{Example.} Let $\spc{Q}$ be a bounded region in $\EE^2$ formed by two congruent triangles, where the bounding quadrangle $[p x z y]$ is non-convex at $z$ (as in the picture).  
Equip $\spc{Q}$ with the length-metric. 
Then $\spc{Q}$ is $\CAT0$
by Reshetnyak gluing  (\ref{thm:gluing}). 
For triangle ${\trig p x y}_\spc{Q}$ in $\spc{Q}$ and its model triangle $\trig{\tilde p}{\tilde x}{\tilde y}$ in $\EE^2$,  
we have 
\[\dist{\tilde x}{\tilde y}{}=\dist{x}{y}{\spc{Q}}=\dist{x}{z}{}+\dist{z}{y}{}.\]
Then the map $F$ defined by matching line-of-sight parameters satisfies $F(\tilde x)=x$ and $\dist{x}{F(\tilde w)}{}>\dist{\tilde x}{\tilde w}{}$ if $\tilde w$ is near the midpoint $\tilde z$ of $[\tilde x\tilde y]$ and lies on $[\tilde p\tilde z]$. 
Indeed, by the first variation formula (\ref{1st-var+}), for $\eps=1-s$ we have
\[\dist{\tilde x}{\tilde w}{}
=\dist{\tilde x}{\tilde \gamma_\frac12(s)}{}
=\dist{x}{z}{}+o(\eps)\] and 
\[\dist{x}{F(\tilde w)}{}
=\dist{x}{\gamma_\frac12(s)}{}
=\dist{x}{z}{}-\eps\cdot\cos\mangle\hinge z p x+o(\eps).\]  
Thus $F$ is not majorizing.

\medskip

In  the following proofs, $x^1 \dots x^n$ ($n\ge 3$) denotes a broken geodesic with vertices $x^1,\dots,x^n$, and $[x^1\dots x^n ]$ denotes the corresponding (closed) polygon.
%that is, respectively, the join of geodesics $[x^1x^2]$, $[x^2,x^3],\dots,[x^{n-1}x^n]$ and the closed curve formed by joining $[x^1x^2],\dots,[x^{n-1}x^n], [x^n x^1]$.
For a subset $R$ of the ambient metric space,
we denote by $[x^0\dots x^n ]_R$ a polygon in the length metric of $R$.

Our first lemma gives a model space construction based on repeated application of Lemma~\ref{lem:quadrangle}.
Recall that convex and concave curves with respect to a poitn are defined in~\ref{def:convex-devel}.

\begin{thm}{Lemma}\label{lem:majorize-subgraph}
In $\Lob2{\kappa}$, let  
$\beta$ be a curve from $x$ to $y$ 
that is concave with respect  to $p$.
Let $D$  be the subgraph of $\beta$ with respect to $p$.
Assume 
\[\length\beta\z+\dist{p}{x}{}+\dist{p}{y}{}<2\cdot\varpi\kappa.\]
\begin{subthm}{curvilinear} 
Then $\beta$ forms a geodesic $[x y]_D$ in $D$ and therefore $\beta$, $[p x]$ and $[p y]$ form a triangle 
${\trig p x y}_D$ in the length-metric of $D$.
\end{subthm}
\begin{subthm} {short-to-subgraph}
Let $\trig{\tilde p}{\tilde x}{\tilde y}$ be the model triangle for triangle 
${\trig p x y}_D$.  Then there is a short map $G\:\Conv\trig{\tilde p}{\tilde x}{\tilde y}\to D$ such that $\tilde p\mapsto p$, $\tilde x\mapsto x$, $\tilde y\mapsto y$, and $G$ is length-preserving on each side of $\trig{\tilde p}{\tilde x}{\tilde y}$.  In particular, $\Conv\trig{\tilde p}{\tilde x}{\tilde y}$ majorizes triangle $[p x y]_D$ in $D$ under $G$.
\end{subthm}
\end{thm} 


\parit{Proof.}
We prove the lemma for a broken geodesic $\beta$;
the general case then follows by approximation.
Namely, since $\beta$ is concave 
it can be approximated by broken geodesics that are concave with respect to $p$, 
with their lengths converging to $\length \beta$. 
Passing to a partial limit we will obtain the needed map $G$.  

Suppose $\beta=x^0x^1\dots x^n$ is a broken geodesic with $x^0=x$ and $x^n=y$.
Consider a sequence of broken geodesics $\beta_i=x^0x^1\dots x^{i-1}y_i$ such that $\dist{p}{y_i}{}=\dist{p}{y}{}$ and 
$\beta_i$ has same length as $\beta$; 
that is, 
\[\dist{x^{i-1}}{y_i}{}=\dist{x^{i-1}}{x^{i}}{}+\dist{x^{i}}{x^{i+1}}{}+\dots+\dist{x^{n-1}}{x^n}{}.\]

\begin{center}
\begin{lpic}[t(-0mm),b(2mm),r(0mm),l(0mm)]{pics/D-n(1)}
\lbl[t]{4,0;$p$}
\lbl[t]{100,0;$x^0=x$}
\lbl[tw]{54,1;$x^1$}
\lbl[t]{29.5,9;$x^2$}
\lbl[rw]{25,20,45;$y_3$}
\lbl[rw]{22.5,23,54;$y_2$}
\lbl[rw]{20,25.5,60;$y_1$}
\end{lpic}
\end{center}

Clearly $\beta_n=\beta$.
Sequentially applying Alexandrov's lemma (\ref{lem:alex}) shows that each of the broken geodesics $\beta_{n-1}, \beta_{n-2},\dots,\beta_1$ is concave with respect to $p$.
Let $D_i$ to be subgraph of $\beta_i$ with respect to $p$.
Applying Lemma~\ref{lem:quadrangle} gives a short map $G_i\:D_{i}\to D_{i+1}$ that maps $y_{i}\mapsto y_{i+1}$ and does not move $p$ and $x$ (in fact,  $G_i$ is the identity everywhere except on $\Conv\trig{p}{x^{i-1}}{y_i}$).
Thus the composition 
\[G_{n-1}\circ\dots\circ G_1\: D_1\to D_n\] 
is short.
The result follows since $D_1\iso\Conv\trig{\tilde p}{\tilde x}{\tilde y}$.\qeds

\begin{thm}{Lemma}\label{lem:majorize-triangle}
In  a $\varpi\kappa$-geodesic $\CAT\kappa$ space $\spc{U}$, let $\trig{p}{x}{y}$ be a triangle in $\spc{U}$ of perimeter $<2\cdot\varpi\kappa$. In $\Lob2{\kappa}$, let $\tilde \gamma$ be the $\kappa$-development of $[x y]$ with respect to $p$, where $\tilde \gamma$ has basepoint $\tilde p$ and subgraph $D$.
Consider the map $H\:D\to\spc{U}$ that sends the point with parameter $(t,s)$ under the line-of-sight map for $\tilde \gamma$ from $\tilde p$, to the point with the same parameter under the line-of-sight map $f$ for $[x y]$ from  $p$.  Then $H$ is  length-nonincreasing.
In particular, $D$ majorizes triangle $\trig p x y$.
\end{thm}

\parit{Proof.}
Let $\gamma=\geod_{[x y]}$ and $T=\dist{x}{y}{}$. As in the proof of the development criterion (\ref{thm:concave-devel}), take a partition 
\[0=t^0<t^1<\dots<t^n=T,\]
and set $x^i=\gamma(t^i)$. 
Construct a chain of model triangles  $\trig{\tilde p}{\tilde x^{i-1}}{\tilde x^i}=
\modtrig\kappa(p x^{i-1} {x^i})$, with $\tilde x^0=\tilde x$ and the direction of $[\tilde p\tilde x^i]$ turning counterclockwise as $i$ grows.  
Let $D_n$ be the subgraph with respect to $\tilde p$ of the broken geodesic $\tilde x^0\dots \tilde x^n$.


Let  $\delta_n$ be the maximum radius of a circle inscribed in any of the triangles $\trig{\tilde p}{\tilde x^{i-1}}{\tilde x^i}$.  

Now we construct a map $H_n \: D_n\to\spc{U}$  that increases distances by at most  $2\cdot\delta_n$.

Suppose $x\in D_n$.  Then $x$ lies on or inside some triangle $\trig{\tilde p}{\tilde x^{i-1}}{\tilde x^i}$.  
Define $H_n(x)$ by first mapping $x$ to a nearest point on $\trig{\tilde p}{\tilde x^{i-1}}{\tilde x^i}$ (choosing one if there are several), followed by the natural map to the triangle  $\trig {p}{x^{i-1}}{ x^i}$. 

Since triangles in $\spc{U}$ are $\kappa$-thin (\ref{prop:k-thin}), the restriction of $H_n$ to each triangle $\trig{\tilde p}{\tilde x^{i-1}}{\tilde x^i}$ is short.   
Then the triangle inequality implies that the restriction of $H_n$ to 
\[U_n=\bigcup_{1\le i\le n}\trig{\tilde p}{\tilde x^{i-1}}{\tilde x^i}\]
is short with respect to the length-metric on $D_n$. 
Since nearest-point projection 
from $D_n$ to $U_n$ increases the $D_n$-distance between two points by at most $2\cdot\delta_n$,  the map $H_n$ also increases the $D_n$-distance  by at most $2\cdot\delta_n$. 

Consider $y_n\in D_n$ with $y_n\to y\in D$ and $z_n\in D_n$ with $z_n\to z\in D$. Since $\delta_n\to 0$ under increasingly finer partitions, and geodesics in $\spc{U}$ vary continuously with their endpoints (\ref{thm:alex-patch}),  we have $H_n(x_n)\to 
H(x)$ and $H_n(y_n)\to H(y)$.  Since 
\[\dist{H_n(x_n)}{H_n(y_n)}{} \le \dist{x_n}{y_n}{D_n} + 2\cdot\delta_n,\]
where the lefthand side converges to $\dist{H(x)}{H(y)}{}$  and the righthand side converges to $\dist{x}{y}{D}$, it follows that $H$ is short.
\qeds




\parit{Proof of majorization theorem (\ref{thm:major}).}
We begin by proving the theorem in case $\alpha$ is polygonal.

First suppose $\alpha$ is a triangle, say $\trig p x y$.
By assumption, the perimeter of $\trig p x y$ is less than
$2\cdot\varpi\kappa$.
This is the base case for the induction.

 Let $\tilde \gamma$ be the $\kappa$-development of $[x y]$ with respect to $p$, where $\tilde \gamma$ has basepoint $\tilde p$ and subgraph $D$.
By the development criterion (\ref{thm:concave-devel}),  $\tilde \gamma$ is concave.
By Lemma~\ref{lem:majorize-subgraph},  there is a short map $G\:\Conv\modtrig\kappa(p x y)\to D$.
Further, by Lemma~\ref{lem:majorize-triangle},  $D$ majorizes $\trig p x y$ under a majorizing map $H\:D\to\spc{U}$. Clearly $H\circ G$ is a majorizing map for $\trig p x y$.

Now we claim that any closed $n$-gon $[x^1x^2 \dots x^n ]$ of perimeter less than $2\cdot \varpi\kappa$ in a $\CAT{\kappa}$ space  is majorized by a convex polygonal region \[R_n=\Conv[\tilde x^1\tilde x^2\dots\tilde x^n]\]
under a map $F_n$ such that $F_n\:\tilde x^i\mapsto x^i$ for each $i$. 

\begin{wrapfigure}{r}{35mm}
\begin{lpic}[t(0mm),b(0mm),r(0mm),l(1mm)]{pics/RR(1)}
\lbl[rb]{7,24;$\tilde x^1$}
\lbl[r]{1,10;$\tilde x^2$}
\lbl[lt]{26,8,41;$\dots$}
\lbl[lb]{24.5,23;$\tilde x^{n-1}$}
\lbl[b]{26,38.5;$\dot x^{n}$}
\lbl{17,14;$R_{n-1}$}
\lbl{18,27;$R'$}
\end{lpic}
\end{wrapfigure}

Assume the statement is true for $(n-1)$-gons, $n\ge 4$.  
Then  $[x^1 x^2 \dots x^{n-1}]$  is majorized by a convex polygonal region 
\[R_{n-1}=\Conv(\tilde x^1, \tilde x^2, \dots, \tilde x^{n-1}),\] 
in $\Lob2\kappa$ under a map $F_{n-1}$ satisfying $F_{n-1}(\tilde x^i)=x^i$ for all $i$. 
Take $\dot x^n\in\Lob2{\kappa}$ such that $\trig{\tilde x^1}{\tilde x^{n-1}}{\dot x^n}=\modtrig\kappa(x^1 x^{n-1} x^n)$ 
and this triangle lies on the other side of $[\tilde x^1\tilde x^{n-1}]$ from $R_{n-1}$.  
Let $\dot R\z=\Conv\trig{\tilde x^1}{\tilde x^{n-1}}{\dot x^n}$, 
and $\dot F\:\dot R\to \spc{U}$ be a majorizing map for $\trig { x^1}{x^{n-1}}{ x^n}$ as provided above.

Set 
$R= R_{n-1}\cup \dot R$, where $R$ carries its length-metric.
Since $F_n$ and $F$ agree on $[\tilde x^1 \tilde x^{n-1}]$, we may define $F\:R\to\spc{U}$ by 
\[
F(x)=
\begin{cases}
F_{n-1}(x),\quad & x\in R_{n-1},\\
\dot F(x),\quad & x\in \dot R.\\
\end{cases}
\]
Then $F$ is length-nonincreasing, and is a majorizing map for $[x^1 x^2 \dots x^n ]$ (as in Definition~\ref{def:majorize}).

If $R$ is a convex subset of $\Lob2\kappa$, we are done. 

If $R$ is not convex,  the total internal angle of $R$ at $\tilde x^1$ or $ \tilde x^{n-1} $ or both is $>\pi$.  
By relabeling we may suppose that it holds for $\tilde x^{n-1}$.  

The region $R$ is obtained by glueing $R_{n-1}$ to $\dot R$ by $[x^1x^{n-1}]$.
Thus, by Reshetnyak gluing (\ref{thm:gluing}), $R$ in its length-metric is a $\CAT{\kappa}$-space.  
Moreover $[\tilde x^{n-2}\tilde x^{n-1}]\cup[\tilde x^{n-1} \dot x^n]$ is a geodesic of $R$.
Thus $[\tilde x^1, \tilde x^2, \dots, \tilde x^{n-1} \dot x^n]_R$ is a closed $(n-1)$-gon in $R$, to which the induction hypothesis applies. The resulting short map from a convex region in $ \Lob2\kappa$ to $R$, followed by $F$,  is the desired majorizing map.

\medskip

Note that in fact we proved the following:

\begin{clm}{}
Let $F_{n-1}$ be a majorizing map for the polygon $[x^1x^2\dots x^{n-1}]$,
and $\dot F$ be a majorizing map for the triangle $[x^1x^{n-1}x^{n}]$.
Then there is a majorizing map $F_n$ for the polygon $[x^1x^2\dots x^n]$
such that \[\Im F_{n+1}= \Im F_n\cup\Im \dot F.\]

\end{clm}

We now use this claim to prove the theorem for general curves.

Assume $\alpha\:[0,\ell]\to\spc{U}$ is an  arbitrary closed curve with natural parameter.
Choose a sequence of partitions $0=t^0_n<t^2_n<\dots<t^n_n=\ell$
so that:
\begin{itemize}
\item The set $\{t_{n+1}^i\}_{i=0}^{n+1}$ 
is obtained from the set  $\{t_n^i\}_{i=0}^n$ by adding one element.
\item For some sequence $\eps_n\to0+$,
we have $t^i_n-t^{i-1}_n<\eps_n$ for all $i$.
\end{itemize}

Inscribe in $\alpha$ a sequence of polygons $P_n$ with vertices $\alpha(t^i_n)$.
Apply the claim above, to get a sequence of majorizing maps $F_n\:R_n\to\spc{U}$.
Note that for all $m>n$
\begin{itemize}
\item $\Im F_m$ lies in an  $\eps_n$-neighborhood of $\Im F_n$
\item $\Im F_m\backslash \Im F_n$ lies in an  $\eps_n$-neighborhood of $\alpha$.
\end{itemize}
It follows that the set
\[K=\alpha\cup\left(\bigcup_n\Im F_n\right)\]
is compact.
Therefore the sequence $(F_n)$
has a partial limit as $n\to\infty$; 
say $F$.
Clearly $F$ is a majorizing map for $\alpha$.
\qeds

The following exercise is the rigidity case 
of the Majorization theorem.

\begin{thm}{Exercise}\label{ex:isometric-majorization}
Let $\spc{U}$ be a $\varpi\kappa$-geodesic $\CAT\kappa$ space
and $\alpha\:[0,\ell]\to\spc{U}$ be a closed curve with arclength parametrization.
Assume that $\ell<2\cdot \varpi\kappa$
and there is a closed convex curve $\tilde \alpha\:[0,\ell]\to\Lob{2}{\kappa}$ such that 
\[\dist{\alpha(t_0)}{\alpha(t_1)}{\spc{U}}=\dist{\tilde \alpha(t_0)}{\tilde \alpha(t_1)}{\Lob{2}{\kappa}}\]
for any $t_0$ and $t_1$.
Then there is a distance preserving map $F\:\Conv\tilde \alpha\to \spc{U}$
such that $F\:\tilde \alpha(t)\mapsto \alpha(t)$ for any $t$.
\end{thm}

\begin{thm}{Exercise}
Assume a closed curve $\alpha$ of length $<2\cdot \varpi\kappa$ in a $\varpi\kappa$-geodesic $\CAT\kappa$ space is not a triangle.
Show that exactly one of these statements holds:

\begin{subthm}{}
$\alpha$ is majorized by non-isometric convex regions in $\Lob2\kappa$;
\end{subthm}

\begin{subthm}{}
the majorizing map for $\alpha$ is distance preserving.
\end{subthm}

\end{thm}

For $n=1$, the following lemma states that in a $\CAT\kappa$ space, 
a sharp triangle comparison implies the
presence  of an isometric copy of the convex hull of the model triangle.
This rigidity statement was proved by Alexandrov \cite{alexandrov:devel}.
An analogue for $\Alex{\kappa}$ spaces fails, see Exercise \ref{ex:not-flat}.
  
\begin{thm}{Arm lemma}\label{lem:arm}
Let $\spc{U}$ be a $\varpi\kappa$-geodesic $\CAT\kappa$ space, 
and $P=[x^0x^1\dots x^{n+1}]$ be a polygon of length $<2\cdot \varpi\kappa$ in $\spc{U}$.
Suppose $\tilde P=[\tilde x^0\tilde x^1\dots \tilde x^{n+1}]$ is a convex  polygon in $\Lob{2}{\kappa}$
such that 
\[
\dist{\tilde x^i}{\tilde x^{i-1}}{\Lob{2}{\kappa}}
=
\dist{x^i}{x^{i-1}}{\spc{U}}
\ \ \t{and}\ \ 
\mangle\hinge{x^i}{x^{i-1}}{x^{i+1}}\ge\mangle\hinge{\tilde x^i}{\tilde x^{i-1}}{\tilde x^{i+1}}
\eqlbl{eq:arm}
\]
for all $i$.
Then 

\begin{subthm}{subthm:arm-ineq}
$\dist{\tilde x^0}{\tilde x^{n+1}}{\Lob{2}{\kappa}}
\le
\dist{x^0}{x^{n+1}}{\spc{U}}$.
\end{subthm}

\begin{subthm}{subthm:arm-eq}
Equality holds in (\ref{SHORT.subthm:arm-ineq}) if and only if the map $\tilde x^i\mapsto x^i$ can be extended 
to a distance-preserving map of $\Conv(\tilde x^0,\tilde x^1\dots \tilde x^{n+1})$ onto $\Conv(x^0,x^1\dots x^{n+1})$.
\end{subthm}
\end{thm}

\parit{Proof. \ref{SHORT.subthm:arm-ineq}).}
By majorization (\ref{thm:major}), $P$ is majorized by a convex region $\tilde D$ in $\Lob{2}{\kappa}$.
By Proposition \ref{prop:majorize-geodesic} and the definition of angle,
$\tilde D$ is bounded by a convex polygon $\tilde P_R=[~y^0\tilde y^1\dots \tilde y^{n+1}]$ that satisfies
\begin{align*}
\dist{\tilde y^i}{\tilde y^{i\pm1}}{\Lob{2}{\kappa}}
&=
\dist{x^i}{x^{i\pm1}}{\spc{U}}, \ \ \ \ \dist{\tilde y^0}{\tilde y^{n+1}}{\Lob{2}{\kappa}}
=
\dist{x^0}{x^{n+1}}{\spc{U}},
\\
& \mangle\hinge{\tilde y^i}{\tilde y^{i-1}}{\tilde y^{i+1}}\ge\mangle\hinge{x^i}{x^{i-1}}{x^{i+1}}\ge\mangle\hinge{\tilde x^i}{\tilde x^{i-1}}{\tilde x^{i+1}},
\end{align*}
for $1\le i\le n$, where the final inequality is by \ref{eq:arm}.

By the arm lemma (\ref{lem:arm}), $\dist{\tilde x^0}{\tilde x^{n+1}}{}\le \dist{\tilde y^0}{\tilde y^{n+1}}{}$.
Since $ \dist{\tilde y^0}{\tilde y^{n+1}}{}=\dist{x^0}{x^{n+1}}{}$, the part (\ref{SHORT.subthm:arm-ineq}) follows.

\parit{\ref{SHORT.subthm:arm-eq}).} Suppose equality holds in (\ref{SHORT.subthm:arm-ineq}).  Then angles at the $j$-th vertex of $\tilde P$, $P$ and $\tilde P_R$ are equal for $1\le j\le n$,   and we may take $\tilde P=\tilde P_R$.  

Let $F\:\tilde D\to\spc{U}$ be the majorizing map for $P$, where $\tilde D$ is the convex region bounded by $\tilde P$, and $F|\tilde P$ is length-preserving.  

\begin{clm}{}\label{clm:arm-triangle}
Let $\tilde x,\tilde y,\tilde z$ be three vertices of $\tilde P$, and $x,y,z$ be the corresponding vertices of $P$.  If $\dist{\tilde x}{\tilde y}{}=
\dist{x}{y}{}$, $\dist{\tilde y}{\tilde z}{}=
\dist{x}{z}{}$ and $\mangle\hinge{\tilde y}{\tilde x}{\tilde z} = \mangle\hinge{y}{x}{z}$, then $F|\Conv(\tilde x, \tilde y, \tilde z)$ is distance-preserving.
\end{clm} 

Since $F$ is majorizing, $F$ restricts to   distance-preserving maps from $[\tilde x\tilde y]$ to $[xy]$ and $[\tilde y\tilde z]$ to $[yz]$.
Suppose $\tilde p\in [\tilde x \tilde y]$ and $\tilde q\in[\tilde y\tilde z]$.  Then 
\[
\dist{\tilde p}{\tilde q}{\Lob{2}{\kappa}}
=
\dist{F( \tilde p )}{F(\tilde q)}{\spc{U}}.
 \eqlbl{eq:arm-eq}
\]
Indeed, inequality holds in one direction by majorization, and in the other direction by angle comparison
(\ref{cat-hinge}).
By first variation (\ref{cor:both-end-first-var-cba}), it follows that each pair of corresponding angles of triangles $[\tilde x \tilde y \tilde z]$ and $[x y z]$ are equal.
But then \ref{eq:arm-eq} holds for $p,q$ on any two sides of these triangles, so $F$ is distance-preserving on every geodesic of $\Conv(\tilde p, \tilde x, \tilde y)$.
Hence the claim.

\begin{clm}{}\label{clm:arm-induction}
Suppose $F|\Conv(\tilde x^{n+1},\tilde x^0,\tilde x^1\dots \tilde x^{k})$ is distance-preserving for some $k$, $1\le k\le n-1$.   Then $F|\Conv(\tilde x^{n+1},\tilde x^0,\tilde x^1\dots \tilde x^{k+1})$ is distance-preserving.
\end{clm}


Set $\tilde p=[\tilde x^{k-1}\tilde x^{k+1}] \cap [\tilde x^{k}\tilde x^{n+1}]$ and $p=[x^{k-1}x^{k+1}] \cap [x^{k}x^{n+1}]$.

The following maps are distance-preserving:
\begin{enumerate}
\item[(i)]
$F|\Conv(\tilde x^{k-1},\tilde x^k,\tilde x^{k+1})$,

\item[(ii)]
 $F|\Conv(\tilde x^{k+1},\tilde x^{k-1},\tilde x^{n+1})$,

\item[(iii)]
$F|\Conv(\tilde x^{n+1},\tilde x^{k},\tilde x^{k+1})$.
\end{enumerate}
Indeed, (i) is immediate from \ref{clm:arm-triangle}.  
Therefore $\dist{\tilde x^{k-1}}{\tilde x^{k+1}}{}=\dist{x^{k-1} }{x^{k+1}}{}$, and so $F$ restricts to a distance-preserving map from $[\tilde x^{k-1}\tilde x^{k+1}]$ onto $[x^{k-1} x^{k+1}]$.  With the induction hypothesis, it follows that $F(\tilde p)=p$, hence 
\[
\mangle\hinge{\tilde x^{k-1}}{\tilde x^{k+1}}{\tilde x^{n+1}} = \mangle\hinge{x^{n+1}}{x^{k-1}}{x^{k+1}}.
\dist{F( \tilde p )}{F(\tilde q)}{\spc{U}}.
 \eqlbl{eq:angle-arm-eq}
\] 
Then (ii) follows from \ref{eq:angle-arm-eq} and \ref{clm:arm-triangle}.  Since $\dist{\tilde x^{k}}{\tilde x^{n+1}}{}=\dist{x^{k} }{x^{n+1}}{}$, (iii) follows from \ref{eq:angle-arm-eq} and (i). 

Let $\tilde \gamma$ be a geodesic of $\Conv(\tilde x^{n+1},\tilde x^0,\tilde x^1\dots \tilde x^{k+1})$.  Then $\length \tilde \gamma < \varpi\kappa$.  If $\tilde \gamma$ does not pass the point $[\tilde x^{k-1}\tilde x^{k+1}] \cap [\tilde x^{k}\tilde x^{n+1}]$, it  follows from the induction hypothesis and (i) (ii), (iii) that  $\gamma = F\circ\tilde \gamma$  is a local geodesic of length $< \varpi\kappa$.  By \ref{cor:loc-geod-are-min}, $\gamma$ is a geodesic.  By continuity, $F\circ\tilde \gamma$ is a geodesic for all $\tilde \gamma$. \ref{clm:arm-induction} follows and hence the lemma.
 \qeds
 
\begin{thm}{Exercise}\label{ex:square}
Let $\spc{U}$ be a complete length $\CAT0$ space and for 4 points $x_1,x_2,x_3,x_4\in \spc{U}$
there is a convex quadrilateral
$[\tilde x_1\tilde x_2\tilde x_3\tilde x_4]$
in $\EE^2$
such that 
\[\dist{x_i}{x_j}{\spc{U}}=\dist{\tilde x_i}{\tilde x_j}{\EE^2}\]
for all $i$ and $j$.
Show that $\spc{U}$ contains an isometric copy of the 
\emph{solid quadrilateral}
$\tilde x_1\tilde x_2\tilde x_3\tilde x_4$; that is, the convex hull of $\tilde x_1,\tilde x_2,\tilde x_3,\tilde x_4$ in $\EE^2$
\end{thm}

%%%%%%%%%%%%%%%%%%%%%%%%%%%%%%%%%%%%%%%%%%%%%%%%%%%%%%%%%%%%%%%%%%%%%%%%%%%%%%%%

\section{Hadamard--Cartan theorem}\label{sec:Hadamard--Cartan}

The development of Alexandrov geomtry was greatly influenced by the Hadamard--Cartan theorem.
Its original formulation states that if $M$ is a complete Riemannian manifold with nonpositive sectional curvature, 
then the exponential map at any point $p\in M$ is a covering;
in particular it implies that the universal cover of $M$ is diffeomorphic to the Euclidean space of the same dimension.

In this generality, the theorem appeared in the lectures of Cartan, see \cite{cartan}.
For surfaces in the Euclidean plane, 
the theorem was proved by
Hans von Mangoldt \cite{mangoldt},  
and few years later independently by Hadamard \cite{hadamard}.

Formulations for metric spaces of different generality were proved by 
Busemann (see \cite{busemann-CBA}),
Rinow (see \cite{rinow}), and 
Gromov  \cite[p.119]{gromov:hyp-groups}. 
A detailed proof of Gromov's statement when $\spc{U}$ is proper  was given by Werner Ballmann \cite{ballmann:cartan-hadamard}, using Birkhoff's curve-shortening.  
A proof in the non-proper 
geodesic case 
was given by the first author and Richard Bishop \cite{a-b:h-c}.  
This proof applies more generally, to \emph{convex spaces} (see Exercise \ref{ex:cats-cradle}).
%??? what is it about???
It was pointed out by Bruce 
Kleiner (see \cite{ballmann:lectures}) 
and independently by Martin 
Bridson and Andr\'{e} 
Haefliger in \cite{BH} that 
%the no-conjugate-point theorem (\ref{thm:no-conj-pt}) allows curve-shortening as in Corollary \ref{cor:homotopy-from-p}, and hence 
%the Hadamard--Cartan theorem
this proof
extends to length spaces as well as geodesic spaces, giving:

\begin{thm}{Hadamard--Cartan theorem}
\label{thm:hadamard-cartan}
Let $\spc{U}$ be a complete,  simply connected length locally $\CAT0$ space.
Then $\spc{U}$ is $\CAT0$.
\end{thm}

\parit{Proof.} Since $\varpi\kappa=\infty$,
the map $\map\:\spc{B}\to\spc{U}$ in Theorem
 (\ref{thm:globalization-lift}) is a metric covering. 
%???SHOULD WE DEFINE metric covering???
Since $\spc{U}$ is simply connected, $\map\:\spc{B}\to\spc{U}$ is an isometry.
\qeds

To formulate the generalized Hadamard--Cartan theorem,
we need the following definition.

\begin{thm}{Definition}\label{def:l-s.c.}
Given $\ell\in (0,\infty]$,
a metric space $\spc{X}$ is called 
$\ell$-simply connected 
if it is connected and 
any closed curve of length $<\ell$ 
is null-homotopic in the class of curves of length $<\ell$ in $\spc{X}$.
\end{thm}

Note that there is a subtle difference between 
simply connected and $\infty$-simply connected spaces;
the first states that any closed curve is null-homotopic while the second means that any rectifiable curve is null-homotopic in the class of rectifiable curves.
However, as follows from Proposition~\ref{prop:sc}, for locally $\CAT\kappa$ spaces these two definitions are equivalent.
This fact makes it possible to deduce the Hadamard--Cartan theorem directly from the Generalized Hadamard--Cartan theorem.

For proper spaces, the generalized Hadamard--Cartan theorem was proved by Brian Bowditch \cite{bowditch}.

\begin{thm}{Generalized Hadamard--Cartan theorem}\label{thm:hadamard-cartan-gen}
A complete length space
$\spc{U}$ is $\CAT\kappa$ 
if and only if $\spc{U}$ is $2\cdot\varpi\kappa$-simply connected
and $\spc{U}$ is locally $\CAT\kappa$.
\end{thm}

In the proof we need the following lemma.

\begin{thm}{Lemma}

Assume $\spc{U}$ is a complete length  locally $\CAT\kappa$ space,
$\eps>0$,
and $\gamma_1,\gamma_2\:\mathbb{S}^1\to\spc{U}$ are two closed curves.
Assume 
\begin{subthm}{}
$\length\gamma_1,\length\gamma_2<2\cdot\varpi\kappa-4\cdot\eps$;
\end{subthm}
 
\begin{subthm}{} $\dist{\gamma_1(x)}{\gamma_2(x)}{}<\eps$ for any $x\in\mathbb{S}^1$, and the geodesic $[\gamma_1(x)\gamma_2(x)]$ is uniquely defined and depends continuously on $x$;
\end{subthm}

\begin{subthm}{}  $\gamma_1$ is majorized by a convex region in $\Lob2\kappa$.
\end{subthm}

Then  $\gamma_2$ is majorized by a convex region in $\Lob2\kappa$.
\end{thm}

\parit{Proof.} Let $D$ be a convex region in $\Lob2\kappa$ that majorizes $\gamma_1$ under the map $F\:D\to\spc{U}$ 
(see Definition~\ref{def:majorize}).
Denote by $\tilde \gamma_1$ 
the curve bounding $D$ 
such that $F\circ\tilde \gamma_1=\gamma_1$.
Since  
\begin{align*}
\length\tilde \gamma_1
&=
\length\gamma_1
<
\\
&<
2\cdot\varpi\kappa-4\cdot\eps,
\end{align*}
there is a point $\tilde p\in D$ such that 
$\dist{\tilde p}{\tilde \gamma(x)}{\Lob2\kappa}<\tfrac{\varpi\kappa}2-\eps$
for any $x\in\mathbb{S}^1$.
Denote by $\alpha_x$ the joint of the paths $F\circ\geodpath_{[p\tilde \gamma_1(x)]_{\Lob2\kappa}}$ 
and  $\geodpath_{[\gamma_1(x)\gamma_2(x)]}$ in $\spc{U}$.
Note that $\alpha_x$ depends continuously on $x$ and
$$\length\alpha_x<\tfrac{\varpi\kappa}{2}\ \ \text{and}\ \ \alpha_x(1)=\gamma_2(x)$$ 
for any $x$.

Let us apply the lifting globalization theorem
(\ref{thm:globalization-lift}) for $p=F(\tilde p)$.
We get a $\varpi\kappa$-geodesic $\CAT\kappa$ space $\spc{B}$
and a locally isometric map $\map\:\spc{B}\to\spc{U}$
with $\map(\hat p)=p$ for some $\hat p \in \spc{B}$, and with the lifting property for the curves starting at $p$ with length $<\varpi\kappa/2$.
Applying the lifting property for $\alpha_x$, 
we get existence of a curve $\hat\gamma_2\:\mathbb{S}^1\to \spc{B}$ such that
$$\gamma_2=\map\circ\hat\gamma_2.$$

Since $\spc{B}$ is a geodesic $\CAT\kappa$ space, we can apply the majorization theorem (\ref{thm:major}) for $\hat\gamma_2$.
The composition of the obtained majorization with $\map$ is a majorization of $\gamma_2$.
\qeds

\parit{Proof.}
The ``only if'' part follows from the Reshetnyak majorization theorem (\ref{thm:major}).

Let  $\gamma_t$, $t\in[0,1]$ 
be a null-homotopy of curves in $\spc{U}$;
that is, $\gamma_0(x)=p$ for some $p\in \spc{U}$
and any $x\in\mathbb{S}^1$.
Assume further that $\length \gamma_t<2\cdot\varpi\kappa$ for any $t$.
To prove the ``only if'' part, it is sufficient to show that $\gamma_1$ is majorized by a convex region in $\Lob2\kappa$. 

By semicontinuity of length (\ref{thm:semicont-of-length}),
we can choose  $\eps>0$ sufficiently small that
$$\length \gamma_t<2\cdot\varpi\kappa-4\cdot\eps$$
for all $t$.

By Corollary~\ref{cor:loc-CAT(k)},
we may assume in addition that
$\oBall(\gamma_t(x),\eps)$ is $\CAT\kappa$ 
for any $t$ and $x$.

Choose a partition $0=t_0<t_1<\dots<t_n=1$
so that $\dist{\gamma_{t_i}(x)}{\gamma_{t_{i-1}}(x)}{}<\eps$
for any $i$ and $x$.
According to \ref{thm:cat-unique},
for any $i$,
the geodesic $[\gamma_{t_i}(x)\gamma_{t_{i-1}}(x)]$ depends continuously on $x$.

Note that $\gamma_0=\gamma_{t_0}$ is majorized by a convex region in $\Lob2\kappa$.
Applying the lemma $n$ times, we see that the same holds for $\gamma_1=\gamma_{t_n}$.\qeds

\begin{thm}{Proposition}\label{prop:sc}
Let $\spc{U}$ be a complete length locally $\CAT\kappa$ space.
Then $\spc{U}$ is simply connected if and only if it is $\infty$-simply connected.
\end{thm}

\parit{Proof.}
To prove the ``if'' part, it is sufficient to show that any closed curve in $\spc{U}$ is homotopic to a closed broken geodesic.

Let $\gamma_0$ be a closed curve in $\spc{U}$.
According to Corollary~\ref{cor:loc-CAT(k)},
there is $\eps>0$ such that 
$\oBall(\gamma(x),\eps)$ is $\CAT\kappa$
for any $x$.

Choose a broken geodesic $\gamma_1$ such that $\dist{\gamma_0}{\gamma_1}{}<\eps$ for any $x$.
By \ref{thm:cat-unique}, 
$\geodpath_{[\gamma_0(x)\gamma_1(x)]}$ 
is uniquely defined 
and depends continuously on~$x$.

Hence $\gamma_t(x)=\geodpath_{[\gamma_0(x)\gamma_1(x)]}(t)$ gives a homotopy from $\gamma_0$ to $\gamma_1$.

The proof of the ``only if'' part is similar.
Assume $\gamma_t$ is a homotopy between two rectifiable curves $\gamma_0$ and $\gamma_1$.
Fix $\eps>0$ so that the ball $\oBall(\gamma_t(x),\eps)$ is $\CAT\kappa$
for any $t$ and $x$.
Choose a partition $0=t_0<t_1<\dots<t_n=1$ 
so that 
$$\dist{\gamma_{t_{i-1}}(x)}{\gamma_{t_i}(x)}{}<\tfrac\eps{10}$$
for any $i$ and $x$.
Set $\hat\gamma_{t_0}=\gamma_0$, $\hat\gamma_{t_n}=\gamma_{t_n}$.
For each $0<i<n$, approximate $\gamma_{t_i}$ by a closed broken geodesic $\hat\gamma_{i}$.

Construct the homotopy 
from $\hat\gamma_{t_{i-1}}$ 
to $\hat\gamma_{t_i}$ 
setting 
$$\hat\gamma_t
=
\geodpath_{[\hat\gamma_{t_{i-1}}(x)\hat\gamma_{t_i}(x)]}(t).$$
Since $\eps$ is sufficiently small, 
by ???  we get
$$\length\hat\gamma_t
<
10\cdot(\length\hat\gamma_{t_{i-1}}+\length\hat\gamma_{t_i}).$$
In particular, $\hat\gamma_t$ is rectifiable for all $t$.

Joining the obtained homotopies for all $i$ we obtain a homotoply from $\gamma_0$ to $\gamma_1$ in the class of rectifiable curves.
\qeds

\begin{thm}{Exercise}
Let $\spc{X}$ be a double cover of $\EE^3$ that branches along two distinct lines $\ell^1$ and $\ell^2$.
Show that  $\spc{X}$ is $\CAT0$ if and only if $\ell^1$ intersects $\ell^2$ at a right angle.
\end{thm}

\parbf{About convex spaces.}
A \emph{convex space} $\spc{X}$ is a geodesic space such that the function
$t\mapsto\dist{\gamma(t)}{\sigma(t)}{}$ is convex 
for any two  geodesic paths $\gamma,\sigma:[0,1]\to \spc{X}$.  
A \emph{locally convex space} is a length space in which every point has a neighborhood that is a convex space in the restricted metric.


\begin{thm}{Exercise}\label{ex:cats-cradle}
Assume $\spc{X}$ is a convex space 
such that the angle of any hinge is defined.
Show that $\spc{X}$ is $\CAT{0}$.
\end{thm}

The following exercise gives an analog of Hadamard--Cartan theorem for locally convex spaces;
see also \cite{a-b:h-c}.

\begin{thm}{Exercise}\label{ex:Hadamard--Cartan}
Show that a complete, simply connected, locally convex space is a convex space.
\end{thm}




%\begin{thm}{Flat triangles lemma} \label{lem:flat-triangle}
%Suppose $\spc{U}$ is a $\varpi\kappa$-geodesic $\CAT\kappa$ space.  
%For a triangle $\trig{x}{y}{z}$ in $\spc{U}$ with model triangle $\trig{\tilde x}{\tilde y}{\tilde z} = \tilde \triangle_\kappa(x y z)$, the following are equivalent:

%\begin{subthm}{thm:flat:angle}

%$\mangle\hinge x y z= \angk\kappa x y z$;
%\end{subthm}

%\begin{subthm}{thm:flat:dist}
%$\dist{x}{w}{}=\dist{\tilde x}{\tilde w}{}$ for some  $\tilde w\in\mathopen{]}\tilde x \tilde z\mathclose{[}$, where $w=f(\tilde w)$ and  $f$ is the natural map;
%\end{subthm}

%\begin{subthm}{thm:flat:isometry} 
%there is a distance-preserving map $F\:\Conv\trig{\tilde x}{\tilde y}{\tilde z} \to \spc{U}$ that maps $\tilde x,\tilde y,\tilde z$ to $x,y,z$ respectively.
%\end{subthm} 
%\end{thm}

%??? I think it is better to write (a) <=> (b) and (a) <=> (c)!!!
%??? Maybe make an exerices from it, this also almost follow from majorization...

%\parit{Proof.}
%By angle-sidelength monotonicity (\ref{cor:monoton-cba}),

%\begin{clm}{}\label{clm:ang=>dist}
%If $\mangle\hinge xyz=\angk\kappa xyz$, then  the restriction $f|([xy]\cup[xz])$ is distance-preserving.\end{clm}

%Suppose $v,w\in \mathopen{]}xy\mathclose{[}$,  in the order $x,v,w,y$. Applying \ref{cat-monoton} to triangles $\trig{z}{x}{y}$, 
%$\trig {z}{x}w$ and $\trig {z}v{y}$, we find that $\dist{z}{v}{}\z=\dist{\tilde z}{\tilde v}{}$ 
%%%%if and only if $\distz}{w}{}=\dist{\tilde z}{\tilde w}{}$, where $v=f(\tilde v)$,  $w=f(\tilde w)$.  Thus by the first variation formula, we have
%\begin{clm}{}\label{clm:dist=>ang}
%If $\dist{z}{w}{}=\dist{\tilde z}{\tilde w}{}$ for some $w\in \mathopen{]}xy\mathclose{[}$, then $\mangle\hinge yxz=\angk\kappa yxz$.
%\end{clm}

%\parit{(\ref{SHORT.thm:flat:angle})$\Leftrightarrow$(\ref{SHORT.thm:flat:dist}).}  Apply claims \ref{clm:ang=>dist} and \ref{clm:dist=>ang} for permutations of $x,y, z$.

%\parit{(\ref{SHORT.thm:flat:dist})$\Leftrightarrow$(\ref{SHORT.thm:flat:isometry}).} 
%Note that (\ref{SHORT.thm:flat:dist}) implies that the natural map $f\:\trig{\tilde x}{\tilde y}{\tilde z} \to\trig xyz$ is distance-preserving.Then by the triangle inequality for angles at $x$,\[\mangle\hinge x {v}{ w}= \angk\kappa x {v}{ w}\eqlbl{eq:flat-angle}\] for any $v,w\in \mathopen{]}yz\mathclose{[}$.  

%Let $(t,s)\mapsto \gamma_t(s)$ be the line-of-sight map for  $[y z]$ from  $x$ (defined in \ref{def:sight}),
%and $(t,s)\mapsto \tilde \gamma_t(s)$ be the line-of-sight map for $[\tilde y\tilde z]$ from $\tilde x$.  By \ref{eq:flat-angle} and angle-sidelength monotonicity (\ref{cor:monoton-cba}), the map  
%$F\:\Conv\trig{\tilde x}{\tilde y}{\tilde z} \to \spc{U}$ defined by $F\:\tilde \gamma_t(s)\mapsto \gamma_t(s)$ is distance-preserving.  Hence  (\ref{SHORT.thm:flat:isometry}). 

%The implication (\ref{SHORT.thm:flat:isometry})$\Rightarrow$(\ref{SHORT.thm:flat:dist}) holds trivially.
%\qeds
 
 
 
%\begin{thm}{Flat triangles lemma} \label{lem:flat-triangle}
%Suppose $\spc{U}$ is a $\varpi\kappa$-geodesic $\CAT\kappa$ space.  
%For a triangle 
%$\trig{x}{y}{z}$ in $\spc{U}$ with model triangle $\trig{\tilde x}{\tilde y}{\tilde z} = \tilde \triangle_\kappa(x y z)$, the following are equivalent:

%\begin{subthm}{thm:flat:angle}
%$\mangle\hinge x y z= \angk\kappa x y z$,
%\end{subthm}

%\begin{subthm}{thm:flat:dist}
%$\dist{x}{w}{}=\dist{\tilde x}{\tilde w}{}$ for some (and therefore any) 
%$w\in ]y z[$ and $\tilde w\in]\tilde y\tilde z[$ 
%such that $\dist{y}{w}{}=\dist{\tilde y }{\tilde w}{}$, 
%\end{subthm}

%\begin{subthm}{thm:flat:isometry} 
%there is a distance-preseving %???!!!
% map $F\:\Conv\trig{\tilde x}{\tilde y}{\tilde z} \to \spc{U}$ that maps $\tilde x,\tilde y,\tilde z$ to $x,y,z$ respectively.
%\end{subthm} 
%\end{thm}

%\parit{Proof.}
%Note that from point-on-side and angle comparison 
%(\ref{cat-monoton}+\ref{SHORT.cat-hinge})
% we get the following.

%\begin{clm}{}\label{clm:ang=>dist}
%If $\mangle\hinge xyz=\angk\kappa xyz$, then for any $w\in \mathopen{]}x y\mathclose{[}$ we have $\dist{z}{w}{}=\dist{\tilde z}{\tilde w}{}$, where $\tilde w\in[\tilde x\tilde y]$ such that $\dist{\tilde x}{\tilde w}{}=\dist{x}{w}{}$.
%\end{clm}

%Further,  let $v,w\in \mathopen{]}xy\mathclose{[}$ and they appear on the geodesic in the order $x,v,w,y$.
%Denote by $\tilde v$ and $\tilde w$ the corresponding points on $[\tilde x\tilde y]$;
%i.e, $\dist{x}{v}{}=\dist{\tilde x}{\tilde v}{}$ 
%and $\dist{x}{w}{}\z=\dist{\tilde x}{\tilde w}{}$.
%Applying \ref{cat-monoton}, to triangles $\trig{z}{x}{y}$, 
%$\trig {z}{x}w$ and $\trig {z}v{y}$,
%we get that $\dist{z}{v}{}\z=\dist{\tilde z}{\tilde v}{}$ 
%if and only if
%$\dist{z}{w}{}=\dist{\tilde z}{\tilde w}{}$.
%In particular if $\dist{z}{w}{}\z=\dist{\tilde z}{\tilde w}{}$
%then the value $\angk\kappa xvz$ is the same for all $v\in\mathopen{]}xy]$.

%Hence the Strong angle lemma (\ref{lem:strong-angle-cba}) we get the following.

%\begin{clm}{}\label{clm:dist=>ang}
%If $\dist{x}{w}{}=\dist{\tilde x}{\tilde w}{}$ for some $w\in \mathopen{]}yz\mathclose{[}$
%then 
%$\mangle\hinge yxz=\angk\kappa yxz$.
%\end{clm}

%Claims \ref{clm:ang=>dist} and \ref{clm:dist=>ang}
%applied for permutations of $x,y, z$ proves (\ref{SHORT.thm:flat:angle})$\Leftrightarrow$(\ref{SHORT.thm:flat:dist}).


%\parit{(\ref{SHORT.thm:flat:angle})$+$(\ref{SHORT.thm:flat:dist})$\Rightarrow$(\ref{SHORT.thm:flat:isometry}).}
%Set 
%\begin{align*}
%\gamma&=\geodpath_{[xz]},
%&
%\tilde \gamma&=\geodpath_{[\tilde x\tilde z]},
%\\
%\gamma_t&=\geodpath_{[y  \gamma(t)]},
%&
%\tilde \gamma_t&=\geodpath_{[\tilde y\tilde \gamma(t)]}.
%\end{align*}
%Note that (\ref{SHORT.thm:flat:dist}) implies that
%$\trig{\tilde x}{\tilde \gamma(t^1)}{\tilde \gamma(t^2)}$ is the model triangle for $\trig{x} {\gamma(t^1)}{\gamma(t^2)}$, for $0\le t^1<t^2\le 1$.
%Then (\ref{SHORT.thm:flat:angle}) and  the triangle inequality for angles at $x$ implies
%\[\mangle\hinge x {\gamma(t^1)}{ \gamma(t^2)}= \angk\kappa x {\gamma(t^1)}{ \gamma(t^2)}.\]  By angle-sidelength monotonicity (\ref{cor:monoton-cba}), $\dist{\tilde \gamma_{t^1}(u^1)}{\tilde \gamma_{t^2}(u^2))}{}
%=\dist{\gamma_{t^1}(u^1)}{\gamma_{t^2}(u^2)}{}$ for $0\le u^1,u^2\le 1$. 
%Therefore the map $F\:\Conv\trig{\tilde x}{\tilde y}{\tilde z} \to \spc{U}$ 
%defined by $F\:\tilde \gamma_t(u)\mapsto\gamma_t(u)$ 
%is distance-preserving.  
%Hence (\ref{SHORT.thm:flat:isometry}). 

%The implication (\ref{SHORT.thm:flat:isometry})$\Rightarrow$(\ref{SHORT.thm:flat:dist}) holds trivially.
%\reds

%%%%%%%%%%%%%%%%%%%%%%%%%%%%%%%%%%%%%%%%%%%%%%%%%%%%%%%%%%%%%%%%

\section{Angles}
\label{sec:angles-cba}

Recall that $\o$ denotes a selective nonpriciple ultrafilter on $\NN$, see Section~\ref{ultralimits}. 

\begin{thm}{Angle semicontinuity}\label{lem:ang.semicont}
Let $\spc{U}_n$  be a $\varpi\kappa$-geodesic $\CAT\kappa$  space for each $n\in \NN$
and $\spc{U}_n\to \spc{U}_\o$ as $n\to\o$.
Assume that a sequence of hinges $\hinge{p_n}{x_n}{y_n}$ in $\spc{U}_n$ converges to a hinge $\hinge{p_\o}{x_\o}{y_\o}$ in  $\spc{U}_\o$.
Then 
\[\mangle\hinge{p_\o}{x_\o}{y_\o}
\ge 
\lim_{n\to\o} \mangle\hinge{p_n}{x_n}{y_n}.\]

\end{thm}

%\begin{wrapfigure}[6]{r}{23mm}
%\begin{lpic}[t(0mm),b(10mm),r(0mm),l(0mm)]{pics/ang.semicont(0.12)}
%\lbl[rt]{7,18;$p_n$}
%\lbl[r]{30,94;$\bar x_n$}
%\lbl[b]{116,160;$x_n$}
%\lbl[t]{59,2;$\bar y_n$}
%\lbl[t]{174,20;$y_n$}
%\end{lpic}
%\end{wrapfigure}

\parit{Proof.}
By angle-sidelength monotonicity (\ref{cor:monoton-cba}),
\[\mangle\hinge{p_\o}{x_\o}{y_\o}
=
\inf\set{\angk\kappa{p_\o}{\bar x_\o}{\bar y_\o}}{\bar x_\o \in \mathopen{]}p_\o x_\o],\ \bar y_\o\in \mathopen{]}p_\o y_\o]}.\]

For fixed $\bar x_\o \in \mathopen{]}p_\o x_\o]$ 
and $\bar y_\o\in \mathopen{]}p_\o x_\o]$,
choose $\bar x_n\in \mathopen{]} p x_n ]$ and $\bar y_n\in \mathopen{]} p y_n ]$ so that $\bar x_n\to \bar x_\o$ 
and $\bar y_n\to \bar y_\o$ as $n\to\o$.
Clearly 
\[\angk\kappa{p_n}{\bar x_n}{\bar y_n}
\to 
\angk\kappa{p_\o}{\bar x_\o}{\bar y_\o}\] 
as $n\to\o$.

By angle comparison (\ref{cat-hinge}), $\mangle\hinge{p_n}{x_n}{y_n}\le \angk\kappa{p_n}{\bar x_n}{\bar y_n}$.
Hence the result.
\qeds

Now we verify that the first variation formula 
holds in the $\CAT{}$ setting. 
Compare first variation inequality (\ref{lem:first-var}) for general metric spaces and the
strong angle lemma (\ref{1st-var+}) for $\Alex{}$ spaces. 

\begin{thm}{Strong angle lemma}
\label{lem:strong-angle-cba}
Let $\spc{U}$ be a $\varpi\kappa$-geodesic $\CAT\kappa$  space.
Then for any hinge  $\hinge  p q x$ in $\spc{U}$, 
we have
\[\mangle\hinge p q x
=
\lim_{\bar x\to p}
\set{\angk\kappa p q{\bar x}}{\bar x\in\mathopen{]}px]}
\eqlbl{eq:cba-1st-var+***}\]
for any $\kappa\in\RR$ such that $\dist{p}{q}{}<\varpi\kappa$.
\end{thm}

\parit{Proof.} 
By angle-sidelength  monotonicity  (\ref{cor:monoton-cba}), the righthand side is defined and at least equal to the lefthand side. 

By Lemma~\ref{lem:k-K-angle}, we may take $\kappa = 0$ in \ref{eq:cba-1st-var+***}.  
By the cosine law and the first variation inequality (\ref{lem:first-var}),  
the righthand side is at most equal to the lefthand side.
\qeds



\begin{thm}{First variation}\label{thm:1st-var-cba}
Let $\spc{U}$ be a $\varpi\kappa$-geodesic $\CAT\kappa$  space.
For any nontrivial geodesic $[py]$ in $\spc{U}$ and point $q\ne p$ such that  $\dist{p}{q}{}<\varpi\kappa$, we have 
\[\dist{q}{\geod_{[p y]}(t)}{}
=
\dist{q}{p}{}-t\cdot\cos\mangle\hinge p q y+o(t).
%\eqlbl{eq:cba-1st-var}
\]
\end{thm}

\parit{Proof.}
The first variation equation is equivalent to the strong angle lemma (\ref{lem:strong-angle-cba}), as follows from the Euclidean cosine law.
\qeds

\begin{thm} {First variation (both-endpoints version)}\label{cor:both-end-first-var-cba}
Let  $\spc{U}$ be a $\varpi\kappa$-geodesic $\CAT\kappa$  space.
For any nontrivial geodesics $[py]$ and $[qz]$ in $\spc{U}$  such that $p\ne q$ and $\dist{p}{q}{}<\varpi\kappa$, we have 
\[
\dist{\geod_{[p y]}(t)}{\geod_{[q z]}(\tau)}{}
=
\dist{q}{p}{} - t\cdot\cos\mangle\hinge p q y - \tau\cdot\cos\mangle\hinge q p z+o(t+\tau).
%\eqlbl{eq:cba-1st-var}
\]
\end{thm}

\parit{Proof.}
By the first variation equation (\ref{thm:1st-var-cba}),
\[\begin{aligned}
&\dist{\geod_{[p y]}(t)}{\geod_{[q z]}(\tau)}{} =
\\
&=
\dist{q}{\geod_{[p y]}(t)}{} - \tau\cdot\cos\mangle\hinge q  {\geod_{[p y]}(t)} z +o(\tau)\\
&=\dist{q}{p}{} - t\cdot\cos\mangle\hinge p q y + o(t) -  \tau\cdot\cos\mangle\hinge q  {\geod_{[p y]}(t)} z +o(\tau)=\\
&= \dist{q}{p}{} - t\cdot\cos\mangle\hinge p q y -  \tau\cdot\cos\mangle\hinge q  p z +o(t+\tau).
\end{aligned}
\]
Here the final equality follows from   
\[
\lim_{t\to 0}\mangle\hinge q  {\geod_{[p y]}(t)} z = \mangle\hinge q  p z.
\eqlbl{eq:2-side-variation}
\]
``$\le$'' in \ref{eq:2-side-variation}  holds by angle semicontinuity (\ref{lem:ang.semicont}). ``$\ge$'' holds by the triangle inequality for angles, since angle comparison (\ref{cat-hinge}) gives 
\[
\lim_{t\to 0}\mangle\hinge q p  {\geod_{[p y]}(t)} = 0.
\]
\qedsf

\section{Remarks and open problems}






The following question was known in folklore in the 80's,
%!!! Dave Berg remembers this
but it seems that in print
it was first mentioned in \cite{kopecka-reich}. %!!!
We do not see any reason why it should be true, 
but we also cannot construct a counterexample.

\begin{thm}{Open question}
Let $\spc{U}$ be a complete length $\CAT0$ space and $K\subset \spc{U}$ be a compact set.
Is it true that $K$ lies in a convex compact set $\bar K\subset\spc{U}$?
\end{thm}

The question can  easily be reduced to the case when $K$ is finite;
so far it is not even known if any three points in a complete length $\CAT0$ space lie in a compact convex set.















