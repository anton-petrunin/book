%%%!TEX root = arXiv.tex
%%%%arXiv
\chapter{Fundamentals of curvature bounded above}
\chaptermark{Fundamentals of CBA}

%%%%%%%%%%%%%%%%%%%%%%%%%%%%%%%%%%%%%%%%%%%%%%%%%%%%%%%%%%%%%%%%%%%%%%%%%%%%%%%%%%%%%%%%%%%%
\section{Four-point comparison.} \label{sec:cba-def}

\index{$\CAT{}$}
\begin{thm}{Four-point comparison}\label{def:2+2}
A quadruple of points $p^1,p^2,x^1,x^2$ in a metric space 
satisfies 
\index{$\CAT{}$!$\CAT\kappa$ comparison}
\emph{$\CAT\kappa$ comparison}
if
  
\begin{subthm}{}
$\angk{\kappa}{p^1}{x^1}{x^2} 
\le 
\angk{\kappa}{p^1}{p^2}{x^1}+\angk{\kappa}{p^1}{p^2}{x^2}$, or
\end{subthm}

\begin{subthm}{}
$\angk{\kappa} {p^2}{x^1}{x^2}\le \angk{\kappa} {p^2}{p^1}{x^1} + \angk{\kappa} {p^2}{p^1}{x^2}$, or
\end{subthm}

\begin{subthm}{}
one of the six model angles 
\begin{align*}
\angk{\kappa}{p^1}{x^1}{x^2},\quad&\angk{\kappa}{p^1}{p^2}{x^1},\quad\angk{\kappa}{p^1}{p^2}{x^2},
\\
\angk{\kappa}{p^2}{x^1}{x^2},\quad&\angk{\kappa}{p^2}{p^1}{x^1},\quad\angk{\kappa}{p^2}{p^1}{x^2}
\end{align*}
is undefined.
\end{subthm}
\end{thm}

\begin{wrapfigure}{r}{30 mm}
\vskip-0mm
\centering
\includegraphics{mppics/pic-905}
\end{wrapfigure}

Here is a more intuitive formulation.

\begin{thm}{Reformulation}\label{def:2+2-reformulated}
Let $\spc{X}$ be a metric space.
A quadruple $p^1,p^2,x^1,x^2\in \spc{X}$ satisfies 
$\CAT\kappa$ comparison if one of the following holds:
\begin{subthm}{}
One of the triples 
$(p^1,p^2,x^1)$ 
or 
$(p^1, p^2, x^2)$ 
has perimeter $>2\cdot\varpi\kappa$.
\end{subthm}

\begin{subthm}{}
If $\trig{\tilde p^1}{\tilde p^2}{\tilde x^1}
=
\modtrig\kappa(p^1 p^2 x^1)$ 
and
$\trig{\tilde p}{\tilde p^2}{\tilde x^2}
\z=
\modtrig\kappa p^1 p^2 x^2$, then
\[\dist{\tilde x^1}{\tilde z}{}+\dist{\tilde z}{\tilde x^2}{}\ge \dist{x^1}{x^2}{},\]
for any $\tilde z\in[\tilde p^1\tilde p^2]$.

\end{subthm}

\end{thm}

\begin{thm}{Definition}
\label{def:ccat}
Let $\spc{U}$ be a metric space.

\begin{subthm}{}
$\spc{U}$ is 
\index{$\CAT{}$!$\CAT{\kappa}$ space} 
$\CAT{\kappa}$ 
if any quadruple $p^1,p^2,x^1,x^2\in \spc{X}$  satisfies  $\CAT{\kappa}$ comparison.
\end{subthm}

\begin{subthm}{}
$\spc{U}$ is 
\index{$\CAT{}$!locally $\CAT{\kappa}$ space}
\emph{locally $\CAT{\kappa}$} 
if any point $q\in \spc{U}$ admits a neighborhood $\Omega\ni q$ such that any quadruple $p^1,p^2,x^1,x^2\in \spc{X}$  satisfies  $\CAT\kappa$ comparison.
\end{subthm}

\begin{subthm}{}
$\spc{U}$  is a  
\index{$\CAT{}$!$\CAT{}$ space}
$\CAT{}$ space if  $\spc{U}$  is $\CAT{\kappa}$ for some $\kappa\in\RR$.
\end{subthm}
\end{thm}

%\begin{thm}{Definition}
%\label{def:ccat}
%A metric space $\spc{U}$ 
%is called $\CAT{\kappa}$ if every quadruple $p^1,p^2,x^1,x^2$ satisfies the $\CAT\kappa$ comparison (\ref{def:2+2}).  

%We say that $\spc{U}$ is $\CAT{}$ if it is $\CAT{\kappa}$ for some $\kappa\in\RR$.  
%\end{thm}

The condition $\spc{U}$ is $\CAT\kappa$ should be understood as ``$\spc{U}$ has global curvature $\le\kappa$''.
In Proposition~\ref{prop:inherit-bound}, it will be shown that this formulation makes sense; 
in particular, if $\kappa\le\Kappa$, then any $\CAT\kappa$ space is $\CAT\Kappa$.


This terminology was introduced by Michael Gromov;  
$\CAT{}$ stands for \'Elie Cartan, Alexandr Alexandrov, and Victor Toponogov.
Originally these spaces were called \index{$\mathfrak{R}_\kappa$ domain}\emph{$\mathfrak{R}_\kappa$ domains};
this is Alexandrov's terminology and is still in use.


\begin{thm}{Exercise}\label{ex:ccat-(3+1)}
Let $\spc{U}$ be a metric space.
Show that $\spc{U}$ is $\CAT\kappa$
if and only if every quadruple of points in $\spc{U}$ admits a labeling by $(p,x^1,x^2,x^3)$ such that the three angles 
$\angk\kappa p{x^1}{x^2}$,
$\angk\kappa p{x^2}{x^3}$ and
$\angk\kappa p{x^1}{x^3}$
satisfy all three triangle inequalities or one of these angles is undefined.
\end{thm}

\begin{thm}{Exercise}\label{ex:sba-2+2-short}
Show that $\spc{U}$ is $\CAT\kappa$
if and only if for any quadruple of points 
$p^1,p^2,x^1,x^2$ in $\spc{U}$ such that
$\dist{p^1}{p^2}{},\dist{x^1}{x^2}{}\le \varpi\kappa$,
there is a quadruple $q^1,q^2,y^1,y^2$ in $\Lob m\kappa$
such that 
\begin{align*}
\dist{q^1}{q^2}{}&=\dist{p^1}{p^2}{},
&
\dist{y^1}{y^2}{}&=\dist{x^1}{x^2}{},
&
\dist{q^i}{y^j}{}&\le \dist{p^i}{x^j}{}
\end{align*}
for any $i$ and $j$.
\end{thm}

\begin{thm}{Advanced exercise}\label{ex:berg-nikolaev}
Let $\spc{U}$ be a complete length space such that for any quadruple $p,q,x,y\in\spc{L}$ 
the following inequality holds
\[\dist[2]{p}{q}{}+\dist[2]{x}{y}{}\le \dist[2]{p}{x}{}
+\dist[2]{p}{y}{}+\dist[2]{q}{x}{}+\dist[2]{q}{y}{}.
\eqlbl{eq:berg-nikolaev-CAT}\]
Prove that $\spc{U}$ is $\CAT0$.

Construct a 4-point metric space $\spc{X}$ that satisfies inequality \ref{eq:berg-nikolaev-CAT} for any relabeling of its points by $p,q,x,y$, and such that $\spc{X}$ is not $\CAT{0}$.
\end{thm}

The next proposition follows directly from Definition \ref{def:ccat} and the definitions of ultralimit and ultrapower;
see Section~\ref{ultralimits} for the related definitions.
Recall that $\o$ denotes a fixed selective ultrafilter on $\NN$.


\begin{thm}{Proposition}
\label{prop:CAT^omega}
Let $\spc{U}_n$ be a $\CAT{\kappa_n}$ space for each $n\in\NN$.
Assume $\spc{U}_n\to \spc{U}_\o$ and $\kappa_n\to\kappa_\o$ as $n\to\o$.
Then $\spc{U}_\o$ is $\CAT{\kappa_\o}$.

Moreover, a metric space $\spc{U}$ is $\CAT\kappa$ if and only if so is its ultrapower~$\spc{U}^\o$.

\end{thm} 

\section{Geodesics}

\begin{thm}{Uniqueness of geodesics}\label{thm:cat-unique}\label{thm:cat-complete} 
In a complete length $\CAT\kappa$ space, pairs of points at distance $<\varpi\kappa$ are joined by unique geodesics, and these geodesics depend continuously on their endpoint pairs.
\end{thm}

\parit{Proof.} 
Fix a complete length $\CAT\kappa$ space $\spc{U}$.
Fix two points $p^1,p^2\in \spc{U}$  such that 
\[\dist{p^1}{p^2}{\spc{U}}<\varpi\kappa.\]

Choose a sequence of approximate midpoints $z_n$ between $p^1$ and $p^2$;
that is,  
\[\dist{p^1}{z_n}{},\dist{p^2}{z_n}{}
\to\tfrac12\cdot\dist[{{}}]{p^1}{p^2}{}
\quad\text{as}\quad n\to\infty.
\eqlbl{eq:to|p1p2|/2}\]

By the law of cosines, $\angk{\kappa} {p^1}{z_n}{p^2}$ and $\angk{\kappa} {p^2}{z_n}{p^1}$ are arbitrarily small when $n$ is sufficiently large.

Let us apply $\CAT\kappa$  comparison (\ref{def:2+2}) to the quadruple $p^1$, $p^2$, $z_n$, $z_\kay$ with large $n$ and $\kay$.
We conclude that  $\angk{\kappa} {p}{z_n}{z_\kay}$ is arbitrarily small when $n,\kay$ are sufficiently large and $p$ is either $p^1$ or $p^2$.  
By \ref{eq:to|p1p2|/2} and the law of cosines, the sequence $z_n$ converges.  

Since $\spc{U}$ is complete, the sequence $z_n$ converges to a midpoint between $p^1$ and $p^2$. 
By Lemma~\ref{lem:mid>geod} we obtain  the existence of a geodesic $[p^1p^2]$.

Now suppose $p^1_n\to p^1$, $p^2_n\to p^2$ as $n\to\infty$.
Let $z_n$ be the midpoint of a geodesic $[p^1_n p^2_n]$ and $z$ be the midpoint of a geodesic $[p^1p^2]$.  

It suffices to show that 
\[\dist{z_n}{z}{}\to0
\quad \text{as}\quad 
n\to\infty.
\eqlbl{eq:z_n->z}\]

By the triangle inequality, the $z_n$ are approximate midpoints between $p^1$ and $p^2$.
Apply the $\CAT\kappa$ comparison (\ref{def:2+2}) to the quadruple $p^1$, $p^2$, $z_n$,~$z$. 
For $p=p^1$ or $p=p^2$, we see that $\angk{\kappa} {p}{z_n}{z}$ is arbitrarily small when $n$ is sufficiently large.  
By the law of cosines, \ref{eq:z_n->z} follows.
\qeds

\begin{thm}{Exercise}\label{ex:CAT-mnfld=>ext.geod}
Let $\spc{U}$ be a complete length $\CAT{}$ space.
Assume $\spc{U}$ is a topological manifold.
Show that any geodesic in $\spc{U}$ can be extended 
as a two-side infinite local geodesic.

Moreover the same holds for any locally geodesic locally $\CAT{}$ space $\spc{U}$ with nontrivial local homology groups at any point;
the latter holds in particular if $\spc{U}$ is a homological manifold.
\end{thm}

\begin{thm}{Exercise}\label{ex:complete-space-of-dir}
Assume $\spc{U}$ is a locally compact geodesic $\CAT{}$ space with extendable geodesics;
that is, any geodesic in $\spc{U}$ can be extended to a both-sided infinite local geodesic.

Show that the space of geodesic directions $\Sigma_p'$ is complete for any $p\in \spc{U}$.
\end{thm}

By the uniqueness of geodesics (\ref{thm:cat-unique}),
we have the following.

\begin{thm}{Corollary}\label{cor:cat-ccat}
Any  complete length $\CAT\kappa$ space is $\varpi\kappa$-geodesic.

\end{thm}

\begin{thm}{Proposition}\label{cor:cat-completion} 
The completion $\bar{\spc{U}}$ of any geodesic $\CAT{\kappa}$ space $\spc{U}$ is a complete length $\CAT\kappa$ space.

Moreover, $\spc{U}$ is a geodesic $\CAT\kappa$ space
if and only if there is a complete length $\CAT\kappa$ space $\bar{\spc{U}}$ that contains a $\varpi\kappa$-convex dense set isometric to $\spc{U}$.
\end{thm}

\parit{Proof.} 
By Theorem \ref{thm:cat-complete},
in order to show that  $\bar{\spc{U}}$ is $\CAT{\kappa}$,
it is sufficient to verify that the completion of a length space is a length space; 
this is straightforward.

For the second part, note that the completion $\bar{\spc{U}}$
contains the original space $\spc{U}$ as a dense $\varpi\kappa$-convex subset, and the metric on $\spc{U}$ coincides with the induced length metric from $\bar{\spc{U}}$.
\qeds

Here is a corollary from Proposition~\ref{cor:cat-completion}
and Theorem~\ref{thm:cat-unique}.

\begin{thm}{Corollary}\label{cor:cat-unique}
Let $\spc{U}$ be a  $\varpi\kappa$-geodesic $\CAT\kappa$ space.
Then pairs of points in $\spc{U}$ at distance less than $\varpi\kappa$ are joined by unique geodesics, and these geodesics depend continuously on their endpoint pairs.

Moreover for any pair of points $p,q\in \spc{U}$ and any value
\[\Lip>\sup\set{\frac{\sn\kappa r}{\sn\kappa \dist{p}{q}{}}}{0\le r\le \dist{p}{q}{}}\]
there are neighborhoods $\Omega_p\ni p$ and $\Omega_q\ni q$ such that the map
\[(x,y,t)\mapsto \geodpath_{[xy]}(t)\]
is $\Lip$-Lipschitz in $\Omega_p\times \Omega_q\times[0,1]$.
\end{thm}

\parit{Proof.}
By Proposition~\ref{cor:cat-completion}, any geodesic $\CAT{\kappa}$ space is isometric to a convex dense subset of a complete length $\CAT\kappa$ space.
It remains to apply  Theorem~\ref{thm:cat-unique}.
\qeds


%%%%%%%%%%%%%%%%%%%%%%%%%%%%%%%%%%%%%%%%%%%%%%%%%%%%%%%%%%%%%%%%%%%%%%%%%%%%%%%%%%%%%%%%%

\section{More comparisons}\label{sec:cat-angles}

Here we give a few reformulations of Definition~\ref{def:ccat}.

\begin{wrapfigure}{r}{25 mm}
\vskip-0mm
\centering
\includegraphics{mppics/pic-910}
\end{wrapfigure}

\begin{thm}{Theorem}
\label{thm:defs_of_cat} 
If $\spc{U}$ is a $\CAT\kappa$ space, then 
the following conditions hold for all triples $p,x,y\in \spc{U}$ of perimeter $<2\cdot\varpi\kappa$:

\begin{subthm}{cat-2-sum} (adjacent angle comparison\index{comparison!adjacent angle comparison}) for any geodesic $[x y]$ and $z\in \mathopen{]}x y\mathclose{[}$, we have
\[\angk\kappa z p x
+\angk\kappa z p y\ge \pi.\]
\end{subthm}

\begin{subthm}{cat-monoton}
(point-on-side comparison\index{comparison!point-on-side comparison}) 
for any geodesic $[x y]$ and $z\in \mathopen{]}x y\mathclose{[}$, we have
\[\angk\kappa x p y\ge\angk\kappa x p z,\]
or equivalently, 
\[\dist{\tilde p}{\tilde z}{}\ge \dist{p}{z}{},\]
where $\trig{\tilde p}{\tilde x}{\tilde y}=\modtrig\kappa(p x y)$, $\tilde z\in\mathopen{]} \tilde x\tilde y\mathclose{[}$, $\dist{\tilde x}{\tilde z}{}=\dist{x}{z}{}$.
\end{subthm}

\begin{subthm}{cat-hinge}(hinge comparison\index{comparison!hinge comparison})
for any hinge $\hinge x p y$, the angle 
$\mangle\hinge x p y$ exists and
\[\mangle\hinge x p y\le\angk\kappa x p y,\]
or equivalently,
\[\side\kappa \hinge x p y\le\dist{p}{y}{}.\]
\end{subthm}
%SBA:  a ``hinge'' in English is a ``movable mechanism''.  It must MOVE. BBI uses it correctly:  if you look up ``hinge'' in BBI index, it sends you to monotonicity.

Moreover, if  $\spc{U}$ is  $\varpi\kappa$-geodesic, then the converse holds in each case.  

\end{thm}


\parbf{Remark.}
\label{22remark}
In the following proof, the part \ref{SHORT.cat-hinge}$\Rightarrow$\ref{SHORT.cat-2-sum})
only requires that the $\CAT\kappa$ comparison (\ref{def:2+2}) hold for any quadruple, and does not require the existence of geodesics at distance $<\varpi\kappa$. 
The same is true of the parts \ref{SHORT.cat-2-sum}$\Leftrightarrow$\ref{SHORT.cat-monoton} and
\ref{SHORT.cat-monoton}$\Rightarrow$\ref{SHORT.cat-hinge}.  
Thus the conditions \ref{SHORT.cat-2-sum}, \ref{SHORT.cat-monoton}) and \ref{SHORT.cat-hinge} are valid for any metric space (not necessarily a length space) that satisfies $\CAT\kappa$ comparison (\ref{def:2+2}). 
The converse does not hold; for example, all these conditions are 
vacuously true in a 
totally disconnected space, while 
$\CAT\kappa$ comparison is not.

\parit{Proof; \ref{SHORT.cat-2-sum}}. 
Since the perimeter of $p,x,y$ is $<2\cdot \varpi\kappa$, so is the perimeter of any subtriple of $p,z,x,y$ by the triangle inequality. 
By Alexandrov's lemma (\ref{lem:alex}), 
\[\angk\kappa p z x +\angk\kappa p z y  < \angk{\kappa} p x y \quad \text{or}\quad  \angk\kappa z p x  +\angk\kappa z p y  =\pi.\]
In the former case, the $\CAT\kappa$ comparison (\ref{def:2+2}) applied to the quadruple $p, z, x, y$ implies
\[\angk\kappa z p x  +\angk\kappa z p y  \ge \angk{\kappa} z x y =\pi.\]

\parit{\ref{SHORT.cat-2-sum}$\Leftrightarrow$\ref{SHORT.cat-monoton}.}
Follows from  Alexandrov's lemma (\ref{lem:alex}).

\parit{\ref{SHORT.cat-monoton}$\Rightarrow$\ref{SHORT.cat-hinge}.} 
By \ref{SHORT.cat-monoton}, for $\bar p\in\mathopen{]}x p]$ and $\bar y\in\mathopen{]}x y]$ the function $(\dist{x}{\bar p}{},\dist{x}{\bar y}{})\mapsto\angk\kappa x{\bar p}{\bar y}$ is nondecreasing in each argument.
In particular, 
$\mangle\hinge x p y\z=\inf\angk\kappa x{\bar p}{\bar y}$.
Thus $\mangle\hinge x p y$ exists and is
at most $\angk\kappa x p y$. 

\parit{Converse.} Assume $\spc{U}$ is $\varpi\kappa$-geodesic. 
Let us first show that in this case \ref{SHORT.cat-hinge}$\Rightarrow$\ref{SHORT.cat-2-sum}.

\begin{wrapfigure}{r}{30 mm}
\vskip-4mm
\centering
\includegraphics{mppics/pic-915}
\end{wrapfigure}

Indeed, by \ref{SHORT.cat-hinge} and the triangle inequality for angles (\ref{claim:angle-3angle-inq}),
\[\angk\kappa z p x
+\angk\kappa z p y \ge \mangle\hinge z p x
+\mangle\hinge z p y \ge \pi.\]
It remains to prove the converse for \ref{SHORT.cat-monoton}.

Given a quadruple  $p^1,p^2,x^1,x^2$ whose subtriples have perimeter $<2\cdot\varpi\kappa$, we must verify the $\CAT\kappa$ comparison (\ref{def:2+2}).
In $\Lob2\kappa$, construct the model triangles  $\trig{\tilde p^1}{\tilde p^2}{\tilde x^1} = \modtrig\kappa(p^1 p^2 x^1 )$ 
and $\trig{\tilde p^1}{\tilde p^2}{\tilde x^2}\z= \modtrig\kappa(p^1 p^2 x^2)$, lying on either side of a common segment $[\tilde p^1 \tilde p^2]$.
We may suppose 
\[\angk{\kappa} {p^1}{p^2}{x^1}+\angk{\kappa} {p^1}{p^2}{x^2}
\le
\pi
\quad \text{and}\quad 
\angk{\kappa}{p^2}{p^1}{x^1}+\angk{\kappa} {p^2}{p^1}{x^2}
\le 
\pi,\] 
since otherwise $\CAT\kappa$ comparison holds trivially.  
Then $[\tilde p^1 \tilde p^2]$ and $[\tilde x^1 \tilde x^2]$ intersect, say at $\tilde q$.  

By assumption, there is a geodesic $[p^1 p^2]$.
Choose $q\in[p^1 p^2]$ corresponding to $\tilde q$; 
that is, $\dist{p^1}{q}{}=\dist{\tilde p^1}{\tilde q}{}$.
Then 
\[\dist{x^1}{x^2}{} \le \dist{x^1}{q}{} + \dist{q}{x^2}{} \le \dist{\tilde x^1}{\tilde q}{} + \dist{\tilde q}{\tilde x^2}{} = \dist{\tilde x^1}{\tilde x^2}{},\]
where the second inequality follows from \ref{SHORT.cat-monoton}. 
By monotonicity of the function $a\mapsto\tangle\mc\kappa\{a;b,c\}$ (\ref{increase}),
\begin{align*}
\angk{\kappa} {p^1}{x^1}{x^2} \le  \mangle\hinge{ \tilde p^1}{ \tilde x^1}{ \tilde x^2}
= \angk{\kappa} {p^1}{p^2}{x^1} + \angk{\kappa} {p^1}{p^2}{x^2}.
\end{align*}
\qedsf

Let us display a corollary of the proof of \ref{thm:defs_of_cat},
namely, monotonicity of the model angle with respect to adjacent sidelengths. 

\begin{thm}{Angle-sidelength  monotonicity}\label{cor:monoton-cba} 
Suppose $\spc{U}$ is a $\varpi\kappa$-geodesic $\CAT\kappa$ space, and 
$p,x,y\in \spc{U}$ have  perimeter $<2\cdot \varpi\kappa$.
Then for $\bar y\in\mathopen{]}x y]$, the function 
\[\dist{x}{\bar y}{}\mapsto \angk\kappa x p{\bar y}\] 
is nondecreasing.

In particular, if $\bar p\in \mathopen{]}x p]$, then
\begin{subthm}{two-mono-cba}the function 
\[(\dist{x}{\bar y}{},\dist{x}{\bar p}{})\mapsto \angk\kappa x {\bar p}{\bar y}\] is nondecreasing in each argument,
\end{subthm}
 
\begin{subthm}{cor:monoton-cba:angle=inf} 
$\mangle\hinge{x}{p}{y}
=
\inf\set{\angk\kappa x {\bar p}{\bar y}}{
\bar p\in\mathopen{]}x p],\ 
\bar y\in\mathopen{]}x y]}.$
\end{subthm}
\end{thm}

\begin{thm}{Exercise}\label{mink+CAT=euclid} 
Let $\spc{U}$ be  $\RR^m$ with the metric defined by a norm.
Show that $\spc{U}$ is a complete length $\CAT{}$ space if and only if $\spc{U}\iso\EE^m$.
\end{thm}

\begin{thm}{Exercise}\label{ex:convexity-CAT0}
Assume $\spc{U}$ is a geodesic $\CAT0$ space.
Show that for any two geodesic paths 
$\gamma,\sigma\:[0,1]\to \spc{U}$
the function 
\[t\mapsto\dist{\gamma(t)}{\sigma(t)}{}\] 
is convex.
\end{thm}

\begin{thm}{Proposition}
\label{prop:inherit-bound}
Assume $\kappa<\Kappa$.
Then any complete length $\CAT\kappa$ space is $\CAT\Kappa$.

Moreover a space $\spc{U}$ is $\CAT\kappa$ if  $\spc{U}$ is $\CAT\Kappa$ for all $\Kappa>\kappa$.
\end{thm}

\parit{Proof.}
The first statement follows from Corollary \ref{cor:cat-ccat}, the adjacent-angles comparison (\ref{cat-2-sum}) and the monotonicity of the function $\kappa\mapsto\angk\kappa x y z$ (\ref{k-decrease}).

The second statement follows since the function $\kappa\mapsto\angk\kappa x y z$ is continuous.
\qeds

%%%%%%%%%%%%%%%%%%%%%%%%%%%%%%%%%%%%%%%%%%%%%%%%%%%%%%%%%%%%%%%%%%%%%%%%%%%%%%%%%%%%%%%%%%%%

\section{Thin triangles} \label{sec:thin-triangle}

In this section we define thin triangles
and use them to characterize $\CAT{}$ spaces.
Inheritance for thin triangles with respect to decomposition
is the main result of this section.
It will lead to two fundamental constructions:  
Alexandrov's patchwork globalization  (\ref{thm:alex-patch}) 
and Reshetnyak gluing (\ref{thm:gluing}).
 
\begin{thm}{Definition of $\bm\kappa$-thin triangles}\label{def:k-thin}
Let $\trig{x^1}{x^2}{x^3}$ be a triangle of perimeter $<2\cdot \varpi\kappa$ in a metric space
and
$\trig{\tilde x^1}{\tilde x^2}{\tilde x^3}\z=\modtrig\kappa({x^1}{x^2}{x^3})$.
Consider the \emph{natural map} $\trig{\tilde x^1}{\tilde x^2}{\tilde x^3}\to \trig{x^1}{x^2}{x^3}$ 
that sends a point $\tilde z\in[\tilde x^i\tilde x^j]$ to the corresponding point $z\in[x^ix^j]$
(that is, such that $\dist{\tilde x^i}{\tilde z}{}=\dist{x^i}{z}{}$ and therefore $\dist{\tilde x^j}{\tilde z}{}=\dist{x^j}{z}{}$).

We say the triangle $\trig{x^1}{x^2}{x^3}$ is \index{thin triangle}\emph{$\kappa$-thin} if the natural map $\trig{\tilde x^1}{\tilde x^2}{\tilde x^3}\to \trig{x^1}{x^2}{x^3}$ is short.
\end{thm}

\begin{thm}{Exercise}\label{ex:equality-for-thin}
Let $\spc{U}$ be a $\varpi\kappa$-geodesic $\CAT\kappa$ space.
Let $\trig xyz$ be a triangle in $\spc{U}$
and $\trig{\tilde x}{\tilde y}{\tilde z}$ be its model triangle in $\Lob{2}{\kappa}$.
Prove that the natural map $f\:\trig{\tilde x}{\tilde y}{\tilde z}\to \trig xyz$ 
 is distance-preserving if and only if one of the following conditions hold:

\begin{subthm}{ex:equality-for-thin:angle}
$\mangle\hinge x y z= \angk\kappa x y z$,
\end{subthm}

\begin{subthm}{ex:equality-for-thin:vertex-base}
$\dist{x}{w}{}=\dist{\tilde x}{\tilde w}{}$ for some  $\tilde w\in]\tilde y\tilde z[$ and
$w= f(\tilde w)$,   
\end{subthm}

\begin{subthm}{ex:equality-for-thin:side-side} 
$\dist{v}{w}{}=\dist{\tilde v}{\tilde w}{}$ for some  
$\tilde v\in \mathopen{]}\tilde x \tilde y\mathclose{[}$,  $\tilde w\in\mathopen{]}\tilde x \tilde z\mathclose{[}$
and $v=f(\tilde v)$, $w=f(\tilde w)$.
\end{subthm} 

\end{thm}

{\sloppy 

\begin{thm}{Proposition}\label{prop:k-thin}
Let $\spc{U}$ be a $\varpi\kappa$-geodesic space. 
Then $\spc{U}$ is  $\CAT\kappa$
if and only if every triangle of perimeter $<2\cdot \varpi\kappa$ in $\spc{U}$  is $\kappa$-thin.
\end{thm}

}

\parit{Proof.}
The if part follows from the point-on-side comparison (\ref{cat-monoton}).  
The only-if part follows from angle-sidelength  monotonicity (\ref{two-mono-cba}).
\qeds


\begin{thm}{Corollary}\label{cor:loc-geod-are-min}
Suppose $\spc{U}$ is a $\varpi\kappa$-geodesic $\CAT\kappa$ space.  
Then any local geodesic in $\spc{U}$ of length $<\varpi\kappa$ is length-minimizing.
\end{thm}

\parit{Proof.}
Suppose $\gamma\:[0,\ell]\to\spc{U}$ is a local geodesic  that is not minimizing, with $\ell<\varpi\kappa$.
Choose $a$ to be the maximal value 
such that $\gamma$ is minimizing on $[0,a]$.
Further choose $b>a$ so that $\gamma$ is minimizing on~$[a,b]$.

Since triangle $\trig{\gamma(0)}{\gamma(a)}{\gamma(b)}$ is $\kappa$-thin, we have
\[\dist{\gamma(a-\eps)}{\gamma(a+\eps)}{}<2\cdot\eps\]
for all small $\eps>0$,
a contradiction.
\qeds


Now let us formulate the main result of this section.
The inheritance lemma states that  in any metric space, a triangle is $\kappa$-thin if it decomposes into $\kappa$-thin triangles. 
In contrast, $\kappa$-thickness of triangles (\ref{ex:fat-triangle}) is not inherited in this way.

\begin{wrapfigure}{r}{25 mm}
\vskip-0mm
\centering
\includegraphics{mppics/pic-920}
\end{wrapfigure}

\begin{thm}{Inheritance lemma}
\label{lem:inherit-angle} 
In a metric space, consider a triangle $\trig p x y$ that \index{decomposed triangle}\emph{decomposes} 
into two triangles $\trig p x z$ and $\trig p y z$;
that is, $\trig p x z$ and $\trig p y z$ have common side $[p z]$, and the sides $[x z]$ and $[z y]$ together form the side $[x y]$ of $\trig p x y$.

If the triangle $\trig p x y$ has perimeter $<2\cdot\varpi\kappa$
and both triangles $\trig p x z$ and $\trig p y z$ are $\kappa$-thin, then triangle $\trig p x y$ is  $\kappa$-thin.
\end{thm} 

The following model-space lemma is  extracted from Lemma 2 in \cite{reshetnyak:major}.


\begin{thm}{Lemma}\label{lem:quadrangle}
Let $\trig{\tilde p}{\tilde x}{\tilde y}$ be a triangle in $\Lob2{\kappa}$ and $\tilde z\in[\tilde x\tilde y]$.
Consider the solid triangle $\tilde D=\Conv\trig{\tilde p}{\tilde x}{\tilde y}$.  
Construct  points $\dot p, \dot x, \dot z, \dot y\in \Lob2{\kappa}$ such that 
\begin{align*}
\dist{\dot p}{\dot x}{}&=\dist{\tilde p}{\tilde x}{},
&
\dist{\dot p}{\dot y}{}&=\dist{\tilde p}{\tilde y}{},
&
\dist{\dot p}{\dot z}{}&\le \dist{\tilde p}{\tilde z}{},
\\
\dist{\dot x}{\dot z}{}&=\dist{\tilde x}{\tilde z}{},
&
\dist{\dot y}{\dot z}{}&=\dist{\tilde y}{\tilde z}{},
\end{align*}
where points $\dot x$ and $\dot y$ lie on either side of $[\dot p\dot z]$.
Set 
\[\dot D=\Conv\trig {\dot p}{\dot x}{\dot z}\cup \Conv\trig {\dot p} {\dot y} {\dot z}.\]

Then there is a short map $F\:\tilde D\to \dot D$ that maps $\tilde p$, $\tilde x$, $\tilde y$ and $\tilde z$ to $\dot p$, $\dot x$, $\dot y$ and $\dot z$ respectively.
\end{thm}

{

\begin{wrapfigure}{r}{41 mm}
\vskip-0mm
\centering
\includegraphics{mppics/pic-925}
\end{wrapfigure}

\parit{Proof.} 
By Alexandrov's lemma (\ref{lem:alex}), 
there are nonoverlapping triangles 
$\trig{\tilde p}{\tilde x}{\tilde z_y}\iso\trig {\dot p}{\dot x}{\dot z}$ 
and 
$\trig{\tilde p}{\tilde y}{\tilde z_x}\iso\trig {\dot p}{\dot y}{\dot z}$
 inside triangle $\trig{\tilde p}{\tilde x}{\tilde y}$.

Connect points in each pair
$(\tilde z,\tilde z_x)$, 
$(\tilde z_x,\tilde z_y)$ 
and $(\tilde z_y,\tilde z)$ 
with arcs of circles centered at 
$\tilde y$, $\tilde p$, and $\tilde x$ respectively. 
Define $F$ as follows.

}

\begin{itemize}
\item Map  $\Conv\trig{\tilde p}{\tilde x}{\tilde z_y}$ isometrically onto  $\Conv\trig {\dot p}{\dot x}{\dot y}$;
similarly map $\Conv \trig{\tilde p}{\tilde y}{\tilde z_x}$ onto $\Conv \trig {\dot p}{\dot y}{\dot z}$.

\item If $w$ is in one of the three circular sectors, say at distance $r$ from the center of the circle, let $F(w)$ be the point on  
$[\dot p \dot z]$, 
$[\dot x \dot z]$,
or $[\dot y \dot z]$ whose distance from the left-hand endpoint of the segment is $r$.
\item Finally, if $w$ lies in the remaining curvilinear triangle $\tilde z \tilde z_x \tilde z_y$, 
set $F(w) = \dot z$. 
\end{itemize}
By construction, $F$ meets the conditions of the lemma. 
\qeds


\parit{Proof of \ref{lem:inherit-angle}.}
Construct model triangles $\trig{\dot p}{\dot x}{\dot z}\z=\modtrig\kappa(p x z)$ 
and $\trig {\dot p} {\dot y} {\dot z}\z=\modtrig\kappa(p y z)$ so that $\dot x$ and $\dot y$ lie on opposite sides of $[\dot p\dot z]$.

\begin{wrapfigure}{r}{30 mm}
\vskip-0mm
\centering
\includegraphics{mppics/pic-930}
\end{wrapfigure}

Suppose
\[\angk\kappa{z}{p}{x}+\angk\kappa{z}{p}{y}
<
\pi.\]
Then for some point $\dot w\in[\dot p\dot z]$, we have \[\dist{\dot x}{\dot w}{}+\dist{\dot w}{\dot y}{}
<
\dist{\dot x}{\dot z}{}+\dist{\dot z}{\dot y}{}=\dist{x}{y}{}.\]
Let $w\in[p z]$ correspond to $\dot w$; that is, $\dist{z}{w}{}=\dist{\dot z}{\dot w}{}$. 
Since $\trig p x z$ and $\trig p y z$ are $\kappa$-thin, we have 
\[\dist{x}{w}{}+\dist{w}{y}{}<\dist{x}{y}{},\]
contradicting the triangle inequality. 

Thus 
\[\angk\kappa{z}{p}{x}+\angk\kappa{z}{p}{y}
\ge
\pi.\]
By Alexandrov's lemma (\ref{lem:alex}), this is equivalent to 
\[\angk\kappa x p z\le\angk\kappa x p y.
\eqlbl{eq:for|pz|}\]

Let $\trig{\tilde  p}{\tilde  x}{\tilde  y}=\modtrig\kappa (p x y)$ 
and $\tilde  z\in[\tilde  x\tilde  y]$ correspond to $z$; that is, $\dist{x}{z}{}=\dist{\tilde  x}{\tilde  z}{}$.
Inequality~\ref{eq:for|pz|} is equivalent to $\dist{ p}{ z}{}\le \dist{\tilde  p}{\tilde  z}{}$.
Hence  Lemma~\ref{lem:quadrangle} applies.  Therefore 
there is a short map $F$ that  sends 
$\trig{\tilde  p}{\tilde  x}{\tilde  y}$ to $\dot D=\Conv\trig {\dot p}{\dot x}{\dot z}\cup \Conv\trig {\dot p} {\dot y} {\dot z}$ 
in such a way that 
$\tilde p\mapsto \dot p$,
$\tilde x\mapsto \dot x$,
$\tilde z\mapsto \dot z$
and
$\tilde y\mapsto \dot y$.

By assumption, the natural maps $\trig {\dot p} {\dot x} {\dot z}\to\trig p x z$ and $\trig {\dot p} {\dot y} {\dot z}\to\trig p y z$ are short.  
By composition,  the natural map from $\trig{\tilde  p}{\tilde  x}{\tilde  y}$ to $\trig p y z$ is short, as claimed.
\qeds


%%%%%%%%%%%%%%%%%%%%%%%%%%%%%%%%%%%%%%%%%%%%%%%%%%%%%%%%%%%%%%%%%%%%%%%%%%%%%%%%%%%%%%%%%%%


\section{Function comparison} \label{sec:func-comp}


\index{comparison!function comparison}
In this section we give analytic and geometric ways of viewing the point-on-side comparison (\ref{cat-monoton}) as a convexity condition.

First we obtain a corresponding differential inequality for the distance function in $\spc{U}$;
see Section~\ref{sec:conv-fun} for the definition.
 
\begin{thm}{Theorem}\label{thm:function-comp} 
Suppose $\spc{U}$ is a $\varpi\kappa$-geodesic space. 
Then the following are equivalent:
\begin{subthm}{function-comp-cat} 
$\spc{U}$ is $\CAT\kappa$,
\end{subthm}
\begin{subthm}{function-comp}
for any $p\in \spc{U}$, the function $f=\md\kappa\circ\distfun{p}{}{}$ satisfies 
\[f''+\kappa \cdot f\ge 1\] 
in $\oBall(p,\varpi\kappa)$.
\end{subthm}\end{thm}

\begin{thm}{Corollary}
A geodesic space $\spc{U}$ is $\CAT{0}$ if and only if for any $p\in \spc{U}$, the function $\distfun[2]{p}{}{}\:\spc{U}\to\RR$ is $2$-convex.
\end{thm}


\parit{Proof of \ref{thm:function-comp}.}
Fix a sufficiently short geodesic $[x y]$ in $\oBall(p,\varpi\kappa)$.
We can assume that the model triangle $\trig{\tilde p}{\tilde x}{\tilde y}\z=\modtrig\kappa(p x y)$ is defined. 
Let \begin{align*} 
\tilde r(t)&=\dist{\tilde p}{\geod_{[\tilde x\tilde y]}(t)}{},
& 
r(t)&=\dist{p}{\geod_{[xy]}(t)}{}.                           \end{align*}
Let $\tilde f=\md\kappa\circ\tilde r$ and $f=\md\kappa\circ r$.
By \ref{md-diff-eq}, we have $\tilde f''\z=1-\kappa\cdot  \tilde f$.
Clearly $\tilde f(t)$ and $f(t)$ agree at $t=0$ and $t=\dist{x}{y}{}$. 
The point-on-side comparison (\ref{cat-monoton}) is the condition $r(t)\le\tilde r(t)$  for all $t\in[0,\dist{x}{y}{}]$.
Since $\md\kappa$ is increasing on $[0,\varpi\kappa)$, then $r\le \tilde r$ and $f\le \tilde f$ are equivalent.
Thus the claim follows by Jensen's inequality (\ref{y''-mono}).
\qeds

\begin{thm}{Corollary}\label{cor:convex-balls}
Suppose $\spc{U}$ is a $\varpi\kappa$-geodesic $\CAT\kappa$ space.
Then any ball (closed or open) of radius $R<\tfrac{\varpi\kappa}2$ in $\spc{U}$ is convex.

Moreover, any open ball of radius $\tfrac{\varpi\kappa}2$ is convex
and any closed ball of radius $\tfrac{\varpi\kappa}2$ is $\varpi\kappa$-convex.
\end{thm}

\parit{Proof.}
Suppose $p\in\spc{U}$, $ R\le\varpi\kappa/2$,  and two points 
$x$ and $y$ lie in $\cBall[p, R]$ or $\oBall(p, R)$.
By the triangle inequality, if $\dist{x}{y}{}<\varpi\kappa$, then any
 geodesic $[x y]$ lies in $\oBall(p, \varpi\kappa)$.
 
By the function comparison (\ref{thm:function-comp}), 
the geodesic $[x y]$ lies in $\cBall[p,R]$ or $\oBall(p,R)$ respectively.

Thus any ball (closed or open) of radius $R<\tfrac{\varpi\kappa}2$ is $\varpi\kappa$-convex.
This implies convexity unless there is a pair of points in the ball at distance at least $\varpi\kappa$.
By the  triangle inequality, the latter is possible only for the closed ball of radius $\tfrac{\varpi\kappa}2$.
\qeds

Recall that Busemann functions are defined in Proposition \ref{prop:busemann}.
The following exercise is analogous to Exercise~\ref{ex:busemann-CBB}.

\begin{thm}{Exercise}\label{ex:busemann-CBA}{\sloppy 
Let $\spc{U}$ be a complete length $\CAT\kappa$ space
and $\bus_\gamma\:\spc{U}\to \RR$ be the Busemann function for a half-line $\gamma\:[0,\infty)\to \spc{L}$.

}

\begin{subthm}{}
If $\kappa=0$, then the Busemann function $\bus_\gamma$ is  convex.
\end{subthm}

\begin{subthm}{}
If $\kappa=-1$, then the function $f=\exp\circ\bus_\gamma$ satisfies
\[f''- f\ge 0.\]
\end{subthm}

\end{thm}

\section{Development}\label{sec:development-CBA}
 
Geometrically,  the development construction (\ref{def:devel}) translates distance comparison into a local convexity statement for subsets of $\Lob2\kappa$.
Recall that a curve in $\Lob2\kappa$ is \emph{(locally) concave} with respect to $p$ if (locally) its supergraph with respect to $p$ is a convex subset of $\Lob2\kappa$; see Definition~\ref{def:convex-devel}.

\begin{thm}{Development criterion\index{comparison!development comparison}}\label{thm:concave-devel} 
For a $\varpi\kappa$-geodesic space $\spc{U}$,
the following statements hold:

\begin{subthm}{locally-concave-dev}
For any $p\in \spc{U}$ and any geodesic $\gamma\:[0,T]\to\oBall(p,\varpi\kappa)$, suppose the $\kappa$-development $\tilde \gamma$ in $\Lob2\kappa$ of $\gamma$ with respect to $p$ is locally concave. 
Then $\spc{U}$ is $\CAT\kappa$.
\end{subthm}

\begin{subthm}{concave-dev} 
If $\spc{U}$ is $\CAT\kappa$, then for any geodesic $\gamma\:[0,T]\to\spc{U}$ and $p\in \spc{U}$
such that the triangle $\trig{p}{\gamma(0)}{\gamma(T)}$ has perimeter $<2\cdot\varpi\kappa$,
the $\kappa$-development $\tilde \gamma$ in $\Lob2\kappa$ of $\gamma$ with respect to $p$ is concave. 
\end{subthm}

\end{thm}


\parit{Proof; \ref{SHORT.locally-concave-dev}.}  
Let  $\gamma=\geod_{[x y]}$ and $T=\dist{x}{y}{}$. 
Let $\tilde \gamma\:[0,T]\to\Lob2\kappa$ be the concave $\kappa$-development based at $\tilde p$ of $\gamma$ with respect to $p$. 
Let us show that the function  
\[t\mapsto \angk\kappa x p{\gamma(t)}
\eqlbl{eq:ang-nondecreasing}\]   
is nondecreasing. 

For a partition $0=t^0<t^1<\dots<t^n=T$, let 
\[\tilde y^i=\tilde \gamma(t^i)\quad \text{and}\quad \tau^i=\dist{\tilde y^0}{\tilde y^1}{}+\dist{\tilde y^1}{\tilde y^2}{}+\dots+\dist{\tilde y^{i-1}}{\tilde y^i}{}.\]  
Since $\tilde \gamma$ is locally concave, 
for a sufficiently fine partition the polygonal line $\tilde y^0\tilde y^1\dots\tilde y^n$ is  locally concave with respect to $\tilde p$. 
Alexandrov's lemma (\ref{lem:alex}), applied inductively to pairs of triangles  $\modtrig\kappa \{\tau^{i-1},\dist{p}{\tilde y^0}{},\dist{p}{\tilde y^{i-1}}{}\}$ and  $\modtrig\kappa\{\dist{\tilde y^{i-1}}{\tilde y^i}{}, \dist{p}{\tilde y^{i-1}}{},\dist{p}{\tilde y^{i}}{}\}$, shows that the sequence  $\tilde \mangle\mc\kappa\{\dist{\tilde p}{\tilde y^{i}}{};\dist{\tilde p}{\tilde y^0}{},\tau^i\}$ is nondecreasing.

Taking finer partitions and passing to the limit, we get
\[\max\nolimits_i\{|\tau^i-t^i|\}\to0.\] 
Therefore \ref{eq:ang-nondecreasing} and 
the point-on-side comparison (\ref{cat-monoton}) follows. 



\parit{\ref{SHORT.concave-dev}.}  
Consider a partition $0=t^0<t^1<\dots<t^n=T$, and 
let $x^i\z=\gamma(t^i)$. Construct a chain of model triangles  $\trig{\tilde p}{\tilde x^{i-1}}{\tilde x^i}=\modtrig\kappa(p x^{i-1}x^i)$ with the direction of $[\tilde p\tilde x^i]$ turning counterclockwise as $i$ grows. 
By the angle comparison (\ref{cat-hinge}),
\[\mangle\hinge{\tilde x^i}{\tilde x^{i-1}}{\tilde p}+\mangle\hinge{\tilde x^i}{\tilde x^{i+1}}{\tilde p}\ge\pi.\eqlbl{eq1:concave-devel*}
\] 
Since $\gamma$ is a geodesic, 
 \[\length \gamma = \sum_{i=1}^n\dist{x^{i-1}}{x^i}{}\le \dist{p}{x^0}{}+\dist{p}{x^n}{}.
\eqlbl{eq2:concave-devel*}
\]  
By repeated application of Alexandrov's lemma (\ref{lem:alex}), and inequality~\ref{eq2:concave-devel*}, 
\[\sum_{i=1}^n\mangle\hinge{\tilde p}{\tilde x^{i-1}}{\tilde x^i}
\le
\angk\kappa p{x^0}{x^n}\le\pi.\] 
Then by \ref{eq1:concave-devel*},  the polygonal line $\tilde p\tilde x^0\tilde x^1\dots \tilde x^n$  are concave with respect to~$\tilde p$.

Note that  under finer partitions, the polygonal line $\tilde x^0\tilde x^1\dots \tilde x^n$ approach the development of $\gamma$ with respect to $p$.
Since the polygonal lines are convex, their lengths converge to the length of $\gamma$.
Hence the result. 
\qeds


%%%%%%%%%%%%%%%%%%%%%%%%%%%%%%%%%%%%%%%%%%%%%%%%%%%%%%%%%%%%%%%%%%

\section{Patchwork globalization}\label{sec:patchwork}

If $\spc{U}$ is a $\CAT\kappa$ space, then it is locally $\CAT\kappa$.
The converse does not hold even for complete length space.
For example, $\mathbb{S}^1$ is locally isometric to $\RR$, and so
is locally $\CAT0$, but it is easy to find a quadruple of points in $\mathbb{S}^1$ that violates $\CAT0$ comparison.  

The following theorem was essentially proved by Alexandr Alexandrov \cite[Satz 9]{alexandrov:devel}; 
it gives a global condition on geodesics that is  necessary and sufficient for a locally $\CAT\kappa$ space to be globally $\CAT\kappa$. 
The proof uses thin-triangle decompositions 
and the inheritance lemma (\ref{lem:inherit-angle}). 

\begin{thm}{Patchwork globalization theorem}\label{thm:alex-patch}
For any complete length space~$\spc{U}$, the following two statements are equivalent:

\begin{subthm}{thm:alex-patch:ccat}
$\spc{U}$ is $\CAT\kappa$.
\end{subthm}
 
\begin{subthm}{thm:alex-patch:geo-uni}
$\spc{U}$ is locally $\CAT\kappa$; moreover,  pairs of points in $\spc{U}$ at distance $<\varpi\kappa$ are joined by unique geodesics, and these geodesics depend continuously on their endpoint pairs.
\end{subthm}

\end{thm}

Note that the implication \ref{SHORT.thm:alex-patch:ccat}$\Rightarrow$\ref{SHORT.thm:alex-patch:geo-uni} follows from Theorem~\ref{thm:cat-unique}.

\begin{thm}{Corollary}\label{cor:k-for-k}
Let $\spc{U}$ be a complete length  space 
and $\Omega\subset\spc{U}$ be an open locally $\CAT\kappa$ subset. 
Then for any point $p\in \Omega$ there is $R>0$ such that $\cBall[p,R]$ is a convex subset of $\spc{U}$ 
and $\cBall[p,R]$ is $\CAT\kappa$.
\end{thm}

\parit{Proof.}
Fix $R>0$ such that $\CAT\kappa$ comparison holds in $\oBall(p,R)$.

We may assume that $\oBall(p,R)\subset\Omega$ and $R<\varpi\Kappa$.
The same argument as in the proof of the theorem on uniqueness of geodesics (\ref{thm:cat-unique}) 
shows that any two points in $\cBall[p,\tfrac R2]$ can be joined by a unique geodesic that depends continuously on the endpoints.

The same argument as in the proof of Corollary \ref{cor:convex-balls} shows that $\cBall[p,\tfrac R2]$ is a convex set.
Then \ref{SHORT.thm:alex-patch:geo-uni}$\Rightarrow$\ref{SHORT.thm:alex-patch:ccat} of the patchwork globalization theorem implies that $\cBall[p,\tfrac R2]$ is $\CAT\kappa$.
\qeds

The proof of patchwork globalization uses the following construction:

\begin{thm}{Definition (Line-of-sight map)} \label{def:sight}
%Let $\spc{U}$ be a metric space in which pairs of points at distance $<\varpi\kappa$ are joined by unique geodesics and these geodesics depend continuously on their endpoint pairs. 
Let  $p$ be a point and $\alpha$ be a curve of finite length in  a length space $\spc{U}$. 
Let $\bar\alpha:[0,1]\to\spc{U}$ be the constant-speed parametrization of $\alpha$.
If $\gamma_t\:[0,1]\to\spc{U}$ is a geodesic path from $p$ to $\bar\alpha(t)$, we say that the map $[0,1]\times[0,1]\to\spc{U}$ defined by
\[(t,s)\mapsto\gamma_t(s)\]
is a \index{line-of-sight map}\emph{line-of-sight map} for $\alpha$ with respect to $p$.
\end{thm}

Note that a line-of-sight map is closely related to geodesic homotopy (Section~\ref{sec:Hadamard--Cartan}).

\parit{Proof of \ref{thm:alex-patch}.}
It only remains to prove \ref{SHORT.thm:alex-patch:geo-uni}$\Rightarrow$\ref{SHORT.thm:alex-patch:ccat}.

Let $[p x y]$ be a triangle of perimeter $<2\cdot\varpi\kappa$  in $\spc{U}$. 
According to \ref{prop:k-thin} and \ref{prop:inherit-bound}, it is sufficient to show the triangle $\trig p x y$ is $\kappa$-thin.

Since pairs of points at distance $<\varpi\kappa$ are joined by unique geodesics and these geodesics depend continuously on their endpoint pairs, there is a unique and continuous line-of-sight map for  $[x y]$ with respect to~$p$.    

For a partition \[0\z=t^0\z<t^1\z<\z\dots\z<t^N=1,\] 
let $x^{i,j}=\gamma_{t^i}(t^j)$. 
\begin{figure}[!ht]
\vskip0mm
\centering
\includegraphics{mppics/pic-935}
\end{figure}
Since the line-of-sight map is continuous, we may assume each triangle $\trig{x^{i,j}}{x^{i,j+1}}{x^{i+1,j+1}}$ and $\trig{x^{i,j}}{x^{i+1,j}}{x^{i+1,j+1}}$ is $\kappa$-thin 
(see Proposition~\ref{prop:k-thin}).

Now we show that the $\kappa$-thin property propagates to $\trig p x y$, by repeated application of the inheritance lemma (\ref{lem:inherit-angle}):
\begin{itemize}
\item 
First, for fixed $i$, 
sequentially applying the lemma shows  that the triangles 
$\trig{x}{x^{i,1}}{x^{i+1,2}}$, 
$\trig{x}{x^{i,2}}{x^{i+1,2}}$, 
$\trig{x}{x^{i,2}}{x^{i+1,3}}$,
and so on are $\kappa$-thin. 
\end{itemize}
In particular, for each $i$, the long triangle $\trig{x}{x^{i,N}}{x^{i+1,N}}$ is $\kappa$-thin.
\begin{itemize} 
\item 
Applying the lemma again shows that the  triangles $\trig{x}{x^{0,N}}{x^{2,N}}$, $\trig{x}{x^{0,N}}{x^{3,N}}$, and so on are $\kappa$-thin. 
\end{itemize}
In particular, $\trig p x y=\trig{p}{x^{0,N}}{x^{N,N}}$ is $\kappa$-thin.
\qeds

The following exercise implies that if the space is proper, then one can drop the condition on continuous dependence of geodesics in the formulation of patchwork globalization.

\begin{thm}{Exercise}\label{ex:patchwork}
\begin{subthm}{ex:patchwork:proper}
Suppose pairs of points in a geodesic space $\spc{U}$ are joined by unique geodesics.
Show that if $\spc{U}$ is proper, then 
these geodesics depend continuously on their endpoint pairs.
\end{subthm}

\begin{subthm}{ex:patchwork:complete}
Construct an example of a complete geodesic space $\spc{U}$ such that 
pairs of points in $\spc{U}$ are joined by unique geodesics, but
these geodesics do not depend continuously on their endpoint pairs.
\end{subthm}
\end{thm}



%%%%%%%%%%%%%%%%%%%%%%%%%%%%%%%%%%%%%%%%%%%%%%%%%%%%%%%%%%%%%%%%

\section{Angles}
\label{sec:angles-cba}

Recall that $\o$ denotes a selective nonprincipal ultrafilter on $\NN$, see Section~\ref{ultralimits}. 

\begin{thm}{Angle semicontinuity}\label{lem:ang.semicont}
Suppose $\spc{U}_1,\spc{U}_2,\dots$ is a sequence of $\varpi\kappa$-geodesic $\CAT\kappa$ spaces
and $\spc{U}_n\to \spc{U}_\o$ as $n\to\o$.
Assume that a sequence of hinges $\hinge{p_n}{x_n}{y_n}$ in $\spc{U}_n$ converges to a hinge $\hinge{p_\o}{x_\o}{y_\o}$ in $\spc{U}_\o$ as $n\to\o$.
Then 
\[\mangle\hinge{p_\o}{x_\o}{y_\o}
\ge 
\lim_{n\to\o} \mangle\hinge{p_n}{x_n}{y_n}.\]

\end{thm}

%\begin{wrapfigure}[6]{r}{23mm}
%\begin{lpic}[t(0mm),b(10mm),r(0mm),l(0mm)]{pics/ang.semicont(0.12)}
%\lbl[rt]{7,18;$p_n$}
%\lbl[r]{30,94;$\bar x_n$}
%\lbl[b]{116,160;$x_n$}
%\lbl[t]{59,2;$\bar y_n$}
%\lbl[t]{174,20;$y_n$}
%\end{lpic}
%\end{wrapfigure}

\parit{Proof.}
By the angle-sidelength monotonicity (\ref{cor:monoton-cba}),
\[\mangle\hinge{p_\o}{x_\o}{y_\o}
=
\inf\set{\angk\kappa{p_\o}{\bar x_\o}{\bar y_\o}}{\bar x_\o \in \mathopen{]}p_\o x_\o],\ \bar y_\o\in \mathopen{]}p_\o y_\o]}.\]

For fixed $\bar x_\o \in \mathopen{]}p_\o x_\o]$ 
and $\bar y_\o\in \mathopen{]}p_\o x_\o]$,
choose $\bar x_n\in \mathopen{]} p x_n ]$ and $\bar y_n\in \mathopen{]} p y_n ]$ so that $\bar x_n\to \bar x_\o$ 
and $\bar y_n\to \bar y_\o$ as $n\to\o$.
Clearly 
\[\angk\kappa{p_n}{\bar x_n}{\bar y_n}
\to 
\angk\kappa{p_\o}{\bar x_\o}{\bar y_\o}\] 
as $n\to\o$.

By the angle comparison (\ref{cat-hinge}), $\mangle\hinge{p_n}{x_n}{y_n}\le \angk\kappa{p_n}{\bar x_n}{\bar y_n}$.
Hence the result.
\qeds

Now we verify that the first variation formula 
holds in the $\CAT{}$ setting. 
Compare it to the first variation inequality (\ref{lem:first-var}) which holds for general metric spaces and to the strong angle lemma (\ref{1st-var+}) for $\Alex{}$ spaces. 

\begin{thm}{Strong angle lemma}
\label{lem:strong-angle-cba}
Let $\spc{U}$ be a $\varpi\kappa$-geodesic $\CAT\kappa$  space.
Then for any hinge  $\hinge  p q y$ in $\spc{U}$, 
we have
\[\mangle\hinge p q y
=
\lim_{\bar y\to p}
\set{\angk\kappa p q{\bar y}}{\bar y\in\mathopen{]}py]}
\eqlbl{eq:cba-1st-var+***}\]
for any $\kappa\in\RR$ such that $\dist{p}{q}{}<\varpi\kappa$.
\end{thm}

\parit{Proof.} 
By angle-sidelength  monotonicity  (\ref{cor:monoton-cba}), the right-hand side is defined and bigger than or  equal to the left-hand side. 

By Lemma~\ref{lem:k-K-angle}, we may take $\kappa = 0$ in \ref{eq:cba-1st-var+***}.  
By the cosine law and the first variation inequality (\ref{lem:first-var}),  
the right-hand side is less than or equal to the left-hand side.
\qeds



\begin{thm}{First variation}\label{thm:1st-var-cba}
Let $\spc{U}$ be a $\varpi\kappa$-geodesic $\CAT\kappa$  space.
For any nontrivial geodesic $[py]$ in $\spc{U}$ and point $q\ne p$ such that  $\dist{p}{q}{}<\varpi\kappa$, we have 
\[\dist{q}{\geod_{[p y]}(t)}{}
=
\dist{q}{p}{}-t\cdot\cos\mangle\hinge p q y+o(t).
%\eqlbl{eq:cba-1st-var}
\]
\end{thm}

\parit{Proof.}
The first variation equation is equivalent to the strong angle lemma (\ref{lem:strong-angle-cba}), as follows from the cosine law.
\qeds

\begin{thm} {First variation (both-endpoints version)}\label{cor:both-end-first-var-cba}
Assume that $\spc{U}$ is a $\varpi\kappa$-geodesic $\CAT\kappa$ space.
Then for any nontrivial geodesics $[py]$ and $[qz]$ in $\spc{U}$  such that $p\ne q$ and $\dist{p}{q}{}<\varpi\kappa$, we have 
\[
\dist{\geod_{[p y]}(t)}{\geod_{[q z]}(\tau)}{}
=
\dist{q}{p}{} - t\cdot\cos\mangle\hinge p q y - \tau\cdot\cos\mangle\hinge q p z+o(t+\tau).
%\eqlbl{eq:cba-1st-var}
\]
\end{thm}

\parit{Proof.}
By \ref{cat-hinge},
\[\begin{aligned}
&\dist{\geod_{[p y]}(t)}{\geod_{[q z]}(\tau)}{} \ge
\\
&\ge
\dist{q}{\geod_{[p y]}(t)}{} - \tau\cdot\cos\mangle\hinge q  {\geod_{[p y]}(t)} z +o(\tau)\ge\\
&\ge\dist{q}{p}{} - t\cdot\cos\mangle\hinge p q y + o(t) -  \tau\cdot\cos\mangle\hinge q  {\geod_{[p y]}(t)} z +o(\tau)=\\
&= \dist{q}{p}{} - t\cdot\cos\mangle\hinge p q y -  \tau\cdot\cos\mangle\hinge q  p z +o(t+\tau).
\end{aligned}
\]
Here the final equality follows from   
\[
\lim_{t\to 0}\mangle\hinge q  {\geod_{[p y]}(t)} z = \mangle\hinge q  p z.
\eqlbl{eq:2-side-variation}
\]
The angle semicontinuity (\ref{lem:ang.semicont}) implies ``$\le$'' in \ref{eq:2-side-variation}, and ``$\ge$'' holds by the triangle inequality for angles, since angle comparison (\ref{cat-hinge}) gives 
\[
\lim_{t\to 0}\mangle\hinge q p  {\geod_{[p y]}(t)} = 0.
\]

The opposite inequality follows from \ref{thm:1st-var-cba} and the triangle inequality
\[\dist{\geod_{[p y]}(t)}{\geod_{[q z]}(\tau)}{}
\le
\dist{\geod_{[p y]}(t)}{m}{}+\dist{m}{\geod_{[q z]}(\tau)}{},\]
where $m$ is the midpoint of $[pq]$.
\qeds

We have given elementary proofs of the first-variation statements \ref{lem:strong-angle-cba}, \ref{thm:1st-var-cba} and \ref{cor:both-end-first-var-cba}.
Note however that the no-conjugate-point theorem \ref{thm:no-conj-pt} not only provides proofs of these statements but also extends the statements from geodesics in $\CAT\kappa$ spaces to local geodesics in locally $\CAT\kappa$ spaces as follows:
 
\begin{thm}{First variation for local geodesics}\label{cor:1st-var++cba}
Let $\gamma_t\:[0,1]\to \spc{U}$ be a continuous family of local geodesics in a locally $\CAT\kappa$.
Set
$\alpha(t)=\gamma_t(0)$ and $\beta(t)=\gamma_t(1)$.
Suppose that $\gamma_0$ is unit-speed and $\alpha^+(0)$ and $\beta^+(0)$ are defined.
Then 
\[\length\gamma_t
=
\length\gamma_0
-
(\langle\alpha^+(0),\gamma_0^+(0)\rangle+\langle\beta^+(0),\gamma_0^-(1)\rangle)\cdot t
+
o(t).\]

\end{thm}

\section{Reshetnyak gluing theorem}\label{sec:cba-gluing}

The following theorem was proved by Yuriy Reshetnyak \cite{reshetnyak:major}, assuming $\spc{U}^1$, $\spc{U}^2$ are proper and complete. 
In the following form, the theorem appears in the book of Martin Bridson and Andr\'e Haefliger \cite{bridson-haefliger}.

\begin{thm}{Reshetnyak gluing theorem}\label{thm:gluing}
Suppose 
$\spc{U}^1$, $\spc{U}^2$ are %complete 
$\varpi\kappa$-geodesic spaces 
with isometric complete $\varpi\kappa$-convex sets $A^i\subset\spc{U}^i$.  Let $\iota\:A^1\to A^2$ be an isometry.
Let $\spc{W}=\spc{U}^1\sqcup_{\iota}\spc{U}^2$;
that is, $\spc{W}$ is the gluing of $\spc{U}^1$ and  $\spc{U}^2$ along $\iota$ (see Section~\ref{sec:quotient}).

Then: 
\begin{subthm}{gluing0}
Both canonical mappings $\jmath_i\:\spc{U}^i\to\spc{W}$ are distance-preserving 
and the images $\jmath_i(\spc{U}^i)$ are $\varpi\kappa$-convex subsets in $\spc{W}$.
\end{subthm}

\begin{subthm}{gluing2}
If $\spc{U}^1, \spc{U}^2$ are $\CAT\kappa$,
then so is $\spc{W}$.
\end{subthm} 
\end{thm}

\parit{Proof.} 
Part \ref{SHORT.gluing0}
follows directly from $\varpi\kappa$-convexity of the $A^i$.

\parit{\ref{SHORT.gluing2}.} 
According to \ref{SHORT.gluing0},
we can identify $\spc{U}^i$ with its image $\jmath_i(\spc{U}^i)$ in $\spc{W}$;
in this way, the subsets $A^i\subset \spc{U}^i$ will be identified and denoted further by $A$.
Thus   $A=\spc{U}^1\cap \spc{U}^2\subset \spc{W}$,
and $A$ is $\varpi\kappa$-convex in $\spc{W}$.

Part \ref{SHORT.gluing2} can be reformulated as follows:

\begin{thm}{Reformulation of \ref{gluing2}}\label{thm:gluing2-reformulated}
Let $\spc{W}$ be a 
length space having two 
$\varpi\kappa$-convex subsets $\spc{U}^1,\spc{U}^2\subset\spc{W}$ such that
$\spc{W}=\spc{U}^1\cup\spc{U}^2$.
Assume the subset $A=\spc{U}^1\cap \spc{U}^2$ is complete and $\varpi\kappa$-convex in $\spc{W}$, and $\spc{U}^1$, $\spc{U}^2$ are $\CAT\kappa$ spaces.
Then $\spc{W}$ is a $\CAT\kappa$ space.
\end{thm}

\begin{clm}{}\label{clm:geod-gluing}
If $\spc{W}$ is $\varpi\kappa$-geodesic, then $\spc{W}$ is $\CAT\kappa$.
\end{clm}

Indeed, 
according to \ref{prop:k-thin},
it is sufficient to show that any triangle $\trig {x^0}{x^1}{x^2}$ of perimeter $<2\cdot \varpi\kappa$ 
in $\spc{W}$ is $\kappa$-thin.
This is obviously true if all three points $x^0$, $x^1$, $x^2$ lie in a single $\spc{U}^i$.
Thus, without loss of generality, we may assume that $x^0\in\spc{U}^1$ and $x^1,x^2\in\spc{U}^2$.
\begin{figure}[!ht]
\vskip-0mm
\centering
\includegraphics{mppics/pic-940}
\end{figure}

Choose points $z^1,z^2\in A=\spc{U}^1\cap\spc{U}^2$ 
lying respectively on the sides $[x^0x^1], [x^0x^2]$.
Note that all distances between any pair of points from $x^0$, $x^1$, $x^2$, $z^1$, $z^2$ are less than $\varpi\kappa$.
Therefore
\begin{itemize}
\item triangle $\trig{x^0}{z^1}{z^2}$ lies in $\spc{U}^1$,
\item both triangles $\trig{x^1}{z^1}{z^2}$ and $\trig{x^1}{z^2}{x^2}$ lie in $\spc{U}^2$.
\end{itemize}
In particular each triangle $\trig{x^0}{z^1}{z^2}$,
$\trig{x^1}{z^1}{z^2}$, $\trig{x^1}{z^2}{x^2}$ is $\kappa$-thin.

Applying the inheritance lemma for thin triangles (\ref{lem:inherit-angle}) twice, 
we get that $\trig {x^0}{x^1}{z^2}$ 
and consequently $\trig {x^0}{x^1}{x^2}$ is $\kappa$-thin.
\claimqeds

\begin{clm}{}\label{clm:geod-gluing0 }
$\spc{W}$ is $\CAT\kappa$ if $\kappa\le0$.
\end{clm}
By \ref{clm:geod-gluing} it suffices to prove that $\spc{W}$ is geodesic.

For $p^1\in \spc{U}^1$, $p^2\in \spc{U}^2$, we may choose a sequence $z_n\in A$ such that $\dist{p^1}{z_n}{}+\dist{p^2}{z_n}{}$
 converges to $\dist{p^1}{p^2}{}$, and $\dist{p^1}{z_n}{}$ and $\dist{p^2}{z_n}{}$ converge.  
 Since $A$ is complete, it suffices to show $z_n$ is a Cauchy sequence.  
 In that case, the limit point $z$ of $z_n$ satisfies $\dist{p^1}{z}{}+\dist{p^2}{z}{}=\dist{p^1}{p^2}{}$, so the geodesics $[p^1z]$ in $\spc{U}^1$ and $[p^2z]$ in $\spc{U}^2$ together give a geodesic $[p^1p^2]$ in $\spc{U}$.  
 
 Suppose $z_n$ is not a Cauchy sequence.
 Then there are subsequences  $x_n$ and $y_n$ of $z_n$ satisfying  $\lim\dist{x_n}{y_n}{}>0$.
 Let $m_n$ be the midpoint of $[x_ny_n]$.
 Since $\dist{p^1}{m_n}{}+\dist{p^2}{m_n}{} \ge \dist{p^1}{p^2}{}$, and  $\dist{p^1}{x_n}{}+\dist{p^2}{x_n}{}$ and  $\dist{p^1}{y_n}{}+\dist{p^2}{y_n}{}$
 converge to $\dist{p^1}{p^2}{}$, then for  any $\epsilon >0$, we may assume (taking subsequences and possibly relabeling $p^1$ and $p^2$)
 \[
 \dist{p^1}{m_n}{}
 \ge
 \dist{p^1}{x_n}{}-\epsilon,
 \qquad
 \dist{p^1}{m_n}{}
 \ge
 \dist{p^1}{y_n}{}-\epsilon.
 \]
 
 Since triangle $\trig{p^1}{x_n}{y_n}$ is thin, the analogous inequalities hold for the Euclidean model triangle  $\trig{\tilde p^1}{\tilde x_n}{\tilde y_n}$.  
 Then there is a nondegenerate limit triangle $\trig{p}{x}{y}$ in the Euclidean plane satisfying $\dist{p}{x}{}=\dist{p}{y}{}\le\dist{p}{m}{}$ where $m$ is the midpoint of $[xy]$.  This  contradiction proves the claim.
\claimqeds

Finally suppose $\kappa>0$; by rescaling, take $\kappa=1$. Consider the Euclidean cones $\Cone\spc{U}^i$ (see Section \ref{sec: tangent space}).
By Theorem \ref{thm:warp-curv-bound:cbb:S}, $\Cone\spc{U}^i$ is a $\CAT0$ space for $i=1,2$.

Geodesics contained in the complement of the tip of $\Cone\spc{U}^i$ project to geodesics of length $<\pi$ in $\spc{U}^i$. 
It follows that $\Cone A$ is convex in $\Cone\spc{U}^1$ and $\Cone\spc{U}^2$.
By the cone distance formula, $\Cone A$ is complete since $A$ is complete.
  
Gluing along $\Cone A$ and applying \ref{clm:geod-gluing} and \ref{clm:geod-gluing0 } for $\kappa=0$, we find that 
$\Cone\spc{W}$ is a $\CAT0$ space.  By Theorem \ref{thm:warp-curv-bound:cbb:S}, $\spc{W}$ is a $\CAT1$ space.
\qeds

\begin{thm}{Exercise}\label{ex:two-rays}
Let $Q$ be the nonconvex subset of the plane bounded by two half-lines $\gamma_1$ and $\gamma_2$ with a common starting point and angle $\alpha$ between them.
Assume $\spc{U}$ is a complete length $\CAT0$ space and $\gamma_1',\gamma_2'$ are two half-lines in $\spc{U}$ with a common
starting point and angle $\alpha$ between them.
Show that the space glued from $Q$ and $\spc{U}$ along the corresponding half-lines is a $\CAT{0}$ space.
\end{thm}

\begin{thm}{Exercise}\label{ex:reshetnyak-doubling}
Suppose $\spc{U}$ is a complete length $\CAT0$ space and $A\subset \spc{U}$ is a closed subset.
Assume that the doubling of $\spc{U}$ in $A$ is $\CAT0$. 
Show that $A$ is a convex set of $\spc{U}$.
\end{thm}

\begin{thm}{Exercise}\label{ex:glue-spherical-suspension}
Let  $\spc{U}$ be a complete length $\CAT1$ space and $K\subset \spc{U}$ be a closed $\pi$-convex set.
Assume $K\subset \cBall[p,\tfrac\pi2]$ for some point $p\in K$.
Show that there is a decreasing continuous one-parameter family of closed convex sets $K_t$ for $t\in[0,1]$ such that $K_0=\cBall[p,\tfrac\pi2]$ and $K_1=K$.

(Decreasing means with respect to inclusion; that is $K_{t_0}\supset K_{t_1}$ if $t_0\le t_1$.
Continuous means with respect to Hausdorff distance; that is $K_t\Hto K_{t_0}$ as $t\to t_0$.)
\end{thm}


\begin{thm}{Exercise}\label{ex:AUB}
Let $A$ and $B$ be two closed convex sets in a complete length $\CAT0$ space.
Assume $A\cap B\ne\emptyset$.
Show that the union $A\cup B$ equipped with induced length metric is $\CAT0$. 
\end{thm}

%%%%%%%%%%%%%%%%%%%%%%%%%%%%%%%%%%%%%%%%%%%%%%%%%%%%%%%%%%%%%%%%%%%%%%

\section{Space of geodesics}\label{sec:geod-space}

In this section we prove a no-conjugate-point theorem for spaces with upper curvature bounds and derive from it a number of statements about local geodesics.
These statements will be used to prove the Hadamard--Cartan theorem (\ref{thm:hadamard-cartan}) and the lifting globalization theorem (\ref{thm:globalization-lift}), in much the same way as  the exponential map is used in Riemannian geometry.

\begin{thm}{Proposition}\label{prop:geo-complete}
{\sloppy 
Let $\spc{U}$ be a locally $\CAT\kappa$ space.
 Let $\gamma_n\:[0,1]\to\spc{U}$ be a sequence of local geodesic paths converging to a path $\gamma_\infty\:[0,1]\to\spc{U}$.
Then $\gamma_\infty$ is a local geodesic path.
Moreover 
\[\length\gamma_n\to\length\gamma_\infty\]
as $n\to\infty$.

}
\end{thm}

\parit{Proof.} 
Fix $t\in[0,1]$.
By Corollary~\ref{cor:k-for-k}, we may choose $R$ satisfying $0<R<\varpi\Kappa$,
and such that
the ball $\spc{B}=\oBall(\gamma_\infty(t),R)$ is a convex subset of $\spc{U}$ and forms a $\CAT\kappa$ space.

A local geodesic segment  with length less than $R/2$ that intersects $\oBall(\gamma_\infty(t),R/2)$ cannot leave $\spc{B}$, and hence  is  minimizing by Corollary~\ref{cor:loc-geod-are-min}.
In particular, for all sufficiently large $n$, 
if subsegment of $\gamma_n$ has length less than $R/2$ and contains $\gamma_n(t)$, then it is a geodesic.


Since $\spc{B}$ is $\CAT\kappa$, geodesic segments in $\spc{B}$ depend uniquely and continuously on their endpoint pairs by Theorem~\ref{thm:cat-unique}.  
Thus there is a subinterval $\II$ of $[0,1]$
that contains a neighborhood of $t$ in $[0,1]$
and such that $\gamma_n|_\II$ is minimizing for all large $n$.
It follows that the restriction $\gamma_\infty|_\II$ is a geodesic,
and therefore $\gamma_\infty$ is a local geodesic.
\qeds


The following theorem was proved by the first author and Richard Bishop \cite{alexander-bishop:h-c}.
In analogy with Riemannian geometry, the main statement of the following theorem could be restated as: 
\textit{In a space of curvature $\le\kappa$, two points cannot be conjugate along a local geodesic of length $<\varpi\kappa$.}


\begin{thm}{No-conjugate-point theorem}
\label{thm:no-conj-pt}{\sloppy 
Suppose $\spc{U}$ is a locally complete, length, locally $\CAT\kappa$ space.
Let $\gamma\:[0,1]\to\spc{U}$ be a local geodesic path with length $<\varpi\kappa$.
Then for some neighborhoods $\Omega^0\ni \gamma(0)$ and $\Omega^1\ni\gamma(1)$, 
there is a unique continuous map from the direct product $\Omega^0\times \Omega^1\times[0,1]$ to $\spc{U}$, 

\[(x,y,t)\mapsto\gamma_{x y}(t),\]  
such that 
$\gamma_{x y}\:[0,1]\to\spc{U}$ is a local geodesic path with 
$\gamma_{x y}(0)=x$ and 
$\gamma_{x y}(1)=y$ for each $(x,y)\in\Omega^0\times\Omega^1$,
and the family $\gamma_{x y}$ contains $\gamma$.
Moreover, we can assume that the map 
\[(x,y,t)\mapsto\gamma_{x y}(t)\:\Omega^0\times\Omega^1\times[0,1]\to\spc{U}\] 
is $\Lip$-Lipschitz
for any
$\Lip>\max\set{\tfrac{\sn\kappa r}{\sn\kappa \ell}}{0\le r\le \ell}$.

}
\end{thm}

{\sloppy

\begin{thm}{Patchwork along a geodesic}
\label{lem:patch}
Let $\spc{U}$ be a locally complete, length, locally $\CAT\kappa$ space, 
and $\alpha\:[a,b]\to\spc{U}$ be a local geodesic.

Then there is a complete length $\CAT\kappa$ space $\spc{N}$
with an open set $\hat\Omega\subset \spc{N}$,
a local geodesic $\hat\alpha\:[a,b]\to\hat\Omega$,
and an open locally distance-preserving map 
$\map\:\hat\Omega\looparrowright\spc{U}$ such that
$\map\circ\hat\alpha=\alpha$.

Moreover if $\alpha$ is simple, then one can assume in addition that $\map$ is an open embedding;
thus $\hat\Omega$ is locally isometric to a neighborhood of $\Omega=\map(\hat\Omega)$ of $\alpha$.
\end{thm}

}

This lemma and its proof were suggested by Alexander Lytchak.
The proof proceeds by piecing together $\CAT{\kappa}$  neighborhoods of points on a curve to construct a new $\CAT{\kappa}$ space.  
Exercise~\ref{ex:cats-cradle} is inspired by the original idea of the proof of the no-conjugate-point theorem (\ref{thm:no-conj-pt}) given in \cite{alexander-bishop:h-c}.

\parit{Proof.} 
According to Corollary~\ref{cor:k-for-k},
we can choose $r>0$ such that 
for any $t\in[a,b]$ the closed ball
$\cBall[\alpha(t),r]$ is a convex set that forms a complete length $\CAT\kappa$ space.

Choose balls $\spc{B}_i=\cBall[\alpha(t_i),r]$
for some partition $a\z=t_0<t_1\z<\z\dots\z<t_n\z=b$
in such a way that 
\[\Int\spc{B}_i\supset \alpha([t_{i-1},t_i])\]
for all $i>0$.
We can assume in addition that $\spc{B}_{i-1}\cap \spc{B}_{i+1}\subset \spc{B}_{i}$ if $0<i<n$.

Consider the disjoint union $\bigsqcup_i\spc{B}_i=\set{(i,x)}{x\in\spc{B}_i}$ with the minimal equivalence relation $\sim$ such that $(i,x)\sim(i-1,x)$ for all $i>0$.
Let $\spc{N}$ be the space obtained by gluing the $\spc{B}_i$ along $\sim$.
Note that $A_i=\spc{B}_i\cap\spc{B}_{i-1}$ is convex in $\spc{B}_i$ and in $\spc{B}_{i-1}$.
Applying the Reshetnyak gluing theorem (\ref{thm:gluing}) several times, 
we conclude that $\spc{N}$ is a complete length $\CAT\kappa$ space.

\begin{figure}[!ht]
\vskip-0mm
\centering
\includegraphics{mppics/pic-945}
\end{figure}

For $t\in[t_{i-1},t_i]$, let $\hat\alpha(t)$  be the equivalence class of $(i,\alpha(t))$ in $\spc{N}$.
Let $\hat\Omega$ be the $\eps$-neighborhood of $\hat\alpha$ in $\spc{N}$, where $\eps>0$ is chosen so that $\oBall(\alpha(t),\eps)\subset\spc{B}_i$ for all $t\in[t_{i-1},t_i]$.

Define $\map\:\hat\Omega\to\spc{U}$
by sending the equivalence class of $(i,x)$ to $x$.
It is straightforward to check that $\map\:\spc{N}\to\spc{U}$, $\hat\alpha\:[a,b]\to\spc{N}$ and $\hat\Omega\subset\spc{N}$ satisfy the conclusion of the main part of the lemma.

To prove the final statement in the lemma,
we only have to choose $\eps>0$ so that in addition, $\dist{\alpha(\tau)}{\alpha(\tau')}{}>2\cdot\eps$ if $\tau\le t_{i-1}$ and $t_i\le\tau'$ for some $i$.
\qeds


\parit{Proof of \ref{thm:no-conj-pt}.}
Apply patchwork along $\gamma$ (\ref{lem:patch}). 
\qeds



The No-conjugate-point theorem (\ref{thm:no-conj-pt}) allows us to move a local geodesic  
so that its endpoints follow given trajectories.
The following corollary describes how this process might terminate. 

\begin{thm}{Corollary}\label{cor:geo-hom}{\sloppy 
Let $\spc{U}$ be a locally complete, length, locally $\CAT\kappa$ space.
Suppose $\gamma\:[0,1]\to\spc{U}$ is a local geodesic with length $< \varpi\kappa$.  Let $\alpha^i\:[0,1]\to \spc{U}$, for $i=0,1$, be curves starting at $\gamma(0)$ and $\gamma(1)$ respectively.  

}

Then there is a uniquely determined pair consisting of an interval $\II $ satisfying $0\in \II\subset[0,1]$, and a continuous family of local geodesics $\gamma_t\:[0,1]\to \spc{U}$ for  $t\in \II$, such that  

\begin{subthm}{cor:geo-hom-length}
$\gamma_0=\gamma$, $\gamma_t(0)=\alpha^0(t)$, $\gamma_t(1)=\alpha^1(t)$, and $\gamma_t$ has length $< \varpi\kappa$,
\end{subthm} 

\begin{subthm}{cor:geo-hom-cauchy}
if $\II\ne [0,1]$, then $\II=[0, a)$, where either $\gamma_t$ converges uniformly to a local geodesic $\gamma_a$ of length $\varpi\kappa$, or 
for some fixed $s\in [0,1]$ the curve $\gamma_t(s):[0,a)\to\spc{U}$ is a Lipschitz curve with no limit 
as $t\to a-$.
%??DO WE REALLY NEED SUCH GENERALITY
\end{subthm}

\end{thm}


\parit{Proof.} Uniqueness follows from  Theorem \ref{thm:no-conj-pt}.

Let $\II$ be the maximal interval for which there is a family $\gamma_t$ satisfying condition \ref{SHORT.cor:geo-hom-length}. 
By Theorem~\ref{thm:no-conj-pt}, such an interval exists and is open in $[0,1]$.  Suppose $\II\ne[0,1]$.
Then  $\II=[0,a)$ for some $0<a\le 1$.

For each fixed $s\in [0,1]$, define the curve $\alpha_s:[0,a)\to\spc{U}$ by $\alpha_s(t)\z=\gamma_t(s)$. 
By Theorem~\ref{thm:no-conj-pt}, 
each $\alpha_s$ is locally Lipschitz.  

If $\alpha_s$ for some value of $s$ does not converge as $t\to a-$, then condition \ref{SHORT.cor:geo-hom-cauchy} holds.
If each $\alpha_s$  converges as $t\to a-$, then $ \gamma_t$ converges as $t\to a-$, say to $\gamma_a$.
By  Proposition~\ref{prop:geo-complete}, $\gamma_a$ is a local geodesic and\[\length\gamma_t\to\length\gamma_a\le \varpi\kappa.\]
By maximality of $\II$, $\length\gamma_a=\varpi\kappa$ and so condition \ref{SHORT.cor:geo-hom-cauchy} again holds.
\qeds

\begin{thm}{Corollary}\label{cor:homotopy-from-p}
Let $\spc{U}$ be a complete locally $\CAT\kappa$ length space, and 
$\alpha\:[0,1]\to \spc{U}$ be a path of length $< \varpi\kappa$ that starts at $p$ and ends at $q$.
Then:  

\begin{subthm}{cor:homotopy-from-p-exist}
There is a unique homotopy of local geodesic paths $\gamma_t\:[0,1]\to \spc{U}$
such that $\gamma_0(t)=\gamma_t(0)=p$ and $\gamma_t(1)=\alpha(t)$ for any~$t$.
\end{subthm}

\begin{subthm}{cor:homotopy-from-p-length}
For any $t\in[0,1]$, 
\[\length\gamma_t\le\length(\alpha|_{[0,t]}),\]
and equality holds for given $t$ if and only if the restriction $\alpha|_{[0,t]}$ is a reparametrization of $\gamma_t$.
\end{subthm}

Moreover, instead of completeness of $\spc{U}$, one can assume that the subspace 
\[W=\set{x\in \spc{U}}{\dist{x}{p}{}+\dist{x}{q}{}\le \ell}\] 
is complete.

\end{thm}

\parit{Proof.}
By Corollary \ref{cor:geo-hom}, taking  $\alpha^0(t)=p$ and  $\alpha^1(t)=\alpha(t)$ for all $t\in [0,1]$, there is an interval $\II$ such that \ref{SHORT.cor:homotopy-from-p-exist} holds for all $t\in\II$, and either $\II=[0,1]$ or $\II=[0,a)$ for some $a\le 1$.

By patchwork along a curve (\ref{lem:patch}), the values of $t$ for which condition \ref{SHORT.cor:homotopy-from-p-length} holds form an open subset of $\II$ containing $0$; clearly this subset is also closed in $\II$.
Therefore \ref{SHORT.cor:homotopy-from-p-length} holds on all of $\II$. 
 
Corollary \ref{cor:geo-hom} implies that
$\II=[0,1]$.
Indeed if $\II=[0,a)$, then either $\length\gamma_t\to\varpi\kappa$ as $t\to a-$,
or for some fixed $s\in [0,1]$ the Lipschitz curve $\gamma_t(s):[0,a)\to\spc{U}$ has no limit as $t\to a-$.
Since $\length\alpha<\varpi\kappa$, \ref{cor:geo-hom} implies that neither of these is possible.
\qeds



%%%%%%%%%%%%%%%%%%%%%%%%%%%%%%%%%%%%%%%%%%%%%%%%%%%%%%%%%%%%%%%%%%%%%%%%%%%%%

\section{Lifting globalization}\label{sec:cat-globalize}

The Hadamard--Cartan theorem (\ref{thm:hadamard-cartan}) states that 
the universal metric cover of a complete locally $\CAT0$ space is $\CAT0$.
The lifting globalization theorem gives an appropriate generalization of the above statement to arbitrary curvature bounds;
it could be also described as a global version of Gauss's lemma.



\begin{thm}{Lifting globalization theorem}
\label{thm:globalization-lift}
Suppose $\spc{U}$ is a complete length locally $\CAT\kappa$ space and  $p\in\spc{U}$.
Then there is a complete $\CAT\kappa$ length space $\spc{B}$, 
with a point $\hat p$ such that 
there is a locally distance-preserving map $\map\:\spc{B}\to\spc{U}$
such that $\map(\hat p)=p$ and the following lifting property holds: 
for any path $\alpha\:[0,1]\to\spc{U}$ with $\alpha(0)=p$ and $\length\alpha<\varpi\kappa/2$, 
there is a unique path $\hat\alpha \:[0,1]\to \spc{B}$ such that $\hat\alpha(0)=\hat p$ 
and $\map\circ\hat\alpha\equiv\alpha$.
\end{thm}


Note that the lifting property implies that $\map(\spc{B})\supset\oBall(p,\varpi\kappa/2)$ and by completeness $\map(\spc{B})\supset\cBall[p,\varpi\kappa/2]$.
Also since $\spc{B}$ is $\CAT\kappa$, the closed ball $\cBall[\hat p,\tfrac{\varpi\kappa}2]_{\spc{B}}$ is a weakly convex set in $\spc{B}$ (see \ref{cor:convex-balls});
in particular $\cBall[\hat p,\tfrac{\varpi\kappa}2]_{\spc{B}}$ is a complete length $\CAT\kappa$ space.
Therefore we can assume in addition that $\dist{\hat p}{\hat x}{}\le \varpi\kappa/2$ for any $\hat x\in\spc{B}$;
or equivalently
\[\cBall[\hat p,\tfrac{\varpi\kappa}2]_{\spc{B}}=\spc{B}.\]


Before proving the theorem we state and prove its corollary.

\begin{thm}{Corollary}\label{cor:loc-CAT(k)}
Suppose $\spc{U}$ is a complete length locally $\CAT\kappa$ space.
Then for any $p\in\spc{U}$ there is $\rho_p>0$
such that $\cBall[p,\rho_p]$ is a complete length $\CAT\kappa$ space.

Moreover, we can assume that $\rho_p<\tfrac{\varpi\kappa}2$
for any $p$ and the function $p\mapsto\rho_p$ is 1-Lipschitz.
\end{thm}

\parit{Proof.} 
Assume $\map\:\spc{B}\to \spc{U}$ 
and $\hat p\in \spc{B}$
are provided by the lifting globalization theorem
(\ref{thm:globalization-lift}).

Since $\map$ is local isometry,
we can choose $r>0$ so that the restriction of $\map$ to $\cBall[\hat p,r]$ is distance-preserving.
By the lifting globalization, the image  $\Phi(\cBall[\hat p,r])$ coincides with the ball
$\cBall[p,r]$.
This proves the first part of the theorem.

To prove the second part, let us choose $\rho_p$ to be the maximal value $\le\tfrac{\varpi\kappa}2$ such that $\cBall[p,\rho_p]$ is a complete length $\CAT\kappa$ space.
By Corollary~\ref{cor:convex-balls}, the ball
\[\cBall[q,\rho_p-\dist{p}{q}{}]\] 
is weakly convex in $\cBall[p,\rho_p]$.
Therefore  
\[\cBall[q,\rho_p-\dist{p}{q}{}]\] is a complete length $\CAT\kappa$ space
for any $q\in \oBall(p,\rho_p)$.
In particular, $\rho_q\ge \rho_p-\dist{p}{q}{}$ for any $p,q\in\spc{U}$.
Hence the second statement follows.
\qeds




The proof of the lifting globalization theorem relies heavily on the properties of the space of local geodesic paths discussed in Section~\ref{sec:geod-space}.
The following lemma  is a key step in the proof;
it was proved by the first author and Richard Bishop \cite{alexander-bishop:cbc}. 

\begin{thm}{Radial lemma}\label{lem:radial-glob}
Let $\spc{U}$ be a length locally $\CAT\kappa$ space,
and suppose $p\in\spc{U}$, $R\le\varpi\kappa$.
Assume the ball  $\cBall[p,\bar{R}]$ is complete for any $\bar{R}<R$, and  there is a unique geodesic path, $\geodpath_{[p x]}$, from $p$ to any point $x\in\oBall(p,R)$ 
that depends continuously on $x$.
Then $\oBall(p,\tfrac R2)$ is a $\varpi\kappa$-geodesic $\CAT\kappa$ space.
\end{thm}
 
\parit{Proof.}
Without loss of generality, we may assume  $\spc{U}=\oBall(p,R)$.

Set $f=\md\kappa\circ\distfun{p}{}{}$.  Let us show that
\[f''+\kappa\cdot f\ge 1.
\eqlbl{eq:rad-conv}\]



Fix $z\in \spc{U}$.
We will apply the no-conjugate-point theorem (\ref{thm:no-conj-pt}) for the unique geodesic path $\gamma$
from $p$ to $z$.  
The  notations $\Omega^0$, 
$\Omega^1$,
$\gamma_{x y}$, $\spc{N}$, $\hat{x}$, $\hat{y}$ will be as in the formulation of the lifting globalization theorem (\ref{thm:globalization-lift});
in particular, $z\in\Omega^1$.

By assumption,
$\gamma_{p y}=\geodpath_{[p y]}$ for any $y\in\Omega^1$. 
Consequently,
 $f(y)\z=\md\kappa\dist[{{}}]{\hat{p}}{\hat{y}}{\spc{N}}$.
Applying the function comparison (\ref{thm:function-comp}) in $\spc{N}$,
we have that $f''+\kappa\cdot f\ge 1$ in $\Omega^1$;
whence \ref{eq:rad-conv} follows.
\claimqeds

Fix $r<\tfrac R2$. Proving the following claim takes most of the remaining proof:

\begin{clm}{}\label{clm:B-is-convex}
$\cBall[ p,r]$ is a convex set in $\spc{U}$.
\end{clm}

Choose arbitrary $x,z\in \cBall[ p,r]$.
First note that \ref{eq:rad-conv} implies the following claim.

\begin{clm}{}\label{clm:B-is-almost-convex}
If $ \gamma\:[0,1]\to\spc{U}$ 
is a local geodesic path from $x$ to $z$ and  
$\length \gamma\z<\varpi\kappa$,  
then $\length \gamma \le 2\cdot r$ 
and $ \gamma$ lies completely in $\cBall[ p,r]$.
\end{clm}

Note that  $\dist{x}{z}{}<\varpi\kappa$.
Thus to prove Claim~\ref{clm:B-is-convex}, it is sufficient to show that there is a geodesic path from $x$ to $z$.
Note that by assumption $\cBall[p,2\cdot r]$ is complete.
Therefore Corollary~\ref{cor:homotopy-from-p} implies the following:

\begin{clm}{}\label{clm:loc-geod<path}
Given a path $\alpha\:[0,1]\to\spc{U}$ from $x$ to $z$ with $\length\alpha<2\cdot r$,
there is a local geodesic path $\gamma$ from $x$ to $z$ such that
\[\length\gamma\le\length\alpha.\]

\end{clm}

Further, let us prove the following:

\begin{clm}{}\label{clm:unique-loc-geod}
There is a unique local geodesic path $\gamma_{x z}$ in $\cBall[ p,r]$ from $x$ to $z$.
\end{clm}

Denote by $\Delta_{x z}$ the set of all local geodesic paths in $\cBall[ p,r]$ from $x$ to $z$.
By Corollary \ref{cor:geo-hom}, there is a  bijection $\Delta_{x z}\to\Delta_{p p}$.
According to \ref{eq:rad-conv}, 
$\Delta_{p p}$ contains only the constant path.
Claim~\ref{clm:unique-loc-geod} follows.


Note that 
claims~\ref{clm:B-is-almost-convex}, 
\ref{clm:loc-geod<path} 
and \ref{clm:unique-loc-geod}
imply that $\gamma_{x z}$ is minimizing; hence Claim~\ref{clm:B-is-convex}.

Further, Claim~\ref{clm:B-is-almost-convex} and the no-conjugate-point theorem (\ref{thm:no-conj-pt}) together 
imply that the map $(x,z)\mapsto\gamma_{x z}$ is continuous.

Therefore by the patchwork globalization theorem (\ref{thm:alex-patch}), 
$\cBall[ p,r]$ is a $\varpi\kappa$-geodesic $\CAT\kappa$ space.

Since
\[\oBall( p,R)
=
\bigcup_{r < R}\cBall[ p,r],\] 
then $\oBall( p,R)$ is convex in $\spc{U}$ and 
$\CAT\kappa$ comparison holds  for any quadruple in $\oBall( p,R)$.
Therefore $\oBall( p,\varpi\kappa/2)$ is $\CAT\kappa$.
\qeds


In the following proof, we construct a space $\mathfrak{G}_p$ of  local geodesic paths that start at $p$.
The space $\mathfrak{G}_p$ comes with 
a marked point $\hat p$ 
and the endpoint map $\map\:\mathfrak{G}_p\to\spc{U}$.
One can think of
the map $\map$ as an analog of the exponential map $\exp_p$ in the Riemannian geometry;
in this case,
the space $\mathfrak{G}_p$ corresponds to the ball of radius $\varpi\kappa$ in the tangent space at $p$, equipped with the metric pulled back by $\exp_p$.

We are going to set $\spc{B}=\oBall(\hat p,\varpi\kappa/2)\subset \mathfrak{G}_p$,
and use the radial lemma (\ref{lem:radial-glob}) to prove that $\spc{B}$ is a $\varpi\kappa$-geodesic $\CAT\kappa$ space.

\parit{Proof of \ref{thm:globalization-lift}.}
Suppose $\hat\gamma$ is a homotopy of local geodesic paths that start at $p$.  Thus the map 
\[\hat\gamma\:(t,\tau)\mapsto\hat\gamma_t(\tau)\:[0,1]\times[0,1]\to\spc{U}\] 
is continuous,
and the following holds for each $t$:
\begin{itemize}
\item $\hat\gamma_t(0)=p$,
\item $\hat\gamma_t\:[0,1]\to\spc{U}$ is a local geodesic path in $\spc{U}$.
\end{itemize}

Denote by $\theta(\hat\gamma)$ the length traced by the ends of $\hat\gamma_t$;
that is, $\theta(\hat\gamma)$ is the length of the path $t\mapsto\hat\gamma_t(1)$.

Let $\mathfrak{G}_p$ be the set of all local geodesic paths 
with length $<\varpi\kappa$ in $\spc{U}$ that start at $p$.
Denote by $\hat p\in \mathfrak{G}_p$ the constant path $\hat p(t)\equiv p$.
Given $\alpha,\beta\in \mathfrak{G}_p$, define
\[
\dist{\alpha}{\beta}{\mathfrak{G}_p}
=
\inf_{\hat\gamma} \{\theta(\hat\gamma)\},\]
with the exact lower bound taken along all homotopies 
$\hat\gamma\:[0,1]\times[0,1]\to\spc{U}$ 
such that 
$\hat\gamma_0=\alpha$, 
$\hat\gamma_1=\beta$ 
and $\hat\gamma_t\in \mathfrak{G}_p$ for all $t\in[0,1]$.

By the no-conjugate-point theorem (\ref{thm:no-conj-pt}), we have $\dist{\alpha}{\beta}{\mathfrak{G}_p}>0$ for distinct $\alpha$ and $\beta$;
that is,

\begin{clm}{}
$\dist{{*}}{{*}}{\mathfrak{G}_p}$ is a metric on $\mathfrak{G}_p$.
\end{clm}

Further, again from the no-conjugate-point theorem (\ref{thm:no-conj-pt}), we have

\begin{clm}{}\label{clm:loc-iso}
The map
\[\map\:\xi\mapsto\xi(1)\:\mathfrak{G}_p\to\spc{U}\]
is a local isometry.
In particular, $\mathfrak{G}_p$ is locally $\CAT\kappa$.
\end{clm}

Let $\alpha\:[0,1]\to\spc{U}$ be a path, $\length\alpha<\varpi\kappa$, and $\alpha(0)=p$.
The homotopy constructed in Corollary~\ref{cor:homotopy-from-p} can be regarded as a path in $\mathfrak{G}_p$, say $\hat\alpha\:[0,1]\to \mathfrak{G}_p$,
such that $\hat\alpha(0)=\hat p$ and $\map\circ\hat\alpha=\alpha$;
in particular $\hat\alpha_t(1)\equiv\alpha(t)$ for any $t$. 
By \ref{clm:loc-iso}, 
\[\length(\hat\alpha)_{\mathfrak{G}_p}=\length(\alpha)_{\spc{U}}.\]
Moreover, it follows that $\alpha$ is a local geodesic path of $\spc{U}$  if and only if $\hat\alpha$ is a local geodesic path of $\mathfrak{G}_p$.

Further, from Corollary~\ref{cor:homotopy-from-p},
for any $\xi\in \mathfrak{G}_p$ and path $\hat\alpha\:[0,1]\to\mathfrak{G}_p$ from $\hat p$ to $\xi$,
we have 
\begin{align*}
\length\hat\alpha
&=\length\map\circ\hat\alpha
\ge
\\
&\ge
\length\xi
=
\\
&=\length\hat\xi
\end{align*}
where equality holds only if $\hat\alpha$ is a reparametrization of $\hat\xi$.
In particular, 
\[\dist{\hat p}{\xi}{\mathfrak{G}_p}=\length\xi
\eqlbl{eq:dist=length}\] 
and
$\hat\xi\:[0,1]\to \mathfrak{G}_p$ is the unique geodesic path from $\hat p$ to $\xi$.
Clearly, the map $\xi\mapsto\hat\xi$ is continuous.

By \ref{eq:dist=length} and Proposition~\ref{prop:geo-complete}, 


\begin{clm}{}\label{clm:complete-B} 
For any $\bar R<\varpi\kappa$, the closed ball
$\cBall[\hat p,\bar R]$ in $\mathfrak{G}_p$ is complete.
\end{clm}

Take $\oBall(\hat p,\varpi\kappa/2)$ and $\map$ constructed above.
According to the radial lemma (\ref{lem:radial-glob}), $\oBall(\hat p,\varpi\kappa/2)$ is a $\varpi\kappa$-geodesic $\CAT\kappa$ space.
The map $\map$ extends to its completion $\spc{B}=\cBall[\hat p,\varpi\kappa/2]$. 
All the remaining statements are already proved.
\qeds

\section{Reshetnyak majorization}\label{sec:resh-kirz}

\begin{thm}{Definition}\label{def:majorize}
Let $\spc{X}$ be a metric space,
$\tilde \alpha$ be a simple closed curve of finite length  in $\Lob2{\kappa}$,
and $D\subset\Lob2{\kappa}$ be a closed region bounded by $\tilde \alpha$.
A length-nonincreasing map $F\:D\to\spc{X}$ is called \index{majorizing map}\emph{majorizing} if it is length-preserving on $\tilde \alpha$.

In this case, we say that $D$ \emph{majorizes} the curve $\alpha=F\circ\tilde \alpha$ under the map $F$.
\end{thm}

The following proposition is a consequence of the definition.

\begin{thm}{Proposition}
\label{prop:majorize-geodesic} 
Let  $\alpha$  be a closed curve in a metric space $\spc{X}$.
Suppose $D\subset\Lob2{\kappa}$ majorizes $\alpha$ under $F\: D \to \spc{X}$.  
Then any geodesic subarc of $\alpha$ is the image under $F$ of a subarc of $\partial_{\Lob2{\kappa}} D$ that is geodesic in the length metric of $D$.

In particular, if $D$ is convex, then the corresponding subarc is a geodesic in $\Lob2{\kappa}$.
\end{thm}

\parit{Proof.} For a geodesic subarc $\gamma\:[a,b]\to\spc{X}$ of $\alpha=F\circ\tilde \alpha$, set
\begin{align*}
\tilde r&=\dist{\tilde \gamma(a)}{\tilde \gamma(b)}{D},
&
\tilde \gamma &= (F|_{\Fr D})^{-1}\circ\gamma,
\\
s&=\length \gamma,
&
\tilde s&= \length \tilde \gamma.
\end{align*}
Then
\[\tilde r\ge r = s =\tilde s\ge\tilde r.\]
Therefore $\tilde s=\tilde r$.
\qeds

\begin{thm}{Corollary}\label{cor:maj-triangle}
Let $\trig p x y$ be a triangle of perimeter $<2\cdot\varpi\kappa$ in a metric space $\spc{X}$. Assume a convex region $D\subset \Lob2\kappa$ majorizes $\trig p x y$.
Then $D=\Conv\trig{\tilde p}{\tilde x}{\tilde y}$ for a model triangle $\trig{\tilde p}{\tilde x}{\tilde y}=\modtrig\kappa(p x y)$, and the majorizing map sends  $\tilde p$, $\tilde x$ and $\tilde y$ respectively to $p$, $x$ and $y$.
\end{thm}

Now we come to the main theorem of this section.

\begin{thm}{Majorization theorem}
\label{thm:major}
Any closed curve $\alpha$ with length smaller than $2\cdot \varpi\kappa$ in  a $\varpi\kappa$-geodesic $\CAT\kappa$ space is majorized by a convex region in $\Lob2\kappa$. \end{thm}

This theorem was proved by Yuriy Reshetnyak \cite{reshetnyak:major};
our proof uses a trick that we learned from the lectures of Werner Ballmann \cite{ballmann:lectures}.
Another proof can be built on Kirszbraun's theorem (\ref{thm:kirsz+}), but it works only for complete spaces.

The case when $\alpha$ is a triangle, say $\trig p x y$, is the base  and is nontrivial.
In this case, by Corollary~\ref{cor:maj-triangle}, the majorizing convex region has to be isometric to $\Conv\trig{\tilde p}{\tilde x}{\tilde y}$, where $\trig{\tilde p}{\tilde x}{\tilde y}=\modtrig\kappa(p x y)$.  
There is a majorizing map for $\trig p x y$ whose image $W$ is the image of the line-of-sight map (definition \ref{def:sight}) for $[x y]$ from  $p$,
but as one can see from the following example, the line-of-sight map is not majorizing in general.

\begin{wrapfigure}{r}{30 mm}
\vskip-0mm
\centering
\includegraphics{mppics/pic-950}
\end{wrapfigure}

\parbf{Example.} Let $\spc{Q}$ be a solid quadrangle $[p x z y]$ in $\EE^2$ formed by two congruent triangles, which is non-convex at $z$ (as in the picture).  
Equip $\spc{Q}$ with the length metric. 
Then $\spc{Q}$ is $\CAT0$
by Reshetnyak gluing  (\ref{thm:gluing}). 
For triangle ${\trig p x y}_\spc{Q}$ in $\spc{Q}$ and its model triangle $\trig{\tilde p}{\tilde x}{\tilde y}$ in $\EE^2$,  
we have 
\[\dist{\tilde x}{\tilde y}{}=\dist{x}{y}{\spc{Q}}=\dist{x}{z}{}+\dist{z}{y}{}.\]
Then the map $F$ defined by matching line-of-sight parameters satisfies $F(\tilde x)=x$ and $\dist{x}{F(\tilde w)}{}>\dist{\tilde x}{\tilde w}{}$ if $\tilde w$ is near the midpoint $\tilde z$ of $[\tilde x\tilde y]$ and lies on $[\tilde p\tilde z]$. 
Indeed, by the first variation formula (\ref{1st-var+}), for $\eps=1-s$ we have
\[\dist{\tilde x}{\tilde w}{}
=\dist{\tilde x}{\tilde \gamma_\frac12(s)}{}
=\dist{x}{z}{}+o(\eps)\] and 
\[\dist{x}{F(\tilde w)}{}
=\dist{x}{\gamma_\frac12(s)}{}
=\dist{x}{z}{}-\eps\cdot\cos\mangle\hinge z p x+o(\eps).\]  
Thus $F$ is not majorizing.

\medskip

In  the following proofs, $x^1 \dots x^n$ ($n\ge 3$) denotes a polygonal line $x^1,\dots,x^n$, and $[x^1\dots x^n ]$ denotes the corresponding (closed) polygon.
For a subset $R$ of the ambient metric space,
we denote by $[x^1\dots x^n ]_R$ a polygon in the length metric of $R$.

Our first lemma gives a model space construction based on repeated application of Lemma~\ref{lem:quadrangle} from the proof of the inheritance.
Recall that convex and concave curves with respect to a point are defined in~\ref{def:convex-devel}.

\begin{thm}{Lemma}\label{lem:majorize-subgraph}
In $\Lob2{\kappa}$, let  
$\beta$ be a curve from $x$ to $y$ 
that is concave with respect  to $p$.
Let $D$  be the subgraph of $\beta$ with respect to $p$.
Assume 
\[\length\beta\z+\dist{p}{x}{}+\dist{p}{y}{}<2\cdot\varpi\kappa.\]
\begin{subthm}{curvilinear} 
Then $\beta$ forms a geodesic $[x y]_D$ in $D$ and therefore $\beta$, $[p x]$ and $[p y]$ form a triangle 
${\trig p x y}_D$ in the length metric of $D$.
\end{subthm}
\begin{subthm} {short-to-subgraph}
Let $\trig{\tilde p}{\tilde x}{\tilde y}$ be the model triangle for 
${\trig p x y}_D$.
Then there is a short map $G\:\Conv\trig{\tilde p}{\tilde x}{\tilde y}\to D$ such that $\tilde p\mapsto p$, $\tilde x\mapsto x$, $\tilde y\mapsto y$, and $G$ is length-preserving on each side of $\trig{\tilde p}{\tilde x}{\tilde y}$.
In particular, $\Conv\trig{\tilde p}{\tilde x}{\tilde y}$ majorizes triangle $[p x y]_D$ in $D$ under~$G$.
\end{subthm}
\end{thm} 


\parit{Proof.}
We prove the lemma for a polygonal line $\beta$;
the general case then follows by approximation.
Namely, since $\beta$ is concave 
it can be approximated by polygonal lines that are concave with respect to $p$, 
with their lengths converging to $\length \beta$. 
Passing to a partial limit we will obtain the needed map $G$.  

Suppose $\beta=x^0x^1\dots x^n$ is a polygonal line with $x^0=x$ and $x^n=y$.
Consider a sequence of polygonal lines $\beta_i=x^0x^1\dots x^{i-1}y_i$ such that $\dist{p}{y_i}{}=\dist{p}{y}{}$ and 
$\beta_i$ has same length as $\beta$; 
that is, 
\[\dist{x^{i-1}}{y_i}{}=\dist{x^{i-1}}{x^{i}}{}+\dist{x^{i}}{x^{i+1}}{}+\dots+\dist{x^{n-1}}{x^n}{}.\]

\begin{figure}[!ht]
\vskip-0mm
\centering
\includegraphics{mppics/pic-955}
\end{figure}

Clearly $\beta_n=\beta$.
Sequentially applying Alexandrov's lemma (\ref{lem:alex}) shows that each of the polygonal lines $\beta_{n-1}, \beta_{n-2},\dots,\beta_1$ is concave with respect to $p$.
Let $D_i$ be the subgraph of $\beta_i$ with respect to $p$.
Applying Lemma~\ref{lem:quadrangle} gives a short map $G_i\:D_{i}\to D_{i+1}$ that maps $y_{i}\mapsto y_{i+1}$ and does not move $p$ and $x$ (in fact,  $G_i$ is the identity everywhere except on $\Conv\trig{p}{x^{i-1}}{y_i}$).
Thus the composition 
\[G_{n-1}\circ\dots\circ G_1\: D_1\to D_n\] 
is short.
The result follows since $D_1\iso\Conv\trig{\tilde p}{\tilde x}{\tilde y}$.\qeds

\begin{thm}{Lemma}\label{lem:majorize-triangle}
Let $\trig{p}{x}{y}$ be a triangle of perimeter $<2\cdot\varpi\kappa$ in a $\varpi\kappa$-geodesic $\CAT\kappa$ space $\spc{U}$.
In $\Lob2{\kappa}$, let $\tilde \gamma$ be the $\kappa$-development of $[x y]$ with respect to $p$, where $\tilde \gamma$ has basepoint $\tilde p$ and subgraph $D$.
Consider the map $H\:D\to\spc{U}$ that sends the point with parameter $(t,s)$ under the line-of-sight map for $\tilde \gamma$ with respect to $\tilde p$, to the point with the same parameter under the line-of-sight map $f$ for $[x y]$ with respect to $p$.
Then $H$ is  length-nonincreasing.
In particular, $D$ majorizes triangle $\trig p x y$.
\end{thm}

\parit{Proof.}
Let $\gamma=\geod_{[x y]}$ and $T=\dist{x}{y}{}$. 
As in the proof of the development criterion (\ref{thm:concave-devel}), take a partition 
\[0=t^0<t^1<\dots<t^n=T,\]
and set $x^i=\gamma(t^i)$. 
Construct a chain of model triangles  $\trig{\tilde p}{\tilde x^{i-1}}{\tilde x^i}\z=
\modtrig\kappa(p x^{i-1} {x^i})$, with $\tilde x^0=\tilde x$ and the direction of $[\tilde p\tilde x^i]$ turning counterclockwise as $i$ grows.  
Let $D_n$ be the subgraph with respect to $\tilde p$ of the polygonal line $\tilde x^0\dots \tilde x^n$.


Let  $\delta_n$ be the maximum radius of a circle inscribed in any of the triangles $\trig{\tilde p}{\tilde x^{i-1}}{\tilde x^i}$.  

Now we construct a map $H_n \: D_n\to\spc{U}$  that increases distances by at most  $2\cdot\delta_n$.
Suppose $w\in D_n$.
Then $w$ lies on or inside some triangle $\trig{\tilde p}{\tilde x^{i-1}}{\tilde x^i}$.  
Define $H_n(w)$ by first mapping $w$ to a nearest point on $\trig{\tilde p}{\tilde x^{i-1}}{\tilde x^i}$ (choosing one if there are several), followed by the natural map to the triangle  $\trig {p}{x^{i-1}}{ x^i}$. 

Since triangles in $\spc{U}$ are $\kappa$-thin (\ref{prop:k-thin}), the restriction of $H_n$ to each triangle $\trig{\tilde p}{\tilde x^{i-1}}{\tilde x^i}$ is short.   
Then the triangle inequality implies that the restriction of $H_n$ to 
\[U_n=\bigcup_{1\le i\le n}\trig{\tilde p}{\tilde x^{i-1}}{\tilde x^i}\]
is short with respect to the length metric on $D_n$. 
Since nearest-point projection from $D_n$ to $U_n$ increases the $D_n$-distance between two points by at most $2\cdot\delta_n$, the map $H_n$ also increases the $D_n$-distance by at most $2\cdot\delta_n$. 

Consider converging sequences $v_n\to v$ and $w_n\to w$ such that $v_n,w_n\in D_n$ and therefore $v,w\in D$.
Note that 
\[\dist{H_n(v_n)}{H_n(w_n)}{} \le \dist{v_n}{w_n}{D_n} + 2\cdot\delta_n,\eqlbl{eq:|H(v)-H(w)|}\]
for each $n$.
Since $\delta_n\to 0$ and geodesics in $\spc{U}$ vary continuously with their endpoints (\ref{thm:alex-patch}), we have $H_n(v_n)\to 
H(v)$ and $H_n(w_n)\to H(w)$.
Therefore the left-hand side in \ref{eq:|H(v)-H(w)|} converges to $\dist{H(v)}{H(w)}{}$ and the right-hand side converges to $\dist{v}{w}{D}$, it follows that $H$ is short.
\qeds




\parit{Proof of \ref{thm:major}.}
We begin by proving the theorem in case $\alpha$ is polygonal.

First suppose $\alpha$ is a triangle, say $\trig p x y$.
By assumption, the perimeter of $\trig p x y$ is less than
$2\cdot\varpi\kappa$.
This is the base case for the induction.

 Let $\tilde \gamma$ be the $\kappa$-development of $[x y]$ with respect to $p$, where $\tilde \gamma$ has basepoint $\tilde p$ and subgraph $D$.
By the development criterion (\ref{thm:concave-devel}),  $\tilde \gamma$ is concave.
By Lemma~\ref{lem:majorize-subgraph},  there is a short map $G\:\Conv\modtrig\kappa(p x y)\to D$.
Further, by Lemma~\ref{lem:majorize-triangle},  $D$ majorizes $\trig p x y$ under a majorizing map $H\:D\to\spc{U}$. Clearly $H\circ G$ is a majorizing map for $\trig p x y$.

\begin{wrapfigure}{r}{40 mm}
\vskip-1mm
\centering
\includegraphics{mppics/pic-960}
\vskip0mm
\end{wrapfigure}

Now we claim that any closed $n$-gon $[x^1x^2 \dots x^n ]$ of perimeter less than $2\cdot \varpi\kappa$ in a $\CAT{\kappa}$ space  is majorized by a convex polygonal region \[R_n=\Conv[\tilde x^1\tilde x^2\dots\tilde x^n]\]
under a map $F_n$ such that $F_n\:\tilde x^i\mapsto x^i$ for each $i$. 


Assume the statement is true for $(n-1)$-gons, $n\ge 4$.  
Then  $[x^1 x^2 \dots x^{n-1}]$  is majorized by a convex polygonal region 
\[R_{n-1}=\Conv[\tilde x^1 \tilde x^2,\dots \tilde x^{n-1}],\] 
in $\Lob2\kappa$ under a map $F_{n-1}$ satisfying $F_{n-1}(\tilde x^i)=x^i$ for all $i$. 
Take $\dot x^n\in\Lob2{\kappa}$ such that $\trig{\tilde x^1}{\tilde x^{n-1}}{\dot x^n}=\modtrig\kappa(x^1 x^{n-1} x^n)$ 
and this triangle lies on the other side of $[\tilde x^1\tilde x^{n-1}]$ from $R_{n-1}$.  
Let $\dot R\z=\Conv\trig{\tilde x^1}{\tilde x^{n-1}}{\dot x^n}$, 
and $\dot F\:\dot R\to \spc{U}$ be a majorizing map for $\trig { x^1}{x^{n-1}}{ x^n}$ as provided above.

Set 
$R= R_{n-1}\cup \dot R$, where $R$ carries its length metric.
Since $F_n$ and $F$ agree on $[\tilde x^1 \tilde x^{n-1}]$, we may define $F\:R\to\spc{U}$ by 
\[
F(x)=
\begin{cases}
F_{n-1}(x),\quad & x\in R_{n-1},\\
\dot F(x),\quad & x\in \dot R.\\
\end{cases}
\]
Then $F$ is length-nonincreasing and is a majorizing map for $[x^1 x^2 \dots x^n ]$ (as in Definition~\ref{def:majorize}).

If $R$ is a convex subset of $\Lob2\kappa$, we are done. 

If $R$ is not convex,  the total internal angle of $R$ at $\tilde x^1$ or $ \tilde x^{n-1} $ or both is $>\pi$.  
By relabeling we may suppose this holds for $\tilde x^{n-1}$.  

The region $R$ is obtained by gluing $R_{n-1}$ to $\dot R$ by $[x^1x^{n-1}]$.
Thus, by Reshetnyak gluing (\ref{thm:gluing}), $R$ carrying its length metric is a $\CAT{\kappa}$-space.  
Moreover $[\tilde x^{n-2}\tilde x^{n-1}]\cup[\tilde x^{n-1} \dot x^n]$ is a geodesic of $R$.
Thus $[\tilde x^1 \tilde x^2 \dots \tilde x^{n-2} \dot x^n]_R$ is a closed $(n-1)$-gon in $R$, to which the induction hypothesis applies. The resulting short map from a convex region in $ \Lob2\kappa$ to~$R$, followed by $F$,  is the desired majorizing map.

\medskip

Note that in fact we have proved the following:

\begin{clm}{}
Let $F_{n-1}$ be a majorizing map for the polygon $[x^1x^2\dots x^{n-1}]$,
and $\dot F$ be a majorizing map for the triangle $[x^1x^{n-1}x^{n}]$.
Then there is a majorizing map $F_n$ for the polygon $[x^1x^2\dots x^n]$
such that \[\Im F_{n+1}= \Im F_n\cup\Im \dot F.\]

\end{clm}

We now use this claim to prove the theorem for general curves.

Assume $\alpha\:[0,\ell]\to\spc{U}$ is an  arbitrary closed curve with natural parameter.
Choose a sequence of partitions $0=t^0_n<t^2_n<\dots<t^n_n=\ell$
so that:
\begin{itemize}
\item The set $\{t_{n+1}^i\}_{i=0}^{n+1}$ 
is obtained from the set  $\{t_n^i\}_{i=0}^n$ by adding one element.
\item For some sequence $\eps_n\to0+$,
we have $t^i_n-t^{i-1}_n<\eps_n$ for all $i$.
\end{itemize}

Inscribe in $\alpha$ a sequence of polygons $P_n$ with vertexes $\alpha(t^i_n)$.
Apply the claim above, to get a sequence of majorizing maps $F_n\:R_n\to\spc{U}$.
Note that for all $m>n$ we have
\begin{itemize}
\item $\Im F_m$ lies in an  $\eps_n$-neighborhood of $\Im F_n$,
\item $\Im F_m\backslash \Im F_n$ lies in an  $\eps_n$-neighborhood of $\alpha$.
\end{itemize}
It follows that the set
\[K=\alpha\cup\left(\bigcup_n\Im F_n\right)\]
is compact.
Therefore the sequence $(F_n)$
has a partial limit as $n\to\infty$; 
say $F$.
Clearly $F$ is a majorizing map for $\alpha$.
\qeds

If $p_1\dots p_n$ is a polygon, then values $\theta_i=\pi-\mangle\hinge{p_i}{p_{i-1}}{p_{i+1}}$ for all $i\pmod n$ are called \emph{external angles} of the polygon.
The following exercise is a generalization of Fenchel's theorem.

\begin{thm}{Exercise}\label{ex:fenchel}
Show that the sum of external angles of any polygon in a complete length $\CAT0$ space cannot be smaller than $2\cdot\pi$. 
\end{thm}

\begin{thm}{Very advanced exercise}\label{ex:FM}
Suppose that a simple polygon $\beta$ in a complete length $\CAT0$ space does not bound an embedded disc.
Show that the sum of external angles of $\beta$ cannot be smaller than $4\cdot\pi$.

Give an example of such a polygon $\beta$ with the sum of external angles exactly $4\cdot\pi$.
\end{thm}

The following exercise is the rigidity case 
of the majorization theorem.

{\sloppy 

\begin{thm}{Exercise}\label{ex:isometric-majorization}
Let $\spc{U}$ be a $\varpi\kappa$-geodesic $\CAT\kappa$ space
and $\alpha\:[0,\ell]\to\spc{U}$ be a closed curve with arclength parametrization.
Assume that $\ell\z<2\cdot \varpi\kappa$
and there is a closed convex curve $\tilde \alpha\:[0,\ell]\to\Lob{2}{\kappa}$ such that 
\[\dist{\alpha(t_0)}{\alpha(t_1)}{\spc{U}}=\dist{\tilde \alpha(t_0)}{\tilde \alpha(t_1)}{\Lob{2}{\kappa}}\]
for any $t_0$ and $t_1$.
Then there is a distance-preserving map $F\:\Conv\tilde \alpha\to \spc{U}$
such that $F\:\tilde \alpha(t)\mapsto \alpha(t)$ for any $t$.
\end{thm}

}

\begin{thm}{Exercise}\label{ex:bishop}
Two majorizations $F\:D\to \spc{U}$ and $F'\:D'\to \spc{U}$ will be called \index{majorizing map!equivalent majorizations}\emph{equivalent} if $F'=F\circ\iota$ for an isometry $\iota\:D\to D'$.

Show that a closed rectifiable curve in a $\CAT0$ space has an isometric majorization map if and only if the majorization map is unique up to equivalence.
\end{thm}

The following lemma states, in particular, that in a $\CAT\kappa$ space, 
a sharp triangle comparison implies the
presence of an isometric copy of the convex hull of the model triangle.
The latter statement was proved by Alexandr Alexandrov \cite{alexandrov:devel}.
  
\begin{thm}{Arm lemma}\label{lem:arm}
Let $\spc{U}$ be a $\varpi\kappa$-geodesic $\CAT\kappa$ space, 
and $P=[x^0x^1\dots x^{n+1}]$ be a polygon of length $<2\cdot \varpi\kappa$ in $\spc{U}$.
Suppose $\tilde P=[\tilde x^0\tilde x^1\dots \tilde x^{n+1}]$ is a convex  polygon in $\Lob{2}{\kappa}$
such that 
\[
\dist{\tilde x^i}{\tilde x^{i-1}}{\Lob{2}{\kappa}}
=
\dist{x^i}{x^{i-1}}{\spc{U}}
\quad \text{and}\quad 
\mangle\hinge{x^i}{x^{i-1}}{x^{i+1}}\ge\mangle\hinge{\tilde x^i}{\tilde x^{i-1}}{\tilde x^{i+1}}
\eqlbl{eq:arm}
\]
for all $i$.
Then 

\begin{subthm}{subthm:arm-ineq}
$\dist{\tilde x^0}{\tilde x^{n+1}}{\Lob{2}{\kappa}}
\le
\dist{x^0}{x^{n+1}}{\spc{U}}$.
\end{subthm}

\begin{subthm}{subthm:arm-eq}
Equality holds in \ref{SHORT.subthm:arm-ineq} if and only if the map $\tilde x^i\mapsto x^i$ can be extended 
to a distance-preserving map of $\Conv(\tilde x^0,\tilde x^1\dots \tilde x^{n+1})$ onto $\Conv(x^0,x^1\dots x^{n+1})$.
\end{subthm}
\end{thm}

\parit{Proof; \ref{SHORT.subthm:arm-ineq}.}
By majorization (\ref{thm:major}), $P$ is majorized by a convex region $\tilde D$ in $\Lob{2}{\kappa}$.
By Proposition \ref{prop:majorize-geodesic} and the definition of angle,
$\tilde D$ is bounded by a convex polygon $\tilde P_R=[\tilde y^0\tilde y^1\dots \tilde y^{n+1}]$ that satisfies
\begin{align*}
\dist{\tilde y^i}{\tilde y^{i\pm1}}{\Lob{2}{\kappa}}
&=
\dist{x^i}{x^{i\pm1}}{\spc{U}}, \qquad \dist{\tilde y^0}{\tilde y^{n+1}}{\Lob{2}{\kappa}}
=
\dist{x^0}{x^{n+1}}{\spc{U}},
\\
& \mangle\hinge{\tilde y^i}{\tilde y^{i-1}}{\tilde y^{i+1}}\ge\mangle\hinge{x^i}{x^{i-1}}{x^{i+1}}\ge\mangle\hinge{\tilde x^i}{\tilde x^{i-1}}{\tilde x^{i+1}}
\end{align*}
for $1\le i\le n$; the last inequality follows from \ref{eq:arm}.

The classical arm lemma \cite{sabitov} gives $\dist{\tilde x^0}{\tilde x^{n+1}}{}\le \dist{\tilde y^0}{\tilde y^{n+1}}{}$.
Since $ \dist{\tilde y^0}{\tilde y^{n+1}}{}=\dist{x^0}{x^{n+1}}{}$, part \ref{SHORT.subthm:arm-ineq} follows.

\parit{\ref{SHORT.subthm:arm-eq}.}
Suppose equality holds in \ref{SHORT.subthm:arm-ineq}.
Then angles at the $j$-th vertex of~$\tilde P$, $P$, and $\tilde P_R$ are equal for $1\le j\le n$, and we may take $\tilde P=\tilde P_R$.  

Let $F\:\tilde D\to\spc{U}$ be the majorizing map for $P$, where $\tilde D$ is the convex region bounded by $\tilde P$, and $F|_{\tilde P}$ is length-preserving.  

\begin{clm}{}\label{clm:arm-triangle}
Let $\tilde x,\tilde y,\tilde z$ be three vertexes of $\tilde P$, and $x,y,z$ be the corresponding vertexes of $P$.  If $\dist{\tilde x}{\tilde y}{}=
\dist{x}{y}{}$, $\dist{\tilde y}{\tilde z}{}=
\dist{x}{z}{}$ and $\mangle\hinge{\tilde y}{\tilde x}{\tilde z} = \mangle\hinge{y}{x}{z}$, then $F|_{\Conv(\tilde x, \tilde y, \tilde z)}$ is distance-preserving.
\end{clm} 

Indeed, since $F$ is majorizing, $F$ restricts to   distance-preserving maps from $[\tilde x\tilde y]$ to $[xy]$ and $[\tilde y\tilde z]$ to $[yz]$.
Suppose $\tilde p\in [\tilde x \tilde y]$ and $\tilde q\in[\tilde y\tilde z]$.  Then 
\[
\dist{\tilde p}{\tilde q}{\Lob{2}{\kappa}}
=
\dist{F( \tilde p )}{F(\tilde q)}{\spc{U}}.
 \eqlbl{eq:arm-eq}
\]
This inequality holds in one direction by majorization, and in the other direction by the angle comparison (\ref{cat-hinge}).
By the first variation formula (\ref{cor:both-end-first-var-cba}), it follows that each pair of corresponding angles of triangles $[\tilde x \tilde y \tilde z]$ and $[x y z]$ are equal.
But then \ref{eq:arm-eq} holds for $p,q$ on any two sides of these triangles, so $F$ is distance-preserving on every geodesic of $\Conv(\tilde p, \tilde x, \tilde y)$.
Hence the claim.
\claimqeds

\begin{clm}{}\label{clm:arm-induction}
Suppose $F|_{\Conv(\tilde x^0,\tilde x^1,\dots,\tilde x^{k})}$ is distance-preserving for some $k$, $2\le k\le n-1$.
Then $F|_{\Conv(\tilde x^0,\tilde x^1,\dots, \tilde x^{k+1})}$ is distance-preserving.
\end{clm}

To verify this claim, let 
\[
\tilde p=[\tilde x^{k-1}\tilde x^{k+1}] \cap [\tilde x^{k}\tilde x^{0}]
\quad\text{and}\quad
p=F(p).
\]

Note that the following maps are distance-preserving:
\begin{enumerate}
\item[(i)]
$F|_{\Conv(\tilde x^{k-1},\tilde x^k,\tilde x^{k+1})}$,

\item[(ii)]
 $F|_{\Conv(\tilde x^{k+1},\tilde x^{k-1},\tilde x^{0})}$,

\item[(iii)]
$F|_{\Conv(\tilde x^{0},\tilde x^{k},\tilde x^{k+1})}$.
\end{enumerate}
Indeed, (i) follows from \ref{clm:arm-triangle}.  
Therefore $\dist{\tilde x^{k-1}}{\tilde x^{k+1}}{}=\dist{x^{k-1} }{x^{k+1}}{}$, and so $F$ restricts to a distance-preserving map from $[\tilde x^{k-1}\tilde x^{k+1}]$ onto $[x^{k-1} x^{k+1}]$.  With the induction hypothesis in
\ref{clm:arm-induction},
 it follows that $p=[x^{k-1}x^{k+1}] \cap [x^{k}x^{0}]$, hence 
\[
\mangle\hinge{\tilde x^{k-1}}{\tilde x^{k+1}}{\tilde x^{0}} 
= \mangle\hinge{x^{k-1}}{x^{k+1}}{x^{0}}.
 \eqlbl{eq:angle-arm-eq}
\] 
Then (ii) follows from \ref{eq:angle-arm-eq} and \ref{clm:arm-triangle}.  Since $\dist{\tilde x^{k}}{\tilde x^{0}}{}=\dist{x^{k} }{x^{0}}{}$, (iii) follows from \ref{eq:angle-arm-eq} and (i). 

Let $\tilde \gamma$ be a geodesic of $\Conv(\tilde x^{0},\tilde x^0,\tilde x^1\dots \tilde x^{k+1})$.
Then $\length \tilde \gamma \z< \varpi\kappa$.
If $\tilde \gamma$ does not contain the point $\tilde p$, then by the induction hypothesis in
\ref{clm:arm-induction} and (i)+(ii)+(iii), we get that  $\gamma = F\circ\tilde \gamma$ is a local geodesic of length $<\varpi\kappa$.
By \ref{cor:loc-geod-are-min}, $\gamma$ is a geodesic.

By continuity, $F\circ\tilde \gamma$ is a geodesic for all $\tilde \gamma$;
so \ref{clm:arm-induction} follows.

The base of the induction is provided by \ref{clm:arm-triangle}.
It finishes the proof of part \ref{SHORT.subthm:arm-eq}.
\qeds
 
\begin{thm}{Exercise}\label{ex:square}
Let $\spc{U}$ be a complete length $\CAT0$ space and 
suppose 
for 4 points $x^1,x^2,x^3,x^4\in \spc{U}$
there is a convex quadrangle
$[\tilde x^1\tilde x^2\tilde x^3\tilde x^4]$
in $\EE^2$
such that 
\[\dist{x^i}{x^j}{\spc{U}}=\dist{\tilde x^i}{\tilde x^j}{\EE^2}\]
for all $i$ and $j$.
Show that $\spc{U}$ contains an isometric copy of the 
\emph{solid quadrangle}
$[\tilde x^1\tilde x^2\tilde x^3\tilde x^4]$; that is, the convex hull of $\tilde x^1,\tilde x^2,\tilde x^3,\tilde x^4$ in $\EE^2$.
\end{thm}

%%%%%%%%%%%%%%%%%%%%%%%%%%%%%%%%%%%%%%%%%%%%%%%%%%%%%%%%%%%%%%%%%%%%%%%%%%%%%%%%

\section{Hadamard--Cartan theorem}\label{sec:Hadamard--Cartan}

The development of Alexandrov geometry was greatly influenced by the Hadamard--Cartan theorem.
Its original formulation states that if $M$ is a complete Riemannian manifold with nonpositive sectional curvature, 
then the exponential map at any point $p\in M$ is a covering;
in particular it implies that the universal cover of $M$ is diffeomorphic to the Euclidean space of the same dimension.

In this generality, the theorem appeared in the lectures of \'Elie Cartan \cite{cartan}.
For surfaces in the Euclidean space, 
the theorem was proved by
Hans von Mangoldt \cite{mangoldt},  
and a few years later independently by Jacques Hadamard \cite{hadamard}.

Formulations for metric spaces of different generality were proved by 
Herbert Busemann \cite{busemann-CBA},
Willi Rinow \cite{rinow}, and 
Michael Gromov \cite[p.~119]{gromov:hyp-groups}. 
A detailed proof of Gromov's statement when $\spc{U}$ is proper  was given by Werner Ballmann \cite{ballmann:cartan-hadamard}, using Birkhoff's curve-shortening.  
A proof in the non-proper 
geodesic case 
was given by the first author and Richard Bishop~\cite{alexander-bishop:h-c}.  
This proof applies more generally, to {}\emph{convex spaces} (see Exercise \ref{ex:cats-cradle}).
It was pointed out by Bruce Kleiner \cite{ballmann:lectures} 
and independently by Martin Bridson and Andr\'{e} Haefliger \cite{bridson-haefliger} that 
%the no-conjugate-point theorem (\ref{thm:no-conj-pt}) allows curve-shortening as in Corollary \ref{cor:homotopy-from-p}, and hence the Hadamard--Cartan theorem
this proof extends to length spaces, as well as geodesic spaces, giving the following statement:

\begin{thm}{Hadamard--Cartan theorem}
\label{thm:hadamard-cartan}
Let $\kappa\le 0$, and $\spc{U}$ be a complete, simply connected length locally $\CAT\kappa$ space.
Then $\spc{U}$ is $\CAT\kappa$.
\end{thm}

\parit{Proof.} Since $\varpi\kappa=\infty$, Theorem~\ref{thm:globalization-lift} implies that there is a $\CAT\kappa$ space $\spc{B}$ and a \index{metric covering}\emph{metric covering} $\map\:\spc{B}\to\spc{U}$; that is, $\map$ is a length-preserving covering map. 

Since $\spc{U}$ is simply connected, $\map\:\spc{B}\to\spc{U}$ is an isometry --- hence the result.
\qeds

To formulate the generalized Hadamard--Cartan theorem,
we need the following definition.

\begin{thm}{Definition}\label{def:l-s.c.}
Given $\ell\in (0,\infty]$,
a metric space $\spc{X}$ is called 
$\ell$-simply connected 
if it is connected and 
any closed curve of length $<\ell$ 
is null-homotopic in the class of curves of length $<\ell$ in $\spc{X}$.
\end{thm}

Note that there is a subtle difference between 
simply connected and $\infty$-simply connected spaces;
the first states that any closed curve is null-homotopic while the second means that any rectifiable curve is null-homotopic in the class of rectifiable curves.
However, as follows from Proposition~\ref{prop:sc}, for locally $\CAT\kappa$ spaces these two definitions are equivalent.
This fact makes it possible to deduce the Hadamard--Cartan theorem directly from the generalized Hadamard--Cartan theorem.

\begin{thm}{Generalized Hadamard--Cartan theorem}\label{thm:hadamard-cartan-gen}
A complete length space
$\spc{U}$ is $\CAT\kappa$ 
if and only if $\spc{U}$ is $2\cdot\varpi\kappa$-simply connected
and $\spc{U}$ is locally $\CAT\kappa$.
\end{thm}

For proper spaces, the generalized Hadamard--Cartan theorem was proved by Brian Bowditch \cite{bowditch}.
In the proof we need the following lemma.

\begin{thm}{Lemma}
Assume $\spc{U}$ is a complete length  locally $\CAT\kappa$ space,
$\eps>0$,
and $\gamma_1,\gamma_2\:\mathbb{S}^1\to\spc{U}$ are two closed curves.
Assume 
\begin{subthm}{}
$\length\gamma_i<2\cdot\varpi\kappa-4\cdot\eps$ for $i=1,2$;
\end{subthm}
 
\begin{subthm}{} $\dist{\gamma_1(x)}{\gamma_2(x)}{}<\eps$ for any $x\in\mathbb{S}^1$, and the geodesic $[\gamma_1(x)\gamma_2(x)]$ is uniquely defined and depends continuously on $x$;
\end{subthm}

\begin{subthm}{}  $\gamma_1$ is majorized by a convex region in $\Lob2\kappa$.
\end{subthm}

Then  $\gamma_2$ is majorized by a convex region in $\Lob2\kappa$.
\end{thm}

\parit{Proof.} Let $D$ be a convex region in $\Lob2\kappa$ that majorizes $\gamma_1$ under the map $F\:D\to\spc{U}$ 
(see Definition~\ref{def:majorize}).
Denote by $\tilde \gamma_1$ 
the curve bounding $D$ 
such that $F\circ\tilde \gamma_1=\gamma_1$.
Since  
\begin{align*}
\length\tilde \gamma_1
&=
\length\gamma_1
<
\\
&<
2\cdot\varpi\kappa-4\cdot\eps,
\end{align*}
there is a point $\tilde p\in D$ such that 
$\dist{\tilde p}{\tilde \gamma(x)}{\Lob2\kappa}<\tfrac{\varpi\kappa}2-\eps$
for any $x\in\mathbb{S}^1$.
Denote by $\alpha_x$ the concatenation of the paths $F\circ\geodpath_{[p\tilde \gamma_1(x)]_{\Lob2\kappa}}$ 
and  $\geodpath_{[\gamma_1(x)\gamma_2(x)]}$ in $\spc{U}$.
Note that $\alpha_x$ depends continuously on $x$, and
$$\length\alpha_x<\tfrac{\varpi\kappa}{2}\quad \text{and}\quad \alpha_x(1)=\gamma_2(x)$$ 
hold for any $x$.

Let us apply the lifting globalization theorem (\ref{thm:globalization-lift}) for $p\z=F(\tilde p)$.
We obtain a $\varpi\kappa$-geodesic $\CAT\kappa$ space $\spc{B}$
and a locally distance-preserving map $\map\:\spc{B}\to\spc{U}$
with $\map(\hat p)=p$ for some $\hat p \in \spc{B}$, and with the lifting property for the curves starting at $p$ with length $<\varpi\kappa/2$.
Applying the lifting property for $\alpha_x$, 
we get existence of a curve $\hat\gamma_2\:\mathbb{S}^1\to \spc{B}$ such that
$$\gamma_2=\map\circ\hat\gamma_2.$$

Since $\spc{B}$ is a geodesic $\CAT\kappa$ space, we can apply the majorization theorem (\ref{thm:major}) for $\hat\gamma_2$.
The composition of the obtained majorization with $\map$ is a majorization of $\gamma_2$.
\qeds

\parit{Proof of Theorem \ref{thm:hadamard-cartan-gen}.}
The only-if part follows from the Reshetnyak majorization theorem (\ref{thm:major}).

Let  $\gamma_t$, $t\in[0,1]$ 
be a null-homotopy of curves in $\spc{U}$;
that is, $\gamma_0(x)\z=p$ for some $p\in \spc{U}$
and any $x\in\mathbb{S}^1$.
Assume further that $\length \gamma_t\z<2\cdot\varpi\kappa$ for any $t$.
To prove the if part, it is sufficient to show that $\gamma_1$ is majorized by a convex region in $\Lob2\kappa$ if $\spc{U}$ is locally $\CAT\kappa$. 

By semicontinuity of length (\ref{thm:semicont-of-length}),
we can choose  $\eps>0$ sufficiently small that
$$\length \gamma_t<2\cdot\varpi\kappa-4\cdot\eps$$
for all $t$.

By Corollary~\ref{cor:loc-CAT(k)},
we may assume in addition that
$\oBall(\gamma_t(x),\eps)$ is $\CAT\kappa$ 
for any $t$ and $x$.

Choose a partition $0=t_0<t_1<\dots<t_n=1$
so that $\dist{\gamma_{t_i}(x)}{\gamma_{t_{i-1}}(x)}{}<\eps$
for any $i$ and $x$.
According to \ref{thm:cat-unique},
for any $i$,
the geodesic $[\gamma_{t_i}(x)\gamma_{t_{i-1}}(x)]$ depends continuously on $x$.

Note that $\gamma_0=\gamma_{t_0}$ is majorized by a convex region in $\Lob2\kappa$.
Applying the lemma $n$ times, we see that the same holds for $\gamma_1\z=\gamma_{t_n}$.
\qeds

\begin{thm}{Proposition}\label{prop:sc}
Let $\spc{U}$ be a complete length locally $\CAT\kappa$ space.
Then $\spc{U}$ is simply connected if and only if it is $\infty$-simply connected.
\end{thm}

\parit{Proof; if part.}
It is sufficient to show that any closed curve in $\spc{U}$ is homotopic to a polygon.

Let $\gamma_0$ be a closed curve in $\spc{U}$.
According to Corollary~\ref{cor:loc-CAT(k)},
there is $\eps>0$ such that 
$\oBall(\gamma(x),\eps)$ is $\CAT\kappa$
for any $x$.

Choose a polygon $\gamma_1$ such that $\dist{\gamma_0(x)}{\gamma_1(x)}{}<\eps$ for any $x$.
By \ref{thm:cat-unique}, 
$\geodpath_{[\gamma_0(x)\gamma_1(x)]}$ 
is uniquely defined 
and depends continuously on~$x$.

Hence $\gamma_t(x)=\geodpath_{[\gamma_0(x)\gamma_1(x)]}(t)$ gives a homotopy from $\gamma_0$ to $\gamma_1$.

\parit{Only-if part.} The proof is similar.

Assume $\gamma_t$ is a homotopy between two rectifiable curves $\gamma_0$ and $\gamma_1$.
Fix $\eps>0$ so that the ball $\oBall(\gamma_t(x),\eps)$ is $\CAT\kappa$
for any $t$ and $x$.
Choose a partition $0=t_0<t_1<\dots<t_n=1$ 
so that 
$$\dist{\gamma_{t_{i-1}}(x)}{\gamma_{t_i}(x)}{}<\tfrac\eps{10}$$
for any $i$ and $x$.
Set $\hat\gamma_{t_0}=\gamma_0$, $\hat\gamma_{t_n}=\gamma_{t_n}$.
For each $0<i<n$, approximate $\gamma_{t_i}$ by a polygon $\hat\gamma_{i}$.

Construct the \index{geodesic homotopy}\emph{geodesic homotopy} 
from $\hat\gamma_{t_{i-1}}$ 
to $\hat\gamma_{t_i}$;  
that is,
set 
$$\hat\gamma_t
=
\geodpath_{[\hat\gamma_{t_{i-1}}(x)\hat\gamma_{t_i}(x)]}(t)$$
for $t\in [t_{i-1},t_i]$.
Since $\eps$ is sufficiently small, 
by \ref{cor:cat-unique}, we get that
$$\length\hat\gamma_t
<
10\cdot(\length\hat\gamma_{t_{i-1}}+\length\hat\gamma_{t_i})$$
for any $t\in [t_{i-1},t_i]$.
In particular, $\hat\gamma_t$ is rectifiable for all $t$.

Joining the obtained homotopies for all $i$, we obtain a homotopy from $\gamma_0$ to $\gamma_1$ in the class of rectifiable curves.
\qeds

\begin{thm}{Exercise}\label{ex:cover-branching-along-2-lines}
Let $\spc{X}$ be a double cover of $\EE^3$ that branches along two distinct lines $\ell$ and $m$.
Show that  $\spc{X}$ is $\CAT0$ if and only if $\ell$ intersects $m$ at a right angle.
\end{thm}

\begin{thm}{Exercise}\label{ex:branching}
Let $\spc{U}$ be a complete length $\CAT0$ space.
Assume $\tilde{\spc{U}}\to \spc{U}$ is a metric covering branching along a geodesic.
Show that $\tilde{\spc{U}}$ is $\CAT0$.

More generally, assume $A\subset \spc{U}$ is a closed convex subset and $f\:\spc{X}\to \spc{U}\backslash A$ is a metric cover.
Denote by $\bar{\spc{X}}$ the completion of $\spc{X}$, and 
$\bar f\:\bar{\spc{X}}\to \spc{U}$ the continuous extension of $f$.
Let $\tilde{\spc{U}}$ be the space glued from $\bar{\spc{X}}$ and $A$ by identifying $x$ and $\bar f(x)$ if $\bar f(x)\in A$.
Show that $\tilde{\spc{U}}$ is $\CAT0$.
\end{thm}

\parbf{About convex spaces.}
A \index{convex space}\emph{convex space} $\spc{X}$ is a geodesic space such that the function
$t\mapsto\dist{\gamma(t)}{\sigma(t)}{}$ is convex 
for any two  geodesic paths $\gamma,\sigma:[0,1]\to \spc{X}$.  
A \index{convex space!locally convex space}\emph{locally convex space} is a length space in which every point has a neighborhood that is a convex space in the restricted metric.


\begin{thm}{Exercise}\label{ex:cats-cradle}
Assume $\spc{X}$ is a convex space 
such that the angle of any hinge is defined.
Show that $\spc{X}$ is $\CAT{0}$.
\end{thm}

The following exercise gives an analog of Hadamard--Cartan theorem for locally convex spaces;
see also \cite{alexander-bishop:h-c}.

\begin{thm}{Exercise}\label{ex:Hadamard--Cartan}
Show that a complete, simply connected, locally convex space is a convex space.
\end{thm}

\section{Convex sets}
\label{sec:convex-CBA}

Recall that according to Corollary~\ref{cor:convex-balls}, any ball (closed or open) of radius $R<\tfrac{\varpi\kappa}2$ in a $\varpi\kappa$-geodesic $\CAT\kappa$ space is convex.
From the uniqueness of geodesics in $\CAT\kappa$ spaces (\ref{thm:cat-unique}) we get the following:

\begin{thm}{Observation}
Any weakly $\varpi\kappa$-convex set 
in a complete length $\CAT\kappa$ space is $\varpi\kappa$-convex.
\end{thm}


\begin{thm}{Closest-point projection lemma}\label{lem:closest point}{\sloppy 
Let $\spc{U}$ be a complete length $\CAT\kappa$ space, and $K\subset \spc{U}$ be a closed $\varpi\kappa$-convex set. 
Assume that $\distfun{K}{p}{}<\tfrac{\varpi\kappa}2$ for some point $p\in \spc{U}$.
Then  
there is a unique point $p^*\in K$ that minimizes the distance to $p$;
that is, $\dist{p^*}{p}{}=\distfun{K}{p}{}$. 

}

\end{thm}

\parit{Proof.} 
Fix $r$ properly between $\distfun{K}{p}{}$ and $\tfrac{\varpi\kappa}2$.
By the function comparison (\ref{thm:function-comp}),
the function $f=\md\kappa\circ\distfun{p}{}{}$ is strongly convex in $\cBall[p,r]$.

The lemma follows from Lemma~\ref{lem:argmin(convex)} applied to the subspace $K'\z=K\cap\cBall[p,r]$ 
and the restriction $f|_{K'}$. 
\qeds

\begin{thm}{Exercise}\label{ex:closest-point-projection}
Let  $\spc{U}$ be a complete length $\CAT0$ space and $K\subset \spc{U}$ be a closed convex set.
Show that the closest-point projection $\spc{U}\to K$ is short. 
\end{thm}

\begin{thm}{Advanced exercise}\label{ex:short-retraction-CBA(1)}
Let  $\spc{U}$ be a complete length $\CAT1$ space and $K\subset \spc{U}$ be a closed $\pi$-convex set.
Assume $K\subset \cBall[p,\tfrac\pi2]$ for some point $p\in K$.
Show that there is a short retraction of $\spc{U}$ to~$K$. 
\end{thm}

\begin{thm}{Proposition}\label{lem:dist-to-convex}
Let $\spc{U}$  be a $\varpi\kappa$-geodesic $\CAT\kappa$ space
and $K\subset \spc{U}$ be a closed $\varpi\kappa$-convex set.
Let
\[f=\sn\kappa\circ\distfun{K}{}{}.\]
Then
\[f''+\kappa \cdot f\ge 0\]
holds in $\oBall(K,\tfrac{\varpi\kappa}2)$.
\end{thm}

\parit{Proof.}
It is sufficient to show that Jensen's inequality (\ref{y''-mono})
holds on a sufficiently short 
geodesic $[pq]$ in $\oBall(K,\tfrac{\varpi\kappa}2)$.
We may assume that 
\[\dist{p}{q}{}+\distfun{K}{p}{}+\distfun{K}{q}{}<\varpi\kappa.\eqlbl{eq:sum=<varpi}\]

For each $x\in[pq]$,
we need to find a value $h(x)\in \RR$
such that 
\[
h(p)=f(p),\qquad 
h(q)=f(q),\qquad
h(x)\le f(x)\qquad
\]
for any $x\in [pq]$,
and
\[h''+\kappa\cdot h\ge 0\eqlbl{h''+kh=<0}\]
along $[pq]$.

Denote by $p^{*}$ and $q^{*}$ the closest-point projections of $p$ and $q$ on $K$; 
they are provided by \ref{lem:closest point}.
From \ref{eq:sum=<varpi} and the triangle inequality,
we have
\[\dist{p^*}{q^*}{}<\varpi\kappa.\]
Since $K$ is $\varpi\kappa$-convex, $K\supset[p^*q^*]$;
in particular
\[\distfun{K}{x}{}\le \distfun{[p^*q^*]}{x}{}\]
for any $x\in \spc{U}$.

There is a majorizing map $F: D\to \spc U$ for quadrangle $[pp^{*}q^{*}q]$, as in Definition  \ref{def:majorize} and the Reshetnyak majorization theorem (\ref{thm:major}).
By Proposition \ref{prop:majorize-geodesic}, 
the figure $D$ is a solid convex quadrangle $[\tilde p\tilde p^*\tilde q^*\tilde q]$ in $\Lob2\kappa$ such that 
\begin{align*}
\dist{\tilde p}{\tilde p^*}{\Lob2\kappa}&=\dist{p}{p^*}{\spc{U}}
&
\dist{\tilde p}{\tilde q}{\Lob2\kappa}&=\dist{p}{q}{\spc{U}}
\\
\dist{\tilde q}{\tilde q^*}{\Lob2\kappa}&=\dist{q}{q^*}{\spc{U}}
&
\dist{\tilde p^*}{\tilde q^*}{\Lob2\kappa}&=\dist{p^*}{q^*}{\spc{U}}.
\end{align*}
Given $x\in [pq]$, denote by $\tilde x$ the corresponding point on $[\tilde p\tilde q]$.
Then
\[\distfun{[pq]}{x}{}
\le
\distfun{[\tilde p\tilde q]}{\tilde x}{}.\]
Set 
\[h(x)
=
\sn\kappa\circ
\distfun{[\tilde p\tilde q]}{\tilde x}{}.\]
By straightforward calculations, \ref{h''+kh=<0} holds
and hence the statement follows.
\qeds

\begin{thm}{Corollary}\label{cor::dist-to-convex}
Let $\spc{U}$  be a complete length $\CAT\kappa$ space
and $K\subset \spc{U}$ be a closed  locally convex set.
Then there is an open set $\Omega\supset K$
such that the function 
$f=\sn\kappa\circ\distfun{K}{}{}$
satisfies 
\[f''+\kappa\cdot f\ge 0\]
in $\Omega$.
\end{thm}

\parit{Proof.}
Fix $p\in K$.
By Corollary~\ref{cor:convex-balls},
$\cBall[p,r]$ is convex for all small $r>0$.

Since $K$ is locally convex, there is $r_p>0$ such that 
the intersection
$K'=K\cap \oBall(p,r_p)$ is convex. 

Note that 
\[\distfun{K}{x}{}=\distfun{K'}{x}{}\]
for any $x\in \oBall(p,\tfrac{r_p}2)$.
Therefore the statement holds for 
\[\Omega=\bigcup_{p\in K}\oBall(p,\tfrac{r_p}2).\]
\qedsf



\begin{thm}{Theorem}\label{thm:local-global-convexity}
Assume $\spc{U}$ is a complete length $\CAT\kappa$ space and $K\subset \spc{U}$ is a closed connected locally convex set.
Assume $\dist{x}{y}{}<\varpi\kappa$ for any $x,y\in K$.
Then $K$ is convex.

In particular, if $\kappa\le 0$, then any closed connected locally convex set in $\spc{U}$ is convex.
\end{thm}

The following proof is due to Sergei Ivanov \cite{ivanov:local-global-convexity}.

\begin{wrapfigure}{o}{50 mm}
\vskip-1mm
\centering
\includegraphics{mppics/pic-970}
\vskip0mm
\end{wrapfigure}

\parit{Proof.}
Since $K$ is locally convex,
it is locally path-connected.
Since $K$ is connected and locally path connected it is path-connected.

Fix two points $x,y\in K$. 
Let us connect $x$ to $y$ by a path $\alpha\:[0,1]\to K$.
Since $\dist{x}{\alpha(s)}{}<\varpi\kappa$ for any $s$,
Theorem~\ref{thm:cat-unique} implies that the geodesic $[x\alpha(s)]$ 
is uniquely defined and depends continuously on $s$.

Let $\Omega\supset K$ be the open set provided by Corollary~\ref{cor::dist-to-convex}.
If $[xy]\z=[x\alpha(1)]$ does not completely lie in $K$, then 
there is a value $s\in [0,1]$ such that $[x\alpha(s)]$ 
lies in $\Omega$ but does not completely lie in $K$.
By Corollary~\ref{cor::dist-to-convex},
the function $f=\sn\kappa\circ\distfun{K}{}{\spc{U}}$ 
satisfies the differential inequality
\[f''+\kappa\cdot f\ge 0\eqlbl{f''+kappa f=<0}\]
along $[x\alpha(s)]$.

Since 
\begin{align*}
\dist{x}{\alpha(s)}{}&<\varpi\kappa,
&
f(x)&=f(\alpha(s))=0,
\end{align*}
then the barrier inequality (\ref{barrier}) 
implies that $f(z)\le 0$ for $z\in [x\alpha(s)]$;
that is $[x\alpha(s)]\subset K$, a contradiction.
\qeds

\section{Remarks}

The following question was known in folklore in the 80's,
% Dave Berg remembers this
but it seems that in print
it was first mentioned by Michael Gromov \cite[6.B$_1\mathrm{(f)}$]{gromov:asymt-inv}.
We do not see any reason why it should be true, 
but we also cannot construct a counterexample.

\begin{thm}{Open question}
Let $\spc{U}$ be a complete length $\CAT0$ space and $K\subset \spc{U}$ be a compact set.
Is it true that $K$ lies in a convex compact set $\bar K\subset\spc{U}$?
\end{thm}

The question can  easily be reduced to the case when $K$ is finite;
so far it is not even known if any three points in a complete length $\CAT0$ space lie in a compact convex set.

One of the most beautiful applications of the Reshetnyak gluing theorem is given by Dmitri Burago,  Serge Ferleger,
and Alexey Kononenko \cite{burago-ferleger-kononenko1998-1,burago-ferleger-kononenko1998-2,burago-ferleger-kononenko1998-3,burago-ferleger-kononenko1998-4}.
They use it to study billiards; a short survey on the subject is written by Dmitri Burago \cite{burago-1998};
see also our book \cite{alexander-kapovitch-petrunin-CAT}.

We expect that \ref{lem:patch} can be extended to all curves, not necessarily local geodesics,
but the proof does admit a straightforward generalization.
