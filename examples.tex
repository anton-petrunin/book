%%!TEX root = the-examples.tex
\chapter{Subsets}

In this chapter we consider the subsets of the Euclidean space which inherit a curvature bounds.
We give a characterization for two and three-dimensional cases
and show that the description in higher dimensions is not as simple.

\section{Sets with smooth boundary}

Let $S$ be an hypersurface in the Euclidean space $\EE^m$.
Assume that the orientation on $S$ is given by a choice of orthonormal vector field $n$ at each point. 
Denote by $k_1\le \dots\le k_{m-1}$ the scalar field of principal curvatures at each point of $S$
with respect to $n$.

We say that $S$ is convex if $k_1\ge 0$ at any point and two-convex if $k_2\ge 0$ at any point.

\begin{thm}{Theorem}\label{thm:set-with-smooth-bry}
Let $K$ be a closed connected subset in $\EE^m$ equipped with the induced length metric.
Assume $K$ is bounded by a smooth hypersurface $S$ 
equipped the the orthonormal vector field $n$ which points outside of $K$.
\begin{subthm}{thm:set-with-smooth-bry:CBB}
$K\in \CBB{}{0}$ if and only if $S$ is convex.
\end{subthm}
\begin{subthm}{thm:set-with-smooth-bry:CBA}
$K\in \cCat{}{0}$ if and only $K$ is simply connected and $S$ is two-convex.
\end{subthm}
\end{thm}

\parit{Proof.}
Denote by $\Omega$ the interior of $K$; that is $\Omega=K\backslash S$.
Since $K$ is connected and its boundary is smooth then so is $\Omega$.

\parit{(\ref{SHORT.thm:set-with-smooth-bry:CBB}).} 
If $S$ is convex then $K$ is locally convex.
Since $K$ is connected, by Theorem~\ref{thm:local-global-convexity}, $K$ is convex.
It follows that induced length metric on $K$ is equal to the Euclidean metric. 
In particular (3+1)-point comparison holds for any quadruple of points in $K$.
Hence ``if''-part follows.

If $S$ is not convex then there is a triangle $[pxy]$ in $\EE^m$ such that two sides $[px]$ and $[py]$ lie in $\Omega$, but the side $[xy]$ does not completely lie in $K$.
Such a triangle can be found in the plane spanned by normal and the first principle directions at the point of $S$ where the first principle curvature  is below zero.

It follows that 
\[\dist{x}{y}{K}>\dist{x}{y}{\EE^m}.
\eqlbl{eq:xy_K>xy}\]

On the other hand $[px]$ and $[py]$ form geodesics in $K$ and $\EE^m$
and the $\eps$-neighborhoods of $p$ in $K$ and $\EE^m$ coincide for small $\eps>0$. 
In particular, $\mangle\hinge pxy_K=\mangle\hinge pxy_{\EE^m}$
and by hinge comparison (\ref{angle}) we have
\[\dist{x}{y}{K}\le\dist{x}{y}{\EE^m}.\]
The latter contradicts \ref{eq:xy_K>xy}.

\parit{(\ref{SHORT.thm:set-with-smooth-bry:CBA}).}
Assume $S$ is not two-convex.
Note that in this case there is a triangle $[xyz]$ in $\EE^m$ such that its sides lie in $\Omega$ completely but its convex hull stick out of $K$.

Indeed, choose a point $p\in S$ such that the first two principle curvatures  $k_1$ and $k_2$ are below zero.
Let $\Pi$ be the plane through $x$ spanned by the first two principle directions.
Choose a solid triangle in $\Pi$ which contains $p$ it its interior; we can assume that the sides of this triangle lie in $\Omega$.
Moving this triangle slightly in the direction normal to $S$ provides the needed triangle.

If $K\in\cCat{}{0}$,
then by Reshetnyak majorization theorem,
there is a short map $f$ from the convex hull $\triangle=\Conv(x,y,z)$ in $\EE^m$ to $K$ which preserves the sides of the triangle.
Note that $f$ is the identity map, a contradiction.

It remains to prove ``if''-part.
Since $K$ is simply connected we only need to show that $\curv_p K\le 0$ at any point $p\in K$.

If $p\in\Int K$ then it admits a neighborhood isometric to a subset of $\EE^m$, therefore $\curv_p K\le 0$.

Assume $p\in S=\partial_{\EE^m} K$ and $k_2(p)>0$.
Fix sufficiently small $\eps>0$ and set $K'=K\cap \cBall[p,\eps]$.

Consider the cooredinate system with the origin at $p$
and the principle directions and $n(p)$ as the coordinate directions.
For small $\eps>0$, the set $K'$ 
can be described as a subgraph
\[K'
=
\set
{(x_1,\dots,x_m)\in \cBall[p,\eps]}
{x_m\le f(x_1,\dots,x_{m-1}}.\]
Moreover, if $\eps$ is small, then for any fixed $x_1\in[-\eps,\eps]$ the function 
\[(x_2,\dots,x_{m-1})\mapsto f(x_1,\dots,x_{m-1})\]
is convex.

Fix a negative real value $\lambda<k_1$.
Given $s\in (-\eps,\eps)$,
consider the subgraphs 
\[K_s'
=
\set
{(x_1,\dots,x_m)\in \cBall[p,\eps]}
{x_m\le f(x_1,\dots,x_{m-1}}+\lambda\cdot (x_1-s)^2}.\]
Note that $K_s'$ is forms a convex set, 
\[K'=\bigcup_{s\in[-\eps,\eps]}K_s'\]
and for any choice $a<b<c$ we have
\[K'_b\supset K'_a\cap K'_c.\]

Note that each $K'_s$ forms a convex set in $\EE^m$ for any $s$.
The given a finite set of values $s_1<\dots<s_n$ consider the unions 
\[Q_i=K'_{s_1}\cup\dots\cup K'_{s_i}\]
equipped with the induced intrinsic metric.
Note that $Q_1=K'_{s_1}$ is a $\cCat{}{0}$ space as a convex subset in $\EE^m$.
Assuming that $Q_i\in \cCat{}{0}$, Reshetnyak's gluing theorem implies that $Q_{i+1}\in \cCat{}{0}$ 


\qeds



\section{Sturdy sets}

The following technical assumption on $K$ will help to say away from pathological examples.

\begin{thm}{Definition}
Let $K$ be a closed set in a metric space.
Denote by $\Omega$ the interior of $K$, equip both with the induced length metrics.
We say that the set $K$ is \emph{sturdy} if the induced lenght metric takes finite values and 
if the inclusion $\Omega\hookrightarrow K$ extends to an isometry
from the completion $\overline \Omega$ to $K$.
\end{thm}

As an example of sturdy set one can take a region of plane bounded by rectifiable closed simple curve.
A nonsturdy sets could be found among sets homeomorphic to closed disc 
which a spiraling piece of boundary, so that the intrinsic distance to from the end of the spiral to some (and therefor any) interior point is infinite. 

We are interested in the following general question.

\begin{clm}{}
For which sets $K$ in the Euclidean space 
the constructed space $\overline\Omega$ is $\cCat{}{0}$? 
\end{clm}

Note that if $K$ is strudy, the question can be reformulated in terms of its interior $\Omega$.

\begin{clm}{}
For which open connected open sets $\Omega$ in the Euclidean space 
the (2+2)-point comparison holds?
\end{clm}

Indeed, if the (2+2)-point comparison holds holds it $\Omega$ then the same is true for its completion. 
It remains to note that completion of length space is a length space.

For the inclusions $\phi\:\Omega\hookrightarrow\overline\Omega$ and $\psi\:\Omega\hookrightarrow\Closure\Omega$ there is unique continuous map 
$\theta\:\overline\Omega\to\Closure\Omega$ such that $\psi=\theta\circ\phi$,
but the map $\map$, in general is not onto and not injective.

The map $\theta$ is always length-preserving, and if $K=\Closure\Omega$ is study then it is an isometry.


\section{Model plane}

A finite union of triangles in $\Lob{2}{\kappa}$ will be also called \index{polygonal set}\emph{polygonal set}.


\begin{thm}{Theorem}\label{thm:polygon-CAT}
Let $P$ be a polygonal set in $\Lob2\kappa$ equipped with induced length metric.
Then $P\in\cCat{}{\kappa}$ if and ony if $P$ is $(2\cdot\varpi\kappa)$-simply connected. 
\end{thm}

\parit{Proof.} According to  generalized Hadamard--Cartan theorem~\ref{thm:hadamard-cartan-gen},
it is sufficient to check that 
\[\curv P\le \kappa.
\eqlbl{eq:curv(P)>=k}\]

By \ref{thm:warp-curv-bound:cbb:a}, a $\kappa$-cone over
a closed interval is $\cCat{}{\kappa}$.
By Reshetnyak's gluing theorem, 
the same holds for a $\kappa$-cone over
a collection of closed intervals.

The inequality \ref{eq:curv(P)>=k} follows 
since a neighborhood of any point in $P$
is isometric to a neighborhood of the tip of a $\kappa$-cone over
a collection of closed intervals.
\qeds


Let $[a_1\dots a_n]$
be a simple closed broken geodesic in the model plane $\Lob{2}{\kappa}$.
A closure  
of a bounded component of 
$\Lob{2}{\kappa}\backslash [a_1\dots a_n]$ 
is called \index{solid polygon}\emph{solid polygon}.

If $\kappa\le 0$ then there is unique polygon which is bounded by $[a_1\dots a_n]$,
while in the case $\kappa>0$
there are two such polygons,
one on each side of $[a_1\dots a_n]$.
Note that the half-spheres in $\Lob{}\kappa$ for $\kappa>0$ are included in the class of polygons.


\begin{thm}{Corollary}\label{thm:2-d-cba}
Let $P$ be a polygon in $\Lob{2}{\kappa}$ equipped with the length-metric. 
\begin{subthm}{thm:2-d-cba:k=<0} 
If $\kappa\le 0$ then $P\in\cCat{}{\kappa}$ 
\end{subthm}
\begin{subthm}{thm:2-d-cba:k>0}
If $\kappa> 0$ and then $P\in\cCat{}{\kappa}$ if and only if $P$ does not contain a hemisphere in its interior. 
\end{subthm}
\end{thm}

\parit{Proof; (\ref{SHORT.thm:2-d-cba:k=<0})}
Note that by construction $P$ is simply connected.
Therefore part (\ref{SHORT.thm:2-d-cba:k=<0})
follows from Theorem \ref{thm:polygon-CAT}.

\parit{(\ref{SHORT.thm:2-d-cba:k>0}).}
Since $\kappa>0$,
applying rescaling, we can assume that $\kappa=1$ and therefore $\Lob2\kappa=\mathbb{S}^2$.

Note that if $P$ contains a closed hemisphere it is not $(2\cdot\pi)$-simply connected;
in fact the $\partial_{\mathbb{S}^2}[\Conv(\partial_{\mathbb{S}^2} P)]$ forms a closed geodesic in $P$ with length smaller that $2\cdot\pi$.

If $P$ is not $(2\cdot\pi)$-simply connected
then it contains a closed geodesic $\gamma$ 
of length smaller than $2\cdot\pi$.
Note that $\gamma$ bounds a convex set $K$ in $\mathbb{S}^2$ 
and  $K\supset \partial_{\mathbb{S}^2} P$.
According to Hemisphere lemma (\ref{lem:hemisphere})
$\gamma$ lies outside of a closed hemisphere.
This hemisphere have to lie in the interior of $P$,
a contradiction.
\qeds



The following theorem 
is a partial case of a more general result proved by Bishop in \cite{bishop:jordan};
see also the discussion after Example 9.1.6 in the book of Burago--Burago--Ivanov \cite{BBI}.

\begin{thm}{Theorem}\label{thm:2d-bishop}
Let $\Omega$ be an open  proper simply connected subset in $\Lob2{\kappa}$.
If $\kappa>0$, assume in addition that $\Omega$ does not contain a closed halphsphere.

Consider $\Omega$ with the induced intrinsic metric.
Then the completion $\Omega^*$ of $\Omega$ is a $\Cat{}{\kappa}$ space.
\end{thm}

\parbf{Remark.}
Note that if $\Omega$ is bounded by a simple closed rectifiable curve its completion of $\Omega^*$ is isometric to the closure $\bar{\Omega}$  of $\Omega$ equipped with the length metric. 
In general, the space $\Omega^*$ admits natural length-preserving map in $\bar{\Omega}$, 
but 
this map might be not injective 
and its image might be proper subset of $\bar{\Omega}$.

%%%%%%%%+PIC???

\parit{Proof.}
Fix the points $x^1,x^2,x^3,x^4\in\Omega$.
Note that it is sufficient to prove that 
\begin{clm}{}\label{clm:2+2-Omega}
(2+2)-point $\kappa$-comparison holds for any four points 
$x^1, x^2, x^3, x^4$ in $\Omega$ 
equipped with the length metric.
\end{clm}

Fix $\eps>0$.
Connect $x^i$ to $x^j$ by a curve $\gamma_{ij}$ in $\Omega$;
such that 
\[\length\gamma_{ij}<\dist{x^i}{x^j}{\Omega}+\eps.\]

Note that there is a polygon $P\subset \Omega$ 
which contains all $\gamma_{ij}$.

By Theorem~\ref{thm:2-d-cba}, $P\in \cCat{}{\kappa}$.
In particular the points  $x^1,x^2,x^3,x^4$ satisfy (2+2)-point comparison in $P$.
By the construction,
\[\dist{x^i}{x^j}{P}\lege \dist{x^i}{x^j}{\Omega}\pm\eps\]
for all $i$ and $j$.

Since $\eps>0$ is arbitrary, \ref{clm:2+2-Omega} follows.
\qeds

\section{Three-dimensional Euclidean space}

\parbf{Two-convexity.}
Assume for a subset $A\subset \EE^3$
the following condition holds for any plane $W\subset\EE^3$.
If a closed curve $\gamma\:\SS^1\to A\cap W$ is null-homotopic in $A$
then $\gamma$ 
hull homotopic $\gamma\subset A\cap W$.
In this case we say that $A$ is \index{two-convex}\emph{two-convex}.

This definition is closely related to the one given by Gromov in \cite[\S\textonehalf]{gromov:SaGMC}, see also \cite{panov-petrunin:sweeping}.

Note that according to this definition any simply connected subset in $\EE^2$ is 2-convex.

Note that 
\begin{itemize}
\item Intersection of arbitrary collection of  two-convex sets is two-convex.
\item Interior of two convex set is two-convex.
\end{itemize}

The later makes possible to define \emph{two-convex hull} $\Conv_2 A$ of a set $A$
as the minimal two-convex set containing $A$.

\begin{thm}{Proposition}
Let $A$ be a subset in $\EE^3$
and $B=\Conv_2 A$ is its two-convex hull.
Assume $K\subset \EE^3$ be a convex set which does not intersect $A$.
Then $K\backslash B$ is convex.
\end{thm}

\parit{Proof.}
Assume contrary; 
that is the intersection $K\backslash B$ is a proper subset of $Z=\Conv(K\backslash B)$.

Consider the set $B'=B\backslash Z$. 
Note that $B'$ is a proper subset of $B$,
$B'\supset A$ and $B'$ is two convex.
that is $B$ is not the minimal two-convex subset containing $A$,
a contradiction.
\qeds


The following proposition 
describes a construction which was essentially given by Shefel in \cite{shefel}.
It produce $\Conv_2 \Omega$ for an open set $\Omega\subset\EE^3$.

\begin{thm}{Proposition}\label{prop:2-conv-construction}
Let $\Pi_1,\Pi_2\dots$ be an everywhere dense
sequence of planes in $\EE^3$.
Given an open set $\Omega$ consider 
the recucevly defined sequence of open sets 
$\Omega=\Omega_0\subset\Omega_1\subset\dots$ 
such that 
$\Omega_n$ is the union of $\Omega_{n-1}$ 
and all the bounded compontnets of 
$\EE^3\backslash(\Pi_n\cup \Omega_{n-1})$.
Then 
\[\Conv_2\Omega=\bigcup_n\Omega_n.\]

\end{thm}

\parit{Proof.}
Set 
\[\Omega'=\bigcup_n\Omega_n.\]
Note that $\Omega'$ is a union of open set, in particular it is open.

The inclusion $\Conv_2\Omega\supset\Omega'$
is evident.

It remains to show that $\Omega'$ is two-convex.
Assume contrary; 
that is, there is a plane $\Pi$ 
and a closed curve $\gamma\:\SS^1\to \Pi\cap \Omega'$ 
which is null-homotopic in $\Omega'$,
but not null-homotopic in $\Pi\cap\Omega'$.

By approximation we can assume that $\Pi=\Pi_n$ for a large enough  $n$ 
and that $\gamma$ lies in $\Omega_{n-1}$.
The latter contradicts the $n$-th step in the construction. 
\qeds

\begin{thm}{Corollary}
Two-convex hull of open set in $\EE^3$ is open.
\end{thm}

\parit{Proof.}
Observe that by Proposition~\ref{prop:2-conv-construction}
the two-convex hull of open set is a union of open sets.
\qeds




\begin{thm}{Shefel's Theorem}\label{thm:shefel}
Let $\Omega$ be an open simply connected 
subset of $\EE^3$.
Equip $\Omega$ with the induced length metric 
and let $\Omega^*$ be its completion.
Then $\Omega^*\in\cCat{}0$  
if and only if $\Omega$ is two-convex.
\end{thm}

Let us first formulate and prove a couple of corollaries of Shefel's theorem.

\begin{thm}{Corollary}
Let $\mathfrak{C}$ be a subset in $\EE^3$ 
which can be presented as intersection of some collection 
of open two-convex set.
Equip $\mathfrak{C}$ with induced length metric and denote by $\mathfrak{C}^*$ its completion.
Then any metric component of $\mathfrak{C}^*$ forms a $\cCat{}{0}$ space.
\end{thm}

\parit{Proof.}
Since the intersection of two-convex sets is two-convex,
we may assume that $\mathfrak{C}$ is formed by intersection of a decreasing sequence of open tow-convex sets $\Omega_1\supset\Omega_2\supset\dots$.

Note that if $\Omega_1\subset \Omega_2$ then there the inclusion
$\Omega_1\to \Omega_2$ induce a short map $\Omega_1^*\to \Omega_2^*$.
Therefore the completions of open two-convex sets containing $\mathfrak{C}$ form an inverse system.
The completion $\mathfrak{C}$ is isometric to the inverse limit of this system.

Since every $\Omega^*$ is $\cCat{}{0}$ we obtain that (2+2)-comparison holds for all quadrouple of points in $\mathfrak{C}$.
Hence the same holds for the completion $\mathfrak{C}^*$.
Passing to any metric component of $\mathfrak{C}^*$ we get a $\cCat{}{0}$ space.
\qeds

The following subcorollary is the main statement in Shefel's original paper \cite{shefel-graph}.

Let $U$ be an open set in $\RR^2$.
Recall, that a continuous function $f\:U\to\RR$ is called \index{saddle function}\emph{saddle} if for any linear function $\ell\:\RR^2\to\RR$ the difference 
$f-\ell$
does not have local maxima and minima in $U$.

\begin{thm}{Subcorollary}
Assume $U\subset\EE^2$ is a topological disc and $f\:U\to \RR$ is a saddle locally Lipschitz function then the graph
$z=f(x,y)$ in $\EE^3$ equipped with induced length metric is a $\cCat{}{0}$ space
\end{thm}

\parit{Proof.}
Since the function $f$ is Lipschitz, the metric on the graph
\[\mathfrak{C}=\set{(x,y,z)\in \EE^3}{z=f(x,y),\ (x,y)\in U}\]
is locally bi-Lipschitz to the euclidean metric on $U$.
In particular the completion $\mathfrak{C}$ and its completion $\mathfrak{C}^*$ has one metric component.

It remains to note that $\mathfrak{C}$ can be presented as intersection 
\[\mathfrak{C}
=
\Omega_1\cap\Omega_2\cap\dots\] 
of open two-convex sets 
\[\Omega_n
=
\set{(x,y,z)\in \EE^3}{z\lg f(x,y)\pm\tfrac1n,\ (x,y)\in U}.\]
\qedsf

The proof of Shefel's theorem goes in a few steps.
First we will prove it for the interiors of polytopes.
Recall, that a subset $K$ of $\EE^m$ is called \index{polytope}\emph{polytope} 
if it can be presented as a union of finite number of simplices.

Note any polytope admits a finite triangulation.
Therefore any polytope equipped with induced intrinsic metric 
forms a Euclidean polyhedral space as defined in \ref{def:poly}.

\begin{thm}{Lemma}\label{lem:poly-shefel}
Shefel's Theorem~\ref{thm:shefel} holds if the set $\Omega$ is formed by interior of a polytope.
\end{thm}

\begin{wrapfigure}{r}{20mm}
\begin{lpic}[t(-7mm),b(0mm),r(0mm),l(0mm)]{pics/polytope(1)}
\end{lpic}
\end{wrapfigure}

\parbf{Remark.}
Let $K$ be a polytope and $\Omega$ its interior,
both considered with induced intrinsic metric.
Typically the completion $\Omega^*$ 
is isometric to $K$.
However in general
we only have locally distance preserving map $\Omega^*\to K$;
it has to be onto, 
but in may be not injective. 
An example can be guessed from the picture.

\parit{Proof.}
If $\Omega$ is not two-convex then 
there is a plane $\Pi$ in $\EE^3$ 
which pass through a vertex $v$ of $K$ 
such that punctured neightborhood of $v$ in $\Pi$ lies in $\Omega$.
Choose a plane $\Pi'$ parallel and very close to $\Pi$ which cuts from the complement of $\Omega$ a little piramid $S$ with vertex $v$.
Consider a small triangle $\triangle$ in $\Pi'$ wich surrounds the base of $S$.
Note that $\triangle$ is a geodesic triangle in $\Omega^*$
for which the point-on-side comparison \ref{cat-monoton}
fails.
That is, $\curv_v\Omega^*\nleqslant0$. %???+PIC

Now let us prove the converse.
Since $\Omega$ is two-convex,
by Proposition~\ref{prop:stong-two-convex}, 
any point $v$ on the boundary of $K$ 
admits a conic neighborhood $U$ in $K$ 
such that the intersection $U\cap\Omega$ 
is formed by a finite collection of simply connected components.

It follows that any point $\Omega^*\backslash \Omega$ 
is locally isometric to a cone over spherical polygons.
Moreover since $\Omega$ is two-convex, 
each polygon does not contain a closed hemisphere in its interior. 
By Lemma ??? each of these spherical polygon is $\cCat{}{1}$. 
Therefore, by cone construction (\ref{thm:warp-curv-bound:cbb:a}) we get that $\Omega^*$ is locally $\cCat{}0$.
\qeds



\begin{thm}{Exercise}\label{ex:polygon-slices}
Let $m\ge 3$,
$K\subset \Lob{m}{\kappa}$ be a compact subset.
Assume that for any 2-dimensional subspace $W$ in $\Lob{m}{\kappa}$
the set $K\cap W$ is a polytope. 
Then $K$ is a polytope.
\end{thm}

\begin{thm}{Lemma}\label{lem:loc-concave}
Let $\Omega$ be an open set in $\Lob{3}{\kappa}$.
Then $\Conv_2\Omega$ is an open set.
 
Moreover, the set $\Lambda=\Lob{3}{\kappa}\backslash(\bar\Omega\cup\Conv_2\Omega)$
 is locally convex;
that is, for any $x\in\Lambda$ therew is $\eps>0$ such that the intersection
$\oBall(x,\eps)\cap(\Lambda\backslash\Omega^{(2)})$
is convex.
\end{thm}

\begin{thm}{Lemma}
Closed locally convex set in $\Lob{3}{\kappa}$ is convex. 
\end{thm}




\begin{thm}{Key lemma}\label{lem:key-shefel}
The two-convex hull of the interior of polytope in $\Lob{3}{\kappa}$
is an interior of a polytope.
\end{thm}

\parit{Proof.}
Let $K$ be a polytope in $\Lob{3}{\kappa}$;
denote by $\Omega$ the interior of $K$.

Denote by $F_1,\dots,F_m$ the faces of $K$.

Set $\Omega'=\Conv_2\Omega$ and let $K'$ be the closure of $\Omega'$.
Further, 
for each $i$, 
set $F'_i=F_i\backslash \Omega'$.
In other words, 
$F'_i$ is the subset of facet $F_i$ 
which remains on the boundary of $K'$.

From the construction of two-convex hull (\ref{prop:2-conv-construction})

\begin{clm}{}\label{clm:F'-convex}
$F'_i$ is convex subset of $F_i$.
\end{clm}

Further, since $\Omega'$ is two-convex,
we get the following.

\begin{clm}{}\label{clm:complement-of-F'-convex}
Each connected component of the complement $F_i\backslash F'_i$ is convex.
\end{clm}

Indeed, assume a connected component $A$ of $F_i\backslash F'_i$ fails to be convex.
Then there is a suppoiting line $\ell$ to $A$ touching $A$ at a single point in the interior of $F_i$.
Then one could rotate the plane of $F_i$ slightly arounf $\ell$ and move it parallel to cut a hat from the complement of $\Omega$.
The latter means that $\Omega$ is not two-convex, 
a contradiction.
\claimqeds

From \ref{clm:F'-convex} and \ref{clm:complement-of-F'-convex}, we get that 

\begin{clm}{}$F'_i$ is a convex polygon for each $i$.
\end{clm}

Consider the complement 
$\EE^3\backslash \Omega$ 
equipped with the length metric.
By construction of two-convex hull (\ref{prop:2-conv-construction}), 
the complement $L=\EE^3\backslash (\Omega'\cup K)$
is locally convex;
that is, any points of $L$ admits a convex neighborhood.

Summarizing (1)
$\Omega'$ is a 2-convex open set,
(2) the boundary $\partial\Omega'$ 
contains a finite number of polygons $F_i'$
and the remaing part is locally concave.
It remains to show that (1) and (2) imply that $\partial\Omega'$
is piecewise linear.

The proof of the last statement is left to the reader, 
Exercise~\ref{ex:polygon-slices} should help.
\qeds

\parit{Proof of \ref{thm:shefel}.}
Note that it is sufficient to show that
(2+2)-point $\kappa$-comparison holds for any
4 points $x^1,x^2,x^3,x^4\in\Omega$.

Fix $\eps>0$.
Choose a six broken lines connecting all the pairs of points $x^1,x^2,x^3,x^4$ such that length of each at most $\eps$ bigger than 
the distance between its ends.
Choose a polytope $P$ 
in $\Omega$ such that the interior $\Int P$ is simply connected 
and  it contains all these six broken.

Denote by $\Theta$ the two-convex hull of the interior of $P$.
According to Key Lemma (\ref{lem:key-shefel}) $\Theta$ is an interior of a polytope.
Therefore, by Lemma \ref{lem:poly-shefel}, $\curv \Theta^*\le \kappa$.
By Hadamard--Cartan theorem, the universal cover $\~\Theta^*$ of $\Theta^*$ is a $\cCat{}{0}$ space.
(Despit that $P$ is simply connected, the convex hull of 

Since $\Int P$ is simply connected, the embedding $\Int P\hookrightarrow \Theta^*$
admits a lifting $\iota\:\Int P\hookrightarrow \~\Theta^*$.

By construction $\iota$ almost preserves the distances between the points $x^1,x^2,x^3,x^4$;
namely 
\begin{align*}
\dist{\iota(x^i)}{\iota(x^j)}{\~\Theta^*}\gele \dist{x^i}{x^j}{\Int P}\pm\eps.
\end{align*}

Since $\eps>0$ is arbitrary and (2+2)-comparison holds in $\~\Theta^*$,
we get that (2+2)-comparison holds in $\Omega$ for $x^1,x^2,x^3,x^4$.

The statement follows since the quadruple $x^1,x^2,x^3,x^4\in\Omega$ is arbitrary.
\qeds

\parbf{Variations and generalization.}
A theorem analogous to ??? holds also in $\HH^3$, 
and a local version holds also in $\SS^3$.

As it will be shown in Section \ref{sec:with-bry},
an analog of Theorem~\ref{thm:shefel}
holds in higher dimensions assuming $\Omega$ has smooth boundary.
If the boundary is not smooth this is not longer true already in $\EE^4$;
an example can be found among intersections of two two-convex domains with smooth boundaries.



\section {Higher dimensional case}

Let $K$ be a strudy set in $\EE^m$ equipped with the induced length metric.
If $K\in \cCat{}{0}$ then it is easy to see that $K$ has to have two-convex interior.
Indeed ???.

However the following example shows that two-convexity is not a sufficient condition.

\begin{thm}{Example}
There are example of strudy two-convex set $K$ in $\EE^4$
which is not a $\cCat{}{0}$ space, if considered with the induced length metric. 


Moreover, an example can be found among the intersections $K=W\cap W'$, where 
\[W={(x,y,z,t)\in \EE^4}{z\ge -x^2-y^2}\]
and $W'$ is a rotaition of $W$.
\end{thm}

The following exercise is closely related.

\begin{thm}{Exercise}
Let $\Pi_1,\Pi_2$ be two planes in $\EE^4$ intersecting at single point.
Consider the complement $\Omega=\EE^4\backslash(\Pi_1\cup\Pi_2)$
equipped with induced length metric.
Denote by $\tilde\Omega^*$ the completion of universla cover of $\Omega$.
Show that $\Pi_1\perp\Pi_2$ if and only if $\tilde\Omega^*\in\cCat{}{0}$.
\end{thm}




\section {Manifolds-with-boundary}\label{sec:with-bry}
Let $M$ be a connected $m$-dimensional Riemannian manifold with possibly nonempty boundary $\partial M$.
%By a \emph{Riemannian manifold-with-boundary} $M$ of a given differentiability class, we mean $M$ has possibly nonempty boundary and has an extension across the boundary to a manifold $\overline{M}$ of the same class with empty boundary, such that  each boundary point of $M$ has a neighborhood $U$ in $\overline{M}$ coordinatized by $\R^n$ and for which $M\cap U$ is coordinatized by a halfspace of $\R^n$. The Riemannian metric of $\bar M$ is of differentiability class one degree lower. 
We equip $M$ with the length metric. 

For each point $p\in\partial M$,
let 
\[k_1(p)\le k_2(p)\le\dots\le k_{m-1}(p)\]
denote the principal curvatures
of $\partial M$ at $p$;
the sign convention is made so that
so the closed ball in Euclidean space has positive curvatures 
and the complement of the open ball has negative curvatures.

We say that $M$ has \emph{convex boundary} 
if $k_1(p)\ge 0$ for any $p\in\partial M$. 

\begin{thm}{Proposition}\label{prop:example-mnflds-with-bry:CBB}
Let $M$ be a Riemannian manifold with possibly nonempty boundary.
Then
$M\in\CBB{}{\kappa}$ 
if and only if $M$ has convex boundary 
and the sectional curvature of $M$ is at least $\kappa$.
\end{thm}

\parit{Proof; (\ref{thm:example-mnflds-with-bry:CBB}).}
To prove ``if'' part, we need to present a neighborhood of any given point $p$ in $M$ where the $\kappa$-comparison holds.

If $p$ lies in the interior of $M$, 
the existence of such neighborhood 
follows from the standard Toponogov comparison theorem.

Otherwise 
Without loss of generality, we may assume that $M$ 
is a subset of Riemannian manifold without boundary $N$, 
which has sectional curvature $\ge \kappa$.???

Note that sufficiently small spherical neighborhood of 
any point in $N$ is convex.???\qeds


Let $p\in\partial M$ and $\sigma$ is a sectional direction at $p$ to $\partial M$.
We say that $\sigma$ is a concave direction if the second fundamental form of $\partial M$ restricted to $\sigma$ is negative.

\begin{thm}{Theorem}\label{thm:example-mnflds-with-bry:CBA}
Let $M$ be a Riemannian manifold with possibly nonempty boundary.
Then
$\curv M\le \kappa$
if and only if the sectional curvature of $M$ is at most $\kappa$
and $\partial M$ has sectional curvature at most $\kappa$ 
in all concave sectional directions of $\partial M$.
\end{thm}

We say that $M$ has \emph{two-convex boundary} 
if $k_2(p)\ge 0$ for any $p\in\partial M$.

%\begin{thm}{Corollary}
%Let $M$ be a Riemannian manifold with possibly nonempty two-convex boundary.
%Assume sectional curvature of $M$ and $\partial M$ are at most $\kappa$.
%Then $\curv M\le \kappa$.
%\end{thm}
%
%\parit{Proof.}
%Follows directly from 
%Theorem~\ref{thm:example-mnflds-with-bry:CBA}.
%\qeds

\begin{thm}{Corollary}
Let $M$ be a Riemannian manifold with possibly nonempty two-convex boundary.
Then $\curv M\le{\kappa}$ 
if and only if the sectional curvature of $M$ is at most $\kappa$.
\end{thm}

\parit{Proof.}
If $\partial M$ is two-convex, 
it has no points with concave sectional directions.
It remains to apply Theorem~\ref{thm:example-mnflds-with-bry:CBA}.
\qeds

\begin{thm}{Definition}\label{def:mwb-segment}??? A geodesic of $M$ consists of  \emph{interior
segments}, by which we mean nonconstant open segments  that have zero acceleration in $M$ but
may include points of $\partial M$;  \emph{boundary segments}, nonconstant geodesics of $\partial M$ whose acceleration in $M$ is nonzero (necessarily normal to $\partial M$) on a dense open subset; \emph{switch-points}, where the geodesic switches between interior and boundary segments;  and \emph{chatter-points}, namely accumulation points of switch-points. 
\end{thm}

Examples of single geodesics are easily constructed that have a Cantor set of positive measure of chatter-points.
The existence of chatter-points makes the variational theory of geodesics
difficult to analyze.

%\begin{thm}{Theorem (Regularity of geodesics)}\label{thm:m-w-b-geodesic-''}
%Let $M$ be a Riemannian 
%manifold-with-boundary.
%Let $\gamma$ be a geodesic of $M$ with unit-speed parameter $s$.  Then  after  passing to coordinates, $\gamma''(s)$ exists everywhere except at switch-points, where $\gamma''(s)$ exists one-sidedly.  At chatter-points, $\gamma''(s)$ vanishes.
%\end{thm}

\begin{thm}{Lemma}\label{lem:m-w-b-geodesic-'}
Let $M$ be a Riemannian manifold-with-boundary, and $\gamma$ be a geodesic of $M$. Then:
\begin{subthm}{}
$\gamma$ is continously differentiable.
\end{subthm}
\begin{subthm}{}
The $2$-plane in $T_{\gamma(t)}N$  normal to $\partial M$ and containing $\gamma'(t)$ is an osculating plane for $\gamma$ at $t$.
\end{subthm}
\end{thm}


\parit{Proof.}
Let  $\gamma$ be a unit-speed geodesic of $M$ with $\gamma(t)=p\in\partial M$.  
At an endpoint of $\gamma$, we take $\gamma'$ to mean the corresponding one-sided derivative.

\begin{clm}{}\label{clm:1-sided-geo-mwb}
If $\gamma$ is one-sidedly differentiable at $t$, then $\gamma$ is differentiable at $t$.
\end{clm}

\begin{clm}{}\label{clm:diff-geo-mwb}
$\gamma$ is differentiable.
\end{clm}
Let $N$ be a Riemannian manifold without boundary, of the same dimension as $M$, in which  a neighborhood of $p$ in $M$ is isometrically embedded. We may take  $N=\cup \,H_u$, $-\epsilon_2<u<\epsilon_2$, for a family of equidistant hypersurfaces $H_u$ with unit normal field $v$,  where each $H_u$ intersects $\partial M$ transversely with $v$ pointing out of $M$, and the 
principal curvatures of the $H_u$ are positive toward $-v$. 
%Letting $H_0$ pass through $p$, we may take $H_0$ to lie in the exponential image of a  neighborhood of $o$ in a sphere through $o$  in $T_p M$. 
Set $H_+=\cup_{\,u>0} \,H_u$.

Projection 
$$\map[2]\:H_+ \to H_0$$
 along $v$ is defined and length-nonincreasing.  Moreover, the image of $H_+\cap M$ lies in $H_0\cap M$ because $v$ points into $M$ at points of $\partial M$.

Suppose $\gamma$ is not one-sidedly differentiable, say $\gamma^+(t)$ does not exist.  Then for $i=1,2$ there are distinct unit vectors $v_i \in T_p\partial M$, and points $\gamma(t_{ij})$, $t_{ij}>0$, $t_{ij}\to t$, such that the directions of the preimages of $\gamma(t_{ij})$ in a normal coordinate neighborhood in $N$ of $p$  converge to $v_i$.

Now choose the family $H_u$ so that $v_1$ and $v_2$ lie on opposite sides of $H_0$ at $p$.  Then $\gamma$ may be shortened in $M$ by applying the projection $\map[2]$ to a subsegment of $\gamma$ that lies in $H_+$ and runs between two points in $H_0$.  This contradiction shows $\gamma^+(t)$ exists. By \ref{clm:1-sided-geo-mwb}, $\gamma'(t)$  exists.

\begin{clm}{}\label{clm:C^1-geo-mwb}
$\gamma$ is continuously differentiable. 
\end{clm}

 Suppose $\gamma'$ is not continuous at $t$. Then $\gamma'(t)\in T_p\partial M$. Without loss of generality, there is a sequence $t_j>0$, $t_j\to t^+$, such that  $\gamma'(t_j)\to v\ne \gamma'(0)$.  Then also $v\in T_p\partial M$. In the notation of Claim \ref{clm:diff-geo-mwb}, choose the family $H_u$ so that $\gamma'(t)$ and $v$ lie on opposite sides of $H_0$ at $p$, and $\gamma'(t))$ points into $H_+$.  Then $\gamma$ has a subsegment in $H_\epsilon^+$ that runs between two points in $H_\epsilon$ for some $\epsilon>0$.   
 Thus $\gamma$ may be shortened in $M$ by projecting this subsegment to $H_\epsilon$.  The claim follows by contradiction.

  
\begin{clm}{}\label{clm:C^1-geo-mwb}
 The $2$-plane in $T_{\gamma(t)}N$ that contains $\gamma'(t)$ and is normal to $\partial M$ is an osculating plane for $\gamma$ at $t$.
\end{clm}

Recall that for any differentiable curve $\alpha$ in $N$, a $2$-plane $Q\subset T_{\alpha(t)}N$ is called  an \emph{osculating plane} for $\alpha$ at $t$ if for each $u$ in a neighborhood of $t$ in the parameter interval of $\alpha$, there is a  $2$-plane $Q_u$ containing $\alpha'(t)$ and $\exp _{\alpha(t)}^{-1} (\alpha(u))$ such that $Q=\lim_{u\to t}Q_u$.  If $Q$ exists and $\alpha$ is not a geodesic on a neighborhood of $t$,  then $Q$ is uniquely determined. 

Fix directions $u_1\in T_p\partial M$  orthogonal to $\gamma'(t)$, and $u_2\in  (T_p\partial M)^\perp$.  Let $R$ be the oriented $3$-plane  spanned by $\gamma'(t)$, $u_1$ and $u_2$.  Denote by $P_1(\theta,\varphi)$ and $P_2(\theta,\varphi)$, the two hyperplanes in $T_pN$ that are spanned by $R^\perp$ and  the $2$-plane in $R$ whose intersection with  the $\,\gamma'(t)\,u_1\,$-plane makes angle $\theta\in [0,\pi)$ with $\gamma'(t)$ and whose angle with $u_2$ is $\varphi\in (0,\pi/2)$.
For $i=1,2$, let $S_i(\theta, \varphi, k)$ be the $(n-1)$-sphere of radius $k$ in $T_pN$ passing through $o$, and  tangent to $P_i(\theta,\varphi)$ on the side away from $u_2$.  Consider the exponential images $H_i(\theta, \varphi, k)$  of  neighborhoods of $o$ in $S_i(\theta, \varphi, k)$.  
In the notation of Claim \ref{clm:diff-geo-mwb}, length-decreasing projection implies that on some open interval about $t$, $\gamma$ lies in $H_i(0, \varphi, k)\cup H_i(0, \varphi, k)^+$.  Otherwise, $\gamma(t+\epsilon)$ would enter the side with positive curvatures of $H_i(\theta, \varphi, k)$ for some $\theta<0$.  Then $\gamma|[t, t+\epsilon]$ would lie in that side, in contradiction to $\gamma'(t)$ being tangent to $H_i(0, \varphi, k)$.  It follows that on some open interval about $t$, the exponential preimage in $T_pN$ of $\gamma$  does not enter the open side with positive curvatures of $S_i(0,\varphi, k)$.

Now the claim follows by letting $\varphi\to 0$ and $k\to\infty$.
\qeds

\begin{thm}{Lemma}\label{lem:abs-cont}
Let $\gamma$ be a geodesic of a Riemannian manifold-with-boundary $M$. Then after passing to coordinates, 
$\gamma'$  is  locally Lipschitz continuous.  
\end{thm}

\parit{Proof.}
Specifically,  we extend $M$ to a manifold $N$ without boundary, and regard a  coordinate neighborhood in $N$ of $p=\gamma(t)$ as a neighborhood  in $\R^m$.

\begin{clm}{}\label{clm:pointwise-k-bound}
The pointwise arc-chord curvature  of  $\,\gamma$ (Definitions \ref{def:arc-chord}, \ref{ptwise-arc-chord}) is bounded above.
\end{clm}
It suffices to reparameterize $\gamma$ by Euclidean arc-length, say $\gamma\:[a,b]\to \R^n$, and we do so for notational convenience. 
%For a fixed $s \le
%\pi/\sqrt{k}$, the curvature $k$ of a $k$-curve $\sigma$ in $\Lob2\kappa$ is determined by the ratio $s/r$ of   length   to chord-length (i.e. \emph{arc/chord} ratio) of a segment of $\sigma$ of length $s$.  

As lengths of subarcs of $\gamma$ approach $0$ and corresponding arc/chord ratios approach $1$, an upper bound on arc/chord curvature may be deduced from a lower bound on the maximum length for which the ratio lies in a given small interval $[1,1+\delta]$.    Thus we prove the claim by showing,  given $t\in [a,b]$,  that for some  $\epsilon>0$, there is a uniform upper bound on the arc/chord ratio of $\gamma|[t_1,t_2]$ if  $a\le t_1\le t\le t_2\le b$, $|t_1-t_2|\le \epsilon$. 

We may suppose  $\gamma(t)$  does not lie in a line segment $\gamma|[t'_1,t'_2]$ where  $t_1< t'_1< t< t'_2$ or $t_1=t'_1= t< t'_2$ or $ t'_1< t= t'_2=t_2$,  since otherwise there is nothing to prove. In particular, $\gamma(t)\in\partial M$. We may also assume $\gamma(t_1)$ and $\gamma(t_2)$ lie in $\partial M$. Indeed, if one or both of $\gamma(t_1)$ and $\gamma(t_2)$ lie in $M-\partial M$, consider the maximal segment $\overline{\gamma}$ of $\gamma|[t_1,t_2]$ with endpoints on $\partial M$.  Since $\gamma$ is $C^1$, then by the triangle inequality for the polygonal curve consisting of the segments that extend $\overline{\gamma}$  to $\gamma$ together with the chord of $\overline{\gamma}$,  the arc/chord ratio of $\gamma$ is at most that of $\overline{\gamma}$.

% It follows that if   $|t_1-t_2|$ is sufficiently small, then  $\tilde\alpha$ has  geodesic curvature $\le k+\epsilon$, where  $k$ is an upper bound for the principal curvatures of $\partial M$  in a neighborhood of $\gamma(t_0)$.  


By Lemma \ref{lem:m-w-b-geodesic-'}, the 
Riemannian osculating plane of $\gamma$ at $t$ is defined and normal to $\partial M$.
%, i.e. as $t_1,t_2$ approach $t$, $Q$ approaches the $2$-plane that contains $\gamma'(t)$ and is normal to $\partial M$ at $p$.  
%Moreover, these osculating planes vary continuusly with $t$. 
Then for $\delta\in (0,\pi/2)$, we may choose $\epsilon>0$  such that  the  angle  in $\R^m$ between the normal to $\partial M$ and the $2$-plane $Q$ through points $\gamma(t_1), p=\gamma(t), \gamma(t_2)$ is bounded above by $\pi/2-\delta$, uniformly in $t$. Thus if $\tilde\alpha$ is the curve in $Q$  whose image is the intersection curve $\partial M\cap Q$ 
between $\gamma(t_1)$ and $\gamma(t_2)$, then  $\tilde\alpha$ has  geodesic curvature bounded in absolute value by some uniform constant $A>0$.

Let $\alpha$ be the geodesic of $M\cap Q$ between $\gamma(t_1)$ and $\gamma(t_2)$. 
  In $Q$, $\alpha$ is a $C^1$ convex curve that runs along the boundary of the convex hull of $\tilde\alpha$.  Therefore the total turn of  the tangent line of any subarc of  $\alpha$ with endpoints on $\partial M$  is not more than the total absolute turn of the tangent line of the subarc of  $\tilde\alpha$ with the same endpoints. This total absolute turn is not more than the total turn for a subarc $\sigma'$,  of the same length as $\tilde\alpha$, of a $k$-curve $\sigma$ of constant curvature $k=A$ in $\Lob2\kappa$.  

Consider a hinge whose sides are tangent  at $t_1$ and $t_2$ to $\alpha|[t_1,t_2]$; and an analogous hinge whose sides are tangent  at the endpoints of $\sigma'$ to a subarc $\sigma'$ of $\sigma$ of the same length as $\alpha|[t_1,t_2]$.  We have just seen that the hinge angle is not smaller for $\alpha|[t_1,t_2]$ than for $\sigma'$.  Therefore the 
chord-length of $\alpha|[t_i,t_2]$ is not smaller than  the chord-length of $\sigma'$.  It follows that there is a uniform upper bound on the arc/chord ratio of $\alpha|[t_1,t_2]$ if   $a\le t_1\le t\le t_2\le b$, $|t_1-t_2|\le \epsilon$.

By assumption, $\gamma$ is a Riemannian-shortest curve in $M$ between $\gamma(t_1)$ and $\gamma(t_2)$. The absolute difference of the Riemannian and Euclidean lengths of $\gamma|[t_1,t_2]$ 
 is at most a uniform constant multiple of the former. Hence the (Euclidean) length of $\gamma|[t_1,t_2]$
  is at most a uniform constant  multiple of the length of $\alpha|[t_1,t_2]$.   Therefore the arc/chord ratio of $\gamma|[t_1,t_2]$ is uniformly bounded above, as required.

\begin{clm}{}\label{clm:global-k-bound}$\gamma'$ is Lipschitz continuous. 
\end{clm}
By \ref{clm:pointwise-k-bound} and globalization of arc-chord curvature, 
\ref{subthm:local-global-base-angle}, the angle $\theta$ in $\R^n$ between $\gamma^+(t_1)$(respectively $-\gamma^-(t_2)$)  and the endpoint chord of $\gamma|[t_1,t_2]$ is at most equal to the angle between a circular arc of curvature $k$ and length $t_2-t_1$  in $\R^2$ and  its endpoint chord.  But $\theta$ is also the angle in $\R^n$ between the tangent lines to $\gamma$ at $t_1$ and $t_2$.  The claim follows.
\qeds

Now we are ready for one of our main curvature estimates. We formulate it in terms of an 
isometric imbedding of $M$ in some Euclidean space $\EE^n$. 
%It would be desirable
%to have a more direct link with the natural intrinsic invariants, the sectional
%curvature of $M$, the normal curvature of $B$ in $M$, and the injectivity radii of $M$
%and $B$, but our method has the advantage that it gives the estimate in terms of
%a single number. 

Say a positive number $r$ is a \emph{tubular
radius} for $M$ in $\EE^n$ if every point at distance $r$ or less from $M$ is the center of a
closed ball $\EE^n$ meeting $M$ at a single point. Then the principal curvatures of $M$, as well as  those of $\partial M$ belonging to a normal
vector whose $M$-component is outward from the interior of $M$, are bounded
above by $1/r$. Conversely, if we take an upper bound $k$ of such principal
curvatures, then
$1/k$ will be a tubular radius for a sufficiently small region of $M$.


\begin{thm}{Lemma}\label{lem:diff-inequality} Let $r = 1/k$ be a tubular radius for $M$, and let $\gamma$ and $\sigma$ be
geodesics in $M$ having speed $\le 1$. Let $f(r) = |\gamma(s) - \sigma(s)|$ be the
displacement in $\EE^n$ between corresponding points. Then 
%except at the countably many points where $f''$ fails to exist, we have a differential inequality
$$f'' \ge -k^2f,$$ with strict inequality where $f> 0$.
\end{thm}


%
%\parit{Proof of \ref{thm:m-w-b-geodesic-''} (Regularity of geodesics).}
%\begin{clm}{}\label{clm:chatter-''-0}
%At a chatter-point $\gamma(t_0)$ of $\gamma$, the normal curvature $k_0$ of $\,\partial M$ in the direction of $\gamma'(t_0)$ vanishes.
%\end{clm}
%
%If $k_0>0$, we could shorten $\gamma$ by replacing $\gamma|[t_0-\epsilon,t_0+\epsilon]$ by an interior geodesic of $M$ for $\epsilon$ sufficiently small. If $k_0<0$,  we could shorten $\gamma$ by replacing $\gamma|[t_0-\epsilon,t_0+\epsilon]$ by its normal projection to $\partial M$  for $\epsilon$ sufficiently small. Therefore $k_0=0$.
%
%\begin{clm}{}
%Define $k(t)$ on open boundary segments of $\gamma$ to be the normal curvature of $\partial M$ in the direction of $\gamma'(t)$, and set $k(t)=0$ otherwise.
%Let $x_n$ be the distance from $\partial M$, and  $x_i$, $i < n$, be coordinates on $\partial M$, extended to be
%constant on geodesics of $M$ normal to $\partial M$. Denote the Christoffel symbols of $M$ in these coordinates  by $\Gamma_{ijl}$. Then $\gamma$ satisfies, in an integral sense, the equations
%$$ {x_k}'' \ =\  -\sum_{i,j}x'_ix'_j\Gamma_{ijk},\,\, l< n,
%\eqlbl{eq:geo-eq-tangent}$$
%$$  {x_n}'' \ =\  \ -\ k\ - \sum_{i,j < n}x'_ix'_j\Gamma_{ijn}.
%\eqlbl{eq:geo-eq-normal}$$
%  \end{clm}
%
%On open segments of $\gamma$ in the interior of $M$, we have $k=0$, so  equations \ref{eq:geo-eq-tangent} and  \ref{eq:geo-eq-normal} are the geodesic equations of $\gamma$. On open boundary segments of $\gamma$, we have $x_n=0$, so equations \ref{eq:geo-eq-tangent} and \ref{eq:geo-eq-normal} again hold  since the coordinate matrix of the second fundamental form of $\partial M$ with
%respect to $\partial / \partial x_n$ is $-\Gamma_{ijk}$ where $i, j < n$.
%
%Given the absolute
%continuity of $\gamma'$ and the normality of $\gamma''$ on $B$, it follows that equations \ref{eq:geo-eq-tangent} and \ref{eq:geo-eq-normal} hold everywhere in an
%integral sense. Then \ref{eq:geo-eq-tangent} holds everywhere since the right-hand side is continuous. At the countably many switch points, \ref{eq:geo-eq-normal}  can be
%interpreted as being valid in the limit from either side. Except at switch-points, the right side of \ref{eq:geo-eq-normal}  is
%continuous, and hence the acceleration of $\gamma$ exists. In
%particular, the acceleration exists and is $0$ whenever $k = 0$. 

\begin{thm}{Lemma}\label{lem:mnflds-with-bry:CBA}
Let $M$ and $N$ be  Riemannian manifolds with possibly nonempty boundary, where $M$ is isometrically immersed in $N$.  Then the difference between distances in $M$ and in $N$ respectively is on the order of the cube of either one.
\end{thm}
%
\begin{thm}{Lemma}\label{lem:mnflds-with-bry:CBA}
Let $M$ be a Riemannian manifold with possibly nonempty boundary.
Then every point of $M$ has a $\cCat{}{}$ neighborhood.
\end{thm}

\parit{Proof of \ref{thm:example-mnflds-with-bry:CBA}.}???
\qeds



\section{Convex hypersurfaces in Riemannian manifolds}

The following theorem provides an other source of examples of $\CBB{}{}$-spaces.

\begin{thm}{Theorem}\label{thm:buyalo} 
Let $M$ be an $m$-dimensional Riemannian manifold 
with sectional curvature $\ge \kappa$ 
and $F\subset M$ be a convex hypersurface equipped with the length-metric;
that is, $F$ bounds a convex set. 
Then $F\in\CBB{m-1}\kappa$.
\end{thm}

This theorem, in a slightly weaker form was proved in by Buyalo in \cite{buyalo:convex-surface}, 
and then its statement was made more exact in \cite{akp}.

\begin{thm}{Lemma}\label{lem:smoothing}
Let $M$ be a Riemannian manifold and $f\:M\subto\RR$ be a locally Lipshits $\lambda$-concave subfunction. 
Then there is a sequence of smooth subfunctions $f_n\:M\subto\RR$ and $\lambda_n\in\RR$ such that $f_n\to f$ and $\lambda_n\to \lambda$ as $n\to\infty$ and each $f_n$ is $\lambda_n$-concave.
\end{thm} 

This lemma is a slight generalization of \cite[Theorem 2]{greene-wu}
by Greene and Wu;
it can be proves the same way.
%???Should we include the proof???

\parit{Proof of theorem \ref{thm:buyalo}}. 
Without loss of generality one can assume that 
\begin{enumerate}
\item \label{k>=-1}$\kappa\ge -1$, 
\item $F$ bounds a compact convex set $C$ in $M$, 
\item there is a function $h$ defined in a neighborhood of $C$ such that $h''\le -2$ and $|h(x)|<1/10$ for any $x\in C$,
\item \label{property:unique} there is unique geodesic between any two points in $C$. 
\end{enumerate}
(If not, rescale and pass to the boundary of the convex piece cut by $F$  from a small convex ball centered at $x\in F$, taking $h=-10\cdot\dist[{{}}]{x}{}{}$.)

Consider the function $f=\dist{F}{}{}$.
By Rauch comparison 
(as it is stated in Petersen's book \cite[11.4.8]{petersen:RiemGeom}), 
for any unit-speed geodesic $\gamma$ in the interior of $C$, $(f\circ\gamma)''$ is bounded in the barrier sense by the corresponding value in the model case --- when $M\iso\Lob2{-1}$ and $F$ is a geodesic.  
In particular,
\[(f\circ\gamma)''\le f\circ\gamma.\]


Therefore $f+\eps\cdot h$ is $(-\eps)$-concave in 
$\Omega_\eps=f^{-1}((0,\eps))\cap C$.
Take 
\[K_\eps\z=f^{-1}([\tfrac{1}{3}\cdot\eps,\tfrac{2}{3}\cdot\eps])\cap C.\]
Applying Lemma~\ref{lem:smoothing}, we can find a smooth $(-\tfrac\eps2)$-concave function $f_{\eps}$ which is arbitrarily close to $f+\eps\cdot h$ on $K_\eps$ and which is defined on a neighborhood of $K_\eps$. 
Take a regular value $\theta_\eps\approx\tfrac{1}{2}\cdot\eps$ of $f_\eps$. (In fact one can take $\theta_\eps=\tfrac{1}{2}\cdot\eps$, but it requires a little work.) 
Since $|(h|C)|<1/10$, the level set $F_\eps=f_{\eps}^{-1}(\theta_\eps)$ will lie entirely in $K_\eps$.
Therefore $F_\eps$ forms a smooth closed convex hypersurface.

Let us denote by $\rho$ and $\rho_\eps$ the length-metrics on  correspondingly $F$ and $F_\eps$.
By the Gauss formula, $(F_\eps,\rho_\eps)$ has curvature $\ge\kappa$.
Further, $F_\eps$ bounds a compact convex set $C_\eps$ 
and $F_\eps\to F$, $C_\eps\to C$ in Hausdorff sense as $\eps\to 0$. 
By property (\ref{property:unique}), the restricted metrics from $M$ to $C$ and to $C_\eps$ are length-metrics.
Thus, $C_\eps$ is an Alexandrov space with $F_\eps$ as boundary, that converges in Gromov--Hausdorff sense to $C$.  It follows from ???  that $(F_\eps,\rho_\eps)$  converges in Gromov--Hausdorff sense to $(F,\rho)$.
Therefore $(F,\rho)$ is an Alexandrov space with curvature $\ge \kappa$.\qeds

\noi{\bf Remark.} 
We are not aware of any proof of theorem \ref{thm:buyalo} which is not based on the Gauss formula. 
Finding such a proof would be interesting on its own, and also could lead to the generalization of theorem \ref{thm:buyalo} to the case when $M$ is an Alexandrov space.



\section{Comments and open problems}

\begin{thm}{Shefel's conjecture}
Any saddle hypersurface in $\RR^3$ equipped with the length-metric has curvature $\le 0$ at any point.
\end{thm}

A surface $S$ in a metric space.
We say that $S$ is saddle if for any convex function $f\:\spc{X}\to\RR$,
the restriction $f|S$ has no local maxima. 

\begin{thm}{Generalized Shefel's  conjecture}
Any 2-dimensional saddle surface in a $\Cat{}{\kappa}$ 
equipped with the length-metric has curvature 
$\le \kappa$ at any point.
\end{thm}

It would be esting to get an analog of the flag condition for other simplices 
which (not necessary right-angled).
Such a condition for some spherical Coxeter simplex
could resolve the following problem; see \cite{panov-petrunin} for more details. 

\begin{thm}{Braid space}
Consider $\CC^n$ with coordinates $z_1,z_2,\dots,z_n$.
Let us remove from $\CC^n$ the complex hyperplanes $z_i=z_j$, for all $i\ne j$,
pass to the universal caver and consider the completion $\spc{B}_n$ 
of the obtained space.

Is it true that $\spc{B}_n\in\Cat{}{0}$ for any $n$.
\end{thm}

The above question has an affirmative answer for $n\le 3$ and open for all $n\ge 4$.

\section{Exercises}



\begin{thm}{Exercise}
Show that completion of any flat simply connected 2-dimensional manifold is a $\cCat{}0$ space. 

In particular, the set in $\EE^2$ bounded by closed simple rectifiable curve equipped with induced length metric is  a $\cCat{}0$ space. 
\end{thm}



\begin{thm}{Exercise}
Let $S$ be a 2-dimensional saddle surface in $\EE^3$ which is homeomorphic to a disc.
Show that any closed curve in $S$
of length $\ell$
bounds the area at most $c\cdot\ell^2$, 
for some universal constant $c$.
\end{thm}




