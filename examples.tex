%%!TEX root = the-examples.tex
\chapter{Examples}

\section{Polyhedral spaces}

\begin{thm}{Definition}\label{def:poly}
A length space $\spc{P}$ is called
is  \emph{piecewise $\Lob{}\kappa$} (briefly $\spc{P}\in \PM{}\kappa$)
if it admits a finite triangulation $\tau$ 
such that arbitrary simplex $\sigma$ in $\tau$ is isometric to a simplex in the model space $\Lob{m}{\kappa}$, 
where $m$ is the dimension of $\sigma$.

\begin{subthm}{}
The piecewise $\Lob{}1$ spaces or $\PM{}1$ spaces will be 
will be also called 
\index{spherical polyhedral space}\emph{spherical polyhedral spaces} and denoted as $\PS{}$;
\end{subthm}

\begin{subthm}{}
The piecewise $\Lob{}0$ spaces or $\PM{}0$ spaces will be 
will be also called 
\index{Euclidean polyhedral space}\emph{Euclidean polyhedral spaces}
and denoted as $\PE{}$;
\end{subthm}

\begin{subthm}{}
The piecewise $\Lob{}{-1}$ spaces or $\PM{}{-1}$ spaces will be 
will be also called 
\emph{hyperbolic polyhedral space}\index{hyperbolic polyhedral space}
and denoted as $\PH{}$.
\end{subthm}

If we do not want to specify $\kappa$, we can say that $\spc{P}$ is a \emph{polyhedral space}.

By a \index{triangulation of a polyhedral space space}\emph{triangulation of a polyhedral space} we will understand the triangulation as in the definition. 

\end{thm}

Note that according to the above definition,
any $\PM{}{\kappa}$ space is compact,
but most of the statements below admit straightforward generalization to complete locally $\PM{}{\kappa}$ spaces;
the later class of spaces includes so called 


The dimension of a $\PM{}{\kappa}$ space $\spc{P}$
is defined as the maximal dimension of the simplex 
in one (and therefore any) triangulation of $\spc{P}$.
The class of $\PM{}{\kappa}$ spaces 
with dimension $m$ and $\le m$ will be denoted 
as $\PM{m}{\kappa}$ and $\PM{\le m}{\kappa}$.


\parbf{Links.}
Let $\spc{P}\in \PM{}\kappa$
and let $\sigma$ be a simplex of a triangulation $\tau$ of $\spc{P}$.

The simplices which contain $\sigma$
form an abstract simplicial complex called by link of $\sigma$ 
which will be denoted as $\Link_\sigma$.
If $m$ denotes the dimension of $\sigma$
then the set of vertices of $\Link_\sigma$
is formed by the $(m+1)$-simplices which contain $\sigma$;
the set of its edges are formed by the $(m+2)$-simplices which contain $\sigma$ and so on.

The link $\Link_\sigma$
can be identified with the subcomplex of $\tau$ formed by all the simplices $\sigma'$ such that $\sigma\cap\sigma'=\emptyset$, 
but both $\sigma$ and $\sigma'$ are faces of the same simplex.

The link of any simplex 
cares natural structure of spherical polyhedron. %???WHY???
We will always consider the link with this metric.

\parbf{Tangent space and space of directions.}
Let $\spc{P}\in \PM{}\kappa$ and $\tau$ be a triangulation of $\spc{P}$.
If a point $p\in \spc{P}$ 
lies in the interior of a $\kay$-simplex $\sigma$ of $\tau$ 
then the tangent space $\T_p\spc{P}$
can be naturally isometrically identified with 
\[\EE^\kay\times(\Cone\Link_\sigma).\]
Equivalently, the space of directions $\Sigma_p$
can be isometrically identified with the 
$\kay$-th spherical suspension over $\Link_\sigma$;
that is, 
\[\Sigma_p\iso\Susp^{\kay}(\Link_\sigma).\]

If $\spc{P}\in \PM{m}\kappa$
then $\Sigma_p\in\PS{\le m-1}{}$ for any $p\in \spc{P}$;
that is, the space of directions of $m$-dimensional polyhedral space 
at any point is an spherical polyhedral space of dimension at most $m-1$. 

In particular, 
for any point $p$ in the interior of $\sigma$,
the isometry class of $\Link_\sigma$ together with $\kay=\dim\sigma$
determines the isometry class of $\Sigma_p$ 
 and the other way around.

Further a neighborhood of $p$ is isometric to a neighborhood of the tip of the $\kappa$-cone over $\Sigma_p$.
In fact it is true that any compact length space for which any point has a neighborhood 
which isometric to a neighborhood of the tip of the $\kappa$-cone is a $\PM{}\kappa$ space;
see \cite{lebedeva-petrunin-poly}.

\begin{thm}{Theorem}\label{thm:poly-CBB} Let $\spc{P}\in\PM{m}\kappa$ and $\tau$ be a triangulation of $\spc{P}$.
Then $\spc{P}\in\CBB{}{\kappa}$ if and only if the following conditions hold.

\begin{subthm}{} $\tau$ is \emph{pure}; 
that is, any simplex in $\tau$ is formed by a face of some simplex of dimension exactly $m$. 
\end{subthm}

\begin{subthm}{thm:poly-CBB:m-1}
The link of any simplex of dimension $m-1$ is formed by single point or two points.
\end{subthm}

\begin{subthm}{thm:poly-CBB:connected}
The link of any simplex of dimension $\le m-2$ is connected.
\end{subthm}

\begin{subthm}{thm:poly-CBB:2pi}
The link of any simplex of dimension $m-2$ is 
isometric to a circle of length $\le 2\cdot\pi$
or a closed real interval or length $\le \pi$.
\end{subthm}
\end{thm}

Note that the condition (\ref{SHORT.thm:poly-CBB:connected})
is equivalent to the following:

\begin{itemize}
 \item[{\it \ref{SHORT.thm:poly-CBB:connected}$'$)}] 
Any path $\gamma\:[0,1]\to \spc{P}$ can be approximated by paths
$\gamma_n\:[0,1]\to \spc{P}$ 
such that $\gamma_n(t)$ cross only simplexes of dimension $m$ and $m-1$.
\end{itemize}

\parit{Proof.} We apply induction on $m$.
The base case $m=1$ follows from (\ref{SHORT.thm:poly-CBB:m-1}).

\parit{Step.}
Assume that the theorem is proved if $\dim\spc{P}<m$.
According to the Globalization theorem (\ref{thm:glob}),
it is sufficient to show that 
$$\curv_p\spc{P}\ge \kappa$$ 
for any point $p\in\spc{P}$. 

Fix $p\in \spc{P}$.
Note that a spherical neighborhood of $p$
is isometric
to a the spherical neighborhood of the tip of the tangent $\kappa$-cone 
$$\T_p\mc\kappa\z
=
\Cone\mc\kappa(\Sigma_p).$$
Hence it is sufficient to show that 
$$\curv\T_p\mc\kappa\ge \kappa\eqlbl{eq:curv(T_p)>=k}$$
for any $p\in \spc{P}$.

By Theorem~\ref{thm:warp-curv-bound:cbb:a}, 
the later equivalent to 
\begin{clm}{}\label{clm:curv+diam}
$\diam\Sigma_p\le \pi$ and $\curv\Sigma_p\ge 1$.
\end{clm}


If $m=2$ then \ref{clm:curv+diam} follows from (\ref{SHORT.thm:poly-CBB:m-1}).

To prove the case $m\ge 3$,
note that $\Sigma_p$ is an $(m-1)$-dimensional spherical polyhedral space and all the conditions of the theorem hold for $\Sigma_p$.
It remains to apply the induction hypothesis.\qeds

Recall that the definition of $\ell$-simply connected spaces is given in \ref{def:l-s.c.}.


\begin{thm}{Theorem}\label{thm:PL-CAT}
Let $\spc{P}\in\PM{}\kappa$ and $\tau$ be a triangulation of  $\spc{P}$.

Then $\curv\spc{P}\le\kappa$ 
if and only if any connected component of the link of any simplex $\sigma$ in $\tau$
is $2\cdot\pi$-simply connected.

Equivalently, if and only if any closed local geodesic in $\Link_\sigma$ has length at least $2\cdot\pi$.

\end{thm}

\parit{Proof.}
Let us apply induction on $\dim\spc{P}$.
The base case case $\dim\spc{P}=0$ is evident.

\parit{Step.}
Assume that the theorem is proved in case $\dim\spc{P}<m$.

Fix a point $p\in\spc{P}$.
A neighborhood of $p$ 
is isometric to the $\kappa$-cone over 
the $\Sigma_p$.
By Theorem~\ref{thm:warp-curv-bound:cbb:a}, 
it is sufficient to show that 
\[\Sigma_p\in \Cat{}1\eqlbl{eq:Sigma-in-CAT(1)}\]

Note that $\Sigma_p\in \PM{m-1}1$
and the links of $\Sigma_p$ are isometric to the links of $\spc{P}$. 
By induction hypothesis we get $\curv\Sigma_p\le 1$.
Applying Generalized Hadamard--Cartan theorem \ref{thm:hadamard-cartan-gen},
we get \ref{eq:Sigma-in-CAT(1)}.
\qeds


\section{Flag complexes}


\begin{thm}{Definition}
A simplicial complex $\mathcal{S}$ 
is called \emph{flag complex}\index{flag complex} if whenever $\{v_0,\z\dots,v_\kay\}$
is a set of distinct vertices of $\mathcal{S}$
which are pairwise joined by edges, then $\{v_0,\dots,v_\kay\}$
spans a $\kay$-simplex in $\mathcal{S}$.

If the above condition satisfied only for $\kay=2$ then we say that $\mathcal{S}$ satisfies a \emph{no triangle condition}\index{no triangle condition}.
\end{thm}

Note that every flag complex is determined by its 1-skeleton.


\begin{thm}{Proposition}\label{prop:no-trig}
A simplicial complex $\mathcal{S}$ is flag if and only if 
$\mathcal{S}$ as well as all the links of all its simplices
satisfies the no-triangle-condition.
\end{thm}

\parit{Proof.}
Clearly flag complex satisfies no triangle condition.

Note that link of flag complex is a flag complex.
Therefore no triangle condition holds for any flag complex and all its links.

Now assume a complex $\tau$ and all its links satisfy no triangle condition.
It follows that $\tau$ includes a 2-simplex for each triangle.
Applying the same observation for each edge we get that $\tau$ includes 3-simplex for any complete graph with 4-vertices.
Repeating this observation 
for triangles, 
3-simplexes,
4-simplices
and so on we get that $\tau$ is flag.
\qeds



Spherical polyhedral $\cCat{}{1}$ spaces glued from of right-angled simplices
admit the following combinatorial characterization discovered by Gromov \cite[p. 122]{gromov:hyp-groups}.

A triangulation of spherical polyhedral space 
is called \emph{all-right triangulation} if each simplex of the triangulation is isometric to the spherical simplex all of which angles are right.

\begin{thm}{Theorem}
Assume that a spherical polyhedral space $\spc{P}$
admits an all-right triangulation $\tau$.
Then $\spc{P}\in\cCat{}{1}$
if and only if $\tau$ is flag.
\end{thm}

\parit{Proof.} 
Assume $\spc{P}$ be the spherical polyhedral space 
with triangulation $\tau$
such that each simplex in $\tau$ is isometric to a right-angled spherical simplex.

Assume there are there vertices $v_1,v_2$ and $v_3$ of $\tau$
that pairwise joined by edges, 
but do not span a simplex.
Note that in this case 
$$\mangle\hinge{v_1}{v_2}{v_3}=\mangle\hinge{v_2}{v_3}{v_1}=\mangle\hinge{v_3}{v_1}{v_1}=\pi.$$
Equivalently,
\begin{clm}{}\label{clm:3pi/2}
The joint of the geodesics $[v_1v_2]$, $[v_2v_3]$ and $[v_3v_1]$
forms a locally geodesic loop $\spc{P}$. 
\end{clm}

Now assume that $\spc{P}\in\cCat{}{1}$.
Then so is the link $\Link_\sigma\spc{P}$ of every simplex in $\tau$. 

Each of these links is an all-right spherical complex
and
by Theorem \ref{thm:PL-CAT}, 
none
of these links can contain a geodesic circle of length less than $2\cdot\pi$. 

Therefore Proposition~\ref{prop:no-trig} and \ref{clm:3pi/2}, 
imply the statement.

It remains to prove that if $\tau$ is a flag then $\spc{P}$ is $\cCat{}{1}$. 
Proceeding by
induction on the dimension of $\spc{P}$, 
we may assume that $\Link_\sigma\in\cCat{}{1}$
for every simplex $\sigma$ of $\tau$.
In particular, $\curv\spc{P}\le 1$.

It remains to show that any closed local geodesic $\gamma$ in $\spc{P}$ 
has length at least $2\cdot\pi$.
Recall that star of a simplex $\sigma$ (briefly $\Star_\sigma$)
is formed by union of all simplexes which contain $\sigma$.

Choose a simplex $\sigma$ which contains a point of $\gamma$ in its interior
let $v$ be a vertex of $\sigma$.
Set $f(t)=\cos\dist{v}{\gamma(t)}{}$.
Note that $f''(t)+f(t)=0$ if $f(t)>0$.
It follows that $\gamma$ spends 
spends time $\pi$ in the interior of $\Star_v$.

After leaving $\Star_v$,
the local geodesic $\gamma$ has to enter in an interior of an other simplex say $\sigma'$
which has a vertex $v'$ not joined to $v$ by an edge.

Since $\tau$ is flag, we have that the stars $\Star_v$ and $\Star_{v'}$
do not have common interior points.
The same argument as above shows that $\gamma$ spends time $=\pi$ in interior of $\Star_{v'}$.
Therefore the total length of $\gamma$ is at least $2\cdot\pi$.
\qeds

\begin{thm}{Exercise}
Show that the barycentric subdivision of any simplicial complex is a flag complex.
Conclude that any simplicial complex is homeomorphic to a $\cCat{}{1}$ space.
\end{thm}


\section{Subsets of model space}

\parbf{Two-dimensional case.}
Let $[a_1\dots a_n]$
be a simple closed broken geodesic in the model plane $\Lob{2}{\kappa}$.
The closure $P$ of a bounded component of 
$\Lob{2}{\kappa}\backslash [a_1\dots a_n]$ 
is called \emph{polygon}.

Note that if $\kappa\le 0$ then there is unique polygon which is bounded by $[a_1\dots a_n]$,
while in the case $\kappa> 0$
there are two such polygons,
one on each side of $[a_1\dots a_n]$.

Note that the half-spheres in $\Lob{}\kappa$ for $\kappa>0$ are included in the class of polygons.
If a polygon is not a half-sphere then 
we can assume that the broken geodesic $[a_1\dots a_n]$ nondegenerate, 
that is, all the angles 
$\mangle \hinge{a_1}{a_n}{a_2}$,
$\mangle \hinge{a_2}{a_1}{a_3},
\dots,
\mangle \hinge{a_{n}}{a_{n-1}}{a_{1}}$ are distinct from $\pi$.
In this case $P$ is called $n$-gon;
the points $a_1,\dots,a_n$ are called \index{vertex of polygon}\emph{vertices} of $P$;
the geodesics $[a_na_1]$, $[a_1a_2],\dots,[a_{n-1}a_n]$ 
\index{side of polygon}\emph{sides} of $P$
and the remaining geodesics between the vertices are called \index{diagonal of polygon}\emph{diagonals} of $P$. 

A vertex $a_i$ of $P$ 
is called \index{convex vertex of polygon}\emph{convex} if a the intersection of $P$ with a small ball centered at $a_i$ is convex.
A non-convex vertex is called \index{concave vertex of polygon}\emph{concave}.


\begin{thm}{Theorem}\label{thm:2-d-cba}
Let $P$ be a polygon in $\Lob{2}{\kappa}$ equipped with the length-metric.
\begin{subthm}{}
If $\kappa\le 0$ then $P\in\cCat{}{\kappa}$.
\end{subthm}

\begin{subthm}{thm:2-d-cba:k>0}
If $\kappa>0$ then $P\in\cCat{}{\kappa}$ if  and only if $P$ does not contain half-space in its interior.
\end{subthm}
\end{thm}

In the proof we will need the following classical result in discrete geometry.

\begin{thm}{Proposition}\label{prop:diagonal}
Assume that $n>3$ 
and $P$ is an $n$-gon in $\Lob{2}{\kappa}$.
If $\kappa>0$ then we assume that $P$ does not contain closed half-sphere in its interior.
Then there is a  diagonal, say $[xy]$, of $P$ 
such that $\l]xy\r[$ lies in the interior of $P$.
\end{thm}

\parit{Proof.}
First note that $P$ has at least one convex vertex, say $v$.
Indeed, otherwise its complement is convex.
The later contradicts either
that $P$ is bounded for $\kappa\le  0$ 
or that $P$ does not contains a half-sphere in its interior for $\kappa\le  0$.

Let $[uv]$ and $[vw]$ be the sides of $P$ meeting at $v$.
If the $\l]uw\r[$ lies in the interior of $P$ 
then the proposition is proved.

%???PIC

Otherwise let $x\in[uv]$ be the point closest to $v$
such that $\l]xw\r[$ does not lie completely in the interior of $P$.
It means that $\l]xw\r[$ contains a vertex of $P$, 
say $z$.
In this case $\l]vz\r[$ lies in the interior of $P$.
Hence the result follows.
\qeds


\parit{Proof of Theorem~\ref{thm:2-d-cba}.}
First let us show that the condition in (\ref{SHORT.thm:2-d-cba:k>0}) is necessary if $\kappa>0$. 
Applying rescaling, we can assume that $\kappa=1$ and therefore $\Lob2\kappa=\SS^2$.

Assume $D=\cBall[p,\tfrac{\pi}2]$ lies in  the interior of $P$.
Given $\eps>0$
consider a triangle $[x^1x^2x^3]$ 
such that $\mangle\hinge{p}{x^i}{x^j}=\tfrac23\cdot\pi$ if $i\ne j$
and 
$\dist{p}{x^i}{}=\tfrac{\pi}2+\eps$ for all $i$.
Note that for small $\eps>0$ the triangle $ [x^1x^2x^3]$ lies in $P$.
On the other hand for some $z\in[x^2x^3]$, the geodesic $[x^1z]$ does not lie in $P$ completely.
Hence 
\[\dist{x^1}{z}{P}>\dist{x^1}{z}{\SS^2};\]
that is, the point-on-side comparison (\ref{cat-monoton}) does not hold for $[x^1x^2x^3]$
and hence $P\z\notin \cCat{}{\kappa}{}$.

If $P$ is a half-sphere then obviously $P\in\cCat{}{\kappa}$.

Assume $P$ is an $n$-gon.
To show that $P\in\cCat{}{\kappa}$,
we apply induction $n$.
The base cases $n=2$ and $3$ are trivial.

The proof of the step of induction,
let us apply Proposition~\ref{prop:diagonal}.
The diagonal $[xy]$ cuts $P$ into two polygons say $Q_1$ and $Q_2$.
By the induction hypothesis $Q_1,Q_2\in\cCat{}{\kappa}$.
Notice that the geodesic $[xy]$ forms a convex set
in $Q_1$ and in $Q_2$.

Note that $P$ can be obtained by gluing $Q_1$ and $Q_2$ along $[xy]$.
Therefore, by Reshetnyak's gluing theorem,
$P\in\cCat{}{\kappa}$.
\qeds

The following theorem 
is a partial case of a more general result proved by Bishop in \cite{bishop:jordan};
see also the discussion after Example 9.1.6 in the book of Burago--Burago--Ivanov \cite{BBI}.

\begin{thm}{Theorem}\label{thm:2d-bishop}
Let $\Omega$ be an open  proper simply connected subset in $\Lob2{\kappa}$.
If $\kappa>0$, assume in addition that $\Omega$ does not contain a closed halphsphere.

Consider $\Omega$ with the induced intrinsic metric.
Then the completion $\Omega^*$ of $\Omega$ is a $\cCat{}{\kappa}$ space.
\end{thm}

\parbf{Remark.}
Note that if $\Omega$ is bounded by a simple closed rectifiable curve its completion of $\Omega^*$ is isometric to the closure $\bar{\Omega}$  of $\Omega$ equipped with the length metric. 
In general, the space $\Omega^*$ admits natural length-preserving map in $\bar{\Omega}$, 
but 
this map might be not injective 
and its image might be proper subset of $\bar{\Omega}$.

%%%%%%%%+PIC???

\parit{Proof.}
Fix the points $x^1,x^2,x^3,x^4\in\Omega$.
Note that it is sufficient to prove that 
\begin{clm}{}\label{clm:2+2-Omega}
(2+2)-point $\kappa$-comparison holds for any four points 
$x^1, x^2, x^3, x^4$ in $\Omega$ 
equipped with the length metric.
\end{clm}

Fix $\eps>0$.
Connect $x^i$ to $x^j$ by a curve $\gamma_{ij}$ in $\Omega$;
such that 
\[\length\gamma_{ij}<\dist{x^i}{x^j}{\Omega}+\eps.\]

Note that there is a polygon $P\subset \Omega$ 
which contains all $\gamma_{ij}$.

By Theorem~\ref{thm:2-d-cba}, $P\in \cCat{}{\kappa}$.
In particular the points  $x^1,x^2,x^3,x^4$ satisfy (2+2)-point comparison in $P$.
By the construction,
\[\dist{x^i}{x^j}{P}\lege \dist{x^i}{x^j}{\Omega}\pm\eps\]
for all $i$ and $j$.

Since $\eps>0$ is arbitrary, \ref{clm:2+2-Omega} follows.
\qeds

\parbf{Two-convexity.}
Assume for a subset $A\subset \EE^m$
the following condition holds for any affine $2$-dimensional subspace $W\subset\EE^m$.
If a closed curve $\gamma\:\SS^1\to A\cap W$ is hull homotopic in $A$
then $\gamma$ 
hull homotopic $\gamma\subset A\cap W$.
In this case we say that $S$ is \index{2-convex}\emph{2-convex}.
This definition is closely related to the one given by Gromov in \cite[\S\textonehalf]{gromov:SaGMC}, see also \cite{panov-petrunin:sweeping}.

Note that according to this definition any simply connected subset in $\EE^2$ is 2-convex.

Note that intersection of arbitrary collection of  two-convex sets is two-convex.
The later makes possible to define \emph{two-convex hull} $\Conv_2 A$ of a set $A$
as the minimal two-convex set containing $A$.

The following proposition describes a construction producing $\Conv_2 \Omega$ for an open set $\Omega\subset\EE^3$.

\begin{thm}{Proposition}\label{prop:2-conv-construction}
Let $\Pi_1,\Pi_2\dots$ be an everywhere dense
sequence of planes in $\EE^3$.
Given an open set $\Omega$ consider 
the recucevly defined sequence of open sets 
$\Omega=\Omega_0\subset\Omega_1\subset\dots$ 
such that 
$\Omega_n$ is the union of $\Omega_{n-1}$ 
and all the bounded compontnets of 
$\EE^3\backslash(\Pi_n\cup \Omega_{n-1})$.
Then 
\[\Conv_2\Omega=\bigcup_n\Omega_n.\]

In particular the two-hull of open set is open.
\end{thm}

\parit{Proof.}
Set 
\[\Omega'=\bigcup_n\Omega_n.\]
Note that $\Omega'$ is a union of open set, in particular it is open.

The inclusion $\Conv_2\Omega\supset\Omega'$
is evident.

It remains to show that $\Omega'$ is two-convex.
Assume contrary; 
that for a plane $\Pi$ 
there is a closed curve $\gamma\:\SS^1\to \Pi\cap \Omega'$ 
which is null-homotopic in $\Omega'$,
but not null-homotopic in $\Pi\cap\Omega'$.

By approximation we can assume that $\Pi=\Pi_n$ and $n$ is large enogh so that $\gamma$ lies in $\Omega_{n-1}$.
The latter contradicts the $n$-th step in the construction. 
\qeds

\parbf{Three-dimensional case.}
A subset $K$ of $\EE^m$ is called \index{polytope}\emph{polytope} 
if it can be presented as a union of finite number of simplices.

Note any polytope admits a triangulation admits a finite triangulation.
Therefore any polytope equipped with induced intrinsic metric 
forms a Euclidean polyhedral space as defined in \ref{def:poly}.


\begin{thm}{Theorem}\label{thm:3d-subset}
Let $\Omega$ be the interior of a polytope $K$ in $\EE^3$.
Equip $\Omega$ with the induced length metric
and denote by $\Omega^*$ its completion. 
Then $\curv\Omega^*\le 0$ 
if and only if $\Omega$ is two-convex.
\end{thm}

\begin{wrapfigure}{r}{20mm}
\begin{lpic}[t(-7mm),b(0mm),r(0mm),l(0mm)]{pics/polytope(1)}
\end{lpic}
\end{wrapfigure}

\parbf{Remark.}
In the most cases, $\Omega^*$ is isometric to $K$ with induced intrinsic metric,
but in general we only have locally distance preserving map $\Omega^*\to K$ which is onto but in general may be not injective. 
An example can be seen on the picture.

\parit{Proof.}
If $\Omega$ is not two-convex then 
there is a plane $\Pi$ in $\EE^3$ 
which pass through a vertex $v$ of $K$ 
such that punctured neightborhood of $v$ in $\Pi$ lies in $\Omega$.
Choose a plane $\Pi'$ parallel and very close to to $\Pi$ which cuts from the complement of $\Omega$ a little piramid $S$ with vertex $v$.
Consider a small triangle $\triangle$ in $\Pi'$ wich surrounds the base of $S$.
Note that $\triangle$ is a geodesic triangle in $\Omega^*$
for which the point-on-side comparison \ref{cat-monoton}
fails to be true.
That is, $\curv_v\Omega^*\nleqslant0$.

The converse follows from 

To prove the converse we will need the following claim.
\begin{clm}{}
Assume that a spherical polygon $R$ does not contain half sphere in its interior.
Then $R$ admits a triangulation by weakly convex triangles all of which vertices are vertices of $R$.
\end{clm}

Indeed, ???\claimqeds

Applying Reshetnyak gluing theorem (???) we get that $R\in\cCat{2}{1}$.

\qeds


\begin{thm}{Shefel's Theorem}\label{thm:shefel}
Let $\Omega$ be an open simply connected 
subset of $\EE^3$.
Equip $\Omega$ with the induced length metric 
and let $\Omega^*$ be its completion.
Then $\Omega^*$ has curvature $\le\kappa$ at every point 
if and only if $\Omega$ is two-convex.
\end{thm}


\begin{thm}{Exercise}\label{ex:polygon-slices}
Let $m\ge 3$,
$K\subset \Lob{m}{\kappa}$ be a compact subset.
Assume that for any 2-dimensional subspace $W$ in $\Lob{m}{\kappa}$
the set $K\cap W$ is a polytope. 
Then $K$ is a polytope.
\end{thm}

\begin{thm}{Lemma}\label{lem:loc-concave}
Let $\Omega$ be an open set in $\Lob{3}{\kappa}$.
Then $\Conv_2\Omega$ is an open set.
 
Moreover, the set $\Lambda=\Lob{3}{\kappa}\backslash(\bar\Omega\cup\Conv_2\Omega)$
 is locally convex;
that is, for any $x\in\Lambda$ therew is $\eps>0$ such that the intersection
$\oBall(x,\eps)\cap(\Lambda\backslash\Omega^{(2)})$
is convex.
\end{thm}

\begin{thm}{Lemma}
Closed locally convex set in $\Lob{3}{\kappa}$ is convex. 
\end{thm}




\begin{thm}{Key lemma}
The two-convex hull of the interior of polytope in $\Lob{3}{\kappa}$
is an interior of a polytope.
\end{thm}

\parit{Proof.}
Let $K$ be a polytope in $\Lob{3}{\kappa}$;
denote by $\Omega$ the interior of $K$.

Denote by $F_1,\dots,F_m$ the faces of $K$.

Set $\Omega'=\Conv_2\Omega$ and let $K'$ be the closure of $\Omega'$.
Further, 
for each $i$, 
set $F'_i=F_i\backslash \Omega'$.
In other words, 
$F'_i$ is the subset $F_i$ 
which remains on the boundary of $K'$.

From the construction of two-convex hull (\ref{prop:2-conv-construction})
we get that.

\begin{clm}{}\label{clm:F'-convex}
$F'_i$ is convex subset of $F_i$.
\end{clm}

Further, since $\Omega'$ is two-convex,
we get the following.

\begin{clm}{}\label{clm:complement-of-F'-convex}
Each connected component of the complement $F_i\backslash F'_i$ is convex.
\end{clm}

Indeed, assume a connected component $A$ of $F_i\backslash F'_i$ fails to be convex.
Then there is a suppoiting line $\ell$ to $A$ touching $A$ at a single point in the interior of $F_i$.
Then one could rotate the plane of $F_i$ slightly arounf $\ell$ and move it paralleley to cut a hat from the complement of $\Omega$.
The latter means that $\Omega$ is not two-convex, 
a contradiction.
\claimqeds

From \ref{clm:F'-convex} and \ref{clm:complement-of-F'-convex}, we get that 

\begin{clm}{}$F'_i$ is a convex polygon for each $i$.
\end{clm}

Consider the complement 
$\EE^3\backslash \Omega$ 
equipped with the length metric.
Note that the complement $L=\EE^3\backslash (\Omega'\cup K)$
is locally convex;
that is, any points of $L$ admits a convex neighborhood.

In particular, if $W$ is a 2-dimensional subspace of $\Lob3\kappa$ 
then the intersection $W\cap K'$ is bounded 
by finite number of locally concave arcs and the  intersections $F_i'\cap W$.
Note that the locally concave arcs have their ends on the ends of $F_i'\cap W$;
the later is formed by a line segment or it degenerates to a point.

\begin{clm}{}\label{clm:intersection-is-polytope}
For any 2-dimesional subspace $W\subset\Lob{3}{\kappa}$,
the intersection $W\cap \Conv_2 \Omega$ is interior of a polytope.
\end{clm}

Indeed, set $L=W\cap \Conv_2 \Omega$.
According to Lemma~\ref{lem:loc-concave}, 
the boundary of $L$ is formed by the intersections s $F'_i\cap W$
and locally concave arcs.

Arguing by contradiction, 
assume that $L$ is not interior of a polytope then there is a smooth extremal point on one of these arcs; 
that is, there is a point $x\in\partial_W L$ such that ???.
Without loss of generality, we can assume that $x$ does not lie on a line connecting a pair of vertices of $F_1',\dots,F_n'$.

Consider the local supporting plane $\Pi$ to $\Lambda$ at $x$
and let $Q_x\ni x$ be the connected component of $\Pi\cap\Lambda'$.

\begin{clm}{}\label{clm:Q-is-interval}
$Q_x$ is an open interval. 
\end{clm}

Indeed,???\claimqeds

Let $Q_x=\l]ab\r[$.
Clearly $a\in\partial F_i'$ and $b\in\partial F_j'$
for some $i$ and $j$.
Since $x$ does not lie on a line connecting the vertices,
we can assume that $a$ is not a vertex of $F_i'$;
denote by $[vw]$ the edge of $F_i'$ containing $a$.

\begin{clm}{}\label{clm:[vw]cPi}
$Q_x$ contains the interior of the triangle $[bvw]$.
\end{clm}

Indeed,???\claimqeds

Note that the claims \ref{clm:[vw]cPi} and \ref{clm:Q-is-interval} contradict eachother.
Therefore \ref{clm:intersection-is-polytope} is proved.


Note that the complement $\Omega^\complement$ is a polytope.
In particular, $\Omega^\complement$ is a union of finite number of convex polytopes,
say 
$\Omega^\complement=R_1\cup\dots\cup R_n$.
For each $i$, the complement $R_i'=R_i\backslash \Conv_2\Omega$ is a convex set.
According to 
Exercise~\ref{ex:polygon-slices} 
and \ref{clm:intersection-is-polytope}, 
$R_i'$ is a convex polytope for each $i$.
Hence $(\Conv_2\Omega)^\complement=\bigcup_i R_i'$ is a polytope.
\qeds

\parit{Proof of \ref{thm:shefel}.}
Note that it is sufficient to show that
(2+2)-point $\kappa$-comparison holds for any
4 points $p^1,p^2,x^1,x^2\in\Omega$.

Fix $\eps>0$.
Choose a six broken lines connecting all the pairs of points $p^1,p^2,x^1,x^2$ such that length of each at most $\eps$ bigger than 
the distance between its ends.
Choose a simply connected polytope $P$ 
in $\Omega$ which contains all these broken lines.
Denote by $P'$ the two-convex hull of 
$P$.

According to ???, $P'$ is a polytope
Therefore $\curv P'\le \kappa$. 
\qeds

\parbf{Variations and generalization.}
A theorem analogous to ??? holds also in $\HH^3$, 
and a local version holds also in $\SS^3$.

As it will be shown in Section \ref{sec:with-bry},
an analog of Theorem~\ref{thm:shefel}
holds in higher dimensions assuming $\Omega$ has smooth boundary.
If the boundary is not smooth this is not longer true already in $\EE^4$;
an example can be found among intersections of two two-convex domains with smooth boundaries.
The following exercise is closely related.

\begin{thm}{Exercise}
Let $\Pi_1,\Pi_2$ be two planes in $\EE^4$ intersecting at single point.
Consider the complement $\Omega=\EE^4\backslash(\Pi_1\cup\Pi_2)$
equipped with induced length metric.
Denote by $\tilde\Omega^*$ the completion of universla cover of $\Omega$.
Show that $\Pi_1\perp\Pi_2$ if and only if $\tilde\Omega^*\in\cCat{}{0}$.
\end{thm}


\section {Manifolds-with-boundary}\label{sec:with-bry}
Let $M$ be a connected $m$-dimensional Riemannian manifold with possibly nonempty boundary $\partial M$.
%By a \emph{Riemannian manifold-with-boundary} $M$ of a given differentiability class, we mean $M$ has possibly nonempty boundary and has an extension across the boundary to a manifold $\overline{M}$ of the same class with empty boundary, such that  each boundary point of $M$ has a neighborhood $U$ in $\overline{M}$ coordinatized by $\R^n$ and for which $M\cap U$ is coordinatized by a halfspace of $\R^n$. The Riemannian metric of $\bar M$ is of differentiability class one degree lower. 
We equip $M$ with the length metric. 

For each point $p\in\partial M$,
let 
\[k_1(p)\le k_2(p)\le\dots\le k_{m-1}(p)\]
denote the principal curvatures
of $\partial M$ at $p$;
the sign convention is made so that
so the closed ball in Euclidean space has positive curvatures 
and the complement of the open ball has negative curvatures.

We say that $M$ has \emph{convex boundary} 
if $k_1(p)\ge 0$ for any $p\in\partial M$. 

\begin{thm}{Proposition}\label{prop:example-mnflds-with-bry:CBB}
Let $M$ be a Riemannian manifold with possibly nonempty boundary.
Then
$M\in\CBB{}{\kappa}$ 
if and only if $M$ has convex boundary 
and the sectional curvature of $M$ is at least $\kappa$.
\end{thm}

\parit{Proof; (\ref{thm:example-mnflds-with-bry:CBB}).}
To prove ``if'' part, we need to present a neighborhood of any given point $p$ in $M$ where the $\kappa$-comparison holds.

If $p$ lies in the interior of $M$, 
the existence of such neighborhood 
follows from the standard Toponogov comparison theorem.

Otherwise 
Without loss of generality, we may assume that $M$ 
is a subset of Riemannian manifold without boundary $N$, 
which has sectional curvature $\ge \kappa$.???

Note that sufficiently small spherical neighborhood of 
any point in $N$ is convex.???\qeds


Let $p\in\partial M$ and $\sigma$ is a sectional direction at $p$ to $\partial M$.
We say that $\sigma$ is a concave direction if the second fundamental form of $\partial M$ restricted to $\sigma$ is negative.

\begin{thm}{Theorem}\label{thm:example-mnflds-with-bry:CBA}
Let $M$ be a Riemannian manifold with possibly nonempty boundary.
Then
$\curv M\le \kappa$
if and only if the sectional curvature of $M$ is at most $\kappa$
and $\partial M$ has sectional curvature at most $\kappa$ 
in all concave sectional directions of $\partial M$.
\end{thm}

We say that $M$ has \emph{two-convex boundary} 
if $k_2(p)\ge 0$ for any $p\in\partial M$.

%\begin{thm}{Corollary}
%Let $M$ be a Riemannian manifold with possibly nonempty two-convex boundary.
%Assume sectional curvature of $M$ and $\partial M$ are at most $\kappa$.
%Then $\curv M\le \kappa$.
%\end{thm}
%
%\parit{Proof.}
%Follows directly from 
%Theorem~\ref{thm:example-mnflds-with-bry:CBA}.
%\qeds

\begin{thm}{Corollary}
Let $M$ be a Riemannian manifold with possibly nonempty two-convex boundary.
Then $\curv M\le{\kappa}$ 
if and only if the sectional curvature of $M$ is at most $\kappa$.
\end{thm}

\parit{Proof.}
If $\partial M$ is two-convex, 
it has no points with concave sectional directions.
It remains to apply Theorem~\ref{thm:example-mnflds-with-bry:CBA}.
\qeds

\begin{thm}{Definition}\label{def:mwb-segment}??? A geodesic of $M$ consists of  \emph{interior
segments}, by which we mean nonconstant open segments  that have zero acceleration in $M$ but
may include points of $\partial M$;  \emph{boundary segments}, nonconstant geodesics of $\partial M$ whose acceleration in $M$ is nonzero (necessarily normal to $\partial M$) on a dense open subset; \emph{switch-points}, where the geodesic switches between interior and boundary segments;  and \emph{chatter-points}, namely accumulation points of switch-points. 
\end{thm}

Examples of single geodesics are easily constructed that have a Cantor set of positive measure of chatter-points.
The existence of chatter-points makes the variational theory of geodesics
difficult to analyze.

%\begin{thm}{Theorem (Regularity of geodesics)}\label{thm:m-w-b-geodesic-''}
%Let $M$ be a Riemannian 
%manifold-with-boundary.
%Let $\gamma$ be a geodesic of $M$ with unit-speed parameter $s$.  Then  after  passing to coordinates, $\gamma''(s)$ exists everywhere except at switch-points, where $\gamma''(s)$ exists one-sidedly.  At chatter-points, $\gamma''(s)$ vanishes.
%\end{thm}

\begin{thm}{Lemma}\label{lem:m-w-b-geodesic-'}
Let $M$ be a Riemannian manifold-with-boundary, and $\gamma$ be a geodesic of $M$. Then:
\begin{subthm}{}
$\gamma$ is continously differentiable.
\end{subthm}
\begin{subthm}{}
The $2$-plane in $T_{\gamma(t)}N$  normal to $\partial M$ and containing $\gamma'(t)$ is an osculating plane for $\gamma$ at $t$.
\end{subthm}
\end{thm}


\parit{Proof.}
Let  $\gamma$ be a unit-speed geodesic of $M$ with $\gamma(t)=p\in\partial M$.  
At an endpoint of $\gamma$, we take $\gamma'$ to mean the corresponding one-sided derivative.

\begin{clm}{}\label{clm:1-sided-geo-mwb}
If $\gamma$ is one-sidedly differentiable at $t$, then $\gamma$ is differentiable at $t$.
\end{clm}

\begin{clm}{}\label{clm:diff-geo-mwb}
$\gamma$ is differentiable.
\end{clm}
Let $N$ be a Riemannian manifold without boundary, of the same dimension as $M$, in which  a neighborhood of $p$ in $M$ is isometrically embedded. We may take  $N=\cup \,H_u$, $-\epsilon_2<u<\epsilon_2$, for a family of equidistant hypersurfaces $H_u$ with unit normal field $v$,  where each $H_u$ intersects $\partial M$ transversely with $v$ pointing out of $M$, and the 
principal curvatures of the $H_u$ are positive toward $-v$. 
%Letting $H_0$ pass through $p$, we may take $H_0$ to lie in the exponential image of a  neighborhood of $o$ in a sphere through $o$  in $T_p M$. 
Set $H_+=\cup_{\,u>0} \,H_u$.

Projection 
$$\map[2]\:H_+ \to H_0$$
 along $v$ is defined and length-nonincreasing.  Moreover, the image of $H_+\cap M$ lies in $H_0\cap M$ because $v$ points into $M$ at points of $\partial M$.

Suppose $\gamma$ is not one-sidedly differentiable, say $\gamma^+(t)$ does not exist.  Then for $i=1,2$ there are distinct unit vectors $v_i \in T_p\partial M$, and points $\gamma(t_{ij})$, $t_{ij}>0$, $t_{ij}\to t$, such that the directions of the preimages of $\gamma(t_{ij})$ in a normal coordinate neighborhood in $N$ of $p$  converge to $v_i$.

Now choose the family $H_u$ so that $v_1$ and $v_2$ lie on opposite sides of $H_0$ at $p$.  Then $\gamma$ may be shortened in $M$ by applying the projection $\map[2]$ to a subsegment of $\gamma$ that lies in $H_+$ and runs between two points in $H_0$.  This contradiction shows $\gamma^+(t)$ exists. By \ref{clm:1-sided-geo-mwb}, $\gamma'(t)$  exists.

\begin{clm}{}\label{clm:C^1-geo-mwb}
$\gamma$ is continuously differentiable. 
\end{clm}

 Suppose $\gamma'$ is not continuous at $t$. Then $\gamma'(t)\in T_p\partial M$. Without loss of generality, there is a sequence $t_j>0$, $t_j\to t^+$, such that  $\gamma'(t_j)\to v\ne \gamma'(0)$.  Then also $v\in T_p\partial M$. In the notation of Claim \ref{clm:diff-geo-mwb}, choose the family $H_u$ so that $\gamma'(t)$ and $v$ lie on opposite sides of $H_0$ at $p$, and $\gamma'(t))$ points into $H_+$.  Then $\gamma$ has a subsegment in $H_\epsilon^+$ that runs between two points in $H_\epsilon$ for some $\epsilon>0$.   
 Thus $\gamma$ may be shortened in $M$ by projecting this subsegment to $H_\epsilon$.  The claim follows by contradiction.

  
\begin{clm}{}\label{clm:C^1-geo-mwb}
 The $2$-plane in $T_{\gamma(t)}N$ that contains $\gamma'(t)$ and is normal to $\partial M$ is an osculating plane for $\gamma$ at $t$.
\end{clm}

Recall that for any differentiable curve $\alpha$ in $N$, a $2$-plane $Q\subset T_{\alpha(t)}N$ is called  an \emph{osculating plane} for $\alpha$ at $t$ if for each $u$ in a neighborhood of $t$ in the parameter interval of $\alpha$, there is a  $2$-plane $Q_u$ containing $\alpha'(t)$ and $\exp _{\alpha(t)}^{-1} (\alpha(u))$ such that $Q=\lim_{u\to t}Q_u$.  If $Q$ exists and $\alpha$ is not a geodesic on a neighborhood of $t$,  then $Q$ is uniquely determined. 

Fix directions $u_1\in T_p\partial M$  orthogonal to $\gamma'(t)$, and $u_2\in  (T_p\partial M)^\perp$.  Let $R$ be the oriented $3$-plane  spanned by $\gamma'(t)$, $u_1$ and $u_2$.  Denote by $P_1(\theta,\varphi)$ and $P_2(\theta,\varphi)$, the two hyperplanes in $T_pN$ that are spanned by $R^\perp$ and  the $2$-plane in $R$ whose intersection with  the $\,\gamma'(t)\,u_1\,$-plane makes angle $\theta\in [0,\pi)$ with $\gamma'(t)$ and whose angle with $u_2$ is $\varphi\in (0,\pi/2)$.
For $i=1,2$, let $S_i(\theta, \varphi, k)$ be the $(n-1)$-sphere of radius $k$ in $T_pN$ passing through $o$, and  tangent to $P_i(\theta,\varphi)$ on the side away from $u_2$.  Consider the exponential images $H_i(\theta, \varphi, k)$  of  neighborhoods of $o$ in $S_i(\theta, \varphi, k)$.  
In the notation of Claim \ref{clm:diff-geo-mwb}, length-decreasing projection implies that on some open interval about $t$, $\gamma$ lies in $H_i(0, \varphi, k)\cup H_i(0, \varphi, k)^+$.  Otherwise, $\gamma(t+\epsilon)$ would enter the side with positive curvatures of $H_i(\theta, \varphi, k)$ for some $\theta<0$.  Then $\gamma|[t, t+\epsilon]$ would lie in that side, in contradiction to $\gamma'(t)$ being tangent to $H_i(0, \varphi, k)$.  It follows that on some open interval about $t$, the exponential preimage in $T_pN$ of $\gamma$  does not enter the open side with positive curvatures of $S_i(0,\varphi, k)$.

Now the claim follows by letting $\varphi\to 0$ and $k\to\infty$.
\qeds

\begin{thm}{Lemma}\label{lem:abs-cont}
Let $\gamma$ be a geodesic of a Riemannian manifold-with-boundary $M$. Then after passing to coordinates, 
$\gamma'$  is  locally Lipschitz continuous.  
\end{thm}

\parit{Proof.}
Specifically,  we extend $M$ to a manifold $N$ without boundary, and regard a  coordinate neighborhood in $N$ of $p=\gamma(t)$ as a neighborhood  in $\R^m$.

\begin{clm}{}\label{clm:pointwise-k-bound}
The pointwise arc-chord curvature  of  $\,\gamma$ (Definitions \ref{def:arc-chord}, \ref{ptwise-arc-chord}) is bounded above.
\end{clm}
It suffices to reparameterize $\gamma$ by Euclidean arc-length, say $\gamma\:[a,b]\to \R^n$, and we do so for notational convenience. 
%For a fixed $s \le
%\pi/\sqrt{k}$, the curvature $k$ of a $k$-curve $\sigma$ in $\Lob2\kappa$ is determined by the ratio $s/r$ of   length   to chord-length (i.e. \emph{arc/chord} ratio) of a segment of $\sigma$ of length $s$.  

As lengths of subarcs of $\gamma$ approach $0$ and corresponding arc/chord ratios approach $1$, an upper bound on arc/chord curvature may be deduced from a lower bound on the maximum length for which the ratio lies in a given small interval $[1,1+\delta]$.    Thus we prove the claim by showing,  given $t\in [a,b]$,  that for some  $\epsilon>0$, there is a uniform upper bound on the arc/chord ratio of $\gamma|[t_1,t_2]$ if  $a\le t_1\le t\le t_2\le b$, $|t_1-t_2|\le \epsilon$. 

We may suppose  $\gamma(t)$  does not lie in a line segment $\gamma|[t'_1,t'_2]$ where  $t_1< t'_1< t< t'_2$ or $t_1=t'_1= t< t'_2$ or $ t'_1< t= t'_2=t_2$,  since otherwise there is nothing to prove. In particular, $\gamma(t)\in\partial M$. We may also assume $\gamma(t_1)$ and $\gamma(t_2)$ lie in $\partial M$. Indeed, if one or both of $\gamma(t_1)$ and $\gamma(t_2)$ lie in $M-\partial M$, consider the maximal segment $\overline{\gamma}$ of $\gamma|[t_1,t_2]$ with endpoints on $\partial M$.  Since $\gamma$ is $C^1$, then by the triangle inequality for the polygonal curve consisting of the segments that extend $\overline{\gamma}$  to $\gamma$ together with the chord of $\overline{\gamma}$,  the arc/chord ratio of $\gamma$ is at most that of $\overline{\gamma}$.

% It follows that if   $|t_1-t_2|$ is sufficiently small, then  $\tilde\alpha$ has  geodesic curvature $\le k+\epsilon$, where  $k$ is an upper bound for the principal curvatures of $\partial M$  in a neighborhood of $\gamma(t_0)$.  


By Lemma \ref{lem:m-w-b-geodesic-'}, the 
Riemannian osculating plane of $\gamma$ at $t$ is defined and normal to $\partial M$.
%, i.e. as $t_1,t_2$ approach $t$, $Q$ approaches the $2$-plane that contains $\gamma'(t)$ and is normal to $\partial M$ at $p$.  
%Moreover, these osculating planes vary continuusly with $t$. 
Then for $\delta\in (0,\pi/2)$, we may choose $\epsilon>0$  such that  the  angle  in $\R^m$ between the normal to $\partial M$ and the $2$-plane $Q$ through points $\gamma(t_1), p=\gamma(t), \gamma(t_2)$ is bounded above by $\pi/2-\delta$, uniformly in $t$. Thus if $\tilde\alpha$ is the curve in $Q$  whose image is the intersection curve $\partial M\cap Q$ 
between $\gamma(t_1)$ and $\gamma(t_2)$, then  $\tilde\alpha$ has  geodesic curvature bounded in absolute value by some uniform constant $A>0$.

Let $\alpha$ be the geodesic of $M\cap Q$ between $\gamma(t_1)$ and $\gamma(t_2)$. 
  In $Q$, $\alpha$ is a $C^1$ convex curve that runs along the boundary of the convex hull of $\tilde\alpha$.  Therefore the total turn of  the tangent line of any subarc of  $\alpha$ with endpoints on $\partial M$  is not more than the total absolute turn of the tangent line of the subarc of  $\tilde\alpha$ with the same endpoints. This total absolute turn is not more than the total turn for a subarc $\sigma'$,  of the same length as $\tilde\alpha$, of a $k$-curve $\sigma$ of constant curvature $k=A$ in $\Lob2\kappa$.  

Consider a hinge whose sides are tangent  at $t_1$ and $t_2$ to $\alpha|[t_1,t_2]$; and an analogous hinge whose sides are tangent  at the endpoints of $\sigma'$ to a subarc $\sigma'$ of $\sigma$ of the same length as $\alpha|[t_1,t_2]$.  We have just seen that the hinge angle is not smaller for $\alpha|[t_1,t_2]$ than for $\sigma'$.  Therefore the 
chord-length of $\alpha|[t_i,t_2]$ is not smaller than  the chord-length of $\sigma'$.  It follows that there is a uniform upper bound on the arc/chord ratio of $\alpha|[t_1,t_2]$ if   $a\le t_1\le t\le t_2\le b$, $|t_1-t_2|\le \epsilon$.

By assumption, $\gamma$ is a Riemannian-shortest curve in $M$ between $\gamma(t_1)$ and $\gamma(t_2)$. The absolute difference of the Riemannian and Euclidean lengths of $\gamma|[t_1,t_2]$ 
 is at most a uniform constant multiple of the former. Hence the (Euclidean) length of $\gamma|[t_1,t_2]$
  is at most a uniform constant  multiple of the length of $\alpha|[t_1,t_2]$.   Therefore the arc/chord ratio of $\gamma|[t_1,t_2]$ is uniformly bounded above, as required.

\begin{clm}{}\label{clm:global-k-bound}$\gamma'$ is Lipschitz continuous. 
\end{clm}
By \ref{clm:pointwise-k-bound} and globalization of arc-chord curvature, 
\ref{subthm:local-global-base-angle}, the angle $\theta$ in $\R^n$ between $\gamma^+(t_1)$(respectively $-\gamma^-(t_2)$)  and the endpoint chord of $\gamma|[t_1,t_2]$ is at most equal to the angle between a circular arc of curvature $k$ and length $t_2-t_1$  in $\R^2$ and  its endpoint chord.  But $\theta$ is also the angle in $\R^n$ between the tangent lines to $\gamma$ at $t_1$ and $t_2$.  The claim follows.
\qeds

Now we are ready for one of our main curvature estimates. We formulate it in terms of an 
isometric imbedding of $M$ in some Euclidean space $\EE^n$. 
%It would be desirable
%to have a more direct link with the natural intrinsic invariants, the sectional
%curvature of $M$, the normal curvature of $B$ in $M$, and the injectivity radii of $M$
%and $B$, but our method has the advantage that it gives the estimate in terms of
%a single number. 

Say a positive number $r$ is a \emph{tubular
radius} for $M$ in $\EE^n$ if every point at distance $r$ or less from $M$ is the center of a
closed ball $\EE^n$ meeting $M$ at a single point. Then the principal curvatures of $M$, as well as  those of $\partial M$ belonging to a normal
vector whose $M$-component is outward from the interior of $M$, are bounded
above by $1/r$. Conversely, if we take an upper bound $k$ of such principal
curvatures, then
$1/k$ will be a tubular radius for a sufficiently small region of $M$.


\begin{thm}{Lemma}\label{lem:diff-inequality} Let $r = 1/k$ be a tubular radius for $M$, and let $\gamma$ and $\sigma$ be
geodesics in $M$ having speed $\le 1$. Let $f(r) = |\gamma(s) - \sigma(s)|$ be the
displacement in $\EE^n$ between corresponding points. Then 
%except at the countably many points where $f''$ fails to exist, we have a differential inequality
$$f'' \ge -k^2f,$$ with strict inequality where $f> 0$.
\end{thm}


%
%\parit{Proof of \ref{thm:m-w-b-geodesic-''} (Regularity of geodesics).}
%\begin{clm}{}\label{clm:chatter-''-0}
%At a chatter-point $\gamma(t_0)$ of $\gamma$, the normal curvature $k_0$ of $\,\partial M$ in the direction of $\gamma'(t_0)$ vanishes.
%\end{clm}
%
%If $k_0>0$, we could shorten $\gamma$ by replacing $\gamma|[t_0-\epsilon,t_0+\epsilon]$ by an interior geodesic of $M$ for $\epsilon$ sufficiently small. If $k_0<0$,  we could shorten $\gamma$ by replacing $\gamma|[t_0-\epsilon,t_0+\epsilon]$ by its normal projection to $\partial M$  for $\epsilon$ sufficiently small. Therefore $k_0=0$.
%
%\begin{clm}{}
%Define $k(t)$ on open boundary segments of $\gamma$ to be the normal curvature of $\partial M$ in the direction of $\gamma'(t)$, and set $k(t)=0$ otherwise.
%Let $x_n$ be the distance from $\partial M$, and  $x_i$, $i < n$, be coordinates on $\partial M$, extended to be
%constant on geodesics of $M$ normal to $\partial M$. Denote the Christoffel symbols of $M$ in these coordinates  by $\Gamma_{ijl}$. Then $\gamma$ satisfies, in an integral sense, the equations
%$$ {x_k}'' \ =\  -\sum_{i,j}x'_ix'_j\Gamma_{ijk},\,\, l< n,
%\eqlbl{eq:geo-eq-tangent}$$
%$$  {x_n}'' \ =\  \ -\ k\ - \sum_{i,j < n}x'_ix'_j\Gamma_{ijn}.
%\eqlbl{eq:geo-eq-normal}$$
%  \end{clm}
%
%On open segments of $\gamma$ in the interior of $M$, we have $k=0$, so  equations \ref{eq:geo-eq-tangent} and  \ref{eq:geo-eq-normal} are the geodesic equations of $\gamma$. On open boundary segments of $\gamma$, we have $x_n=0$, so equations \ref{eq:geo-eq-tangent} and \ref{eq:geo-eq-normal} again hold  since the coordinate matrix of the second fundamental form of $\partial M$ with
%respect to $\partial / \partial x_n$ is $-\Gamma_{ijk}$ where $i, j < n$.
%
%Given the absolute
%continuity of $\gamma'$ and the normality of $\gamma''$ on $B$, it follows that equations \ref{eq:geo-eq-tangent} and \ref{eq:geo-eq-normal} hold everywhere in an
%integral sense. Then \ref{eq:geo-eq-tangent} holds everywhere since the right-hand side is continuous. At the countably many switch points, \ref{eq:geo-eq-normal}  can be
%interpreted as being valid in the limit from either side. Except at switch-points, the right side of \ref{eq:geo-eq-normal}  is
%continuous, and hence the acceleration of $\gamma$ exists. In
%particular, the acceleration exists and is $0$ whenever $k = 0$. 

\begin{thm}{Lemma}\label{lem:mnflds-with-bry:CBA}
Let $M$ and $N$ be  Riemannian manifolds with possibly nonempty boundary, where $M$ is isometrically immersed in $N$.  Then the difference between distances in $M$ and in $N$ respectively is on the order of the cube of either one.
\end{thm}
%
\begin{thm}{Lemma}\label{lem:mnflds-with-bry:CBA}
Let $M$ be a Riemannian manifold with possibly nonempty boundary.
Then every point of $M$ has a neighborhood in $\CBA{}{}$.
\end{thm}

\parit{Proof of \ref{thm:example-mnflds-with-bry:CBA}.}
\qeds



\section{Convex hypersurfaces in Riemannian manifolds}

The following theorem provides an other source of examples of $\CBB{}{}$-spaces.

\begin{thm}{Theorem}\label{thm:buyalo} 
Let $M$ be an $m$-dimensional Riemannian manifold 
with sectional curvature $\ge \kappa$ 
and $F\subset M$ be a convex hypersurface equipped with the length-metric;
that is, $F$ bounds a convex set. 
Then $F\in\CBB{m-1}\kappa$.
\end{thm}

This theorem, in a slightly weaker form was proved in by Buyalo in \cite{buyalo:convex-surface}, 
and then its statement was made more exact in \cite{akp}.

\begin{thm}{Lemma}\label{lem:smoothing}
Let $M$ be a Riemannian manifold and $f\:M\subto\RR$ be a locally Lipshits $\lambda$-concave subfunction. 
Then there is a sequence of smooth subfunctions $f_n\:M\subto\RR$ and $\lambda_n\in\RR$ such that $f_n\to f$ and $\lambda_n\to \lambda$ as $n\to\infty$ and each $f_n$ is $\lambda_n$-concave.
\end{thm} 

This lemma is a slight generalization of \cite[Theorem 2]{greene-wu}
by Greene and Wu;
it can be proves the same way.
%???Should we include the proof???

\parit{Proof of theorem \ref{thm:buyalo}}. 
Without loss of generality one can assume that 
\begin{enumerate}
\item \label{k>=-1}$\kappa\ge -1$, 
\item $F$ bounds a compact convex set $C$ in $M$, 
\item there is a function $h$ defined in a neighborhood of $C$ such that $h''\le -2$ and $|h(x)|<1/10$ for any $x\in C$,
\item \label{property:unique} there is unique geodesic between any two points in $C$. 
\end{enumerate}
(If not, rescale and pass to the boundary of the convex piece cut by $F$  from a small convex ball centered at $x\in F$, taking $h=-10\cdot\dist[{{}}]{x}{}{}$.)

Consider the function $f=\dist{F}{}{}$.
By Rauch comparison 
(as it is stated in Petersen's book \cite[11.4.8]{petersen:RiemGeom}), 
for any unit-speed geodesic $\gamma$ in the interior of $C$, $(f\circ\gamma)''$ is bounded in the barrier sense by the corresponding value in the model case --- when $M\iso\Lob2{-1}$ and $F$ is a geodesic.  
In particular,
\[(f\circ\gamma)''\le f\circ\gamma.\]


Therefore $f+\eps\cdot h$ is $(-\eps)$-concave in 
$\Omega_\eps=f^{-1}((0,\eps))\cap C$.
Take 
\[K_\eps\z=f^{-1}([\tfrac{1}{3}\cdot\eps,\tfrac{2}{3}\cdot\eps])\cap C.\]
Applying Lemma~\ref{lem:smoothing}, we can find a smooth $(-\tfrac\eps2)$-concave function $f_{\eps}$ which is arbitrarily close to $f+\eps\cdot h$ on $K_\eps$ and which is defined on a neighborhood of $K_\eps$. 
Take a regular value $\theta_\eps\approx\tfrac{1}{2}\cdot\eps$ of $f_\eps$. (In fact one can take $\theta_\eps=\tfrac{1}{2}\cdot\eps$, but it requires a little work.) 
Since $|(h|C)|<1/10$, the level set $F_\eps=f_{\eps}^{-1}(\theta_\eps)$ will lie entirely in $K_\eps$.
Therefore $F_\eps$ forms a smooth closed convex hypersurface.

Let us denote by $\rho$ and $\rho_\eps$ the length-metrics on  correspondingly $F$ and $F_\eps$.
By the Gauss formula, $(F_\eps,\rho_\eps)$ has curvature $\ge\kappa$.
Further, $F_\eps$ bounds a compact convex set $C_\eps$ 
and $F_\eps\to F$, $C_\eps\to C$ in Hausdorff sense as $\eps\to 0$. 
By property (\ref{property:unique}), the restricted metrics from $M$ to $C$ and to $C_\eps$ are length-metrics.
Thus, $C_\eps$ is an Alexandrov space with $F_\eps$ as boundary, that converges in Gromov--Hausdorff sense to $C$.  It follows from ???  that $(F_\eps,\rho_\eps)$  converges in Gromov--Hausdorff sense to $(F,\rho)$.
Therefore $(F,\rho)$ is an Alexandrov space with curvature $\ge \kappa$.\qeds

\noi{\bf Remark.} 
We are not aware of any proof of theorem \ref{thm:buyalo} which is not based on the Gauss formula. 
Finding such a proof would be interesting on its own, and also could lead to the generalization of theorem \ref{thm:buyalo} to the case when $M$ is an Alexandrov space.



\section{Comments and open problems}

\begin{thm}{Shefel's conjecture}
Any saddle hypersurface in $\RR^3$ equipped with the length-metric has curvature $\le 0$ at any point.
\end{thm}

A surface $S$ in a metric space.
We say that $S$ is saddle if for any convex function $f\:\spc{X}\to\RR$,
the restriction $f|S$ has no local maxima. 

\begin{thm}{Generalized Shefel's  conjecture}
Any 2-dimensional saddle surface in a $\Cat{}{\kappa}$ 
equipped with the length-metric has curvature 
$\le \kappa$ at any point.
\end{thm}

It would be interesting to get an analog of the flag condition for other simplices 
which (not necessary right-angled).
Such a condition for some spherical Coxeter simplex
could resolve the following problem; see \cite{panov-petrunin} for more details. 

\begin{thm}{Braid space}
Consider $\CC^n$ with coordinates $z_1,z_2,\dots,z_n$.
Let us remove from $\CC^n$ the complex hyperplanes $z_i=z_j$, for all $i\ne j$,
pass to the universal caver and consider the completion $\spc{B}_n$ 
of the obtained space.

Is it true that $\spc{B}_n\in\Cat{}{0}$ for any $n$.
\end{thm}

The above question has an affirmative answer for $n\le 3$ and open for all $n\ge 4$.

\section{Exercises}

\begin{thm}{Exercise}
Assume  $\spc{P}\in\PM{}\kappa$ and $\dim \spc{P}\ge 2$. 
Show that 
\begin{enumerate}
\item If $\spc{P}\CBB{}{\kappa'}$ then $\kappa'\le \kappa$ and $\spc{P}\CBB{}{\kappa}$
\item If $\spc{P}\cCat{}{\kappa'}$ then $\kappa'\ge \kappa$ and $\spc{P}\cCat{}{\kappa}$
\end{enumerate}

\end{thm}


\begin{thm}{Exercise}
Show that if in a Euclidean polyhedral space such that
any two points can be connected by unique geodesic then $\spc{P}\in\cCat{}{0}$.
\end{thm}


\begin{thm}{Exercise}
Show that completion of any flat simply connected 2-dimensional manifold is a $\cCat{}0$ space. 

In particular, the set in $\EE^2$ bounded by closed simple rectifiable curve equipped with induced length metric is  a $\cCat{}0$ space. 
\end{thm}



\begin{thm}{Exercise}
Let $S$ be a 2-dimensional saddle surface in $\EE^n$ which is homeomorphic to a disc.
Show that any closed curve in $S$
of length $\ell$
bounds the area at most $c_n\cdot\ell^2$, 
where $c_n$ is a constant which depend only on $n$.
\end{thm}

\begin{thm}{Exercise}
Let $\phi_1,\phi_2,\dots,\phi_k\:\EE^m\to \EE^m$ be commuting short retractions.
Set $A_i=\Im \phi_i$ for all $i$;
note that each $A_i$ is a weakly convex set.

Assume $\Gamma$ is a finite graph 
(without loops and multiple edges) 
with edges labeled by $A_1,A_2,\dots, A_n$.
Denote by $\spc{X}$ the space obtained by taking 
a copy of $\EE^m$ for each vertex of $\Gamma$ and 
gluing two such copies along $A_i$ if the corresponding vertices are joint by an edge labeled by $A_i$.

Show that $\spc{X}\in\cCat{}{0}$
\end{thm}



