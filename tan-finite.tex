%%%%%%%%%%%%%%%%%%%%%%%%%%%%%%%%%%%%%%%%%%%%%%%%%%%%%%%%%%%%%%%%%%%%%%%%%%%%

\chapter{Tangent space in the finite dimensional case}

\section{Finite dimensional case}

In this section we will show that in finite dimensional case the notion of tangent space behaves on a more reasonable way.
Note however, 
we use tangent cone in the proof of properties of dimension of Alexandrov space.
Thus, the previous sections of this chapter are nesessury for complete understanding the finite dimensional case.

\begin{thm}{Theorem}\label{thm:tan4finite} 
Let $\spc{L}\in \CBB m\kappa$. 
Then for any point $p\in \spc{L}$, 
$\T_p\in\CBB m0$, $\Sigma_p\in\CBB{m-1}1$ and moreover $\T_p=\T^\o_p$.
\end{thm}


This theorem gives a possibility of making definitions and proving theorems by induction on the dimension of the space.
Often a problem in Riemannian geometry is solved by 
reducing to a problem in Eiemannian geometry 
in a tangent space at a point.
In Alexandrov's geometry, instead one can reduce the problem 
to a space of directions and uses induction on dimension. 
For instance, see the  
definition of boundary (\ref{sec:bry}), 
in the proof of existence of quasigeodesics, in particular the part connected with Milka's lemma (\ref{lem:milka}) ???MORE???.

\begin{thm}{Technical lemma}\label{lem:tan-tech}
Let $\spc{L}_n\in \CBB{}{\kappa}$ and $(\spc{L}_n,p_n)\GHto(\spc{L},p)$.
Then, given $\eps>0$ for any array of diections $\xi_1,\xi_2,\dots,\xi_\kay\in\Sigma_p$
there is a sequence of  arrays $\xi_{1,n},\xi_{2,n},\dots,\xi_{\kay,n}\in\Sigma_{p_n}$
such that \[\mangle(\xi_{i,n},\xi_{j,n})>\mangle(\xi_i,\xi_j)-\eps\]
for all $i$, $j$ and all large $n$.
\end{thm}

\parit{Proof.}
For each $\xi_i$, choose $x_i\in \spc{L}$ so that $\dir{p}{x_i}$ sufficiently close to $\xi_i$ 
(say $\mangle(\dir{p}{x_i},\xi_i)<\tfrac\eps2$).
By the definition of angle (Section~\ref{sec:angles}), 
one can choose $\bar x_i\in\l]px_i\r[$ such that
$\angk{\kappa}{p}{\bar x_i}{\bar x_j}$ is sufficiently close to $\mangle\hinge{p}{\bar x_i}{\bar x_j}$.
In particular,
\[\angk{\kappa}{p}{\bar x_i}{\bar x_j}>\mangle(\xi_i,\xi_j)-\eps\] 
for all $i,j$.

For each $i$, choose a sequence $\Str(p_n)\ni \bar x_{i,n}\to \bar x_i\in\spc{L}$.
Clearly, $\angk{\kappa}{p_n}{\bar x_{i,n}}{\bar x_{j,n}}\to\angk{\kappa}{p}{\bar x_i}{\bar x_j}$.

Therefore, for all large $n$, we have 
\[\mangle\hinge{p_n}{\bar x_{i,n}}{\bar x_{j,n}}\ge\angk{\kappa}{p_n}{\bar x_{i,n}}{\bar x_{j,n}}
>
\mangle(\xi_i,\xi_j)-\eps.\]
Thus, one can take $\xi_{i,n}=\dir{p_n}{\bar x_{i,n}}$.
\qeds

\begin{thm}{Lemma}\label{lem:comp-sigma}
Let $\spc{L}\in \CBB m\kappa$ and $p\in \spc{L}$. 
Then $\Sigma_p\spc{L}$ is compact and $\T_p\spc{L}$ is proper.
\end{thm}

The proof is very similar to the proof of \ref{E-comeagre}???.

\parit{Proof.} Assume $\Sigma_p$ is not compact.
Then for some fixed $\delta>0$ and any $\kay\in\NN$ 
one could find an array of directions $\xi_1,\xi_2,\dots,\xi_\kay\in \Sigma_p$, 
such that $\mangle(\xi_i,\xi_j)>\delta$ for all $i\not=j$.

Applying the technical lemma (\ref{lem:tan-tech}) for $\spc{L}_n=\spc{L}$ and $p_n\to p$, 
we get that 
\[\pack_\delta\Sigma_q\ge\kay\] 
for any point $q$ sufficiently close to $p$.
According to \ref{LinDim+-f}, one can choose $q$ arbitrary close $p$ so that $\Sigma_q\iso\SS^{m-1}$.
Thus, one arrives to a contradiction for $\kay>\pack_\delta\SS^{m-1}$.

Finally, $\T_p$ is a proper space since $\T_p=\Cone\Sigma_p$ ???why???.
\qeds

\parit{Proof of \ref{thm:tan4finite}.} 
According to theorem \ref{thm:tan-is-CBB}, 
there is a natural distance preserving map $\iota: \T_p\hookrightarrow \T^\o_p$.
Thus it is enough to show that  $\iota(\T_p)=\T^\o_p$.
I.e., given $v_\o\in \T^\o_p$ we have to construct $v\in \T_p$ such that $\iota(v)=v_\o$.

Fix $v_\o\in \T^\o_p$, and choose a sequence $v_n\in \Str(p)$, such that $n\cdot\dist[{{}}]{p}{v_n}{}$ is bounded 
and $n\blow v_n\to v_\o\in\T^\o_p$ as $n\to\o$.
Set 
\[v=\lim_{n\to\o} n\cdot\ddir{p}{v_n}.\]
Since $\T_p$ is proper (\ref{lem:comp-sigma}) and $n\dist[{{}}]{p}{v_n}{}$ is bounded, this limit is defined.

From \ref{thm:T-in-T^w}, we get??? that $\iota(v)=v_\o$
\qeds

\begin{thm}{Corollary}\label{thm:tan-split} 
Let $\spc{L}\in\CBB m\kappa$, $p\in \spc{L}$.
Then
\[\T_{p}\iso \Lin_p \oplus \Lin_p^\perp,\] 
where $\Lin_p$ is the linear subspace of $\T_p$ (see  defininiton \ref{def:opp+Lin}) 
and $\Lin_p^\perp$ are subcone of $\T_x$ perpendicular to $\Lin_p$.
\end{thm}

Note that in the infinite dimesional case, this question remains open, 
see \ref{open:split-inf-dim}.

\parit{Proof.}
Follows from splitting theorem \ref{thm:splitting} and \ref{thm:tan4finite}.
\qeds

\begin{thm}{Corollary}
Let $\spc{L}\in\CBB m\kappa$, $p\in \spc{L}$.
Assume $\dim\Lin_p= m-1$ then $\T_p\iso\RR_{\ge0}\times\RR^{m-1}$ and in particular $p\in\partial \spc{L}$.
\end{thm}

\parit{Proof.}
From Corollary~\ref{thm:tan-split}, 
we get 
\[\T_{p}\iso \Lin_p^\perp \oplus \Lin_p.\]
Thus $\dim \Lin_p^\perp=1$ 
and according to ???, $\Lin_p^\perp$ is isometric to a Riemannian manifold with possibly non-empty boundary.
Since $\Lin_p^\perp$ is also a cone, it might be isometric either to $\RR$ or $\RR_{\ge0}$.
In the first case we get that $\T_p\iso\RR^m$, i.e. $\dim\Lin_p=m$, a contradiction.
Therefore $\T_p\iso\RR_{\ge0}\times\RR^{m-1}$, and in particular $p\in \partial\spc{L}$.
\qeds

As a corollary of above theorem and ???, we get that $\lam\blow(\spc{L},p)\GHto(\T_p,0)$ as $\lam\to\infty$,
the following theorem says that there is an almost canonical choice Hausdorff approximations $\lam\blow L\to \T_p$.

\begin{thm}{Theorem}\label{thm:approx4tan} 
Let $\spc{L}\in\CBB{m}{}$ 
and $p\in \spc{L}$.
Then there is a Gromov--Hausdorff convergence $\lam\blow\spc{L}\xGHto{\tau}\T_p$ as $\lam\to\infty$
such that 
\begin{subthm}{}
for any map $\alpha\:[0,\eps)\to \spc{L}$ with $\alpha(0)=p$ and $\alpha^+(0)=v\in\T_p$,
we have $\lam\cdot\alpha(\tfrac1\lam)\xto{\tau} v$.
\end{subthm}

\begin{subthm}{}
Assume $f\:\spc{L}\to\RR$ is a semiconcave locally Lipschitz subfintion such that $p\in\Dom f$.
Define $w_\lam\:\lam\blow\spc{L}\to\RR\:\lam\blow x\mapsto\lam\cdot[f(x)-f(p)]$ then $w_\lam\xto\tau\d_pf$.
\end{subthm}
\end{thm}

Further, the convergence $\tau_p$ as in this theorem will be called \emph{canonical convergence}\index{canonical convergence}.

\parit{Proof.}
???
\qeds

\begin{thm}{Semicontinuity of space of directions}\label{thm:simicont-Sigma}
Let $\spc{L}_n\in\CBB m\kappa$ and $\spc{L}_n\xGHto{a_n} \spc{L}$ and $\spc{L}_n\ni p_n\to p\in \spc{L}$.
Then for any $\eps>0$,
we have $\Sigma_{p_n}>\Sigma_p-\eps$ for all large $n$.
(In other words, 
if $\Sigma_{p_n}\GHto\Sigma$ then $\Sigma\ge \Sigma_p$.)
\end{thm}

\parit{Proof of \ref{thm:simicont-Sigma}.}
Accordinng to Lemma~\ref{lem:comp-sigma}, $\Sigma_p$ is compact.
Thus, we can choose a finite $\tfrac\eps2$-net $\xi_1,\xi_2,\dots,\xi_\kay$ in $\Sigma_p$. 
According to the technical lemma (\ref{lem:tan-tech}), there is a sequence of arrays $\xi_{1,n},\xi_{2,n},\dots,\xi_{\kay,n}$ in $\Sigma_{p_n}$ 
such that $\mangle(\xi_{i,n},\xi_{j,n})\z>\mangle(\xi_i,\xi_j)-\tfrac\eps2$ for all $i$ and $j$ and large enough $n$.

Define a sequence of maps $\map_n\:\Sigma_p\to\Sigma_{p_n}$ the following way:
for each $\zeta\in\Sigma_p$ choose a $\xi_i$ such that $\mangle(\xi_i,\zeta)<\tfrac\eps2$ and set $\map_n(\zeta)=\xi_{i,n}\in\Sigma_{p_n}$.
From above, for all large $n$, we have $\mangle(\map_n(\zeta),\map_n(\zeta'))>\mangle(\zeta,\zeta')-\eps$ for any $\zeta,\zeta'\in \Sigma_p$;
i.e. $\Sigma_{p_n}>\Sigma_p-\eps$ for all large $n$.\qeds

\begin{thm}{Corollary}\label{cor:simicont-Sigma}
Let $\spc{L}\in\CBB m\kappa$ and $p_n\to p$ be a converging sequence of points in $\spc{L}$.
Then for any $\eps>0$, $\Sigma_{p_n}>\Sigma -\eps$ for all large $n$.

Moreover, if $\dir{p}{p_n}\to\xi\in \Sigma_p$ then $\Sigma_{p_n}>\Sigma_\xi \T_p-\eps$ for all large $n$.
\end{thm}

We use this corollary to get Lemma~\ref{lem:amost=Sigma}.
Also, since the set of Euclidean points is dense (\ref{LinDim+-f})
the above corollary implies the following.

\begin{thm}{Corollary}\label{cor:S>Sigma}
Let $L\in\CBB m\kappa$.
Then for any $p\in \spc{L}$, 
$\SS^{m-1}\ge \Sigma_p$.
\end{thm}

\parit{Proof of \ref{cor:simicont-Sigma}.}
Both parts obtained by applying Theorem~\ref{thm:simicont-Sigma} to particular data.
For the first part, use constant sequence $\spc{L}_n=\spc{L}$.
For the second part, use $\tfrac{1}{\dist{p}{p_n}{}}\blow\spc{L}\GHto \T_p$ 
and 
$\tfrac{1}{\dist{p}{p_n}{}}\blow p_n\to\xi$.%???NOTATION OF GH-CONVERGENCE HAS TO BE FIXED???
\qeds

\begin{thm}{Semicontinuity of space of differential}\label{thm:simicont-differential}
Let $\spc{L}_n\in\CBB m\kappa$ and  $a_n,a_\infty, p\in \spc{L}$.
Assume $a_n\to a_\infty$ as $n\to\infty$.
Then $\liminf\d_p\dist{a_n}{}{}\le \d_p\dist{a_\infty}{}{}$
\end{thm}

\begin{thm}{Sum lemma}
Let $\spc{L}_n\in\CBB m\kappa$ and  $p\in \spc{L}$ and $v^1,v^2,\dots,v^\kay\in \T_p$.
Then there is a vector $w\in \T_p$ such that for any semiconcave locally Lipschitz subfunction 
$f\:\spc{L}\subto \RR$
we have
$\d_pf(w)\ge\sum_i\d_pf(v^i)$.
\end{thm}




\section{Almost parallel sets}

\begin{thm}{Definition}\label{def:parallel-set}
Let $\spc{L}\in\CBB{m}{}$.
A nonempty set $P\subset\spc{L}$ is called $\eps$-parallel
if there is a point array $(a^1,a^2,\dots,a^n)$ in  $\spc{L}$
such that the following two conditions hold.

\begin{subthm}{def:parallel-set:net}
For any $p\in P$,
the set of directions $\{\dir{p}{a^1},\dir{p}{a^2},\dots,\dots,\dir{p}{a^n}\}$
forms an $\eps$-dense set in $\Sigma_p$.
\end{subthm}

\begin{subthm}{def:parallel-set:almost}
For any $i,j$ and $p,q\in P$ we have 
$$|\mangle\hinge{p}{a^i}{a^j}-\mangle\hinge{q}{a^i}{a^j}|\le\eps.$$
\end{subthm}
\end{thm}

\parbf{Remarks.}
\begin{itemize}

\item From the definition it follows that 
that $\Sigma_p$ is $(3\cdot\eps)$-isometric to $\Sigma_q$.
Indeed, it is straightforward to check that 
there is map $f\:\Sigma_p\to \Sigma_q$ which sends a direction $\xi\in \Sigma_p$
to  $\dir{q}{a^i}\in \Sigma_q$ 
such that $\mangle(\dir{p}{a^i},\xi)<\eps$
and any such map $f$ is a $(3\cdot\eps)$-isometry.


\item Note that any one-point set $\{p\}$ in $\spc{L}$
forms an $\eps$-parallel for any $\eps>0$.
To prove the later one has to construct a point array $(a^i)$
which satisfies (\ref{SHORT.def:parallel-set:net}). 
\end{itemize}




\begin{thm}{Proposition}
Given a positve integer $m$,
for any $\eps>0$ there is $\delta>0$
such that the following holds.
If $\spc{L}\in\CBB{}{}$ and $X\subset\spc{L}$ is an $\delta$-parallel set
then $X$ admits a $e^{\pm\eps}$-bi-Lipschitz embedding in $\EE^\kay$,
where $\kay$ is the maximal integer such that
$$\Sigma_x\ge\SS^{\kay-1}-\eps$$
for some $x\in X$.
\end{thm}

\parit{Proof.}
Choose a maximal subarray $(c,b^1\dots,b^\kay)$ of $\bm{a}$
such that ???.
Let us show that the restriction of distance map 
$\dist{\bm{b}}{}{}\:\spc{L}\to\RR^\kay$ to
$X$
satisfies the proposition.

According to ??? $\dist{\bm{b}}{}{}$ is ???-Lipschitz.
It remains to show that 
$$|\dist{\bm{b}}{x}{}-\dist{\bm{b}}{y}{}|>???\cdot\dist{x}{y}{\spc{L}}$$
for any $x,y\in X$.



Fix $x,y\in X$.
Note that 
$$\mangle\hinge{x}{a^i}{y}+\mangle\hinge{y}{a^i}{x}
>
\pi-???$$
for any $a^i$.
Let $a^{i}$ and $a^{j}$ be the elements of the array $\bm{a}$ such that
$$\mangle\hinge{x}{a^{i}}{y},
\ 
\mangle\hinge{y}{a^{j}}{x}
<
\eps.$$
From above it follwos that 
$\mangle\hinge{x}{a^i}{a^j}>\pi-???$
???
\qeds



\begin{thm}{Theorem}
Let $\spc{L}\in\CBB{m}{}$ 
and $\mathfrak{C}\subset\spc{L}$ be a closed set.
Given $\eps>0$
there is $p\in\mathfrak{C}$ and $r>0$
such that 
$\oBall(p,r)\cap\mathfrak{C}$
is $\eps$-parallel.
\end{thm}

\begin{thm}{Corollary}
Let $\spc{L}\in\CBB{m}{}$.
Given $\eps>0$,
there is a covering of $\spc{L}$ by countable number of closed  $\eps$-parallel sets. 
 
\end{thm}


\parit{Proof.}
Let $n=\max\set{\pack_\eps\Sigma_x}{x\in\mathfrak{C}}$.
Consider set 
\[\mathfrak{C}_n=\set{x\in\mathfrak{C}}{\pack_\eps\Sigma_x=n}.\]
According to \ref{cor:rank>=k-open}, $\mathfrak{C}_n$ is open rel $\mathfrak{C}$.
Define a function $s\:\mathfrak{C}_n\to\RR$ as
\[s(x)
=
\sup\set{\sum_{i<j}\mangle(\xi^i,\xi^j)}%
{\begin{aligned}
\{&\xi^i\}\ \t{is a maximal}
\\
&\eps\t{-packing in}\ \Sigma_x
 \end{aligned}
}.\]

Set $s=\sup\set{s(x)}{x\in \mathfrak{C}_n}$.
Let us choose a point 
$p\in\mathfrak{C}_n$ such that $s(p)$ is almost maximal,
say 
\[s(p)>s-\tfrac{\eps}{10\cdot n^2}.\]
Choose a maximal $\eps$-packing $(\xi^1,\xi^2,\dots,\xi^n)$ of $\Sigma_p$ 
such that 
\[\sum_{i<j}\mangle(\xi^i,\xi^j)>s-\tfrac{\eps}{10\cdot n^2}.\eqlbl{eq:almost-max}\]
Without loss of genrality, we can assume that each $\xi^i$
is a geodesic direction;
and 
moreover
\begin{itemize}
\item For each $i$ there is $a^i\in\spc{L}$
such that $\dir p{a^i}$ is uniquely defined and $\xi^i=\dir p{a^i}$.
\item For each $i\ne j$,  
$$\angk\kappa p{a^i}{a^j}+\tfrac{\eps}{10\cdot n^2}>\mangle\hinge p{a^i}{a^j}\ \ 
\text{and}
\ \ \angk\kappa p{a^i}{a^j}>\eps.$$
\end{itemize}
  

Fix sufficiently small $r>0$
and arbitrary $q\in \oBall(p,r)\cap \mathfrak{C}$.
Since $\mathfrak{C}_n$ is open rel $\mathfrak{C}$,
we can assume that $\pack_\eps\Sigma_q=n$.
Choose
 geodesics $[qa^i]$ 
and set $\xi^i_q=\dir q{a^i}$.
(These geodesics exist according to \ref{cor:dim>proper}.)
From above we have that
we get that $\{\xi^1_q,\xi^2_q,\dots,\xi^n_q\}$ is a maximal $\eps$-packing, and therefore $\eps$-net in $\Sigma_q$.
I.e., the condition \ref{def:parallel-set:net} follows.

Since $r$ is small, we also have that
\[\mangle(\xi^i_q,\xi^j_q)>\mangle(\xi^i,\xi^j)-\tfrac{\eps}{10\cdot n^2}.\] 
\[\sum_{i<j}\mangle(\xi^i_q,\xi^j_q)\le s.\]
These two inequalities together with \ref{eq:almost-max} imply \ref{def:parallel-set:almost}.
\qeds
%%%%%%%%%%%%%%%%%%%%%%%%%%%%%%%%%%%%%%%%%%%%%%%%%%%%%%%%%%%%%%%%%%%%%%%%




\section{Remarks and open problems}

It is not known whether exist a proper space with the same property as in Halbeisen's example (section \ref{halbeisen}).
In other words, 
is it possible to  prove the first conclusion of Theorem~\ref{thm:tan4finite} 
for infinite dimensional proper $\CBB{}{}$-spaces?
Namely:

\begin{thm}{Open question}\label{open:Halb-proper}
Assume $\spc{L}\in\CBB{}{}$ is proper. 
Is it true that the tangent cone $\T_p$ belongs to $\CBB{}{0}$ for any $p\in\spc{L}$?
\end{thm}

\begin{thm}{Open question}\label{open:split-inf-dim}
Let $\spc{L}\in\CBB{}{\kappa}$ and $p\in \spc{L}$.

Is it true that $\T_p$ spits isometrically
\[\T_p=\Lin_p^\perp\oplus\Lin_p\]
where 
$\Lin_p^\perp=\set{v\in \T_p}{\<v,x\>=0\ \t{for any}\ x\in\Lin_p}$?
\end{thm}

Compare to Theorem~\ref{thm:tan-split}


\section{Exercises}

\begin{thm}{Exercise}
Let $\spc{L}\in\CBB{}{}$, assume $[pq_i]$ covrege as to $[pq]$, $p\not=q$. 

Show that if $\dim\spc{L}<\infty$ then $\mangle\hinge p{q_i}{q}\to0$.

Show that if $\dim\spc{L}=\infty$ that is not longer true.
\end{thm}




