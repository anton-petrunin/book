%%!TEX root = the-baricentric-simplex.tex
\chapter{Dimension of CAT spaces}\label{chap:web+bary}

In this chapter we discuss  constructions introduced by Bruce Kleiner \cite{kleiner}.

The material of this chapter is used mostly for $\CAT{}$ spaces, 
but the results in section~\ref{sec:web-general} find some applications for finite-dimensional $\Alex{}$ spaces as well.

\section{The case of complete geodesic spaces}\label{sec:web-general}

The following construction gives a $\kay$-dimensional submanifold 
for a given ``nondegenerate'' array of $\kay+1$ strongly convex functions.

\begin{thm}{Definition}\label{def:ordung}
For two real arrays $\bm{v}$, $\bm{w}\in \RR^{\kay+1}$,
$\bm{v}=(v^0,v^1,\dots,v^\kay)$ 
and 
$\bm{w}=(w^0,w^1,\dots,w^\kay)$, 
we will write
$\bm{v}\succcurlyeq\bm{w}$ if $v^i\ge w^i$ for each $i$.
\end{thm}

Given a subset $Q\subset \RR^{\kay+1}$, 
denote by $\Up Q$ \label{PAGE.def:Up}
the smallest upper set containing $Q$,  
and by 
$\Min Q$ the set of minimal elements of $Q$ with respect to $\succcurlyeq$;
that is,
\begin{align*}
\Up Q 
&=
\set{\bm{v}\in\RR^{\kay+1}}{\exists\, \bm{w}\in Q\ \text{such that}\ \bm{v}\succcurlyeq\bm{w}},
\\
\Min Q 
&=
\set{\bm{v}\in Q}{\text{if}\ \bm{v}\succcurlyeq\bm{w}\in Q\ \text{then}\ \bm{w}=\bm{v}}.
\end{align*}


\begin{thm}{Definition}\label{def:web}
Let  $\bm{f}=(f^0,f^1,\dots,f^\kay)\:\spc{X}\to \RR^{\kay+1}$ be a function array on a metric space $\spc{X}$.
The set 
\[\Web\bm{f}
\df
\bm{f}^{-1}\left[\Min\bm{f}(\spc{X})\right]
\subset 
\spc{X}\] 
will be called the \index{web}\emph{web} of $\bm{f}$.
\end{thm}

Given an array $\bm{f}=(f^0,f^1,\dots,f^\kay)$,
we denote by $\bm{f}^{\without i}$ the subarray of $\bm{f}$ with $f^i$ removed;
that is, 
\[\bm{f}^{\without i\,}\df(f^0,\dots,f^{i-1},f^{i+1},\dots,f^\kay).\]
Clearly 
$\Web\bm{f}^{\without i}\subset \Web\bm{f}$.
Define the \index{web!inner web}\emph{inner web} of $\bm{f}$ 
as 
\[\InWeb\bm{f}
=
\Web\bm{f}\backslash\left(\bigcup_{i}\Web\bm{f}^{\without i}\right).\]


We say that a function array is \emph{nondegenerate} 
if $\InWeb\bm{f}\not=\emptyset$.

\parbf{Example.} 
If $\spc{X}$ is a geodesic space, 
then $\Web(\distfun{x}{}{},\distfun{y}{}{})$ is the union of all geodesics from $x$ to $y$, and 
\[\InWeb(\distfun{x}{}{},\distfun{y}{}{})=\Web(\distfun{x}{}{},\distfun{y}{}{})\backslash\{x,y\}.\]

\parbf{Barycenters.}
Let us denote by $\Delta^\kay\subset \RR^{\kay+1}$\index{$\Delta^m$} 
the \index{standard simplex}\emph{standard $\kay$-simplex}; 
that is, $\bm{x}=(x^0,x^1,\dots,x^\kay)\in\Delta^\kay$ if $\sum_{i=0}^\kay x^i=1$ and $x^i\ge0$ for all $i$.

Let $\spc{X}$ be a metric space 
and $\bm{f}=(f^0,f^1,\dots,f^\kay)\:\spc{X}\to \RR^{\kay+1}$ be a function array.
Consider the map $\spx{\bm{f}}\:\Delta^\kay\to \spc{X}$\index{$\spx{\bm{f}}$} defined by 
\[\spx{\bm{f}}(\bm{x})=\argmin\sum_{i=0}^\kay x^i\cdot f^i,\]
where $\argmin f$\index{$\argmin$} denotes a point of minimum of $f$.
The map $\spx{\bm{f}}$ will be called a \index{barycentric simplex of function array}\emph{barycentric simplex} of $\bm{f}$.
Note that for a general function array $\bm{f}$, 
the value $\spx{\bm{f}}(\bm{x})$ might be undefined or nonuniquely defined.

It is clear from the definition that $\spx{\bm{f}^{\without i}}$ 
coincides with the restriction of $\spx{\bm{f}}$ to the corresponding face of $\Delta^\kay$.


\begin{thm}{Theorem}\label{thm:web}
Let $\spc{X}$ be a complete geodesic space 
and $\bm{f}\z=(f^0,f^1,\dots,f^\kay)\:\spc{X}\to\RR^{\kay+1}$ 
be an array of strongly convex and locally Lipschitz functions.
Then $\bm{f}$ defines a $C^{\frac12}$-embedding 
$\Web\bm{f}\hookrightarrow\RR^{\kay+1}$.

Moreover,
\begin{subthm}{thm:web:Up-convex}
$W=\Up[\bm{f}(\spc{X})]$ is a convex closed subset of $\RR^{\kay+1}$,
and

$S=\Fr_{\RR^{\kay+1}} W$ is a convex hypersurface in $\RR^{\kay+1}$.
\end{subthm}

\begin{subthm}{thm:web:f(web)=min}
\[\bm{f}(\Web\bm{f})=\Min W \subset S\]
and
\[\bm{f}(\InWeb\bm{f})= \Int_{S}(\Min W).\]
\end{subthm}

\begin{subthm}{thm:web:bary}
The barycentric simplex 
$\spx{\bm{f}}\:\Delta^\kay\to \spc{X}$ is a uniquely defined Lipshitz map and $\Im\spx{\bm{f}}=\Web\bm{f}$.
In particular $\Web\bm{f}$ is compact.
\end{subthm}

\begin{subthm}{thm:web:lip-const}
Let us equip $\Delta^\kay$ with the metric induced by the $\ell^1$-norm on $\RR^{\kay+1}$.
Then the Lipschitz constant of $\spx{\bm{f}}\:\Delta^\kay\to\spc{U}$ can be estimated in terms of 
positive lower bounds on $(f^i)''$ 
and Lipschitz constants of $f^i$
in a neighborhood of $\Web\bm{f}$ for all $i$.
\end{subthm}


In particular, by (\ref{SHORT.thm:web:Up-convex}) and (\ref{SHORT.thm:web:f(web)=min}),  $\InWeb\bm{f}$ is $C^{\frac12}$-homeomorphic to an open set of $\RR^\kay$.
\end{thm}

The proof is preceded by a few preliminary statements.

\begin{thm}{Lemma}\label{lem:argmin(convex)}
Suppose $\spc{X}$ is a complete geodesic space and $f\:\spc{X}\to\RR$ is a locally Lipschitz, strongly convex function. Then the minimum point 
of $f$
is uniquely defined.
\end{thm}

\parit{Proof.}
Without loss of generality, we can assume that $f$ is $1$-convex.
In particular, the following claim holds:
\begin{clm}{}\label{midpoint}
 if $z$ is a midpoint of the geodesic $[x y]$, then 
\[s\le f(z)
\le
\tfrac{1}{2}\cdot f(x)+\tfrac{1}{2}\cdot f(y)-\tfrac{1}{8}\cdot\dist[2]{x}{y}{},
\]
where $s$ is the infimum of $f$.
\end{clm}

\parit{Uniqueness.}
Assume that $x$ and $y$ are distinct minimum points of $f$. 
From \ref{midpoint} we have
\[f(z)<f(x)=f(y),\] 
a contradiction. 

\parit{Existence.}
Fix a point $p\in \spc{X}$, and
let $\Lip\in\RR$ be a Lipschitz constant of $f$ in a neighborhood of $p$.

Consider the function $\phi(t)=f\circ\geod_{[px]}(t)$.
Clearly $\phi$ is $1$-convex and $\phi^+(0)\ge -\Lip$.
Setting $\ell=\dist{p}{x}{}$, we have 
\begin{align*}
f(x)
&=
\phi(\ell)
\ge
\\
&\ge
f(p)-\Lip\cdot\ell+\tfrac{1}{2}\cdot\ell^2
\ge
\\
&\ge f(p)-\tfrac{1}{2}\cdot{\Lip^2}.
\end{align*}

In particular,
\begin{align*}
s
&\df
\inf\set{f(x)}{x\in \spc{X}}
\ge
\\
&\ge
f(p)-\tfrac{1}{2}\cdot{\Lip^2}.
\end{align*}

Choose a sequence of points $p_n\in \spc{X}$ such that $f(p_n)\to s$.
Applying \ref{midpoint} for $x\z=p_n$, $y\z=p_m$, we see that $(p_n)$ is Cauchy. Thus $p_n$ converges to a minimum point of $f$.
\qeds

\begin{thm}{Definition}
Let $Q$ be a closed subset of $\RR^{\kay+1}$.
A vector $\bm{x}\z=(x^0,x^1,\dots,x^\kay)\in\RR^{\kay+1}$
is \index{subnormal vector}\emph{subnormal} to $Q$ at a point $\bm{v}\in Q$ 
if
\[\<\bm{x},\bm{w}-\bm{v}\>
\df
\sum_ix^i\cdot(w^i-v^i)
\ge 0\]
for any $\bm{w}\in Q$.
\end{thm}


\begin{thm}{Lemma}\label{lem:Up-convex}{\sloppy 
Let $\spc{X}$ be a complete geodesic space 
and $\bm{f}\z=(f^0,f^1,\dots,f^\kay)\:\spc{X}\to\RR^{\kay+1}$ 
be an array of strongly convex and locally Lipschitz functions.
Let $W=\Up\bm{f}(\spc{X})$.
Then: 

}
\begin{subthm}{lem:Up-convex:Up-convex}
$W$ is a closed convex set, bounded below with respect to $\succcurlyeq$.
\end{subthm}

\begin{subthm}{lem:Up-convex:subnormal}
If $\bm{x}$ is a subnormal vector to $W$, then $\bm{x}\succcurlyeq\bm{0}$.
\end{subthm}

\begin{subthm}{lem:Up-convex:surface}
 $S=\Fr_{\RR^{\kay+1}}W$ is a complete convex hypersurface in $\RR^{\kay+1}$.
\end{subthm}

\end{thm}

\parit{Proof.}
Denote by $\bar W$ the closure of $W$.

Convexity of the $f^i$ implies that
for any two points $p,q\in \spc{X}$ and $t\in[0,1]$ we have
\[(1-t)\cdot\bm{f}(p)+t\cdot \bm{f}(q)
\succcurlyeq
\bm{f}\circ\geodpath_{[p q ]}(t),
\eqlbl{n-convex}\]
where $\geodpath_{[p q]}$ denotes a geodesic path from $p$ to $q$. 
Therefore $W$, as well as $\bar W$, are convex sets in $\RR^{\kay+1}$.


Let
\[w^i=\min\set{f^i(x)}{x\in\spc{X}}.\]
By Lemma~\ref{lem:argmin(convex)}, $w^i$ is finite for each $i$.
Clearly $\bm{w}=(w^0,w^1,\dots,w^\kay)$ is a lower bound of $\bar W$ with respect to $\succcurlyeq$.

It is clear that $W$ has nonempty interior,
and $W\not=\RR^{\kay+1}$ since $W$ is bounded below.
Therefore $S=\Fr_{\RR^{\kay+1}}W=\Fr_{\RR^{\kay+1}}\bar W$
is a complete convex hypersurface in $\RR^{\kay+1}$.

Since $\bar W$ is closed and bounded below, we also have
\[\bar W=\Up[\Min\bar W].
\eqlbl{eq:W=Up Min W}\]

Choose an arbitrary $\bm{v}\in S$.
Let $\bm{x}\in\RR^{\kay+1}$ be a subnormal vector to $\bar W$ at $\bm{v}$. 
In particular, 
$\<\bm{x},\bm{y}\>
\ge
0$ 
for any $\bm{y}\succcurlyeq\bm{0}$;
that is, $\bm{x}\succcurlyeq\bm{0}$.

Further, according to Lemma~\ref{lem:argmin(convex)}, 
the function 
$\sum_i x^i\cdot f^i$ has a uniquely defined minimum point, say $p$.
Clearly 
\[\bm{v}\succcurlyeq\bm{f}(p)\quad\text{and}\quad \bm{f}(p)\in \Min W.\eqlbl{eq:v>f(p)}\]

Note that for any $\bm{u}\in \bar W$ there is $\bm{v}\in S$ such that $\bm{u}\succcurlyeq\bm{v}$. 
Therefore \ref{eq:v>f(p)} implies 
\[\bar W\subset\Up[\Min W]\subset W.\]
Hence
$\bar W=W$; that is, $W$ is closed.
\qeds









\parit{Proof of \ref{thm:web}}; \textit{(\ref{SHORT.thm:web:Up-convex})+(\ref{SHORT.thm:web:f(web)=min}).}
Without loss of generality we may assume that all $f^i$ are $1$-convex.

Given $\bm{v}=(v^0,v^1,\dots,v^\kay)\in\RR^{\kay+1}$, consider the function 
$h_{\bm{v}}\: \spc{X}\to \RR$ defined by
\[h_{\bm{v}}(p)=\max_i\{f^i(p)-v^i\}.\]
Note that $h_{\bm{v}}$ is $1$-convex.
Let 
$$\map(\bm{v})\df\argmin h_{\bm{v}}.$$
According to Lemma~\ref{lem:argmin(convex)}, $\map(\bm{v})$ is uniquely defined.

From the definition of web (\ref{def:web}) 
we have
$\map\circ\bm{f}(p)=p$ for any $p\in \Web\bm{f}$;
that is, $\map$ is a left inverse to the restriction $\bm{f}|_{\Web\bm{f}}$.
In particular, 
\[\Web\bm{f}=\Im\map.
\eqlbl{eq:Web=Im}\]

Given $\bm{v},\bm{w}\in\RR^{\kay+1}$,
set $p=\map (\bm{v})$ and $q=\map (\bm{w})$.
Since $h_{\bm{v}}$ and $h_{\bm{w}}$ are 1-convex, we have
\begin{align*}
h_{\bm{v}}(q)
&\ge 
h_{\bm{v}}(p)+\tfrac{1}{2}\cdot\dist[2]{p}{q}{},
&
h_{\bm{w}}(p)
&\ge 
h_{\bm{w}}(q)+\tfrac{1}{2}\cdot\dist[2]{p}{q}{}.
\end{align*}
Therefore,
\begin{align*}
\dist[2]{p}{q}{}
&\le 
2\cdot\sup_{x\in\spc{X}}\{ |h_{\bm{v}}(x)-h_{\bm{w}}(x)| \}
\le
\\
&\le 
2\cdot\max_{i}\{|v^i-w^i|\}.
\end{align*}
In particular,
$\map$ is $C^{\frac{1}{2}}$-continuous,
or $\bm{f}|_{\Web\bm{f}}$ is a $C^{\frac{1}{2}}$-embedding.

As in Lemma~\ref{lem:Up-convex},
let $W=\Up\bm{f}(\spc{X})$ and $S=\Fr_{\RR^{\kay+1}}W$.
Then
$S$ is a convex hypersurface in $\RR^{\kay+1}$.
Clearly $\bm{f}(\Web\bm{f})\z=\Min W\subset S$.
From the definition of inner web, we have
$\bm{v}\in \bm{f}(\InWeb\bm{f})$ 
if and only if 
$\bm{v}\in S$ and
for any $i$ there is $\bm{w}=(w^0,w^1,\dots,w^\kay)\in W$ such that $w^j<v^j$ for all $j\not=i$.
Thus $\bm{f}(\InWeb\bm{f})$ is open in $S$.
That is, $\InWeb\bm{f}$ is $C^{\frac{1}{2}}$-homeomorphic to an open set in a convex hypersurface $S\subset\RR^{\kay+1}$,
and hence to an open set of $\RR^{\kay}$, as claimed.










\parit{(\ref{SHORT.thm:web:bary})+(\ref{SHORT.thm:web:lip-const}).}
Since $f^i$ is $1$-convex, for any $\bm{x}=(x^0,x^1,\dots,x^\kay)\in\Delta^\kay$ 
the convex combination 
\[\left(\sum_i x^i\cdot f^i\right)\:\spc{X}\to\RR\] 
is also $1$-convex.
Therefore, according to Lemma~\ref{lem:argmin(convex)}, the barycentric simplex 
%$\spx{\bm{f}}(\bm{x})$ is defined for any $\bm{x}\in\Delta^\kay$.
$\spx{\bm{f}}$ is uniquely defined on $\Delta^\kay$.
 
For $\bm{x},\bm{y}\in\Delta^\kay$,
let 
\begin{align*}
f_{\bm{x}}
&=\sum_i x^i\cdot f^i,
&
f_{\bm{y}}
&=\sum_i y^i\cdot f^i,
\\
p
&=\spx{\bm{f}}(\bm{x}),
&
q
&=\spx{\bm{f}}(\bm{y}),
\\
\ell&=\dist{p}{q}{}.
\end{align*}
Note the following:
\begin{itemize}
\item The function $\phi(t)=f_{\bm{x}}\circ\geod_{[p q]}(t)$ has minimum at $0$. 

Therefore $\phi^+(0)\ge 0$
\item The function $\psi(t)=f_{\bm{y}}\circ\geod_{[p q]}(t)$ has minimum at $\ell$. 

Therefore $\psi^-(\ell)\ge 0$.
\end{itemize}
From $1$-convexity of $f_{\bm{y}}$, we have
$\psi^+(0)+\psi^-(\ell)+\ell\le0$.

Let $\Lip$ be a Lipschitz constant for all $f^i$ in a neighborhood $\Omega\ni p$.
Then 
\[\psi^+(0)
\le 
\phi^+(0)+\Lip\cdot\|\bm{x}-\bm{y}\|_1,\] 
where $\|\bm{x}-\bm{y}\|_1=\sum_{i=0}^\kay|x^i-y^i|$.
That is, given $\bm{x}\in\Delta^\kay$, there is a constant $\Lip$ such that
\begin{align*}
\dist{\spx{\bm{f}}(\bm{x})}{\spx{\bm{f}}(\bm{y})}{}
&=
\ell
\le
\\
&\le 
\Lip\cdot\|\bm{x}-\bm{y}\|_1
\end{align*}
for any $\bm{y}\in\Delta^\kay$.
In particular, there is $\eps>0$ such that if $\|\bm{x}-\bm{y}\|_1<\eps,$ $\|\bm{x}-\bm{z}\|_1 <\eps$, then $\spx{\bm{f}}(\bm{y})$, $\spx{\bm{f}}(\bm{z})\in\Omega$. 
Thus the same argument as above implies 
\[\dist{\spx{\bm{f}}(\bm{y})}{\spx{\bm{f}}(\bm{z})}{}
=
\ell\le \Lip\cdot\|\bm{y}-\bm{z}\|_1\]
for any $\bm{y}$ and $\bm{z}$ sufficiently close to $\bm{x}$; that is, $\spx{\bm{f}}$ is locally Lipschitz.
Since $\Delta^\kay$ is compact, $\spx{\bm{f}}$ is Lipschitz.

Clearly $\spx{\bm{f}}(\Delta^\kay)\subset \Web \bm{f}$.
It remains to show that $\spx{\bm{f}}(\Delta^\kay)\supset \Web \bm{f}$.
According to Lemma~\ref{lem:Up-convex},
$W=\Up\bm{f}(\spc{X})$ is a closed convex set in $\RR^{\kay+1}$.
Let $p\in \Web \bm{f}$. 
Clearly $\bm{f}(p)\in \Min W\subset S\z=\Fr_{\RR^{\kay+1}}W$.
Let $\bm{x}$ be a subnormal vector to $W$ at $\bm{f}(p)$.
According to Lemma~\ref{lem:Up-convex}, 
$\bm{x} \succcurlyeq\bm{0}$.
Without loss of generality we may assume that $\sum_i x^i=1$;
that is, $\bm{x}\in \Delta^\kay$.
By Lemma~\ref{lem:argmin(convex)},
$p$ is the unique minimum point of $\sum_i x^i\cdot f^i$;
that is, $p=\spx{\bm{f}}(\bm{x})$.
\qeds








\section{The case of CAT spaces}

Let $\bm{a}=(a^0,a^1,\dots,a^\kay)$ be a point array in a metric space $\spc{U}$.
Recall that 
$\distfun{\bm{a}}{}{}$
denotes the distance map
\[(\distfun{a^0}{}{},\distfun{a^1}{}{},\dots,\distfun{a^\kay}{}{})\:\spc{U}\to\RR^{\kay+1},\]
which can be also regarded as a function array.
The \index{radius of a point array}\emph{radius of the point array} $\bm{a}$ is defined to be the radius of the set $\{a^0,a^1,\dots,a^\kay\}$;
that is,
\[\rad\bm{a}=\inf\set{r>0}{\exists z\in\spc{U}\ \text{such that}\ a^i\in \oBall(z,r)\ \text{for any}\ i}.\]

Fix $\kappa\in\RR$.
Let $\bm{a}=(a^0,a^1,\dots,a^\kay)$ be a point array of radius $<\tfrac{\varpi\kappa}2$
in a metric space $\spc{U}$.
Consider the function array $\bm{f}=(f^0,f^1,\dots,f^\kay)$ 
where 
\[f^i(x)\z=\md\kappa\dist[{{}}]{a^i}{x}{}.\]
Assuming the barycentric simplex $\spx{\bm{f}}$ is defined,
then $\spx{\bm{f}}$ is called the \index{$\kappa$-barycentric simplex}\emph{$\kappa$-barycentric simplex} for the point array $\bm{a}$;
it will be denoted by $\spx{\bm{a}}\mc\kappa$.
The points $a^0,a^1,\dots,a^\kay$ are called 
\index{vertexes  of the $\kappa$-barycentric simplex}\emph{vertexes of the $\kappa$-barycentric simplex}.
Note that once we say the $\kappa$-barycentric simplex is defined, 
we automatically assume that $\rad\bm{a}<\tfrac{\varpi\kappa}2$.


\begin{thm}{Theorem}\label{thm:cat-bary-web}
Let $\spc{U}$ be a complete length $\CAT\kappa$ space
and $\bm{a}\z=(a^0,a^1,\dots a^\kay)$ be a point array with radius $<\tfrac{\varpi\kappa}{2}$.
Then: 

\begin{subthm}{thm:cat-bary-web:Lip}
The $\kappa$-barycentric simplex $\spx{\bm{a}}\mc\kappa\:\Delta^\kay\to \spc{U}$ 
is defined. 
Moreover, $\spx{\bm{a}}\mc\kappa$ is a Lipschitz map,
and if $\Delta^\kay$ is equipped with the $\ell^1$-metric, then its Lipschitz constant can be estimated in terms of $\kappa$ and the radius of $\bm{a}$ (in particular it does not depend on $\kay$).
\end{subthm}


\begin{subthm}{thm:cat-bary-web:web=Im(bary)}
$\Web(\distfun{\bm{a}}{}{})=\Im \spx{\bm{a}}\mc\kappa$.
Moreover, if a closed convex set $K\subset\spc{U}$ contains all $a^i$, then $\Web(\distfun{\bm{a}}{}{})\subset K$.
\end{subthm}

\begin{subthm}{thm:cat-bary-web:mnfld}
The restriction%
\footnote{Recall that $\distfun{\bm{a}^{\without 0}}{}{}$ denotes the array $(\distfun{a^1}{}{},\dots,\distfun{a^\kay}{}{})$.}
$\distfun{\bm{a}^{\without 0}}{}{}|_{\InWeb(\distfun{\bm{a}}{}{})}$ is an open $C^{\frac12}$-embedding in $\RR^\kay$.
Thus there is an inverse of 
$\distfun{\bm{a}^{\without 0}}{}{}|_{\InWeb(\distfun{\bm{a}}{}{})}$, say $\map\:\RR^\kay\subto\spc{U}$.

The subfunction $f=\distfun{a^0}{}{}\circ\map$ is semiconvex and locally Lipschitz.
Moreover, if $\kappa\le 0$, then $f$ is convex.
\end{subthm}


In particular, $\Web(\distfun{\bm{a}}{}{})$ is a compact set and
$\InWeb(\distfun{\bm{a}}{}{})$ is $C^{\frac12}$-homeomorphic to an open subset of $\RR^\kay$.

\end{thm}

\begin{thm}{Definition}\label{prop-def:web-embedding}
The submap $\map\:\RR^{\kay}\subto \spc{X}$ of Theorem~\ref{thm:cat-bary-web:mnfld}
will be called the \index{web embedding}\emph{$\distfun{\bm{a}}{}{}$-web embedding} 
with \index{brace}\emph{brace} $\distfun{a^0}{}{}$.
The terminology invokes Theorem~\ref{thm:cat-bary-web:mnfld}.
%S: Is adding this sentence helpful?
\end{thm}

\begin{thm}{Definition}
Let $\spc{U}$ be a complete length $\CAT\kappa$ space
and $\bm{a}\z=(a^0,a^1,\dots a^\kay)$ be a point array with radius $<\tfrac{\varpi\kappa}{2}$.
If $\InWeb(\distfun{\bm{a}}{}{})$ is nonempty, then the point array $\bm{a}$ is called \emph{nondegenerate}.
\end{thm}

Lemma~\ref{lem:nondeg-test-with-balls} will provide examples of nondegenerate point arrays,
which can be used in Theorem~\ref{thm:cat-bary-web:mnfld}.

\begin{thm}{Corollary}\label{cor:LinDim>bary}
Let $\spc{U}$ be a complete length $\CAT\kappa$ space,
$\bm{a}\z=(a^0,a^1,\dots a^m)$ be a nondegenerate point array 
of radius $<\tfrac{\varpi\kappa}{2}$ in $\spc{U}$
and $\sigma=\spx{\bm{a}}\mc\kappa$ be the corresponding $\kappa$-baricentric simplex.
Then for some $\bm{x}\in \Delta^m$,
the differential $\dd_{\bm{x}}\sigma$ is linear 
and the image $\Im\dd_{\bm{x}}\sigma$
forms a subcone isometric to an $m$-dimensional Euclidean space in the tangent cone $\T_{\sigma(\bm{x})}$.
\end{thm}


\parit{Proof.}
Denote the distance map  $\distfun{\bm{a}^{\without 0}}{}{}$ by $\tau\:\spc{U}\to\RR^m$.

According to Theorem~\ref{thm:cat-bary-web},
$\sigma$ is Lipschitz
and  the distance map $\tau$  
gives an open embedding of  
$\InWeb(\distfun{\bm{a}}{}{})=\sigma(\Delta^m)\backslash\sigma(\partial\Delta^m)$.
Note that $\tau$ is Lipschitz.
According to Rademacher's theorem (\ref{thm:Rademacher-CBB+CBA}), 
the differential 
$\dd_{\bm{x}}(\tau\circ\sigma)$
is linear for almost all $\bm{x}\in\Delta^m$.
Further, since $\InWeb(\distfun{\bm{a}}{}{})\z\ne\emptyset$,
the area formula %?? ref
implies that $\dd_{\bm{x}}(\tau\circ\sigma)$ is surjective on a set of positive masure of points $\bm{x}\in\Delta^m$.

Note that $\dd_{\bm{x}}(\tau\circ\sigma)=(\dd_{\sigma(\bm{x})}\tau)\circ(\dd_{\bm{x}}\sigma)$.
Applying Rademacher's theorem again, we have linearity of 
$\dd_{\bm{x}}\sigma$  for almost all $\bm{x}\in\Delta^m$;
at these points $\Im\dd_{\bm{x}}\sigma$ forms a subcone isometric to a Euclidean space in $\T_{\sigma(\bm{x})}$.
Clearly the dimension of $\Im\dd_{\bm{x}}(\tau\circ\sigma)$ is at least as big as the dimension of $\Im\dd_{\bm{x}}\sigma$.
Hence the result.
\qeds


\parit{Proof of \ref{thm:cat-bary-web}.}
Fix $z\in\spc{U}$ and $r<\tfrac{\varpi\kappa}2$
such that $\dist{z}{a^i}{}<r$ for all $i$.
Note that the set $K\cap \cBall[z,r]$ is convex, closed, and contains all $a^i$.
Applying the theorem on short retract (Exercise~\ref{ex:short-retraction-CBA(1)}),
we get the second part of~(\ref{SHORT.thm:cat-bary-web:web=Im(bary)}).

\medskip

The remaining statements are proved first in  the case $\kappa\le 0$,  
and then the remaining case $\kappa>0$ is reduced to the case $\kappa=0$.

\parit{Case $\kappa\le 0$.}
Consider the function array $f^i=\md\kappa\circ \distfun{a^i}{}{}$.
From the definition of web (\ref{def:web}),
it is clear that $\Web(\distfun{\bm{a}}{}{})=\Web\bm{f}$.
Further, from the definition of $\kappa$-barycentric simplex,
$\spx{\bm{a}}\mc\kappa=\spx{\bm{f}}$.

All the functions $f^i$ are strongly convex (see \ref{function-comp}).
Therefore (\ref{SHORT.thm:cat-bary-web:Lip}), (\ref{SHORT.thm:cat-bary-web:web=Im(bary)}) and the first statements in (\ref{SHORT.thm:cat-bary-web:mnfld}) follow from Theorem \ref{thm:web}.

\parit{Case $\kappa>0$.}
Applying rescaling, we may assume $\kappa=1$,
so $\varpi\kappa=\varpi1\z=\pi$.

Let $\mathring{\spc{U}}=\Cone\spc{U}$.
By \ref{thm:warp-curv-bound:cbb:S}, $\mathring{\spc{U}}$ is $\CAT0$.
Let us denote by $\iota$ the natural embedding of $\spc{U}$ as the unit sphere in $\mathring{\spc{U}}$,  and by  
$\proj\:\mathring{\spc{U}}\subto\spc{U}$ the submap
defined by $\proj(v)=\iota^{-1}(v/|v|)$ for all 
$v\not=\0$.
Note that there is $z\in\spc{U}$ and $\eps>0$ such that
the set 
\[K_\eps
=
\set{v\in\mathring{\spc{U}}}%
{\<\iota(z),v\>\ge\eps}\] 
contains all $\iota(a^i)$.
Then 
$\0\notin K_\eps$, 
and
the set $K_\eps$ is closed and convex.
The latter follows from Exercise~\ref{ex:busemann-CBA},
since $v\mapsto -\<\iota(z),v\>$ is a Busemann function.

Denote by $\iota(\bm{a})$ the point array $(\iota(a^0), \iota(a^1),\dots,\iota(a^\kay))$ in $\mathring{\spc{U}}$. 
From the case $\kappa=0$,
we get that $\Im \spx{\iota(\bm{a})}\mc0\subset K_\eps$.
In particular $\Im \spx{\iota(\bm{a})}\mc0\not\ni \0$ and thus $\proj\circ\spx{\iota(\bm{a})}\mc0$ is defined.
Direct calculations show 
\[\spx{\bm{a}}\mc1
=
\proj\circ\spx{\iota(\bm{a})}\mc0
\quad\text{and}\quad
\Web(\distfun{\bm{a}}{}{})=\proj[\Web(\distfun{\iota(\bm{a})}{}{})].\]
Thus the case $\kappa=1$ of the theorem is reduced to the case $\kappa=0$,
which is proved already.
\qeds



\begin{thm}{Lemma}\label{lem:nondeg-test-with-balls}
Let $\spc{U}$ be a complete length $\CAT\kappa$ space,
$\bm{a}\z=(a^0,a^1,\dots a^\kay)$ be an array of radius $<\tfrac{\varpi\kappa}2$, 
and $B^i=\cBall[a^i,r^i]$ for some array of positive reals $(r^0,r^1,\dots,r^\kay)$.
Assume that
$\bigcap_i B^i=\emptyset$,
but
$\bigcap_{i\not=j} B^i\not=\emptyset$
for any $j$.
Then $\bm{a}$ is nondegenerate. 
\end{thm}

\parit{Proof.} 
Without loss of generality we may assume that $\spc{U}$ is geodesic and $\diam\spc{U}\z<\varpi\kappa$.
If not, choose $z\in\spc{U}$ and $r<\tfrac{\varpi\kappa}{2}$ so that
$\dist{z}{a^i}{}\le r$
for each $i$, 
and consider $\cBall[z,r]$ instead of $\spc{U}$.
The latter can be done since $\cBall[z,r]$ is convex and closed, 
so $\cBall[z,r]$ is a complete length $\CAT\kappa$ space 
and $\Web(\distfun{\bm{a}}{}{})\subset\cBall[z,r]$;
see \ref{cor:convex-balls} and \ref{thm:cat-bary-web:web=Im(bary)}.

By Theorem~\ref{thm:cat-bary-web}, $\Web(\distfun{\bm{a}}{}{})$ is a compact set;
therefore there is a point $p\in\Web(\distfun{\bm{a}}{}{})$
minimizing the function 
\[f(x)=\max_i\{\distfun{B^i}{x}{}\}=\max\{0,\dist{a^0}{x}{}-r^0,\dots,\dist{a^\kay}{x}{}-r^\kay\}.\]

By the definition of web (\ref{def:web}), 
$p$ is also the minimum point of $f$ on $\spc{U}$.
Let us prove the following claim:

\begin{clm}{}
 $p\notin B^j$ for any $j$.
\end{clm}

Indeed, 
assume the contrary; that is, 
\[
p\in B^j
\eqlbl{eq:p-in-Bj}
\] 
for some $j$.
Then $p$ is a point of local minimum for the function 
\[h^j(x)=\max_{i\not=j}\{\distfun{B^i}{x}{}\}.\]
Hence 
\[\max_{i\not=j}\{\mangle\hinge p x {a^i}\}\ge \tfrac\pi2
\]
for any $x\in\spc{U}$.
From the angle comparison (\ref{cat-hinge}), it follows that 
$p$ is a global minimum of $h^j$ and hence
\[
p\in \bigcap_{i\not=j} B^i.
\]
The latter and \ref{eq:p-in-Bj} contradict $\bigcap_i B^i=\emptyset$. \claimqeds 
\noindent 

From the definition of web, it also follows that 
\[\Web(\distfun{\bm{a}^{\without j}}{}{})\subset \bigcup_{i\not=j}B^i.\]
Indeed, if $q\in \bigcap_{i\not=j}B^i$ and $q'\notin \bigcup_{i\not=j}B^i$,
then $\dist{a_i}{q}{}\z<\dist{a_i}{q'}{}$ for any $i\ne j$ therefore $q'\notin \Web(\distfun{\bm{a}^{\without j}}{}{})$.
Therefore the claim implies that
$p\notin\Web(\distfun{\bm{a}^{\without j}}{}{})$ for each $j$;
that is, $p\in\InWeb(\distfun{\bm{a}}{}{})$.
\qeds


\section{Dimension}\label{sec:dim-cba}

See Chapter \ref{ch:dim} for definitions of various dimension-like invariants of metric spaces.

We start with two examples.

The first example shows that the dimension of complete length $\CAT{}$ spaces is not local;
that is, such spaces might have open sets with different linear dimensions.

Such an example can be constructed by gluing at one point two Euclidean spaces of different dimensions.
According to Reshetnyak's gluing theorem (\ref{thm:gluing}), this construction gives a $\CAT{0}$ space.

The second example provides a complete length $\CAT{}$ space 
with topological dimension 1 and arbitrary large Hausdorff dimension.
Thus for complete length $\CAT{}$ spaces, one should not expect any relations between topological and Hausdorff dimensions except for the one provided by Szpilrajn's theorem (\ref{thm:szpilrajn}).

To construct the second type of example,
note that the completion of any metric tree has topological dimension 1 and is $\CAT\kappa$ for any $\kappa$.
Start with a binary tree $\Gamma$, and a sequence $\eps_n>0$ such that $\sum_n\eps_n<\infty$.
Define the metric on $\Gamma$
by prescribing the length of an edge from level $n$ to  level $n+1$ to be $\eps_n$.
For an appropriately chosen sequence $\eps_n$, the completion of $\Gamma$ will contain a Cantor set of arbitrarily large Hausdorff dimension.

\medskip

The following is a version of a theorem proved by Bruce Kleiner \cite{kleiner}, with an improvement made by Alexander Lytchak \cite{lytchak:diff}.

\begin{thm}{Theorem}\label{thm:dim-infty-CBA}
For any complete length $\CAT\kappa$ space $\spc{U}$, the following statements are equivalent:

\begin{subthm}{LinDim-CBA} $\LinDim\spc{U}\ge m$.
\end{subthm}

\begin{subthm}{thm:dim-infty-CBA:bary} 
For some $z\in \spc{U}$ there is an array of $m+1$ balls $B^i\z=\oBall(a^i,r^i)$ with $a^0,a^1,\dots,a^m\in \oBall(z,\frac{\varpi\kappa}2)$ 
such that 
\[\bigcap_i B^i=\emptyset
\quad\text{and}\quad
\bigcap_{i\not=j} B^i\not=\emptyset
\quad \text{for each $j$}.\]

\end{subthm}


\begin{subthm}{thm:dim-infty-CBA:mnfld} 
There is a $C^{\frac{1}{2}}$-embedding $\map\:\cBall[1]_{\EE^m}\hookrightarrow \spc{U}$;
that is, $\map$ is bi-H\"older with exponent $\tfrac{1}{2}$.
\end{subthm}

\begin{subthm}{thm:dim-infty-CBA:TopDim}
There is a closed separable set $K\subset\spc{U}$ such that 
\[\TopDim K\ge m.\]
\end{subthm}

\end{thm}

\parbf{Remarks.}
Theorem \ref{thm:loc-lip-inverse} gives a stronger version of part (\ref{SHORT.thm:dim-infty-CBA:mnfld}) in the finite-dimensional case.
Namely, a complete length $\CAT{}$ space with linear dimension $m$ 
admits a bi-Lipschitz embedding $\map$ of an open set of $\RR^m$.
Moreover, the Lipschitz constants of $\map$ can be made arbitrarily close to~$1$.

\begin{thm}{Corollary}\label{cor:dim-CBA}
For any separable complete length $\CAT{}$ space $\spc{U}$, we have
\[\TopDim\spc{U}=\LinDim\spc{U}.\]

\end{thm}


Any simplicial complex can be equipped with a length metric
such that each $\kay$-simplex 
is isometric to the standard simplex
\[\Delta^\kay
=
\set{(x_0,\dots,x_\kay)\in \RR^{\kay+1}}
{x_i\ge 0,\quad x_0+\dots+x_\kay=1}\]
with the metric induced by the $\ell^1$-norm on $\RR^{\kay+1}$.
This metric will be called the \index{$\ell^1$-metric}\emph{$\ell^1$-metric} on the simplicial complex.

\begin{thm}{Lemma}\label{lem:approximation-cba}
Let $\spc{U}$ be a complete length $\CAT\kappa$ space
and $\rho\:\spc{U}\to\RR$ be a continuous positive function.
Then there is a simplicial complex $\spc{N}$ equipped with $\ell^1$-metric,
a locally Lipschitz map $\map\:\spc{U}\to \spc{N}$,  
and a Lipschitz map $\map[2]\:\spc{N}\to\spc{U}$ such that:

\begin{subthm}{lem:approximation-cba:displacement}
The displacement of the composition $\map[2]\circ\map\:\spc{U}\to\spc{U}$ is bounded by $\rho$;
that is,
\[\dist{x}{\map[2]\circ\map(x)}{}<\rho(x)\] 
for any $x\in\spc{U}$.
\end{subthm}

\begin{subthm}{lem:approximation-cba:im}
If $\LinDim\spc{U}\le m$ 
then the $\map[2]$-image of any closed simplex in $\spc{N}$ 
coincides with the image of its $m$-skeleton.
\end{subthm}

\end{thm}

\parit{Proof.}
Without loss of generality, we may assume that for any $x$ we have $\rho(x)\z<\rho_0$
for some fixed $\rho_0<\tfrac{\varpi\kappa}{2}$.

By Stone's theorem, any metric space is paracompact.
Thus, we can choose a locally finite covering $\set{\Omega_\alpha}{\alpha\in\IndexSet}$ of $\spc{U}$ such that $\Omega_\alpha\subset \oBall(x,\tfrac{1}{3}\cdot\rho(x))$ for any $x\in \Omega_\alpha$. 

Denote by $\spc{N}$ the nerve of the covering $\{\Omega_\alpha\}$;
that is, $\spc{N}$ is an abstract simplicial complex with 
%set of vertexes formed by $\IndexSet$, 
vertex set $\IndexSet$,
such that
%that is, $\spc{N}$ is an abstract simplicial complex with set of vertices formed by $\IndexSet$ such that
$\{\alpha^0,\alpha^1,\dots,\alpha^n\}\subset\IndexSet$ 
are vertexes of a simplex if and only if
$\Omega_{\alpha^0}
\cap
\Omega_{\alpha^1}
\cap\dots\cap
\Omega_{\alpha^n}
\not=
\emptyset$.

Fix a Lipschitz partition of unity 
$\phi_\alpha\:\spc{U}\to [0,1]$ subordinate to $\{\Omega_\alpha\}$.
Consider the map $\map\:\spc{U}\to \spc{N}$ such that the barycentric coordinate of $\map(p)$ is $\phi_\alpha(p)$.
Note that $\map$ is locally Lipschitz. 
Clearly the $\map$-preimage of any open simplex in $\spc{N}$ lies in $\Omega_\alpha$ for some $\alpha\in\IndexSet$.

For each $\alpha\in\IndexSet$, 
choose $x_\alpha\in\Omega_\alpha$.
Let us extend the map $\alpha\mapsto x_\alpha$
to a map $\map[2]\:\spc{N}\to\spc{U}$ that is $\kappa$-barycentric on each simplex.
According to Theorem~\ref{thm:cat-bary-web:Lip}, this extension exists, 
$\map[2]$ is Lipschitz, 
and its Lipschitz constant depends only on $\rho_0$ and $\kappa$.

\parit{(\ref{SHORT.lem:approximation-cba:displacement})}
Fix $x\in\spc{U}$. Denote by $\Delta$ the minimal simplex that contains $\map(x)$, 
and let $\alpha^0,\alpha^1,\dots,\alpha^n$ be the vertexes of $\Delta$.
%Denote by $\Delta$ the minimal simplex that contains $\map(x)$;
%and let $(\alpha^0,\alpha^1,\dots,\alpha^n)$ be the vertices of $\Delta$.
Note that $\alpha$ is a vertex of $\Delta$ if and only if $\phi_{\alpha}(x)>0$.
Thus
\[\dist{x}{x_{\alpha^i}}{}<\tfrac{1}{3}\cdot\rho(x)\] 
for any $i$.
Therefore 
\[\diam\map[2](\Delta)
\le
\max_{i,j}\{\dist{x_{\alpha^i}}{x_{\alpha^j}}{}\}
<
\tfrac{2}{3}\cdot\rho(x).\]
In particular, 
\[\dist{x}{\map[2]\circ\map(x)}{}\le\dist{x}{x_{\alpha^0}}{}+\diam \map[2](\Delta) <\rho(x).\]

\parit{(\ref{SHORT.lem:approximation-cba:im})}
Assume the contrary;
that is, $\map[2](\spc{N})$ is not included in the $\map[2]$-image of the $m$-skeleton of $\spc{N}$.
Then for some $\kay>m$,
there is a $\kay$-simplex $\Delta^\kay$ in $\spc{N}$
such that the barycentric simplex $\sigma=\map[2]|_{\Delta^\kay}$ is nondegenerate; 
that is, 
$$W=\map[2](\Delta^\kay)\backslash\map[2](\partial\Delta^\kay)\not=\emptyset.
$$
Applying Corollary~\ref{cor:LinDim>bary}
gives $\LinDim\spc{U}\ge \kay$, a contradiction.
\qeds






\parit{Proof of \ref{thm:dim-infty-CBA}.} 
Note that
\begin{itemize}
\item The implication (\ref{SHORT.thm:dim-infty-CBA:bary})$\Rightarrow$(\ref{SHORT.thm:dim-infty-CBA:mnfld})
follows directly from Lemma~\ref{lem:nondeg-test-with-balls}
and Theorem~\ref{thm:cat-bary-web:mnfld}.
\item The implication 
(\ref{SHORT.thm:dim-infty-CBA:mnfld})$\Rightarrow$(\ref{SHORT.thm:dim-infty-CBA:TopDim}) 
is trivial.
\end{itemize}
 
\parit{(\ref{SHORT.thm:dim-infty-CBA:TopDim})$\Rightarrow$(\ref{SHORT.LinDim-CBA}).}
According to Theorem~\ref{thm:stable-value}, 
there is a continuous map $f\:K\to \RR^{m}$ with a stable value.
By the Tietze extension theorem, it is possible to extend $f$ 
to a continuous map $F\:\spc{U}\to \RR^{m}$.

Fix $\eps>0$.
Since $F$ is continuous, there is a continuous positive function $\rho$ defined on $\spc{U}$ such that 
\[\dist{x}{y}{}<\rho(x)
\quad\Rightarrow\quad
|F(x)- F(y)|<\tfrac13\cdot\eps.\]
Apply Lemma~\ref{lem:approximation-cba} for the function $\rho$.
For the resulting simplicial complex $\spc{N}$ 
 and maps $\map\:\spc{U}\to \spc{N}$, $\map[2]\:\spc{N}\to \spc{U}$, we have
\[|F\circ \map[2]\circ\map(x)-F(x)|<\tfrac13\cdot\eps\] 
for any $x\in \spc{U}$.

According to Lemma~\ref{lem:lip-approx},
there is a locally Lipschitz map $F_\eps\:\spc{U}\to \RR^{m+1}$ 
such that $|F_\eps(x)-F(x)|<\tfrac13\cdot\eps$ for any $x\in \spc{U}$.

Note that
$\map(K)$ is contained in a countable subcomplex of $\spc{N}$, say $\spc{N}'$.
Indeed, since $K$ is separable, there is a countable dense collection of points $\{x_n\}$ in $K$.
Denote by $\Delta_n$ the minimal simplex of $\spc{N}$ that contains $\map(x_n)$.
Then $\map(K)\subset\bigcup_i\Delta_n$.

Arguing by contradiction,
assume $\LinDim\spc{U}<m$.
By \ref{lem:approximation-cba:im},
the image $F_\eps\circ\map[2]\circ\map(K)$ lies in the $F_\eps$-image of the $(m-1)$-skeleton of $\spc{N}'$;
In particular it can be covered 
by a  countable collection of Lipschitz images of $(m-1)$-simplexes.
Hence
$\bm{0}\in \RR^m$ is not a stable value of the restriction $F_\eps\circ\map[2]\circ\map|_K$.
Since $\eps>0$ is arbitrary, 
then $\bm{0}\in \RR^m$ is not a stable value of $f=F|_K$, a contradiction.

\parit{(\ref{SHORT.LinDim-CBA})$\Rightarrow$(\ref{SHORT.thm:dim-infty-CBA:bary}).} 
The following claim is a consequence of the definition of tangent space.

\begin{clm}{}\label{clm:finite-config-cba}
Let $q\in \spc{U}$ and $\dot x^1,\dot x^2,\dots,\dot x^n\in \T_q$.
Then given $\delta>0$,
there is an array $(x^1,x^2,\dots,x^n)$ of points  in $\spc{U}$ 
such that 
\[\mangle(\dot x^i,\ddir q{x^i})<\delta,\]
and for some fixed $\lam>0$ we have
\[\tfrac1\lam\cdot\dist[{{}}]{q}{x^i}{}=|\dot x^i|\quad
\text{and}\quad |\tfrac1\lam\cdot\dist[{{}}]{x^i}{x^j}{}-\dist{\dot x^i}{\dot x^j}{}|<\delta\] 
for all $i$ and $j$.

Moreover the value $\lam$ can be taken arbitrarily small.
\end{clm}

\parit{Proof of the claim.} For each $i$ choose a geodesic $\gamma^i$ 
from $q$ that goes almost in the directions of $\dot x^i$.
Then take the point $x^i$ on $\gamma^i$ at distance $\lam\cdot|\dot x^i|$ from $q$.
\claimqeds


Choose $q\in \spc{U}$ such that $\T_q$ contains a subcone $E$ isometric to $m$-dimensional Euclidean space.
Note that one can choose $\eps>0$ 
and a point arrray $(\dot a^0,\dot a^1,\dots,\dot a^m)$ in $E\subset \T_q$ 
such that 
$\bigcap_i\cBall[\dot a^i,1+\eps]=\emptyset$
and $\bigcap_{i\not=j}\cBall[\dot a^i,1\z-\eps]\z{\not=}\emptyset$ for each $j$.

Applying Claim \ref{clm:finite-config-cba}, we get a point array 
$(a^0,a^1,\dots,a^m)$ in $\spc{U}$
such that $\bigcap_i\cBall[a^i,\lam]=\emptyset$
and $\bigcap_{i\not=j}\cBall[a^i,\lam]\not=\emptyset$ for each $j$.
Since $\lam>0$ can be chosen arbitrarily small, 
 (\ref{SHORT.thm:dim-infty-CBA:bary}) follows.
\qeds


\section{Finite-dimensional spaces}

Recall that a web embedding and its brace are defined in \ref{prop-def:web-embedding}.

{\sloppy 

\begin{thm}{Theorem}\label{thm:loc-lip-inverse}
Suppose  $\spc{U}$ is a complete length $\CAT\kappa$ space such that 
$\LinDim\spc{U}=m$,
and $\bm{a}=(a^0,a^1,\dots a^m)$ is a point array in $\spc{U}$ 
with radius $<\tfrac{\varpi\kappa}{2}$.
Then 
the $\distfun{\bm{a}}{}{}$-web embedding $\map\:\RR^m\subto\spc{U}$ with brace $\distfun{a^0}{}{}$ is locally Lipschitz.
\end{thm}

}

Note that if $\bm{a}$ is degenerate,
that is, if $\InWeb(\distfun{\bm{a}}{}{})=\emptyset$, 
then
the domain of the web embedding $\map$ above is empty and hence the conclusion of the theorem trivially holds.

\begin{thm}{Lemma}\label{lem:nondeg-bs-test}
Let $\spc{U}$ be a complete length $\CAT\kappa$ space,
and $\bm{a}\z=(a^0,a^1,\dots a^\kay)$ be a point array with radius $<\tfrac{\varpi\kappa}{2}$.
Then for any $p\in \InWeb(\distfun{\bm{a}}{}{})$,
there is $\eps>0$ such that 
if for some $q\in \Web(\distfun{\bm{a}}{}{})$ and $b\in\spc{U}$
we have 
\[\dist{p}{q}{}<\eps,
\quad 
\dist{p}{b}{}<\eps
\quad\text{and}\quad 
\mangle\hinge{q}{b}{a^i}<\tfrac\pi2+\eps\]
for each $i$,
then the array $b,a^0,a^1,\dots,a^m$ is nondegenerate.
\end{thm}


\parit{Proof.}
Without loss of generality, we may assume that $\spc{U}$ is geodesic and $\diam\spc{U}<\varpi\kappa$.
If not, consider instead of $\spc{U}$,
a ball $\cBall[z,r]\subset\spc{U}$ 
for some $z\in\spc{U}$ 
and $r<\tfrac{\varpi\kappa}{2}$
such that $\dist{z}{a^i}{}\le r$ for each $i$.

From the angle comparison (\ref{cat-hinge}), it follows that 
$p\in\InWeb\bm{a}$ if and only if both of the following conditions hold:
\begin{enumerate}
\item $\max_i\{\mangle\hinge p{a^i}{u}\}\ge \tfrac\pi2$ for any $u\in\spc{U}$,
\item\label{prop:<pi/2} for each $i$ there is $u^i\in\spc{U}$ such that $\mangle\hinge p{a^j}{u^i}<\tfrac\pi2$ for all $j\not=i$.
\end{enumerate}

Due to the semicontinuity of angles (\ref{lem:ang.semicont}),
there is $\eps>0$ such that for any $x\in \oBall(p,10\cdot\eps)$ we have
\[
\mangle\hinge {x}{a^j}{u^i}
<
\tfrac\pi2-10\cdot\eps
\quad\text{for all}\quad j\not=i.
\eqlbl{eq:<pi/2-eps}\]

Now assume that for sufficiently small $\eps>0$
there are points $b\in\spc{U}$ and $q\in\Web(\distfun{\bm{a}}{}{})$ such that 
\[\dist{p}{q}{}<\eps,\quad \dist{p}{b}{}<\eps,\quad \mangle\hinge{q}{b}{a^i}<\tfrac\pi2+\eps\quad\text{for all}\quad i.
\eqlbl{eq:2nd-angle}\]
According to Theorem~\ref{thm:cat-bary-web:web=Im(bary)},
for all small $\eps>0$ we have 
\[\rad\{b,a^0,a^1,\dots,a^\kay\}<\tfrac{\varpi\kappa}2.\]
Fix a sufficiently small $\delta>0$
and let 
\[v^i=\geod_{[q u^i]}(\tfrac13\cdot\delta)\quad\text{and}\quad w^i=\geod_{[v^i b]}(\tfrac23\cdot\delta).\]
Clearly
\begin{align*}
\dist{b}{w^i}{}
&=
\dist{b}{v^i}{}-\tfrac23\cdot\delta
\le
\\
&\le
\dist{b}{q}{}-\tfrac13\cdot\delta.
\\
\intertext{Further, the inequalities \ref{eq:<pi/2-eps} and \ref{eq:2nd-angle} imply}
\dist{a^j}{w^i}{}
&<
\dist{a^j}{v^i}{}+\tfrac23\cdot\eps\cdot\delta
<
\\
&<
\dist{a^i}{q}{}-\eps\cdot\delta
<
\\
&<
\dist{a^i}{q}{}
\end{align*}
for all $i\not=j$.

Set $B^i=\cBall[a^i,\dist{a^i}{q}{}]$ and $B^{m+1}=\cBall[b,\dist{a^i}{q}{}-\tfrac13\cdot\delta]$.
Clearly 
\begin{align*}
&\!\!\!\!\bigcap_{i\not=m+1} B^i=\{q\},
\\
&\bigcap_{i\not=j}B^i\ni w^j\quad\text{for}\quad j\not=m+1,
\\
&\bigcap_{i}B^i=\{q\}\cap B^{m+1}=\emptyset.
\end{align*}
Lemma~\ref{lem:nondeg-test-with-balls} finishes the proof.
\qeds


\parit{Proof of \ref{thm:loc-lip-inverse}.}
%By \ref{prop-def:web-embedding}, it is sufficient to 
%show that $\map$ is locally Lipshitz.
Suppose $\map$ is not locally Lipshitz.; that is, there are sequences $\bm{y}_n, \bm{z}_n\to \bm{x}\in\Dom\map$ such that
\[\frac{\dist{\map(\bm{y}_n)}{\map(\bm{z}_n}{})}{|\bm{y}_n-\bm{z}_n|}
\to\infty
\quad
\text{as}
\quad
n\to\infty.
\eqlbl{eq:nonlip}\]
Set $p=\map(\bm{x})$,
$q_n=\map(\bm{y}_n)$, 
and $b_n=\map(\bm{z}_n)$.
By \ref{prop-def:web-embedding}, $p$, $q_n$, $b_n\in\InWeb(\distfun{\bm{a}}{}{})$
and $q_n,b_n\to p$ as $n\to\infty$.
Choose an arbitrary $\eps>0$.
Note that \ref{eq:nonlip} implies
\[\mangle\hinge{q_n}{a^i}{b_n}<\tfrac\pi2+\eps
\]
for all $i>0$ and all large $n$.
Further, according to \ref{prop-def:web-embedding}, the subfunction
$(\distfun{\bm{a}^0}{}{})\circ\map$ is locally Lipschitz.
Therefore we also have 
\[\mangle\hinge{q_n}{a^0}{b_n}<\tfrac\pi2+\eps
\]
for all large $n$.
According to Lemma~\ref{lem:nondeg-bs-test}, the point array $b_n,a^0,\dots,a^\kay$ for large $n$ is nondegenerate.

Applying Corollary~\ref{cor:LinDim>bary},
we have a contradiction.
\qeds



\section{Remarks and open problems}

The following conjecture (in an equivalent form)
appears in \cite{kleiner}, see also \cite[p.~133]{gromov:asymt-inv}.

\begin{thm}{Conjecture}
For any complete length $\CAT{}$ space $\spc{U}$, we have
\[\TopDim\spc{U}
=
\LinDim\spc{U}.\]

\end{thm}

By Corollary~\ref{cor:dim-CBA}, this conjecture holds for separable spaces.


















