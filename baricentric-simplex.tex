%array^
\chapter{Webs and barycenters %ready
}\label{chap:web+bary}

The material of this chapter is used mostly for $\cCat{}{}$ spaces, 
although the results in section~\ref{sec:web-general} find some applications for finite dimensional $\Alex{}$ spaces as well.
The analogous material for $\Alex{}$ spaces is described in Chapter~\ref{chap:dist-maps}.


\section{The case of complete geodesic spaces}\label{sec:web-general}

The following construction gives a $\kay$-dimensional submanifold  
for a given ``nondegenerate'' array of $\kay+1$ strongly convex functions.
This construction was introduced by Kleiner in \cite{kleiner}.

\parbf{Webs.}
Let us introduce a partial order $\succcurlyeq$ on $\RR^{\kay+1}$.

\begin{thm}{Definition}\label{def:ordung}
For two real atrrays $\bm{v}$, $\bm{w}\in \RR^{\kay+1}$,
$\bm{v}=(v^0,v^1,\dots,v^\kay)$ 
and 
$\bm{w}=(w^0,w^1,\dots,w^\kay)$, 
we will write
$\bm{v}\succcurlyeq\bm{w}$ if $v^i\ge w^i$ for each $i$.
\end{thm}

Given a subset $Q\subset \RR^{\kay+1}$, 
denote by $\Up Q$,\label{PAGE.def:Up}
the smallest upper set containing $Q$ 
and 
$\Min Q$, the set of minimal elements of $Q$ with respect to $\succcurlyeq$;
that is,
\begin{align*}
\Up Q 
&=
\set{\bm{v}\in\RR^{\kay+1}}{\exists\, \bm{w}\in Q\ \t{such that}\ \bm{v}\succcurlyeq\bm{w}},
\\
\Min Q 
&=
\set{\bm{v}\in Q}{\t{if}\ \bm{v}\succcurlyeq\bm{w}\in Q\ \t{then}\ \bm{w}=\bm{v}}.
\end{align*}


\begin{thm}{Definition}\label{def:web}
Let $\spc{X}$ be a metric space 
and $\bm{f}=(f^0,f^1,\dots,f^\kay)\:\spc{X}\to \RR^{\kay+1}$ be a function array.
The set 
\[\Web\bm{f}
\df
\bm{f}^{-1}\l[\Min\bm{f}(\spc{X})\r]
\subset 
\spc{X}\] 
will be called the \emph{web}\index{web} on $\bm{f}$.
\end{thm}

Given an array $\bm{f}=(f^0,f^1,\dots,f^\kay)$,
we denote by $\bm{f}^{\without i}$ the subarray of $\bm{f}$ with $f^i$ removed;
that is, 
\[\bm{f}^{\without i\,}\df(f^0,\dots,f^{i-1},f^{i+1},\dots,f^\kay).\]
Clearly 
$\Web\bm{f}^{\without i}\subset \Web\bm{f}$.
Define the \emph{inner web}\index{inner web} of $\bm{f}$ 
as 
\[\InWeb\bm{f}
=
\Web\bm{f}\backslash\l(\bigcup_{i}\Web\bm{f}^{\without i}\r).\]


We say that a function array is \emph{nondegenerate}\index{nondegenerate function array} 
if $\InWeb\bm{f}\not=\emptyset$.

\parbf{Example.} 
If $\spc{X}$ is a geodesic space, 
then $\Web(\dist{x}{}{},\dist{y}{}{})$ is the union of all geodesics from $x$ to $y$ and 
$\InWeb(\dist{x}{}{},\dist{y}{}{})=\Web(\dist{x}{}{},\dist{y}{}{})\backslash\{x,y\}$.

\parbf{Barycenters.}
Let us denote by $\Delta^\kay\subset \RR^{\kay+1}$\index{$\Delta^m$} 
the \emph{standard $\kay$-simplex}\index{standard simplex}; 
that is, $\bm{x}=(x^0,x^1,\dots,x^\kay)\in\Delta^\kay$ if $\sum_{i=0}^\kay x^i=1$ and $x^i\ge0$ for all $i$.

Let $\spc{X}$ be a metric space 
and $\bm{f}=(f^0,f^1,\dots,f^\kay)\:\spc{X}\to \RR^{\kay+1}$ be a function array.
Consider the map $\spx{\bm{f}}\:\Delta^\kay\to \spc{X}$,\index{$\spx{\bm{f}}$} defined by 
\[\spx{\bm{f}}(\bm{x})=\argmin\sum_{i=0}^\kay x^i\cdot f^i,\]
where $\argmin f$\index{$\argmin$} denotes a point of minimum of $f$.
The map $\spx{\bm{f}}$ will be called a \emph{barycentric simplex}\index{barycentric simplex of function array} of $\bm{f}$.
Note that for general function array $\bm{f}$, 
the value $\spx{\bm{f}}(\bm{x})$ might be undefined or nonuniquely defined.

It is clear from the  definition, that $\spx{\bm{f}^{\without i}}$ 
coincides with the restriction of $\spx{\bm{f}}$ to the corresponding face of $\Delta^\kay$.


\begin{thm}{Theorem}\label{thm:web}
Let $\spc{X}$ be a complete geodesic space 
and $\bm{f}\z=(f^0,f^1,\dots,f^\kay)\:\spc{X}\to\RR^{\kay+1}$ 
be an array of strongly convex and locally Lipschitz functions.
Then $\bm{f}$ defines a $C^{\frac12}$-embedding 
$\Web\bm{f}\hookrightarrow\RR^{\kay+1}$.

Moreover,
\begin{subthm}{thm:web:Up-convex}
$W=\Up[\bm{f}(\spc{X})]$ is a convex closed subset of $\RR^{\kay+1}$
and

$S=\Fr_{\RR^{\kay+1}} W$ is a convex hypersurface in $\RR^{\kay+1}$.
\end{subthm}

\begin{subthm}{thm:web:f(web)=min}
$\bm{f}(\Web\bm{f})=\Min W \subset S$ and
$\bm{f}(\InWeb\bm{f})= \Int_{S}(\Min W)$.
\end{subthm}

\begin{subthm}{thm:web:bary}
The barycentric simplex 
$\spx{\bm{f}}\:\Delta^\kay\to \spc{X}$ is a uniquely defined Lipshitz map and $\Im\spx{\bm{f}}=\Web\bm{f}$.

In particular $\Web\bm{f}$ is compact.
\end{subthm}

\begin{subthm}{thm:web:lip-const}
Let us equip $\Delta^\kay$ with metric induced by $\ell^1$-norm on $\RR^{\kay+1}$.
Then the Lipschitz constant of $\spx{\bm{f}}\:\Delta^\kay\to\spc{U}$ can be estimated in terms of 
positive lower bounds on $(f^i)''$ 
and Lipshitz constants of $f^i$
in a neigborhood of $\Web\bm{f}$ for all $i$.
\end{subthm}


In particular, from (\ref{SHORT.thm:web:Up-convex}) and (\ref{SHORT.thm:web:f(web)=min}), it follows that $\InWeb\bm{f}$ is $C^{\frac12}$-homeomorphic to an open set of $\RR^\kay$.
\end{thm}

The proof is preceded by few preliminary statements.

\begin{thm}{Lemma}\label{lem:argmin(convex)}
Assume $\spc{X}$ is a complete geodesic space and let  $f\:\spc{X}\to\RR$ be a locally Lipschitz, strongly convex function.  Then the minimum point 
of $f$
is uniquely defined.
\end{thm}

\parit{Proof.}
Without loss of generality, we can assume that $f$ is $1$-convex;
in particular the following clim holds.
\begin{clm}{}\label{midpoint}
 if $z$ is a midpoint of $[x y]$ then  
\[s\le f(z)
\le
\tfrac{1}{2}\cdot f(x)+\tfrac{1}{2}\cdot f(y)-\tfrac{1}{8}\cdot\dist[2]{x}{y}{}.
\]
\end{clm}

\parit{Uniqueness.}
Assume that $x$ and $y$ are distinct minimum points of $f$. 
From \ref{midpoint}, we get
\[f(z)<f(x)=f(y),\] 
a contradiction. 

\parit{Existence.}
Fix a point $p\in  \spc{X}$; 
let $\Lip\in\RR$ be a Lipschitz constant of $f$ in a neighborhood of $p$.

Consider function $\phi(t)=f\circ\geod_{[px]}(t)$.
Clearly $\phi$ is $1$-convex and $\phi^+(0)\ge -\Lip$.
Setting $\ell=\dist{p}{x}{}$, we get 
\begin{align*}
f(x)
&=
\phi(\ell)
\ge
\\
&\ge
f(p)-\Lip\cdot\ell+\tfrac{1}{2}\cdot\ell^2
\ge
\\
&\ge f(p)-\tfrac{1}{2}\cdot{\Lip^2}.
\end{align*}

In particular,
\[s
\df
\inf\set{f(x)}{x\in \spc{X}}
\ge
f(p)-\tfrac{1}{2}\cdot{\Lip^2}.\]

Choose a sequence of points $p_n\in \spc{X}$  such that $f(p_n)\to s$.
Applying \ref{midpoint}, for $x\z=p_n$, $y\z=p_m$, we get that $(p_n)$ is converging in itself. 
Hence, $p_n$ converges to a minimum point of $f$.
\qeds

\begin{thm}{Definition}
Let $Q$ be a closed subset of $\RR^{\kay+1}$.
A vector $\bm{x}\z=(x^0,x^1,\dots,x^\kay)\in\RR^{\kay+1}$
is \emph{subnormal}\index{subnormal vector} to $Q$ at point $\bm{v}\in \Fr_{\RR^{\kay+1}}Q$ 
if
\[\<\bm{x},\bm{w}-\bm{v}\>
\df
\sum_ix^i\cdot(w^i-v^i)
\ge 0\]
for any $\bm{w}\in Q$.
\end{thm}


\begin{thm}{Lemma}\label{lem:Up-convex}
Let $\spc{X}$ be a complete geodesic space 
and $\bm{f}\z=(f^0,f^1,\dots,f^\kay)\:\spc{X}\to\RR^{\kay+1}$ 
be an array of strongly convex and locally Lipschitz functions.
Set $W=\Up\bm{f}(\spc{X})$.
Then: 
\begin{subthm}{lem:Up-convex:Up-convex}
$W$ is a closed convex set which is bounded below with respect to $\succcurlyeq$.
\end{subthm}

\begin{subthm}{lem:Up-convex:subnormal}
If $\bm{x}$ is a subnormal vector to $W$, then $\bm{x}\succcurlyeq\bm{0}$.
\end{subthm}

\begin{subthm}{lem:Up-convex:surface}
 $S=\Fr_{\RR^{\kay+1}}W$ is a complete convex hypersurface in $\RR^{\kay+1}$.
\end{subthm}

\end{thm}

\parit{Proof.}
Denote by $\bar W$ the closure of $W$.

Convexity of $f^i$ implies that
for any two points $p,q\in \spc{X}$ and $t\in[0,1]$ we have
\[(1-t)\cdot\bm{f}(p)+t\cdot \bm{f}(q)
\succcurlyeq
\bm{f}\circ\geodpath_{[p q]}(t).
\eqlbl{n-convex}\]
where $\geodpath_{[p q]}$ denotes a geodesic path from $p$ to $q$. 
Therefore $W$ as well as $\bar W$ are convex sets in $\RR^{\kay+1}$.

Set
\[w^i=\min\set{f^i(x)}{x\in\spc{X}}.\]
By Lemma~\ref{lem:argmin(convex)}, $w^i$ is finite for each $i$.
Clearly $\bm{w}=(w^0,w^1,\dots,w^\kay)$ is a lower bound of $\bar W$ with respect to $\succcurlyeq$.

It is clear that $W$ has nonempty interior
and $W\not=\RR^{\kay+1}$ since it is bounded below.
Therefore $S=\Fr_{\RR^{\kay+1}}W=\Fr_{\RR^{\kay+1}}\bar W$
is a complete convex hypersurface in $\RR^{\kay+1}$.

Since $\bar W$ is closed and bounded below, we also have
\[\bar W=\Up[\Min\bar W].
\eqlbl{eq:W=Up Min W}\]

Choose arbitrary $\bm{v}\in S$.
Let $\bm{x}\in\RR^{\kay+1}$ be a subnormal vector to $\bar W$ at $\bm{v}$. 
In particular, 
$\<\bm{x},\bm{y}\>
\ge
0$ 
for any $\bm{y}\succcurlyeq\bm{0}$;
that is, $\bm{x}\succcurlyeq\bm{0}$.

Further, according to Lemma~\ref{lem:argmin(convex)}, 
the function  
$\sum_i x^i\cdot f^i$ has uniquely defined minimum point, say $p$.
Clearly 
\[\bm{v}\succcurlyeq\bm{f}(p)\ \ \t{and}\ \  \bm{f}(p)\in \Min W.\eqlbl{eq:v>f(p)}\]

Note that for any $\bm{u}\in \bar W$ there is $\bm{v}\in S$ such that $\bm{u}\succcurlyeq\bm{v}$. 
Therefore \ref{eq:v>f(p)} implies that
\[\bar W\subset\Up[\Min  W]\subset W.\]
Hence
$\bar W=W$; that is, $W$ is closed.
\qeds









\parit{Proof of Theorem~\ref{thm:web}.}
Without loss of generality we may assume that all $f^i$ are $1$-convex.

Given $\bm{v}=(v^0,v^1,\dots,v^\kay)\in\RR^{\kay+1}$, consider the function 
$h_{\bm{v}}\: \spc{X}\to \RR$ defined as
\[h_{\bm{v}}(p)=\max_i\{f^i(p)-v^i\}.\]
Note that $h_{\bm{v}}$ is $1$-convex.
Set 
$$\map(\bm{v})\df\argmin h_{\bm{v}}.$$
According to Lemma~\ref{lem:argmin(convex)}, $\map(\bm{v})$ is uniquely defined.

From the definition of web (\ref{def:web}) 
we have
$\map\circ\bm{f}(p)=p$ for any $p\in \Web\bm{f}$;
that is,  $\map$ is a left inverse to the restriction $\bm{f}|\Web\bm{f}$.
In particular, 
\[\Web\bm{f}=\Im\map.
\eqlbl{eq:Web=Im}\]

Given $\bm{v},\bm{w}\in\RR^{\kay+1}$,
set $p=\map (\bm{v})$ and $q=\map (\bm{w})$.
Since $h_{\bm{v}}$ and $h_{\bm{w}}$ are 1-convex, we have
\begin{align*}
h_{\bm{v}}(q)
&\ge 
h_{\bm{v}}(p)+\tfrac{1}{2}\cdot\dist[2]{p}{q}{},
&
h_{\bm{w}}(p)
&\ge 
h_{\bm{w}}(q)+\tfrac{1}{2}\cdot\dist[2]{p}{q}{}.
\end{align*}
Therefore,
\begin{align*}
\dist[2]{p}{q}{}
&\le 
2\cdot\sup_{x\in\spc{X}}\{ |h_{\bm{v}}(x)-h_{\bm{w}}(x)| \}
\le
\\
&\le 
2\cdot\max_{i}\{|v^i-w^i|\}.
\end{align*}
In particular,
$\map$ is $C^{\frac{1}{2}}$-continuous,
or $\bm{f}|\Web\bm{f}$ is a $C^{\frac{1}{2}}$-embedding.

As in Lemma~\ref{lem:Up-convex},
set $W=\Up\bm{f}(\spc{X})$ and $S=\Fr_{\RR^{\kay+1}}W$.
Recall that
$S$ is a convex hypersurface in $\RR^{\kay+1}$.
Clearly $\bm{f}(\Web\bm{f})=\Min W\subset S$.
From the definition of inner web we have
$\bm{v}\in \bm{f}(\InWeb\bm{f})$  
if and only if 
$\bm{v}\in S$ and
for any $i$ there is $\bm{w}=(w^0,w^1,\dots,w^\kay)\in W$ such that $w^j<v^j$ for all $j\not=i$.
Thus, $\bm{f}(\InWeb\bm{f})$ is open in $S$.
I.e., $\InWeb\bm{f}$ is $C^{\frac{1}{2}}$-homeomorphic to an open set in a convex hypersurface $S\subset\RR^{\kay+1}$.










\parit{(\ref{SHORT.thm:web:bary})+(\ref{SHORT.thm:web:lip-const}).}
Since $f^i$ is $1$-convex, for any $\bm{x}=(x^0,x^1,\dots,x^\kay)\in\Delta^\kay$, 
the convex combination 
\[\l(\sum_i x^i\cdot f^i\r)\:\spc{X}\to\RR\] 
is also $1$-convex.
Therefore, according to Lemma~\ref{lem:argmin(convex)}, $\spx{\bm{f}}(\bm{x})$ is defined for any $\bm{x}\in\Delta^\kay$.
 
For $\bm{x},\bm{y}\in\Delta^\kay$,
set 
\begin{align*}
f_{\bm{x}}
&=\sum_i x^i\cdot f^i,
&
f_{\bm{y}}
&=\sum_i y^i\cdot f^i,
\\
p
&=\spx{\bm{f}}(\bm{x}),
&
q
&=\spx{\bm{f}}(\bm{y}),
\\
\ell&=\dist{p}{q}{}
\end{align*}
Note the following:
\begin{itemize}
\item The function $\phi(t)=f_{\bm{x}}\circ\geod_{[p q]}(t)$ has minimum at $0$. 

Therefore $\phi^+(0)\ge 0$
\item The function $\psi(t)=f_{\bm{y}}\circ\geod_{[p q]}(t)$ has minimum at $\ell$. 

Therefore $\psi^-(\ell)\ge 0$.
\end{itemize}
From $1$-convexity of $f_{\bm{y}}$, we have
$\psi^+(0)+\psi^-(\ell)+\ell\le0$.

Let $\Lip$ be a Lipschitz constant for all $f^i$ in a neighborhood $\Omega\ni p$.
Then 
\[\psi^+(0)
\le 
\phi^+(0)+\Lip\cdot\|\bm{x}-\bm{y}\|_1,\] 
where $\|\bm{x}-\bm{y}\|_1=\sum_{i=0}^\kay|x^i-y^i|$.
I.e., given $\bm{x}\in\Delta^\kay$, there is a constant $\Lip$ such that
\begin{align*}
\dist{\spx{\bm{f}}(\bm{x})}{\spx{\bm{f}}(\bm{y})}{}
&=
\ell
\le
\\
&\le 
\Lip\cdot\|\bm{x}-\bm{y}\|_1
\end{align*}
for any $\bm{y}\in\Delta^\kay$.
In particular, there is $\eps>0$ such that if $\|\bm{x}-\bm{y}\|_1,$ $\|\bm{x}-\bm{z}\|_1 <\eps$, then $\spx{\bm{f}}(\bm{y})$, $\spx{\bm{f}}(\bm{z})\in\Omega$. 
Thus, the same argument as above implies 
\[\dist{\spx{\bm{f}}(\bm{y})}{\spx{\bm{f}}(\bm{z})}{}
=
\ell\le \Lip\cdot\|\bm{y}-\bm{z}\|_1\]
for any $\bm{y}$ and $\bm{z}$ sufficiently close to $\bm{x}$; that is, $\spx{\bm{f}}$ is locally Lipschitz.
Since  $\Delta^\kay$ is compact,  $\spx{\bm{f}}$ is Lipschitz.

Clearly $\spx{\bm{f}}(\Delta^\kay)\subset \Web \bm{f}$;
it remains to show that $\spx{\bm{f}}(\Delta^\kay)\supset \Web \bm{f}$.
According to Lemma~\ref{lem:Up-convex},
$W=\Up\bm{f}(\spc{X})$ is a closed convex set in $\RR^{\kay+1}$.
Let $p\in \Web \bm{f}$; 
clearly $\bm{f}(p)\in \Min W\subset S=\Fr_{\RR^{\kay+1}}W$.
Let $\bm{x}$ be a subnormal vector to $W$ at $\bm{f}(p)$.
According to Lemma~\ref{lem:Up-convex}, 
$\bm{x} \succcurlyeq\bm{0}$.
Without loss of generality we may assume that $\sum_i x^i=1$;
that is, $\bm{x}\in \Delta^\kay$.
By Lemma~\ref{lem:argmin(convex)},
$p$ is the unique minimum point of $\sum_i x^i\cdot f^i$;
that is, $p=\spx{\bm{f}}(\bm{x})$.
\qeds








\section{The case of CAT spaces}

Let $\bm{a}=(a^0,a^1,\dots,a^\kay)$ be a point array in a metric space $\spc{U}$.
Recall that 
$\dist{\bm{a}}{}{}$
denotes the distance map $(\dist{a^0}{}{},\dist{a^1}{}{},\dots,\dist{a^\kay}{}{})\:\spc{U}\to\RR^{\kay+1}$
which can be also regarded as a function array.
The \emph{radius of the point array}\index{radius of the point array} $\bm{a}$ is defined to be the radius of the set $\{a^0,a^1,\dots,a^\kay\}$;
that is,
\[\rad\bm{a}=\inf\set{r>0}{a^i\in \oBall(z,r)\ \t{for all}\ i \ \t{and some}\ z\in\spc{U}}.\]%!!!

Fix $\kappa\in\RR$.
Let $\bm{a}=(a^0,a^1,\dots,a^\kay)$ be a point array of radius $<\tfrac{\varpi\kappa}2$
in a metric space $\spc{U}$.
Consider function array $\bm{f}=(f^0,f^1,\dots,f^\kay)$ 
where 
\[f^i(x)\z=\md\kappa\dist[{{}}]{a^i}{x}{}.\]
Assuming the barycentric simplex $\spx{\bm{f}}$ is defined,
then $\spx{\bm{f}}$ is called the \emph{$\kappa$-barycentric simplex}\index{$\kappa$-barycentric simplex} for the point array $\bm{a}$;
it will be denoted by $\spx{\bm{a}}\mc\kappa$.
The points $a^0,a^1,\dots,a^\kay$ are called 
\emph{vertexes of $\kappa$-barycentric simplex}%
\index{vertexes of barycentric simplex}.
Note that once we say the $\kappa$-barycentric simplex is defined, 
we automatically assume that $\rad\bm{a}<\tfrac{\varpi\kappa}2$.


\begin{thm}{Theorem}\label{thm:cat-bary-web}
Let $\spc{U}$ be a complete length $\CAT\kappa$ space
and $\bm{a}=(a^0,a^1,\dots a^\kay)$ be a point array with radius $<\tfrac{\varpi\kappa}{2}$.
Then: 

\begin{subthm}{thm:cat-bary-web:Lip}
The $\kappa$-barycentric simplex $\spx{\bm{a}}\mc\kappa\:\Delta^\kay\to \spc{U}$ 
is defined. 
Moreover, $\spx{\bm{a}}\mc\kappa$ is a Lipschitz map,
and if $\Delta^\kay$ is equipped with the $\ell^1$-metric then its Lipschitz contant can be estimated in terms of $\kappa$ and the radius of $\bm{a}$ (in particular it does not depend on $\kay$).
\end{subthm}

\begin{subthm}{thm:cat-bary-web:web=Im(bary)}
$\Web\dist{\bm{a}}{}{}=\Im \spx{\bm{a}}\mc\kappa$.
Moreover, if a closed convex set $K$ contains all $a^i$ then $\Web\dist{\bm{a}}{}{}\subset K$.
\end{subthm}


\begin{subthm}{thm:cat-bary-web:mnfld}
The restriction%
\footnote{Recall that $\dist{\bm{a}^{\without 0}}{}{}$ denotes array $(\dist{a^1}{}{},\dots,\dist{a^\kay}{}{})$.}
$\dist{\bm{a}^{\without 0}}{}{}|\InWeb\dist{\bm{a}}{}{}$ is an open $C^{\frac12}$-embedding in $\RR^\kay$.
Thus there is an inverse of $\dist{\bm{a}^{\without 0}}{}{}|\InWeb\dist{\bm{a}}{}{}$, say $\map\:\RR^\kay\subto\spc{U}$.

The subfunction $f=\dist{a^0}{}{}\circ\map$ is semiconvex and locally Lipschitz.
Moreover, if $\kappa\le 0$, then $f$ is convex.
\end{subthm}

In particular, $\Web\dist{\bm{a}}{}{}$ is compact and
$\InWeb\dist{\bm{a}}{}{}$ is $C^{\frac12}$-homeomorphic to an open subset of $\RR^\kay$.
\end{thm}

\begin{thm}{Definition}\label{prop-def:web-embedding}
The submap $\map\:\RR^{\kay}\subto \spc{X}$ as in Theorem~\ref{thm:cat-bary-web:mnfld}
will be called the \emph{$\dist{\bm{a}}{}{}$-web embedding}\index{web embedding} 
with \emph{brace} $\dist{a^0}{}{}$.
\end{thm}%!!!???

\begin{thm}{Definition}
Let $\spc{U}$ be a complete length $\CAT\kappa$ space
and $\bm{a}=(a^0,a^1,\dots a^\kay)$ be a point array with radius $<\tfrac{\varpi\kappa}{2}$.
If  $\InWeb\dist{\bm{a}}{}{}\not=\emptyset$, then the point array $\bm{a}$ is called \emph{nondegenerate}\index{nondegenerate point array}.
\end{thm}

Lemma~\ref{lem:nondeg-test-with-balls} will provide examples of nondegenerate point arrays,
which can be used in Theorem~\ref{thm:cat-bary-web:mnfld}.

\begin{thm}{Corollary}\label{cor:LinDim>bary}
Let $\spc{U}$ be a complete length $\CAT\kappa$ space,
$\bm{a}=(a^0,a^1,\dots a^m)$ be a nondegenerate point array  
of radius $<\tfrac{\varpi\kappa}{2}$ in $\spc{U}$
and $\sigma=\spx{\bm{a}}\mc\kappa$ be corresponding $\kappa$-baricentric simplex.
Then for some $\bm{x}\in \Delta^m$
the differential $\d_{\bm{x}}\sigma$ is linear 
and the image $\Im\d_{\bm{x}}\sigma$
forms a subcone of the tangent cone $\T_{\sigma(\bm{x})}$ which is 
isometric to an $m$-dimensional Euclidean space.
\end{thm}


\parit{Proof.}
For shortness, denote by $\tau\:\spc{U}\to\RR^m$
 the distance map $\dist{\bm{a}^{\without 0}}{}{}$.

According to Theorem~\ref{thm:cat-bary-web},
$\sigma$ is Lipschitz
and 
the distance map 
$\tau$,
gives an open embedding of 
$\InWeb\dist{\bm{a}}{}{}=\sigma(\Delta^m)\backslash\sigma(\partial\Delta^m)$.
Clearly $\tau$ its Lipschitz.
According to Rademacher's theorem (\ref{thm:rademacher})
the differential 
$\d_{\bm{x}}(\tau\circ\sigma)$
is linear for almost all $\bm{x}\in\Delta^m$.
Further since $\InWeb\dist{\bm{a}}{}{}\not=\emptyset$,
the area formula ??? implies that $\d_{\bm{x}}(\tau\circ\sigma)$ is surjective on a set of positive masure of points  $\bm{x}\in\Delta^m$.

Note that $\d_{\bm{x}}(\tau\circ\sigma)=(\d_{\sigma(\bm{x})}\tau)\circ(\d_{\bm{x}}\sigma)$.
Applying Rademacher's theorem again, we get that
$\d_{\bm{x}}\sigma$ is linear for almost all $\bm{x}\in\Delta^m$,
at these points $\Im\d_{\bm{x}}\sigma$ forms a subcone of $\T_{\sigma(\bm{x})}$ which is isometric to a Euclidean space.
Clearly the dimension of $\Im\d_{\bm{x}}(\tau\circ\sigma)$ is at least as big as dimension of $\Im\d_{\bm{x}}\sigma$.
Hence the result.
\qeds


\parit{Proof of Theorem \ref{thm:cat-bary-web}.}
Fix $z\in\spc{U}$ and $r<\tfrac{\varpi\kappa}2$
such that $\dist{z}{a^i}{}<r$ for all $i$.
Note that the set $K\cap \cBall[z,r]$ is convex, closed, and contains all $a^i$.
Applying the theorem on short retract (\ref{thm:short-retract}),
we get the second part of (\ref{SHORT.thm:cat-bary-web:web=Im(bary)}).

The rest of the statements are proved first in case $\kappa\le 0$ 
and then the remaining case $\kappa>0$ is reduced to the case $\kappa=0$.

\parit{Case $\kappa\le 0$.}
Consider function array $f^i=\md\kappa\circ \dist[{{}}]{a^i}{}{}$.
From the definition of web (\ref{def:web}),
it is clear that $\Web\dist{\bm{a}}{}{}=\Web\bm{f}$.
Further, from the definition of $\kappa$-barycentric simplex,
$\spx{\bm{a}}\mc\kappa=\spx{\bm{f}}$.

All the functions $f^i$ are strongly convex (see \ref{function-comp}).
Thus, (\ref{thm:cat-bary-web:Lip}), (\ref{thm:cat-bary-web:web=Im(bary)}) and the first statements in (\ref{thm:cat-bary-web:web=Im(bary)}) 
 follow from Theorem \ref{thm:web}.
%???CHECK convexity ???

\parit{Case $\kappa>0$.}
Applying rescaling, we can assume $\kappa=1$;
so $\varpi\kappa=\varpi1=\pi$.

Set $\mathring{\spc{U}}=\Cone\spc{U}$;
according to ???, $\mathring{\spc{U}}\in\cCat{}{0}$.
Let us denote by $\iota$ the natural embedding of $\spc{U}$ as the unit sphere in $\mathring{\spc{U}}$ and 
$\proj\:\mathring{\spc{U}}\subto\spc{U}$ the submap,
defined as $\proj(v)=\iota^{-1}(v/|v|)$ for all 
$v\not=o$ in 
$\mathring{\spc{U}}$.
Note that there is $z\in\spc{U}$ and $\eps>0$ such that
the set 
\[K_\eps
=
\set{v\in\mathring{\spc{U}}}%
{\<\iota(z),v\>\ge\eps}\] 
contains all $\iota(a^i)$.
Note that 
$o\notin K_\eps$
and
the set $K_\eps$ is closed and convex.
The later follows from Theorem~\ref{thm:busemann},
since $v\mapsto -\<\iota(z),v\>$ is a Busemann function.


Denote by $\iota(\bm{a})$ the point array $(\iota(a^0), \iota(a^1),\dots,\iota(a^\kay))$ in $\mathring{\spc{U}}$. 
From the case $\kappa=0$,
we get $\Im \spx{\iota(\bm{a})}\mc0\subset K_\eps$.
In particular $\Im \spx{\iota(\bm{a})}\mc0\not\ni o$ and thus $\proj\circ\spx{\iota(\bm{a})}\mc0$ is defined.
Direct calculations show 
\[\spx{\bm{a}}\mc1
=
\proj\circ\spx{\iota(\bm{a})}\mc0
\ \ \t{and}\ \ 
\Web\dist[{{}}]{\bm{a}}{}{}=\proj[\Web\dist[{{}}]{\iota(\bm{a})}{}{}].\]
This reduces the case $\kappa=1$ of the theorem to the case $\kappa=0$,
which is proved already.
\qeds



\begin{thm}{Lemma}\label{lem:nondeg-test-with-balls}
Let $\spc{U}$ be a complete length $\CAT\kappa$ space,
$\bm{a}=(a^0,a^1,\dots a^\kay)$ be an array of radius $<\tfrac{\varpi\kappa}2$
and $B^i=\cBall[a^i,r^i]$ for some array of positive reals $(r^0,r^1,\dots,r^\kay)$.
Assume that
$\bigcap_i B^i=\emptyset$,
but
$\bigcap_{i\not=j} B^i\not=\emptyset$
for any $j$.
Then $\bm{a}$ is nondegenerate. 
\end{thm}

\parit{Proof.} 
Without loss of generality we can assume that $\spc{U}$ is geodesic and  $\diam\spc{U}\z<\varpi\kappa$.
If not, choose a point $z\in\spc{U}$ and $r<\tfrac{\varpi\kappa}{2}$ such that
$\dist{z}{a^i}{}\le r$
for each $i$
and consider $\cBall[z,r]$ instead of $\spc{U}$.
The later can be done since $\cBall[z,r]$ is convex and closed, 
thus $\cBall[z,r]$ is a complete length $\CAT\kappa$ space 
and $\Web\dist{\bm{a}}{}{}\subset\cBall[z,r]$;
see \ref{cor:convex-balls} and \ref{thm:cat-bary-web:web=Im(bary)}.

By Theorem~\ref{thm:cat-bary-web}, $\Web\dist{\bm{a}}{}{}$ is a compact set,
therefore there is a point $p\in\Web\dist{\bm{a}}{}{}$
minimizing the function 
\[f(x)=\max_i\{\dist{B^i}{x}{}\}=\max\{0,\dist{a^0}{x}{}-r^0,\dots,\dist{a^\kay}{x}{}-r^\kay\}.\]

By the  definition of web (\ref{def:web}), 
$p$ is also the minimum point of $f$ on $\spc{U}$.
Now let us show the following claim.

\begin{clm}{}
 $p\notin B^j$ for any $j$.
\end{clm}

Indeed, 
assume the contrary; that is, 
\[
p\in B^j
\eqlbl{eq:p-in-Bj}
\] 
for some $j$.
Then $p$ is a point of local minimum for the function 
\[h^j(x)=\max_{i\not=j}\{\dist{B^i}{x}{}\}.\]
Hence 
\[\max_{i\not=j}\{\mangle\hinge p x {a^i}\}\ge \frac\pi2
\]
for any $x\in\spc{U}$.
From angle comparison (\ref{cat-hinge}), it follows that 
$p$ is a global minimum of $h^j$ and hence
\[
p\in \bigcap_{i\not=j} B^i.
\]
The latter and \ref{eq:p-in-Bj} contradict $\bigcap_i B^i=\emptyset$. \claimqeds

From the definition of web, it also follows that 
\[\Web\dist{\bm{a}^{\without j}}{}{}\subset \bigcup_{i\not=j}B^i.\]
Therefore, the claim implies that
$p\notin\Web\dist{\bm{a}^{\without j}}{}{}$ for each $j$;
that is, $p\in\InWeb\dist{\bm{a}}{}{}$.
\qeds







\section{The case of finite-dimensional CAT spaces}

Recall that web embedding and its brace is defined in \ref{prop-def:web-embedding}.

\begin{thm}{Theorem}\label{thm:loc-lip-inverse}
Let $\spc{U}$ be a complete length $\CAT\kappa$ space,
$\LinDim\spc{U}=m$,
and $\bm{a}=(a^0,a^1,\dots a^m)$ be a point array in $\spc{U}$ 
with radius $<\tfrac{\varpi\kappa}{2}$.
Then 
the $\dist{\bm{a}}{}{}$-web embedding $\map\:\RR^m\subto\spc{U}$ with brace $\dist{a^0}{}{}$ is locally Lipschitz.
\end{thm}

Note that if $\bm{a}$ is degenerate,
that is, if $\InWeb\dist{\bm{a}}{}{}=\emptyset$, 
then
the domain of web embedding $\map$ above is empty and hence the conclusion of the theorem trivially holds.


As will be seen in Section~\ref{sec:web-embedding},
the web embedding as in the above theorem
is much better than bi-Lipschitz.
In particular, it %!!!differentiability and convexity properties are used???
can be used to construct bi-Lipschitz embeddings with constants arbitrarily close to $1$.

\begin{thm}{Lemma}\label{lem:nondeg-bs-test}
Let $\spc{U}$ be a complete length $\CAT\kappa$ space,
and $\bm{a}=(a^0,a^1,\dots a^\kay)$ be a point array with radius $<\tfrac{\varpi\kappa}{2}$.
Then for any $p\in \InWeb\dist{\bm{a}}{}{}$,
there is $\eps>0$ such that 
if for some $q\in \Web\dist{\bm{a}}{}{}$ and $b\in\spc{U}$
we have $\dist{p}{q}{}<\eps$, $\dist{p}{b}{}<\eps$ and $\mangle\hinge{q}{b}{a^i}<\tfrac\pi2+\eps$,
then the array $b,a^0,a^1,\dots,a^m$ is nondegenerate.
\end{thm}


\parit{Proof.}
Without loss of generality, we may assume that $\spc{U}$ is geodesic and $\diam\spc{U}<\varpi\kappa$.
If not, consider instead of $\spc{U}$,
a ball $\cBall[z,r]\subset\spc{U}$ 
for some $z\in\spc{U}$ 
and $r<\tfrac{\varpi\kappa}{2}$
such that $\dist{z}{a^i}{}\le r$ for each $i$.

From angle comparison (\ref{cat-hinge}), it follows that 
$p\in\InWeb\bm{a}$ if and only if both of the following conditions hold:
\begin{enumerate}
\item $\max_i\{\mangle\hinge p{a^i}{u}\}\ge \tfrac\pi2$ for any $u\in\spc{U}$,
\item\label{prop:<pi/2} for each $i$ there is $u^i\in\spc{U}$ such that $\mangle\hinge p{a^j}{u^i}<\tfrac\pi2$ for all $j\not=i$.
\end{enumerate}

Due to the semicontinuity of angles (\ref{lem:ang.semicont}),
there is $\eps>0$ such that for any $x\in \oBall(p,10\cdot\eps)$ we have
\[
\mangle\hinge {x}{a^j}{u^i}
<
\tfrac\pi2-10\cdot\eps
\ \ \t{for all}\ \ j\not=i.
\eqlbl{eq:<pi/2-eps}\]

Now assume for sufficiently small $\eps>0$
there are points $b\in\spc{U}$ and $q\in\Web\dist{\bm{a}}{}{}$  such that 
\[\dist{p}{q}{}<\eps,\ \  \dist{p}{b}{}<\eps,\ \  \mangle\hinge{q}{b}{a^i}<\tfrac\pi2+\eps\ \ \t{for all}\ \ i.
\eqlbl{eq:2nd-angle}\]
Acording to Theorem~\ref{thm:cat-bary-web:web=Im(bary)},
for all small $\eps>0$, we have 
\[\rad\{b,a^0,a^1,\dots,a^\kay\}<\tfrac{\varpi\kappa}2.\]
Fix sufficiently small $\delta>0$
and set 
\[v^i=\geod_{[q u^i]}(\tfrac13\cdot\delta)\ \ \t{and}\ \  w^i=\geod_{[v^i b]}(\tfrac23\cdot\delta).\]
Clearly,
\begin{align*}
\dist{b}{w^i}{}
&=
\dist{b}{v^i}{}-\tfrac23\cdot\delta
\le
\\
&\le
\dist{b}{q}{}-\tfrac13\cdot\delta.
\\
\intertext{Further, the inequalities \ref{eq:<pi/2-eps} and \ref{eq:2nd-angle} imply}
\dist{a^j}{w^i}{}
&<
\dist{a^j}{v^i}{}+\tfrac23\cdot\eps\cdot\delta
<
\\
&<
\dist{a^i}{q}{}-\eps\cdot\delta
<
\\
&<
\dist{a^i}{q}{}
\end{align*}
for all $i\not=j$.

Set $B^i=\cBall[a^i,\dist{a^i}{q}{}]$ and $B^{m+1}=\cBall[b,\dist{a^i}{q}{}-\tfrac13\cdot\delta]$.
Clearly 
\begin{align*}
&\!\!\!\!\bigcap_{i\not=m+1} B^i=\{q\},
\\
&\bigcap_{i\not=j}B^i\ni w^j\ \ \t{for}\ \ j\not=m+1.
\\
&\bigcap_{i}B^i=\{q\}\cap B^{m+1}=\emptyset.
\end{align*}
Lemma~\ref{lem:nondeg-test-with-balls} finishes the proof.
\qeds


\parit{Proof of \ref{thm:loc-lip-inverse}.}
From \ref{prop-def:web-embedding}, it is sufficient to show that 
the $\map$ is Lipshitz.
Assume the contrary; that is, there are sequences $\bm{y}_n, \bm{z}_n\to \bm{x}\in\Dom\map$ such that
\[\frac{\dist{\map(\bm{y}_n)}{\map(\bm{z}_n}{})}{|\bm{y}_n-\bm{z}_n|}
\to\infty
\ \ 
\t{as}
\ \ 
n\to\infty.
\eqlbl{eq:nonlip}\]
Set $p=\map(\bm{x})$,
$q_n=\map(\bm{y}_n)$, 
and $b_n=\map(\bm{z}_n)$.
From \ref{prop-def:web-embedding}, $p$, $q_n$, $b_n\in\InWeb\dist{\bm{a}}{}{}$
and $q_n,b_n\to p$ as $n\to\infty$.
Fix arbitrary $\eps>0$.
Note that \ref{eq:nonlip} implies
\[\mangle\hinge{q_n}{a^i}{b_n}<\tfrac\pi2+\eps
\]
for all $i>0$ and all large $n$.
Further, according to \ref{prop-def:web-embedding}, the subfunction
$(\dist{\bm{a}^0}{}{})\circ\map$ is locally Lipschitz.
Therefore we also have 
\[\mangle\hinge{q_n}{a^0}{b_n}<\tfrac\pi2+\eps
\]
for all large $n$.
According to Lemma~\ref{lem:nondeg-bs-test}, the point array $b_n,a^0,\dots,a^\kay$ for large $n$ is nondegenerate.

Applying Corollary~\ref{cor:LinDim>bary},
we get a contradiction
\qeds






















