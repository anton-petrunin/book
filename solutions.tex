%%!TEX root = the-solutions.tex

\chapter{Solutions}

%%!TEX root = arXiv.tex
\parbf{Exercise~\ref{exr-crofton}.}
Suppose $\alpha$ is a closed spherical curve. 
By Crofton's formula length of curve $\alpha$ on the sphere equals to $\pi\cdot n_\alpha$, where $n_\alpha$ denotes the average number of crossings of $\alpha$ with equators.

Since $\alpha$ is closed, almost all equators cross it at even number of points (we assume that $\infty$ is an even number.
If $\length \alpha<2\cdot\pi$ then $n_\alpha<2$.
Therefore there is an equator that does not cross $\alpha$ --- hence the result.


\parbf{Exercise~\ref{ex:complete=>complete};}
\textit{(\ref{SHORT.ex:complete=>complete:complete}).}
Note that any Cauchy sequence $(x_n)$ in $(\spc{X},\yetdist{}{}{})$ is also Cauchy in $\spc{X}$.
Since $\spc{X}$ is complete, $x_n$ converges; denote its limit by $x_\infty$.

Passing to a subsequence, we may assume that $\yetdist{x_{n-1}}{x_n}{}<\tfrac1{2^n}$.
It follows that there is a 1-Lipschitz curve $\alpha\:[0,1]\to (\spc{X},\yetdist{}{}{})$ such that $x_n=\alpha(\tfrac1{2^n})$ and $x_\infty=\alpha(0)$.
In particular $\yetdist{x_n}{x_\infty}{}\to0$ and $n\to\infty$.

\parit{(\ref{SHORT.ex:complete=>complete:compact}).}
Fix two points $x,y\in\spc{X}$ such that $\ell=\yetdist{x}{y}{}<\infty$.
Let $\alpha_n$ be a sequence of paths from $x$ to $y$ such that $\length(\alpha_n)\to\ell$ as $n\to \infty$.
Without loss of generality we may assume that each $\alpha_n$ is $(\ell+1)$-Lipschitz.

Since $\spc{X}$ is compact, there is a partial limit of $\alpha_n$ as $n\to \infty$,
denote it by $\alpha_\infty$.
By semicontinuity of length, $\length\alpha_\infty\le\ell$;
that is $\alpha$ is a shortest path in $\spc{X}$.

\parbf{Exercise~\ref{ex:no-geod}.}
The following example is suggested by Fedor Nazarov \cite{nazarov}.

\medskip

Consider the unit ball $(B,\rho_0)$
in the space $c_0$ of all sequences converging to zero equipped with the sup-norm.

Consider another metric $\rho_1$ which is different from $\rho_0$ by the conformal factor
\[\phi(\bm{x})=2+\tfrac{1}2\cdot x_1+\tfrac{1}4\cdot x_2+\tfrac{1}8\cdot x_3+\dots,\]
where $\bm{x}=(x_1,x_2\,\dots)\in B$.
That is, if $\bm{x}(t)$, $t\in[0,\ell]$, is a curve parametrized by $\rho_0$-length 
then its $\rho_1$-length is 
\[\length_{\rho_1}\bm{x}=\int\limits_0^\ell\phi\circ\bm{x}.\]
Note that the metric $\rho_1$ is bi-Lipschitz to~$\rho_0$.

Assume $\bm{x}(t)$ and $\bm{x}'(t)$ are two curves parametrized by $\rho_0$-length that differ only in the $m$-th coordinate, denoted by $x_m(t)$ and $x_m'(t)$ correspondingly.
Note that if $x'_m(t)\le x_m(t)$ for any $t$ and 
the function $x'_m(t)$ is locally $1$-Lipschitz at all $t$ such that $x'_m(t)< x_m(t)$, then 
\[\length_{\rho_1}\bm{x}'\le \length_{\rho_1}\bm{x}.\]
Moreover this inequality is strict if $x'_m(t)< x_m(t)$ for some~$t$.

Fix a curve $\bm{x}(t)$, $t\in[0,\ell]$, parametrized by  $\rho_0$-length.
We can choose $m$ large, so that $x_m(t)$ is sufficiently close to $0$ for any~$t$.
In particular, for some values $t$, we have $y_m(t)<x_m(t)$, where
\[y_m(t)=(1-\tfrac t\ell)\cdot x_m(0)
+\tfrac t\ell\cdot x_m(\ell)
-\tfrac 1{100}\cdot \min\{t,\ell-t\}.\]
Consider the curve $\bm{x}'(t)$ as above with
\[x'_m(t)=\min\{x_m(t),y_m(t)\}.\]
Note that $\bm{x}'(t)$ and $\bm{x}(t)$ have the same end points, and by the above
\[\length_{\rho_1}\bm{x}'<\length_{\rho_1}\bm{x}.\]
That is, for any curve $\bm{x}(t)$ in $(B,\rho_1)$, we can find a shorter curve $\bm{x}'(t)$ with the same end points.
In particular, $(B,\rho_1)$ has no geodesics.

\parbf{Exercise~\ref{exercise from BH}.}
The following example is taken from~\cite{BH}.

\medskip

\begin{wrapfigure}{r}{25mm}
\begin{lpic}[t(-0mm),b(-1mm),r(0mm),l(2mm)]{pics/square(1)}
\lbl[r]{3,11;{$\dots$}}
\lbl[r]{-.5,1;{$p$}}
\lbl[r]{-.5,21;{$q$}}
\lbl[l]{12,11;{$\spc{X}$}}
\end{lpic}
\end{wrapfigure}

Consider the following subset of $\R^2$ equipped with the induced length metric
\[
\spc{X}
=
\bigl((0,1]\times\{0,1\}\bigr)
\cup
\bigl(\{1,\tfrac12,\tfrac13,\dots\}\times[0,1]\bigr)
\]
Note that $\spc{X}$ is locally compact and geodesic.

Its completion $\bar{\spc{X}}$ is isometric to the closure of $\spc{X}$ equipped with the induced length metric;
$\bar{\spc{X}}$ is obtained from $\spc{X}$ by adding two points $p=(0,0)$ and $q=(0,1)$.

The point $p$ admits no compact neighborhood in $\bar{\spc{X}}$ 
and there is no geodesic connecting $p$ to $q$ in~$\bar{\spc{X}}$. \qeds 

\parbf{Exercise~\ref{ex:compact-in-lenght}}
Let $\spc{X}$ be a compact metric space.
Let us identify $\spc{X}$ with its image under Kuratowsky embedding.
Denote by $\spc{K}$ the \emph{linear} convex hull of $\spc{X}$ is the space of bounded functions on $\spc{X}$;
that is $x\in \spc{K}$ if and only if $x$ can not be separated from $\spc{X}$ by a hyperplane.

Since $\spc{X}$ is compact, so is $\spc{K}$.
It remains to observe that $\spc{K}$ is a length space.

\parbf{Exercise~\ref{ex:nonconvex-limit}.} Let $\spc{X}_n$ be the plane with the metric induced by $\ell^n$-norm and $f_n(x,y)=x$ for all $n$.
Observe that $\spc{X}_\o$ is the plane with the metric induced by $\ell^\infty$-norm where the limit function $f_\o(x,y)=x$ is not convex.

\parbf{Exercise~\ref{ex:adjacent-angles}.}
If $\mangle\hinge pxz+\mangle\hinge pyz< \pi$, then by the triangle inequality for angles (\ref{claim:angle-3angle-inq}) we have $\mangle\hinge pxy< \pi$.
The latter implies that a short $[xy]$ fails to be a geodesic around $p$.

\parbf{Exercise~\ref{ex:tangent-vect=o(t)}.}
By the definition of right derivative, there is a gedesic $\gamma$ such that both limits 
\[\limsup_{\eps\to0+}\frac{\dist{\alpha(\eps)}{\gamma(\eps)}{\spc{X}}}{\eps}
\quad\text{and}\quad
\limsup_{\eps\to0+}\frac{\dist{\beta(\eps)}{\gamma(\eps)}{\spc{X}}}{\eps}\]
are arbitrarily small.
Therefore 
\[\limsup_{\eps\to0+}\frac{\dist{\alpha(\eps)}{\beta(\eps)}{\spc{X}}}{\eps}=0.\]

\parbf{Exercise~\ref{ex:both-sided-diff}.}
Follows directly from the definition of right and left derivatives.

\parbf{Exercise~\ref{ex:schroeder-foetch}.}
Choose two non-Euclidean norms $|{*}|_{\spc{X}}$ and $|{*}|_{\spc{Y}}$ on $\RR^{10}$ such that the sum $|{*}|_{\spc{X}}+|{*}|_{\spc{Y}}$ is Euclidean.
See \cite{schroeder-foetch} for more details.

\parbf{Exercise~\ref{ex:(3+1)-expanding}.} 
Assume $\dist{p}{x^i}{}=\dist{q}{y^i}{}$ for each $i$.
Observe and use that
\[\dist{x^i}{x^j}{}\le\dist{y^i}{y^j}{}
\quad\iff\quad
\angk\kappa p{x^i}{x^j}\le \angk\kappa q{y^i}{y^j}.\]

\parbf{Exercise~\ref{ex:cbb-area}.} Follows from the overlap lemma (\ref{lem:extend-overlap}).

\parbf{Exercise~\ref{ex:cbb-wald}.} 

\parbf{Exercise~\ref{mink+alex=euclid}.}

\parbf{Exercise~\ref{ex:nongeod-cbb}.}
Modify the induced length metric on the unit sphere in the Hilbert space in small neighborhoods of a countable collection of points. To prove that the obtained space $\Alex0$, you may need to use the technique from Halbeisen's example (\ref{Halbeisen's example}).

\parbf{Exercise~\ref{ex:almost.geod}.} Mimic the proof of Theorem~\ref{thm:almost.geod}.

\parbf{Exercise~\ref{ex:non-split-almost.geod}.}

\parbf{Exercise~\ref{ex:cbb-geod-overlap}.}

\parbf{Exercise~\ref{ex:G-delta-not-thru}.}
Any nonnegatively curved metric on the plane with everywhere dense set of singular points will do the job;
by singular point we understand a point with total angle around it strictly smaller than $2\cdot\pi$.

Indeed, if $x_i$ is a singular point then there is $\eps_i>0$ such that no geodesic with ends outside of $\oBall(x_i,r)$ can meet the ball $\oBall(x_i,\eps_i\cdot r)$.
The set 
\[\Omega_n=\bigcup_i \oBall(x_i,\tfrac{\eps_i}n)\]
is open and everywhere dense.
Note that $\Omega_n$ may intersect a geodesic only at $\tfrac1n$-end of it.
The intersection of $\Omega_n$ is a G-delta dense set that does not intersect interior of any geodesic.



\parbf{Exercise~\ref{ex:equality-alexlemma}.}

\parbf{Exercise~\ref{ex:urysohn}.} See the construction of Urysohn's space \cite[3.11$\tfrac{3}{2}_+$]{gromov-MS}.

\parbf{Exercise~\ref{ex:lebedeva-petrunin}.}
Read \cite{lebedeva-petrunin}.

\parbf{Exercise~\ref{ex:fat-triangle}.} Apply angle-sidelength  monotonicity (\ref{cor:monoton}) twice. 

\parbf{Exercise~\ref{ex:busemann}.} The first part follows from angle-sidelength  monotonicity (\ref{cor:monoton}).
An example for the second part can be found among metrics on $\RR^2$ induced by a norm. (Compare to Exercise~\ref{mink+alex=euclid}.)

\parbf{Exercises \ref{ex:busemann-CBB} and \ref{ex:busemann-CBA}.}
By the definition of Busemann function,
\begin{align*}
\exp(&\sqrt{-\kappa}\cdot\bus_\gamma) 
= \exp \left[\lim_{t\to \infty} \sqrt{-\kappa}\cdot(\distfun{{\gamma (t)}}{}{} - t)\right]=
\\
&= \lim_{t\to \infty} \biggl(\exp \left[\sqrt{-\kappa}\cdot(\distfun{\gamma (t)}{}{} -t)\right]
+\exp\left[\sqrt{-\kappa}\cdot(-d_{\gamma (t)}-t)\right]\biggr)=
\\
&=  \lim_{t\to \infty} \left(2\cdot \cosh \left[\sqrt{-\kappa}\cdot\distfun{\gamma (t)}{}{}\right]\cdot \exp\left[\sqrt{-\kappa}\cdot(-t)\right]\right).
\end{align*}

By the function comparison definitions of $\CAT\kappa$ space (\ref{function-comp}) or $\Alex{\kappa}$ space (\ref{comp-kappa}),  for any $p\in \spc{U}$ the function\\ $f=\cosh \sqrt{-\kappa}\circ\distfun{p}{}{}$ satisfies $f''+\kappa \cdot f\ge 0$ (respectively  $f''+\kappa \cdot f\le 0$). The result follows.

\parbf{Exercise~\ref{ex:noncomplete-globalization}.} Read \cite{petrunin:globalization}.

\parbf{Exercise~\ref{ex:fixed-point}.} If $\diam(\spc{L}/G)>\tfrac\pi2$, then for some $x\in \spc{L}$ we have
\[\sup \set{\distfun{G\cdot x}(y)}{y\in \spc{L}}
>
\tfrac\pi2.\]
Use comparison to show that there is unique point $y^{*}$ that lies on the maximal distance from the orbit $G\cdot x$.
Observe that $y^{*}$ is a fixed point.

\parbf{Exercise~\ref{ex:kleiner}.}
This exercise is based on the main idea in \cite{hsiang-kleiner}.

\medskip

Assume there are 4 such points $x_1,x_2,x_3,x_4$
Note that from comparison the average of $\mangle\hinge{x_i}{x_j}{x_k}$ is at larger than $\tfrac\pi3$.
On the other hand, since each $x_i$ has spaces of directions $\le\tfrac12\cdot\mathbb{S}^n$, the average of $\mangle\hinge{x_i}{x_j}{x_k}$ is at most $\tfrac\pi3$ --- a contradiction.


\parbf{Exercise~\ref{ex:ccat-(3+1)}.}

\parbf{Exercise~\ref{ex:sba-2+2-short}.}

\parbf{Exercise~\ref{ex:CAT-mnfld=>ext.geod}.}
Suppose that a geodesic $[px]$ is not expendable behind $x$.
We may assume that $\dist{p}{x}{}<\varpi\kappa$;
otherwise move $p$ along the geodesic toward to $x$.

By the uniqueness of geodesics (\ref{thm:cat-unique}), any point $y$ in a neighborhood $\Omega\ni x$ is connected to $p$ by a unique geodesic path, denote it by $\gamma_y$.
Moreover, $h_t(y)=\gamma_y(t)$ defines a homotopy between the identity map of $\Omega$ and the constant map with value $p$;
this is the  so called \index{geodesic homotopy}\emph{geodesic homotopy}.

Since $[px]$ is not extendable, $x\notin h_t(\Omega)$ for any $t<1$.
In particular the local homology groups vanish at $x$ --- a contradiction.

\parbf{Exercise~\ref{ex:complete-space-of-dir}.} Choose a sequence of geodesic directions $\xi_n$ at $p$; denote by $\gamma_n$ the corresponding geodesics.
Since the space $\spc{U}$ is locally compact, we may pass a converging subsequence of $(\gamma_n)$; denote its limit by limit $\gamma_\infty$ and its direction by $\xi_\infty$.
By comparison, $\xi_\infty$ is a limit of $(\xi_n)$.

\parbf{Exercise~\ref{ex:convexity-CAT0}.} %???+PIC
It sufficient to show that if $v$ and $y$ are midpoints of geodesics $[uw]$ and $[xz]$ in $\spc{U}$, then
\[\dist{v}{y}{}\le \tfrac12\cdot(\dist{u}{x}{}+\dist{w}{z}{}).\]

Denote by $p$ the midpoint of $[uz]$.
Applying angle-sidelength  monotonicity (\ref{cor:monoton-cba}) twice, we get
\[\dist{v}{p}{}\le \tfrac12\cdot\dist{w}{z}{}.\]
The same way we obtain 
\[\dist{y}{p}{}\le \tfrac12\cdot\dist{u}{x}{}.\]
It remains to add these two inequalities and apply the triangle inequality.

(This inequality also follows directly from the majorization theorem (\ref{thm:major}).)

\parbf{Exercise~\ref{ex:equality-for-thin}.}

\parbf{Exercise~\ref{ex:busemann-CBA}.}
See the solution of Exercises~\ref{ex:busemann-CBB}.

\parbf{Exercise~\ref{ex:closest-point-projection}.}

\parbf{Exercise~\ref{ex:short-retraction-CBA(1)}.}
Without loss of generality, we may assume that $p\in K$.

If $\dist{K}{x}{}\ge\pi$, then set $\map[2](x)=p$.

Otherwise, if $\dist{K}{x}{}<\pi$, by Closest-point projection lemma~\ref{lem:closest point}, 
there is unique point $x^*\in K$ that minimizes distance to $x$;
that is, $\dist{x^*}{x}{}=\dist{K}{x}{}$.
Let us define $\ell_x$, $\phi_x$ and $\psi_x$ using the floowing identities:
\begin{align*}
\ell_x&=\dist{p}{x^*}{},
\\
\phi_x&=\tfrac\pi2-\dist[{{}}]{x^*}{x}{},
\\
\sin\psi_x&=\sin\phi_x\cdot\sin\ell_x, 
\ \ 0\le \psi_x\le \tfrac\pi2.
\intertext{Set}
\map[2](x)&=\geod_{[px^*]}(\psi_x).
\end{align*}

Note that $\map[2]$ is a retraction to $K$; 
that is,
$\map[2](x)\in K$ for any $x\in \spc{U}$
and 
$\map[2](a)=a$ for any $a\in K$.

Let us show that $\map[2]$ is short.
Given $x,y\in\oBall(K,\tfrac\pi2)$, set
\begin{align*}
x'&=\map[2](x)
&
y'&=\map[2](y)
\\
r&=\dist{x}{y}{}
&
r'&=\dist{x'}{y'}{}
\\
d&=\dist{x^*}{y^*}{}
&
\alpha&=\angk1{p}{x^*}{y^*}
\end{align*}

Note that 
\[\cos r\le 
\cos\phi_x\cdot\cos\phi_y
-
\cos d\cdot\sin\phi_x\cdot\sin\phi_y.\eqlbl{eq:cos(r)}\]

Indeed, if $x,y\notin K$,
then 
$\mangle\hinge{x^*}{x}{y*}, 
\mangle\hinge{y^*}{y}{x*}
\ge 
\tfrac\pi2$
and
the inequality~\ref{eq:cos(r)} follows from the Arm lemma (\ref{lem:arm}).
If $x\in K$ and $y\notin K$, we get \ref{eq:cos(r)}, by angle comparison (\ref{cat-hinge}) 
since $\mangle\hinge{y^*}{y}{x*}\ge \tfrac\pi2$.
The same way \ref{eq:cos(r)} is proved 
in case $x\notin K$ and $y\in K$.
Finally, if $x,y\in K$, $\phi_x=\phi_y=\tfrac\pi2$ and $r=d$;
that is, the inequality trivially holds.

Further note that
\[\cos\alpha
=
\frac{\cos d-\cos \ell_x\cdot\cos\ell_y}{\sin\ell_x\cdot\sin\ell_y}.\]
Applying angle-sidelength  monotonicity (\ref{cor:monoton-cba}) we get
\begin{align*}
\cos r'&\ge
\cos\psi_x\cdot\cos\psi_y
-
\cos \alpha \cdot\sin\psi_x\cdot\sin\psi_y=
\\
&=
\cos\psi_x\cdot\cos\psi_y
-(\cos d-\cos \ell_x\cdot\cos\ell_y)\cdot\sin\phi_x\cdot\sin\phi_y\ge
\\
&\ge \cos\psi_x\cdot\cos\psi_y
-\cos d\cdot\sin\phi_x\cdot\sin\phi_y
\end{align*}


Note that 
$\psi_x\le \phi_x$
and
$\psi_y\le \phi_y$;
in particular,
\[
\cos\phi_x\cdot\cos\phi_y\le \cos\psi_x\cdot\cos\psi_y.
\]
Hence 
\[\cos r'\ge \cos r;\]
that is, the restriction $\map[2]|\oBall(K,\tfrac\pi2)$ is short.
Clearly $\map[2]$ is continuous,
since the complement of $\oBall(K,\tfrac\pi2)$ is mapped to $p$,
we get that $\map[2]$ is short; that is,
\[r'\le r \eqlbl{eq:cos=<cos}\]
for any $x,y\in\spc{U}$.

If we have equality in \ref{eq:cos=<cos}
then 
\[\cos\ell_x\cdot\cos\ell_y\cdot\sin\phi_x\cdot\sin\phi_y=0.\]
If $K\subset \oBall(p,\tfrac\pi2)$, then $\ell_x,\ell_y<\tfrac\pi2$;
which implies that $x\in K$ or $y\in K$.
Without loss of generality we may assume that $x\in K$.

It remains to show that if $y\notin K$ 
then the inequality~\ref{eq:cos=<cos}
is strict.
If $\dist{K}{y}{}\ge\tfrac\pi2$, then \ref{eq:cos=<cos} holds since 
the left hand side is $<\tfrac\pi2$,
while right hand side is $\ge \tfrac\pi2$.
If $\dist{K}{y}{}<\tfrac\pi2$, then $\phi_y>0$ and clearly $\psi_y<\phi_y$,
hence the inequality~\ref{eq:cos=<cos} is strict.
\qeds

We fail to find a transparent geometric proof of the statement above.
Below you will find a geometric way to think about the construction; 
%%%DOWN
compare to the construction 
in the proof of Kirszbraun's theorem (\ref{thm:kirsz+}).
%%%UP

\parit{Geometric interpretation of the map $\map[2]$.}
Set $\mathring{\spc{U}}=\Cone \spc{U}$;
denote by $\mathring{K}$ the subcone of $\mathring{\spc{U}}$ spanned by $K$.
The space $\spc{U}$ can be naturally identified with the unit sphere in $\mathring{\spc{U}}$;
that is, the set 
\[\set{z\in \mathring{\spc{U}}}{|z|=1}.\]

According to \ref{thm:warp-curv-bound:cat}, $\mathring{\spc{U}}$ is $\CAT0$.
Note that $\mathring{K}$ forms a convex closed subset of $\mathring{\spc{U}}$.
According to \ref{lem:closest point}, for any point $x$ there is unique point $\hat x\in \mathring{K}$
that minimize the distance to $x$;
that is, $\dist{\hat x}{x}{}=\dist{K}{x}{}$.
(If $|\hat x|\ne0$, then in the notations above we have
$x^*=\tfrac1{|\hat x|}\cdot\hat x$.)

Consider the ray $t\mapsto t\cdot p$ in  $\mathring{\spc{U}}$.
According to ???, %ASK Stephanie???
for given $s\in \mathring{\spc{U}}$
the geodesics $\geod_{[s\ t\cdot p]}$ converge as $t\to\infty$ to a ray, 
say $\alpha_s\:[0,\infty)\to \mathring{\spc{U}}$.



Note that if $|x|=1$, then $|\hat x|\le 1$.
By assumption for any $a\in K$ the function $t\mapsto |\alpha_a(t)|$ is monotonicity increasing.
Therefore there is unique value $t_x\ge 0$ such that
$|\alpha_{\hat x}(t_x)|=1$.
Consider the map $\map[2]\:\spc{U}\to K$
defined as 
\[\map[2](x)=\alpha_{\hat x}(t_x).\]

\parbf{Exercise~\ref{ex:two-rays}.}
Consider the angle $A$ in the plane of measure $\pi-\alpha$.
Note that $A$ is $\CAT0$.
Therefore by Reshetnyak gluing theorem \ref{thm:gluing},
by gluing a side of $A$ to $\gamma_1$ in $\spc{U}$ we obtain a $\CAT0$ space, say $\spc{U}'$.

Note that $\gamma_2$ together with the other side of $A$ forms a both sides infinite geodesic, say $\gamma$ in $\spc{U}'$.
In particular, $\gamma$ is a convex set isometric to $\RR$.

Glue a half-plane along its boundary to $\gamma$.
By Reshetnyak gluing theorem \ref{thm:gluing} the obtained space is $\CAT0$.

It remains to note that this space can be obtained directly by gluing $\spc{U}$ to with $Q$ along $\gamma_1$ and $\gamma_2$.

\parbf{Exercise~\ref{ex:glue-spherical-suspension}.}
Since $K$ is $\pi$-convex, it is $\CAT1$.
By \ref{thm:warp-curv-bound:cat}, the spherical suspension $\Susp K$ is $\CAT1$ as well.
Let us glue $\Susp K$ to $\spc{U}$ by along $K$;
according to Reshetnyak's gluing theorem, the obtained space, say $\spc{U}'$ is $\CAT1$.

Consider the geodesic path $\gamma\:[0,1]$ from $p$ to a pole of the suspension in $\spc{U}'$.
Set $K_t=\spc{U}\cap\cBall[\gamma(t),\tfrac\pi2]$.
By \ref{cor:convex-balls}, $K_t$ is $\pi$-convex of any $t$ and monotonicity of the family should be evident.

\parit{Remark.}
Note that if one applies Shrafutdinov's construction to the family of convex sets provided by the exercise we get a short strong deformation retraction from $\cBall[p,\tfrac\pi2]$ to $K$;
that is, there is a family of maps $\phi_t\:\cBall[p,\tfrac\pi2]\to \cBall[p,\tfrac\pi2]$ such that 
the function $t\mapsto \dist{\phi_t(x)}{\phi_t(y)}{}$ is nonincreasing for any pair of points $x,y\in\cBall[p,\tfrac\pi2]$, $\phi_t(x)=x$ for any $x\in K$ and $\phi_1(\cBall[p,\tfrac\pi2])=K$. 
Moreover we can assume that there is a family of short maps $\phi_t(\cBall[p,\tfrac\pi2])= K_t$ and $\phi_t(x)=x$ for any $t$ and $x\in K_t$.
It leads to another solution of Exercise~\ref{ex:short-retraction-CBA(1)}.

\parbf{Exercise~\ref{ex:reshetnyak-doubling}.}

\parbf{Exercise~\ref{ex:branching}.}

\parbf{Exercise~\ref{ex:glue-spherical-suspension}.}

\parbf{Exercise~\ref{ex:isometric-majorization}.}
\textit{(Easier way.)} 
Let 
$(t,s)\mapsto \gamma_t(s)$ be the line-of-sight map 
for $\alpha$ from $\alpha(0)$,
and 
$(t,s)\mapsto \tilde \gamma_t(s)$ be the line-of-sight map 
for $\tilde \alpha$ from $\tilde \alpha(0)$.
Consider the map  $F\:\Conv\tilde \alpha\to \spc{U}$ such that 
$F\:\tilde \gamma_t(s)\mapsto \gamma_t(s)$.

Show that $F$ majorizes $\alpha$
and conclude that $F$ is distance-preserving.

\parit{(Harder way.)}
Prove and apply the following lemma together with the Majorization theorem.
\begin{thm}{Lemma}\label{lem:short+convex}
Let $\alpha$ and $\beta$ be two convex curves in $\Lob2\kappa$.
Assume 
\[\length \alpha=\length\beta<2\cdot\varpi\kappa\]
and there is a short bijecction $f\:\alpha\to\beta$.
Then $f$ is an isometry.
\end{thm}

\parbf{Exercise~\ref{ex:bishop}.}

\parbf{Exercise~\ref{ex:square}.}

\parbf{Exercise~\ref{ex:cover-branching-along-2-lines}.}

\parbf{Exercise~\ref{ex:cats-cradle}.}

\parbf{Exercise~\ref{ex:Hadamard--Cartan}.}

\parbf{Exercise~\ref{ex:CBB+CBA}.}

\parbf{Exercise~\ref{ex:5-point-CBA=>CBB}.}

\parbf{Exercise~\ref{ex:sturm}.}

\parbf{Exercise~\ref{ex:(3+1)-nonsufficient}.}

\parbf{Exercise~\ref{ex:strut+embedding}.}

\parbf{Exercise~\ref{ex:flat-in-CAT}.}

\parbf{Exercise~\ref{ex:flat-in-CBB}.}

\parbf{Exercise~\ref{ex:not-flat}.}

\parbf{Exercise~\ref{CBA-n-point}.}

\parbf{Exercise~\ref{ex:riemannian-kirszbraun-equality}.}

\parbf{Exercise~\ref{ex:perunin-stadler}.}

\parbf{Exercise~\ref{ex:isbell}.}

\parbf{Exercise~\ref{ex:kirszbrun-source}.}

\parbf{Exercise~\ref{ex:warp=<}.}

\parbf{Exercise~\ref{ex:convexity-in-cone}.}

\parbf{Exercise~\ref{ex:spherical-join}.}

\parbf{Exercise~\ref{ex:norays}.}

\parbf{Exercise~\ref{ex:d_q dist_p(v)=-<dri p q, v>-CAT}.}

\parbf{Exercise~\ref{ex:d_q dist_p(v)=-<dri p q, v>}.}

\parbf{Exercise~\ref{ex:d dist(grad)<0}.}

\parbf{Exercise~\ref{ex:df(v)=<grad f,v>}.}

\parbf{Exercise~\ref{ex:compact-dimension-cbb}.}

\parbf{Exercise~\ref{ex:sharafutdinov}.}

\parbf{Exercise~\ref{ex:elf-contracting}.}

\parbf{Exercise~\ref{ex:grad-curve-condition}.}

\parbf{Exercise~\ref{ex:grad-curve-analitic}.}

\parbf{Exercise~\ref{ex:grad-flow-bry}.}

\parbf{Exercise~\ref{ex:geodesic}.}

\parbf{Exercise~\ref{ex:gexp}.}

\parbf{Exercise~\ref{ex:bry-cover}.}


\parbf{ Exercise ~\ref{exr-crofton}.}
%By rescaling we can assume that $\kappa=1$. 
Let $\alpha$ be a closed curve in  $\mathbb{S}^2$ of length $\le 2\pi$.  We wish to prove that it's contained in a hemisphere in $\mathbb{S}^2$.
By approximation it's sufficient to prove this for  smooth curves of length $< 2\pi$ with transverse self-intersections. Furthermore, by changing such  a curve near its self-intersection points  it can be approximated with respect to the Hausdorff distance by simple closed curves. 
Thus, without loss of generality we can assume that $\alpha$ is a simple closed curve of length $<2\pi$.
By Crofton's formula we have that
\[
L(\alpha)=\frac 1 4 \int _{\mathbb{S}^2}\#(\alpha\cap v^\perp) \dd_v\vol_2
\]

Obviously,  if $\#(\alpha\cap v^\perp) =0$, then $\alpha$ is contained in one of the hemispheres determined by $v^\perp$. By the intermediate value theorem the same holds true if $\#(\alpha\cap v^\perp) =1$.
Suppose  $\#(\alpha\cap v^\perp) \ge 2$ for all $v\in\mathbb{S}^2$. Then Crofton's formula implies that
$L(\alpha)\ge \frac 1 4 \int_{\mathbb{S}^2}2=2\pi$. \qeds

\parbf{ Exercise ~\ref{ex:no-geod}.}
%Given a  metric graph $\Gamma$ let $\Mid(\Gamma)$ be the set of vertices of $\Gamma$ together with the baricenters of all edges in $\Gamma$ with the induced metric from $\Gamma$.

Given a metric graph $\Gamma$ define $P_k(\Gamma)$ as follows. Let $\Gamma^b$ be the barycentric subdivision of $\Gamma$ with the natural metric. For any two adjacent vertices $p,q\in\Gamma^b$ substitute the edge $[pq]$ by  a countable collection of intervals $\{I_i\}_{i\ge 1}$ of length $\dist{p}{q}{}+\frac{\dist{p}{q}{}}{2^ki}$ where one end of each $I_i$ is glued to $p$ and the other to $q$. Note that the resulting space $P_k(\Gamma)$ is again a metric graph  with an inner metric. 

Let $\spc{X}_0=[0,1]$ and define $\spc{X}_k$ for $k\ge 1$ inductively as $\spc{X}_k=P_k(\spc{X}_{k-1})$.

Let $\spc{Y}_k$ be the set of vertices of $\spc{X}_k$ with the induced metric. By construction the inclusion $\spc{Y}_k\subset \spc{Y}_{k+1}$ is distance preserving.

Let $\spc{Y}_\infty=\cup_{k\ge 1}\spc{Y}_k$ with the obvious metric and let $\spc{Y}=\bar {\spc{Y}}_\infty$ be its metric completion. Then $\spc{Y}$ is a length space since it satisfies the almost midpoint property. But it is not hard to see that no two distinct points in $\spc{Y}$ can be connected by a shortest geodesic. \qeds

\parbf{ Exercise ~\ref{exercise from BH}.}
The following example is from~\cite{BH}.

Consider the following subset of $\R^2$:

\[
\spc{X}=(0,1]\times\{0\}\cup (0,1]\times\{1\}\cup_{n\ge 1}\{1/n\}\times[0,1]
\]
Consider the induced inner metric on $\spc{X}$. It's obviously locally compact and geodesic.
However, it's immediate to check that its metric completion $\bar{\spc{X}}=[0,1]\times\{0\}\cup [0,1]\times\{1\}\cup_{n\ge 1}\{1/n\}\times[0,1]$ is neither. \qeds

\parbf{Besicovitch inequality \ref{ex:besicovitch-inq}.}
Fix $\eps>0$ and let
$f_1,f_2\dots\:\EE^m\subto \spc{X}$
be the short submaps such that 
\[\Im\Phi\subset\bigcup_n\Im f_n\]
and 
\[\sum_n\vol_m(\Dom f_n)
<
\LongMes_m(\Im\Phi)+\eps.\]

Consider the functions $\psi^i=\distfun{A^i}{}{}$
and the map $\bm{\psi}=(\psi^1,\dots,\psi^n)\:\spc{X}\to\RR^n$.
Note that for each $i$ and $n$,
the composition $\psi^i\circ f_n$ is 1-Lipschitz.
It follows that
\begin{align*}
|[\dd_x(\bm\psi\circ f_n)]^{\wedge n}|
&\le \lip(\psi^1\circ f_n)\cdots\lip(\psi^m\circ f_n)
\le
\\
&
\le 1.
\end{align*}
for almost all $x\in\Dom f_n$.

Note that 
\[\Im\bm\psi\circ\map
\supset
[0,a^1]\times\dots\times[0,a^n].\]


Applying Federer's area formula,
for 
$\bm\psi\circ f_n\:\RR^m\subto\RR^m$, 
we get 
\begin{align*}\vol_n(\Dom f_n)
&=
\int\limits_{\Im \bm\psi\circ f_n}|[\dd_x(\bm\psi\circ f_n)]^{\wedge n}|\cdot\dd_x\vol_m\le
\\
&\le \vol_n(\Im \bm\psi\circ f_n)
\end{align*}

\begin{align*}
\LongMes_m \spc{X}+\eps
&\ge \sum_n\,\vol_n(\Dom f_n)\ge
\\
&\ge \sum_n\,\int\limits_{\Dom f_n}\bigl|[\dd_x(\bm\psi\circ f_n)]^{\wedge n}\bigr|\cdot\dd_x\vol_n=
\\
&=\sum_n\,\vol_n(\Im \bm\psi\circ f_n)\ge
\\
&\ge \vol_n[\bm\psi(\spc{X})].
\end{align*}
Since $\eps>0$ is arbitrary, 
the result follows.
\qeds

\parbf{Exercise \ref{ex:ultra-unique-geod}.}
It is sufficient to show that midpoint $z$ of $[pq]$ lies in $\spc{X}$.
Take a sequnce of points $z_n\in \spc{X}$ such that $z_n\to z$ as $n\to\o$.

Arguing by contradiction,
assume that $z_\o\notin\spc{X}$.
Then, according to Lemma~\ref{lem:X-X^w}, there is a subsequence $(z'_n)$ of $(z_n)$ such that $z'_n\to z'\not= z$ as $n\to\o$.
Clearly $z'$ is a midpoint for $p$ and $q$.

According to \ref{cor:ulara-geod} $\spc{X}^\o$ is geodesic.
Choose two geodesics $[p z']$ and $[z' q]$;
together they form a geodesic from $p$ to $q$ in $\spc{X}^\o$ that is distinct from $[p q]$, a contradiction.
\qeds



\parbf{Exercise \ref{ex:lip+dist}.}
Applying partition of unity, we may assume that support of $f$ lies in the domain $\Omega$ that admits a bi-Lipschitz distance embedding $\bm{a}\:\spc{L}\to\RR^\kay$.
Choose sufficiently big constant; $\Const\ge ???$ will do.
Define 
$$\phi(\bm{x})
=
\min\set{\Const\cdot|\distfun{\bm{a}}{p}-\bm{x}|+f(p)}{p\in \Omega}.$$

\parbf{Exercise \ref{ex:d(grad)<0}.}
Let $\phi\can F\circ\distfun{\bm{a}}{}{}$ 
and $\psi\can G\circ\distfun{\bm{b}}{}{}$; 
clearly,
\begin{align*}
\dd_p\phi(v)
&=\sum_i\partial_i F\cdot (\dd_p\distfun{a^i}{}{})(v),
\\
\dd_p\psi(v)
&=\sum_i\partial_j G\cdot (\dd_p\distfun{b^j}{}{})(v).
\end{align*}
Applying the definition of gradient (\ref{def:grad}),
Theorem \ref{thm:differential-of-dist}
and the identities above, 
we get that 
for any choice of geodesics $[pa^i]$ the following holds
\begin{align*}
\dd_p\phi(\nabla_p\psi)
&=\sum_i\partial_i F
\cdot
(\dd_p\distfun{a^i}{}{})(\nabla_p\psi)
\le
\\
&\le
-\sum_i\partial_i F
\cdot
\<\dir{p}{a^i},\nabla_p\psi\>
\le
\\
&\le
-\sum_i
\partial_i F
\cdot
\dd_p\psi(\dir{p}{a^i})
=
\\
&=
-\sum_{i,j}
\partial_i F
\cdot
\partial_j G
\cdot
(\dd_p\distfun{b^j}{}{})(\dir{p}{a^i})
\le
\\
&\le
\sum_{i,j}
\partial_i F
\cdot
\partial_j G
\cdot
\cos\angk\kappa{p}{a^i}{b^j}=
\\
&=\sdk\kappa{p}{\phi}{\psi}
\end{align*}
\qedsf

\parbf{Exercise \ref{ex:df(v)=<grad f,v>}.}
Recall that given two vectors $v,w\in \T_p$ we write 
$v+w=0$ if $|v|=|w|$ and $\mangle(v,w)=\pi$.


According to ???,
for almost all $t\in\II$,
the right and left derivatives 
$\alpha^+(t),\alpha^-(t)\in \T_{\alpha(t)}$
are defined and $\alpha^+(t)+\alpha^-(t)=0$.
In particular, 
\[\<\nabla_pf,\alpha^+(t)\>+\<\nabla_pf,\alpha^-(t)\>
\ae 0\]
By ???, $f\circ\alpha$ is differentiable for almost all $t\in\II$.
Therefore 
\[\dd_pf(\alpha^+(t))+\dd_pf(\alpha^-(t))\ae
0.\]
Since 
\[\<\nabla_pf,\alpha^\pm(t)\>\ge \dd_pf(\alpha^\pm(t)),\]
we get the result.


\parbf{Exercise \ref{ex:nan-li}.}
Consider space $\hat{\spc{L}}=\spc{L}\times\{-1,+1\}$ with involution
$\psi\:(x,s)\mapsto (x,-s)$.
Let $\sim$ be the minimal equivalence relation on $\hat{\spc{L}}$
such that $(x,s)\approx (\iota(x),-s)$ for any $x\in\partial\spc{L}$. 

According to the Gluing theorem (\ref{thm:gluing-cbb}),
$\hat{\spc{L}}/\sim$ is an $m$-dimensional complete length $\Alex\kappa$ space.
Note that $\psi$ induce an isometry on $\hat{\spc{L}}/\sim$.

Finally notice that $\spc{L}/\iota=(\hat{\spc{L}}/\sim)/\psi$ 
and apply Theorem \ref{thm:CBB/G}.







\parbf{Exercise~\ref{ex:funny-S}.} In the proof we apply the following lemma from \cite{edwards}; 
it follows from the disjoint discs property.


\begin{thm}{Lemma}\label{lem:homomanifold-characterization}
Let $\spc{S}$ be a simplicial complex that 
is an $m$-dimensional homology manifold for some $m\ge 5$.
Assume all the vertices of
$\spc{S}$ have simply connected links.
Then $\spc{S}$ is a topological manifold.
\end{thm}


It is sufficient to construct a simplicial complex $\spc{S}$
such that 
\begin{itemize}
\item $\spc{S}$ is a closed $(m-1)$-dimensional homology manifold;
\item $\pi_1(\spc{S}\backslash\{v\})\ne0$ for some vertex $v$ in $\spc{S}$;
\item $\spc{S}\sim \mathbb{S}^{m-1}$; that is, $\spc{S}$ is homotopy equivalent to $\mathbb{S}^{m-1}$.
\end{itemize}

Indeed, assume such $\spc{S}$ is constructed.
Then the suspension
$\spc{R}\z=\Susp\spc{S}$
is an $m$-dimensional homology manifold with a natural triangulation coming from $\spc{S}$.
By Lemma~\ref{lem:homomanifold-characterization},
$\spc{R}$ is a topological manifold.
According to generalized Poincar\'{e} conjecture,
$\spc{R}\simeq\mathbb{S}^m$;
that is
$\spc{R}$ is homeomorphic to $\mathbb{S}^m$.
Since $\Cone \spc{S}\simeq \spc{R}\backslash\{s\}$ where $s$ denotes a south pole of the suspension 
and $\EE^m\simeq \mathbb{S}^m\backslash\{p\}$
for any point $p\in \mathbb{S}^m$
we get 
\[\Cone \spc{S}\simeq\EE^m.\]

Let us construct $\spc{S}$.
Fix an $(m-2)$-dimensional homology sphere $\Sigma$ with a triangulation such that $\pi_1\Sigma\ne0$.
According to \cite{kervaire} %it is a good readable paper, but I am sure the existance follows from sometheng written before
an example of that type exists for any $m\ge 5$.

Remove from $\Sigma$ one $(m-2)$-simplex.
Denote the obtained complex by $\Sigma'$.
Since $m\ge 5$, we have $\pi_1\Sigma=\pi_1\Sigma'$.

Consider the product $\Sigma'\times [0,1]$. 
Attach to it the cone over its boundary $\partial (\Sigma'\times [0,1])$.
Denote by $\spc{S}$ the obtained simplicial complex
and by $v$ the tip of the attached cone.

Note that $\spc{S}$ is homotopy equivalent to the spherical suspension over $\Sigma$ which is a simply connected homology sphere and hence is homotopy equivalent to $\mathbb{S}^{m-1}$.
  Hence  $\spc{S}\sim\mathbb{S}^{m-1}$.

The complement $\spc{S}\backslash\{v\}$ is homotopy equivalent to $\Sigma'$.
Therefore 
\[
\pi_1(\spc{S}\backslash\{v\})
=\pi_1\Sigma'
=\pi_1\Sigma\ne 0.
\]
That is, $\spc{S}$ satisfies the conditions above.


\parbf{Exercise~\ref{ex:set-with-smooth-bry:CBB}.}
Denote by $\Omega$ the interior of $K$; that is $\Omega=K\backslash S$.
Since $K$ is connected and its boundary is smooth, so is $\Omega$.

Recall that $k_1(p),\dots, k_{m-1}(p)$ denote the principle curvatures taken in the nondecreasing order of the surface $S$ at the point $p$. 

\parit{``if''-part.} 
Note that if $S$ is convex, then $K$ is locally convex;
that is any point $p\in K$ admits a neighborhood $U\ni p$ such that $U\cap K$ is convex.

Since $K$ is connected, by Theorem~\ref{thm:local-global-convexity}, $K$ is convex.
It follows that induced length metric on $K$ coincides with the Euclidean metric. 
In particular (3+1)-point comparison holds for any quadruple of points in $K$.

\begin{wrapfigure}{r}{20mm}
\begin{lpic}[t(-3mm),b(0mm),r(0mm),l(0mm)]{pics/pxy-nonconvex(1)}
\lbl[r]{1.75,11;$q$}
\lbl[r]{11,21;$x$}
\lbl[r]{9.5,1;$y$}
\end{lpic}
\end{wrapfigure}

\parit{``Only-if''-part.} 
If $S$ is not convex, then there is a triangle $[qxy]$ in $\EE^m$ such that two sides $[qx]$ and $[qy]$ lie in $\Omega$, but the side $[xy]$ does not completely lie in $K$.
Such a triangle can be found in the plane spanned by the normal vector $\nu(p)$ and the first principle direction at a point $p\in S$ where $k_1(p)<0$.

Clearly
\[\dist{x}{y}{K}>\dist{x}{y}{\EE^m}.
\eqlbl{eq:xy_K>xy}\]

On the other hand $[qx]$ and $[qy]$ form geodesics in $K$ and $\EE^m$.
Since $q$ lies in the interior of $K$, 
\[\mangle\hinge qxy_K=\mangle\hinge qxy_{\EE^m}.\]

If $K$ is $\Alex{0}$, then by hinge comparison (\ref{angle}) we have
\[\dist{x}{y}{K}\le\dist{x}{y}{\EE^m}.\]
The latter contradicts \ref{eq:xy_K>xy}.
\qeds





\parbf{Exercise~\ref{ex:poly-shefel}.}
If $\Omega$ is not two-convex, then there is a plane $\Pi$ in $\EE^3$ that contains a vertex $v$ of $K$ such that punctured neighborhood of $v$ in $\Pi$ lies in $\Omega$.
Choose a plane $\Pi'$ parallel and very close to $\Pi$ that cuts from the complement of $\Omega$ a little piramid $S$ with vertex~$v$.
Consider a small triangle $\triangle$ in $\Pi'$ wich surrounds the base of $S$.
Note that $\triangle$ is a geodesic triangle in $\Omega^*$
for which the point-on-side comparison \ref{cat-monoton}
fails.
That is, $\Omega^*$ is not locally $\CAT0$. %???+PIC

\parit{``if'' part.}
Since $\Omega$ is two-convex,
by Proposition~\ref{prop:stong-two-convex}, 
any point $v$ on the boundary of $K$ 
admits a conic neighborhood $U$ in $K$ 
such that the intersection $U\cap\Omega$ 
is formed by a finite collection of simply connected components.

It follows that any point $\Omega^*\backslash \Omega$ 
is locally isometric to a cone over spherical polygons.
Moreover since $\Omega$ is two-convex, 
each polygon does not contain a closed hemisphere in its interior. 
By Lemma ??? each of these spherical polygon is $\CAT1$. 
Therefore, by cone construction (\ref{thm:warp-curv-bound:cbb:a}) we get that $\Omega^*$ is locally $\CAT0$.
\qeds
