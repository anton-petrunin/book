%%!TEX root = the-solutions.tex

\chapter{Solutions}

\parbf{Exercise~\ref{ex:two-rays}.}
Consider the angle $A$ in the plane of measure $\pi-\alpha$.
Note that $A$ is $\CAT0$.
Therefore by Reshetnyak gluing theorem \ref{thm:gluing},
by gluing a side of $A$ to $\gamma_1$ in $\spc{U}$ we obtain a $\CAT0$ space, say $\spc{U}'$.

Note that $\gamma_2$ together with the other side of $A$ forms a both sides infinite geodesic, say $\gamma$ in $\spc{U}'$.
In particular, $\gamma$ is a convex set isometric to $\RR$.

Glue a half-plane along its boundary to $\gamma$.
By Reshetnyak gluing theorem \ref{thm:gluing} the obtained space is $\CAT0$.

It remains to note that this space can be obtained directly by gluing $\spc{U}$ to with $Q$ along $\gamma_1$ and $\gamma_2$.

\parbf{ Exercise ~\ref{exr-crofton}.}
%By rescaling we can assume that $\kappa=1$. 
Let $\alpha$ be a closed curve in  $\SS^2$ of length $\le 2\pi$.  We wish to prove that it's contained in a hemisphere in $\SS^2$.
By approximation it's sufficient to prove this for  smooth curves of length $< 2\pi$ with transverse self-intersections. Furthermore, by changing such  a curve near its self-intersection points  it can be approximated with respect to the Hausdorff distance by simple closed curves. 
Thus, without loss of generality we can assume that $\alpha$ is a simple closed curve of length $<2\pi$.
By Crofton's formula we have that
\[
L(\alpha)=\frac 1 4 \int _{\SS^2}\#(\alpha\cap v^\perp) \d_v\vol_2
\]

Obviously,  if $\#(\alpha\cap v^\perp) =0$, then $\alpha$ is contained in one of the hemispheres determined by $v^\perp$. By the intermediate value theorem the same holds true if $\#(\alpha\cap v^\perp) =1$.
Suppose  $\#(\alpha\cap v^\perp) \ge 2$ for all $v\in\SS^2$. Then Crofton's formula implies that
$L(\alpha)\ge \frac 1 4 \int_{\SS^2}2=2\pi$. \qeds

\parbf{ Exercise ~\ref{ex:no-geod}.}
%Given a  metric graph $\Gamma$ let $\Mid(\Gamma)$ be the set of vertices of $\Gamma$ together with the baricenters of all edges in $\Gamma$ with the induced metric from $\Gamma$.

Given a metric graph $\Gamma$ define $P_k(\Gamma)$ as follows. Let $\Gamma^b$ be the barycentric subdivision of $\Gamma$ with the natural metric. For any two adjacent vertices $p,q\in\Gamma^b$ substitute the edge $[pq]$ by  a countable collection of intervals $\{I_i\}_{i\ge 1}$ of length $\dist{p}{q}{}+\frac{\dist{p}{q}{}}{2^ki}$ where one end of each $I_i$ is glued to $p$ and the other to $q$. Note that the resulting space $P_k(\Gamma)$ is again a metric graph  with an inner metric. 

Let $\spc{X}_0=[0,1]$ and define $\spc{X}_k$ for $k\ge 1$ inductively as $\spc{X}_k=P_k(\spc{X}_{k-1})$.

Let $\spc{Y}_k$ be the set of vertices of $\spc{X}_k$ with the induced metric. By construction the inclusion $\spc{Y}_k\subset \spc{Y}_{k+1}$ is distance preserving.

Let $\spc{Y}_\infty=\cup_{k\ge 1}\spc{Y}_k$ with the obvious metric and let $\spc{Y}=\bar {\spc{Y}}_\infty$ be its metric completion. Then $\spc{Y}$ is a length space since it satisfies the almost midpoint property. But it is not hard to see that no two distinct points in $\spc{Y}$ can be connected by a shortest geodesic. \qeds

\parbf{ Exercise ~\ref{exercise from BH}.}
The following example is from~\cite{BH}.

Consider the following subset of $\R^2$:

\[
\spc{X}=(0,1]\times\{0\}\cup (0,1]\times\{1\}\cup_{n\ge 1}\{1/n\}\times[0,1]
\]
Consider the induced inner metric on $\spc{X}$. It's obviously locally compact and geodesic.
However, it's immediate to check that its metric completion $\bar{\spc{X}}=[0,1]\times\{0\}\cup [0,1]\times\{1\}\cup_{n\ge 1}\{1/n\}\times[0,1]$ is neither. \qeds

\parbf{Besicovitch inequality \ref{ex:besicovitch-inq}.}
Fix $\eps>0$ and let
$f_1,f_2\dots\:\EE^m\subto \spc{X}$
be the short submaps such that 
\[\Im\Phi\subset\bigcup_n\Im f_n\]
and 
\[\sum_n\vol_m(\Dom f_n)
<
\LongMes_m(\Im\Phi)+\eps.\]

Consider the functions $\psi^i=\distfun{A^i}{}{}$
and the map $\bm{\psi}=(\psi^1,\dots,\psi^n)\:\spc{X}\to\RR^n$.
Note that for each $i$ and $n$,
the composition $\psi^i\circ f_n$ is 1-Lipschitz.
It follows that
\begin{align*}
|[\d_x(\bm\psi\circ f_n)]^{\wedge n}|
&\le \lip(\psi^1\circ f_n)\cdots\lip(\psi^m\circ f_n)
\le
\\
&
\le 1.
\end{align*}
for almost all $x\in\Dom f_n$.

Note that 
\[\Im\bm\psi\circ\map
\supset
[0,a^1]\times\dots\times[0,a^n].\]


Applying Federer's area formula,
for 
$\bm\psi\circ f_n\:\RR^m\subto\RR^m$, 
we get 
\begin{align*}\vol_n(\Dom f_n)
&=
\int\limits_{\Im \bm\psi\circ f_n}|[\d_x(\bm\psi\circ f_n)]^{\wedge n}|\cdot\d_x\vol_m\le
\\
&\le \vol_n(\Im \bm\psi\circ f_n)
\end{align*}

\begin{align*}
\LongMes_m \spc{X}+\eps
&\ge \sum_n\,\vol_n(\Dom f_n)\ge
\\
&\ge \sum_n\,\int\limits_{\Dom f_n}\bigl|[\d_x(\bm\psi\circ f_n)]^{\wedge n}\bigr|\cdot\d_x\vol_n=
\\
&=\sum_n\,\vol_n(\Im \bm\psi\circ f_n)\ge
\\
&\ge \vol_n[\bm\psi(\spc{X})].
\end{align*}
Since $\eps>0$ is arbitrary, 
the result follows.
\qeds

\parbf{Exercise \ref{ex:ultra-unique-geod}.}
It is sufficient to show that midpoint $z$ of $[pq]$ lies in $\spc{X}$.
Take a sequnce of points $z_n\in \spc{X}$ such that $z_n\to z$ as $n\to\o$.

Arguing by contradiction,
assume that $z_\o\notin\spc{X}$.
Then, according to Lemma~\ref{lem:X-X^w}, there is a subsequence $(z'_n)$ of $(z_n)$ such that $z'_n\to z'\not= z$ as $n\to\o$.
Clearly $z'$ is a midpoint for $p$ and $q$.

According to \ref{cor:ulara-geod} $\spc{X}^\o$ is geodesic.
Choose two geodesics $[p z']$ and $[z' q]$;
together they form a geodesic from $p$ to $q$ in $\spc{X}^\o$ which is distinct from $[p q]$, a contradiction.
\qeds



\parbf{Exercise \ref{ex:lip+dist}.}
Applying partiiton of unity, we may assume that support of $f$ lies in the domain $\Omega$ which admits a bi-Lipschitz distance embedding $\bm{a}\:\spc{L}\to\RR^\kay$.
Choose sufficiently big constant; $\Const\ge ???$ will do.
Define 
$$\phi(\bm{x})
=
\min\set{\Const\cdot|\distfun{\bm{a}}{p}-\bm{x}|+f(p)}{p\in \Omega}.$$

\parbf{Exercise \ref{ex:d(grad)<0}.}
Let $\phi\can F\circ\distfun{\bm{a}}{}{}$ 
and $\psi\can G\circ\distfun{\bm{b}}{}{}$; 
clearly,
\begin{align*}
\d_p\phi(v)
&=\sum_i\partial_i F\cdot (\d_p\distfun{a^i}{}{})(v),
\\
\d_p\psi(v)
&=\sum_i\partial_j G\cdot (\d_p\distfun{b^j}{}{})(v).
\end{align*}
Applying the definition of gradient (\ref{def:grad}),
Theorem \ref{thm:differential-of-dist}
and the identities above, 
we get that 
for any choice of geodesics $[pa^i]$ the following holds
\begin{align*}
\d_p\phi(\nabla_p\psi)
&=\sum_i\partial_i F
\cdot
(\d_p\distfun{a^i}{}{})(\nabla_p\psi)
\le
\\
&\le
-\sum_i\partial_i F
\cdot
\<\dir{p}{a^i},\nabla_p\psi\>
\le
\\
&\le
-\sum_i
\partial_i F
\cdot
\d_p\psi(\dir{p}{a^i})
=
\\
&=
-\sum_{i,j}
\partial_i F
\cdot
\partial_j G
\cdot
(\d_p\distfun{b^j}{}{})(\dir{p}{a^i})
\le
\\
&\le
\sum_{i,j}
\partial_i F
\cdot
\partial_j G
\cdot
\cos\angk\kappa{p}{a^i}{b^j}=
\\
&=\sdk\kappa{p}{\phi}{\psi}
\end{align*}
\qedsf

\parbf{Exercise \ref{ex:df(v)=<grad f,v>}.}
Recall that given two vectors $v,w\in \T_p$ we write 
$v+w=0$ if $|v|=|w|$ and $\mangle(v,w)=\pi$.


According to ???,
for almost all $t\in\II$,
the right and left derivatives 
$\alpha^+(t),\alpha^-(t)\in \T_{\alpha(t)}$
are defined and $\alpha^+(t)+\alpha^-(t)=0$.
In particular, 
\[\<\nabla_pf,\alpha^+(t)\>+\<\nabla_pf,\alpha^-(t)\>
\ae 0\]
By ???, $f\circ\alpha$ is differentiable for almost all $t\in\II$.
Therefore 
\[\d_pf(\alpha^+(t))+\d_pf(\alpha^-(t))\ae
0.\]
Since 
\[\<\nabla_pf,\alpha^\pm(t)\>\ge \d_pf(\alpha^\pm(t)),\]
we get the result.


\parbf{Exercise \ref{ex:nan-li}.}
Consider space $\hat{\spc{L}}=\spc{L}\times\{-1,+1\}$ with involution
$\psi\:(x,s)\mapsto (x,-s)$.
Let $\sim$ be the minimal equivalence relation on $\hat{\spc{L}}$
such that $(x,s)\approx (\iota(x),-s)$ for any $x\in\partial\spc{L}$. 

According to the Gluing theorem (\ref{thm:gluing-cbb}),
$\hat{\spc{L}}/\sim$ is an $m$-dimensional complete length $\Alex\kappa$ space.
Note that $\psi$ induce an isometry on $\hat{\spc{L}}/\sim$.

Finally notice that $\spc{L}/\iota=(\hat{\spc{L}}/\sim)/\psi$ 
and apply Theorem \ref{thm:CBB/G}.

\parbf{Exercises \ref{ex:busemann-CBB} and \ref{ex:busemann-CBA}}
By the definition of Busemann function,
\begin{align*}
\exp(\sqrt{-\kappa}\cdot\bus_\gamma) 
&= \exp \lim_{t\to \infty} \sqrt{-\kappa}\cdot(\distfun{{\gamma (t)}}{}{} - t) 
\\
&= \lim_{t\to \infty} \left(\exp \sqrt{-\kappa}(d_{\gamma (t)} -t) + \exp
\sqrt{-\kappa}\cdot(-d_{\gamma (t)}-t)\right)\\
&=  \lim_{t\to \infty} 2 \cosh \sqrt{-\kappa}d_{\gamma (t)} \exp\sqrt{-\kappa}\cdot(-t).
\end{align*}

By the function comparison definitions of $\CAT\kappa$ space (\ref{function-comp}) or $\Alex{\kappa}$ space (\ref{comp-kappa}),  for any $p\in \spc{U}$ the function $f=\cosh \sqrt{-\kappa}\circ\distfun{p}{}{}$ satisfies $f''+\kappa \cdot f\ge 0$ (respectively  $f''+\kappa \cdot f\le 0$). The result follows.










\parbf{Exercise~\ref{ex:short-retraction-CBA(1)}.}
Without loss of generality, we may assume that $p\in K$.

If $\dist{K}{x}{}\ge\pi$, then set $\map[2](x)=p$.

Otherwise, if $\dist{K}{x}{}<\pi$, by Closest-point projection lemma~\ref{lem:closest point}, 
there is unique point $x^*\in K$ that minimizes distance to $x$;
that is, $\dist{x^*}{x}{}=\dist{K}{x}{}$.
Let us define $\ell_x$, $\phi_x$ and $\psi_x$ using the floowing identities:
\begin{align*}
\ell_x&=\dist{p}{x^*}{},
\\
\phi_x&=\tfrac\pi2-\dist[{{}}]{x^*}{x}{},
\\
\sin\psi_x&=\sin\phi_x\cdot\sin\ell_x, 
\ \ 0\le \psi_x\le \tfrac\pi2.
\intertext{Set}
\map[2](x)&=\geod_{[px^*]}(\psi_x).
\end{align*}

Note that $\map[2]$ is a retraction to $K$; 
that is,
$\map[2](x)\in K$ for any $x\in \spc{U}$
and 
$\map[2](a)=a$ for any $a\in K$.

Let us show that $\map[2]$ is short.
Given $x,y\in\oBall(K,\tfrac\pi2)$, set
\begin{align*}
x'&=\map[2](x)
&
y'&=\map[2](y)
\\
r&=\dist{x}{y}{}
&
r'&=\dist{x'}{y'}{}
\\
d&=\dist{x^*}{y^*}{}
&
\alpha&=\angk1{p}{x^*}{y^*}
\end{align*}

Note that 
\[\cos r\le 
\cos\phi_x\cdot\cos\phi_y
-
\cos d\cdot\sin\phi_x\cdot\sin\phi_y.\eqlbl{eq:cos(r)}\]

Indeed, if $x,y\notin K$,
then 
$\mangle\hinge{x^*}{x}{y*}, 
\mangle\hinge{y^*}{y}{x*}
\ge 
\tfrac\pi2$
and
the inequality~\ref{eq:cos(r)} follows from the Arm lemma (\ref{lem:arm}).
If $x\in K$ and $y\notin K$, we get \ref{eq:cos(r)}, by angle comparison (\ref{cat-hinge}) 
since $\mangle\hinge{y^*}{y}{x*}\ge \tfrac\pi2$.
The same way \ref{eq:cos(r)} is proved 
in case $x\notin K$ and $y\in K$.
Finally, if $x,y\in K$, $\phi_x=\phi_y=\tfrac\pi2$ and $r=d$;
that is, the inequality trivially holds.

Further note that
\[\cos\alpha
=
\frac{\cos d-\cos \ell_x\cdot\cos\ell_y}{\sin\ell_x\cdot\sin\ell_y}.\]
Applying angle-sidelength  monotonicity (\ref{cor:monoton-cba}) we get
\begin{align*}
\cos r'&\ge
\cos\psi_x\cdot\cos\psi_y
-
\cos \alpha \cdot\sin\psi_x\cdot\sin\psi_y=
\\
&=
\cos\psi_x\cdot\cos\psi_y
-(\cos d-\cos \ell_x\cdot\cos\ell_y)\cdot\sin\phi_x\cdot\sin\phi_y\ge
\\
&\ge \cos\psi_x\cdot\cos\psi_y
-\cos d\cdot\sin\phi_x\cdot\sin\phi_y
\end{align*}


Note that 
$\psi_x\le \phi_x$
and
$\psi_y\le \phi_y$;
in particular,
\[
\cos\phi_x\cdot\cos\phi_y\le \cos\psi_x\cdot\cos\psi_y.
\]
Hence 
\[\cos r'\ge \cos r;\]
that is, the restriction $\map[2]|\oBall(K,\tfrac\pi2)$ is short.
Clearly $\map[2]$ is continuous,
since the complement of $\oBall(K,\tfrac\pi2)$ is mapped to $p$,
we get that $\map[2]$ is short; that is,
\[r'\le r \eqlbl{eq:cos=<cos}\]
for any $x,y\in\spc{U}$.

If we have equality in \ref{eq:cos=<cos}
then 
\[\cos\ell_x\cdot\cos\ell_y\cdot\sin\phi_x\cdot\sin\phi_y=0.\]
If $K\subset \oBall(p,\tfrac\pi2)$, then $\ell_x,\ell_y<\tfrac\pi2$;
which implies that $x\in K$ or $y\in K$.
Without loss of generality we may assume that $x\in K$.

It remains to show that if $y\notin K$ 
then the inequality~\ref{eq:cos=<cos}
is strict.
If $\dist{K}{y}{}\ge\tfrac\pi2$, then \ref{eq:cos=<cos} holds since 
the left hand side is $<\tfrac\pi2$,
while right hand side is $\ge \tfrac\pi2$.
If $\dist{K}{y}{}<\tfrac\pi2$, then $\phi_y>0$ and clearly $\psi_y<\phi_y$,
hence the inequality~\ref{eq:cos=<cos} is strict.
\qeds

We fail to find a transparent geometric proof of the statement above.
Below you will find a geometric way to think about the construction; 
%%%DOWN
compare to the construction 
in the proof of Kirszbraun's theorem (\ref{thm:kirsz+}).
%%%UP

\parit{Geometric interpretation of the map $\map[2]$.}
Set $\mathring{\spc{U}}=\Cone \spc{U}$;
denote by $\mathring{K}$ the subcone of $\mathring{\spc{U}}$ spanned by $K$.
The space $\spc{U}$ can be naturally identified with the unit sphere in $\mathring{\spc{U}}$;
that is, the set 
\[\set{z\in \mathring{\spc{U}}}{|z|=1}.\]

According to ??? $\mathring{\spc{U}}$ is $\CAT0$.
Note that $\mathring{K}$ forms a convex closed subset of $\mathring{\spc{U}}$.
According to ???, for any point $x$ there is unique point $\hat x\in \mathring{K}$
which minimize the distance to $x$;
that is, $\dist{\hat x}{x}{}=\dist{K}{x}{}$.
(If $|\hat x|\ne0$, then in the notations above we have
$x^*=\tfrac1{|\hat x|}\cdot\hat x$.)

Consider the ray $t\mapsto t\cdot p$ in  $\mathring{\spc{U}}$.
According to ???, %ASK Stephanie???
for given $s\in \mathring{\spc{U}}$
the geodesics $\geod_{[s\ t\cdot p]}$ converge as $t\to\infty$ to a ray, 
say $\alpha_s\:[0,\infty)\to \mathring{\spc{U}}$.



Note that if $|x|=1$, then $|\hat x|\le 1$.
By assumption for any $a\in K$ the function $t\mapsto |\alpha_a(t)|$ is monotonicity increasing.
Therefore there is unique value $t_x\ge 0$ such that
$|\alpha_{\hat x}(t_x)|=1$.
Consider the map $\map[2]\:\spc{U}\to K$
defined as 
\[\map[2](x)=\alpha_{\hat x}(t_x).\]


\parbf{Exercise~\ref{ex:isometric-majorization}.}
\textit{(Easier way.)} 
Let 
$(t,s)\mapsto \gamma_t(s)$ be the line-of-sight map 
for $\alpha$ from $\alpha(0)$,
and 
$(t,s)\mapsto \~\gamma_t(s)$ be the line-of-sight map 
for $\~\alpha$ from $\~\alpha(0)$.
Consider the map  $F\:\Conv\~\alpha\to \spc{U}$ such that 
$F\:\~\gamma_t(s)\mapsto \gamma_t(s)$.

Show that $F$ majorizes $\alpha$
and conclude that $F$ is distance-preserving.

\parit{(Harder way.)}
Prove and apply the following lemma together with the Majorization theorem.
\begin{thm}{Lemma}\label{lem:short+convex}
Let $\alpha$ and $\beta$ be two convex curves in $\Lob2\kappa$.
Assume 
\[\length \alpha=\length\beta<2\cdot\varpi\kappa\]
and there is a short bijecction $f\:\alpha\to\beta$.
Then $f$ is an isometry.
\end{thm}

\parbf{Exercise~\ref{ex:lebedeva-petrunin}.}
Read \cite{lebedeva-petrunin}.

\parbf{Exercise~\ref{ex:funny-S}.} In the proof we apply the following lemma from \cite{edwards}; 
it follows from the disjoint discs property.


\begin{thm}{Lemma}\label{lem:homomanifold-characterization}
Let $\spc{S}$ be a simplicial complex which 
is an $m$-dimensional homology manifold for some $m\ge 5$.
Assume all the vertices of
$\spc{S}$ have simply connected links.
Then $\spc{S}$ is a topological manifold.
\end{thm}


It is sufficient to construct a simplicial complex $\spc{S}$
such that 
\begin{itemize}
\item $\spc{S}$ is a closed $(m-1)$-dimensional homology manifold;
\item $\pi_1(\spc{S}\backslash\{v\})\ne0$ for some vertex $v$ in $\spc{S}$;
\item $\spc{S}\sim \mathbb{S}^{m-1}$; that is, $\spc{S}$ is homotopy equivalent to $\mathbb{S}^{m-1}$.
\end{itemize}

Indeed, assume such $\spc{S}$ is constructed.
Then the suspension
$\spc{R}\z=\Susp\spc{S}$
is an $m$-dimensional homology manifold with a natural triangulation coming from $\spc{S}$.
By Lemma~\ref{lem:homomanifold-characterization},
$\spc{R}$ is a topological manifold.
According to generalized Poincar\'{e} conjecture,
$\spc{R}\simeq\mathbb{S}^m$;
that is
$\spc{R}$ is homeomorphic to $\mathbb{S}^m$.
Since $\Cone \spc{S}\simeq \spc{R}\backslash\{s\}$ where $s$ denotes a south pole of the suspension 
and $\EE^m\simeq \mathbb{S}^m\backslash\{p\}$
for any point $p\in \mathbb{S}^m$
we get 
\[\Cone \spc{S}\simeq\EE^m.\]

Let us construct $\spc{S}$.
Fix an $(m-2)$-dimensional homology sphere $\Sigma$ with a triangulation such that $\pi_1\Sigma\ne0$.
According to \cite{kervaire} %it is a good readable paper, but I am sure the existance follows from sometheng written before
an example of that type exists for any $m\ge 5$.

Remove from $\Sigma$ one $(m-2)$-simplex.
Denote the obtained complex by $\Sigma'$.
Since $m\ge 5$, we have $\pi_1\Sigma=\pi_1\Sigma'$.

Consider the product $\Sigma'\times [0,1]$. 
Attach to it the cone over its boundary $\partial (\Sigma'\times [0,1])$.
Denote by $\spc{S}$ the obtained simplicial complex
and by $v$ the tip of the attached cone.

Note that $\spc{S}$ is homotopy equivalent to the spherical suspension over $\Sigma$ which is a simply connected homology sphere and hence is homotopy equivalent to $\mathbb{S}^{m-1}$.
  Hence  $\spc{S}\sim\mathbb{S}^{m-1}$.

The complement $\spc{S}\backslash\{v\}$ is homotopy equivalent to $\Sigma'$.
Therefore 
\[
\pi_1(\spc{S}\backslash\{v\})
=\pi_1\Sigma'
=\pi_1\Sigma\ne 0.
\]
That is, $\spc{S}$ satisfies the conditions above.

\parbf{Exercise~\ref{ex:urysohn}.} See the construction of Urysohn's space \cite[3.11$\tfrac{3}{2}_+$]{gromov-MS}.

\parbf{Exercise~\ref{ex:set-with-smooth-bry:CBB}.}
Denote by $\Omega$ the interior of $K$; that is $\Omega=K\backslash S$.
Since $K$ is connected and its boundary is smooth, so is $\Omega$.

Recall that $k_1(p),\dots, k_{m-1}(p)$ denote the principle curvatures taken in the nondecreasing order of the surface $S$ at the point $p$. 

\parit{``if''-part.} 
Note that if $S$ is convex, then $K$ is locally convex;
that is any point $p\in K$ admits a neighborhood $U\ni p$ such that $U\cap K$ is convex.

Since $K$ is connected, by Theorem~\ref{thm:local-global-convexity}, $K$ is convex.
It follows that induced length metric on $K$ coincides with the Euclidean metric. 
In particular (3+1)-point comparison holds for any quadruple of points in $K$.

\begin{wrapfigure}{r}{20mm}
\begin{lpic}[t(-3mm),b(0mm),r(0mm),l(0mm)]{pics/pxy-nonconvex(1)}
\lbl[r]{1.75,11;$q$}
\lbl[r]{11,21;$x$}
\lbl[r]{9.5,1;$y$}
\end{lpic}
\end{wrapfigure}

\parit{``Only-if''-part.} 
If $S$ is not convex, then there is a triangle $[qxy]$ in $\EE^m$ such that two sides $[qx]$ and $[qy]$ lie in $\Omega$, but the side $[xy]$ does not completely lie in $K$.
Such a triangle can be found in the plane spanned by the normal vector $\nu(p)$ and the first principle direction at a point $p\in S$ where $k_1(p)<0$.

Clearly
\[\dist{x}{y}{K}>\dist{x}{y}{\EE^m}.
\eqlbl{eq:xy_K>xy}\]

On the other hand $[qx]$ and $[qy]$ form geodesics in $K$ and $\EE^m$.
Since $q$ lies in the interior of $K$, 
\[\mangle\hinge qxy_K=\mangle\hinge qxy_{\EE^m}.\]

If $K$ is $\Alex{0}$, then by hinge comparison (\ref{angle}) we have
\[\dist{x}{y}{K}\le\dist{x}{y}{\EE^m}.\]
The latter contradicts \ref{eq:xy_K>xy}.
\qeds





\parbf{Exercise~\ref{ex:poly-shefel}.}
If $\Omega$ is not two-convex, then there is a plane $\Pi$ in $\EE^3$ which contains a vertex $v$ of $K$ such that punctured neighborhood of $v$ in $\Pi$ lies in $\Omega$.
Choose a plane $\Pi'$ parallel and very close to $\Pi$ which cuts from the complement of $\Omega$ a little piramid $S$ with vertex~$v$.
Consider a small triangle $\triangle$ in $\Pi'$ wich surrounds the base of $S$.
Note that $\triangle$ is a geodesic triangle in $\Omega^*$
for which the point-on-side comparison \ref{cat-monoton}
fails.
That is, $\Omega^*$ is not locally $\CAT0$. %???+PIC

\parit{``if'' part.}
Since $\Omega$ is two-convex,
by Proposition~\ref{prop:stong-two-convex}, 
any point $v$ on the boundary of $K$ 
admits a conic neighborhood $U$ in $K$ 
such that the intersection $U\cap\Omega$ 
is formed by a finite collection of simply connected components.

It follows that any point $\Omega^*\backslash \Omega$ 
is locally isometric to a cone over spherical polygons.
Moreover since $\Omega$ is two-convex, 
each polygon does not contain a closed hemisphere in its interior. 
By Lemma ??? each of these spherical polygon is $\CAT1$. 
Therefore, by cone construction (\ref{thm:warp-curv-bound:cbb:a}) we get that $\Omega^*$ is locally $\CAT0$.
\qeds
