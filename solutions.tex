\chapter{Solutions}

\parbf{Besicovitch inequality \ref{ex:besicovitch-inq}.}
Fix $\eps>0$ and let
$f_1,f_2\dots\:\EE^m\subto \spc{X}$
be the short submaps such that 
\[\Im\Phi\subset\bigcup_n\Im f_n\]
and 
\[\sum_n\vol_m(\Dom f_n)
<
\LongMes_m(\Im\Phi)+\eps.\]

Consider the functions $\psi^i=\dist{A^i}{}{}$
and the map $\bm{\psi}=(\psi^1,\dots,\psi^n)\:\spc{X}\to\RR^n$.
Note that for each $i$ and $n$,
the composition $\psi^i\circ f_n$ is 1-Lipschitz.
It follows that
\begin{align*}
|[\d_x(\bm\psi\circ f_n)]^{\wedge n}|
&\le \lip(\psi^1\circ f_n)\cdots\lip(\psi^m\circ f_n)
\le
\\
&
\le 1.
\end{align*}
for almost all $x\in\Dom f_n$.

Note that 
\[\Im\bm\psi\circ\map
\supset
[0,a^1]\times\dots\times[0,a^n].\]


Applying Federer's area formula,
for 
$\bm\psi\circ f_n\:\RR^m\subto\RR^m$, 
we get 
\begin{align*}\vol_n(\Dom f_n)
&=
\int\limits_{\Im \bm\psi\circ f_n}|[\d_x(\bm\psi\circ f_n)]^{\wedge n}|\cdot\d_x\vol_m\le
\\
&\le \vol_n(\Im \bm\psi\circ f_n)
\end{align*}

\begin{align*}
\LongMes_m \spc{X}+\eps
&\ge \sum_n\,\vol_n(\Dom f_n)\ge
\\
&\ge \sum_n\,\int\limits_{\Dom f_n}\bigl|[\d_x(\bm\psi\circ f_n)]^{\wedge n}\bigr|\cdot\d_x\vol_n=
\\
&=\sum_n\,\vol_n(\Im \bm\psi\circ f_n)\ge
\\
&\ge \vol_n[\bm\psi(\spc{X})].
\end{align*}
Since $\eps>0$ is arbitrary, 
the result follows.
\qeds

\parbf{Exercise \ref{ex:ultra-unique-geod}.}
It is sufficient to show that midpoint $z$ of $[pq]$ lies in $\spc{X}$.
Take a sequnce of points $z_n\in \spc{X}$ such that $z_n\to z$ as $n\to\o$.

Arguing by contradiction,
assume that $z_\o\notin\spc{X}$.
Then, according to Lemma~\ref{lem:X-X^w}, there is a subsequence $(z'_n)$ of $(z_n)$ such that $z'_n\to z'\not= z$ as $n\to\o$.
Clearly $z'$ is a midpoint for $p$ and $q$.

According to \ref{cor:ulara-geod} $\spc{X}^\o$ is geodesic.
Choose two geodesics $[p z']$ and $[z' q]$;
together they form a geodesic from $p$ to $q$ in $\spc{X}^\o$ which is distinct from $[p q]$, a contradiction.
\qeds



\parbf{Exercise \ref{ex:lip+dist}.}
Applying partiiton of unity, we may assume that support of $f$ lies in the domain $\Omega$ which admits a bi-Lipschitz distance embedding $\bm{a}\:\spc{L}\to\RR^\kay$.
Choose sufficiently big constant; $\Const\ge ???$ will do.
Define 
$$\phi(\bm{x})
=
\min\set{\Const\cdot|\dist{\bm{a}}{p}{}-\bm{x}|+f(p)}{p\in \Omega}.$$

\parbf{Exercise \ref{ex:d(grad)<0}.}
Let $\phi\can F\circ\dist{\bm{a}}{}{}$ 
and $\psi\can G\circ\dist{\bm{b}}{}{}$; 
clearly,
\begin{align*}
\d_p\phi(v)
&=\sum_i\partial_i F\cdot (\d_p\dist{a^i}{}{})(v),
\\
\d_p\psi(v)
&=\sum_i\partial_j G\cdot (\d_p\dist{b^j}{}{})(v).
\end{align*}
Applying the definition of gradient (\ref{def:grad}),
Theorem \ref{thm:differential-of-dist}
and the identities above, 
we get that 
for any choice of geodesics $[pa^i]$ the following holds
\begin{align*}
\d_p\phi(\nabla_p\psi)
&=\sum_i\partial_i F
\cdot
(\d_p\dist{a^i}{}{})(\nabla_p\psi)
\le
\\
&\le
-\sum_i\partial_i F
\cdot
\<\dir{p}{a^i},\nabla_p\psi\>
\le
\\
&\le
-\sum_i
\partial_i F
\cdot
\d_p\psi(\dir{p}{a^i})
=
\\
&=
-\sum_{i,j}
\partial_i F
\cdot
\partial_j G
\cdot
(\d_p\dist{b^j}{}{})(\dir{p}{a^i})
\le
\\
&\le
\sum_{i,j}
\partial_i F
\cdot
\partial_j G
\cdot
\cos\angk\kappa{p}{a^i}{b^j}=
\\
&=\sdk\kappa{p}{\phi}{\psi}
\end{align*}
\qedsf

\parbf{Exercise \ref{ex:df(v)=<grad f,v>}.}
Recall that given two vectors $v,w\in \T_p$ we write 
$v+w=0$ if $|v|=|w|$ and $\mangle(v,w)=\pi$.


According to ???,
for almost all $t\in\II$,
the right and left derivatives 
$\alpha^+(t),\alpha^-(t)\in \T_{\alpha(t)}$
are defined and $\alpha^+(t)+\alpha^-(t)=0$.
In particular, 
\[\<\nabla_pf,\alpha^+(t)\>+\<\nabla_pf,\alpha^-(t)\>
\ae 0\]
By ???, $f\circ\alpha$ is differentiable for almost all $t\in\II$.
Therefore 
\[\d_pf(\alpha^+(t))+\d_pf(\alpha^-(t))\ae
0.\]
Since 
\[\<\nabla_pf,\alpha^\pm(t)\>\ge \d_pf(\alpha^\pm(t)),\]
we get the result.


\parbf{Exercise \ref{ex:nan-li}.}
Consider space $\hat{\spc{L}}=\spc{L}\times\{-1,+1\}$ with involution
$\psi\:(x,s)\mapsto (x,-s)$.
Let $\sim$ be the minimal equivalence relation on $\hat{\spc{L}}$
such that $(x,s)\approx (\iota(x),-s)$ for any $x\in\partial\spc{L}$. 

According to the Gluing theorem (\ref{thm:gluing-cbb}),
$\hat{\spc{L}}/\sim\in \CBB{m}{\kappa}$.
Note that $\psi$ induce an isometry on $\hat{\spc{L}}/\sim$.

Finally notice that $\spc{L}/\iota=(\hat{\spc{L}}/\sim)/\psi$ 
and apply Theorem \ref{thm:CBB/G}.

\parbf{Exercises \ref{ex:busemann-CBB} and \ref{ex:busemann-CBA}}
By the definition of Busemann function,
\begin{align*}
\exp(\sqrt{-\kappa}\cdot\bus_\gamma) 
&= \exp \lim_{t\to \infty} \sqrt{-\kappa}\cdot(\dist{{\gamma (t)}}{}{} - t) 
\\
&= \lim_{t\to \infty} \left(\exp \sqrt{-\kappa}(d_{\gamma (t)} -t) + \exp
\sqrt{-\kappa}\cdot(-d_{\gamma (t)}-t)\right)\\
&=  \lim_{t\to \infty} 2 \cosh \sqrt{-\kappa}d_{\gamma (t)} \exp\sqrt{-\kappa}\cdot(-t).
\end{align*}

By the function comparison definitions of $\Cat{}{\kappa}$ (\ref{function-comp}) or $\CBB{}{\kappa}$ (\ref{comp-kappa}),  for any $p\in \spc{U}$ the function $f=\cosh \sqrt{-\kappa}\circ\dist{p}{}{}$ satisfies $f''+\kappa \cdot f\ge 0$ (respectively  $f''+\kappa \cdot f\le 0$). The result follows.

