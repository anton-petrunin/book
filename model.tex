%%!TEX root = the-model.tex
\chapter{Model plane}

\section{Trigonometry}\label{model}

Given a real number $\kappa$, the \emph{model $\kappa$-\hspace{0pt}plane}\index{plane!$\kappa$-plane} will be a complete simply connected 2-\hspace{0pt}dimensional Riemannian manifold of constant curvature~$\kappa$.

The  model $\kappa$-plane  will be denoted by $\Lob2\kappa$.
\begin{itemize}
\item If $\kappa>0$, $\Lob2\kappa$\index{$\Lob2\kappa$} is isometric to a sphere of radius $\tfrac{1}{\sqrt{\kappa}}$; the unit sphere $\Lob21$ will be also denoted by $\mathbb{S}^2$.
\item If $\kappa=0$, $\Lob2\kappa$ is the Euclidean plane, which is also denoted by $\EE^2$ 
\item If $\kappa<0$, $\Lob2\kappa$ is the  Lobachevsky plane with curvature $\kappa$.
\end{itemize}



Set $\varpi\kappa=\diam\Lob2\kappa$\index{$\varpi\kappa$}, so 
$\varpi\kappa=\infty$ if $\kappa\le0$ and $\varpi\kappa=\pi/\sqrt{\kappa}$ if $\kappa>0$.

The distance between points $x,y\in \Lob2\kappa$ will be denoted by $\dist{x}{y}{}$\index{$\dist{*}{*}{}$}, and $[x y]$\index{$[{*}{*}]$} 
will denote the segment connecting $x$ and $y$. 
The segment $[x y]$ is uniquely defined for $\kappa\le 0$ and for $\kappa>0$ it is defined uniquely if $\dist{x}{y}{}<\varpi\kappa=\pi/\sqrt{\kappa}$.

A triangle in $\Lob2\kappa$, with vertices $x,y,z$ will be denoted by $\trig x y z$\index{$\trig{{*}}{{*}}{{*}}$}.
Formally, a triangle is an ordered set of its sides, so $\trig x y z$ is just a short notation for a triple $([y z],[z x],[x y])$.

The angle of $\trig x y z$ at $x$ will be denoted by $\mangle\hinge xyz$\index{$\mangle$}.

By $\modtrig\kappa\{a,b,c\}$\index{$\modtrig\kappa$!$\modtrig\kappa\{{*},{*},{*}\}$} we denote a triangle in 
$\Lob2\kappa$ with sidelengths $a,b,c$, so 
$\trig x y z=\modtrig\kappa\{a,b,c\}$ means that $x,y,z\in \Lob2\kappa$  are such that 
\[\dist{x}{y}{}=c,\quad \dist{y}{z}{}=a,\quad \dist{z}{x}{}=b.\]
For $\modtrig\kappa\{a,b,c\}$ to be defined, the sides $a,b,c$ must satisfy the triangle inequality.  If $\kappa>0$, we 
require 
in addition that $a+b+c<2\cdot\varpi\kappa$; 
otherwise $\modtrig\kappa\{a,b,c\}$ is considered to be undefined.

\parbf{Trigonometric functions.}
We will need three ``trigonometric functions'' in $\Lob2\kappa$: $\cs\kappa$, $\sn\kappa$ and $\md\kappa$;
$\cs{}$ stands for \emph{cosine}, $\sn{}$ stands for \emph{sine} and $\md{}$, for \emph{modified distance}. 

They are defined as the solutions of the following initial value problems respectively:
\[
\begin{cases}
 x''+\kappa\cdot x=0,\\
 x(0)=1,\\
 x'(0)=0.
 \end{cases} 
  \quad 
 \begin{cases}
 y''+\kappa\cdot y=0,\\
 y(0)=0,\\
 y'(0)=1.
 \end{cases} 
\quad
 \begin{cases}
 z''+\kappa\cdot z=1,\\
 z(0)=0,\\
 z'(0)=0.
 \end{cases}  
\]

Namely we set $\cs\kappa(t)=x(t)$, $\sn\kappa(t)=y(t)$ and 
\[
\md\kappa(t)=
\begin{cases}
z(t)& \textrm{ if } 0\le t\le \varpi\kappa,
\\
\tfrac{2}{\kappa}& \textrm{ if  } t> \varpi\kappa.
\end{cases}
\]

Here are the tables which relate our trigonometric functions to the standard ones, where 
we take $\kappa>0$:\index{$\md\kappa$}\index{$\sn\kappa$}\index{$\cs\kappa$}
\begin{align*}
&\sn{\pm\kappa}=\tfrac{1}{\sqrt{\kappa}}\cdot \sn{\pm1}({x}\cdot {\sqrt{\kappa}});
&&\cs{\pm\kappa}=\cs{\pm1}({x}\cdot {\sqrt{\kappa}});
\\
&\sn{-1} x=\sinh x;&&\cs{-1} x=\cosh x;\\
&\sn{0} x=x;
&&\cs{0} x=1;\\
&\sn{1} x=\sin x;	&&\cs{1} x= \cos x.
\end{align*}
	
\begin{align*}
\md{\pm\kappa}
&=
\tfrac{1}{\kappa}\cdot \md{\pm1}({x}\cdot {\sqrt{\kappa}});
\\
\md{-1} x&=\cosh x-1;
\\
\md{0} x&=\tfrac{1}{2}\cdot x^2; 
\\
\md{1} x
&=
\left[
\begin{aligned}
&1-\cos x&&\t{for}&&x\le \pi,
\\
&2&&\t{for}&&x >\pi.
\end{aligned}
\right.	&
\end{align*}

\begin{wrapfigure}{r}{31mm}
\begin{lpic}[t(-5mm),b(0mm),r(0mm),l(0mm)]{pics/treug(1)}
\lbl[b]{30,18,-90;{\small $a=\side\kappa\{\phi;b,c\}$}}
\lbl[br]{16,27;$b$}
\lbl[tr]{16,10;$c$}
\lbl[l]{5,19;{\scriptsize $\phi\ =\tangle\mc\kappa\{a;b,c\}$}}
\end{lpic}
\end{wrapfigure}


Note that
%\[\md\kappa(0)=(\md\kappa)'(0)=0\ \ \t{and}\ \ 
%(\md\kappa)''(x)=1-\kappa\cdot\md\kappa(x)\ \ \t{for}\ \ x<\varpi\kappa.\]
%Also,
\[
\mdk(x)=\int\limits_0^x
\snk(x)\cdot\dd x \ \t{ for }\ \ x\le \varpi\kappa
\]
%In other words, $\md\kappa(x)=1/\kappa$ for all $x\ge\varpi\kappa$ and otherwise
%\[\md\kappa(x) = 
%\tfrac{1}{2}\cdot x^2+\sum_{n>1}\tfrac{(-\kappa)^{n-1}}{(2\cdot n)!}\cdot x^{2\cdot n}
%=
%\left[\begin{aligned} 
%\tfrac{1}{\kappa}&\cdot\left[1-\cos(x\cdot\sqrt{\kappa} )\right]  && {\text{if }} & \kappa &> 0 \\
% \tfrac{1}{2}&\cdot x^2                          && {\text{if }} & \kappa&=0 \\
%\tfrac{1}{\kappa}&\cdot\left[1-\cosh(x\cdot\sqrt{-\kappa})\right] && {\text{if }} & \kappa &< 0
%\end{aligned}\r..\]
%Set $\sn\kappa=(\md\kappa)'$ and $\cs\kappa=(\md\kappa)''$\index{$\sn\kappa$}\index{$\cs\kappa$}.
%The functions $\sn\kappa, \cs\kappa:\RR_{\ge}\to\RR$ satisfy the equation $y''+\kappa\cdot y=0$,  
%and have the same initial conditions as sine and cosine respectively.



Let $\phi$ be the angle of $\modtrig\kappa\{a,b,c\}$  
opposite to $a$.
In this case we will write \label{page:model-side}\index{$\side\kappa$!$\side\kappa \{{*};{*},{*}\}$}
\[a
=
\side\kappa\{\phi;b,c\}
\ \  \text{or}\ \ 
\phi
=
\tangle\mc\kappa\{a;b,c\}.\]

The functions $\side\kappa$ and $\tangle\mc\kappa$ will be called correspondingly the \emph{model side} and the \emph{model angle}.
Set 
\[
\side\kappa\{\phi;b,-c\}
=\side\kappa\{\phi;-b,c\}
\df\side\kappa\{\pi-\phi;b,c\},\]
this way we define $\side\kappa\{\phi;b,c\}$ for some negative values of $b$ and $c$.


\begin{thm}{Properties of standard functions}\label{md-equalities}

\begin{subthm}{md-diff-eq}
For fixed $a$ and $\phi$, the function 
\[y(t)=\md\kappa\left(\side\kappa\{\phi;a,t\}\right)\]
 satisfies the following differential equation:
\[y''+\kappa\cdot y=1.\]
\end{subthm}

\begin{subthm}{sn-diff-eq}
Let $\alpha\:[a,b]\to\Lob2\kappa$ be a unit-speed geodesic, and $A$ be the image of a complete geodesic.  If $f(t)$ is the distance from $\alpha(t)$ to $A$, the function 
\[y(t)=\sn\kappa (f(t))\]
 satisfies the following differential equation:
\[y''+\kappa\cdot y=0\]
for $y\ne 0$.
\end{subthm}

\begin{subthm}{increase}
For fixed $\kappa$, $b$ and $c$, the function 
\[a\mapsto\tangle\mc\kappa\{a;b,c\}\]
is increasing and defined on a real interval.
Equivalently, the function
\[\phi\mapsto\side\kappa \{\phi;b,c\}\]
is increasing and defined if $b,c<\varpi\kappa$ and $\phi\in[0,\pi]$\footnote{Formally speaking, if $\kappa>0$ and $b+c\ge \varpi\kappa$, it is defined only for $\phi\in[0,\pi)$, but $\side\kappa \{\phi;b,c\}$ can be extended to $[0,\pi]$ as a continuos function.}.
\end{subthm}

\begin{subthm}{k-decrease}
For fixed $\phi,a,b,c$, the functions
\[\kappa\mapsto \tangle\mc\kappa\{a;b,c\}\ \ \text{and}\ \ \kappa\mapsto \side\kappa \{\phi;b,c\}\]
are respectively non-decreasing (in fact, decreasing, if $a\notin\{|b-c|, b+c\}$) and non-increasing (in fact, increasing, if $\phi\notin\{0,\pi\}$).
\end{subthm}

\begin{subthm}{lem:alex-0}(Alexandrov's lemma)
Assume that for real numbers $a$, $b$, $a'$, $b'$, $x$ and $\kappa$, the following two expressions are defined
\begin{enumerate}
\item $\tangle\mc\kappa\{a;b,x\}+\tangle\mc\kappa\{a';b',x\}-\pi$,
\item $\tangle\mc\kappa\{a';b+b',a\}-\tangle\mc\kappa\{x;a,b\}$,
\end{enumerate}
Then they have the same sign.
\end{subthm}
\end{thm}


All the properties except Alexandrov's lemma (\ref{SHORT.lem:alex-0}) can be shown by direct calculation. Alexandrov's lemma is reformulated in \ref{lem:alex} and is proved there.



\parbf{Cosine law.}

The above formulas easily imply  the cosine law in $\Lob2\kappa$, which can be expressed as follows
\[\cos\phi
=\left[\begin{aligned}
&\frac{b^2+c^2-a^2}{2\cdot b\cdot c}
&\text{if}&\ \ \kappa=0
\\
&\frac{\cs\kappa a-\cs\kappa b\cdot\cs\kappa c}{\kappa\cdot\sn\kappa b\cdot\sn\kappa c}
&\text{if}&\ \ \kappa\not=0
&\frac{\cs\kappa a-\cs\kappa b\cdot\cs\kappa c}{\kappa\cdot\sn\kappa b\cdot\sn\kappa c}
&\text{if}&\ \ \kappa\not=0
\end{aligned}\right.\]

However, rather than using these explicit formulas,  we mainly will use
the properties of $\tangle\mc\kappa$ and $\side\kappa$ listed in \ref{md-equalities}.

\section{Hemisphere lemma}\label{curves-in-model}

%By a  \emph{rectifiable curve}  we mean the image of a closed bounded interval under some locally Lipschitz map (equivalently, a \emph{$1$-rectifiable set} on the sense of \ref{sec:rectifiable}).

%\begin{thm}{Lemma}\label{lem:lenght-of-convex-curve}
%Let $\alpha_n$ be a sequence of closed convex curves in $\Lob2{\kappa}$ wich convege to a curve $\alpha$, then $\length\alpha_n\to \length \alpha$.
%\end{thm}

%\parit{Proof.}???\qeds


\begin{thm}{Hemisphere lemma}
\label{lem:hemisphere}
For $\kappa>0$, any closed path of length $<2\cdot \varpi\kappa$ (respectively, $\le2\cdot \varpi\kappa$) in $\Lob2\kappa$ lies in an open (respectively, closed) hemisphere. 
\end{thm}

\parit{Proof.} By rescaling, we may assume that $\kappa=1$ and thus $\varpi\kappa=\pi$ and $\Lob2{\kappa}=\mathbb{S}^2$.
Let $\alpha$ be a closed curve in $\mathbb{S}^2$ of length $2\cdot\ell$.

%???+PIC

Assume $\ell<\pi$.
Let $\check\alpha$ be a subarc of $\alpha$ of length $\ell$, with endpoints $p$ and $q$. 
Since $\dist{p}{q}{}\le\ell<\pi$, there is a unique geodesic $[pq]$ in $\mathbb{S}^2$.  
Let $z$ be the midpoint of  $[pq]$.  
We claim that $\alpha$ lies in the open hemisphere centered at $z$.  
If not, $\alpha$ intersects the boundary  great circle in a point, say $r$.
Without loss of generality we may assume that $r\in\check\alpha$. 
The arc $\check\alpha$ together with its reflection in $z$ form a closed curve of length $2\cdot \ell$ that contains $r$ and its antipodal point $r'$.
Thus 
\[\ell=\length \check\alpha\ge \dist{r}{r'}{}=\pi,\] 
a contradiction.

If $\ell=\pi$, then either $\alpha$ is a local geodesic, and hence a great circle, 
or $\alpha$ may be strictly shortened by substituting a geodesic arc for a subarc of $\alpha$ 
whose endpoints $p^1,p^2$ are arbitrarily close to some point $p$ on  $\alpha$.
In the latter case,  $\alpha$ lies in a closed hemisphere obtained as a limit of closures of open hemispheres  containing the shortened curves as $p^1,p^2$ approach $p$.
\qeds



%\begin{thm}{Lemma}\label{lem:proj-to-conv}Let $K\subset\Lob2\kappa$ be a convex closed set.Then there is a short map $\map\:\Lob2\kappa\to K$ such that $\map(x)=x$ for any $x\in K$. \end{thm}

\begin{thm}{Exercise}\label{exr-crofton}
Build a proof of Hemisphere lemma
\ref{lem:hemisphere} based on Crofton's formula.
\end{thm}







