%%!TEX root = all.tex
%%arrays-up
\chapter{Gradient exponent}\label{chap:gexp}%ready

One of the technical difficulties in Alexandrov's geometry comes from
nonextendability of geodesics. 
In particular, for $\spc{L}\in\CBB{}{}$ and a point $p\in \spc{L}$, 
the exponential map, $\exp_p\:\T_p\to \spc{L}$, if defined the usual way, can
be undefined in an arbitrary small neighborhood of the origin in $\T_p$. 
Here we construct its analog --- the \emph{gradient exponential map},
$\gexp_p\:\T_p\to\spc{L}$, 
which practically solves this problem. 
It has many important properties of the ordinary exponential map
and in certain respects it is ``better'',
even in Riemannian case.


\section{Radial curves: definition}

The radial curves are specially reparametrized gradient curves for distance functions.
This parametrization makes them to behave as unit-speed geodesics in a natural comparison sense.

\begin{thm}{Definition}\label{def:rad-curv}
Assume $\spc{L}\in\CBB{}{}$, 
$\kappa\in\RR$, 
and $p\in \spc{L}$.
A curve 
$$\sigma\:[s_{\min},s_{\max})\to \spc{L}$$  
is called 
\emph{$(p,\kappa)$-radial curve} 
if
$$s_{\min}
\z=
\dist{p}{\sigma(s_{\min})}{}\in(0,\tfrac{\varpi\kappa}2),$$ 
and it satisfies the following differential equation
\[\sigma^+(s)
\z=
\frac{\tg\kappa\dist[{{}}]{p}{\sigma(s)}{}}{\tg\kappa s}
\cdot
\nabla_{\sigma(s)}\dist{p}{}{}.
\eqlbl{eq:rad}\]
for any $s\in[s_{\min},s_{\max})$, here $\tg\kappa x=\frac{\sn\kappa x}{\cs\kappa x}$.

If $x=\sigma(s_{\min})$, we say that $\sigma$ \emph{starts in}  $x$.
\end{thm}

Note that according to the definition $s_{\max}\le\tfrac{\varpi\kappa}2$.

Further you will see that the $(p,\kappa)$-radial curves 
work best for $\CBB{}{\kappa}$-spaces.



\begin{thm}{Definition}\label{def:rad-geod}
Let $\spc{L}\in\CBB{}{}$
and $p\in\spc{L}$.
A unit-speed geodesic  $\gamma\:\II\to \spc{L}$  is called 
\emph{$p$-radial geodesic}\index{radial geodesic} if 
$\dist{p}{\gamma(s)}{}\equiv s$.
\end{thm}

The proof of the following two propositions follows directly from the definitions: 

\begin{thm}{Proposition}\label{prop:rad-geod}
Let $\spc{L}\in\CBB{}{}$
and $p\in\spc{L}$.
Assume $\tfrac{\varpi\kappa}{2}
\ge 
s_{\max}$.
Then any $p$-radial geodesic 
$\gamma\:[s_{\min},s_{\max})
\to 
\spc{L}$ 
is a $(p,\kappa)$-radial curve.
\end{thm}

\begin{thm}{Proposition}\label{prop:dist<s}
Let $\spc{L}\in\CBB{}{}$, 
$p\in\spc{L}$ 
and $\sigma\:[s_{\min},s_{\max})\to \spc{L}$ be a $(p,\kappa)$-radial curve.
Then for any $s\in [s_{\min},s_{\max})$, 
we have $\dist{p}{\sigma(s)}{}\le s$.

Moreover, 
if for some $s_0$ we have $\dist{p}{\sigma(s_0)}{}= s_0$ 
then the restriction $\sigma|[s_{\min},s_0]$ is a $p$-radial geodesic.
\end{thm}

\begin{thm}{Existence and uniqueness}\label{rad-curv-exist}
Let $\spc{L}\in\CBB{}{}$, 
$\kappa\in\RR$, 
$p\in\spc{L}$, 
and $x\in \spc{L}$.
Assume
$0
<
\dist{p}{x}{}
<
\tfrac{\varpi\kappa}2$.
Then there is unique $(p,\kappa)$-radial curve $\sigma\:[\dist{p}{x}{},\tfrac{\varpi\kappa}2)\to \spc{L}$ 
which starts in $x$.
\end{thm}


\parit{Proof; existence.}
Set\index{$\itg\kappa$} 
\[\itg\kappa\:[0,\tfrac{\varpi\kappa}2)\to\RR,
\ \ 
\itg\kappa (t)=\int\limits_0^t\tg\kappa\under t\cdot\d\under t.\]
Clearly $\itg\kappa$ is smooth and increasing.
From \ref{prop:conv-comp} it follows that the composition 
\[f=\itg\kappa\circ\dist{p}{}{}\] 
is semiconcave in $\oBall(p,\tfrac{\varpi\kappa}2)$.

According to \ref{thm:exist-grad-curv}, there is an $f$-gradient curve $\alpha\:[0,t_{\max})\to \spc{L}$ defined on the maximal interval such that $\alpha(0)=x$.

Now consider solution of differential equation $\tau(t)$, $\tau'=(\tg\kappa\tau)^2$ and $\tau(0)=r$. 
Note that $\tau(t)$ is also a gradient curve  for function $\itg\kappa$ defined on $[0,\tfrac{\varpi\kappa}2)$.
Direct calculations show that composition $\alpha\circ\tau^{-1}$ 
forms an $(p,\kappa)$-radial curve.

\parit{Uniqueness.} Assume $\sigma^1,\sigma^2$ be two $(p,\kappa)$-radial curves which starts in $x$.
Then compositions $\sigma^i\circ\tau$ both give $f$-gradient curves.
By Picard's theorem (\ref{thm:picard}), we have
$\sigma^1\circ\tau 
\equiv 
\sigma^2\circ\tau$.
Therefore $\sigma^1(s)=\sigma^2(s)$ 
for any $s\ge r$ such that both sides are defined.
\qeds

\section{Radial comparisons}

In this section we show that radial curves in some comparison sense behave as unit-speed geodesics.

\begin{thm}{Radial monotonicity}\label{rad-mon}
Let $\spc{L}\in\CBB{}{\kappa}$ and
$p, q$ be distinct points in $\spc{L}$.
Assume $\sigma\:  [s_{\min},\tfrac{\varpi\kappa}2)\to \spc{L}$
is a $(p,\kappa)$-radial curve.
Then the function 
\[s\mapsto 
\tangle\mc\kappa\{
\dist{q}{\sigma(s)}{};
\dist{p}{q}{},
s
\}\]
is nonincreasing in all the domain of definition.
\end{thm}

From Radial monotonicity,
by straightforward calculations one gets the following.

\begin{thm}{Corollary}\label{cor:rad-comp}
Let $\kappa\le0$,
$\spc{L}\in\CBB{}{\kappa}$ 
and $p, q\in \spc{L}$.
Assume $\sigma\:[s_{\min},\tfrac{\varpi\kappa}2)\to \spc{L}$ is a $(p,\kappa)$-radial curves.
Then for any $w\ge 1$, 
the function
\[
s\mapsto \tangle\mc\kappa\{\dist{q}{\sigma(s)}{};\dist{p}{q}{},w\cdot s\}
\]
is non-increasing in whole domain of definition.
\end{thm}


\begin{thm}{Radial comparison}\label{rad-comp}
Let $\spc{L}\in\CBB{}{\kappa}$ 
and $p\in \spc{L}$.
Assume $\rho\:  [r_{\min},\tfrac{\varpi\kappa}2)\to \spc{L}$
and    $\sigma\:[s_{\min},\tfrac{\varpi\kappa}2)\to \spc{L}$
are two $(p,\kappa)$-radial curves.
Set
\[\phi_{\min}=\angkk\kappa p{\rho(r_{\min})}{\sigma(s_{\min})}.
\]
Then for any $r\in[r_{\min},\tfrac{\varpi\kappa}2)$ and  $s\in[s_{\min},\tfrac{\varpi\kappa}2)$,
we have
\[
\tangle\mc\kappa\{\dist{\rho(r)}{\sigma(s)}{};r,s\}
\le \phi_{\min},
\]
or equivalently,
\[
\dist{\rho(r)}{\sigma(s)}{}
\le \side\kappa\{\phi_{\min};r,s\}.
\]

\end{thm}


We prove both of the theorems simultaneously.
The proof is an application of \ref{lem:grad-lip} plus trigonometric manipulations.
We give a prove first in the simplest case $\kappa=0$
and then the harder case $\kappa\ne 0$.

We proof case $\kappa=0$ separately since it is easier to follow.
The arguments for both cases are nearly the same, 
but the case $\kappa\not=0$ require an extra idea.
In fact once the case $\kappa\not=0$ is proved, 
the case $\kappa=0$ can be obtained by a limit procedure.

\parit{Proof of \ref{rad-mon} and \ref{rad-comp} in case $\kappa=0$.}
Set

\begin{wrapfigure}[5]{r}{60mm}
\begin{lpic}[t(0mm),b(0mm),r(0mm),l(0mm)]{pics/rad-notation(0.16)}
\lbl[tr]{54,5;$p$}
\lbl[rb]{57,159;$\rho(r)$}
\lbl[r]{25,85;$\rho$}
\lbl[bl]{151,140;$\sigma(s)$}
\lbl[rt]{147,65;$\sigma$}
\lbl[br]{110,80;$S$}
\lbl[bl]{56,80;$R$}
\lbl[bl]{120,10;$\spc{L}$}
\lbl[b]{105,155;$\ell$}
\lbl[b]{305,150;$\ell$}
\lbl[rb]{252,80;$r$}
\lbl[br]{308,80;$s$}
\lbl[bl]{258,38;$\phi$}
\lbl[bl]{310,10;$\Lob2\kappa$}
\end{lpic}
\end{wrapfigure}


\begin{align*}
R=R(r)&=\dist{p}{\rho(r)}{},
\\
S=S(s)&=\dist{p}{\sigma(s)}{},
\\
\ell=\ell(r,s)&=\dist{\rho(r)}{\sigma(s)}{},
\\
\phi=\phi(r,s)&=\tangle\mc0\{\ell(r,s);r,s\}.
\end{align*}
It will be sufficient to prove the following two inequalities:
\[\frac{\partial^+\phi}{\partial r}(s_{\min},r)\le 0,\ \ \ \ \ \ \ \ \frac{\partial^+\phi}{\partial s}(s,r_{\min})\le 0\leqno(*)\mc0_\phi\]
\[
s\cdot\frac{\partial^+\phi}{\partial s}
+
r\cdot\frac{\partial^+\phi}{\partial r}\le 0.
\leqno(**)\mc0_\phi
\]

\begin{wrapfigure}{r}{45mm}
\begin{lpic}[t(-5mm),b(0mm),r(0mm),l(0mm)]{pics/r-s(0.18)}
\lbl[lb]{114,16;$r_{\min}$}
\lbl[lb]{14,66; $s_{\min}$}
\lbl[b]{177,179; {\small $(r_0,s_0)$}}
%\lbl[lb]{125,128,45; {\tiny $(r(t),s(t))$}}
\end{lpic}
\end{wrapfigure}
Once they are proved,
the radial monotonicity follows from $(*)\mc0_\phi$.
The radial comparison follows from both $(*)\mc0_\phi$ and $(**)\mc0_\phi$.
Indeed, one can connect $(s_{\min},r_{\min})$ and $(s_0,r_0)$ in $[s_{\min},\infty)\times[r_{\min},\infty)$ 
by a join of coordinate line and a segment defined by $r/s=r_0/s_0$.
According to $(*)\mc0_\phi$ and $(**)\mc0_\phi$, the value of $\phi$ does not increase while pair $(r,s)$ moving along this join.
Thus $\phi(r_0,s_0)\le\phi(r_{\min},s_{\min})=\phi_{\min}$.

It remains to show $(*)\mc0_\phi$ and $(**)\mc0_\phi$. 
First let us rewrite the inequalities $(*)\mc0_\phi$ and $(**)\mc0_\phi$ in an equivalent form:
\[
\frac{\partial^+\ell}{\partial s}(s,r_{\min})
\le 
\cos\tangle\mc0\{r_{\min};s,\ell\},
\ \ 
\frac{\partial^+\ell}{\partial r}(s_{\min},r)
\le 
\cos\tangle\mc0\{s_{\min};r,\ell\},\leqno(*)\mc0_\ell\]

\[
s\cdot\frac{\partial^+\ell}{\partial s}
+
r\cdot\frac{\partial^+\ell}{\partial r}\le 
 s\cdot\cos\tangle\mc0\{r;s,\ell\}
+
r\cdot\cos\tangle\mc0\{s;r,\ell\}=\ell.
\leqno(**)\mc0_\ell
\]

Set 
\[f=\tfrac{1}{2}\cdot\dist[2]{p}{}{}.\leqno(A)\mc0\] 
Clearly $f$ is $1$-concave and
\[\rho^+(r)=\tfrac{1}{r}\cdot\nabla_{\rho(r)} f\ \ \t{and}\ \ \sigma^+(s)=\tfrac{1}{s}\cdot\nabla_{\sigma(s)} f.\leqno(B)\mc0\]
Thus from \ref{lem:grad-lip}, we have
\[\frac{\partial^+\ell}{\partial r}
=
-\tfrac{1}{r}\cdot\<\nabla_{\rho(r)} f,\dir{\rho(r)}{\sigma(s)}\>
\le\frac{{\ell^2}+{R^2}-{S^2}}{2\cdot\ell\cdot r}.\leqno(C)\mc0\]
Since $R(r)\le r$ and $S(s_{\min})=s_{\min}$, we get 
\[
\begin{aligned}
\frac{\partial^+\ell}{\partial r}(r,s_{\min})
&\le
\frac{{\ell^2}+r^2-s_{\min}^2}{2\cdot\ell\cdot r}
=\\
&=
\cos\tangle\mc0\{s_{\min};r,\ell\},
\end{aligned}
\leqno(D)\mc0
\]
which is the first inequality in $(*)\mc0_\ell$.
By switching places of $\rho$ and $\sigma$ we obtain the second inequality in $(*)\mc0_\ell$.
Further, summing together $(C)\mc0$ with its mirror-inequality for $\frac{\partial^+\ell}{\partial s}$, we get
\[r\cdot\frac{\partial^+\ell}{\partial r}
+
s\cdot\frac{\partial^+\ell}{\partial s}\le \frac{{\ell^2}+{R^2}-{S^2}}{2\cdot\ell }+\frac{{\ell^2}+{S^2}-{R^2}}{2\cdot\ell }= \ell\leqno(E)\mc0\]
which is $(**)\mc0_\ell$.
\qeds

\parit{Proof of \ref{rad-mon} and \ref{rad-comp} in case $\kappa\not=0$.} Set as before
\begin{align*}
R=R(r)&=\dist{p}{\rho(r)}{},&\ell&=\ell(r,s)=\dist{\rho(r)}{\sigma(s)}{}
\\
S=S(s)&=\dist{p}{\sigma(s)}{},&\phi&=\phi(r,s)=\tangle\mc\kappa\{\ell(r,s);r,s\}.
\end{align*}
The statement follows from the following three inequalities:
\[
\begin{aligned}
&\frac{\partial^+\phi}{\partial r}(s_{\min},r)\le 0 
&
&\frac{\partial^+\phi}{\partial s}(s,r_{\min})\le 0
\end{aligned}
\leqno(*)\mc\pm_\phi
\]
\[
\sn\kappa s\cdot\cs\kappa S\cdot\frac{\partial^+\phi}{\partial s}
+
\sn\kappa r\cdot\cs\kappa R\cdot\frac{\partial^+\phi}{\partial r}\le 0
\leqno(**)\mc\pm_\phi
\]

Once they are proved,
the radial monotonicity follows from $(*)\mc\pm_\phi$.
The radial comparison follows from both $(*)\mc0_\phi$ and $(**)\mc\pm_\phi$.
Indeed, functions $s\mapsto \sn\kappa s\cdot\cs\kappa S$ and $r\mapsto \sn\kappa r\cdot\cs\kappa R$ are Lipschitz.
Thus there is a solution for differential equation
\[(r',s')=(\sn\kappa s\cdot\cs\kappa S,\sn\kappa r\cdot\cs\kappa R)\] 
with any initial data. $(r_0,s_0)\in[r_{\min},\tfrac{\varpi\kappa}2)\times[s_{\min},\tfrac{\varpi\kappa}2)$.
(Unlike case $\kappa=0$ the solution can not be written explicitly.)
Since $\sn\kappa s\cdot\cs\kappa S$, $\sn\kappa r\cdot\cs\kappa R>0$, this solution $(r(t),s(t))$ must meet one of coordinate rays
$\{r_{\min}\}\times[s_{\min},\tfrac{\varpi\kappa}2)$ or $[r_{\min},\tfrac{\varpi\kappa}2)\times\{s_{\min}\}$.
I.e., one can connect pair $(s_{\min},r_{\min})$ to $(s_0,r_0)$ by a join of coordinate line and the solution $(r(t),s(t))$.
According to $(*)\mc\pm_\phi$ and $(**)\mc\pm_\phi$, the value of $\phi$ does not increase while pair $(r,s)$ moving along this join.
Thus $\phi(r_0,s_0)\le\phi(r_{\min},s_{\min})$.

As before we rewrite the inequalities $(*)\mc\pm_\phi$ and $(**)\mc\pm_\phi$ in terms of $\ell$:
\[
\begin{aligned}
\frac{\partial^+\ell}{\partial s}(s,r_{\min})
&\le 
\cos\tangle\mc\kappa\{r_{\min};s,\ell\},
\\
\frac{\partial^+\ell}{\partial r}(s_{\min},r)
&\le 
\cos\tangle\mc\kappa\{s_{\min};r,\ell\},
\end{aligned}
\leqno(*)\mc\pm_\ell
\]

\[
\begin{aligned}
\sn\kappa s&\cdot\cs\kappa S\cdot\frac{\partial^+\ell}{\partial s}
+
\sn\kappa r\cdot\cs\kappa R\cdot\frac{\partial^+\ell}{\partial r}\le 
\\
&\le\sn\kappa s\cdot\cs\kappa S\cdot\cos\tangle\mc\kappa\{r;s,\ell\}
+
\sn\kappa r\cdot\cs\kappa R\cdot\cos\tangle\mc\kappa\{s;r,\ell\}
\end{aligned}
\leqno(**)\mc\pm_\ell
\]
Further, set
\[f=-\tfrac{1}{\kappa}\cdot\cs\kappa\circ\dist{p}{}{}
=
\md\kappa\circ\dist{p}{}{}-\tfrac{1}{\kappa}.\leqno(A)\mc\pm\]
Clearly $f''+\kappa\cdot  f\le 0$ and
\[
\begin{aligned}
\rho^+(r)&=\frac{1}{\tg\kappa r\cdot\cs\kappa R}\cdot\nabla_{\rho(r)} f
\\
\sigma^+(s)&=\frac{1}{\tg\kappa s\cdot\cs\kappa S}\cdot\nabla_{\sigma(s)} f.
\end{aligned}
\leqno(B)\mc\pm\]
Thus from \ref{lem:grad-lip}, we have
\[\begin{aligned}
\frac{\partial^+\ell}{\partial r}
&=
-\frac{1}{\tg\kappa r\cdot\cs\kappa R}
\cdot
\<\nabla_{\rho(r)} f,\dir{\rho(r)}{\sigma(s)}\>
\le
\\
&\le
\frac
{1}
{\tg\kappa r\cdot\cs\kappa R}
\cdot
\frac
{\cs\kappa S-\cs\kappa R\cdot\cs\kappa\ell}
{\kappa\cdot\sn\kappa\ell}
=
\\
&=
\frac
{\frac{\cs\kappa S}{\cs\kappa R}-\cs\kappa\ell}
{\kappa\cdot\tg\kappa r\cdot\sn\kappa\ell}.
\end{aligned}
\leqno(C)\mc\pm\]
Note that for all $\kappa\not=0$,
the function $x\mapsto\frac{1}{\kappa\cdot\cs\kappa x}$ is increasing.
Thus, since $R(r)\le r$ and $S(s_{\min})=s_{\min}$, we get 
\[\begin{aligned}
\frac{\partial^+\ell}{\partial r}(r,s_{\min})
&\le 
\frac
{\frac{\cs\kappa s_{\min}}{\cs\kappa r}-\cs\kappa\ell}
{\kappa\cdot\tg\kappa r\cdot\sn\kappa\ell}
=
\\
&=
\frac
{{\cs\kappa s_{\min}}-\cs\kappa\ell\cdot\cs\kappa r}
{\kappa\cdot\sn\kappa r\cdot\sn\kappa\ell}=
\\
&=\cos\tangle\mc\kappa\{s_{\min};r,\ell\},
  \end{aligned}\leqno(D)\mc\pm\]
which is the first inequality in $(*)\mc\pm_\ell$ for $\kappa\not=0$.
By switching places of $\rho$ and $\sigma$ we obtain the second inequality in $(*)\mc\pm_\ell$.
Further, summing together $(C)\mc\pm$ with its mirror-inequality for $\frac{\partial^+\ell}{\partial s}$, we get
\[\begin{aligned}
\sn\kappa r\cdot\cs\kappa R\cdot\frac{\partial^+\ell}{\partial r}
&+
\sn\kappa s\cdot\cs\kappa S\cdot\frac{\partial^+\ell}{\partial s}\le
\\
&\le
\frac
{{\cs\kappa S}\cdot\cs\kappa r-\cs\kappa\ell\cdot\cs\kappa R\cdot\cs\kappa r}
{\kappa\cdot\sn\kappa\ell}
+\\
&\ \ \ \ +
\frac
{{\cs\kappa R}\cdot\cs\kappa s-\cs\kappa\ell\cdot\cs\kappa S\cdot\cs\kappa s}
{\kappa\cdot\sn\kappa\ell}=
\\
&=
\sn\kappa r\cdot\cs\kappa R\cdot
\frac
{\cs\kappa s-\cs\kappa\ell\cdot\cs\kappa r}
{\kappa\cdot\sn\kappa r\cdot\sn\kappa\ell}
+\\
&\ \ \ \ +
\sn\kappa s\cdot\cs\kappa S\cdot
\frac
{\cs\kappa r-\cs\kappa\ell\cdot\cs\kappa s}
{\kappa\cdot\sn\kappa s\cdot\sn\kappa\ell}
=
\\
&=\sn\kappa r\cdot\cs\kappa R\cdot\cos\tangle\mc\kappa\{r;s,\ell\}
+\\
&\ \ \ \ +\sn\kappa s\cdot\cs\kappa S\cdot\cos\tangle\mc\kappa\{s;r,\ell\}
\end{aligned}
\leqno(E)\mc\pm\]
which is $(**)\mc\pm_\ell$.\qeds




























\section{Gradient exponent}\label{sec:gexp}

Let $\spc{L}\in\CBB{}{\kappa}$, 
$p\in \spc{L}$ 
and $\xi\in \Sigma_p$.
Consider a sequence of points $x_n\in \spc{L}$ so that $\dir p{x_n}\to \xi$.
Set $r_n=\dist{p}{x_n}{}$;
denote by $\sigma_n\:[r_n,\tfrac{\varpi\kappa}2)\to \spc{L}$ be the $(p,\kappa)$-radial curve which starts in $x_n$.

Directly from radial comparison (\ref{rad-comp}), 
we have that $\sigma_n\:[r_n,\tfrac{\varpi\kappa}2)\to \spc{L}$ 
converge to a curve $\sigma_\xi\:(0,\tfrac{\varpi\kappa}2)\to \spc{L}$ 
and this limit is independent from the choice of the sequence $x_n$.
Set $\sigma_\xi(0)=p$ and if $\kappa>0$ define \[\sigma_\xi(\tfrac{\varpi\kappa}2)
=
\lim_{t\to\frac{\varpi\kappa}2}\sigma_\xi(t).\]
The obtained curve $\sigma_\xi$ will be called \emph{$(p,\kappa)$-radial curve in direction $\xi$}.

Let us define \emph{gradient exponent} as  
\[
\gexp\mc\kappa_p\:r\cdot\xi\mapsto\sigma_\xi(r)\:\cBall[\0,\tfrac{\varpi\kappa}2]\subset \T_p\to \spc{L}.
\]

Here are properties of radial curves reformulated in terms of gradient exponent:

\begin{thm}{Theorem}\label{thm:prop-gexp}
Let $\spc{L}\in\CBB{}{\kappa}$. 
Then
\begin{subthm}{}
 If $p,q\in \spc{L}$ be points such that $\dist{p}{q}{}\le\tfrac{\varpi\kappa}2$ then for any geodesic $[pq]$ in $\spc{L}$, we have
\[\gexp\mc\kappa_p(\ddir p q)=q\] 
\end{subthm}

\begin{subthm}{thm:prop-gexp:short} 
For any $v,w\in \cBall[\0,\tfrac{\varpi\kappa}2]\subset \T_p$,
\[\dist{\gexp\mc\kappa_p v}{\gexp\mc\kappa_p w}{}
\le
\side\kappa\hinge{\0}v w.\]
In other words, if we denote by $\mathcal{T}_{p}\mc\kappa$ the set $\cBall[\0,\tfrac{\varpi\kappa}2]\subset \T_p$ 
equipped with metric $\dist{v}{w}{\mathcal{T}\mc\kappa_{p}}=\side\kappa\hinge{\0}v w$, 
then 
\[\gexp\mc\kappa_p:\mathcal{T}\mc\kappa_{p}\to \spc{L}\] 
is a short map.
\end{subthm}

\begin{subthm}{gexp-mono} 
Let $\spc{L}\in\CBB{}{\kappa}$,
$p, q\in \spc{L}$ 
and $\dist{p}{q}{}\le \tfrac{\varpi\kappa}2$.
If $v\in\T_p$, $|v|\le 1$ and 
\[\sigma(t)=\gexp\mc\kappa_p(t\cdot v)\] then
the function
\[
s
\mapsto 
\tangle\mc\kappa(\sigma|_0^s,q)
\df
\tangle\mc\kappa\{\dist{q}{\sigma(s)}{};\dist{q}{\sigma(0)},s\}
\]
is non-increasing in whole domain of definition.
\end{subthm}
\end{thm}

\parit{Proof.}
Follows directly from construction of $\gexp\mc\kappa_p$ and the radial comparison (\ref{rad-comp}).
\qeds

Applying the theorem above together with \ref{LinDim+-f},
we obtain the following.

\begin{thm}{Corollary}\label{cor:short-map-to-ball}
Let $\spc{L}\in\CBB{m}{\kappa}$, $p\in\spc{L}$ and $0<R\le\tfrac{\varpi\kappa}2$.
Then there is a short map 
$f\:\cBall[R]_{\Lob{m}{\kappa}}\to \spc{L}$
such that $\Im f= \cBall[p,R]\subset \spc{L}$.
\end{thm}


\section{A generalization}

Here we generalize the above constructions of radial curves and gradient exponent.
Roughly, we show that one can use a distance function 
$\dist{A}{}{}$ to any closed set $A$ instead of the distant function to one point.
We only give the corresponding definitions and state the results,
the proofs are straightforward generalization of corresponding one-point-set version. 

First we give a more general form of the definition of radial curves (\ref{def:rad-curv}) and the definition of radial geodesic (\ref{def:rad-geod}):

\begin{thm}{Definition}
Assume $\spc{L}\in\CBB{}{}$, 
$\kappa\in\RR$, 
and $A\subset \spc{L}$ be a closed subset.
A curve $\sigma\:[s_{\min},s_{\max})\to \spc{L}$  is called 
\emph{$(A,\kappa)$-radial curve} 
if
$s_{\min}
\z=
\dist{A}{\sigma(s_{\min})}{}\in(0,\tfrac{\varpi\kappa}2)$, 
and it satisfies the following differential equation
\[\sigma^+(s)
\z=
\frac{\tg\kappa\dist[{{}}]{p}{\sigma(s)}{}}{\tg\kappa s}
\cdot
\nabla_{\sigma(s)}\dist{A}{}{}.\]
for any $s\in[s_{\min},s_{\max})$, here $\tg\kappa x=\frac{\sn\kappa x}{\cs\kappa x}$.

If $x=\sigma(s_{\min})$, we say that $\sigma$ \emph{starts in}  $x$.
\end{thm}

\begin{thm}{Definition}
Let $\spc{L}\in\CBB{}{}$
and $A\subset \spc{L}$ be a closed subset.
A unit-speed geodesic  $\gamma\:\II\to \spc{L}$  is called 
\emph{$A$-radial geodesic}\index{radial geodesic} if 
$\dist{A}{\gamma(s)}{}\equiv s$.
\end{thm}

The following propositions are analogous to the propositions \ref{prop:rad-geod} and \ref{prop:dist<s}.
Their proofs follow directly from the definitions: 

\begin{thm}{Proposition}
Let $\spc{L}\in\CBB{}{}$,
$A\subset\spc{L}$ be a closed subset.
Assume that 
$\tfrac{\varpi\kappa}{2}
\ge 
s_{\max}$.
Then any $\dist{A}{}{}$-radial geodesic 
$\gamma\:[s_{\min},s_{\max})
\to 
\spc{L}$ 
is an $(A,\kappa)$-radial curve.
\end{thm}

\begin{thm}{Proposition}
Let $\spc{L}\in\CBB{}{}$,
$A\subset\spc{L}$ be a closed subset 
and $\sigma\:[s_{\min},s_{\max})\to \spc{L}$ be a $(A,\kappa)$-radial curve.
Then for any $s\in [s_{\min},s_{\max})$, 
we have $\dist{A}{\sigma(s)}{}\le s$.

Moreover, if for some $s_0$ we have $\dist{A}{\sigma(s_0)}{}= s_0$ 
then the restriction $\sigma|_{[s_{\min},s_0]}$ is a $A$-radial geodesic.
\end{thm}

Here is the corresponding generalization of existence and uniqueness 
for $(A,\kappa)$-radial curves;
it can be proved the same way as \ref{rad-curv-exist}

\begin{thm}{Existence and uniqueness}
Let $\spc{L}\in\CBB{}{}$, 
$\kappa\in\RR$, 
$A\subset\spc{L}$ be a closed subset, 
and $x\in \spc{L}$.
Assume
$0
<
\dist{A}{x}{}
<
\tfrac{\varpi\kappa}2$.
Then there is unique $(A,\kappa)$-radial curve $\sigma\:[\dist{A}{x}{},\tfrac{\varpi\kappa}2)\to \spc{L}$ 
which starts in $x$.
\end{thm}

Next we formulate radial monotonicity and radial comparison for $(A,\kappa)$-radial curves.
The proof of these two statements are almost exactly the same as the proof of \ref{rad-mon} and \ref{rad-comp}.

\begin{thm}{Radial monotonicity}\label{gen-rad-mon}
Let $\spc{L}\in\CBB{}{\kappa}$,
$A\subset \spc{L}$ be a closed subset
and $q\in\spc{L}\backslash A$.
Assume $\sigma\:  [s_{\min},\tfrac{\varpi\kappa}2)\to \spc{L}$
is an $(A,\kappa)$-radial curve.
Then the function 
\[s\mapsto 
\tangle\mc\kappa\{
\dist{q}{\sigma(s)}{};
\dist{A}{q}{},
s
\}\]
is nonincreasing in all the domain of definition.
\end{thm}

To formulate generalized radial comparison,
we need to introduce a short cut notation.
Given a set $A$ and two points $x$ and $y$ in a metric space define
\[
\angkk\kappa A{x}{y}
\df
\tangle\mc\kappa\{
\dist{x}{y}{};
\dist{A}{x}{},
\dist{A}{y}{}
\}
\]
Note that distances $\dist{x}{y}{}$, 
$\dist{A}{x}{}$ and 
$\dist{A}{y}{}$ might not satisfy the triangle inequality.
Therefore the model angle 
$\angkk\kappa A{x}{y}$ might be undefined even for $\kappa\le0$.

\begin{thm}{Radial comparison}\label{gen-rad-comp}
Let $\spc{L}\in\CBB{}{\kappa}$ 
and $A\subset \spc{L}$ be a closed set.
Assume $\rho\:  [r_{\min},\tfrac{\varpi\kappa}2)\to \spc{L}$
and    $\sigma\:[s_{\min},\tfrac{\varpi\kappa}2)\to \spc{L}$
are two $\dist{A}{}{}$-radial curves for curvature $\kappa$.
Assume further that 
\[\phi_{\min}
=
\angkk\kappa A{\rho(r_{\min})}{\sigma(s_{\min})}
\]
is defined.
Then for any $r\in[r_{\min},\tfrac{\varpi\kappa}2)$ and  $s\in[s_{\min},\tfrac{\varpi\kappa}2)$,
we have
\[
\dist{\rho(r)}{\sigma(s)}{}
\le \side\kappa\{\phi_{\min};r,s\}.
\]

\end{thm}

Finally, 
assume $p$ be an isolated point of a closed set $A$ in $\spc{L}\in\CBB{}{}{}$.
Applying the same limiting procedure as in Section \ref{sec:gexp},
for any $\xi\in\Sigma_p$
one can construct an $(A,\kappa)$-radial curve $\sigma_\xi$
such that $\sigma_\xi(0)=p$ and $\sigma^+(0)=\xi$.
This way we obtain a map $\gexp\mc\kappa_A\:\T_p\subto\spc{L}$,
$r\cdot\xi\mapsto\sigma_\xi(r)$.
For the constructed map, the following analog of \ref{thm:prop-gexp} holds;
the proof is straightforward.

\begin{thm}{Theorem}
Let $\spc{L}\in\CBB{}{\kappa}$ and $A\subset\spc{L}$ is a closed subset with an isolated point $p\in A$.
Then
\begin{subthm}{}
Assume $\dist{A}{q}{}=\dist{p}{q}{}\le\tfrac{\varpi\kappa}2$ be 
an $A$-radial geodesic then
\[\gexp\mc\kappa_A(\ddir p q)=q\] 
\end{subthm}

\begin{subthm}{} 
For any $v,w\in \cBall[\0,\tfrac{\varpi\kappa}2]\subset \T_p$,
\[\dist{\gexp\mc\kappa_p v}{\gexp\mc\kappa_p w}{}
\le
\side\kappa\hinge{\0}v w.\]
In other words, if we denote by $\mathcal{T}_{p}\mc\kappa$ the set $\cBall[\0,\tfrac{\varpi\kappa}2]\subset \T_p$ 
equipped with metric $\dist{v}{w}{\mathcal{T}\mc\kappa_{p}}=\side\kappa\hinge{\0}v w$, 
then 
\[\gexp\mc\kappa_p:\mathcal{T}\mc\kappa_{p}\to \spc{L}\] 
is a short map.
\end{subthm}

\begin{subthm}{gexp-mono-1} 
Let $\spc{L}\in\CBB{}{\kappa}$,
$p, q\in \spc{L}$ 
and $\dist{p}{q}{}\le \tfrac{\varpi\kappa}2$.
If $v\in\T_p$, $|v|\le 1$ and 
\[\sigma(t)=\gexp\mc\kappa_p(t\cdot v)\] then
the function
$
s\mapsto \tangle\mc\kappa(\sigma|_0^s,q)
$
is non-increasing in whole domain of definition.
\end{subthm}
\end{thm}


\section{Remarks}
Let $\kappa\ge 0$. 
Assume that for some function $\psi$, the curves defined by equation 
\[\sigma^+(s)=\psi(s,\dist{p}{\sigma(s)}{})\cdot\nabla_{\sigma(s)}\dist{p}{}{}\]
satisfy radial comparison \ref{rad-comp}.
Then in fact $\sigma(s)$ are radial curves; 
i.e. 
\[\psi(s,\dist{p}{\sigma(s)}{})= \frac{\tg\Kappa\dist[{{}}]{p}{\sigma(s)}{})}{\tg\Kappa s},\]
see exercise \ref{ex:gexp}.

In case $\kappa<0$, such $\psi$ is not unique.
In particular one can take curves defined by simpler equation
\[\sigma^+(s)
=
\frac{\sn\kappa \dist[{{}}]{p}{\sigma(s)}{}}{\sn\kappa s}\cdot\nabla_{\sigma(s)}\dist{p}{}{}
=
\frac{1}{\sn\kappa s}\cdot\nabla_{\sigma(s)}(\md\kappa\circ\dist{p}{}{}).\]
Among all curves of that type, the radial curves for curvature $\kappa$ 
as defined in \ref{def:rad-curv} maximize the growth of $\dist{p}{\sigma(s)}{}$.











\section{Exercises}
\begin{thm}{Exercise}
\label{ex:gexp} 
Let $\spc{L}\subset\EE^2$ be Euclidean halfplane. 
Clearly $\spc{L}\in\CBB20$.
Given a point $x\in \EE^2$ let us denote by $\proj(x)$ the closest point on $\spc{L}$. 

Apply the Radial comparison (\ref{rad-comp}) to show that for any interior point $p\in \spc{L}$ we have 
\[\gexp_p v=\proj(p+v).\]
\end{thm}

\begin{thm}{Exercise}
Let $\spc{L}\in \CBB{m}{\kappa}$ and $\rad\spc{L}=R$.
Prove that there is a $(\sn\kappa R)$-Lipschitz map $\map\:\SS^{m-1}\to\spc{L}$ such that $\Im\map\supset\partial\spc{L}$.
\end{thm}

\begin{thm}{Exercise}\label{ex:geodesic}
Let $\spc{L}\in \CBB{}{\kappa}$ 
and $x,y,z\in \spc{L}$.
Assume \[\angk\kappa zxy=\pi.\]
Show that there is a geodesic $[xy]$
which pass through $z$.

Compare to Exercise~\ref{ex:flat-in-CBB}
\end{thm}


