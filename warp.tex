%%!TEX root = arXiv.tex
%arXiv

\chapter{Warped products}
The warped product is a construction that produces 
a new metric space, denoted by $\spc{B}\warp f \spc{F}$,
from two metric spaces $\spc{B}$, $\spc{F}$ and a function $f\:\spc{B}\to\RR_{\ge0}$. %The space $\spc{B}$ is called the base and $\spc{F}$, the fiber of the warped product.

Many important constructions such as direct product, cone, spherical suspension and join
can be defined using warped products.

\section{Definitions}\label{sec:wp-def}

First we define $\spc{B}\times_f\spc{F}$  for length spaces $\spc{B}$ and $ \spc{F}$, and locally Lipschitz function $f\:\spc{B}\to [0,\infty)$.   Then we expand the definition to allow arbitrary metric spaces $ \spc{F}$.
For length spaces $\spc{B}$ and $ \spc{F}$,
consider the topological space $(B\times F)/\!\sim$, where the elements of $\{p\}\times F$ are identified if $f( p )=0$. We denote an equivalence class in $(B\times F)/\!\sim$ by any of its representatives $(p,\phi)$.

We refer to $\spc{B}$ and $\spc{F}$ as \emph{base} and \emph{fiber} respectively; 
and to $\spc{B}\times\{\phi_0\}$ for $\phi_0\in  \spc{F}$ and $\{p_0\}\times \spc{F}$ 
for $p_0\in \spc{B}$ 
as \emph{horizontal} and \emph{vertical leaves}.

For any curve $\gamma\:[0,1]\to(B\times F)/\!\sim$, we write $\gamma=(\gamma_\spc{B},\gamma_\spc{F})$ where 
$\gamma_\spc{B}$   is the projection of $\gamma$ to $\spc{B}$,   
and $\gamma_F$ is the projection, where
it is defined, of $\gamma$ to $\spc{F}$.  That is, $\gamma_\spc{B}$ is defined for
all $t\in[0,1]$,  and  $\gamma_\spc{F}$ is defined for all  $t\in[0,1]$ such that $f(t)\ne 0$. We write $\gamma_\spc{F}\: I_+\to \spc{F}$ where $I_+=[0,1]-I_0$,    $I_0=(f\circ\gamma_\spc{B})^{-1}(0)$.
Set  $I_+=\sqcup_{i=1,2,\ldots} \,I_i$, where the $I_i$ are the maximal open subintervals on which $f\circ\gamma_\spc{B}>0$.

We say $\gamma$ is Lipschitz if $\gamma_\spc{B}$ and $\gamma_\spc{F}|I_i$ are Lipschitz and  $\gamma_\spc{F}|I_i$, $i=1,\ldots$, have uniform Lipschitz constant. 
For $\gamma$ Lipschitz, set
\[
\length_f \gamma\: = \int _0^1 \sqrt{
v_\spc{B}^2+ (f\circ\gamma_\spc{B})^2\cdot v_\spc{F}^2}\cdot dt,
\eqlbl{eq:length}
\]
where $\int$ is Lebesgue integral, $v_\spc{B}$ is the speed of $\gamma_\spc{B}$, $v_\spc{F}|I_+$ is  the speed of $\gamma_\spc{F}|I_+$ and $v_\spc{F}|I_0=0$.   
The integrand is defined  almost everywhere on $[0,1]$ by Rademacher's theorem ??, and is bounded. We refer to the integrand as the \emph{$f$-speed} of $\gamma$.

Equivalently,  
\[
\length_f\gamma\  \ =\ \  \sum_i \int _{I_i}\sqrt{
v_\spc{B}^2+ (f\circ\gamma_\spc{B})^2\cdot v_\spc{F}^2}  \ \ +\ \  \length\, (\gamma_\spc{B}|I_0).
\]
%  Here the first term
% is defined 
% independently of enumeration because the summands are positive.
 
 Then the \emph{warped product $\spc{B}\times_f\spc{F}$ with warping function $f$} is the metric  space with distance
 \[
 |xy|_{\spc{B}\times_f\spc{F}}\: =\inf\{\length_f\gamma\: 
\gamma(0)=x, \gamma(1)=y\}
 \]
 where the curves $\gamma\:[0,1]\to \spc{B}\times_f\spc{F}$ are Lipschitz. 

\medskip

\parbf{Remark.} 
Even if $f>0$, mere continuity of $f$ does not suffice to give a warped product with the properties we need. In particular,  the $f$-length formula \ref{eq:length} may not agree with the metric length.

For example, let $\spc{B}$ be the union of intervals $B^i=[0,1/i]$ glued at $0$, $f(0)=1$, $(f|B_i)(1/i)=1/i$, 
$\spc{F}=[0,1]$. Then $f$ is not locally Lipschitz. We may take $f$ to be continuous and positive. 
The curve $\gamma(t)=(0,t)$, $0\le t \le 1$, covers a vertical leaf. Since  any two points of $\gamma$ are joined by Lipschitz curves of length $\le 3/i$ for all $i$,
this leaf is a single point in the metric topology of $\spc{B}\warp{f} \spc{F}$; thus the metric topology is finer than the product topology of $B\times F$. The $f$-length of $\gamma$ is $1$ while the metric length is $0$. 

If  $f$ is  locally Lipschitz, then the $f$-length and metric length agree (see Section \ref{sec:wp-properties}).  If in addition $f$ is positive, the metric distance is locally bounded above and below by that of direct products of $\spc{B}$ with positive rescalings of $\spc{F}$, and so the metric and product topologies of of $\spc{B}\warp{f} \spc{F}$ agree. 

\vspace{3mm}

The  Fiber-independence theorem (\ref{thm:fiber-independence}) implies that if $\iota\:A\to \check A$ is an isometry between two subsets
$A\subset \spc{F}$ and $\check A\subset \check{\spc{F}}$
in length spaces $ \spc{F}$ and $\check{\spc{F}}$, and $\spc{B}$ is a length space, then for any warping function $f\:\spc{B}\to\RR_{\ge0}$,
the map $\iota$ induces an isometry between the sets 
$\spc{B}\warp{f} A \subset \spc{B}\warp{f} \spc{F}$ and $\spc{B}\warp{f}\check{A}\subset \spc{B}\warp{f} \check{\spc{F}}$.

The latter observation allows us to define the warped product $\spc{B}\warp{f} \spc{F}$ where the fiber $\spc{F}$ does not carry its length metric.
Indeed according to ???, any metric space is a subspace $\spc{F}$ of a length space, say $\spc{F}'$.
Therefore we can take the warped product $\spc{B}\warp{f} \spc{F}'$
and identify $\spc{B}\warp{f} \spc{F}$ with its subspace consisting of all pairs $(b,\phi)$ such that $\phi\in \spc{F}$.
According to the Fiber-independence theorem \ref{thm:fiber-independence}, the resulting space does not depend on the choice of $\spc{F}'$.

For any $(p,\phi),(q,\psi) \in \spc{B}\times \spc{F}$, the Fiber-independence theorem gives an explicit formula:
\[
|(p,\phi)(q,\psi)|_{\spc{B}\times_f \spc{F}} =
|(p,0)(q,|\phi \psi|_{\spc{F}}|_{\spc{B}\times_f\R}.
\]


\section{Basic properties}
\label{sec:wp-properties}

\begin{thm}{Proposition}
The warped product $\spc{\spc{B}}\warp{f}\spc{F}$ satisfies:

\begin{subthm}{horiz-leaf-proj}
The projection $(p,\phi_0)\mapsto p$  of any  horizontal leaf $\spc{B}\times\{\phi_0\}$, with its length metric,  is an isometry onto $\spc{B}$.
\end{subthm}

\begin{subthm}{vert-leaf-proj}
If $f(p_0)\ne0$, the projection $(p_0,\phi)\mapsto \phi$ of the vertical leaf $\{p_0\}\times \spc{F}$, with its length metric,  is a homothety onto $\spc{F}$ with multiplier $1/f(p_0)$.
\end{subthm}


\begin{subthm}{horiz-leaf-isometry}
Each horizontal leaf $\spc{B}\times\{\phi_0\}$ is isometrically embedded in $\spc{B}\warp{f}\spc{F}$.
\end{subthm}


\begin{subthm}{Df>0}If  $f$ achieves its minimum at $p_0$, then the vertical leaf $\{p_0\} \times \spc{F}$ is isometrically embedded in $\spc{B}\warp{f}\spc{F}$.
\end{subthm}

\end{thm}


\parit{Proof.} 
Claims  (\ref{SHORT.horiz-leaf-proj}), (\ref{SHORT.vert-leaf-proj}) and (\ref{SHORT.Df>0})  are immediate from the 
$f$-length formula \ref{eq:length}.
Also by \ref{eq:length}, the projection of
$\spc{B}\warp{f}\spc{F}$ onto $\spc{B}\times\{\phi_0\}$ given by  $(p,\phi)\mapsto (p,\phi_0)$   is length-nonincreasing, as is the projection onto $\{p_0\} \times \spc{F}$ given by  $(p,\phi)\mapsto (p_0,\phi)$  if $p_0$ is a local minimum point of $f$.  
Hence (\ref{SHORT.horiz-leaf-isometry}) and (\ref{SHORT.Df>0}).
\qeds

A horizontal leaf need not be convex even if $\spc{B}\warp{f}\spc{F}$ is a geodesic space, since vanishing of the warping function~$f$ allows geodesics to bifurcate into distinct horizontal leaves.
% (see Proposition \ref{prop:f=0} (\ref{f=0-fiber-choice})).  
For instance, suppose $\alpha:[0,1]\to \spc{B}$ is a geodesic of $\spc{B}$ satisfying $\{0,1\}\subset\alpha^{-1}(Z)\neq[0,1]$, where $Z$ is the zero set of $f$. Then for any distinct $\phi_1, \phi_2 \in \spc{F}$, the geodesic $(\alpha,\phi_2)$ of  $\spc{B} \warp{f}\spc{F}$ has its endpoints in $\spc{B}\times\{\phi_1\}$ but does not lie in $\spc{B}\times\{\phi_1\}$.

\begin{thm}{Exercise}\label{ex:chohom-1=warped-product}
Show that the orbits of an isometric cohomogenity-1 group action on a Riemannian manifold
form the vertical fibers of a warped product.
\end{thm}


Distance in a warped product is fiber-independent, in the sense that distances may be calculated by substituting for $\spc{F}$ a different length space:

\begin{thm}{Fiber-independence theorem}\label{thm:fiber-independence}
Consider length spaces $\spc{B}$, $\spc{F}$ and  $\check{\spc{F}}$,  and a locally Lipschitz function
$f:\spc{B}\to\R_{\ge 0}$.  
Assume $p,q\in \spc{B}$, $\phi,\psi\in \spc{F}$ and $\check{\phi},\check{\psi}\in \check{\spc{F}}$:
Then 
\[
\begin{aligned}
\dist{\phi}{\psi}{\spc{F}}
&
\ge\dist{\check{\phi}}{\check{\psi}}{\check{\spc{F}}}
\\
&\Downarrow
\\
\dist{(p,\phi)}{(q,\psi)}{\spc{B}\warp{f}\spc{F}}
&\ge\dist{(p,\check{\phi})}{(q,\check{\psi})}{\spc{B}\warp{f}\check{\spc{F}}}
\end{aligned}
%\eqlbl{eq:dist-fiber-indep}
\]
	
\end{thm}

\parit{Proof.} 
%Fix a path $\gamma=(\gamma_{\spc{B}},\gamma_{\spc{F}})$ 
%from $(p,\phi)$ to $(q,\psi)$.
Let $\gamma$ be a path in $\spc{B}\times \spc{F})/\!\sim$. 

Suppose $f\circ\gamma_\spc{B}>0$.
Since $\dist{\phi}{\psi}{\spc{F}}
\ge\dist{\check{\phi}}{\check{\psi}}{\check{\spc{F}}}$,
there is a Lipschitz path $\gamma_{\check{\spc{F}}}$ 
from $\check\phi$ to $\check\psi$ in $\check{\spc{F}}$ such that
\[(\speed\gamma_{\spc{F}})(t)
\ge
(\speed\gamma_{\check{\spc{F}}})(t)\]
for almost all $t\in[0,1]$.
Consider the path $\check\gamma=(\gamma_{\spc{B}},\gamma_{\check{\spc{F}}})$ from $(p,\check\phi)$ to $(q,\check\psi)$ in $\spc{B}\warp{f}\check{\spc{F}}$.
Clearly
\[\length_f\gamma\ge \length_f\check\gamma.\]

Suppose $(f\circ\gamma_\spc{B})^{-1}(0)\ne\emptyset$.  Then there is a Lipschitz path $\check\gamma$  from $(p,\check\phi)$ to $(q,\check\psi)$ in $\spc{B}\warp{f}\check{\spc{F}}$
such that $\check\gamma_{\spc{F}}$
is constant on each maximal subinterval on which  $f\circ(\gamma)_{\spc{B}}>0$.
By the $f$-length formula, 
\[\length_f\gamma \ge 
\length\gamma_{\spc{B}}=\length_f\check\gamma.
\]

The statement follows.
\qeds

\begin{thm}{Exercise}\label{ex:warp=<}
Let $\spc{B}$ and $\spc{F}$ be two length spaces and $f,g\:\spc{B}\to \RR_\ge$ be two locally Lipschitz nonnegative  functions.
Assume $f(b)\le g(b)$ for any $b\in\spc{B}$.
Show that 
$\spc{B}\warp{f}\spc{F}\le \spc{B}\warp{g}\spc{F}$;
that is, there is a distance noncontracting map $\spc{B}\warp{f}\spc{F}\to \spc{B}\warp{g}\spc{F}$.
\end{thm}

To see that  $\length_f$, as defined by \ref{eq:length} in Section \ref{sec:wp-def}, agrees with the length induced by  $|**|_{\spc{B}\times_f\spc{F}}$, we may proceed as follows. Denote  $\length_f$ by $L$.  Set
\[
L_{\,\Sigma}\,\,=\,\sup_{t_0<\ldots<t_n}\,\sum_{i=1}^n \,d_{\,i}\,,
\]
where the supremum is taken over all partitions  \,$t_0<\ldots<t_n$\, of $[0,1]$, and where, letting $\bar t_i$ be a minimum point of $(f\circ\gamma_B)|[t_{i-1}\,t_i]$\,, 
\[
d_i=
\begin{cases}
\bigl |\gamma(t_i)\,\,\gamma(t_{i-1})\bigr |_{\,B\,\times\,(f\circ\gamma_B)(\bar t_i)\,\cdot F}\,
& \text{if }(f\circ\gamma_B)(\bar t_i)>0,\\
\bigl |\gamma_B(t_i)\,\,\gamma_B(t_{i-1})\bigr |_B
\quad & \text{if } (f\circ\gamma_B)(\bar t_i)=0.
\end{cases}
 \]
Here $B\,\times\,(f\circ\gamma_B)(\bar t_i)\cdot F$\, denotes the Cartesian product  of $B$ with a rescaling of $F$.
 Our choice of  $\bar t_i$ ensures that any sequence of successively refined   sums is nondecreasing.
 
Then  the proof that \,$L=L_{\Sigma}$\,  proceeds  as in the classical case, for length of an absolutely continuous curve in $\R^2$.
(see \cite[p. 245-247]{goffman}.)

Since $f$ is  locally Lipschitz, then outside any neighbhood in $\spc{B}\times \spc{F})/\!\sim$ of a point $x$ we have   $|x*|_{\spc{B}\times_f\spc{F}}$ bounded away from $0$. Therefore $\length_f$ determines a length structure in the sense of \cite{BBI}. Moreover, from the formula for $L_{\Sigma}$, 
%since $f$ is  locally uniformly continuous, 
the length $L=\length_f=L_{\Sigma}$ is a lower semi-continous functional on the space of Lipschitz curves with respect to pointwise convergence.
%, i.e. if \,$\gamma_i\to\gamma$\, then \,$\liminf\,(\length\gamma_i) \ge\length\gamma$. 
By \cite[Theorem 2.4.3]{BBI}, $\length_f$ agrees with the length induced by the  metric of  $\spc{B}\warp{f}\spc{F}$.



\section{Examples}

\parbf{Direct product.}
The simplest example is the \emph{direct product} $\spc{B}\times \spc{F}$, which could be also written as the warped product $\spc{B}\warp1 \spc{F}$.  
For $p,q\in \spc{B}$ and $\phi,\psi\in \spc{F}$, the direct product metric simplifies to
\[
\dist{(p,\phi)}{(q,\psi)}{} =\sqrt{\dist[2]{p}{q}{} + \dist[2]{\phi}{\psi}{}}.
\]
This is taken as the defining formula for the direct product of two arbitrary metric spaces $\spc{B}$ and $\spc{F}$. 

\parbf{Cones.}
The \emph{Euclidean cone} $\Cone\spc{F}$ over a metric space $\spc{F}$
can be defined as the warped product $[0,\infty)\warp{\id} \spc{F}$.
For $s,t\in [0,\infty)$ and $\phi,\psi\in \spc{F}$, 
the cone metric is given by the cosine rule
\[
\dist{(s,\phi)}{(t,\psi)}{} 
=
\sqrt{s^2+t^2-2\cdot s\cdot t\cdot \cos\alpha},
\]
where $\alpha= \max\{\pi, \dist{\phi}{\psi}{}\}$.

Instead of the Euclidean cosine rule, 
we may use the cosine rule in $\Lob2\kappa$:
\[
\dist{(s,\phi)}{(t,\psi)}{} 
=
\side\kappa\{\alpha;s,t\}.
\]
In this way we get  \emph{$\kappa$-cones} over $\spc{F}$, denoted by $\Cone\mc\kappa\spc{F}=[0,\infty)\warp{\sn\kappa} \spc{F}$ for $\kappa\le 0$
and $\Cone\mc\kappa\spc{F}=[0,\varpi\kappa]\warp{\sn\kappa} \spc{F}$ for $\kappa>0$.

The $1$-cone $\Cone\mc1\spc{F}$ is also called the \emph{spherical suspension} over $\spc{F}$,  and is also denoted by $\Susp\spc{F}$.
That is,
\[
\Susp\spc{F}=[0,\pi]\warp{\sin}\spc{F}.
\]

\begin{thm}{Exercise}\label{ex:convexity-in-cone}
Let $\spc{F}$ be a length space and $A\subset  \spc{F}$.
Show that $\Cone\mc\kappa A$ is convex in $\Cone\mc\kappa\spc{F}$ 
if and only if $A$ is $\pi$-convex in $\spc{F}$.
\end{thm}

The elements of the Euclidean cone $\spc{K}=\Cone\spc{F}$
will often be referred to as \index{vector}\emph{vectors}.
The tip of $\spc{K}$ is usually denoted by $\0$ or $\0_{\spc{K}}$.
The \index{absolute value}\emph{absolute value} $|w|$ of the vector $w$ is defined as $\dist{o}{w}{\spc{K}}$;
that is, the distance from $w$ to the origin.
The \index{scalar product}\emph{scalar product} $\<v,w\>$
of two vectors $v,w\in\spc{K}$
is defined by 
\[\<v,w\>
\df
\bigl(\dist[2]{v}{w}{\spc{K}}-|v|^2-|w|^2\bigr)/2.
\]
%??? continue

\parbf{Doubling.}
The doubling space $\spc{W}$ of a metric space $\spc{V}$ on a closed subset $A\subset\spc{V}$
can be also defined as a special type of warped product.
Consider the metric space $\spc{F}$ consisting of two points with distance $\pi$ from each other.
Then $\spc{W}$ is isometric to the warped product 
with base $\spc{V}$, 
fiber $\spc{F}$ and warping function $\distfun{A}{}{}$;
that is
\[\spc{W}\iso\spc{V}\warp{\distfun{A}{}{}}\spc{F}.\]


\section{1-dimensional base}

The following theorems  provide conditions for the spaces and functions in a warped product with 1-dimensional base to have curvature bounds.  These theorems are baby cases of the characterization of curvature bounds in warped products given in \cite{alexander-bishop:warps,alexander-bishop:warp1}.
%%%further section(s) to be added in next arxiv version

\begin{thm}{Theorem}\label{thm:warp-curv-bound:cbb}
\begin{subthm}{thm:warp-curv-bound:cbb:a}
If $\spc{L}$ is a complete length $\Alex{1}$ space and $\diam\spc{L}\le\pi$
then 
\begin{align*}
\Susp\spc{L}&=[0,\pi]\warp{\sin}\spc{L}\quad\text{is  $\Alex1$},
\\
\Cone\spc{L}&=[0,\infty)\warp{\id}\spc{L}\quad\text{is  $\Alex0$},
\\
\Cone\mc{-1}\spc{L}&=[0,\infty)\warp{\sinh}\spc{L}\quad\text{is  $\Alex{-1}$}.
\end{align*}
Moreover the converse also holds in each of the three cases.
\end{subthm}

\begin{subthm}{thm:cbb-product}
If $\spc{L}$ is a complete length $\Alex0$ space
then 
\begin{align*}
\RR\times\spc{L}&\quad\text{is a complete length $\Alex0$ space},
\\
\RR\warp{\exp}\spc{L}&\quad\text{is a complete length $\Alex{-1}$ space.}
\end{align*}
Moreover the converse also holds in each of the two cases.
\end{subthm}

\begin{subthm}{}
If $\spc{L}$ is a complete length $\Alex{-1}$ space,
then $\RR\warp{\cosh}\spc{L}$ is a complete length $\Alex{-1}$ space.
Moreover the converse also holds.
\end{subthm}
\end{thm}

%%%%DOWN

\begin{thm}{Theorem}\label{thm:warp-curv-bound:cat}
Let $\spc{L}$ be a metric space.
\begin{subthm}{thm:warp-curv-bound:cbb:S}
If $\spc{L}$ is $\CAT{1}$
then 
\begin{align*}
\Susp\spc{L}&=[0,\pi]\warp{\sin}\spc{L}\quad\text{is  \CAT{1}},
\\
\Cone\spc{L}&=[0,\infty)\warp{\id}\spc{L}\quad\text{is  \CAT{0}},
\\
\Cone\mc{-1}\spc{L}&=[0,\infty)\warp{\sinh}\spc{L}\quad\text{is  \CAT{-1}}.
\end{align*}
Moreover, the converse also holds in each of the three cases.
\end{subthm}

\begin{subthm}{thm:warp-curv-bound:cbb:E}
If $\spc{L}$ is a complete length $\CAT0$ space
then 
$\RR\times\spc{L}$ is $\CAT0$ 
and 
$\RR\warp{\exp}\spc{L}$ is $\CAT{-1}$.
Moreover the converse also holds in each of the two cases.
\end{subthm}

\begin{subthm}{thm:warp-curv-bound:cbb:H}
If $\spc{L}$ is $\CAT{-1}$
then 
$\RR\warp{\cosh}\spc{L}$
is $\CAT{-1}$.
Moreover the converse also holds.
\end{subthm}
\end{thm}

%%%%UP

In the proof of the above two theorems %%%ONE THEOREM
we will use the following proposition.

\begin{thm}{Proposition}\label{prop:warp-examples}

\begin{subthm}{prop:warp-examples:S}
\begin{align*}
\Susp\mathbb S^{m-1}&=[0,\pi]\warp{\sin}\mathbb S^{m-1}\iso\SS^m,
\\
\Cone\mathbb S^{m-1}&=[0,\infty)\warp{\id}\mathbb S^{m-1}\iso\EE^m,
\\
\Cone\mc{-1}\mathbb S^{m-1}&=[0,\infty)\warp{\sinh}\mathbb S^{m-1}\iso\Lob{m}{-1}.
\end{align*}
\end{subthm}



\begin{subthm}{}
\begin{align*}
\RR\times&\EE^{m-1}\iso\EE^{m},
\\
\RR\warp{\exp}\EE^{m-1}&\iso\Lob{m}{-1}.
\end{align*}
\end{subthm}

\begin{subthm}{}
\[\RR\warp{\cosh}\Lob{m-1}{-1}\iso\Lob{m}{-1}.\]
\end{subthm}

\end{thm}

The proof is left to the reader.

\parit{Proof of \ref{thm:warp-curv-bound:cbb}.}
Let us prove the last statement in (\ref{SHORT.thm:warp-curv-bound:cbb:S}); the remaining statements are similar.
Each proof is based on the Fiber-independence theorem~\ref{thm:fiber-independence} 
and 
the corresponding statement in Proposition~\ref{prop:warp-examples}.


Choose an arbitrary quadruple of points 
\[(s,\phi),(t^1,\phi^1),(t^2,\phi^2),(t^3,\phi^3)\in[0,\infty)\warp{\sinh} \spc{L}.\]
Since $\diam\spc{L}\le\pi$,
{(1+\textit{n})-point comparison (\ref{thm:pos-config})  provides a quadruple of points $\psi,\psi^1,\psi^2,\psi^3\in\SS^3$ such that 
\[\dist{\psi}{\psi^i}{\SS^3}=\dist{\phi}{\phi^i}{\spc{L}}\] 
and
\[\dist{\psi^i}{\psi^j}{\SS^3}\ge\dist{\phi^i}{\phi^j}{\spc{L}}\]
for all $i$ and $j$.

According to Proposition~\ref{prop:warp-examples:S}, 
\[\Cone\mc{-1}\SS^3=[0,\infty)\warp{\sinh}\SS^3\iso\Lob{4}{-1}.\]

Consider the quadruple of points 
\[(s,\psi),(t^1,\psi^1),(t^2,\psi^2),(t^3,\psi^3)\in \Cone\mc{-1}\SS^3=\Lob{4}{-1}.\]

By the Fiber-independence theorem~\ref{thm:fiber-independence},
\[\dist{(s,\psi)}{(t^i,\psi^i)}{[0,\infty)\warp{\sinh}\SS^3}=\dist{(s,\phi)}{(t^i,\phi^i)}{[0,\infty)\warp{\sinh}\spc{L}}\]
and
\[\dist{(t^i,\psi^i)}{(t^j,\psi^j)}{[0,\infty)\warp{\sinh}\SS^3}\ge\dist{(t^i\phi^i)}{(t^j,\phi^j)}{[0,\infty)\warp{\sinh}\spc{L}}\]
for all $i$ and $j$.
Since four points of $\Lob{4}{-1}$ lie in an isometrically embedded copy of $\Lob{3}{-1}$, it remains to apply Exercise \ref{ex:(3+1)-expanding}.\qeds
