%%!TEX root = arXiv.tex
%arXiv

\chapter{Warped products}
The warped product is a construction that produces 
a new metric space, denoted by $\spc{B}\warp f \spc{F}$,
from two metric spaces $\spc{B}$, $\spc{F}$ and a function $f\:\spc{B}\to\RR_{\ge0}$. 

Many important constructions such as direct product, cone, spherical suspension and join
can be defined using warped products.

\section{Definition}
\label{sec:wp-def}
\label{sec:wp-properties}

First we define the warped product for length spaces and then we expand the definition to allow arbitrary metric spaces $\spc{F}$.

Let $\spc{B}$ and $\spc{F}$ be length spaces and $f\:\spc{B}\to [0,\infty)$ be a continuous function.

For any path $\gamma\:[0,1]\to \spc{B}\times \spc{F}$, we write $\gamma=(\gamma_\spc{B},\gamma_\spc{F})$ where 
$\gamma_\spc{B}$ is the projection of $\gamma$ to $\spc{B}$,   
and $\gamma_\spc{F}$ is the projection to $\spc{F}$.
If $\gamma_\spc{B}$ and $\gamma_\spc{F}$ are Lipschitz, set
\[
\length_f \gamma \df \int\limits_0^1 \sqrt{
v_\spc{B}^2+ (f\circ\gamma_\spc{B})^2\cdot v_\spc{F}^2}\cdot dt,
\eqlbl{eq:length}
\]
where $\int$ is Lebesgue integral, $v_\spc{B}$ and $v_\spc{F}$ are the speeds of $\gamma_\spc{B}$ and $\gamma_\spc{F}$ respectively. 

Note that $\length_f \gamma\ge \length \gamma_\spc{B}$.
The integral in \ref{eq:length} can be broken into the sum of two parts: one for the restriction of $\length_f\gamma$ as in \ref{eq:length} to the nonzero set of $f\circ\gamma_\spc{B}$, and the other for the length of $\gamma_\spc{B}$ restricted to the zero set.

Consider the pseudometric on $\spc{B}\warp f\spc{F}$ defined by
 \[
 \dist{x}{y}{}
 \df 
 \inf\set{\length_f\gamma}{\gamma(0)=x, \gamma(1)=y}
 \]
where the exact lower bound is taken for all Lipschitz paths $\gamma\:[0,1]\to \spc{B}\times\spc{F}$. 
The corresponding metric space is called the \emph{warped product with base $\spc{B}$, fiber $\spc{F}$ and warping function $f$}; it will be denoted by $\spc{B}\warp f\spc{F}$.

The points in $\spc{B}\warp f\spc{F}$ can be described by corresponding pairs $(p,\phi)\in\spc{B}\times\spc{F}$; note that if $f(p)=0$ for some $p\in \spc{B}$, then $(p,\phi)\z=(p,\psi)$ for any $\phi,\psi\in \spc{F}$.

Note we are not claiming that every Lipschitz curve in $\spc{B}\warp f \spc{F}$ may be reparametrized as the image of a Lipschitz curve in $\spc{B}\times \spc{F}$; in fact it is not true.

\begin{thm}{Proposition}
The warped product $\spc{\spc{B}}\warp{f}\spc{F}$ satisfies:

\begin{subthm}{horiz-leaf-proj}
The projection $(p,\phi_0)\mapsto p$ is a submetry which when restricted to any horizontal leaf $\spc{B}\times\{\phi_0\}$
is an isometry to $\spc{B}$.
\end{subthm}

\begin{subthm}{vert-leaf-proj}
If $f(p_0)\ne0$, the projection $(p_0,\phi)\mapsto \phi$ of the vertical leaf $\{p_0\}\times \spc{F}$, with its length metric,  is a homothety onto $\spc{F}$ with multiplier $\tfrac1{f(p_0)}$.
\end{subthm}

\begin{subthm}{Df>0}If  $f$ achieves its (local) minimum at $p_0$, then the vertical leaf $\{p_0\} \times \spc{F}$, with its length metric, is (locally) isometrically embedded in $\spc{B}\warp{f}\spc{F}$.
\end{subthm}

\end{thm}


\parit{Proof.} 
Claim  (\ref{SHORT.vert-leaf-proj}) follows from the $f$-length formula \ref{eq:length}.

Also by \ref{eq:length}, the projection of
$\spc{B}\warp{f}\spc{F}$ onto $\spc{B}\times\{\phi_0\}$ given by  $(p,\phi)\mapsto (p,\phi_0)$ is length-nonincreasing; hence (\ref{SHORT.horiz-leaf-proj}).

The projection $(p,\phi)\mapsto (p_0,\phi)$ of a neighborhood of the vertical leaf $\{p_0\} \times \spc{F}$ to $\{p_0\} \times \spc{F}$ is length-nonincreasing if $p_0$ is a local minimum point of $f$. 
If $p_0$ is a global minimum point of $f$, then the same holds for the projection of whole space.
Hence (\ref{SHORT.Df>0}).
\qeds

Note that any horizontal leaf is weakly convex, but does not have to be convex even if $\spc{B}\warp{f}\spc{F}$ is a geodesic space, since vanishing of the warping function~$f$ allows geodesics to bifurcate into distinct horizontal leaves.
For instance, if there is a geodesic with the ends in the zero set 
\[Z=\set{(p,\phi)\in\spc{B} \warp{f}\spc{F})}{f(p)=0},\] 
then there is a geodesic with the same ends in each horizontal leaf.

\begin{thm}{Proposition}
Suppose $\spc{B}$ and $\spc{F}$ are length spaces and $f\:\spc{B}\to [0,\infty)$ is a continuous function.
Then the warped product $\spc{B}\warp f\spc{F}$ is a length space.
\end{thm}

\parit{Proof.}
It is sufficient to show that for any $\alpha\:[0,1]\to \spc{B}\warp f\spc{F}$ there is a path 
$\beta\:[0,1]\to \spc{B}\times\spc{F}$ such that 
\[\length \alpha\ge \length_f\beta.\]

If $f\circ\alpha_{\spc{B}}(t)>0$ for any $t$, then the vertical projection $\alpha_{\spc{F}}$ is defined.
In this case, let $\beta(t)=(\alpha_{\spc{B}}(t),\alpha_{\spc{F}}(t))\in \spc{B}\times\spc{F}$;
clearly 
\[\length \alpha= \length_f\beta.\]

If $f\circ\alpha_{\spc{B}}(t_0) = 0$ for some $t_0$, let $\beta$ be the concatenation of three curves in $\spc{B}\times{\spc{F}}$;
namely: 
(1) the horizontal curve $(\alpha_\spc{B}(t),\phi)$ for $t\le t_0$,
(2) a vertical path in form $(s,\phi)$ to $(s,\psi)$
and 
(3) the horizontal curve $(\alpha_\spc{B}(t),\psi)$ for $t\ge t_0$.
By \ref{eq:length}, the $f$-length of the middle path in the concatenation is vanishing;
therefore the $f$-length of $\alpha$ is can not be smaller than length of $\alpha_{\spc{B}}$;
that is,
\[\length_f\alpha \ge 
\length\alpha_{\spc{B}}=\length_f\beta.
\]
The statement follows.
\qeds

Distance in a warped product is fiber-independent, in the sense that distances may be calculated by substituting for $\spc{F}$ a different length space:

\begin{thm}{Fiber-independence theorem}\label{thm:fiber-independence}
Consider length spaces $\spc{B}$, $\spc{F}$ and $\check{\spc{F}}$, and a locally Lipschitz function
$f\:\spc{B}\to\R_{\ge 0}$.  
Assume $p,q\in \spc{B}$, $\phi,\psi\in \spc{F}$ and $\check{\phi},\check{\psi}\in \check{\spc{F}}$:
Then 
\[
\begin{aligned}
\dist{\phi}{\psi}{\spc{F}}
&
\ge\dist{\check{\phi}}{\check{\psi}}{\check{\spc{F}}}
\\
&\Downarrow
\\
\dist{(p,\phi)}{(q,\psi)}{\spc{B}\warp{f}\spc{F}}
&\ge\dist{(p,\check{\phi})}{(q,\check{\psi})}{\spc{B}\warp{f}\check{\spc{F}}}
\end{aligned}
%\eqlbl{eq:dist-fiber-indep}
\]
In particular,
\[
\dist{(p,\phi)}{(q,\psi)}{\spc{B}\warp f \spc{F}} =
\dist{(p,0)}{(q,\ell)}{\spc{B}\warp f\R},
\]
where $\ell=\dist{\phi}{\psi}{\spc{F}}$.
\end{thm}

\parit{Proof.} 
Let $\gamma$ be a path in $(\spc{B}\times \spc{F})$. 

Since $\dist{\phi}{\psi}{\spc{F}}
\ge\dist{\check{\phi}}{\check{\psi}}{\check{\spc{F}}}$,
there is a Lipschitz path $\gamma_{\check{\spc{F}}}$ 
from $\check\phi$ to $\check\psi$ in $\check{\spc{F}}$ such that
\[(\speed\gamma_{\spc{F}})(t)
\ge
(\speed\gamma_{\check{\spc{F}}})(t)\]
for almost all $t\in[0,1]$.
Consider the path $\check\gamma=(\gamma_{\spc{B}},\gamma_{\check{\spc{F}}})$ from $(p,\check\phi)$ to $(q,\check\psi)$ in $\spc{B}\warp{f}\check{\spc{F}}$.
Clearly
\[\length_f\gamma\ge \length_f\check\gamma.\]
\qedsf

\begin{thm}{Exercise}\label{ex:warp=<}
Let $\spc{B}$ and $\spc{F}$ be two length spaces and $f,g\:\spc{B}\to \RR_\ge$ be two locally Lipschitz nonnegative  functions.
Assume $f(b)\le g(b)$ for any $b\in\spc{B}$.
Show that 
$\spc{B}\warp{f}\spc{F}\le \spc{B}\warp{g}\spc{F}$;
that is, there is a distance noncontracting map $\spc{B}\warp{f}\spc{F}\to \spc{B}\warp{g}\spc{F}$.
\end{thm}

\section{Extended definitions}

The fiber-independence theorem implies that 
\[
\dist{(p,\phi)}{(q,\psi)}{\spc{B}\warp f \spc{F}} =
\dist{(p,0)}{(q,\dist{\phi}{\psi}{\spc{F}})}{\spc{B}\warp f\R}
\]
for any $(p,\phi),(q,\psi) \in \spc{B}\times \spc{F}$.
In particular, if $\iota\:A\to \check A$ is an isometry between two subsets
$A\subset \spc{F}$ and $\check A\subset \check{\spc{F}}$
in length spaces $\spc{F}$ and $\check{\spc{F}}$, and $\spc{B}$ is a length space, then for any warping function $f\:\spc{B}\to\RR_{\ge0}$,
the map $\iota$ induces an isometry between the sets 
$\spc{B}\warp{f} A \subset \spc{B}\warp{f} \spc{F}$ and $\spc{B}\warp{f}\check{A}\subset \spc{B}\warp{f} \check{\spc{F}}$.

The latter observation allows us to define the warped product $\spc{B}\z{\warp{f}} \spc{F}$ where the fiber $\spc{F}$ does not carry its length metric.
Indeed we can use Kuratowsky embedding to realize $\spc{F}$ as a subspace in a length space, say $\spc{F}'$.
Therefore we can take the warped product $\spc{B}\warp{f} \spc{F}'$
and identify $\spc{B}\warp{f} \spc{F}$ with its subspace consisting of all pairs $(b,\phi)$ such that $\phi\in \spc{F}$.
According to the Fiber-independence theorem \ref{thm:fiber-independence}, the resulting space does not depend on the choice of $\spc{F}'$.


\section{Examples}

\parbf{Direct product.}
The simplest example is the \emph{direct product} $\spc{B}\times \spc{F}$, which could be also written as the warped product $\spc{B}\warp1 \spc{F}$.  
For $p,q\in \spc{B}$ and $\phi,\psi\in \spc{F}$, the direct product metric simplifies to
\[
\dist{(p,\phi)}{(q,\psi)}{} =\sqrt{\dist[2]{p}{q}{} + \dist[2]{\phi}{\psi}{}}.
\]
This is taken as the defining formula for the direct product of two arbitrary metric spaces $\spc{B}$ and $\spc{F}$. 

\parbf{Cones.}
The \emph{Euclidean cone} $\Cone\spc{F}$ over a metric space $\spc{F}$
can be defined as the warped product $[0,\infty)\warp{\id} \spc{F}$.
For $s,t\in [0,\infty)$ and $\phi,\psi\in \spc{F}$, 
the cone metric is given by the cosine rule
\[
\dist{(s,\phi)}{(t,\psi)}{} 
=
\sqrt{s^2+t^2-2\cdot s\cdot t\cdot \cos\alpha},
\]
where $\alpha= \max\{\pi, \dist{\phi}{\psi}{}\}$.

Instead of the Euclidean cosine rule, 
we may use the cosine rule in $\Lob2\kappa$:
\[
\dist{(s,\phi)}{(t,\psi)}{} 
=
\side\kappa\{\alpha;s,t\}.
\]
In this way we get  \emph{$\kappa$-cones} over $\spc{F}$, denoted by $\Cone\mc\kappa\spc{F}=[0,\infty)\warp{\sn\kappa} \spc{F}$ for $\kappa\le 0$
and $\Cone\mc\kappa\spc{F}=[0,\varpi\kappa]\warp{\sn\kappa} \spc{F}$ for $\kappa>0$.

The $1$-cone $\Cone\mc1\spc{F}$ is also called the \emph{spherical suspension} over $\spc{F}$,  and is also denoted by $\Susp\spc{F}$.
That is,
\[
\Susp\spc{F}=[0,\pi]\warp{\sin}\spc{F}.
\]

\begin{thm}{Exercise}\label{ex:convexity-in-cone}
Let $\spc{F}$ be a length space and $A\subset  \spc{F}$.
Show that $\Cone\mc\kappa A$ is convex in $\Cone\mc\kappa\spc{F}$ 
if and only if $A$ is $\pi$-convex in $\spc{F}$.
\end{thm}

The elements of the Euclidean cone $\spc{K}=\Cone\spc{F}$
will often be referred to as \index{vector}\emph{vectors}.
The tip of $\spc{K}$ is usually denoted by $\0$ or $\0_{\spc{K}}$.
The \index{absolute value}\emph{absolute value} $|w|$ of the vector $w$ is defined as $\dist{\0}{w}{\spc{K}}$;
that is, the distance from $w$ to the tip.
The \index{scalar product}\emph{scalar product} $\<v,w\>$
of two vectors $v,w\in\spc{K}$
is defined by 
\[\<v,w\>
\df
\bigl(\dist[2]{v}{w}{\spc{K}}-|v|^2-|w|^2\bigr)/2.
\]
%??? continue

\parbf{Doubling.}
The doubling space $\spc{W}$ of a metric space $\spc{V}$ on a closed subset $A\subset\spc{V}$
can be also defined as a special type of warped product.
Consider the fiber $\mathbb{S}^0$ consisting of two points with distance $2$ from each other.
Then $\spc{W}$ is isometric to the warped product 
with base $\spc{V}$, 
fiber $\mathbb{S}^0$ and warping function $\distfun{A}{}{}$;
that is,
\[\spc{W}\iso\spc{V}\warp{\distfun{A}{}{}}\mathbb{S}^0.\]




\section{1-dimensional base}

The following theorems  provide conditions for the spaces and functions in a warped product with 1-dimensional base to have curvature bounds.  These theorems are baby cases of the characterization of curvature bounds in warped products given in \cite{alexander-bishop:warps,alexander-bishop:warp1}.
%%%further section(s) to be added in next arxiv version

\begin{thm}{Theorem}\label{thm:warp-curv-bound:cbb}
\begin{subthm}{thm:warp-curv-bound:cbb:a}
If $\spc{L}$ is a complete length $\Alex{1}$ space and $\diam\spc{L}\le\pi$
then 
\begin{align*}
\Susp\spc{L}&=[0,\pi]\warp{\sin}\spc{L}\quad\text{is  $\Alex1$},
\\
\Cone\spc{L}&=[0,\infty)\warp{\id}\spc{L}\quad\text{is  $\Alex0$},
\\
\Cone\mc{-1}\spc{L}&=[0,\infty)\warp{\sinh}\spc{L}\quad\text{is  $\Alex{-1}$}.
\end{align*}
Moreover the converse also holds in each of the three cases.
\end{subthm}

\begin{subthm}{thm:cbb-product}
If $\spc{L}$ is a complete length $\Alex0$ space
then 
\begin{align*}
\RR\times\spc{L}&\quad\text{is a complete length $\Alex0$ space},
\\
\RR\warp{\exp}\spc{L}&\quad\text{is a complete length $\Alex{-1}$ space.}
\end{align*}
Moreover the converse also holds in each of the two cases.
\end{subthm}

\begin{subthm}{}
If $\spc{L}$ is a complete length $\Alex{-1}$ space,
then $\RR\warp{\cosh}\spc{L}$ is a complete length $\Alex{-1}$ space.
Moreover the converse also holds.
\end{subthm}
\end{thm}

%%%%DOWN

\begin{thm}{Theorem}\label{thm:warp-curv-bound:cat}
Let $\spc{L}$ be a metric space.
\begin{subthm}{thm:warp-curv-bound:cbb:S}
If $\spc{L}$ is $\CAT{1}$
then 
\begin{align*}
\Susp\spc{L}&=[0,\pi]\warp{\sin}\spc{L}\quad\text{is  \CAT{1}},
\\
\Cone\spc{L}&=[0,\infty)\warp{\id}\spc{L}\quad\text{is  \CAT{0}},
\\
\Cone\mc{-1}\spc{L}&=[0,\infty)\warp{\sinh}\spc{L}\quad\text{is  \CAT{-1}}.
\end{align*}
Moreover, the converse also holds in each of the three cases.
\end{subthm}

\begin{subthm}{thm:warp-curv-bound:cbb:E}
If $\spc{L}$ is a complete length $\CAT0$ space
then 
$\RR\times\spc{L}$ is $\CAT0$ 
and 
$\RR\warp{\exp}\spc{L}$ is $\CAT{-1}$.
Moreover the converse also holds in each of the two cases.
\end{subthm}

\begin{subthm}{thm:warp-curv-bound:cbb:H}
If $\spc{L}$ is $\CAT{-1}$
then 
$\RR\warp{\cosh}\spc{L}$
is $\CAT{-1}$.
Moreover the converse also holds.
\end{subthm}
\end{thm}

%%%%UP

In the proof of the above two theorems %%%ONE THEOREM
we will use the following proposition.

\begin{thm}{Proposition}\label{prop:warp-examples}

\begin{subthm}{prop:warp-examples:S}
\begin{align*}
\Susp\mathbb S^{m-1}&=[0,\pi]\warp{\sin}\mathbb S^{m-1}\iso\mathbb{S}^m,
\\
\Cone\mathbb S^{m-1}&=[0,\infty)\warp{\id}\mathbb S^{m-1}\iso\EE^m,
\\
\Cone\mc{-1}\mathbb S^{m-1}&=[0,\infty)\warp{\sinh}\mathbb S^{m-1}\iso\Lob{m}{-1}.
\end{align*}
\end{subthm}



\begin{subthm}{}
\begin{align*}
\RR\times&\EE^{m-1}\iso\EE^{m},
\\
\RR\warp{\exp}\EE^{m-1}&\iso\Lob{m}{-1}.
\end{align*}
\end{subthm}

\begin{subthm}{}
\[\RR\warp{\cosh}\Lob{m-1}{-1}\iso\Lob{m}{-1}.\]
\end{subthm}

\end{thm}

The proof is left to the reader.

\parit{Proof of \ref{thm:warp-curv-bound:cbb}.}
Let us prove the last statement in (\ref{SHORT.thm:warp-curv-bound:cbb:S}); the remaining statements are similar.
Each proof is based on the Fiber-independence theorem~\ref{thm:fiber-independence} 
and 
the corresponding statement in Proposition~\ref{prop:warp-examples}.


Choose an arbitrary quadruple of points 
\[(s,\phi),(t^1,\phi^1),(t^2,\phi^2),(t^3,\phi^3)\in[0,\infty)\warp{\sinh} \spc{L}.\]
Since $\diam\spc{L}\le\pi$,
{(1+\textit{n})-point comparison (\ref{thm:pos-config})  provides a quadruple of points $\psi,\psi^1,\psi^2,\psi^3\in\mathbb{S}^3$ such that 
\[\dist{\psi}{\psi^i}{\mathbb{S}^3}=\dist{\phi}{\phi^i}{\spc{L}}\] 
and
\[\dist{\psi^i}{\psi^j}{\mathbb{S}^3}\ge\dist{\phi^i}{\phi^j}{\spc{L}}\]
for all $i$ and $j$.

According to Proposition~\ref{prop:warp-examples:S}, 
\[\Cone\mc{-1}\mathbb{S}^3=[0,\infty)\warp{\sinh}\mathbb{S}^3\iso\Lob{4}{-1}.\]

Consider the quadruple of points 
\[(s,\psi),(t^1,\psi^1),(t^2,\psi^2),(t^3,\psi^3)\in \Cone\mc{-1}\mathbb{S}^3=\Lob{4}{-1}.\]

By the Fiber-independence theorem~\ref{thm:fiber-independence},
\[\dist{(s,\psi)}{(t^i,\psi^i)}{[0,\infty)\warp{\sinh}\mathbb{S}^3}=\dist{(s,\phi)}{(t^i,\phi^i)}{[0,\infty)\warp{\sinh}\spc{L}}\]
and
\[\dist{(t^i,\psi^i)}{(t^j,\psi^j)}{[0,\infty)\warp{\sinh}\mathbb{S}^3}\ge\dist{(t^i,\phi^i)}{(t^j,\phi^j)}{[0,\infty)\warp{\sinh}\spc{L}}\]
for all $i$ and $j$.
Since four points of $\Lob{4}{-1}$ lie in an isometrically embedded copy of $\Lob{3}{-1}$, it remains to apply Exercise \ref{ex:(3+1)-expanding}.\qeds

\begin{thm}{Exercise}
\index{spherical join}\emph{Spherical join} $\spc{U}\star\spc{V}$ of two metric spaces $\spc{U}$ and $\spc{V}$
is defined as the unit sphere equipped with the angle metric in the product of Euclidean cones $\Cone \spc{U}\times \Cone\spc{V}$.

Assume $\spc{U}$ and $\spc{V}$ are nonempty spaces.
\begin{subthm}{}Show that $\spc{U}\star\spc{V}$ is $\CAT1$ if and only if $\spc{U}$ and $\spc{V}$ are $\CAT1$.
\end{subthm}

\begin{subthm}{}Show that $\spc{U}\star\spc{V}$ is $\Alex1$ if and only if $\spc{U}$ and $\spc{V}$ are $\Alex1$.
\end{subthm}

\end{thm}


\section{General case}

In this section we formulate general results on curvature bounds of warped products 
proved by the first author and Richard Bishop \cite{alexander-bishop:warps}

\begin{thm}{Theorem}\label{thm:warp-CBB}
Let $\spc{B}$ be a complete finite-dimensional $\Alex\kappa$ length space, and the function $f\:\spc{B}\to \RR_\ge$ satisfy $f''+\kappa\cdot f\le 0$.
Denote by $Z\subset \spc{B}$ the zero set of~$f$.
Suppose $\spc{F}$ is a complete finite-dimensional $\Alex{\kappa'}$ space.
Then the warped product $\spc{B}\warp f\spc{F}$ is $\Alex\kappa$ in the following two cases:
\begin{subthm}{thm:warp-CAT:Z=0}
If $Z= \emptyset$ and 
\[\kappa'\ge \kappa\cdot f^2(b)\]
for any $b\in \spc{B}$.
\end{subthm}

\begin{subthm}{thm:warp-CAT:Zne0}
If $Z\ne \emptyset$ and
\[|d_zf|^2\le\kappa'\]
for any $z\in Z$.
\end{subthm}

\end{thm}

\begin{thm}{Theorem}\label{thm:warp-CAT}
Let $\spc{B}$ be a 
complete  
$\CAT\kappa$ length space, and the function $f\:\spc{B}\to \RR_\ge$ satisfy $f''+\kappa\cdot f\ge 0$, 
where $f$ is Lipschitz on bounded sets or $B$ is locally compact.
Denote by $Z\subset \spc{B}$ the zero set of~$f$.
Suppose $\spc{F}$ is a 
complete  $\CAT{\kappa'}$  space.
Then the warped product $\spc{B}\warp f\spc{F}$ is $\CAT\kappa$ in the following two cases:
\begin{subthm}{}
If $Z= \emptyset$ and 
\[\kappa'\le \kappa\cdot f^2(b)\]
for any $b\in \spc{B}$.
\end{subthm}

\begin{subthm}{}
If $Z\ne \emptyset$, 
\[[(d_zf)\dir zb]^2\ge\kappa'\] for any minimizing geodesic $[zb]$ from $Z$ to a point $b\in\spc{B}$ and 
\[\kappa'\le \kappa\cdot f^2(b)\] for any $b\in \spc{B}$ such that $\distfun{X}{b}{}\ge \tfrac{\varpi\kappa}2$.
\end{subthm}

\end{thm}

\parit{The proofs are coming.}
