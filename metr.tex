%%!TEX root =the-metr.tex
%array^
\chapter{Metric spaces}
In this chapter we fix some conventions and notations. We are assuming that the reader is familiar with basic notions in metric geometry.

\section{Metrics and their relatives}
\label{sec:metric spaces}

\parbf{Definitions.}
Let $\II$ be a subinterval of $[0,\infty]$.
A function $\rho$ defined on $\spc{X}\times\spc{X}$ is called an $\II$-valued metric if the following conditions hold:
\begin{itemize}
 \item $\rho(x,x)=0$ for any $x$;
 \item $\rho(x,y)\in \II$ for any pair $x\ne y$;
 \item $\rho(x,y)+\rho(x,z)\ge \rho(y,z)$ for any triple of points $x,y, z$.
\end{itemize}
The value $\rho(x,y)$ is also called the \index{distance}\emph{distance} between $x$ and $y$.


The above definition will be used for four choices of the interval $\II$: $(0,\infty)$, $(0,\infty]$, $[0,\infty)$ and $[0,\infty]$.
Any $\II$-valued metric can be referred to briefly as a metric;
the interval should be apparent from context but by default a metric is $(0,\infty)$-valued. 
If we need to be more specific we may also use the following names:
\begin{itemize}
\item a $(0,\infty)$-valued metric may be called a \index{genuine metric}\emph{genuine metric}.
\item a $(0,\infty]$-valued metric may be called an \index{$\infty$-metric}\emph{$\infty$-metric}.
\item a $[0,\infty)$-valued metric may be called  a \index{genuine pseudommetric}\emph{genuine pseudommetric}.
\item A $[0,\infty]$-valued metric may be called a \index{pseudometric}\emph{pseudometric} or \index{$\infty$-pseudometric}\emph{$\infty$-pseudometric}.
\end{itemize}

A metric space is a set equipped with a metric.
The distance between points $x$ and $y$ in a metric space $\spc{X}$ will  usually be denoted by \[\dist{x}{y}{}\qquad\text{or}\quad \dist{x}{y}{\spc{X}};\]
the latter will be used if we need to emphasize that we are working in the space $\spc{X}$.

The function $\distfun{x}{}{}\:\spc{X}\to\RR$ defined as 
\[\distfun{x}{}{}\:y\mapsto\dist{x}{y}{}\]
will be called the \index{distance function}\emph{distance function} from $x$. 

Any  subset $A$ in a metric space $\spc{X}$ will be also considered as a \index{subspace}\emph{subspace};
that is, a metric space with the metric defined by restricting the metric of $\spc{X}$ to $A\times A\subset \spc{X}\times\spc{X}$.

The \index{direct product}\emph{direct product} $\spc{X}\times \spc{Y}$ of two metric spaces $\spc{X}$ and $\spc{Y}$ is defined as the metric space carrying  the metric
\[
\dist{(p,\phi)}{(q,\psi)}{} =\sqrt{\dist[2]{p}{q}{} + \dist[2]{\phi}{\psi}{}}
\]
for $p,q\in \spc{X}$ and $\phi,\psi\in \spc{Y}$. 


\parbf{Zero and infinity.}
Genuine metric spaces are the main objects of study in this book.
However, the  generalizations above are useful
in  various definitions and constructions.
For example, the construction of length metric (see Section~\ref{sec:intrinsic}) uses infinite distances.
The following definition gives another example.

\begin{thm}{Definition}\label{def:disjoint-union}
Assume $\{\spc{X}_\alpha\}_{\alpha\in\IndexSet}$ is a collection of $\infty$-metric spaces.
The disjoint union 
$$\bm{X}=\bigsqcup_{\alpha\in\IndexSet}\spc{X}_\alpha$$ 
has a natural $\infty$-metric on it defined as follows:
given two points $x\in\spc{X}_\alpha$ and $y\in\spc{X}_\beta$
set 
\[
\begin{matrix}
\dist{x}{y}{\bm{X}}=\infty&\text{if}\quad\alpha\ne\beta,
\\
\dist{x}{y}{\bm{X}}=\dist{x}{y}{\spc{X}_\alpha}&\text{if}\quad\alpha=\beta.
\end{matrix}
\]
The resulting $\infty$-metric space $\bm{X}$ will be called the \index{disjoint union of metric spaces}\emph{disjoint union} of $\{\spc{X}_\alpha\}_{\alpha\in\IndexSet}$, denoted by \[\bigsqcup_{\alpha\in\IndexSet}\spc{X}_\alpha.\]
\end{thm}

Now let us give examples showing that vanishing and infinite distance between distinct points is useful.


Suppose a set ${\spc{X}}$ comes with a set of metrics $\dist{}{}{\alpha}$ for $\alpha\in\IndexSet$.
Then 
\[\dist{x}{y}{}=\sup\set{\dist{x}{y}{\alpha}}{\alpha\in\IndexSet}\]
is in general only an $\infty$-metric;
that is,  even if the metrics $\dist{}{}{\alpha}$ are genuine, then $\dist{}{}{}$ might be $(0,\infty]$-valued.

Let $\spc{X}$ be a set,
$\spc{Y}$ be a metric space and  
 $\map\:\spc{X}\to\spc{Y}$ be a map.
If $\map$ is not injective,
then the {}\emph{pullback}
\[\dist{x}{y}{\spc{X}}=\dist{\map(x)}{\map(y)}{\spc{Y}}\]
defines only a pseudometric on $\spc{X}$.

\parbf{Corresponding metric space and metric component.}
The following two observations show that
nearly any question about metric spaces can be reduced to a question about genuine metric spaces.

Assume $\spc{X}$ is a pseudometric space.
Set
$x\sim y$ if $\dist{x}{y}{}=0$. 
Note that if $x\sim x'$, then $\dist{y}{x}{}=\dist{y}{x'}{}$ for any $y\in\spc{X}$.
Thus, $\dist{}{}{}$ defines a metric on the
quotient set $\spc{X}/{\sim}$.
In this way we obtain a metric space $\spc{X}'$.
The space $\spc{X}'$ is called the 
\emph{corresponding metric space} for the pseudometric space $\spc{X}$.
Often we do not distinguish between $\spc{X}'$ and~$\spc{X}$. 


Set $x\approx y$ if and only if $\dist{x}{y}{}<\infty$;
this is another equivalence relation on $\spc{X}$.
The equivalence class of a point $x\in\spc{X}$ will be called the \index{metric component}\emph{metric component} 
 of $x$; it will be denoted as $\spc{X}_x$.
One could think of $\spc{X}_x$ as  $\oBall(x,\infty)_{\spc{X}}$, the open ball centered at $x$ and radius $\infty$ in $\spc{X}$; see definition below.

It follows that any $\infty$-metric space is a {}\emph{disjoint union} of genuine metric spaces, the metric components of the original $\infty$-metric space; see Definition~\ref{def:disjoint-union}

To summarize this discussion: Given a $[0,\infty]$-valued metric space $\spc{X}$, we may pass to the corresponding $(0,\infty]$-valued metric space $\spc{X}'$ and break the latter  into a disjoint union of metric components, each of which is  a genuine metric space.


\section{Notations}

\parbf{Balls.}
Given $R\in[0,\infty]$ and a point $x$ in a metric space $\spc{X}$, the sets
\begin{align*}
\oBall(x,R)&=\set{y\in \spc{X}}{\dist{x}{y}{}<R},
\\
\cBall[x,R]&=\set{y\in \spc{X}}{ \dist{x}{y}{}\le R}%V: I suggest using just square brackets or just a bar over B for closed balls. Using both is an overkill. %S. I prefer B(A,R) and \bar B(A,R) where A is radius, B and \bar B tell open or closed.  This is easiest and makes most sense. A: for me [] reinforse the meaning, but I will agree on \bar B(x,R)
\end{align*}
are called respectively the \index{open ball}\emph{open} and the \index{closed ball}\emph{closed  balls} of radius $R$ with center $x$.

If we need to emphasize that these balls are taken in the space $\spc{X}$,
we write 
\[\oBall(x,R)_{\spc{X}}\quad\text{and}\quad \cBall[x,R]_{\spc{X}}\]
respectively.

Since in the model space $\Lob{m}{\kappa}$ all balls of the same radius are isometric, 
often we will not need to specify the center of the ball,
and may write 
\[\oBall(R)_{\Lob{m}{\kappa}}\quad\text{and}\quad\cBall[R]_{\Lob{m}{\kappa}}\] respectively.

A set $A\subset\spc{X}$ is called \index{bounded set}\emph{bounded} if $A\subset\oBall(x,R)$ for some $x\in\spc{X}$ and $R<\infty$.

\parbf{Distances to sets.}
For subset $A\subset \spc{X}$, 
let us denote the distance from $A$ to a point $x$ in $\spc{X}$ as $\distfun{A}{x}$\index{$\distfun{A}{x}$};
that is,
\[\distfun{A}{x}
\df
\inf\set{\dist{a}{x}{}}{a\in A}.\] 
We define the distance between sets $A$ and $B$  as 
\[\dist{A}{B}{}
\df
\inf\set{\dist{a}{b}{}}{a\in A,\ b\in B}.\]%???maybe a bad notation --- this is not a distance

For any subset $A\subset\spc{X}$,  the sets
\begin{align*}
\oBall(A,R)&=\set{y\in \spc{X}}{ \distfun{A}{y}{}<R},
\index{\oBall(A,R)}
\\
\cBall[A,R]&=\set{y\in \spc{X}}{\distfun{A}{y}{}\le R}
\index{\cBall[A,R]}
\end{align*}
are called respectively the  \index{open tubular neighborhood}\emph{open} and \index{closed tubular neighborhood}\emph{closed $R$-neighborhoods} of $A$.

\parbf{Diameter, radius and packing.}
Let $\spc{X}$ be a metric space.
Then the \index{diameter}\emph{diameter} of $\spc{X}$ is  defined as
\[\diam \spc{X}=\sup\set{\dist{x}{y}{}}{x,y\in \spc{X}}.\]

The \index{radius}\emph{radius} of $\spc{X}$ is  defined as
\[\rad \spc{X}=\inf\set{R>0}{\oBall(x,R)= \spc{X}\ \text{for some}\ x\in\spc{X}}.\]

The packing number \index{$\eps$-pack}\emph{$\eps$-pack} of $\spc{X}$ is the maximal number  (possibly infinite) of points in $\spc{X}$ at distance $>\eps$ from each other;  it is denoted by $\pack_\eps\spc{X}$.
If $m=\pack_\eps\spc{X}<\infty$, then a set $\{x^1,x^2,\dots,x^m\}$ in $\spc{X}$ 
such that $\dist{x^i}{x^j}{}>\eps$ is called a \index{maximal packing}\emph{maximal $\eps$-packing} in $\spc{X}$.


\parbf{G-delta sets.}
Recall that an arbitrary union of open balls in a metric space is called an \index{open set}\emph{open set}.
A subset of a metric space is called a \emph{G-delta set} if it can be presented as an intersection of  a countable number of open subsets. 

Often we will use the following classical result:

\begin{thm}{Baire's theorem}
Let $\spc{X}$ be a complete metric space 
and $\{\Omega_n\}$, $n\in \NN$, be a collection of open dense subsets of $\spc{X}$.
Then $\bigcap_{n=1}^\infty\Omega_n$ is dense in $\spc{X}$.
\end{thm}
 

\parbf{Proper spaces.}
A metric space $\spc{X}$ is called \index{proper space}\emph{proper} if all closed bounded sets in $\spc{X}$ are compact. 
This condition is equivalent to each of the following statements:
\begin{enumerate}
\item For some (and therefore any) point $p\in \spc{X}$ and any $R<\infty$, 
the closed ball $\cBall[p,R]\subset\spc{X}$ is compact. 
\item The function $\distfun{p}{}{}\:\spc{X}\to\RR$ is proper for some (and therefore any) point $p\in \spc{X}$.
\end{enumerate}

We will often use the following two classical statements:

\begin{thm}{Proposition}
Proper metric spaces are separable and second countable.
\end{thm}



\begin{thm}{Proposition}\label{compact=seq-compact}
Let $\spc{X}$ be a metric space. Then the following are equivalent
\begin{enumerate}[i)]
\item $\spc{X}$ is compact;
\item $\spc{X}$ is \index{sequentially compact}\emph{sequentially compact}; that is, any sequence of points in $\spc{X}$ contains a convergent subsequence;
\item $\spc{X}$ is complete and for any $\eps>0$ there is a finite $\eps$-net \index{$\eps$-net} in $\spc{X}$; that is, there is a finite collection of points $p_1,\ldots,p_{N}$ such that $\bigcup_i\oBall(p_i,\eps)=\spc{X}$.
\end{enumerate}
\end{thm}

\section{Length spaces}\label{sec:intrinsic}

A \index{curve}\emph{curve} in $\spc{X}$ is a continuous map $\alpha\:\II\to \spc{X}$, where $\II$ is a {}\emph{real interval} (that is, an arbitrary convex subset of $\RR$).

\begin{thm}{Definition}\label{def:length}
Let $\spc{X}$ be a metric space.
Given a curve $\alpha\: \II\to \spc{X}$, we define its \index{length}\emph{length} as 
\[
\length \alpha \df \sup\set{\sum_{i\ge 1} \dist{\alpha(t_i)}{\alpha (t_{i-1})}{}}{t_0,\dots,t_n\in \II,t_0\le\ldots\le t_n}
\]
\end{thm}

The following lemma is an easy exercise.
%The same definition works for arbitrary map $\alpha\:\II\to \spc{X}$, not necessary continuous.

\begin{thm}{Lower semicontinuity of length}\label{thm:semicont-of-length}
Assume $\alpha_n\:\II\to \spc{X}$ is a sequence of curves that converges pointwise to a curve $\alpha_\infty\:\II\to \spc{X}$.
Then 
\[\length\alpha_\infty\le \liminf_{n\to\infty}\length\alpha_n.\]

\end{thm}

Given two points $x$ and $y$ in a metric space $\spc{X}$,
consider the value
\[\yetdist{x}{y}{}=\inf_{\alpha}\{\length\alpha\},\]
where infimum is taken for all paths $\alpha$ from $x$ to $y$.

It is easy to see that $\yetdist{}{}{}$ defines a $(0,\infty]$-valued metric on  $\spc{X}$;
it will be called the induced \index{length metric}\emph{length metric} on $\spc{X}$.
Clearly 
\[\yetdist{x}{y}{}\ge \dist{x}{y}{}\]
for any $x,y\in \spc{X}$.

It  easily follows from the definition that the length of a curve $\alpha$ with respect to $\yetdist{}{}{}$ is equal to the length of $\alpha$ with respect to $\dist{}{}{}$. In particular, iterating the construction produces the same $\yetdist{}{}{}$.

\begin{thm}{Definition}\label{def:length-space}
If $\yetdist{x}{y}{}=\dist{x}{y}{}$ for any pair of points $x,y$ in a metric space $\spc{X}$, then $\spc{X}$ is called a \index{length space}\emph{length space}.
\end{thm}

In other words, a metric space $\spc{X}$ is a
\emph{length space}
if for any $\eps>0$ and any two points $x,y\in \spc{X}$ with $\dist{x}{y}{}<\infty$ there is a path $\alpha\:[0,1]\to\spc{X}$ connecting%
\footnote{That is, such that $\alpha(0)=x$ and $\alpha(1)=y$.}
 $x$ to $y$
such that 
\[\length\alpha<\dist{x}{y}{}+\eps.\]

In this book, most of the time we consider length spaces.
If $\spc{X}$ is a length space, 
and $A\subset \spc{X}$ the set $A$ comes with the inherited metric from $\spc{X}$ 
which might be not a length metric.
The corresponding length metric on $A$ will be denoted as $\dist{}{}{A}$.


\parbf{Variations of the definition.}
We will need the following variations of Definition~\ref{def:length-space}:

\begin{itemize}
\item Assume $R>0$.
If $\yetdist{x}{y}{}=\dist{x}{y}{}$ for any pair $\dist{x}{y}{}<R$, then $\spc{X}$ is called an \index{$R$-length space}\emph{$R$-length space}.
\item If any point in $\spc{X}$ admits a neighborhood  $\Omega$ such that $\yetdist{x}{y}{}=\dist{x}{y}{}$ for any pair of points $x,y\in \Omega$
then  $\spc{X}$ is called a \index{locally length space}\emph{locally length space}.
\item A metric space $\spc{X}$ is called 
\emph{geodesic}%
\index{geodesic space}
if for any two points $x,y\in \spc{X}$ with $\dist{x}{y}{}<\infty$ there is a geodesic $[x y]$ in $\spc{X}$.
\item Assume $R>0$. A metric space $\spc{X}$ is called 
\emph{$R$-geodesic}%
\index{$R$-geodesic space}
if for any two points $x,y\in \spc{X}$ such that $\dist{x}{y}{}<R$ there is a geodesic $[x y]$ in $\spc{X}$.
\end{itemize}

Note that the notions of $\infty$-length spaces and length spaces are the same.
Clearly, any geodesic space is a length space 
and any $R$-geodesic space is $R$-length.

\begin{thm}{Example} 
Let $\spc{X}$ be obtained by gluing a countable collection of disjoint intervals $\II_n$ of length $1+\tfrac1n$ where for each $\II_n$ one end is glued to $p$ and the other to $q$.
Then $\spc{X}$ carries a natural complete length metric such that $\dist{p}{q}{}=1$, but there is no geodesic connecting $p$ to~$q$.
\end{thm}

\begin{thm}{Exercise}\label{ex:complete=>complete}
Let $\spc{X}$ be a metric space
and $\yetdist{}{}{}$ be the length metric on it.
Show the following
\begin{subthm}{ex:complete=>complete:complete}
If $\spc{X}$  is complete, then  $(\spc{X},\yetdist{}{}{})$ is complete.
\end{subthm}

\begin{subthm}{ex:complete=>complete:compact}
If $\spc{X}$ is compact, then $(\spc{X},\yetdist{}{}{})$ is geodesic.
\end{subthm}
\end{thm}


\begin{thm}{Exercise}\label{ex:no-geod}
Give an example of a complete length space such that no pair of distinct points can be joined by a geodesic.
\end{thm}

\begin{thm}{Definition}
Let $\spc{X}$ be a metric space and $x,y\in\spc{X}$.

\begin{enumerate}[(i)]
\item A point $z\in \spc{X}$ is called a \index{midpoint}\emph{midpoint} for $x$ and $y$
% V: the standard phrase in English is "midpoint between $x$ and $y$".  I would suggest switching to that everywhere.
if 
\[\dist{x}{z}{}=\dist{y}{z}{}=\tfrac12\cdot\dist[{{}}]{x}{y}{}.\]
\item Assume $\eps\ge 0$.
A point $z\in \spc{X}$ is called  an \index{$\eps$-midpoint}\emph{$\eps$-midpoint} for $x$ and $y$
if 
\[\dist{x}{z}{},\quad\dist{y}{z}{}\le\tfrac12\cdot\dist[{{}}]{x}{y}{}+\eps.\]
\end{enumerate}

\end{thm}

Note that a $0$-midpoint is the same as a midpoint.


\begin{thm}{Lemma}\label{lem:mid>geod}
Let $\spc{X}$ be a complete metric space.

\begin{subthm}{lem:mid>length}
Assume that for any pair of points $x,y\in \spc{X}$ and any $\eps>0$
there is an $\eps$-midpoint $z$.
Then  $\spc{X}$ is a length space.
\end{subthm}

\begin{subthm}{lem:mid>geod:geod}
Assume that for any pair of points $x,y\in \spc{X}$ 
there is a midpoint $z$.
Then  $\spc{X}$ is a geodesic space.
\end{subthm}

\begin{subthm}{lem:mid>geod:R}
If for some $R>0$, the assumptions (\ref{SHORT.lem:mid>length}) or (\ref{SHORT.lem:mid>geod:geod}) hold only for pairs of points $x,y\in \spc{X}$ such that $\dist{x}{y}{}<R$, 
then  $\spc{X}$ is an $R$-length or $R$-geodesic space respectively.

\end{subthm}

\end{thm}


\parit{Proof.}
Fix a pair of points $x,y\in \spc{X}$.
Let $\eps_n=\frac\eps{2^{2\cdot n}}$,
$\alpha(0)=x$ and $\alpha(1)=y$.

Let $\alpha(\tfrac12)$ be an $\eps_1$-midpoint for $\alpha(0)$ and $\alpha(1)$.
Further, let $\alpha(\frac14)$ 
and $\alpha(\frac34)$  be $\eps_2$-midpoints 
for the pairs $(\alpha(0),\alpha(\tfrac12)$ 
and $(\alpha(\tfrac12),\alpha(1)$ respectively.
Applying the above procedure recursively,
on the $n$-th step we define $\alpha(\tfrac{\kay}{2^n})$,
for every odd integer $\kay$ such that $0<\tfrac\kay{2^n}<1$, to be an $\eps_{n}$-midpoint for the already defined
$\alpha(\tfrac{\kay-1}{2^n})$ and $\alpha(\tfrac{\kay+1}{2^n})$.


In this way we define $\alpha(t)$ for all dyadic rationals $t$ in $[0,1]$.
If $t\in[0,1]$ is not a dyadic rational, consider a sequence of dyadic rationals $t_n\to t$ as $n\to\infty$.
By completeness of $\spc{X}$, the sequence $\alpha(t_n)$ converges;
let $\alpha(t)$ be its limit.
It is easy to see that $\alpha(t)$
does not depend on the choice of the sequence $t_n$,
and $\alpha\:[0,1]\to\spc{X}$ is a path from $x$ to $y$.
Moreover,
\[\begin{aligned}
\length\alpha&\le \dist{x}{y}{}+\sum_{n=1}^\infty 2^{n-1}\cdot\eps_n\le
\\
&\le \dist{x}{y}{}+\tfrac\eps2.
\end{aligned}
\eqlbl{eq:eps-midpoint}
\]
Since $\eps>0$ is arbitrary, we have (\ref{SHORT.lem:mid>length}).

To prove (\ref{SHORT.lem:mid>geod:geod}), 
one should repeat the same argument 
taking midpoints instead of $\eps_n$-midpoints.
In this case \ref{eq:eps-midpoint} holds for $\eps_n=\eps=0$.

The proof of (\ref{lem:mid>geod:R}) is obtained by a straightforward modification of the proofs above.
\qeds

\begin{thm}{Corollary}
A proper length space is geodesic.
\end{thm}

This follows from Lemma \ref{lem:mid>geod} since in a compact set a sequence of $\tfrac1n$-midpoints $z_n$ contains a convergent subsequence.

{\sloppy

\begin{thm}{Hopf--Rinow theorem}\label{thm:Hopf-Rinow}
Any complete, locally compact length space is proper.
\end{thm}

}

\parit{Proof.}
Let $\spc{X}$ be a locally compact length space.
Given $x\in \spc{X}$, denote by $\rho(x)$ the supremum of all $R>0$ such that
the closed ball $\cBall[x,R]$ is compact.
Since $\spc{X}$ is locally compact 
$$\rho(x)>0\quad\text{for any}\quad x\in \spc{X}.\eqlbl{eq:rho>0}$$
It is sufficient to show that $\rho(x)=\infty$ for some (and therefore any) point $x\in \spc{X}$.

Assume the contrary; that is, $\rho(x)<\infty$.

\begin{clm}{} $B=\cBall[x,\rho(x)]$ is compact for any $x$.
\end{clm}

Indeed, $\spc{X}$ is a length space;
therefore for any $\eps>0$, 
the set $\cBall[x,\rho(x)-\eps]$ is a compact $\eps$-net in $B$.
Since $B$ is closed and hence complete, it is compact by Proposition~\ref{compact=seq-compact}.
\claimqeds

\begin{clm}{} $|\rho(x)-\rho(y)|\le \dist{x}{y}{\spc{X}}$,
in particular $\rho\:\spc{X}\to\RR$ is a continuous function.
\end{clm}

Indeed, assume the contrary; that is, $\rho(x)+\dist{x}{y}{}<\rho(y)$ for some $x,y\in \spc{X}$. 
Then 
$\cBall[x,\rho(x)+\eps]$ is a closed subset of $\cBall[y,\rho(y)]$ for some $\eps>0$.
Then  compactness of $\cBall[y,\rho(y)]$ implies compactness of $\cBall[x,\rho(x)+\eps]$, a contradiction.\claimqeds

Set $\eps=\min_{y\in B}\{\rho(y)\}$; 
the minimum is defined since $B$ is compact.
From \ref{eq:rho>0}, we have $\eps>0$.

Choose a finite $\tfrac\eps{10}$-net $\{a_1,a_2,\dots,a_n\}$ in $B$.
The union $W$ of the closed balls $\cBall[a_i,\eps]$ is compact.
Clearly 
$\cBall[x,\rho(x)+\frac\eps{10}]\subset W$.
Therefore $\cBall[x,\rho(x)+\frac\eps{10}]$ is compact,
a contradiction.
\qeds

\begin{thm}{Exercise}\label{exercise from BH}
Construct a geodesic space that is locally compact,
but whose completion is neither geodesic nor locally compact.
\end{thm}

\section{Convex sets}


\begin{thm}{Definition} 
\label{def:convex-set}
Let $\spc{X}$ be a geodesic space and $A\subset\spc{X}$.

We say $A$ is 
\emph{convex}%
\index{convex set}
if for every two points $p,q\in A$ any geodesic $[pq]$ lies in $A$.

We say $A$ is 
\emph{weakly convex}%
\index{weakly convex set}
if for every two points $p,q\in A$
there is a geodesic $[pq]$ 
that lies in $A$.

We say  $A$ is 
\emph{totally convex}
\index{totally convex set}
if for every two points $p,q\in A$, every local geodesic from $p$ to $q$ lies in $A$.

If for some $R\in (0,\infty]$ these definitions are applied only for pairs of points such that $\dist{p}{q}{}<R$ and only for the geodesics of length $<R$,
then $A$ is called respectively 
\emph{$R$-convex},%
\index{$R$-convex set} 
\emph{weakly  $R$-convex}%
or \emph{totally  $R$-convex}.

A set $A\subset\spc{X}$ is called 
\emph{locally convex}%
\index{locally convex set}
if every point $a\in A$ admits an open neighborhood $\Omega\ni a$
such that for every two points $p,q\in A\cap\Omega$ every geodesic $[pq]\subset \Omega$ lies in $A$.
Similarly one defines locally weakly convex and locally totally convex sets. %???DO we neeed all of them???
\end{thm}

\parbf{Remarks.} Let us state a few observations that easily follow  from the definition.
\begin{itemize}
\item The notion of \emph{(weakly) convex set} is the same as {}\emph{(weakly) $\infty$-convex set}.
\item The inherited metric on a weakly convex set coincides with its length metric.
\item Any open set is locally convex by definition.
\end{itemize}

The following proposition states that weak convexity survives under taking ulralimits.
An analogous statement about convexity does not hold;
for example there is a sequence of convex discs in $\mathbb{S}^2$ that converges to the half sphere $\mathbb{S}^2_+$, which is not convex.

\begin{thm}{Proposition}\label{prop:weak-convex-stable}
Let $\spc{X}_n,$ be a sequence of geodesic spaces.
Let $\o$ be an ultrafilter on $\mathbb N$; see Definition ~\ref{def:ultrafilter}.
Assume that  $A_n\subset \spc{X}_n$ is a sequence of weakly convex sets, and 
$\spc{X}_n\to \spc{X}_\o$ and $A_n\to A_\o\subset \spc{X}_\o$ as $n\to\o$.
Then $A_\o$ is a weakly convex set of $\spc{X}_\o$.
\end{thm}

\parit{Proof.}
Fix $x_\o,y_\o\in A_\o$.
Consider sequences $x_n,y_n\in A_n$ such that $x_n\to x_\o$ and $y_n\to y_\o$ as $n\to\o$.

Denote by $\alpha_n$ a geodesic path from $x_n$ to $y_n$ that lies in $A_n$.
Set
\[\alpha_\o(t)=\lim_{n\to \o}\alpha_n(t).\]

Note that $\alpha_\o$ is a geodesic path that lies in $A_\o$.
The proposition follows.
\qeds






\section{Quotient spaces}\label{sec:quotient}

\parbf{Quotient spaces.} Assume $\spc{X}$ is a metric space
with an equivalence relation $\sim$.
Note that given two pseudometrics $\rho_1$ and $\rho_2$ on $\spc{X}/{\sim}$,
their maximum 
\[\rho(x,y)=\max\{\rho_1(x,y),\rho_2(x,y)\}\]
is also a pseudometric.
If for these two pseudometrics $\rho_1$ and $\rho_2$ the projections $\spc{X}\to(\spc{X}/{\sim},\rho_i)$ are short, then the same is true for $\rho(x,y)$.

It follows that 
the quotient space $\spc{X}/{\sim}$ admits a natural quotient pseudometric;
this is the maximal pseudometric on  $\spc{X}/{\sim}$ 
that makes the quotient map 
$\spc{X}\to\spc{X}/{\sim}$ short.
The corresponding metric space will be also denoted as $\spc{X}/{\sim}$
and will be called the \index{quotient space}\emph{quotient space} of $\spc{X}$ by the equivalence relation $\sim$.

In general  the points of the metric space $\spc{X}/{\sim}$
are formed by equivalence classes in $\spc{X}$
for a wider equivalence relation.
However, in  most of the cases we will consider, 
the set of equivalence classes will coincide with the set of points in the metric space $\spc{X}/{\sim}$.



\begin{thm}{Proposition}\label{prop:length-X}
Let $\spc{X}$ be a length space and 
$\sim$ be an equivalence relation on $\spc{X}$. Then $\spc{X}/{\sim}$
is a length space.
\end{thm}

\parit{Proof.}
Let $\spc{Y}$ be an arbitrary metric space.
Since $\spc{X}$ is a length space,
a map $f\:\spc{X}\to \spc{Y}$ is short if and only if 
\[\length(f\circ\alpha)\le \length\alpha\]
for any curve
$\alpha\:\II\to \spc{X}$.
Denote by $\yetdist{}{}{}$ the length metric on $\spc{Y}$.
It follows that if $f\:\spc{X}\to \spc{Y}$ is short
then so is 
$f\:\spc{X}\to (\spc{Y},\yetdist{}{}{})$.

Consider the quotient map 
$f\:\spc{X}\to\spc{X}/{\sim}$.
Recall that the space $\spc{X}/{\sim}$ is defined by the maximal pseudometric that makes $f$ short.

Denoting by $\yetdist{}{}{}$ the length metric on $\spc{X}/{\sim}$,
it follows that
\[f\:\spc{X}\to(\spc{X}/{\sim},\yetdist{}{}{})\]
is also short.

Note that 
\[\yetdist{x}{y}{}\ge\dist{x}{y}{\spc{X}/{\sim}}\]
for any $x,y\in \spc{X}/{\sim}$.
From maximality of $\dist{}{}{\spc{X}/{\sim}}$, we get
\[\yetdist{x}{y}{}=\dist{x}{y}{\spc{X}/{\sim}}\]
for any $x,y\in \spc{X}/{\sim}$;
that is, $\spc{X}/{\sim}$ is a length space.
\qeds  

\parbf{Group actions.}
Assume a group $G$ acts on a metric space $\spc{X}$.
Consider a relation $\sim$ on $\spc{X}$
defined by $x\sim y$ if there is $g\in G$ such that $x=g\cdot y$.
Note that $\sim$ is an equivalence relation.

In this case the quotient space $\spc{X}/{\sim}$ will also be denoted by $\spc{X}/G$, and can be regarded as the space of $G$-orbits in $\spc{X}$.

Assume that the group $G$ acts on $\spc{X}$ by isometries.
Then the distance between orbits $G\cdot x$ and $G\cdot y$ in $\spc{X}/G$
can be defined directly: 
\[\dist{G\cdot x}{G\cdot y}{\spc{X}/G}=\inf\set{\dist{x}{g\cdot y}{\spc{X}}
=
\dist{g^{-1}\cdot x}{y}{\spc{X}}}{g\in G}.\]

If the $G$-orbits are closed, then $\dist{G\cdot x}{G\cdot y}{\spc{X}/G}=0$ if and only if $G\cdot x=G\cdot y$.
In this case the quotient space $\spc{X}/G$ is a genuine metric space.

The following proposition follows from the definition of a quotient space:

\begin{thm}{Proposition}\label{prop:submetry-X/G}
Assume $\spc{X}$ is a metric space and the  group $G$ acts on $\spc{X}$ by isometries.
Then the projection $\pi\co \spc{X}\to\spc{X}/G$ is a submetry;
%???V: Submetries are only defined in chapter 5. some ocmment is an order here or a forward reference to chapter 5???
that is, $\pi(\oBall(p,r))=\oBall(\pi(p),r)$ for any $p\in \spc{X}, r>0$ (see Definition~\ref{def:submetry}).
\end{thm}


\section{Gluing and doubling}\label{sec:doubling}

\parbf{Gluing.}
Recall that the disjoint union of metric spaces can be also considered as a metric space; see Definition~\ref{def:disjoint-union}.
Therefore the quotient space construction works as well for an equivalence relation on the disjoint union of metric spaces.

Consider two metric spaces $\spc{X}_1$ and $\spc{X}_2$
with subsets $A_1\subset\spc{X}_1$ and $A_2\subset\spc{X}_2$,
and a bijection $\phi\:A_1\to A_2$.
Consider the minimal equivalence relation on $\spc{X}_1\sqcup\spc{X}_2$
such that $a\sim \phi(a)$ for any $a\in A_1$.
In this case, the corresponding quotient space 
$(\spc{X}_1\sqcup\spc{X}_2)/{\sim}$ will be called the \index{gluing}\emph{gluing of $\spc{X}$ and $\spc{Y}$ along $\phi$} and denoted by
\[\spc{X}_1\sqcup_{\phi}\spc{X}_2.\]

Note that if the map $\phi\:A_1\to A_2$ is distance-preserving,
then the projections $\iota_i\:\spc{X}_i\to \spc{X}_1\sqcup_{\phi}\spc{X}_2$ are also distance-preserving, and 
\[\dist{\iota_1(x_1)}{\iota_2(x_2)}{\spc{X}_1\sqcup_{\phi}\spc{X}_2}
=
\inf_{a_2=\phi(a_1)}
\{\,\dist{x_1}{a_1}{\spc{X}_1}+\dist{x_2}{a_2}{\spc{X}_2}\,\}\]
for any $x_1\in \spc{X}_1$ and $x_2\in \spc{X}_2$.

\parbf{Doubling.}
Let $\spc{V}$ be a metric space 
and $A\subset \spc{V}$ be a closed subset.
A metric space $\spc{W}$ glued from two copies of $\spc{V}$ along $A$ is called the \index{doubling}\emph{doubling of $\spc{V}$ in $A$}.

The space $\spc{W}$ is completely described by the following properties:
\begin{itemize}
\item The space $\spc{W}$ contains $\spc{V}$ as a subspace; 
in particular the set $A$ can be treated as a subset of $\spc{W}$.
\item There is an isometric involution of $\spc{W}$ which is called \index{reflection}\emph{reflection in $A$};
further it will be denoted as $x\mapsto x'$.
\item For any $x\in \spc{W}$ we have $x\in \spc{V}$ or $x'\in \spc{V}$ and 
\[
\dist{x'}{y}{\spc{W}}
=\dist{x}{y'}{\spc{W}}
=\inf_{a\in A}\{\dist{x}{a}{\spc{V}}+\dist{a}{y}{\spc{V}}\}
\]
for any $x,y\in \spc{V}$.
\end{itemize}




The image of $\spc{V}$ under reflection in $A$ will be denoted by $\spc{V}'$.
The subspace $\spc{V}'$ is an isometric copy of $\spc{V}$.
Clearly $\spc{V}\cup\spc{V}'=\spc{W}$ and $\spc{V}\cap\spc{V}'=A$.
Moreover $a=a'$ $\iff$ $a\in A$.

The following proposition follows directly from the definitions.

\begin{thm}{Proposition}\label{prop:doubling}
Assume $\spc{W}$ is the doubling of the metric space $\spc{V}$ in its closed subset $A$.
Then: 

\begin{subthm}{}
If $\spc{V}$ is a complete length space, then so is $\spc{W}$.
\end{subthm}
 
\begin{subthm}{}
If $\spc{V}$ is proper, then so is $\spc{W}$.
In this case, for any $x,y\in\spc{V}$ there is $a\in A$ such that 
\[\dist{x}{a}{\spc{V}}+\dist{a}{y}{\spc{V}}=\dist{x}{y'}{\spc{W}}.\]
\end{subthm}

\begin{subthm}{prop:doubling:projection}
Given $x\in \spc{W}$, set $\bar x=x$ if $x\in \spc{V}$,
and $\bar x=x'$ otherwise. The map $\spc{W}\to\spc{V}$ defined by $x\mapsto \bar x$ is short and length-preserving.
In particular, if $\gamma$ is a geodesic in $\spc{W}$ with ends in $\spc{V}$, then $\bar\gamma$ is a geodesic in $\spc{V}$ with the same ends.
\end{subthm}
\end{thm}



\section{Kuratowsky embedding}

Given a metric space $\spc{X}$, 
let us denote by $\Bnd(\spc{X},\RR)$ the space of all bounded functions on $\spc{X}$ equipped with the sup-norm
\[\|f\|=\sup_{x\in\spc{X}}\{|f(x)|\}.\]

\parbf{Kuratowski embedding.}
Given a point $p\in\spc{X}$, consider the 
map $\kur_p\:\spc{X}\to\Bnd(\spc{X},\RR)$ 
defined by $\kur_p x=\distfun{x}{}{}-\distfun{p}{}{}$.
The map $\kur_p$ will be called the \index{Kuratowski map}\emph{Kuratowski map at $p$}.

From the triangle inequality, we have
\[\|\kur_p x-\kur_p y\|
=
\sup_{z\in\spc{X}}\{|\dist{x}{z}{}-\dist{y}{z}{}|\}
=
\dist{x}{y}{}.\]
Therefore, for any $p\in\spc{X}$, the Kuratowski map gives a distance-preserving map $\kur_p\:\spc{X}\hookrightarrow\Bnd(\spc{X},\RR)$.
Thus we can (and often will) consider the space $\spc{X}$ as a subset of  $\Bnd(\spc{X},\RR)$.


\begin{thm}{Exercise}\label{ex:compact-in-lenght}
Show that any compact metric space is isometric to a subspace in a compact length space.
\end{thm}
