%%!TEX root = the-metr.tex
%array^
\chapter{Metric spaces}
In this chapter we fix some conventions assuming that the reader is familiar with basic notions in metric geometry.

\section{Definitions}
\label{sec:metric spaces}

\begin{thm}{Definition}
Let $\ushort{\spc{X}}$ be a set
and 
$$\dist{}{}{}\:\ushort{\spc{X}}\times\ushort{\spc{X}}\to[0,\infty]$$
 be a function, $(x,y)\mapsto \dist{x}{y}{}$ 
which satisfies the following conditions:

\begin{subthm}{}
 $\dist{x}{x}{}=0$ for any $x\in \ushort{\spc{X}}$;
\end{subthm}


\begin{subthm}{}(symmetry)
$\dist{x}{y}{}=\dist{y}{x}{}$ for any $x,y\in \ushort{\spc{X}}$;
\end{subthm}

\begin{subthm}{}
 (triangle inequality) $\dist{x}{z}{}\le\dist{x}{y}{}+\dist{y}{z}{}$ for any $x,y,z\in \ushort{\spc{X}}$.
\end{subthm}

Then 
\begin{itemize}
\item The pair $\spc{X}=(\ushort{\spc{X}},\dist{}{}{})$ 
is called a \emph{$[0,\infty]$-metric space}.
\item The set $\ushort{\spc{X}}$ is called \emph{underlying set}\index{underlying set} of $\spc{X}$. 
The elements of $\ushort{\spc{X}}$ are called \emph{points}\index{point} of the metric space $\spc{X}$.
\item The function $\dist{}{}{}$ is called 
\emph{$[0,\infty]$-metric}\index{$[0,\infty]$-metric} of $\spc{X}$,
\end{itemize}
\end{thm}

With a slight abuse of notations we will write $x\in\spc{X}$ meaning that $x$ is a point of $\spc{X}$.

The value $\dist{x}{y}{}$ is called the \emph{distance}\index{distance} from $x$ to $y$.
If we need to emphasize 
that the distance is taken in the space ${\spc{X}}$, 
we will write $\dist{x}{y}{\spc{X}}$;

The function $\distfun{x}{}{}\:\ushort{\spc{X}}\to\RR$ defined as $\distfun{x}{}{}\:y\mapsto\dist{x}{y}{}$ will be called the \emph{distance function}\index{distance function} from $x$. 

We will need three variations of the above definition, which we denote by $(0,\infty]$-metric, $[0,\infty)$-metric and $(0,\infty)$-metric,
the intervals $(0,\infty]$, $[0,\infty)$, $(0,\infty)$ describe possible values for the distance between distinct points in the space.

\begin{thm}{Definition}
A $[0,\infty]$-metric space is called a \emph{$(0,\infty]$-metric space}\index{$(0,\infty]$-metric space}
if $\dist{x}{y}{}=0$ if and only if $x=y$.
In this case the function $(x,y)\mapsto \dist{x}{y}{}$ is called a \emph{$(0,\infty]$-metric}\index{$(0,\infty]$-metric}.
\end{thm}

\begin{thm}{Definition}
A $[0,\infty]$-metric space is called a \emph{$[0,\infty)$-metric space}\index{$[0,\infty)$-metric space}
if $\dist{x}{y}{}<\infty$ for any two points $x, y$.
In this case the function $(x,y)\mapsto \dist{x}{y}{}$ is called a \emph{$[0,\infty)$-metric}\index{$[0,\infty)$-metric}.
\end{thm}

\begin{thm}{Definition}
A $[0,\infty]$-metric space is called a \emph{$(0,\infty)$-metric space}\index{$(0,\infty)$-metric space}
if $0<\dist{x}{y}{}<\infty$ for any two points $x\ne y$.
In this case the function $(x,y)\mapsto \dist{x}{y}{}$ is called a \emph{$(0,\infty)$-metric}\index{$[0,\infty)$-metric}.
\end{thm}

The main object of study in this book are $(0,\infty)$-metric spaces which will be also called  \emph{metric spaces}.
However, the notion of $[0,\infty]$-metric spaces which allow for infinite distances between  points is useful
in  various definitions and constructions such as the following one

\begin{itemize}
\item Assume $\{\spc{X}_\alpha\}_{\alpha\in\IndexSet}$ be a collection of $[0,\infty]$-metric spaces.
The disjoint union 
$$\bm{X}=\bigsqcup_{\alpha\in\IndexSet}\spc{X}_\alpha$$ 
has a natural $[0,\infty]$-metric on it defined as follows:
given two points $x\in\spc{X}_\alpha$ and $y\in\spc{X}_\beta$
set 
\[
\begin{matrix}
\dist{x}{y}{\bm{X}}=\infty&\text{if}\ \ \alpha\ne\beta,
\\
\dist{x}{y}{\bm{X}}=\dist{x}{y}{\spc{X}_\alpha}&\text{if}\ \ \alpha=\beta.
\end{matrix}
\]
The obtained metric space $\bm{X}$ will be called the \emph{disjoint union}\index{disjoint union of metric spaces} of $\{\spc{X}_\alpha\}_{\alpha\in\IndexSet}$ and  denoted as $\bigsqcup_{\alpha\in\IndexSet}\spc{X}_\alpha$.

\item Another example is the construction of length-metric given in Section~\ref{sec:intrinsic}.
\end{itemize}

Now let us give some constructions which show that allowing distances between distinct point might to vanish can also be useful.
\begin{itemize}

\item Suppose a set ${\spc{X}}$ comes with a set of $[0,\infty]$-metrics $\dist{}{}{\alpha}$ for $\alpha\in\IndexSet$.
Then set
\[\dist{x}{y}{}=\inf\set{\dist{x}{y}{\alpha}}{\alpha\in\IndexSet}.\]
Clearly, $\dist{}{}{}$ is a $[0,\infty]$-metric.
But it need not be a $(0,\infty]$-metric  even if all $\dist{}{}{\alpha}$ are $(0,\infty)$-metrics.

\item Let $\spc{X}$ be a set,
$\spc{Y}$ be a metric space and let 
 $\map\:\spc{X}\to\spc{Y}$ be a map.
Then one can consider the \emph{pullback metric} on $\spc{X}$
defined as 
\[\dist{x}{y}{\spc{X}}=\dist{\map(x)}{\map(y)}{\spc{Y}}\]
for any $x,y\in\spc{X}$.

Note that the pullback metric $\dist{}{}{\spc{X}}$ is only  $[0,\infty)$ even if the metric on 
$\spc{Y}$ is $(0,\infty)$, since $\map$ need not be injective.

\end{itemize}

On the other hand, 
as we will see below,
nearly any question about the other 3 types of metric spaces can be reduced to a question about $(0,\infty)$-metric spaces.


\begin{itemize}
\item 
If $\spc{X}$ is a $[0,\infty]$-metric space 
and $x,x'\in\spc{X}$ are such that $\dist{x}{x'}{}=0$, then $\dist{y}{x}{}=\dist{y}{x'}{}$ for any $y\in\spc{X}$.
Thus, $\dist{}{}{}$ defines a metric on the
quotient set $\ushort{\spc{X}}/{\sim}$ with the equivalence relation 
$x\sim y$ iff $\dist{x}{y}{}=0$.
This way we obtain a $(0,\infty]$-metric space $\spc{X}'$.
Often we will not distinguish between the $(0,\infty]$-metric space $\spc{X}'$
and the $[0,\infty]$-metric space $\spc{X}$. 


\item The relation $x\sim y$ iff $\dist{x}{y}{}<\infty$ is an equivalence relation.
The equivalence class of a point $x\in\spc{X}$ will be called the \emph{metric component}\index{metric component} %???Is it OK???
 of $x$; it will be denoted as $\spc{X}_x=\oBall(x,\infty)$.
Thus, any metric space is a disjoint union of its metric components in the sense defined above.
\end{itemize}

\parbf{Related definitions.}
For a metric space as above,
\begin{itemize}
\item Given $R\in[0,\infty]$ and $x\in \spc{X}$, the sets
\begin{align*}
\oBall(x,R)&=\{y\in \spc{X}\mid \dist{x}{y}{}<R\},
\\
\cBall[x,R]&=\{y\in \spc{X}\mid \dist{x}{y}{}\le R\}
\end{align*}


are called respectively the  \emph{open}\index{open ball} and  the \emph{closed  balls}\index{closed ball}   of radius $R$ with center $x$.
\begin{itemize}
\item If we need to emphasize that these balls are taken in the metric space $\spc{X}$,
we write $\oBall(x,R)_{\spc{X}}$
and $\cBall[x,R]_{\spc{X}}$
correspondingly.

\item In case $\spc{X}=\Lob{m}{\kappa}$, we often will not need to specify the center of the ball,
so we may write $\oBall(R)_{\Lob{m}{\kappa}}$
and $\cBall[R]_{\Lob{m}{\kappa}}$ correspondingly.
\end{itemize}
\item A set $A\subset\spc{X}$ is called \emph{bounded}\index{bounded set} if $A\subset\oBall(x,R)$ for some $x\in\spc{X}$ and $R<\infty$.
\item
For subsets $A, B\subset \spc{X}$, 
%let us define the distance from $A$ to a point $x$ in $\spc{X}$ as 
%\[\dist{A}{x}{}
%\df
%\inf\set{\dist{a}{x}{}}{a\in A},\] 
%and  ??? seems redundant as this is a special case of distances between sets. V.
we define the distance between $A$ and $B$  as 
\[\dist{A}{B}{}
\df
\inf\set{\dist{a}{b}{}}{a\in A,\ b\in B}.\]
\item
For any subset $A\subset\spc{X}$,  the sets
\begin{align*}
\oBall(A,R)&=\{y\in \spc{X}\mid \dist{A}{y}{}<R\},
\\
\cBall[A,R]&=\{y\in \spc{X}\mid \dist{A}{y}{}\le R\}
\end{align*}
are called respectively the  \emph{open}\index{open tubular neighborhood} and \emph{closed $R$-neighborhoods}\index{closed neighborhood} of $A$.

\item Similarly, define
\begin{align*}
\Sphere[A,R]&=\{y\in \spc{X}\mid \dist{A}{y}{}=R\},
\end{align*}
In particular, if $A=\{x\}$ is a one-point set in $\spc{X}$ 
then  $\Sphere[x,R]$ 
is the metric sphere of radius $R$ around $x$.

\item
For $0<r<R$, let 
\[
\Ann [A,r,R]=\{x\in\spc{L}| \text{ such that }r\le \dist{x}{A}{}\le R\}
\]

be the closed annulus around $A$.
\item Any  subset $A$ in a metric space $\spc{X}$ will be also considered as a \emph{subspace}\index{subspace};
that is, a metric space with the metric defined as restriction of the metric of $\spc{X}$ to $A\times A\subset \spc{X}\times\spc{X}$.
\end{itemize}

\parbf{Types of metric spaces.}
\begin{itemize}
%\item ???A metric space is called \emph{finite}\index{finite metric space},
%if its underlying set is finite. ??? I don't think there is any need to define this. V.

\item
A metric space $\spc{X}$ is called \emph{complete}\index{complete space} if any converging in itself sequence converges.
Namely, a sequence of points $x_n\in\spc{X}$ \emph{convergence in itself}\index{converges in itself} or \index{Cauchy sequence}\emph{Cauchy} if for any $\eps>0$, we have
$\dist{x_n}{x_m}{}<\eps$ for all sufficiently large $n$ and $m$.
A sequence of points $x_n\in\spc{X}$ is called \emph{convergent}\index{convergent sequence} if there is $x\in \spc{X}$ such that for any $\eps>0$ we have $\dist{x_n}{x}{}<\eps$ for all sufficiently large $n$.

Let $\spc{X}$ be arbitrary metric space.
Given two converging in itself sequences $(x_n)$ and $(x'_n)$ in $\spc{X}$,
the following limit is always defined:
\[\rho((x_n),(x'_n))
=
\lim_{n\to\infty}\dist{x_n}{x_n'}{\spc{X}}\]
and $\rho$ gives a pseudometric on the set of all senseless in $\spc{X}$ which are converging in itself.
The corresponding metric space $\spc{Z}$ is called the \emph{completion} of $\spc{X}$,  briefly, $\spc{Z}=\overline{\spc{X}}$.

Note that there is a natural embedding $\spc{X}\hookrightarrow\overline{\spc{X}}$; 
each point $x\in\spc{X}$ is mapped to a constant sequence $(x,x,\dots)$.
Thus the space $\spc{X}$ can be (and will be) considered as a subset of $\overline{\spc{X}}$.

\item A metric space $\spc{X}$ is called \emph{locally complete}\index{locally complete} if for each point $p\in \spc{X}$ there is some $R>0$ such that  the closed ball $\cBall[p,R]\subset\spc{X}$ is complete.

\item A metric space $\spc{X}$ is called \emph{proper}\index{proper space} if all closed bounded sets in $\spc{X}$ are compact. 
This condition is equivalent to each of the following statements:
\begin{enumerate}
\item For some (and therefore any) point $p\in \spc{X}$ and any $R<\infty$, 
the closed ball $\cBall[p,R]\subset\spc{X}$ is compact. 
\item The function $\distfun{p}{}{}\:\spc{X}\to\RR$ is proper for some (and therefore any) point $p\in \spc{X}$.
\end{enumerate}
\end{itemize}


\begin{thm}{Proposition}
Proper metric spaces are separable/second countable.
\end{thm}
\parit{Proof.}
Let $\spc{X}$ be  proper metric space and let $p\in \spc{X}$.
Since $\spc{X}=\cup_{n=1}^\infty\cBall[p,n]$, to see that $X$ is separable it is sufficient to construct a countable dense subset in  each $\cBall[p,n]$. 
For every $m\ge 1$  take  an open cover of $\cBall[p,n]$ by open $1/m$-balls.
By compactness of $\cBall[p,n]$ it admits a finite subcover $\oBall(q^m_1,1/m),\ldots, \oBall(q^m_{N_m},1/m)$. Then $\cup_{m, j} \{q^m_j\}$ is a countable dense subset of $\cBall[p,n]$.
Lastly, to see that $\spc{X}$ is second countable, let $\{p_i\}_{i=1}^\infty$ be a countable dense subset in $\spc{X}$. \\ Then $\cup_{i,j\ge 1} \oBall(p_i,\frac 1 j)$ is a countable basis of topology of $\spc{X}$.
\qeds


\begin{thm}{Proposition}\label{compact=seq-compact}
Let $\spc{X}$ be a metric space. Then the following are equivalent
\begin{enumerate}[i)]
\item $\spc{X}$ is compact;
\item $\spc{X}$ is \emph{sequentially compact}; that is, any sequence of points in $\spc{X}$ contains a convergent subsequence;
\item $\spc{X}$ is complete and for any $\eps>0$ there is a finite $\eps$-net in $\spc{X}$; that is, there is a finite collection of points $p_1,\ldots,p_{N}$ such that $\cup_i\oBall(p_i,\eps)=\spc{X}$.
\end{enumerate}
\end{thm}
\parit{Proof.}
i) $\Rightarrow$ ii)  and ii) $\Rightarrow$  iii) are  obvious. We will leave iii) $\Rightarrow$  ii) as an exercise.

Let us prove that ii) implies i). Let $\{U_\alpha\}_{\alpha\in\mathcal A}$ be an open cover of $\spc{X}$. 
We claim that there is an $\eps>0$ such that for any $p\in \spc{X}$ there is $\alpha\in\mathcal A$ such that $\oBall(p_i,\eps)\subset U_\alpha$. Such $\eps$ is called the \emph{ Lebesgue number} \index{ Lebesgue number} of the cover  $\{U_\alpha\}_{\alpha\in\mathcal A}$.

Suppose no such $\eps$ exists.
Then there exists a sequence of points $p_i\in\spc{X}$ such that $\oBall(p_i,1/i)$ is not contained in any $U_\alpha$.
By sequential compactness by passing to a subsequence we can assume that $p_i\to p$.
Then $p\in U_{\alpha_0}$ for some $\alpha_0$ and since $U_{\alpha_0}$ is open it follows that $\oBall(p_i,1/i)\subset U_{\alpha_0}$ for all large $i$.
This is a contradiction and hence the cover $\{U_\alpha\}_{\alpha\in\mathcal A}$ admits  a positive Lebesgue number $\eps$.
Then by iii) there is a finite $\eps$ net $p_1,\ldots,p_{N}$ in $\spc{X}$. 
By the choice of $\eps$ for each $i$ there is $\alpha_i$ such that
$\oBall(p_i,\eps)\subset U_{\alpha_i}$. 
Then $\cup_{i=1}^NU_{\alpha_i}=\spc{X}$.
\qeds 

A subset of a metric space is called G-delta set if it can be presented as an intersection of countable number of open subset.

\begin{thm}{Bair's theorem}
Let $\spc{X}$ be a complete metric space 
and $\{\Omega_n\}$, $n\in \NN$ be a collection of open dense subsets of $\spc{X}$.
Then $\bigcap_{n=1}^\infty\Omega_n$ is dense in $\spc{X}$.
\end{thm}

\parit{Proof.}
Let $U\subset\spc{X}$ be open. We need to show that $U\cap\bigl(\bigcap_{n=1}^\infty\Omega_n\bigr)\ne\emptyset$. Since $\Omega_1$ is open and dense, there is $p_1\in\spc{X}$ and $r_1>0$ such that $\cBall[p_1,r_1]\subset  U\cap \Omega_1$.
Since $\Omega_2$ is open and dense there is $p_2\in\spc{X},0<r_2<r_1/2$ such that $\cBall[p_2,r_2]\subset \oBall(p_1,r_1)\cap \Omega_2\subset U\cap\Omega_1\cap\Omega_2$. Proceeding by induction we can construct a sequence of balls $\cBall[p_i,r_i]$ such that $0<r_i<\frac {r_{i-1}}{2}$ satisfying  $\cBall[p_i,r_i]\subset \oBall(p_{i-1},r_{i-1})\cap \Omega_i\subset U\cap \bigl(\bigcap_{k=1}^i\Omega_k
 \bigr)$. Then the sequence $\{p_i\}$ converges in itself and hence, by completeness of $\spc{X}$ we have that $p_i\to p$. By construction, $p_i\in \cBall[p_j,r_j]$ for all $i\ge j$ and hence the same is true for $p$. Thus, $p\in U\cap\bigl(\bigcap_{n=1}^\infty\Omega_n\bigr)\ne\emptyset $. \qeds
 
 Recall that a family of maps between metric spaces $\{f_\alpha\co \spc{X}\to\spc{Y}\}_{\alpha\in\mathcal A}$ is called \emph{equicontinuous} if for any $\eps>0$ there is $\delta>0$ such that for any $p,q\in\spc{X}$ with $\dist{p}{q}{}<\delta$ and any $\alpha\in\mathcal A$ it holds that $\dist{f(p)}{f(q)}{}<\eps$.
\begin{thm}{Arzel\`a-Ascoli theorem}\cite[Lemma 3.10]{BH}

Let  $\{f_\alpha\co \spc{X}\to\spc{Y}\}_{\alpha\in\mathcal A}$  be an equicontinuous family  of maps where $\spc{Y}$ is compact and $\spc{X}$ is separable. 

Then $\{f_\alpha\}$ contains a uniformly convergent subsequence $f_{\alpha_i}\to f\co\spc{X}\to\spc{Y}$.
\end{thm}

\section{Invariants of metric spaces}

Let $\spc{X}$ be a metric space.
Then
\begin{itemize}
\item 
The \emph{diameter}\index{Diameter} of $\spc{X}$ is  defined as
\[\diam \spc{X}=\sup\set{\dist{x}{y}{}}{x,y\in \spc{X}}.\]

\item The \emph{radius}\index{Radius} of $\spc{X}$ is  defined as
\[\rad \spc{X}=\inf\set{R>0}{\oBall(x,R)= \spc{X}\ \t{for some}\ x\in\spc{X}}.\]

\item The packing number \emph{$\eps$-pack}\index{$\eps$-pack} of $\spc{X}$ is the maximal number  (possibly infinite) of points in $\spc{X}$ at distance $>\eps$ from each other;  it is denoted by $\pack_\eps\spc{X}$.
If $m=\pack_\eps\spc{X}<\infty$, then a set $\{x^1,x^2,\dots,x^m\}$ in $\spc{X}$ 
such that $\dist{x^i}{x^j}{}>\eps$ is called a \emph{maximal $\eps$-packing}\index{maximal $\eps$-packing} in $\spc{X}$.

\end{itemize}





















\section{Length spaces}\label{sec:intrinsic}

\begin{thm}{Definition}
Let $\spc{X}$ be a metric space.  Given $[a,b]\subset [-\infty,\infty]$ and a continuous curve $\alpha\co [a,b]\to \spc{X}$ we define its {\emph length} as 
\[
\length \alpha \df \sup_{a\le t_0\le t_1\le\ldots\le t_n\le b}\Sigma_i \dist{\alpha(t_i)}{\alpha_{i+1}}{}
\]
\end{thm}
It is easy to see that if $\tau\co [c,d]\to [a,b]$ is a homeomorphism, then $\length \alpha=\length(\alpha\circ\tau)$.


Given two points $x$ and $y$ in a metric space $\spc{X}$
consider the value
\[\yetdist{x}{y}{}=\inf_{\alpha}\{\length\alpha\},\]
where infimum is taken for all paths $\alpha$ from $x$ to $y$.

It is easy to see that $\yetdist{}{}{}$ defines a metric on  $\spc{X}$;
it will be called the \emph{length-metric}\index{length-metric} on $\spc{X}$.

\begin{thm}{Definition}
If $\yetdist{x}{y}{}=\dist{x}{y}{}$ for any pair of points $x,y\in\spc{X}$, then $\spc{X}$ is called a \emph{length space}\index{length space}.
\end{thm}
In other words, a metric space $\spc{X}$ is a
\emph{length space}
if for any $\eps>0$ and any two points $x,y\in \spc{X}$ with $\dist{x}{y}{}<\infty$ there is a path $\alpha\:[0,1]\to\spc{X}$ connecting%
\footnote{That is, such that $\alpha(0)=x$ and $\alpha(1)=y$.}
 $x$ to $y$
such that 
\[\length\alpha<\dist{x}{y}{}+\eps.\]

Let us give a few more variations of  the notion of length spaces.

\begin{itemize}
\item Assume $R>0$.
If $\yetdist{x}{y}{}=\dist{x}{y}{}$ for any pair $\dist{x}{y}{}<R$, then $\spc{X}$ is called an \emph{$R$-length space}\index{$R$-length space}.
\item If any point in $\spc{X}$ admits a neighborhood  $\Omega$ such that $\yetdist{x}{y}{}=\dist{x}{y}{}$ for any pair of points $x,y\in \Omega$
then  $\spc{X}$ is called a \emph{locally length space}\index{locally length space}.
\item A metric space $\spc{X}$ is called 
\emph{geodesic}%
\index{geodesic space}
if for any two points $x,y\in \spc{X}$ with $\dist{x}{y}{}<\infty$ there is a geodesic $[x y]$ in $\spc{X}$.
\item Assume $R>0$. A metric space $\spc{X}$ is called 
\emph{$R$-geodesic}%
\index{$R$-geodesic space}
if for any two points $x,y\in \spc{X}$ with $\dist{x}{y}{}<R$ there is a geodesic $[x y]$ in $\spc{X}$.
\end{itemize}

Note that the notions of $\infty$-length spaces and length spaces are the same.
Clearly, any geodesic space is a length space 
and any $R$-geodesic space is an $R$-length space.
\parbf{Example.} 
Let $\spc{X}$ be obtained by gluing a countable collection of disjoint intervals $I_i$ of length $1+1/i$ where for each $I_i$ one end is glued to $p=\{0\}$ and the other to $q=\{1\}$.
Then $\spc{X}$ carries a natural complete length metric  with respect to which $\dist{p}{q}{}=1$ but there is no geodesic connecting $p$ to $q$.

\begin{thm}{Exercise}\label{ex:complete=>complete}
Let $\spc{X}$ be a metric space
and $\yetdist{}{}{}$ is the length metric on it.
Show the following
\begin{subthm}{}
If complete  $(\spc{X},\yetdist{}{}{})$ is complete.
\end{subthm}

\begin{subthm}{}
If $\spc{X}$ is compact, then $(\spc{X},\yetdist{}{}{})$ is geodesic.
\end{subthm}
\end{thm}


\begin{thm}{Exercise}\label{ex:no-geod}
Give an example of a complete length space for which no pair of distinct points can be joined by a geodesic.
\end{thm}

In this book, most of the time we consider length spaces.
If $\spc{X}$ is length space, 
and $A\subset \spc{X}$.
The set $A$ comes with the inherited metric from $\spc{X}$ 
which might be not a length-metric.
The corresponding length-metric on $A$ will be denoted as $\dist{}{}{A}$.

\begin{thm}{Definition}
Let $\spc{X}$ be a metric space and $x,y\in\spc{X}$.

\begin{enumerate}[(i)]
\item A point $z\in \spc{X}$ is called a \emph{midpoint} of $x$ and $y$
if 
\[\dist{x}{z}{}=\dist{y}{z}{}=\tfrac12\cdot\dist[{{}}]{x}{y}{}.\]
\item Assume $\eps\ge 0$.
A point $z\in \spc{X}$ is called \emph{$\eps$-midpoint} of $x$ and $y$
if 
\[\dist{x}{z}{},\dist{y}{z}{}<\tfrac12\cdot\dist[{{}}]{x}{y}{}+\eps.\]
\end{enumerate}

\end{thm}

Note that a $0$-midpoint is the same as a midpoint.


\begin{thm}{Lemma}\label{lem:mid>geod}
Let $\spc{X}$ be a complete metric space and $R>0$.
\begin{subthm}{lem:mid>length}
Assume that for any pair of points $x,y\in \spc{X}$  
such that $\dist{x}{y}{}<R$ and any $\eps>0$
there is a $\eps$-midpoint $z$.
Then the space $\spc{X}$ is an $R$-length space.
\end{subthm}

\begin{subthm}{lem:mid>geod:geod}
Assume that for any pair of points $x,y\in \spc{X}$ 
such that $\dist{x}{y}{}<R$
there is a midpoint $z$.
Then the space $\spc{X}$ is an $R$-geodesic space.
\end{subthm}
\end{thm}

\parit{Proof.}
Let $x,y\in \spc{X}$ be a pair of points such that $\dist{x}{y}{}<R$.

Set $\eps_n=\frac\eps{2^{2\cdot n}}$.

Set $\alpha(0)=x$ and $\alpha(1)=y$.

Set $\alpha(\tfrac12)$ to be an $\eps_1$-midpoint of $\alpha(0)$ and $\alpha(1)$.
Further, set $\alpha(\frac14)$ 
and $\alpha(\frac34)$ to be $\eps_2$-midpoints 
for the pairs $(\alpha(0),\alpha(\tfrac12)$ 
and $(\alpha(\tfrac12),\alpha(1)$ respectively.
Applying the above procedure recursively,
on the $n$-th step we define $\alpha(\tfrac{\kay}{2^n})$
for every odd integer $\kay$ such that $0<\tfrac\kay{2^n}<1$, 
as an $\eps_{n}$-midpoint of the already defined
$\alpha(\tfrac{\kay-1}{2^n})$ and $\alpha(\tfrac{\kay+1}{2^n})$.


In this way we define $\alpha(t)$ for $t\in W$,
where $W$ denotes the set of dyadic rationals in $[0,1]$.
For any $t\in[0,1]$ consider a sequence of $t_n\in W$ such that $t_n\to t$ as $n\to\infty$.
Note that the sequence $\alpha(t_n)$ converges;
define $\alpha(t)$ as its limit.
It is easy to see that $\alpha(t)$
does not depend on the choice of the sequence $t_n$
and $\alpha\:[0,1]\to\spc{X}$ is a path from $x$ to $y$.
Moreover,
\[\begin{aligned}
\length\alpha&\le \dist{x}{y}{}+\sum_{n=1}^\infty 2^{n-1}\cdot\eps_n\le
\\
&\le \dist{x}{y}{}+\tfrac\eps2.
\end{aligned}
\eqlbl{eq:eps-midpoint}
\]
Since $\eps>0$ is arbitrary, we get that (\ref{SHORT.lem:mid>length}).

To prove (\ref{SHORT.lem:mid>geod:geod}), 
one should repeat the same argument 
taking midpoints instead of $\eps_n$-midpoints.
In this case \ref{eq:eps-midpoint} holds for $\eps_n=\eps=0$.
\qeds

Since in a compact space a sequence of $1/n$-midpoints $z_n$ contains a convergent subsequence, Lemma  \ref{lem:mid>geod} immediately implies

\begin{thm}{Proposition}
A proper length space is geodesic.
\end{thm}

\begin{thm}{Hopf--Rinow theorem}\label{thm:Hopf-Rinow}
Any complete, locally compact length space is proper.
\end{thm}

\parit{Proof.}
Let $\spc{X}$ be a locally compact length space.
Given $x\in \spc{X}$, denote by $\rho(x)$ the supremum of all $R>0$ such that
the closed ball $\cBall[x,R]$ is compact.
Since $\spc{X}$ is locally compact 
$$\rho(x)>0\ \ \text{for any}\ \ x\in \spc{X}.\eqlbl{eq:rho>0}$$
It is sufficient to show that $\rho(x)=\infty$ for some (and therefore any) point $x\in \spc{X}$.

Assume the contrary; that is, $\rho(x)<\infty$.

\begin{clm}{} $B=\cBall[x,\rho(x)]$ is compact for any $x$.
\end{clm}

Indeed, $\spc{X}$ is a length space;
therefore for any $\eps>0$, 
the set $\cBall[x,\rho(x)-\eps]$ is a compact $\eps$-net in $B$.
Since $B$ is closed and hence complete, it has to be compact by Proposition~\ref{compact=seq-compact}.
\claimqeds

\begin{clm}{} $|\rho(x)-\rho(y)|\le \dist{x}{y}{\spc{X}}$,
in particular $\rho\:\spc{X}\to\RR$ is a continuous function.
\end{clm}

Indeed, 
assume the contrary; that is, $\rho(x)+|x-y|<\rho(y)$ for some $x,y\in \spc{X}$. 
Then 
$\cBall[x,\rho(x)+\eps]$ is a closed subset of $\cBall[y,\rho(y)]$ for some $\eps>0$.
Then  compactness of $\cBall[y,\rho(y)]$ implies compactness of $\cBall[x,\rho(x)+\eps]$, a contradiction.\claimqeds

Set $\eps=\min_{y\in B}\{\rho(y)\}$; 
the minimum is defined since $B$ is compact.
From \ref{eq:rho>0}, we have $\eps>0$.

Choose a finite $\tfrac\eps{10}$-net $\{a_1,a_2,\dots,a_n\}$ in $B$.
The union $W$ of the closed balls $\cBall[a_i,\eps]$ is compact.
Clearly 
$\cBall[x,\rho(x)+\frac\eps{10}]\subset W$.
Therefore $\cBall[x,\rho(x)+\frac\eps{10}]$ is compact;
a contradiction.
\qeds

\begin{thm}{Exercise}(from \cite{BH})\label{exercise from BH}
Construct a geodesic space which is locally compact,
but whose completion is neither geodesic nor locally compact.
\end{thm}

\section{Convex sets}

\begin{thm}{Definition} 
\label{def:convex-set}
Let $\spc{X}$ be a metric space. 
A set $A\subset\spc{X}$ is called 
\emph{convex}%
\index{convex set}
if for every two points $p,q\in A$ there every geodesic $[pq]$ of $\spc{X}$ 
lies in $A$.
It is called 
\emph{weakly convex}%
\index{convex set!weakly convex set}\index{weakly convex set}
if for every two points $p,q\in A$ there is a geodesic $[pq]$ of $\spc{X}$ 
which lies in $A$.
If in addition every local geodesic from $p$ to $q$ lies in $A$, then $A$ is called \emph{totally convex}\index{totally convex set}

If these definition is applied only when  $\dist{p}{q}{}<R$, $R\in (0,\infty]$ and only for the geodesics of length $<R$
then $A$ is called correspondingly 
\emph{$R$-convex},
\emph{weakly  $R$-convex}%
\index{weakly convex set!weakly $R$-convex set} 
and \emph{totally  $R$-convex}
\index{totally convex set!totally $R$-convex set}.

A set $A\subset\spc{X}$ is called 
\emph{locally convex}
if every point $a\in A$ admits an open neighborhood $\Omega\ni a$
suc that for every two points $p,q\in A\cap\Omega$ every geodesic $[pq]\subset \Omega$ lies in $A$.
Similarly one defines  locally weakly convex and locally totally convex sets. %???DO we neeed all of them???
\end{thm}

\parbf{Remarks.}
\begin{itemize}
\item Note that \emph{(weakly) convex sets} can be also regarded as \emph{(weakly) $\infty$-convex sets}.
\item Note that inherited metric on a weakly convex set coinsides with its length-metric.
\item Note that any open set is locally convex by definition.
\end{itemize}

The following proposition states that weak convexity survive under ulralimit.
An analogus statement about convexity does not hold;
for example there is a sequence of convex discs in $\SS^2$ which converges to the half sphere $\SS^2_+$, which is not convex.

\begin{thm}{Proposition}\label{prop:weak-convex-stable}
Let $(\spc{X}_n,\star_n)$ be a sequence of metric spaces with marked points.
Assume $A_n\subset \spc{X}_n$ be a sequence of weakly convex sets and 
$\spc{X}_n\to \spc{X}_\o$ and $A_n\to A_\o\subset \spc{X}_\o$ as $n\to\o$.
Then $A_\o$ is a weakly convex set of $\spc{X}_\o$.
\end{thm}

\parit{Proof.}
Fix $x_\o,y_\o\in A_\o$.
Consider the sequences $x_n,y_n\in A_n$ such that $x_n\to x_\o$ and $y_n\to y_\o$ as $n\to\o$.

Denote by $\alpha_n$ a geodesic path from $x_n$ to $y_n$ which lies in $A_n$.
Set
\[\alpha_\o(t)=\lim_{n\to \o}\alpha_n(t).\]

Note that $\alpha_\o$ is a geodesic path which lies in $A_\o$.
Hence the result follows.
\qeds






\section{Quotient spaces}\label{sec:quotient}

\parbf{Quotient spaces.} Assume $\spc{X}$ is a metric space
with an equivalence relation $\sim$.
Note that given two pseudometrics $\rho_1$ and $\rho_2$ on $\spc{X}/{\sim}$
their maximum 
\[\rho(x,y)=\max\{\rho_1(x,y),\rho_2(x,y)\}\]
is also a pseudometric;
also if for these two pseudometric $\rho_1$ and $\rho_2$ the projection $\spc{X}\to\spc{X}/{\sim}$ is short, then so it true for  $\rho(x,y)$.

It follows that 
the quotient space $\spc{X}/{\sim}$ admits natural pseudometric;
this is the maximal pseudometric on  $\spc{X}/{\sim}$ 
which makes the quotient map 
$\spc{X}\to\spc{X}/{\sim}$ to be short.
The corresponding metric space will be also denoted as $\spc{X}/{\sim}$
and will be called the \emph{quotient space}\index{quotient space} of $\spc{X}$ by the equivalence relation $\sim$.

In general case the points of the metric space $\spc{X}/{\sim}$
are formed by equivalence classes in $\spc{X}$
for a wider equivalence relation, say $\approx$; 
that is, such that $x\sim y$ implies $x\approx y$ for any $x,y\in \spc{X}$.
But in the most of the cases we will consider, 
the set of equivalence classes will coincide with the set of points in the metric space $\spc{X}/{\sim}$.



\begin{thm}{Proposition}\label{prop:length-X}
Let $\spc{X}$ be a length space and 
$\sim$ be an equivalence relation on $\spc{X}$, then $\spc{X}/{\sim}$
is a length space.
\end{thm}

\parit{Proof.}
Let $\spc{Y}$ be arbitrary metric space.
Since $\spc{X}$ is a length space,
the map $f\:\spc{X}\to \spc{Y}$ is short if and only if 
\[\length(f\circ\alpha)\le \length\alpha\]
for any curve
$\alpha\:\II\to \spc{X}$.

Denote by $\yetdist{}{}{}$ the length-metric on $\spc{Y}$.
It follows that if $f\:\spc{X}\to \spc{Y}$ is short
then so is 
$f\:\spc{X}\to (\spc{Y},\yetdist{}{}{})$.

Consider the quotient map 
$f\:\spc{X}\to\spc{X}/{\sim}$.
Recall that the space $\spc{X}/{\sim}$ is defined by the maximal pseudometric which makes $f$ to be short.

Denote by $\yetdist{}{}{}$ the length-metric on $\spc{X}/{\sim}$.
It follows that
\[f\:\spc{X}\to(\spc{X}/{\sim},\yetdist{}{}{})\]
is also short.

Note that 
\[\yetdist{x}{y}{}\ge\dist{x}{y}{\spc{X}/{\sim}}\]
for any $x,y\in \spc{X}/{\sim}$.
From maximality of $\dist{}{}{\spc{X}/{\sim}}$, we get
\[\yetdist{x}{y}{}=\dist{x}{y}{\spc{X}/{\sim}}\]
for any $x,y\in \spc{X}/{\sim}$;
that is, $\spc{X}/{\sim}$ is a length space.
\qeds

\parbf{Gluing.}
Recal that disjoint union of metric spaces can be also considered as a metric.
Therefore the above construction works as well for an equivalence relation on a disjoint union of metric spaces.

Consider two metric spaces $\spc{X}_1$ and $\spc{X}_2$
with subset $A_1\subset\spc{X}_1$ and $A_2\subset\spc{X}_2$
and a bijection $\phi\:A_1\to A_2$.
Consider the minimal equivalence relation on $\spc{X}_1\sqcup\spc{X}_2$
such that $a\sim \phi(a)$ for any $a\in A_1$.
In this case, we say that the corresponding quotient space 
$(\spc{X}_1\sqcup\spc{X}_2)/{\sim}$ will be called \emph{gluing}\index{gluing} of $\spc{X}$ and $\spc{Y}$ along $\phi$ and denoted as
\[\spc{X}_1\sqcup_{\phi}\spc{X}_2.\]

Assume that the map $\phi\:A_1\to A_2$ is distance preserving for the inherited metrics.
Then the projections $\iota_i\:\spc{X}_i\to \spc{X}_1\sqcup_{\phi}\spc{X}_2$ are also distance preserving and 
\[\dist{\iota_1(x_1)}{\iota_2(x_2)}{\spc{X}_1\sqcup_{\phi}\spc{X}_2}
=
\inf
\set{\dist{x_1}{a_1}{\spc{X}_1}+\dist{x_2}{a_2}{\spc{X}_2}}
{ a_2=\phi(a_1)}\]
for any $x_1\in \spc{X}_1$ and $x_2\in \spc{X}_2$.  

\parbf{Group actions.}
Assume a group $G$ acts on a metric space $\spc{X}$.
Consider relation $\sim$ on $\spc{X}$
defined as $x\sim y$ if there is $g\in G$ such that $x=g\cdot y$.
Note that $\sim$ is an equivalence relation.

In this case the quotiont space $\spc{X}/{\sim}$ will ba also denoted as $\spc{X}/G$.

Assume $G$ acts on $\spc{X}$ by isometrics and it has closed orbits.

The case of isometric actions will be important for us.

Assume that the group $G$ acts on $\spc{X}$ by isometries 
and it has closed orbits.
Let us denote by $\spc{X}/G$ the space of orbits of $G$.

Let us equip $\spc{X}/G$ with a metric.
Given two orbits $G\cdot x$ and $G\cdot y$ in $\spc{X}/G$
set 
\[\dist{G\cdot x}{G\cdot y}{\spc{X}/G}=\inf\set{\dist{x}{g\cdot y}{\spc{X}}
=
\dist{g^{-1}\cdot x}{y}{\spc{X}}}{g\in G}.\]
Note that since $G$-orbits assumed to be closed $\dist{G\cdot x}{G\cdot y}{\spc{X}/G}=0$ if and only if $G\cdot x=G\cdot y$.
It straightforward to check the remaining axioms of metric.

\begin{thm}{Proposition}\label{prop:submetry-X/G}
Assume $\spc{X}$ is metric space and group $G$ acts on $\spc{X}$ by isometries.
Then the projection $\spc{X}\to\spc{X}/G$ is a submetry.
\end{thm}

\parit{Proof.}
We need to show that the quotient map is short ans 1-co-Lipschitz.
Both statements follow from the definition of quotient space.
\qeds

\section{Doubling}\label{sec:doubling}

Let $\spc{V}$ be a metric space 
and $A\subset \spc{V}$ be a closed subset.
A metric space $\spc{W}$ glued from two copies of $\spc{V}$ along $A$ is called \emph{doubling of $\spc{V}$ in $A$}\index{doubling}.

More formally the space $\spc{W}$ can be described by the following properties.
\begin{itemize}
\item The space $\spc{W}$ contains $\spc{V}$ as a subspace; 
in particular the set $A$ can be treated as a subset of $\spc{W}$.
\item There is an isometric involution of $\spc{W}$ which is called \index{reflection}\emph{reflection in $A$};
further it will be denoted as $x\mapsto x'$.
\item For any $x\in \spc{W}$ we have $x\in \spc{V}$ or $x'\in \spc{V}$ and 
\[
\dist{x'}{y}{\spc{W}}
=\dist{x}{y'}{\spc{W}}
=\inf_{a\in A}\{\dist{x}{a}{\spc{V}}+\dist{a}{y}{\spc{V}}\}.
\]
for any $x,y\in \spc{V}$.
\end{itemize}




The image of $\spc{V}$ under the reflection in $A$ will be denoted by $\spc{V}'$.
The subspace $\spc{V}'$ is an isometric copy of $\spc{V}$.
Clearly $\spc{V}\cup\spc{V}'=\spc{W}$ and $\spc{V}\cap\spc{V}'=A$.
Moreover $a=a'$ $\iff$ $a\in A$.

The following proposition follows directly from the definitions.

\begin{thm}{Proposition}\label{prop:doubling}
Assume $\spc{W}$ is the doubling of metric space $\spc{V}$ in its closed subset $A$.
Then 

\begin{subthm}{}
If $\spc{V}$ is a complete length space, then so is $\spc{W}$.
\end{subthm}
 
\begin{subthm}{}
If $\spc{V}$ is proper, then so is $\spc{W}$.
In this case, for any $x,y\in\spc{V}$ there is $a\in A$ such that 
\[\dist{x}{a}{\spc{V}}+\dist{a}{y}{\spc{V}}=\dist{x}{y'}{\spc{W}}.\]
\end{subthm}

\begin{subthm}{prop:doubling:projection}
Given $x\in \spc{W}$ set $\bar x=x$ if $x\in \spc{V}$
and $\bar x=x'$ otherwise. The map $\spc{W}\to\spc{V}$ defined by $x\mapsto \bar x$ is short and length-preserving.
In particular if $\gamma$ is a geodesic in $\spc{W}$ with ends in $\spc{V}$, then $\bar\gamma$ is a geodesic in $\spc{V}$ with the same ends.
\end{subthm}
\end{thm}



\section{Kuratowsky embedding}

Given a metric space $\spc{X}$, 
let us denote by $\Bnd(\spc{X},\RR)$ the space of all bounded functions on $\spc{X}$ equipped with the supremum norm,
\[\|f\|=\sup_{x\in\spc{X}}\{|f(x)|\}.\]

\parbf{Kuratowski embedding.}
Given a point $p\in\spc{X}$, consider 
map $\kur_p\:\spc{X}\to\Bnd(\spc{X},\RR)$, 
defined as $\kur_p x=\distfun{x}{}{}-\distfun{p}{}{}$.
The map $\kur_p$ will be called \emph{Kuratowski map at $p$}%
\index{Kuratowski map}.

From the triangle inequality, we have
\[\|\kur_p x-\kur_p y\|
=
\sup_{z\in\spc{X}}\{|\dist{x}{z}{}-\dist{y}{z}{}|\}
=
\dist{x}{y}{}.\]
Therefore, for any $p\in\spc{X}$, the Kuratowski map gives a distance preserving map $\kur_p\:\spc{X}\hookrightarrow\Bnd(\spc{X},\RR)$.
Thus, we can (and often will) consider space $\spc{X}$ as a subset of  $\Bnd(\spc{X},\RR)$.


\begin{thm}{Exercise}
Show that any compact space is isometric to a subspace in a compact length space.
\end{thm}