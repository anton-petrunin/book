%%!TEX root =the-euclid.tex
%arXiv
\chapter{The ghost of Euclid
}

\section{Geodesics, triangles and hinges}
\label{sec:geods}

\parbf{Geodesics and their relatives.}
Let $\spc{X}$ be a metric space 
and $\II$\index{$\II$} be a real interval. 
A globally isometric map $\gamma\:\II\to \spc{X}$ is called a \emph{unit-speed geodesic}\index{unit-speed geodesic}%
\footnote{Various authors call it differently: \emph{shortest path}, \emph{minimizing geodesic}.}; 
in other words, $\gamma\:\II\to \spc{X}$ is a unit-speed geodesic if the equality
\[\dist{\gamma(s)}{\gamma(t)}{\spc{X}}=|s-t|\]
holds for any pair $s,t\in \II$.

A unit-speed geodesic $\gamma\:\RR_{\ge0}\to \spc{X}$ is called a \emph{ray}\index{ray}.

A unit-speed geodesic  $\gamma\:\RR\to \spc{X}$ is called a \emph{line}\index{line}.

%geodesic between points is not defined???
A unit-speed geodesic between $p$ and $q$ in $\spc{X}$ will be denoted by $\geod_{[p q]}$\index{$\geod_{[{*}{*}]}$}.
We will always assume $\geod_{[p q]}$ is parametrized starting at $p$; 
that is, $\geod_{[p q]}(0)=p$ and $\geod_{[p q]}(\dist{p}{q}{})=q$.
The image of $\geod_{[p q]}$ will be denoted by $[p q]$\index{$[{*}{*}]$} and called a \emph{geodesic}\index{geodesic}.
The term \emph{geodesic}\index{geodesic} will also be used for  a linear reparametrization of a unit-speed geodesic;
when a confusion is possible we call the latter a \emph{constant-speed geodesic}\index{constantspeed geodesic}.
%???MAYBE BETTER CALL IT GEODESIC CURVE???
%%%%%%%%???MAYBE BETTER TO KEEP CONSTANTSPEED GEODESIC. IT IS CLEAR.  A GEODESIC CURVE POSSIBLY MIGHT HAVE A DIFFERENT PARAMETER. WE SHOULD NOT EVEN RAISE THIS POSSIBILITY IN THE READER'S MIND.
%%%%%%%%%%%S:  SORRY BUT I NEED OTHER PARAMETERS IN WARPED PRODUCTS.  SO CAN WE CALL THESE ``PREGEODESICS'''AS IN BRIDSON-HAEFLIGER.
With slight abuse of notation, we will use $[p q]$ also for the class of all linear reparametrizations of $\geod_{[p q]}$.

We may write $[p q]_{\spc{X}}$ 
to emphasize that the geodesic $[p q]$ is in the space  ${\spc{X}}$.
Also we use the following short-cut notation:
\begin{align*}
\l] p q \r[&=[pq]\backslash\{p,q\},
&
\l] p q \r]&=[pq]\backslash\{p\},
&
\l[ p q \r[&=[pq]\backslash\{q\}.
\end{align*}

%A curve that is a reparametrization, not necessarily linear,  of a geodesic will be called a \emph{pregeodesic}.


In general, a geodesic between $p$ and $q$ need not exist and if it exists, it need not to be unique.
However,  once we write $\geod_{[p q]}$ or $[p q]$ we mean that we fixed a choice of geodesic.

A constant-speed geodesic $\gamma\:[0,1]\to\spc{X}$ is called a \emph{geodesic path}\index{geodesic path}.
Given a geodesic $[p q]$,
we denote by $\geodpath_{[pq]}$ the corresponding geodesic path;
that is, 
$$\geodpath_{[pq]}(t)\z\equiv\geod_{[pq]}(t\cdot\dist[{{}}]{p}{q}{}).$$

A curve $\gamma\:\II\to \spc{X}$  is called a \emph{local geodesic}\index{geodesic!local geodesic}, if for any $t\in\II$ there is a neighborhood $U\ni t$ in $\II$ such that the restriction $\gamma|_U$ is a constant-speed geodesic.  If $\II=[0,1]$, then $\gamma$ is called a \emph{local geodesic path}.

\begin{thm}{Proposition}\label{prop:busemann}
Suppose $\spc{X}$ is a metric space and $\gamma\:[0,\infty)\to \spc{X}$ is a ray. 
Then the \index{Busemann function}\emph{Busemann function} $\bus_\gamma\:\spc{X}\to \RR$ 
\[\bus_\gamma(x)=\lim_{t\to\infty}\dist{\gamma(t)}{x}{}- t\eqlbl{eq:def:busemann*}\]
is defined
and $1$-Lipschitz.
\end{thm}

\parit{Proof.}
As  follows from the triangle inequality, the function \[t\mapsto\dist{\gamma(t)}{x}{}- t\] is nonincreasing in $t$.  
Clearly $\dist{\gamma(t)}{x}{}- t\ge-\dist{\gamma(0)}{x}{}$.
Thus the limit in \ref{eq:def:busemann*} is defined.
\qeds


\parbf{Triangles.}
For a triple of points $p,q,r\in \spc{X}$, a choice of triple of geodesics $([q r], [r p], [p q])$ will be called a \emph{triangle}\index{triangle}, and we will use the short notation 
$\trig p q r=([q r], [r p], [p q])$\index{$\trig {{*}}{{*}}{{*}}$}.
Again, given a triple $p,q,r\in \spc{X}$, there may be no triangle 
$\trig p q r$, simply because one of the pairs of these points cannot be joined by a geodesic.  Or there may be many different triangles, any of which can be denoted by $\trig p q r$.
Once we write $\trig p q r$, it means we have chosen such a triangle; 
that is, made a choice of each $[q r], [r p]$ and $[p q]$.

The value 
\[\dist{p}{q}{}+\dist{q}{r}{}+\dist{r}{p}{}\] 
will be called the \emph{perimeter of triangle} $\trig p q r$;
it obviously coincides with perimeter of triple $p$, $q$, $r$ as defined below.

\parbf{Hinges.}
Let $p,x,y\in \spc{X}$ be a triple of points such that $p$ is distinct from $x$ and $y$.
A pair of geodesics $([p x],[p y])$ will be called a  \emph{hinge}\index{hinge}, and will be denoted by 
$\hinge p x y=([p x],[p y])$\index{$\hinge{{*}}{{*}}{{*}}$}.

%%%%%%%%%%%%%%%%%%%%%%%%%%%%%%%%%%%%%%%%%%%%%%%%%%%%%%%%%%%%%%%%%%%%%%%%%%%%%%%%%%%%%%











\section{Model angles and triangles}\label{sec:mod-tri/angles}

Let $\spc{X}$ be a metric space, 
$p,q,r\in \spc{X}$ 
and $\kappa\in\RR$. 
Let us define the \emph{model triangle}\index{model triangle} $\trig{\~p}{\~q}{\~r}$ 
(briefly, 
$\trig{\~p}{\~q}{\~r}=\modtrig\kappa(p q r)$%
\index{$\modtrig\kappa$!$\modtrig\kappa({*}{*}{*})$}) to be a triangle in the model plane $\Lob2\kappa$ such that
\[\dist{\~p}{\~q}{}=\dist{p}{q}{},
\ \ \dist{\~q}{\~r}{}=\dist{q}{r}{},
\ \ \dist{\~r}{\~p}{}=\dist{r}{p}{}.\]

In the notation of Section~\ref{model}, 
$\modtrig\kappa(p q r)=\modtrig\kappa\{\dist{q}{r}{},\dist{r}{p}{},\dist{p}{q}{}\}$.

If $\kappa\le 0$, the  model triangle is  always defined, that is, exists and is unique up to isometry of $\Lob2\kappa$.
If $\kappa>0$, the model triangle is said to be defined if in addition
\[\dist{p}{q}{}+\dist{q}{r}{}+\dist{r}{p}{}< 2\cdot\varpi\kappa.\]
In this case it also exists and is unique up to isometry of $\Lob2\kappa$.
The value $\dist{p}{q}{}+\dist{q}{r}{}+\dist{r}{p}{}$ will be called the  \emph{perimeter of triple} $p$, $q$, $r$.

If for  $p,q,r\in \spc{X}$,
$\trig{\~p}{\~q}{\~r}=\modtrig\kappa(p q r)$ is defined 
and $\dist{p}{q}{},\dist{p}{r}{}>0$, the angle measure of 
$\trig{\~p}{\~q}{\~r}$ at $\~ p$ will be called the \emph{model angle} of triple $p$, $q$, $r$, and will be denoted by
$\angk\kappa p q r$%
\index{$\tangle\mc\kappa$!$\angk\kappa{{*}}{{*}}{{*}}$}.

In the notation of Section~\ref{model}, 
\[\angk\kappa p q r=\tangle\mc\kappa\{\dist{q}{r}{};\dist{p}{q}{},\dist{p}{r}{}\}.\]

\begin{wrapfigure}[6]{r}{28mm}
\begin{lpic}[t(-0mm),b(6mm),r(0mm),l(0mm)]{pics/lem_alex1(1)}
\lbl[lb]{10,23;$x$}
\lbl[rt]{1.5,.5;$p$}
\lbl[bl]{25,7.5;$y$}
\lbl[lb]{17,15;$z$}
\end{lpic}
\end{wrapfigure}

\begin{thm}{Alexandrov's lemma}
\index{Alexandrov's lemma}
\index{lemma!Alexandrov's lemma}
\label{lem:alex}  
Let $p,q,r,z$ be distinct points in a metric space such that $z\in \l]p r\r[$ and 
\[\dist{p}{q}{}+\dist{q}{r}{}+\dist{r}{p}{}< 2\cdot\varpi\kappa.\]
Then 
the following expressions have the same sign:
\begin{subthm}{lem-alex-difference}
$\angk\kappa p q z
-\angk\kappa p q r$,
\end{subthm} 

\begin{subthm}{lem-alex-angle}
$\angk\kappa z q p
+\angk\kappa z q r -\pi$.
\end{subthm}

Moreover,
\[\angk\kappa q p r \ge \angk\kappa q p z +  (see Section \ref{sec:Euclid-construction}) \angk\kappa q z r,\]
with equality if and only if the expressions in (\ref{SHORT.lem-alex-difference}) and (\ref{SHORT.lem-alex-angle}) vanish.
\end{thm}

\parit{Proof.} By the triangle inequality, 
\[
\dist{p}{q}{}+\dist{q}{z}{}+\dist{z}{p}{}\le \dist{p}{q}{}+\dist{q}{r}{}+\dist{r}{p}{}< 2\cdot\varpi\kappa.
\]
Therefore the model triangle $\trig{\~p}{\~q}{\~z}=\modtrig\kappa p q z$ is defined.
Take 
a point $\~r$ on the extension of 
$[\~p \~z]$ beyond $\~z$ so that $\dist{\~p}{\~r}{}=\dist{p}{r}{}$ (and therefore $\dist{\~p}{\~z}{}=\dist{p}{z}{}$). 
 
From monotonicity of function $a\mapsto\tangle\mc\kappa\{a;b,c\}$ (\ref{increase}), 
the following expressions have the same sign:
\begin{enumerate}[(i)]
\item $\mangle\hinge{\~p}{\~q}{\~r}-\angk\kappa{p}{q}{r}$;
\item $\dist{\~p}{\~r}{}-\dist{p}{r}{}$;
\item $\mangle\hinge{\~z}{\~q}{\~r}-\angk\kappa{z}{q}{r}$.
\end{enumerate}
Since 
\[\mangle\hinge{\~p}{\~q}{\~r}=\mangle\hinge{\~p}{\~q}{\~z}=\angk\kappa{p}{q}{z}\]
and
\[ \mangle\hinge{\~z}{\~q}{\~r}
=\pi-\mangle\hinge{\~z}{\~p}{\~q}
=\pi-\angk\kappa{z}{p}{q},\]
the first statement follows.

For the second statement, construct $\trig{\~q}{\~z}{r'}=\modtrig\kappa q z r$ on the opposite side of $[\~q\~z]$ from $\trig{\~p}{\~q}{\~z}$.  
Since
\[\dist{\~p}{r'}{}\le \dist{\~p}{\~z}{} + \dist{\~z}{r'}{}=\dist{p}{z}{}+\dist{z}{r}{}=\dist{p}{r}{},\]
then 
\begin{align*}
\angk\kappa{q}{p}{z} + \angk\kappa{q}{z}{r} 
&
= 
\mangle\hinge{\~q}{\~p}{\~z}+ \mangle\hinge{\~q}{\~z}{r'} 
=
\\
&
= 
\mangle\hinge{\~q}{\~p}{r'}
\le
\\
&\le  \angk\kappa q p r.
\end{align*}
Equality holds if and only  if $\dist{\~p}{r'}{}=\dist{p}{r}{}$, 
as required.
\qeds

%%%%%%%%%%%%%%%%%%%%%%%%%%%%%%%%%%%%%%%%%%%%%%%%%%%%%%%%%%%%%%%%%%%%%%%%%

\section{Angles and the first variation}\label{sec:angles}

Given a hinge $\hinge p x y$, we define its \emph{angle}\index{angle} to be \index{$\mangle$!$\mangle\hinge{{*}}{{*}}{{*}}$}
\[\mangle\hinge p x y
\df
\lim_{\bar x,\bar y\to p} \angk\kappa p{\bar x}{\bar y},\eqlbl{eq:angle-def}\]
for $\bar x\in\l]p x\r]$ and $\bar y\in\l]p y\r]$, if this limit exists.

Similarly to $\angk\kappa p{x}{y}$, 
we will use the short notation\index{$\side\kappa$!$\side\kappa \hinge{{*}}{{*}}{{*}}$}
\[\side\kappa \hinge p x y=
\side\kappa \l\{\mangle\hinge p x y;\dist{p}{x}{},\dist{p}{y}{}\r\},\]
where the right hand side is defined on page~\pageref{page:model-side}.
The value $\side\kappa \hinge p x y$ will be called the  \emph{model side}
 of hinge $\hinge p x y$.

\begin{thm}{Lemma}\label{lem:k-K-angle}
For any $\kappa,\Kappa\in\RR$, there exists  $\Const\in\RR$ such that
\[|\angk\Kappa p{x}{y}-\angk\kappa p{x}{y}|
\le 
\Const\cdot\dist[{{}}]{p}{x}{}\cdot\dist[{{}}]{p}{y}{},
\eqlbl{eq:k-K}\]
whenever  the lefthand side is defined.
\end{thm}

Lemma~\ref{lem:k-K-angle} implies that 
the definition of angle is independent of $\kappa$.
In particular, one can take $\kappa=0$ in \ref{eq:angle-def};
thus the angle can be calculated from the  cosine law:
\[\cos\angk{0}{p}{x}{y}
=
\frac{\dist[2]{p}{x}{}+\dist[2]{p}{y}{}-\dist[2]{x}{y}{}}{2\cdot \dist[{{}}]{p}{x}{}\cdot\dist[{{}}]{p}{y}{}}.\]

\parit{Proof.}
The function $\kappa\mapsto \angk\kappa p{x}{y}$ is nondecreasing (see \ref{k-decrease}).
Thus, for $\Kappa>\kappa$, we have
\begin{align*}
0\le \angk\Kappa p{x}{y}-\angk{\kappa}p{x}{y}
\le& \angk\Kappa p{x}{y}+\angk\Kappa {x}p{y}+\angk\Kappa {y}p{x}-
\\
&-\angk\kappa p{x}{y}-\angk\kappa {x}p{y}-\angk\kappa {y}p{x}
= 
\\
=&\Kappa\cdot\area\modtrig\Kappa(pxy)-\kappa\cdot\area\modtrig\kappa(pxy).
\end{align*}
Thus, \ref{eq:k-K} follows since 
%???WHY???
\[0
\le
\area\modtrig\kappa(pxy)\le \area\modtrig\Kappa(pxy),
%\le
%O\l(\dist[{{}}]{p}{x}{}\cdot\dist[{{}}]{p}{y}{}\r),
\]
%!!!
where the latter area is at most $\dist[{{}}]{p}{x}{}\cdot\dist[{{}}]{p}{y}{}$.
%???It is true without assuming $K\le 0$??? if $K\le 0$ and hence, by central projection from a sphere to its tangent plane, is at most $\tan\dist[{{}}]{p}{x}{}\cdot\tan\dist[{{}}]{p}{y}{}$ if $K=1$.
\qeds



\begin{thm}{Triangle inequality for angles}
\label{claim:angle-3angle-inq}
Let  $[px^1]$, $[px^2]$ and $[px^3]$ %$\gamma^1, \gamma^2, \gamma^3$ 
be three geodesics in a metric space.
If all of the angles $\alpha^{i j}=\mangle\hinge p {x^i}{x^j}$ are defined then they satisfy the triangle inequality:
\[\alpha^{13}\le \alpha^{12}+\alpha^{23}.\]

\end{thm}


\parit{Proof.}
Since $\alpha^{13}\le\pi$, we can assume that $\alpha^{12}+\alpha^{23}< \pi$.
Set $\gamma^i\z=\geod_{[px^i]}$.
Given any $\eps>0$, for all sufficiently small $t,\tau,s\in\RR_+$ we have
\begin{align*}
\dist{\gamma^1(t)}{\gamma^3(\tau)}{}
\le 
\ &\dist{\gamma^1(t)}{\gamma^2(s)}{}+\dist{\gamma^2(s)}{\gamma^3(\tau)}{}<\\
<
\ &\sqrt{t^2+s^2-2\cdot t\cdot  s\cdot \cos(\alpha^{12}+\eps)}\ +
\\
&+\sqrt{s^2+\tau^2-2\cdot s\cdot \tau\cdot \cos(\alpha^{23}+\eps)}\le
\\
\intertext{Below we define 
$s(t,\tau)$ so that for 
$s=s(t,\tau)$, this chain of inequalities can be continued as follows:}
\le
\ &\sqrt{t^2+\tau^2-2\cdot t\cdot \tau\cdot \cos(\alpha^{12}+\alpha^{23}+2\cdot \eps)}.
\end{align*}

\begin{wrapfigure}{r}{25mm}
\begin{lpic}[t(-4mm),b(-0mm),r(0mm),l(0mm)]{pics/s-choice(1)}
\lbl[w]{12,35,50;$\,t\,$}
\lbl[w]{12,12,-50;$\,\tau\,$}
\lbl[w]{14,22.5,4;$\,s\,$}
\lbl[lw]{9,20,-25;{$=\alpha^{12}+\eps$}}
\lbl[lw]{9,26,28;{$=\alpha^{23}+\eps$}}
\end{lpic}
\end{wrapfigure}

Thus for any $\eps>0$, 
\[\alpha^{13}\le \alpha^{12}+\alpha^{23}+2\cdot \eps.\]
Hence the result follows.

To define $s(t,\tau)$, consider three rays $\~\gamma^1$, $\~\gamma^2$, $\~\gamma^3$ on a Euclidean plane starting at one point, such that $\mangle(\~\gamma^1,\~\gamma^2)=\alpha^{12}+\eps$, $\mangle(\~\gamma^2,\~\gamma^3)=\alpha^{23}+\eps$ and $\mangle(\~\gamma^1,\~\gamma^3)=\alpha^{12}+\alpha^{23}+2\cdot \eps$.
We parametrize each ray by the distance from the starting point.
Given two positive numbers $t,\tau\in\RR_+$, let $s=s(t,\tau)$ be 
the number such that 
$\~\gamma^2(s)\in[\~\gamma^1(t)\ \~\gamma^3(\tau)]$. 
Clearly $s\le\max\{t,\tau\}$, so $t,\tau,s$ may be taken sufficiently small.
\qeds 

\begin{thm}{Exercise}
Prove that the sum of adjacent angles is at least $\pi$.

More precisely: let $\spc{X}$ be a complete length space and $p,x,y,z\in \spc{X}$.
If $p\in \l] x y \r[$, then 
\[\mangle\hinge pxz+\mangle\hinge pyz\ge \pi\]
whenever  each angle on the left-hand side is defined.
\end{thm}


\begin{thm}{First variation inequality}\label{lem:first-var}
Assume for hinge $\hinge q p x$ 
the angle $\alpha=\mangle\hinge q p x$ is defined then
\[\dist{p}{\geod_{[qx]}(t)}{}
\le
\dist{q}{p}{}-t\cdot \cos\alpha+o(t).\]

\end{thm}

\parit{Proof.} Take sufficiently small $\eps>0$.
For all sufficiently small $t>0$, we have that 
\begin{align*}
 \dist{\geod_{[qp]}(t/\eps)}{\geod_{[qx]}(t)}{}
&\le 
\tfrac{t}{\eps}\cdot \sqrt{1+\eps^2 -2\cdot \eps\cdot \cos\alpha}+o(t)\le
\\
&\le \tfrac{t}{\eps} -t\cdot \cos\alpha + t\cdot \eps.
\end{align*}
Applying triangle inequality, we get 
\begin{align*}
\dist{p}{\geod_{[qx]}(t)}{}
&\le \dist{p}{\geod_{[qp]}(t/\eps)}{}+\dist{\geod_{[qp]}(t/\eps)}{\geod_{[qx]}(t)}{}
\le 
\\
&\le
\dist{p}{q}{} -t\cdot \cos\alpha + t\cdot \eps
\end{align*}
for any $\eps>0$ and all sufficiently small $t$.
Hence the result.
\qeds


\section{Constructions}
\label{sec:Euclid-constructions}


\parbf{Cone.}
The \index{cone}\emph{
Euclidean cone} $\spc{Y}=\Cone\spc{X}$ 
over a metric space $\spc{X}$
is defined as the metric space whose underlying set consists of
equivalence classes in
$[0,\infty)\times \spc{X}$ with the equivalence relation ``$\sim$'' given by $(0,p)\sim (0,q)$ for any points $p,q\in\spc{X}$,
and whose metric is given by the cosine rule
\[
\dist{(s,p)}{(t,q)}{\spc{Y}} 
=
\sqrt{s^2+t^2-2\cdot s\cdot t\cdot \cos\alpha},
\]
where $\alpha= \min\{\pi, \dist{p}{q}{\spc{X}}\}$.

The point in  $\Cone{\spc{X}}$ formed by the equivalence class of $\{0\}\times\spc{X}$ is called the \index{tip of the cone}\emph{tip of the cone} and is denoted by $0$ or $0_{\spc{Y}}$.
The distance $\dist{0}{v}{\spc{Y}}$ is called the norm of $v$ and is denoted by $|v|$ or $|v|_{\spc{Y}}$.

\section{Space of directions and tangent space}
\label{sec:tangent-space+directions}

Let $\spc{X}$ be a metric space.
If the angle $\mangle\hinge pxy$ is defined for any hinge $\hinge pxy$ in $\spc{X}$,
then we will say that the space $\spc{X}$ has \index{defined angles}\emph{defined angles}.

%We shall see that, according to 
%the corollaries \ref{cor:monoton:sup} 
%and \ref{cor:monoton-cba:angle=inf},
%$\Alex{}$ and $\CAT{}$ spaces have defined angles.
 
Let $\spc{X}$ be a space with defined angles. For $p\in \spc{X}$,
consider the set $\mathfrak{S}_p$ 
of all nontrivial unit-speed geodesics starting at $p$.
By \ref{claim:angle-3angle-inq}, the triangle inequality holds for $\mangle$ on $\mathfrak{S}_p$,
that is, $(\mathfrak{S}_p,\mangle)$ 
forms a pseudometric space.

The metric space corresponding to  $(\mathfrak{S}_p,\mangle)$ is called the \emph{space of geodesic directions} at $p$, denoted by $\Sigma'_p$ or $\Sigma'_p\spc{X}$.
The elements of $\Sigma'_p$ are called \emph{geodesic directions} at $p$.
Each geodesic direction is formed by an equivalence class of geodesics starting from $p$ 
for the equivalence relation 
\[[px]\sim[py]\ \ \iff\ \ \mangle\hinge pxy=0.\]



The completion of $\Sigma'_p$ is called \emph{space of directions} at $p$ and is denoted as $\Sigma_p$ or $\Sigma_p\spc{X}$.
The elements of $\Sigma_p$ are called \emph{directions} at $p$.

The Euclidean cone $\Cone\Sigma_p$ over the space of directions $\Sigma_p$ is called \emph{tangent space} at  $p$ and denoted by $\T_p$ or $\T_p\spc{X}$.

The tangent space $\T_p$ could be also defined directly, without introducing the space of direction.
To do so, consider the set $\mathfrak{T}_p$ of all geodesics starting at $p$, with arbitrary speed.
Given $\alpha,\beta\in \mathfrak{T}_p$,
set 
\[\dist{\alpha}{\beta}{\mathfrak{T}_p}
=
\lim_{\eps\to0} 
\frac{\dist{\alpha(\eps)}{\beta(\eps)}{\spc{X}}}\eps
\eqlbl{eq:dist-in-T_p}.\]
If the angles in $\spc{X}$ are defined, then so is
the limit in \ref{eq:dist-in-T_p}.
In this case it defines a pseudometric on $\mathfrak{T}_p$.


The corresponding metric space admits a natuaral isometric identification with the cone $\T'_p=\Cone\Sigma'_p$.
The elements of $\T'_p$ are formed by the equivalence classes for the relation 
\[\alpha\sim\beta\ \ \iff\ \ \dist{\alpha(t)}{\beta(t)}{\spc{X}}=o(t).\]
The completion of $\T'_p$ is therefore naturally isometric to $\T_p$.

The elements of $\T_p$ will be called \index{tangent vector}\emph{tangent vectors} at $p$,
despite that $\T_p$ is only a cone --- not a vector space.
The elements of $\T'_p$ will be called \index{geodesic tangent vector}\emph{geodesic tangent vectors} at $p$.

\section{Remarks}
\label{page:upper-angle}
In Alexandrov geometry, angles defined as in Section \ref{sec:angles} always exist (see Theorem~\ref{angle} and Corollary~\ref{cor:monoton-cba:angle=inf}).

For general metric spaces, this angle may not exist, \index{$\mangle$!$\mangle^\text{up}$}
and it is more natural to consider the \emph{upper angle}\index{angle!upper angle}  defined as
\[\mangle^\text{up}\hinge p x y
\df
\limsup_{\bar x,\bar y\to p} \angk\kappa p{\bar x}{\bar y},\]
where $\bar x\in\l]p x\r]$ and $\bar y\in\l]p y\r]$.
The triangle inequality (\ref{claim:angle-3angle-inq}) holds for upper angles as well.

Euclidean cones are the simplest examples of warped products (see Chapter \ref{}). 