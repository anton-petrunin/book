%%!TEX root =the-euclid.tex
\chapter{The ghost of Euclid%ready
}

\section{Geodesics, triangles and hinges}
\label{sec:geods}

\parbf{Geodesics and their relatives.}
Let $\spc{X}$ be a metric space 
and $\II$\index{$\II$} be a real interval. 
A globally isometric map $\gamma\:\II\to \spc{X}$ is called a \emph{unit-speed geodesic}\index{unit-speed geodesic}%
\footnote{Various authors call it differently: \emph{shortest path}, \emph{minimizing geodesic}.}; 
in other words, $\gamma\:\II\to \spc{X}$ is a unit-speed geodesic if 
\[\dist{\gamma(s)}{\gamma(t)}{\spc{X}}=|s-t|\]
for any pair $s,t\in \II$.

A unit-speed geodesic $\gamma\:\RR_{\ge0}\to \spc{X}$ is called a \emph{ray}\index{ray}.

A unit-speed geodesic  $\gamma\:\RR\to \spc{X}$ is called a \emph{line}\index{line}.

%geodesic between points is not defined???
A unit-speed geodesic between $p$ and $q$ in $\spc{X}$ will be denoted by $\geod_{[p q]}$\index{$\geod_{[{*}{*}]}$}.
We assume $\geod_{[p q]}$ is parametrized starting at $p$; 
i.e. $\geod_{[p q]}(0)=p$ and $\geod_{[p q]}(\dist{p}{q}{})=q$.
The image of $\geod_{[p q]}$ will be denoted by $[p q]$\index{$[{*}{*}]$} and called a \emph{geodesic}\index{geodesic}.
The term \emph{geodesic}\index{geodesic} will also be used for  a linear reparametrization of a unit-speed geodesic;
when a confusion is possible we call the latter a \emph{constant-speed geodesic}\index{constantspeed geodesic}.
%???MAYBE BETTER CALL IT GEODESIC CURVE???
%%%%%%%%???MAYBE BETTER TO KEEP CONSTANTSPEED GEODESIC. IT IS CLEAR.  A GEODESIC CURVE POSSIBLY MIGHT HAVE A DIFFERENT PARAMETER. WE SHOULD NOT EVEN RAISE THIS POSSIBILITY IN THE READER'S MIND.
%%%%%%%%%%%S:  SORRY BUT I NEED OTHER PARAMETERS IN WARPED PRODUCTS.  SO CAN WE CALL THESE ``PREGEODESICS'''AS IN BRIDSON-HAEFLIGER.
With slight abuse of notation, we will use $[p q]$ also for the class of all linear reparametrizations of $\geod_{[p q]}$.

We may write $[p q]_{\spc{X}}$ 
to emphasize that the geodesic $[p q]$ is in the space  ${\spc{X}}$.
Also we use the following short-cut notation:
\begin{align*}
\l] p q \r[&=[pq]\backslash\{p,q\},
&
\l] p q \r]&=[pq]\backslash\{p\},
&
\l[ p q \r[&=[pq]\backslash\{q\}.
\end{align*}

A curve that is a reparametrization, not necessarily linear,  of a geodesic will be called a \emph{pregeodesic}.


In general, a geodesic between $p$ and $q$ need not exist and if it exists, it need not to be unique.  However,  once we write $\geod_{[p q]}$ or $[p q]$ we mean that we made a choice of geodesic.

A constant-speed geodesic $\gamma\:[0,1]\to\spc{X}$ is called a \emph{geodesic path}\index{geodesic path}.
Given a geodesic $[p q]$,
we denote by $\geodpath_{[pq]}$ the corresponding geodesic path;
i.e., 
$$\geodpath_{[pq]}(t)\z\equiv\geod_{[pq]}(t\cdot\dist[{{}}]{p}{q}{}).$$



A curve $\gamma\:\II\to \spc{X}$  is called a \emph{local geodesic}\index{geodesic!local geodesic}, if for any $t\in\II$ there is a neighborhood $U\ni t$ in $\II$ such that the restriction $\gamma|_U$ is a constant-speed geodesic.  If $\II=[0,1]$, then $\gamma$ is called a \emph{local geodesic path}.

\parbf{Triangles.}
For a triple of points $p,q,r\in \spc{X}$, a choice of triple of geodesics $([q r], [r p], [p q])$ will be called a \emph{triangle}\index{triangle} and we will use short notation 
$\trig p q r=([q r], [r p], [p q])$\index{$\trig {{*}}{{*}}{{*}}$}.
Again given a triple $p,q,r\in \spc{X}$ it may be no triangle 
$\trig p q r$ simply because one of the pairs of these points can not be joined by a geodesic, and also it maybe many different triangles with these vertexes, any of which can be denoted by $\trig p q r$.
Once we write $\trig p q r$, it means we made a choice of such a triangle, 
i.e. a choice of each $[q r], [r p]$ and $[p q]$.
The value $\dist{p}{q}{}+\dist{q}{r}{}+\dist{r}{p}{}$ will be called \emph{perimeter of triangle} $\trig p q r$;
it obviousely coinsides with perimeter of triple $p$, $q$, $r$ defined below.

\parbf{Hinges.}
Let $p,x,y\in \spc{X}$ be a triple of points such that $p$ is distinct from $x$ and $y$.
A pair geodesics $([p x],[p y])$ will be called \emph{hinge}\index{hinge} and briefly, it will be denoted by 
$\hinge p x y=([p x],[p y])$\index{$\hinge{{*}}{{*}}{{*}}$}.


%%%%%%%%%%%%%%%%%%%%%%%%%%%%%%%%%%%%%%%%%%%%%%%%%%%%%%%%%%%%%%%%%%%%%%%%%%%%%%%%%%%%%%
\section{Model triangles and angles.}\label{sec:mod-tri/angles}

Let $\spc{X}$ be a metric space, 
$p,q,r\in \spc{X}$ 
and $\kappa\in\RR$. 
Let us define its \emph{model triangle}\index{model triangle} $\trig{\~p}{\~q}{\~r}$ 
(briefly, 
$\trig{\~p}{\~q}{\~r}=\modtrig\kappa(p q r)$%
\index{$\modtrig\kappa$!$\modtrig\kappa({*}{*}{*})$}) to be a triangle in the model plane $\Lob2\kappa$ such that
\[\dist{\~p}{\~q}{}=\dist{p}{q}{},
\ \ \dist{\~q}{\~r}{}=\dist{q}{r}{},
\ \ \dist{\~r}{\~p}{}=\dist{r}{p}{}.\]

In the notation of Section~\ref{model}, 
$\modtrig\kappa(p q r)=\modtrig\kappa\{\dist{q}{r}{},\dist{r}{p}{},\dist{p}{q}{}\}$.

If $\kappa\le 0$ model triangle allways defined, it exists and unique up to isometry of $\Lob2\kappa$.
If $\kappa>0$, the model triangle is said to be defined if in addition
\[\dist{p}{q}{}+\dist{q}{r}{}+\dist{r}{p}{}< 2\cdot\varpi\kappa.\]
In this case it also exists and unique up to isometry of $\Lob2\kappa$.
The value $\dist{p}{q}{}+\dist{q}{r}{}+\dist{r}{p}{}$ will be called \emph{perimeter of triple} $p$, $q$, $r$.

If for  $p,q,r\in \spc{X}$,
$\trig{\~p}{\~q}{\~r}=\modtrig\kappa(p q r)$ is defined 
and $\dist{p}{q}{},\dist{p}{r}{}>0$, the angle measure of 
$\trig{\~p}{\~q}{\~r}$ at $\~ p$ will be called \emph{model angle} of triple $p$, $q$, $r$ and it will be denoted by
$\angk\kappa p q r$%
\index{$\tangle\mc\kappa$!$\angk\kappa{{*}}{{*}}{{*}}$}.

In the notation of Section~\ref{model}, 
$\angk\kappa p q r=\tangle\mc\kappa\{\dist{q}{r}{};\dist{p}{q}{},\dist{p}{r}{}\}$.

\begin{wrapfigure}[10]{r}{35mm}
\begin{lpic}[t(0mm),b(-10mm),r(0mm),l(0mm)]{pics/lem_alex1(0.4)}
\lbl[br]{17,59;$p$}
\lbl[r]{1,2;$q$}
\lbl[l]{86,13;$r$}
\lbl[lb]{67,32;$z$}
\end{lpic}
\end{wrapfigure}

\begin{thm}{Alexandrov's lemma}
\index{Alexandrov's lemma}
\index{lemma!Alexandrov's lemma}
\label{lem:alex}  
Let $p,q,r,z$ be distinct points in a metric space such that $z\in \l]p r\r[$ and 
\[\dist{p}{q}{}+\dist{q}{r}{}+\dist{r}{p}{}< 2\cdot\varpi\kappa.\]
Then 
the following expressions have the same sign:
\begin{subthm}{lem-alex-difference}
$\angk\kappa p q z
-\angk\kappa p q r$,
\end{subthm} 

\begin{subthm}{lem-alex-angle}
$\angk\kappa z q p
+\angk\kappa z q r -\pi$.
\end{subthm}

Moreover,
\[\angk\kappa q p r \ge \angk\kappa q p z +  \angk\kappa q z r,\]
with equality if and only if the expressions in (\ref{SHORT.lem-alex-difference}) and (\ref{SHORT.lem-alex-angle}) vanish.
\end{thm}

\parit{Proof.} By the triangle inequality, 
\[
\dist{p}{q}{}+\dist{q}{z}{}+\dist{z}{p}{}\le \dist{p}{q}{}+\dist{q}{r}{}+\dist{r}{p}{}< 2\cdot\varpi\kappa.
\]
Therefore the model triangle $\trig{\~p}{\~q}{\~z}=\modtrig\kappa p q z$ is defined.
Take 
a point $\~r$ on the extension of 
$[\~p \~z]$ beyond $\~z$ so that $\dist{\~p}{\~r}{}=\dist{p}{r}{}$ (and therefore $\dist{\~p}{\~z}{}=\dist{p}{z}{}$). 
 
From monotonicity of function $a\mapsto\tangle\mc\kappa\{a;b,c\}$ (\ref{increase}), 
the following expressions have the same sign:
\begin{enumerate}[(i)]
\item $\mangle\hinge{\~p}{\~q}{\~r}-\angk\kappa{p}{q}{r}$;
\item $\dist{\~p}{\~r}{}-\dist{p}{r}{}$;
\item $\mangle\hinge{\~z}{\~q}{\~r}-\angk\kappa{z}{q}{r}$.
\end{enumerate}
Since 
\[\mangle\hinge{\~p}{\~q}{\~r}=\mangle\hinge{\~p}{\~q}{\~z}=\angk\kappa{p}{q}{z}\]
and
\[ \mangle\hinge{\~z}{\~q}{\~r}
=\pi-\mangle\hinge{\~z}{\~p}{\~q}
=\pi-\angk\kappa{z}{p}{q},\]
the first statement follows.

For the second statement, construct $\trig{\~q}{\~z}{r'}=\modtrig\kappa q z r$ on the opposite side of $[\~q\~z]$ from $\trig{\~p}{\~q}{\~z}$.  
Since
\[\dist{\~p}{r'}{}\le \dist{\~p}{\~z}{} + \dist{\~z}{r'}{}=\dist{p}{z}{}+\dist{z}{r}{}=\dist{p}{r}{},\]
then 
\begin{align*}
\angk\kappa{q}{p}{z} + \angk\kappa{q}{z}{r} 
&
= 
\mangle\hinge{\~q}{\~p}{\~z}+ \mangle\hinge{\~q}{\~z}{r'} 
=
\\
&
= 
\mangle\hinge{\~q}{\~p}{r'}
\le
\\
&\le  \angk\kappa q p r.
\end{align*}
Equality holds if and only  if $\dist{\~p}{r'}{}=\dist{p}{r}{}$, 
as required.
\qeds








\section{Quadruples}\label{sec:quad}

Let $\spc{X}$ be a metric space.
Consider quadruple of points in $\spc{X}$;
say $\{x^1,x^2,x^3,x^4\}$.
The points $x^i$ will be called \emph{vertices}\index{vertex of the quadruple} of the quadruple.

The quadruple $\{x^1,x^2,x^3,x^4\}$
is called \emph{$\kappa$-admissible}\index{$\kappa$-admissible quadruple} if all 4 model triangles 
$\modtrig\kappa(x^ix^jx^\kay)$ are defined;
otherwise it is called \emph{$\kappa$-inadmissible}\index{$\kappa$-inadmissible quadruple}.

A $\kappa$-admissible quadruple $\{x^1,x^2,x^3,x^4\}$
is called \emph{$\kappa$-model}\index{$\kappa$-model quadruple}
if it is isometric to a quadruple in $\Lob3\kappa$.
The later is equivalent to the fact that 
for one (and therefore any) permutation $(i,j,\kay,\ell)$
of $(1,2,3,4)$, 
we have that the three angles $\angk\kappa{x^i}{x^j}{x^\kay}$,
$\angk\kappa{x^i}{x^\kay}{x^\ell}$ and $\angk\kappa{x^i}{x^j}{x^\ell}$
satisfy all three triangle inequalities 
and
\[\angk\kappa{x^i}{x^j}{x^\kay}+\angk\kappa{x^i}{x^\kay}{x^\ell}+\angk\kappa{x^i}{x^j}{x^\ell}
\le 
2\cdot\pi.\]

Assume $\{x^1,x^2,x^3,x^4\}$ is $\kappa$-admissible.
If 
\[\angk\kappa{x^i}{x^j}{x^\kay}+\angk\kappa{x^i}{x^\kay}{x^\ell}+\angk\kappa{x^i}{x^j}{x^\ell}
\le 
2\cdot\pi\]
for any permutation $(i,j,\kay,\ell)$
of $(1,2,3,4)$, we call 
%it 
$\{x^1,x^2,x^3,x^4\}$ 
\emph{$\kappa$-supermodel}\index{$\kappa$-supermodel quadruple}.
%Note that any $\kappa$-supermodel quadruple is $\kappa$-model.

A $\kappa$-admissible quadruple $\{x^1,x^2,x^3,x^4\}$
is called \emph{$\kappa$-submodel} if it is $\kappa$-model
or for some permutation $(i,j,\kay,\ell)$
of $(1,2,3,4)$ we have 
\[\angk\kappa{x^i}{x^j}{x^\kay}+\angk\kappa{x^i}{x^\kay}{x^\ell}+\angk\kappa{x^i}{x^j}{x^\ell}
>
2\cdot\pi.\eqlbl{eq:>2pi}\]
%%S
In other words, 
$\kappa$-submodels 
consist of all $\kappa$-models and all $\kappa$-admissibles that are not $\kappa$-supermodels, and likewise $\kappa$-supermodels 
consist of all $\kappa$-models and all $\kappa$-admissibles that are not $\kappa$-submodels.

Since \ref{eq:>2pi} forces the three angles $\angk\kappa{x^i}{x^j}{x^\kay}$,
$\angk\kappa{x^i}{x^\kay}{x^\ell}$ and $\angk\kappa{x^i}{x^j}{x^\ell}$ to 
satisfy all three triangle inequalities, then $\kappa$-submodels also may be described as all $\kappa$-admissibles $\{x^1,x^2,x^3,x^4\}$ such that for some permutation $(i,j,\kay,\ell)$ of $(1,2,3,4)$, the three angles $\angk\kappa{x^i}{x^j}{x^\kay}$,
$\angk\kappa{x^i}{x^\kay}{x^\ell}$ and $\angk\kappa{x^i}{x^j}{x^\ell}$
satisfy all three triangle inequalities. 

The following proposition gives yet another description of $\kappa$-submodel.

\begin{thm}{Proposition}\label{prop:submodel}
A $\kappa$-admissible 
quadruple is $\kappa$-submodel 
if and only if one of the following two equivalent conditions holds.

\begin{subthm}{}
For any labeling of the vertices 
by $(p^1,p^2,x^1,x^2)$, at least one of the following two inequalities holds
\begin{enumerate}[(i)]
\item $\angk{\kappa}{p^1}{x^1}{x^2} 
\le 
\angk{\kappa}{p^1}{p^2}{x^1}+\angk{\kappa}{p^1}{p^2}{x^2}$
\item $\angk{\kappa} {p^2}{x^1}{x^2}\le \angk{\kappa} {p^2}{p^1}{x^1} + \angk{\kappa} {p^2}{p^1}{x^2}$.
\end{enumerate}
\end{subthm}

\begin{subthm}{}
There is some labeling of the vertices by $(p,x^1,x^2,x^3)$ so that the three angles 
$\angk\kappa p{x^1}{x^2}$,
$\angk\kappa p{x^2}{x^3}$ and
$\angk\kappa p{x^1}{x^3}$
satisfy all three triangle inequalities.
\end{subthm}

\end{thm}


\parit{Proof.}
Follows from Overlap lemma (\ref{lem:extend-overlap}).
\qeds

%%%%%%%%%%%%%%%%%%%%%%%%%%%%%%%%%%%%%%%%%%%%%%%%%%%%%%%%%%%%%%%%%%%%%%%%%

\section{Angles and the first variation.}\label{sec:angles}

Given a hinge $\hinge p x y$, we define its \emph{angle}\index{angle} as 
follows:\index{$\mangle$!$\mangle\hinge{{*}}{{*}}{{*}}$}
\[\mangle\hinge p x y
\df
\lim_{\bar x,\bar y\to p} \angk\kappa p{\bar x}{\bar y},\]
where $\bar x\in\l]p x\r]$ and $\bar y\in\l]p y\r]$.

Similarly to $\angk\kappa p{x}{y}$, 
we will use short notation\index{$\side\kappa$!$\side\kappa \hinge{{*}}{{*}}{{*}}$}
\[\side\kappa \hinge p x y=
\side\kappa \l\{\mangle\hinge p x y;\dist{p}{x}{},\dist{p}{y}{}\r\}.\]
The value $\side\kappa \hinge p x y$ will be called the  \emph{model side}
 of hinge $\hinge p x y$.�

\parbf{Remark.}
We stick to this definition since in Alexandrov's geometry angles always defined (see Theorem~\ref{angle} and Corollary~\ref{cor:monoton-cba:angle=inf}).
For general metric spaces, the angle need not be defined\index{$\mangle$!$\mangle^\text{up}$}
and it is more natural to consider \emph{upper angle}\index{angle!upper angle} which is defined as
\[\mangle^\text{up}\hinge p x y
\df
\limsup_{\bar x,\bar y\to p} \angk\kappa p{\bar x}{\bar y},\]
where $\bar x\in\l]p x\r]$ and $\bar y\in\l]p y\r]$.
A good discussion of different definition of angles is given in ???.


\begin{thm}{Lemma}\label{lem:k-K-angle}
For any $\kappa,\Kappa\in\RR$, there exists  $\Const\in\RR$ such that
\[|\angk\Kappa p{x}{y}-\angk\kappa p{x}{y}|
\le 
\Const\cdot\dist[{{}}]{p}{x}{}\cdot\dist[{{}}]{p}{y}{},
\eqlbl{eq:k-K}\]
whenever  the lefthand side is defined.
\end{thm}

Lemma~\ref{lem:k-K-angle} implies that 
the definition of angle is independent of $\kappa$.
In particular, one can take $\kappa=0$ in the definition, so that the angle can be calculated from the  cosine law:
\[\cos\angk{0}{p}{x}{y}
=
\frac{\dist[2]{p}{x}{}+\dist[2]{p}{y}{}-\dist[2]{x}{y}{}}{2\cdot \dist[{{}}]{p}{x}{}\cdot\dist[{{}}]{p}{y}{}}.\]

\parit{Proof.}
The function $\kappa\mapsto \angk\kappa p{x}{y}$ is nondecreasing (see \ref{k-decrease}).
Thus, for $\Kappa>\kappa$, we have
\begin{align*}
0\le \angk\Kappa p{x}{y}-\angk{\kappa}p{x}{y}
\le& \angk\Kappa p{x}{y}+\angk\Kappa {x}p{y}+\angk\Kappa {y}p{x}-
\\
&-\angk\kappa p{x}{y}-\angk\kappa {x}p{y}-\angk\kappa {y}p{x}
= 
\\
=&\Kappa\cdot\area\modtrig\Kappa(pxy)-\kappa\cdot\area\modtrig\kappa(pxy).
\end{align*}
Thus, \ref{eq:k-K} follows since 
%???WHY???
\[0
\le
\area\modtrig\kappa(pxy)\le \area\modtrig\Kappa(pxy),
%\le
%O\l(\dist[{{}}]{p}{x}{}\cdot\dist[{{}}]{p}{y}{}\r),
\]
%!!!
where the latter area is at most $\dist[{{}}]{p}{x}{}\cdot\dist[{{}}]{p}{y}{}$.

%???It is true without assuming $K\le 0$??? if $K\le 0$ and hence, by central projection from a sphere to its tangent plane, is at most $\tan\dist[{{}}]{p}{x}{}\cdot\tan\dist[{{}}]{p}{y}{}$ if $K=1$.
\qeds



\begin{thm}{Triangle inequality for angles}
\label{claim:angle-3angle-inq}
Let  $[px^1]$, $[px^2]$ and $[px^3]$ %$\gamma^1, \gamma^2, \gamma^3$ 
be three geodesics in a metric space.
If all of the angles $\alpha^{i j}=\mangle\hinge p {x^i}{x^j}$ are defined then they satisfy the triangle inequality:
\[\alpha^{13}\le \alpha^{12}+\alpha^{23}.\]

\end{thm}

\parbf{Remark.}
The above theorem also holds for upper angles, see ???.

\begin{wrapfigure}{l}{30mm}
\begin{lpic}[t(-5mm),b(-30mm),r(0mm),l(0mm)]{pics/s-choice(0.33)}
\lbl[rb]{45,101;$t$}
\lbl[rt]{45,30;$\tau$}
\lbl[W]{50,65;$s\ \ $}
\lbl[l]{18,60,-25;$=\alpha^{12}+\eps$}
\lbl[l]{18,69,24;$=\alpha^{23}+\eps$}
\end{lpic}
\end{wrapfigure}

\parit{Proof.} 
Since $\alpha^{13}\le\pi$, we can assume that $\alpha^{12}+\alpha^{23}< \pi$.
Set $\gamma^i=\geod_{[px^i]}$.
Given any $\eps>0$, for all sufficiently small $t,\tau,s\in\RR_+$ we have
\begin{align*}
\dist{\gamma^1(t)}{\gamma^3(\tau)}{}
\le 
&\dist{\gamma^1(t)}{\gamma^2(s)}{}+\dist{\gamma^2(s)}{\gamma^3(\tau)}{}<\\
<
&\sqrt{t^2+s^2-2\cdot t\cdot  s\cdot \cos(\alpha^{12}+\eps)}+
\\
&+\sqrt{s^2+\tau^2-2\cdot s\cdot \tau\cdot \cos(\alpha^{23}+\eps)}\le
\\
\intertext{(Below we define 
$s(t,\tau)$ so that for 
$s=s(t,\tau)$, this chain of inequalities continues the following way.)}
\le
&\sqrt{t^2+\tau^2-2\cdot t\cdot \tau\cdot \cos(\alpha^{12}+\alpha^{23}+2\cdot \eps)}.
\end{align*}
Thus for any $\eps>0$, 
\[\alpha^{13}\le \alpha^{12}+\alpha^{23}+2\cdot \eps.\]
Hence the result.

To define $s(t,\tau)$, consider three rays $\~\gamma^1$, $\~\gamma^2$, $\~\gamma^3$ on a Euclidean plane starting at one point, such that $\mangle(\~\gamma^1,\~\gamma^2)=\alpha^{12}+\eps$, $\mangle(\~\gamma^2,\~\gamma^3)=\alpha^{23}+\eps$ and $\mangle(\~\gamma^1,\~\gamma^3)=\alpha^{12}+\alpha^{23}+2\cdot \eps$.
We parametrize each ray by length from the starting point.
Given two positive numbers $t,\tau\in\RR_+$, let $s=s(t,\tau)$ be %a 
the 
number such that 
$\~\gamma^2(s)\in[\~\gamma^1(t)\ \~\gamma^3(\tau)]$. Clearly $s\le\max\{t,\tau\}$, % i.e. if $t$ and $\tau$ are both sufficiently small then so is $s$.
so $t,\tau,s$ may be taken sufficiently small.
\qeds

\begin{thm}{First variation inequality}\label{lem:first-var}
Assume for hinge $\hinge q p x$ 
the angle $\alpha=\mangle\hinge q p x$ is defined then
\[\dist{p}{\geod_{[qx]}(t)}{}
\le
\dist{q}{p}{}-t\cdot \cos\alpha+o(t).\]

\end{thm}

\parit{Proof.} Take sufficiently small small $\eps>0$.
For all sufficiently small $t>0$, we have that 
\begin{align*}
 \dist{\geod_{[qp]}(t/\eps)}{\geod_{[qx]}(t)}{}
&\le 
\tfrac{t}{\eps}\cdot \sqrt{1+\eps^2 -2\cdot \eps\cdot \cos\alpha}+o(t)\le
\\
&\le \tfrac{t}{\eps} -t\cdot \cos\alpha + t\cdot \eps.
\end{align*}
Applying triangle inequality, we get 
\begin{align*}
\dist{p}{\geod_{[qx]}(t)}{}
&\le \dist{p}{\geod_{[qp]}(t/\eps)}{}+\dist{\geod_{[qp]}(t/\eps)}{\geod_{[qx]}(t)}{}
\le 
\\
&\le
\dist{p}{q}{} -t\cdot \cos\alpha + t\cdot \eps
\end{align*}
for any $\eps>0$ and all sufficiently small $t$.
Hence the result.
\qeds


%%%%%%%%%%%%%%%%%%%%%%%%%%%%%%%%%%%%%%%%%


\section{Exercises}
\begin{thm}{Exercise}
Prove that the sum of adjacent angles is at least $\pi$.

More precisely: let $\spc{X}$ be a complete length space and $p,x,y,z\in \spc{X}$.
If $p\in \l] x y \r[$, then 
\[\mangle\hinge pxz+\mangle\hinge pyz\ge \pi\]
whenever  each angle on the left-hand side is defined.
\end{thm}