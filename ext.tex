%%!TEX root = all.tex
\chapter{Extremal sets}
\label{extremal}

Imagine that you want to move a heavy box inside an empty room by pushing it around. 
If the box is located in the middle of the room, you  can push it in any direction. 
But once it is pushed against a wall you can not push it back to the center;
and once it is pushed into a corner you cannot push it anywhere anymore. 
The same happens in Alexandrov spaces
when one tries to move a point  by pushing it along a gradient flow, 
but the role of walls and corners is played by  extremal subsets.

Extremal subsets first appeared in the study of their special case --- the boundary of
an Alexandrov's space. 
They were introduced in \cite{perelman-petrunin:extremal}, and
were studied further in \cite{petrunin:extremal}, \cite{perelman:collapsing}.

The presence of extremal subsets in a $\Alex{}$ space
makes it behave as something new, 
intersting and
different from Riemannian manifolds.
It gives an additional combinatoric structure which
reflects geometry and topology of the space itself, as well as of nearby spaces.

\section{New Def}

Let $\spc{L}$ be an $m$-dimensional complete length $\Alex{}$ space and 
$E\subset \spc{L}$ be a closed subset.
Recall that tangent cone $\T_pE$ is defined in ???.

\begin{thm}{Definition}
Let $\spc{L}$ be an $m$-dimensional complete length $\Alex{}$ space.
A closed subset $E\subset \spc{L}$ 
is called \index{extremal set}\emph{extremal}
if for any $p\in E$ and any $v\in\T_p$ there is $w\in \T_pE$ such that
\[\<v,x\>\ge \<w,x\>\]
for any $x\in \T_p$.
\end{thm}

\begin{thm}{Exercise}
Let $\spc{L}$ be an $m$-dimensional complete length $\Alex{}$ space and $p\in \spc{L}$.
Show that one-point set $\{p\}$ forms an extremal set
if and only if $\diam\Sigma_p\le\tfrac\pi2$.
\end{thm}

\begin{thm}{Lemma}
Let 
\end{thm}




\section{The definition}

\begin{thm}{Definition}
Let $\spc{L}$ be an $m$-dimensional complete length $\Alex{}$ space.
A closed subset $E\subset \spc{L}$ 
is called \index{extremal set}\emph{extremal}
if for any point $p\notin E$ the following condition fulfilled.
If $q\in E$ is a point which minimize the distance to $p$
then $\mangle \hinge q p x\le\tfrac\pi2$ for any $x\in\spc{L}$.
Equivalently, $q$ is a critical point of the distance function $\distfun{p}{}{}$,  
see Definition \ref{def:critical-point}.
\end{thm}

\begin{thm}{Proposition}\label{prop:extrimal+nbhd}
Let $\spc{L}$ be an $m$-dimensional complete length $\Alex{}$ space.
Then given a point $p\in \spc{L}$ there is $\eps>0$ such that 
if $\dist{E}{p}{}<\eps$ then $E\ni p$
for any extremal set $E\subset \spc{L}$.
\end{thm}

\parit{Proof.}
According to ???, there is $\eps>0$ such that all the values in $(0,\eps)$ are regular for the distance function $\distfun{p}{}{}$.
It remains to apply the definition of extremal set.
\qeds

\begin{thm}{Proposition}
The union of arbitrary collection of extremal sets is extremal. 
\end{thm}

\parit{Proof.}
Let $\{E_\alpha\}_{\alpha\in\IndexSet}$ be a collection of extremal sets
and 
$$E=\bigcup_\alpha E_\alpha.$$

By Proposition \ref{prop:extrimal+nbhd}, $E$ is closed.

Now assume $p\notin E$; and let $q\in E$ is a point which minimize the distance to $p$. 
Choose $\alpha$ so that $q\in E_\alpha$.
Since $E_\alpha$ is extremal, the point $q$ is critical for the distance function $\distfun{p}{}{}$.
Hence the result follows.
\qeds

\begin{thm}{Definition}\label{def:prim-extr-set}
An extremal set $E$ will be called \index{primitive extremal set}\emph{primitive}
if it can not be presented as a union of extremal sets which are proper subsets of $E$.
\end{thm}

\parbf{Examples.}
Note that according to the definition whole space and empty set are extremal sets.

Let $\spc{Q}$ be unit square in $\EE^2$.
Note that $\spc{Q}$ is a two-dimensional complete length $\Alex0$ space.

The empty set, 
whole $\spc{Q}$ as well as each vertex 
and each side of $\spc{Q}$ forms a primitive extremal subset.
In addition the unions of these primitive extremal sets are also extremal.
(The total number of extremal sets of square should be 47 and 10 of them are primitive.)

On the other hand, the regular pentagon $\spc{P}$ in $\EE^2$
has only 3 extremal sets: $\emptyset$, $\partial\spc{P}$ and $\spc{P}$
and all these sets are primitive.

\section{Extremal sets of the quotient spaces}

Recall that quotient spaces 
were considered in Section~\ref{sec:quotient-CBB}.
The following two theorems provide an important sourse of extremal subsets.

\begin{thm}{Theorem}
Let $\spc{L}$ be an $m$-dimensional complete length $\Alex{}$ space
and $G\acts\spc{L}$ is an isometric group action.
Assume $A$ is a connected component of the fixed point set of the action.
Then $\proj(E)$ forms an extremal subset in $\spc{L}/G$.
\end{thm}

\begin{thm}{Theorem}
Let $\spc{L}$ be an $m$-dimensional complete length $\Alex{}$ space.
Assume $G\acts\spc{L}$ is an isometric group action;
and $\proj\:\spc{L}\to \spc{L}/G$ is the projection to the orbit space.

Then for any extremal set
$E\subset \spc{L}$,
its projection $\proj(E)\subset \spc{L}/G$ forms an extremal set.

\end{thm}




\section{Topology of Extremal sets}

The following proposition gives yet big number of examples.

\begin{thm}{Proposition}\label{prop:t-cone} 
Let $\spc{L}$ be an $m$-dimensional complete length $\Alex\kappa$ space.
Given a cone $K$ 
consider set $S_K\subset \spc{L}$ of all points 
$p\in \spc{L}$ such that there is a homeomorphism $\T_p\to K$ which sends tip of $\T_p$ to the tip of $K$.
Let $Z$ be a path-connected components of $S_K$.
Then the closure of $Z$ forms an extremal subset.
\end{thm}

Note that according to ???, the set $S_K\subset \spc{L}$ can be also defined as set of all points such that its spherical neigborhood is homeomorphic to $K$.

\parit{Proof.}
Fix a path-connected component $Z$ of $S_K$ and let $E$ be the closure of $Z$.
Choose arbitrary point $p\notin E$ and let $q\in E$ be a point which minimize the distance $\dist{p}{q}{}$.

Assume $\mangle\hinge qpx>\tfrac\pi2$;
in other words the $q$ is a regular point for the function $\distfun{p}{}{}$.
Then by Morse lemma (\ref{lem:morse}),
there is a neighborhood $\Omega\ni q$ which splits along $\distfun{p}{}{}$;
that is, there is a homeomorphism $h\:\II\times \Lambda\to\Omega$,
where $\II$ is an open real inteval containing $\dist{p}{q}{}$
such that $\dist{p}{h(t,x)}{}=t$ for any $(t,x)\in \II\times \Lambda$.

Choose a point $q'=h(t_0,x_0)\in \Omega\cap Z$.
Clearly the curve $t\mapsto h(t,x)$ belongs to $Z$.
Choose $t_1\in \II$ such that $t_1<\dist{p}{q}{}$
then $q''=h(t_1,x_0)\in Z$ and $\dist{p}{q''}{}<\dist{p}{q}{}$,
a contradiction.
\qeds




\section{Invariance with respect to gradient flow.}

Extremal subsets can be defined as ``ideals'' of the gradient flow.
Recall that $\GF^t_f$ denotes the $f$-gradient flow for time $t$, 
see Chapter~\ref{chap:grad-flow}.

\begin{thm}{Definition}\label{def:extrim} 
Let $\spc{L}$ be an $m$-dimensional complete length $\Alex\kappa$ space.
A subset $E\subseteq \spc{L}$ is an \index{extremal subset}\emph{extremal subset}, if for any native semiconcave function
$f$ on $\spc{L}$, 
$t\ge 0$ and $x\in E$, we have $\GF_f^t(x)\in E$.
\end{thm}

Note that according to the definition whole space and empty set are extremal subsets.
Any other extremal subset of the space will be called \index{proper extremal subset} \emph{proper}.

\begin{thm}{Proposition}\label{prop:extremal-closed-plus} 
Extremal subsets in complete length $\Alex{}$ spaces are closed.

Moreover, if $\spc{L}$ is an $m$-dimensional complete length $\Alex{}$ space, then for any $p\in \spc{L}$ there is $\eps>0$
such that if an extremal set $E\subset \spc{L}$
intersects $\eps$-neighborhood of $p$ then $p\in E$.
\end{thm}

\begin{thm}{Proposition}
Let $\spc{L}$ be an $m$-dimensional complete length $\Alex{}$ space and $E\subset \spc{L}$ be an extremal subset.
Then the length-metric of $E$ is locally finite.
More over, for any point $p\in E$ and $\eps>0$ there is $\delta>0$
such that any point $q\in E\cap \oBall(p,\delta)$ can be joint to $p$
by a curve which lies in $E$ and has length $<\eps$.

\end{thm}

Both propositions above follow from the following lemma.

\begin{thm}{Lemma}
Let $\spc{L}$ be an $m$-dimensional complete length $\Alex{}$ space and $p\in \spc{L}$.
Then there is a native convex subfunction $f\:\spc{L}\subto\RR$
such that $d_pf(\xi)<-1$ for any $\xi\in\Sigma_p$.

\end{thm}

\parit{Proof.}
Fix small $\eps>0$ (any $\eps<\tfrac\pi2$ will do).

Choose a finite set of geodesic directions 
$\dir p {a_1},\dir p {a_2},\dots,\dir p {a_n}\in \Sigma_p$ 
such that for any direction $\xi\in \Sigma_p$ there is a direction $\xi_i$
such that $\mangle(\xi,\dir p {a_i})<\eps$.
Without loss of generality, we can assume $a_i$ lie on the same distance from $p$, say $\dist{p}{a_i}{}=r$ for all $i$ and fixed $r>0$.

Consider the function 
$$f(x)=\min_i\{\dist{a_i}{x}{}\}.$$

Choose $\eps>0$ so that the restriction $f|\oBall(p,\eps)$ is $\lambda$-concave
and ...

Let $x(t)=\GF_f^t x$.
If $x(t)\ne p$ for some $t$ then 
\begin{align*}
f(p)-f(x(t))
\ge 
-\d_pf(\dir{p}{x})\cdot \dist{p}{x}{}-\lambda\cdot \dist[2]{p}{x}{}
\ge ???
\end{align*}
Therefore 
$(\d_x\distfun{p}{}{})(\nabla_xf)\le ???$ 
if $0<\dist{p}{x}{}<???$.

If follows that if $x(t)=\GF^t_f$ then 
$\tfrac{\d}{\d t}{}\dist{p}{x(t)}{}\le ???$.
I.e., $\GF^t_f$ shrinks $\eps$-neighborhood to $p$ in time ???. 


Note that $f$ is a native semiconcave in a neighborhood of $p$.
Moreover for some $d_pf(\xi)\le -\delta$ for some fixed $\delta>0$ and any $\xi\in\Sigma_p$. 

Let us pass to a spherical neighborhood???
\qeds




\begin{enumerate}[(i)]
\item An Alexandrov's space itself, as well as the empty set, forms an extremal subsets.
\item A point $p\in \spc{L}$ forms a one-point extremal subset if its
space of directions $\Sigma_p$ has a diameter $\le\tfrac\pi2$

\item 
Let $\spc{L}/G$ be a factor of an Alexandrov's space by an isometry group and $S_H\subset \spc{L}$ be the set of points with stabilizer conjugate to a subgroup $H\subset G$ (or its connected component). 
Then the closure of the projection of $S_H$ in $\spc{L}/G$ forms an extremal subset.

WHy???

For example: A cube can be presented as a quotient of a flat torus by a discrete isometry group, and each face of the cube forms an extremal subset.

\end{enumerate}


\noi The following theorem gives an equivalence of our definition of extremal subset and the definition given in
\cite{perelman-petrunin:extremal}:

\begin{thm}{Theorem}\label{thm:dist-extr}  A closed subset $E$ in
an Alexandrov's space $\spc{L}$ is extremal if and only if for any $q\in \spc{L}\backslash E$,
the following condition is fulfilled:

If $\distfun{q}{}{}$ has a local minimum on $E$ at a point $p$, then $p$ is a critical
point of $\distfun{q}{}{}$ on $\spc{L}$; 
that is, $\nabla_p\distfun{q}{}{}=\0$.
\end{thm}

\parit{Proof.} For the ``only if'' part, note that if $p\in E$ is not a critical point of
$\distfun{q}{}{}$, then one can find a point $x\in \Str(p)$ (thus $\dir p x$ is
uniquely defined) such that 
\[(\d_p\distfun{q}{}{})(\dir p x)>0.\] 
Since $\nabla_p\distfun{x}{}{}$ is polar to $\dir p x$ (see
???) we get 
\[(\d_p\distfun{q}{}{})(\nabla_p\distfun{x}{}{})<0,\] 
see ???.
Hence, the gradient flow $\GF_{\distfun{x}{}{}}^t$ pushes the point $p$ closer to $q$, which
contradicts the fact that $p$ is a minimum point of $\distfun{q}{}{}$ on $E$.

To prove the ``if'' part, it is sufficient to show that if $F\subset \spc{L}$ satisfies the condition of the theorem,
then for any $p\in F$, and any semiconcave function $f$, either $\nabla_p f=\0$ or 
$\tfrac{\nabla_p f}{|\nabla_p f|}\in \Sigma_p F$.
If so, an $f$-gradient curve
can be obtained as a limit of broken lines with vertexes on $F$, and from
uniqueness, any gradient curve which starts at $F$ lives in $F$.

Let us use induction on $\dim \spc{L}$. 
Note that if $F\subset \spc{L}$ satisfies the condition,
then the same is true for $\Sigma_p F\subset\Sigma_p$, for any $p\in F$.
Then using the inductive hypothesis we get that $\Sigma_p F\subset \Sigma_p$ is an extremal subset.

If $p$ is isolated, then clearly $\diam \Sigma_p\le \tfrac\pi2$ and therefore $\nabla_p
f=o$, so we can assume $\Sigma_p F\not=\emptyset$. 

Note that $\phi=\d_p f|\Sigma_p$ satisfies $\phi''+\phi\le 0$.
Take $\xi=\tfrac{\nabla_p f}{|\nabla_p f|}$, so 
$\xi\in \Sigma_p$ is the maximal point of $\phi$.
Let $\eta\in \Sigma_p F$ be a direction closest  to $\xi$. Then
$\mangle(\xi,\eta)\le \tfrac\pi2$; 
otherwise $F$ would not satisfy the condition in the theorem for a point $q$ with $\dir{p}{q}\approx\xi$.
Hence, since $\Sigma_p F\subset \Sigma_p$ is an extremal subset, $\nabla_\eta\phi\in \Sigma_\eta \Sigma_p F$ and
therefore 
\[\d_\eta\phi(\dir\eta\xi)
\le
\<\nabla_\eta\phi,\dir\eta\xi\>\le0.\]
Hence, $\phi(\eta)\ge\phi(\xi)$, and therefore $\xi=\eta$; 
that is, $\tfrac{\nabla_p f}{|\nabla_p f|}\in \Sigma_p F$. 
\qeds

From this theorem it follows that in the definition of extremal subset (\ref{def:extrim}),
one has to check only squares of distance functions. 
Namely:

\begin{thm}{Corollary}
Let $\spc{L}$ be an $m$-dimensional complete length $\Alex\kappa$ space. 
Then $E\subset \spc{L}$ is an extremal subset, if for any point $p\in \spc{L}$, and any $x\in E$, we
have $\GF_{\dist[2]{p}{}{}}^t(x)\in E$ for any $t\ge 0$.
\end{thm}

In particular, applying Lemma~\ref{lem:stable-grad-curves} we get

\begin{thm}{Lemma}
\label{lem:limit-extr} The limit of extremal subsets is an extremal subset. 

Namely, if $\spc{L}_n$ is an $m$-dimensional complete length $\Alex\kappa$ space, $\spc{L}_n\GHto \spc{L}$  and $E_n\subset \spc{L}_n$ is a sequence
of extremal subsets such that $E_n\to E\subset \spc{L}$ then $E$ is an extremal subset of
$\spc{L}$. 
\end{thm}

The following is yet another important technical lemma:

\begin{thm} {Lemma} {\rm \cite[3.1(2)]{perelman-petrunin:extremal}}
\label{lem:dist-to-extr}
Let $\spc{L}$ be a compact $m$-dimensional length $\Alex\kappa$ space. 
Then there is $\eps>0$
such that $\distfun{E}{}{}$ has no critical values in $(0,\eps)$. 
Moreover,
\[|\nabla_x\distfun{E}{}{}|>\eps\ \ \text{if}\ \ 0<\dist{E}{x}{}<\eps.\]

For a non-compact $\spc{L}$,  the same holds 
for the restriction
 $\distfun{E}{}{}|\Omega$ to any bounded open $\Omega\subset \spc{L}$.
\end{thm}



\parit{Proof.} 
Follows from Lemma~\ref{lem:tuda-suda} 
and Theorem~\ref{thm:dist-extr}.\qeds

\section{Tangent cone}

Note that the tangent cone \label{T_pE}$\T_p E$ of an extremal subset $E\subset \spc{L}$ is defined; 
that is,
for any $p\in E$, subsets $\lam E$ in $(\lam \spc{L},p)$ converge to a subcone of $\T_p E\subset \T_p \spc{L}$ as
$\lam\to\infty$.
Indeed, assume $E\subset \spc{L}$ is an extremal subset and $p\in E$.
For any $\xi\in \Sigma_p E$\footnote{\label{U_pX}For a closed subset $X\subset \spc{L}$, and $p\in X$, $\Sigma_p X\subset \Sigma_p$ denotes the set of tangent directions to $X$ at $p$; that is, the
set of limits of $\dir p{q_n}$ for $q_n\to p$, $q_n\in X$.}, the radial curve $\gexp(t\cdot\xi)$ lies in $E$.%
\footnote{that follows from the fact that the curves 
$t\mapsto\gexp(t\cdot\dir p{q_n})$ starting with $q_n$ belong to $E$
and their converge to $\gexp(t\cdot\xi)$} 
In particular, there is a curve which goes in any tangent direction of $E$.
Therefore, as $\lam\to\infty$,  $(\lam E\subset \lam \spc{L},p)$ converges to
a subcone $\T_p E\subset \T_p \spc{L}$, which is simply cone over $\Sigma_p E$ (see also
\cite[3.3]{perelman-petrunin:extremal})

???The same argument shows that for any $p\in \spc{L}\backslash E$ the fuction $\inf_{\xi\in\Sigma_xE}\{\mangle(\dir x p,\xi)\}$ is locally bounded away from zero.

Next we list some properties of tangent cones of extremal subsets, compare
\cite[1.4]{perelman-petrunin:extremal}

\begin{thm}{Theorem}
\label{ext-tangent} A closed subset $E\subset \spc{L}$ is  extremal if and only if 
at any point $p\in E$, its tangent cone $\T_p E\subset \T_p \spc{L}$ is defined,
and it is an extremal subset of the tangent cone $\T_p \spc{L}$.
\end{thm}

Here is an equivalent formulation in terms of the space of directions: 

\begin{thm}{Theorem}
For any $p\in E$,
either 
(a) $\Sigma_p E=\emptyset$ and $\diam \Sigma_p\le\tfrac\pi2$ or 
(b) $\Sigma_p E=\{\xi\}$ is one point extremal subset and $\cBall[\xi,\tfrac\pi2]=\Sigma_p$
or
(c) $\Sigma_p E$ is extremal subset of $\Sigma_p$ with at least two points.)
\end{thm}

\parit{Proof.}???
$\T_p E$ is extremal as a limit of extremal subsets, see
Lemma~\ref{lem:limit-extr}. 
On the other hand for any semiconcave function $f$ and $p\in E$, the differential $\d_p f:\T_p\to\RR$ is concave and since $\T_p E\subset \T_p$ is extremal we have $\nabla_p f\in \T_p E$. 
I.e. gradient curves can be approximated by broken
geodesics with vertices on $E$, see page~\pageref{grad-constr}.
\qeds

The following theorem was prove in \cite[3.4--5]{perelman-petrunin:extremal}.

\begin{thm}{Theorem} 
If $E$ and $F$ are extremal sets 
\begin{subthm}{E+F} The set $E\cap F$ is extremal 
and 
\[\T_p(E\cup F)=\T_p E\cup \T_p F\]
for any $p\in E\cap F$.
\end{subthm}

\begin{subthm}{EF} The set $E\cup F$  is extremal
and 
\[\T_p(E\cap F)=\T_p E\cap \T_p F\]
for any $p\in E\cup F$
\end{subthm}

\begin{subthm}{E-F} The set $\overline{E\backslash F}$ is extremal 
and 
\[\T_p(\overline{E\backslash
F})=\overline{\T_p E\backslash \T_p F}\]
for any $p\in \overline{E\backslash F}$.
\end{subthm}

In particular, if $\T_p E=\T_p F$ then $E$ and $F$ coincide in a neighborhood of $p$.
\end{thm}

\parit{Proof.}???
 The properties (\ref{SHORT.E+F}) and (\ref{SHORT.EF}) are obvious. 
The property (\ref{SHORT.E-F}) follows from
Property~\ref{ext-tangent} and Lemma~\ref{lem:dist-to-extr}.
\qeds

\section{Branching coverings}

Here is a partial converse to Theorem ???

\begin{thm}{Theorem}
Let $\spc{X}$ be a length space with isometric action $\ZZ_2\acts\spc{X}$.
Assume that the quotient space $\spc{L}=\spc{X}/\ZZ_2$
 is an $m$-dimensional complete length $\Alex\kappa$ space; denote by $E\subset \spc{L}$ the projection of fixed point set of $\ZZ_2\acts\spc{X}$.

Then $spc{X}$ is $\Alex\kappa$ space if and only if $E$ is an extremal set in $\spc{L}$.
\end{thm}



\section{Intrisic geometry of extremal sets}

\begin{enumerate}
\item 
Let $(\spc{L}_n)$ be a sequence of $m$-dimensional complete length $\Alex\kappa$ spaces, 
$\spc{L}_n\GHto \spc{L}$ without collapse 
(that is, $\dim \spc{L}=m$) and $E_n\subset \spc{L}_n$ be extremal subsets. 
Assume $E_n\to E\subset \spc{L}$ as subsets. 
Then 
\begin{enumerate}[(i)]
\item\cite[9.1]{kapovitch:stability} \label{lim-dim-extr} For all large $n$, there is a homeomorphism of pairs $(\spc{L}_n,E_n)\to(\spc{L},E)$. 
In particular, for all large $n$, $E_n$ is homeomorphic to $E$, 
\item\cite[1.2]{petrunin:extremal}\label{lim-intr-extr} $E_n\GHto E$ as metric spaces with the
length-metrics from $\spc{L}_n$ and $\spc{L}$.
\end{enumerate}
The first property is a coproduct of the proof of Perelman's stability theorem.
The proof of the second is an application of quasigeodesics.
\end{enumerate}


Let us denote by $\Ext(x)$ the minimal extremal subset which contains a point
$x\in \spc{L}$. 
Extremal subsets which can be obtained this way will be called
\emph{primitive}. 
Set \[\Ext^\circ(x)=\{y\in \Ext(x)\mid\Ext(y)=\Ext(x)\};\]
let us call $\Ext^\circ(x)$ the \emph{main part} of $\Ext(x)$.
$\Ext^\circ(x)$ is the same as $\Ext(x)$ with its proper extremal subsets removed.
From Property~\ref{E-F} on page~\pageref{E-F},  $\Ext^\circ(x)$ is open and everywhere
dense in $\Ext(x)$.
Clearly the main parts of primitive extremal subsets form a disjoint
covering of $M$.

\begin{enumerate}
\item \cite[3.8]{perelman-petrunin:extremal} {\it Stratification.}\label{strata} The main part of a primitive extremal subset is a
topological manifold. In particular, the main parts of primitive extremal subsets
stratify Alexandrov's space into topological manifolds. 

This follows from Theorem~\ref{thm:dist-extr} and the Morse lemma (property~\ref{morse} page~\pageref{morse}); see also example~\ref{ex:t-cone}, page~\pageref{ex:t-cone}.
\end{enumerate}


\begin{thm}{Lieberman's Lemma}\label{lib-lem} 
Let $\spc{L}$ be an $m$-dimensional complete length $\Alex{}$ space and $E\subset \spc{L}$ be extremal set.
Then any unit-speed geodesic for the
length-metric on $E$ is a quasigeodesic in $\spc{L}$.
\end{thm}

???ADJUST THE PROOF TO THE NEW DEF OF QG???

\parit{Proof.} 
Let $\gamma:\II\to E$ be unit-speed geodesic in an extremal subset
$E\subset \spc{L}$ and $f$ be a native $\lambda$-concave subfunction such that
$\gamma\subset\Dom f$.
Assume $f\circ\gamma$ is not $\lambda$-concave, then there is a non-negative
Lipschitz function $\tau$ with support in $(a,b)$ such that
\[\int\limits_a^b\l[(f\circ\gamma)'\cdot\tau'+\lambda\cdot\tau\r]\cdot\d s
< 
0\]
Then as follows from Lemma~\ref{lem:grad-variation}, for small $t\ge 0$
\[\gamma_t(s)=\GF^{t\cdot\tau(s)}_f \circ\gamma_0(s)\]
gives a length-contracting homotopy of curves relative to ends 
and according to
Definition~\ref{def:extrim}, it stays in $E$, a contradiction.\qeds

The next theorem implies that extrinsic and length-metrics on extrimal subset are bi-Lipschitz equivalent.

\begin{thm}{Theorem \cite[3.2(3)]{perelman-petrunin:extremal}}
Let $\spc{L}$ be an $m$-dimensional complete length $\Alex\kappa$ space,
$E\subset \spc{L}$ be an extremal set 
and $\Omega\subset \spc{L}$ be a bounded open subset.

Then there is $\eps>0$ such that for any $x,y\in E\cap\Omega$, we have
\[\dist{x}{y}{E} \le \tfrac{1}{\eps}\cdot\dist[{{}}]{x}{y}{}.\]

Moreover $\eps=\eps(m,\kappa,\diam\Omega,\vol\Omega)$.
\end{thm}

\parit{Proof.}
???From Lemma~\ref{lem:tuda-suda}, 
it follows that for two sufficiently close
points $x,y\in E$ near $p$ there is a point $z$ so that
$\<\nabla_x\distfun{z}{}{},\dir xy\> >\eps$ or $\<\nabla_y\distfun{z}{}{},\dir y x\>
>\eps$. 
Then, for the corresponding point, say $x$, the gradient curve
$t\to\GF^t_{\distfun{z}{}{}}(x)$ lies in $E$, it is $1$-Lipschitz and the distance
$\dist{\GF^t_{\distfun{z}{}{}} (x)}{y}{}$ is decreasing with the speed of at least $\eps$. 
Hence the result.
\qeds


\begin{thm}{Theorem} 
Let $(\spc{L}_n)$ be a non-collapsing sequence of $m$-dimensional complete length $\Alex\kappa$ spaces, $\spc{L}_n\GHto \spc{L}$ 
and
$E_n \subset \spc{L}_n$ be a sequence extremal subsets which converges $E_n\to E$ as subsets.

Equip subsets $E_n\subset \spc{L}_n$ and $E\subset\spc{L}$ with length-metrics,
then $E_n \GHto E$.
\end{thm}

We first prove that inner and external on extremal subsets are locally bi-Lipschtz:

\parit{Proof.} 
Assume the contrary, then, after passing to a subsequence,
one can choose two sequences $p_n,$ $q_n \in E_n$, $p_n \to p\in E$ and $q_n \to q\in E$ such that
\[\dist{p}{q}{F} < \lim_{n\to\infty} \dist{p_n}{q_n}{F_n}.\]
 
Therefore we need to show only that
$\dist{p}{q}{F} \ge \limsup_{n\to\infty} \dist{p_n}{q_n}{F_n}$. 
Set $\yetdist{p}{q}{F}=\limsup_{n\to\infty}\dist{p_n}{q_n}{F_n}$; this is easily seen to be a metric. 
From the previous paragraph, $\yetdist{p}{q}{}$ does not depend on the choice of sequences $(p_n)$ and $(q_n)$, 
and we have $\yetdist{p}{q}{}<\tfrac{1}{\eps}\cdot\dist[{{}}]{p}{q}{}$, because from above $\eps$ can be found uniformly for all $\spc{L}_n$ in the absence of collapse.

Let $\gamma\:[a,b]\to E$ be a shortest path in $E$ between $p$ and $q$ which is parametrized by arc-length. 
Assume $\dist{p}{q}{F}<\yetdist{p}{q}{F}$. 
Then, from \cite[5.14]{busemann}, for some
$t_0 \in[a, b]$ and $\eps > 0$ there is a sequence $t_i \to t_0\pm$ such that
\[\yetdist{\gamma(t_0)}{\gamma(t_i)}{} \ge (1+\eps)\cdot|t_i-t_0|.\]
Setting $r = \gamma(t_0)$ and $s = \gamma(t_i)$, take sequences $r_n$, $s_n \in E_n$ such that $r_n \to r$ and $s_n \to s$. 
Let $\gamma_i$ in $E$ be the limit curve to the shortest paths between $r_n$ and $s_n$ in $E_n$. 
By \cite[2.3(3)]{perelman-petrunin:qg}??? and the generalized
Lieberman lemma, $\gamma_i$ is a quasigeodesic between $\gamma(t_0)$ and $\gamma(t_i)$. 
From above, $\length\gamma_i \ge (1 + \eps)\cdot|t_i - t_0 |$. 
Now consider the limit $(\tfrac{1}{|t_0-t_i |}\blow \spc{L}, r) \to (\T_r,\0)$.
Consider the curve in $\T_r$ given by
$\gamma^*(t) = \lim_{i\to\infty}\gamma_i(\frac{t-t_0}{t_1-t0}),$
where ??? denotes the image of $\gamma_i$ in $\frac{1}{|t_0-t_i |}\spc{L}$. 
Then $\gamma^*$ is a quasigeodesic between $\0$ and the tangent vector $\gamma^+(t_0)$ which has length not less then
$1 + \eps$. 
This is a contradiction since $|\gamma^\pm(t_0)|=1$ by [Perelman and Petrunin 1994, 2.3(2)].



\begin{thm}{The first variation formula\cite[1.4]{petrunin:extremal}}\label{1st-var}
Assume $\spc{L}$ is an $m$-dimensional complete length $\Alex\kappa$ space and
$E\subset \spc{L}$ is an extremal subset.
Recall that $\dist{*}{*}{E}$ denotes its length-metric. 
Let $p,q \in E$ 
and $\alpha(t)$ be a curve in $E$ starting from $p$ in direction
$\alpha^+(0)\in \Sigma_p E$.
Then 
\[\dist{\alpha(t)}{q}{E}=\dist{p}{q}{E}-\cos\phi\cdot t + o(t).\]
where $\phi$ is the minimal length-distance in $\Sigma_p E$ between
$\alpha^+(0)$ and a direction of a shortest path in $E$ from $p$ to $q$ (if $\phi>\pi$, we assume $\cos\phi=-1$).
\end{thm}

\parit{Proof.}???\qeds



\section{Exercises}

A $\Alex{}$ space is called \index{orientable}\emph{orientable} if its main part is an orientable manifold.

\begin{thm}{Exercise}
Show that for any  complete length $\Alex{}$ space $\spc{L}$ there is an orientable complete length $\Alex{}$ space $\bar{\spc{L}}$ without boundary and an involution $\iota\:\bar{\spc{L}}\to\bar{\spc{L}}$
such that $\spc{L}$ is isometric to the quotient space $\bar{\spc{L}}/\iota$.
\end{thm}



