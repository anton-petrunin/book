\chapter{Polyhedral spaces}

\section{Definitions}

\begin{thm}{Definition}\label{def:poly}
A length space $\spc{P}$ is called  
\emph{$\Lob{}\kappa$-polyhedral space} (briefly $\spc{P}\in \PM{}\kappa$, where $\mathbf{P}$ stands for \emph{piecewise})
if it admits a finite triangulation $\tau$ 
such that an arbitrary simplex $\sigma$ in $\tau$ is isometric to a simplex in the model space $\Lob{m}{\kappa}$, 
where $m$ is the dimension of $\sigma$.

If we do not wish to specify $\kappa$, we will say that $\spc{P}$ is a \emph{polyhedral space}.

By a 
\index{triangulation of a polyhedral space space}\emph{triangulation of a polyhedral space} 
we will understand the triangulation as in the definition. 

\begin{subthm}{}
The $\Lob{}1$-polyhedral spaces will also be 
called 
\index{spherical polyhedral space}\emph{spherical polyhedral spaces};
\end{subthm}

\begin{subthm}{}
The $\Lob{}0$-polyhedral spaces will also be 
called 
\index{Euclidean polyhedral space}\emph{Euclidean polyhedral spaces};
\end{subthm}

\begin{subthm}{}
The $\Lob{}{-1}$-polyhedral spaces will also be 
called 
\emph{hyperbolic polyhedral space}\index{hyperbolic polyhedral space}.
\end{subthm}
\end{thm}

Note that according to the above definition,
all polyhedral spaces are compact.
However, 
most of the statements below admit straightforward generalizations 
to \index{polyhedral spaces!locally polyhedral spaces}\emph{locally polyhedral spaces};
that is, complete length spaces,  
any point of which admits a closed neighborhood isometric to a polyhedral space.
The latter class of spaces includes in particular the infinite covers of polyhedral spaces.

The dimension of a polyhedral space $\spc{P}$
is defined as the maximal dimension of the simplex 
in one (and therefore any) triangulation of $\spc{P}$.
The class of $\PM{}{\kappa}$ spaces 
of dimension $m$ will be denoted 
as $\PM{m}{\kappa}$.


\parbf{Links.}
Let $\spc{P}$ be a polyhedral space
and $\sigma$ be a simplex in a triangulation $\tau$ of $\spc{P}$.

The simplices which contain $\sigma$
form an abstract simplicial complex called the \emph{link} of $\sigma$, 
denoted by $\Link_\sigma$.
If $m$ is  the dimension of $\sigma$
then the set of vertices of $\Link_\sigma$
is formed by the $(m+1)$-simplices which contain $\sigma$;
the set of its edges are formed by the $(m+2)$-simplices 
which contain $\sigma$, and so on.

The link $\Link_\sigma$
can be identified with the subcomplex of $\tau$ 
formed by all the simplices $\sigma'$ 
such that $\sigma\cap\sigma'=\emptyset$ 
but both $\sigma$ and $\sigma'$ are faces of the same simplex.

The points in $\Link_\sigma$ can be identified with the normal directions to $\sigma$ at a point in $\sigma$.
The angle metric between directions makes  $\Link_\sigma$ into a spherical polyhedral space.
We will always consider the link with this metric.

\parbf{Tangent space and space of directions.}
Let $\spc{P}$ be a polyhedral space and let  $\tau$ be a triangulation of $\spc{P}$.
If a point $p\in \spc{P}$ 
lies in the interior of a $\kay$-simplex $\sigma$ of $\tau$ 
then the tangent space $\T_p\spc{P}$
is  naturally isometric to
\[\EE^\kay\times(\Cone\Link_\sigma).\]
Equivalently, the space of directions $\Sigma_p$
can be isometrically identified with the 
$\kay$-th spherical suspension over $\Link_\sigma$;
that is, 
\[\Sigma_p\iso\Susp^{\kay}(\Link_\sigma).\]

If $\spc{P}$ is an $m$-dimensional polyhedral space,
then for any $p\in \spc{P}$
the space of directions $\Sigma_p$ is a spherical polyhedral space
of dimension at most $m-1$. 

In particular, 
for any point $p$ in $\sigma$,
the isometry class of $\Link_\sigma$ together with $\kay=\dim\sigma$
determines the isometry class of $\Sigma_p$ 
 and the other way around.

A small neighborhood of $p$ is isometric to a neighborhood of the tip of the $\kappa$-cone over $\Sigma_p$.
In fact, if this propery holds at any point of a compact length space $\spc{P}$
then  $\spc{P}\in \PM{}\kappa$;
see \cite{lebedeva-petrunin-poly}.

%%%%%%%%%%  DOWN

\parbf{Curvature bounded below.}
The following theorem provides a combinatorial description of polyhedral spaces with curvature bounded below.

\begin{thm}{Theorem}\label{thm:poly-CBB} Let $\spc{P}\in\PM{m}\kappa$ and $\tau$ be a triangulation of $\spc{P}$.
Then $\spc{P}\in\CBB{}{\kappa}$ if and only if the following conditions hold.

\begin{subthm}{} $\tau$ is \emph{pure}; 
that is, any simplex in $\tau$ is  a face of some simplex of dimension exactly $m$. 
\end{subthm}

\begin{subthm}{thm:poly-CBB:m-1}
The link of any simplex of dimension $m-1$ is formed by single point or two points.
\end{subthm}

\begin{subthm}{thm:poly-CBB:connected}
The link of any simplex of dimension $\le m-2$ is connected.
\end{subthm}

\begin{subthm}{thm:poly-CBB:2pi}
Any link of any simplex of dimension $m-2$
has diameter at most $\pi$.
\end{subthm}
\end{thm}

The condition (\ref{SHORT.thm:poly-CBB:connected})
can be reformulated in the following way:

\begin{itemize}
 \item[{\it \ref{SHORT.thm:poly-CBB:connected}$\,'\!$)}] 
Any path $\gamma\:[0,1]\to \spc{P}$ can be approximated by paths
$\gamma_n\:[0,1]\to \spc{P}$ 
which  cross only simplexes of dimension $m$ and $m-1$.
\end{itemize}

Further, modulo the other conditions,
the condition (\ref{SHORT.thm:poly-CBB:2pi})
is equivalent to the following:


\begin{itemize}
 \item[{\it \ref{SHORT.thm:poly-CBB:2pi}$\,'\!$)}] 
The link of any simplex of dimension $m-2$ is 
isometric to a circle of length $\le 2\cdot\pi$
or a closed real interval of length $\le \pi$.
\end{itemize}

\parit{Proof.} We apply induction on $m$.
The base case $m=1$ follows from the assumption (\ref{SHORT.thm:poly-CBB:m-1}).

\parit{ Induction Step.}
Assume that the theorem is proved for polyhedral spaces  of $\dim <m$. Suppose  $\dim\spc{P}=m$.

We will only give a proof of the ``if" direction. The ``only if" direction is similar and is left to the reader.


According to the Globalization theorem (\ref{thm:glob}),
it is sufficient to show that 
$$\curv_p\spc{P}\ge \kappa$$ 
for any point $p\in\spc{P}$. 

Fix $p\in \spc{P}$.
Note that a spherical neighborhood of $p$
is isometric
to a  spherical neighborhood of the tip of the tangent $\kappa$-cone 
$$\T_p\mc\kappa\z
=
\Cone\mc\kappa(\Sigma_p).$$
Hence it is sufficient to show that 
$$\T_p\mc\kappa\in\CBB{}{\kappa}
\eqlbl{eq:curv(T_p)>=k}$$
for any $p\in \spc{P}$.

By Theorem~\ref{thm:warp-curv-bound:cbb:a}, 
the latter is equivalent to 
\begin{clm}{}\label{clm:curv+diam}
$\diam\Sigma_p\le \pi$ and $\Sigma_p\in\CBB{}{1}$.
\end{clm}


If $m=2$ then \ref{clm:curv+diam} follows from (\ref{SHORT.thm:poly-CBB:m-1}).

To prove the case $m\ge 3$,
note that $\Sigma_p$ is an $(m-1)$-dimensional spherical polyhedral space and all the conditions of the theorem hold for $\Sigma_p$.
It remains to apply the induction hypothesis.\qeds

\begin{thm}{Exercise}
Assume  $\spc{P}\in\PM{}\kappa$ and $\dim \spc{P}\ge 2$. 
Show that 
if $\spc{P}\in\CBB{}{\kappa'}$ then $\kappa'\le \kappa$ and $\spc{P}\in \CBB{}{\kappa}$.
\end{thm}

%%%%%%%%%% UP???

\parbf{Curvature bounded above.}
Recall that the definition of $\ell$-simply connected space is 
given in \ref{def:l-s.c.}.

The following theorem provides a combinatorial description of polyhedral spaces with curvature bounded above.


\begin{thm}{Theorem}\label{thm:PL-CAT}
Let $\spc{P}\in\PM{}\kappa$ and $\tau$ be a triangulation 
of  $\spc{P}$. Then 

\begin{subthm}{thm:PL-CAT:curc>=k}
$\curv\spc{P}\le\kappa$ 
if and only if any connected component of the link of any simplex $\sigma$ in $\tau$
is $(2\cdot\pi)$-simply connected.
Equivalently, if and only if any closed local geodesic in $\Link_\sigma$ has length at least $2\cdot\pi$.
\end{subthm}

\begin{subthm}{thm:PL-CAT:CAT}
$\spc{P}\in\cCat{}{\kappa}$ 
if and only if $\spc{P}$ is $(2\cdot\varpi\kappa)$-simply connected and any connected component of the link of any simplex $\sigma$ in $\tau$
is $(2\cdot\pi)$-simply connected.
\end{subthm}

\end{thm}


\parit{Proof.}
Let us apply induction on $\dim\spc{P}$.
The base case $\dim\spc{P}=0$ is evident.

\parit{Step.}
Assume that the theorem is proved in the case $\dim\spc{P}<m$. Suppose  $\dim\spc{P}=m$.

We will only give a proof of the ``if" direction.

Fix a point $p\in\spc{P}$.
A neighborhood of $p$ 
is isometric to the $\kappa$-cone over 
the $\Sigma_p$.
By Theorem~\ref{thm:warp-curv-bound:cbb:a}, 
it is sufficient to show that 
\[\Sigma_p\in \Cat{}1.
\eqlbl{eq:Sigma-in-CAT(1)}\]

Note that $\Sigma_p$ is a spherical polyhedral space 
and its  links are isometric to  links of $\spc{P}$. 
By the  induction hypothesis, $\curv\Sigma_p\le 1$.
Applying Generalized Hadamard--Cartan theorem (\ref{thm:hadamard-cartan-gen}),
we get \ref{eq:Sigma-in-CAT(1)}.

To prove (\ref{SHORT.thm:PL-CAT:CAT}) apply Generalized Hadamard--Cartan theorem (\ref{thm:hadamard-cartan-gen}) to $\spc{P}$.
\qeds

\begin{thm}{Exercise}
Show that if in a Euclidean polyhedral space $\spc{P}$
any two points can be connected by a unique geodesic 
then $\spc{P}\in\cCat{}{0}$.
\end{thm}

\begin{thm}{Exercise}
Assume  $\spc{P}\in\PM{}\kappa$ and $\dim \spc{P}\ge 2$. 
Show that 
if $\spc{P}\in \cCat{}{\kappa'}$ then $\kappa'\ge \kappa$ and $\spc{P}\in\cCat{}{\kappa}$.
\end{thm}


\section{Flag complexes}


\begin{thm}{Definition}
A simplicial complex $\mathcal{S}$ 
is called \index{flag complex}\emph{flag} if whenever $\{v_0,\z\dots,v_\kay\}$
is a set of distinct vertices of $\mathcal{S}$
which are pairwise joined by edges, then the vertexes $v_0,\dots,v_\kay$
span a $\kay$-simplex in $\mathcal{S}$.

If the above condition is satisfied only for $\kay=2$, 
then we say $\mathcal{S}$ satisfies 
the \emph{no-triangle condition}\index{no-triangle condition}.
\end{thm}

Note that every flag complex is determined by its 1-skeleton.

\begin{thm}{Exercise}\label{ex:baricenric-flag}
Show that the barycentric subdivision of any simplicial complex is a flag complex.
Conclude that any simplicial complex is homeomorphic to a $\cCat{}{1}$ space.
\end{thm}


\begin{thm}{Proposition}\label{prop:no-trig}
A simplicial complex $\mathcal{S}$ is flag if and only if 
$\mathcal{S}$ as well as all the links of all its simplices
satisfies the no-triangle condition.
\end{thm}

From the definition of flag complex 
we get the following.

\begin{thm}{Lemma}\label{lem:link-of-flag}
Any link of a flag complex is flag.
\end{thm}


\parit{Proof of Proposition~\ref{prop:no-trig}.}
By Lemma~\ref{lem:link-of-flag}, the no-triangle condition holds 
for any flag complex and all its links.

Now assume a complex $\spc{S}$ and all its links satisfy 
the no-triangle condition.
It follows that $\spc{S}$ includes a 2-simplex for each triangle.
Applying the same observation for each edge we get that $\spc{S}$ 
includes a 3-simplex for any complete graph with 4 vertices.
Repeating this observation 
for triangles, 
4-simplexes,
5-simplices
and so on we get that $\spc{S}$ is flag.
\qeds


\parbf{All-right triangulation.}
A triangulation of a spherical polyhedral space 
is called an  \emph{all-right triangulation} 
if each simplex of the triangulation is isometric 
to a spherical simplex all of whose angles are right.
Similarly, we say that a simplicial complex 
is equipped with an  \emph{all-right spherical metric}
if it is a length metric and each simplex is isometric 
to a spherical simplex all of whose angles are right.

Spherical polyhedral $\cCat{}{1}$ spaces glued from of right-angled simplices
admit the following characterization 
discovered by Gromov \cite[p. 122]{gromov:hyp-groups}.

\begin{thm}{Flag condition}\label{thm:flag}
Assume that a spherical polyhedral space $\spc{P}$
admits an all-right triangulation $\tau$.
Then $\spc{P}\in\cCat{}{1}$
if and only if $\tau$ is flag.
\end{thm}

\parit{Proof; ``only-if'' part.} 
Assume there are three vertices $v_1,v_2$ and $v_3$ of $\tau$
that are pairwise joined by edges 
but do not span a simplex.
Note that in this case 
$$\mangle\hinge{v_1}{v_2}{v_3}=\mangle\hinge{v_2}{v_3}{v_1}=\mangle\hinge{v_3}{v_1}{v_1}=\pi.$$
Equivalently,
\begin{clm}{}\label{clm:3pi/2}
The join of the geodesics $[v_1v_2]$, $[v_2v_3]$ and $[v_3v_1]$
forms a locally geodesic loop in $\spc{P}$. 
\end{clm}

Now assume that $\spc{P}\in\cCat{}{1}$.
Then by \ref{thm:warp-curv-bound:cbb:a},
$\Link_\sigma\spc{P}\in\in\cCat{}{1}$ for every simplex $\sigma$ 
in $\tau$. 

Each of these links is an all-right spherical complex
and
by Theorem \ref{thm:PL-CAT}, 
none
of these links can contain a geodesic circle of length less than $2\cdot\pi$. 

Therefore Proposition~\ref{prop:no-trig} and \ref{clm:3pi/2} 
imply the ``only-if'' part.

\parit{``If'' part.} 
By Lemma~\ref{lem:link-of-flag} and Theorem~\ref{thm:PL-CAT},
it is sufficient to show that any closed local geodesic $\gamma$ 
in a flag complex $\spc{S}$ with all-right metric has length at least $2\cdot\pi$.

Fix a flag complex $\spc{S}$.
Recall that the  \index{star of vertex}\emph{star} of a vertex $v$ (briefly $\overline \Star_v$)
is formed by all the simplices  containing $v$. Similarly, $\Star_v$,   the open star of $v$, is the union of all simplices containing $v$ with faces opposite $v$ removed.

Choose a simplex $\sigma$ which contains a point of $\gamma$.
Let $v$ be a vertex of $\sigma$.
Set $f(t)=\cos\dist{v}{\gamma(t)}{}$.
Note that 
\[f''(t)+f(t)=0\] if $f(t)>0$.  
Since the zeroes of $f$ are  $\pi$ apart,
$\gamma$ 
spends time $\pi$ on every visit to $\Star_v$.

After leaving $\Star_v$,
the local geodesic $\gamma$ has to enter another simplex, 
say $\sigma'$, 
which has a vertex $v'$ not joined to $v$ by an edge.

Since $\tau$ is flag, we have that the stars $\Star_v$ and $\Star_{v'}$
do not overlap.
The same argument as above shows that $\gamma$ spends time $\pi$ on every visit to $\Star_{v'}$.
Therefore the total length of $\gamma$ is at least $2\cdot\pi$.
\qeds

\begin{thm}{Exercise}
Assume that a spherical polyhedral space $\spc{P}$
admits a triangulation $\tau$ such that all dihedral angles of all simplexes are at least $\tfrac\pi2$.
Show that $\spc{P}\in\cCat{}{1}$
if $\tau$ is flag.
\end{thm}

\begin{thm}{Exercise}
Let $\spc{U}\in \cCat{}{0}$
and $\phi_1,\phi_2,\dots,\phi_k\:\spc{U}\to \spc{U}$ be commuting short retractions; 
that is 
\begin{itemize}
\item $\phi_i\circ\phi_i=\phi_i$ for each $i$;
\item $\phi_i\circ\phi_j=\phi_j\circ\phi_i$ for any $i$ and $j$;
\item $\dist{\phi_i(x)}{\phi_i(y)}{\spc{U}}\le \dist{x}{y}{\spc{U}}$ for each $i$ and any $x,y\in\spc{U}$.
\end{itemize}
Set $A_i=\Im \phi_i$ for all $i$;
note that each $A_i$ is a weakly convex set.

Assume $\Gamma$ is a finite graph 
(without loops and multiple edges) 
with edges labeled by $1,2,\dots, n$.
Denote by $\spc{U}^\Gamma$ the space obtained by taking 
a copy of $\spc{U}$ for each vertex of $\Gamma$ and 
gluing two such copies along $A_i$ if the corresponding vertices are joint by an edge labeled by $i$.

Show that $\spc{U}^\Gamma\in\cCat{}{0}$
\end{thm}

\parbf{The space of trees.}
The following construction is given by Billera, Holmes and Vogtmann in \cite{BHV}.

Let $\spc{T}_n$ be the set of all metric trees with $n$ end-vertices
labeled by $a_1,\dots,a_n$.
To describe one tree in $\spc{T}_n$ we may fix a topological tree $\tau$ with end vertices $a_1,\dots,a_n$ and all the other verices of degree 3 
and prescribe the lengths of $2\cdot n-3$ edges.
If the lengh of an edge is $0$, we assume that edge degenerates;
such a tree can be also decribed using a different topological tree $\tau'$.
The subset of $\spc{T}_n$ corresponding to the given topological tree $\tau$ can be identified with a convex closed cone in  $\mathbb{R}^{2\cdot n-3}$.
Equip each such subset with the metric induced from $\mathbb{R}^{2\cdot n-3}$ and consider the legth metric on $\spc{T}_n$ induced by these metrics.

\begin{thm}{Exercise}
Show that $\spc{T}_n$ with the described metric is a $\cCat{}0$ space.
\end{thm}



\section{Cubical complexes}

The definition of a cubical complex
mostly repeats the definition of a simplicial complex, 
with simplices replaced by cubes.

Formally, a \index{cubical complex}\emph{cubical complex} is defined as a subcomplex 
of the unit cube in the Euclidean space of large dimension;
that is, a collection of faces of cube
such that together with each face it contains all its sub-faces.
Each cube face in this collection 
will be called a \emph{cube} of the cubical complex.

Note that according to this definition, 
any cubical complex is finite,
that is, contains a finite number of cubes.

The union of all the cubes in a cubical complex $\spc{Q}$ will be called its \emph{underlying space};
it will be denoted by $\spc{Q}$ or by $\ushort{\spc{Q}}$ 
if we need to emphasize that we are talking about a set, 
not a complex.
A homeomorphism from $\ushort{\spc{Q}}$ to a topological space $\spc{X}$ is called a \index{cubulation}\emph{cubulation of} $\spc{X}$.

The underlying space of a cubical complex $\spc{Q}$ will be always considered with the length metric
induced from $\RR^N$.
In particular, with this metric, 
each cube of $\spc{Q}$ is isometric to the unit cube of the same dimension.

It is straightforward to construct a triangulation 
of $\ushort{\spc{Q}}$ 
such that each simplex is isometric to a Euclidean simplex.
In particular $\ushort{\spc{Q}}$ is a Euclidean polyhedral space.

The link of each cube in a cubical complex admits a natural 
all-right triangulation; 
each simplex corresponds to an adjusted cube.

\parbf{Cubical analog of a simplicial complex.}
Let $\spc{S}$ be a simplicial complex and $\{v_1,\dots,v_N\}$ be the set of its vertexes.

Consider $\RR^N$ with the standard basis $\{e_1,\dots,e_N\}$.
Denote by $\square^N$ the standard unit cube in $\RR^N$;
that is 
\[\square^N=\set{(x_1,\dots,x_n)\in \RR^N}{0\le x_i\le 1\ \text{for each}\ i}.\]

Given a $\kay$-dimensional simplex $\<v_{i_0},\dots,v_{i_\kay}\>$ in $\spc{S}$, 
mark the $(\kay\z+1)$-dimensional faces in $\square^N$ (there are  $2^{N-\kay}$ of them)
which are parallel to the coordinate $(k+1)$-plane 
spanned by $e_{i_0},\dots,e_{i_\kay}$.


Note that the set of all marked faces of $\square^{N}$
forms a cubical complex;
it will be called 
the \index{cubical analog}\emph{cubical analog} of $\spc{S}$
and will be denoted as $\square_\spc{S}$.

Note that if a simplicial complex is connected then so is its cubical analog.

\begin{thm}{Proposition}\label{prob:cubical-analog}
Let $\spc{S}$ be a connected simplicial complex
and $\spc{Q}=\square_{\spc{S}}$ be its cubical analog.
Then $\ushort{\spc{Q}}$ is connected 
and the link of any vertex of $\spc{Q}$
is isometric to  ${\spc{S}}$
equipped with the spherical right-angled metric.

In particular, if $\spc{S}$ is a flag complex 
then $\spc{Q}$ is a locally $\cCat{}{0}$ space 
and therefore its universal cover $\tilde{\spc{Q}}$ is a $\cCat{}{0}$ space.
\end{thm}

\parit{Proof.}
The first part of the proposition follows 
from the construction above.

If ${\spc{S}}$ is flag, 
then by Flag condition (\ref{thm:flag}) 
the link of any cube in $\spc{Q}$ is a $\cCat{}{1}$ space.
Therefore, by cone construction (\ref{thm:warp-curv-bound:cbb:a})
$\spc{Q}$
is locally $\cCat{}{0}$ space.
It remains to apply Hadamard--Cartan theorem 
\ref{thm:hadamard-cartan}.
\qeds

From Proposition \ref{prob:cubical-analog}, 
it follows that the cubical analog
of any flag complex is aspherical.
The following exercise states that the  converse also holds, see \cite[5.4]{davis-survey}.

\begin{thm}{Exercise}
A simplicial complex is flag 
if and only if its cubical analog is aspherical.
\end{thm}

\section{Exotic aspherical manifolds}


By Hadamard--Cartan theorem (\ref{thm:hadamard-cartan})
any $\cCat{}{0}$ space is contractible.
Therefore any complete locally $\cCat{}{0}$ space 
is \emph{aspherical};
that is, they have contractible universal covers.
This observation can be used to construct examples of aspherical spaces. 

Let $\spc X$ be a proper topological space.
Recall that $\spc X$ is called 
\index{simply connected space at infinity}\emph{simply connected at infinity} 
if for any compact set $K\subset\spc X$
there is a bigger compact set $K'\supset K$
such that  $\spc X\backslash K'$ is path connected 
and any loop which lies in $\spc X\backslash K'$
is null-homotopic in  $\spc X\backslash K$.

Recall that path connected spaces are not empty by definition.
Therefore compact spaces are not simply connected at infinity.

The following statement was proved by Michael Davis in \cite{davis-noneuclidean}; 
his survey \cite{davis-survey} discusses 
a number of related constructions.

\begin{thm}{Proposition}\label{prop:aspherical}
For any  $m\ge 4$ there is a closed aspherical 
$m$-dimensional piecewise linear manifold
whose universal cover is not simply connected at infinity.

In particular, the universal cover of this manifold 
is not homeomorphic to the $m$-dimensional Euclidean space.
\end{thm}

The proof requires the following lemma.

\begin{thm}{Lemma}\label{lem:example-pi_infty}
Let $\spc{S}$ be a flag complex,
$\spc{Q}=\square_{\spc{S}}$ be its cubical analog
and $\tilde{\spc{Q}}$ be the universal cover of $\spc{Q}$.

Assume  $\tilde{\spc{Q}}$ is simply connected at infinity.
Then $\spc{S}$ is simply connected.
\end{thm}

The proof below can be modified to prove the following extension. 

\begin{thm}{Exercise}\label{ex:example-pi_infty-new}
Under the assumptions of the Lemma~\ref{lem:example-pi_infty}, 
for any vertex $v$ in $\spc{S}$
the complement $\spc{S}\backslash\{v\}$ is simply connected.
\end{thm}


\parit{Proof of Lemma~\ref{lem:example-pi_infty}.}
Assume $\spc{S}$ is not simply connected.
Choose a shortest noncontractible circle $\gamma\:\mathbb{S}^1\to\spc{S}$ formed by the edges of $\spc{S}$.

Note that $\gamma$ forms a 1-dimensional subcompelex of $\spc{S}$ which is a closed local geodesic.
Denote by $G$ the subcomplex of $\spc{Q}$ which corresponds to $\gamma$.

Fix a vertex $v\in G$;
let $G_v$ be the connected component of $G$ containing $v$.
Let $\tilde G$ be 
the inverse image 
of $G_v$ in $\tilde{\spc{Q}}$
for the universal cover $\tilde{\spc{Q}}\to \spc{Q}$.
Fix a point $\tilde v\in\tilde G$ in the inverse image of $v$.
 
Note that 
\begin{clm}{}\label{tilde-G-convex}
$\tilde G$ forms a convex set in $\tilde{\spc{Q}}$.
\end{clm}


Indeed, according to Proposition \ref{prob:cubical-analog},
$\tilde{\spc{Q}}$ is $\cCat{}{0}$.
By ???, %here there should be ref to locconvex+connected => convex in CAT(0)
it is sufficient to show that $\tilde G$ is locally convex in $\tilde{\spc{Q}}$,
or equivalently $G$ is locally convex in $\spc{Q}$.

\begin{wrapfigure}[5]{r}{30mm}
\begin{lpic}[t(-7mm),b(0mm),r(0mm),l(0mm)]{pics/gamma-in-S(1)}
\lbl[t]{15,5;$\xi$}
\lbl[b]{15,17;$\zeta$}
\lbl[r]{14.5,10.5;$e$}
\end{lpic}
\end{wrapfigure}

The latter can only fail if $\gamma$ passes through two vertices, say $\xi$ to $\zeta$ in $\spc{S}$,
which are joined by an edge not in $\gamma$; 
denote this edge by $e$.

Each edge of $\spc{S}$ has length $\tfrac\pi2$.
Therefore each of two circles formed by $e$ and an arc of $\gamma$
from $\xi$ to $\zeta$ is shorter that $\gamma$.
Moreover,
at least one of them is noncontractable 
since $\gamma$ is not.
That is, 
$\gamma$ is not a shortest noncontractible circle 
--- a contradiction.
\claimqeds

Further, note that 
$\tilde G$ is homeomorphic to the plane, 
since $\tilde G$ is 
a 2-dimensional manifold without boundary which 
by the above is $\cCat{}{0}$ and hence is contractible.

Denote by $C_R$ the circle of radius $R$ in $\tilde G$ centered at $\tilde v$.
All $C_R$ are homotopic to each other in $\tilde G\backslash\{\tilde v\}$ and therefore in $\tilde{\spc{Q}}\backslash \{\tilde v\}$.

Note that the map $\tilde{\spc{Q}}\backslash \{\tilde v\}\to \spc{S}$
defined by $x\mapsto\dir{\tilde v}{x}$ maps $C_R$ to a circle homotopic to $\gamma$.
Therefore $C_R$ is not contractible in $\tilde{\spc{Q}}\backslash \{\tilde v\}$.

In particular, 
$C_R$ is not contactable in $\tilde{\spc{Q}}\backslash K$
if $K\supset \tilde v$.
If $R$ is large, 
the circle $C_R$  
lies outside of any compact set $K'$ in $\tilde{\spc{Q}}$.
It follows that $\tilde{\spc{Q}}$ is not simply connected at infinity, a contradiction.
\qeds

\parit{Proof of Proposition~\ref{prop:aspherical}.}
Let $\Sigma^{m-1}$ be an $(m-1)$-dmensional smooth homology sphere which is not simply connected and bounds a contractible smooth compact $m$-dimensional manifold $\spc{W}$. 

For $m\ge 5$ the existence of such $(\spc{W}, \Sigma)$ follows from \cite{kervaire}. 
For $m=4$ it follows from the construction in \cite{mazur}.

Pick any smooth triangulation $\tau$ of $W$ and let $\spc{S}$ be the resulting subcomplex which triangulates $\Sigma$.


We can assume that $\spc{S}$ is flag; 
otherwise pass to the barycentric subdivision 
of $\tau$ and apply Exercise~\ref{ex:baricenric-flag}.


Let $\spc{Q}=\square_{\spc{S}}$ be the cubical analog of $\spc{S}$.

By Proposition~\ref{prob:cubical-analog},
$\spc{Q}$ is a homology manifold.
It follows that $\spc{Q}$ is a piecewise linear manifold 
with a finite number of singularities at its vertices.


Removing a small neighborhood of each vertex in $\spc{Q}$,
we can obtain a piecewise linear manifold 
which boundary is formed by several copies of $\Sigma$.
To each copy of $\Sigma$, 
glue a copy of  $\spc{W}$ along its boundary.
This results in a  closed piecewise linear manifold 
$\spc{M}$ which is homotopically equivalent to $\spc{Q}$.

Finally, by the Lemma~\ref{lem:example-pi_infty},  
the universal cover $\tilde{\spc{Q}}$ of $\spc{Q}$
is not simply connected at infinity.

The same holds for 
the universal cover $\tilde{\spc{M}}$ of $\spc{M}$.
The later follows since homotopy equivalences 
$f\: \spc Q\to \spc M$ and $g\:\spc M\to \spc Q$ 
lift to proper maps 
$\tilde f \: \tilde{\spc{Q}}\to \tilde{\spc{M}}$
and $\tilde g \: \tilde{\spc{M}}\to \tilde{\spc{Q}}$. 
\qeds

The following is a geometric analog of the proposition above;
it was proved by Ancel, Davis and Guilbault in \cite{ADG}. 
%???first names 
%??? I plan to make exercise form this proposition and move the proof of lemma to the end


\begin{thm}{Proposition}\label{prop:loc-CAT-mnfld}
Given $m\ge 5$ there is a Euclidean polyhedral space $\spc{P}$ such that
\begin{subthm}{}
$\spc{P}$ is homeomorphic to a closed $m$-dimensional manifold.
\end{subthm}

\begin{subthm}{}
$\curv\spc{P}\le 0$
\end{subthm}

\begin{subthm}{}
The universal cover of $\spc{P}$ is not simply connected at infinity.
\end{subthm}
\end{thm}

It worth to note that any complete simply connected Riemannian manifold with nonpositive curvature is homeomorphich to the Euclidean space of the same dimension.
In fact by Hadamard--Cartan theorem
(\ref{thm:hadamard-cartan}), 
the exponential map at one point of such manifold is a homeomorphism.
In particular there is no Riemannian analog of the proposition.
Moreover, according to a theorem of Stone, see \cite{stone, davis-januszkiewicz}, there is no piecewise linear analog of the propositon; 
that is the homeomorphism to a manifold in the proposition can not be made piecewise linear. 

\parit{Proof.}
Apply Exercise~\ref{ex:example-pi_infty-new} to the barycentric subdivision of the simplicial complex $\spc{S}$ provided by Exercise~\ref{ex:funny-S}.
\qeds

\begin{thm}{Exercise}\label{ex:funny-S}
Given a positive integer $m\ge 5$
there is an $(m-1)$-dimensional simplicial complex $\spc{S}$ such that $\Cone\spc{S}$ is homeomorphic to $\EE^m$
and $\pi_1(\spc{S}\backslash\{v\})\ne0$ for some vertex $v$ in $\spc{S}$.
\end{thm}


