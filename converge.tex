%%!TEX root = all.tex
\chapter{Convergence of metric spaces}

In this section we discuss the
Gromov--Hausdorff convergence of metric spaces.

It seems that Hausdorff convergence was first introduced by Felix Hausdorff \cite{hausdorff}
and couple of years later an equivalent definition was given by Wilhelm Blaschke \cite{blaschke}.
Further a refinement of this definition was introduced by Zdeněk Frolík \cite{frolik}
and then rediscovered by Robert Wijsman \cite{wijsman},
however this refinement was a step in the direction of the so-called \emph{closed convergence} introduced by Hausdorff in the original book. 
By that reason we call it Hausdorff convergence
instead of
\emph{Hausdorff--Blascke--Frol\'{\i}k--Wijsman convergence}.

The Gromov--Hausdorff convergence was first introduced by David Edwards \cite{edwards}
and rediscovered later by Mikhael Gromov \cite{gromov-polynomial-growth}.
It was essential tool in Gromov's proof that any group of polynomial growth has is a nilpontant subgroup of finite index.
Convergences of metric spaces
was considered earlier, but each time
the definition was limited to very specific type of problems.

The definition of Gromov--Hausdorff convergence or metric spaces use 
the Hausdorff convergence.
It means that the sequence of metric spaces admit a sequence of distance preserving embedding into huge metric space so that the their
images converge in Hausdorff sense.
The Gromov--Hausdorff convergence is also defined in a nonstandard way.

\section{Convergence of subsets}

Let $\spc{X}$ be a metric space and $A\subset \spc{X}$.
We will denote by $\distfun Ax$ the distance from $A$ to a point $x$ in $\spc{X}$;
that is,
$$\distfun Ax\df\inf\set{\dist{a}{x}{}}{a\in A}.$$

\begin{thm}{Definition of Hausdorff convergence}\label{def:hausdorff-coverge}
Given a sequence of nonempty closed sets $(A_n)_{n=1}^\infty$ in a nonempty metric space $\spc{X}$ 
a closed set $A_\infty\subset \spc{X}$ is called Hausdorff limit of $(A_n)_{n=1}^\infty$,
briefly $A_n\Hto A_\infty$ if 
$$\distfun{A_n}x\to\distfun{A_\infty}x\quad \text{as}\quad n\to\infty$$
for any fixed $x\in \spc{X}$.

In this case the sequence of closed sets $(A_n)_{n=1}^\infty$ is called \emph{converging} or \emph{converging in the sense of Hausdorff}.
\end{thm}

%One can extend the definition slightly.
%It is natural to assume that $\dist{\emptyset}{x}{}=\infty$ for every $x$.
%Therefore if $\dist_{A_n}(x)\to\infty$ for some (and therefore any) $x\in \spc{X}$, then we can assume that the Hausdorff limit of $A_n$ is the empty set.
%Using this definition we can remove the condition 




\begin{thm}{Selection theorem}
Let $\spc{X}$ be a proper space
and $(A_n)_{n=1}^\infty$ be a sequence of closed sets in $\spc{X}$.
Assume that for some (and therefore any) point  $x\in\spc{X}$ 
the sequence $\distfun{A_n}x$ is bounded.
Then  $(A_n)_{n=1}^\infty$ has a converging subsequence in the sense of Hausdorff.
\end{thm}

\parit{Proof.}
Since $\spc{X}$ is proper,
we can choose a countable dense set $\{x_1,x_2, \dots\}$ in $\spc{X}$.
Note that the sequence $a_n=\distfun{A_n}x_\kay$ is bounded for each $\kay$. 
Therefore, passing to a subsequence of $(A_n)_{n=1}^\infty$,
we can assume that $\distfun{A_n}x_\kay$ is converging as $n\to\infty$ for any fixed $\kay$.

Note that for each $n$, the function $\distfun{A_n}\:\spc{X}\to\RR$ is 1-Lipschitz and nonnegative.
Therefore the sequence $\distfun{A_n}$ converges pointwise to a 1-Lipschitz nonnegative function $f\:\spc{X}\to\RR$.

Set $A_\infty=f^{-1}(0)$.
Since $f$ is 1-Lipschitz, 
$\distfun{A_\infty}y\ge f(y)$ for any $y\in \spc{X}$.
It remains to show that $\distfun{A_\infty}y\le f(y)$ for any $y$.

Assume contrary;
that is, $f(z)<R<\distfun{A_\infty}z$ for some $z\in \spc{X}$ and $R>0$.
Then for any sufficiently large $n$ there is a point $z_n\in A_n$ such that
$\dist{x}{z_n}{}\le R$.
Since $\spc{X}$ is proper, we can pass to a partial limit $z_\infty$ of $z_n$ as $n\to\infty$.

Clearly that $f(z_\infty)=0$; that is, $z_\infty\in A_\infty$.
On the other hand, 
\[\distfun{A_\infty}y\le\dist{z_\infty}{y}{}\le R<\distfun{A_\infty}y,\] 
a contradiction.
\qeds

\section{Convergence of spaces}

\begin{thm}{Definition}\label{def:comp-metr}
Let $\set{\spc{X}_\alpha}{\alpha\in\IndexSet}$ be a set of metric spaces.
A metric space $\bm{X}$
is called \emph{common space}\index{common space} of $\set{\spc{X}_\alpha}{\alpha\in\IndexSet}$ if its underlying set is formed by the disjoint union $$\bigsqcup_{\alpha\in\IndexSet} \spc{X}_\alpha$$ 
and each inclusion $\iota_\alpha\:\spc{X}_\alpha\hookrightarrow\bm{X}$
is distance preserving.
\end{thm}

\begin{thm}{Definition}\label{def:GH}
Let $\bm{X}$ be a common space for proper metric spaces
$\spc{X}_1,\spc{X}_2,\dots$ and $\spc{X}_\infty$.
Assume that $\spc{X}_n$ forms an open set in $\bm{X}$ for each $n<\infty$ and 
$\spc{X}_n\Hto \spc{X}_\infty$ in $\bm{X}$ as $n\to\infty$.

In this case the topology $\GH$ of $\bm{X}$ is called \emph{Gromov--Hausdorff convergence}\index{Gromov--Hausdorff convergence}
and we write $\spc{X}_n\GHto \spc{X}_\infty$ or $\spc{X}_n\xGHto{\GH} \spc{X}_\infty$;
the later notation is used if we need to consider specific Gromov--Hausdorff convergene $\GH$.
The space $\spc{X}_\infty$ is called the limit space of the sequence $(\spc{X}_n)$ along $\GH$.
\end{thm}

Once we write $\spc{X}_n\GHto \spc{X}_\infty$ we mean that we made a choice of Gromov--Hausdorff convergence.

Note that for a fixed sequence of metric spaces $\spc{X}_1,\spc{X}_2,\dots$ one may construct different Gromov--Hausdorff convergences, say $\spc{X}_n\xGHto{\GH} \spc{X}_\infty$ and $\spc{X}_n\xGHto{\GH'} \spc{X}_\infty'$  and their limit spaces $\spc{X}_\infty$ and $\spc{X}_\infty'$ need not to be isometric to each other. 
For example, for the constant sequence $\spc{X}_n\iso\RR_{\ge0}$, 
one may take $\spc{X}_\infty\iso\RR_{\ge0}$.
In this case a point in the disjont space $\bm{X}$ can be thought as a pair $(x,n)\in \RR_{\ge}\times (\ZZ_>\cup \{\infty\})$ 
and the metric on $\bm{X}$ can be defined the following way
$$\dist{(x,n)}{(y,m)}{\bm{X}}\df|\tfrac1n-\tfrac1m|+|x-y|,$$
where we assume that $0=\tfrac1\infty$.
On the other hand, one can take $\spc{X}_\infty'\iso\RR$,
and consider the metric
\begin{align*}
\dist{(x,n)}{(y,m)}{\bm{X}'}
&=|\tfrac1n-\tfrac1m|+|(x-n)-(y-m)|,
\\
\dist{(x,n)}{(y,\infty)}{\bm{X}'}
&=\tfrac1n+|(x-n)-y|,
\\
\dist{(x,\infty)}{(y,\infty)}{\bm{X}'}
&=|x-y|.
\end{align*}
where $n, m<\infty$.

\begin{thm}{Induced convergences}
Assume $\spc{X}_n\xGHto{\GH}\spc{X}_\infty$,
and $\bm{X}$ as in the definition \ref{def:GH}
and $\iota_n\:\spc{X}_n\hookrightarrow\bm{X}$, $\iota_\infty\:\spc{X}_\infty\hookrightarrow\bm{X}$ are corresponding inclusions.

\begin{subthm}{}
A sequence of points $x_n\in\spc{X}_n$ converges to $x_\infty\in\spc{X}_\infty$ (briefly, $x_n\to x_\infty$ or $x_n\xto{\GH} x_\infty$) 
if $\dist{x_n}{x_\infty}{\bm{X}}\to 0$.
\end{subthm}

\begin{subthm}{}
A sequence of closed sets 
$\mathfrak{C}_n\subset \spc{X}_n$ 
converges to a closed  set 
$\mathfrak{C}_\infty\subset \spc{X}_\infty$ (briefly, $\mathfrak{C}_n\to \mathfrak{C}_\infty$ or $\mathfrak{C}_n\xto{\GH} \mathfrak{C}_\infty$)
if $\mathfrak{C}_n\Hto\mathfrak{C}_\infty$ as subsets of $\bm{X}$.
\end{subthm}

\begin{subthm}{}
A sequence of open sets $\Omega_n\subset \spc{X}_n$ 
converges to an open set $\Omega_\infty\subset \spc{X}_\infty$
(briefly, $\Omega_n\to \Omega_\infty$ 
or $\Omega_n\xto{\GH} \Omega_\infty$)
if the complements $\spc{X}_n\backslash \Omega_n$ converge to the complement $\spc{X}_\infty\backslash \Omega_\infty$ as closed sets. %what if the limit is everything???
\end{subthm}


\begin{subthm}{} Let $\spc{X}_n\xGHto{\GH} \spc{X}_\infty$ and $\spc{Y}_n\xGHto{\theta} \spc{Y}_\infty$. 
A sequence of submaps, $\map_n\:\spc{X}_n\subto \spc{Y}_n$ converges to a submap $\map_\infty\:\spc{X}_\infty\subto \spc{Y}_\infty$ if the following conditions holds
\begin{itemize}
\item $\Dom\map_n\to \Dom\map_\infty$ as a sequence of open sets.

\item For any $x_\infty\in \Dom \map_\infty$ and any sequence $x_n\in \spc{X}_n$ such that $x_n\to x_\infty$
\[\spc{Y}_n\ni \map _n(x_n)\xto\theta \map_\infty(x_\infty)\in\spc{Y}_\infty\] 
as $n\to\infty$.
\end{itemize}
\end{subthm}

\begin{subthm}{} Given a sequence of measures $\mu_n$ on $\spc{X}_n$
denote by $\iota_n\#\mu_n$ the pushforward measures on $\bm{X}$.
We say that $(\mu_n)$ weakly converges to a measure $\mu_\infty$ on $\spc{X}_\infty$ 
(brefly, $\mu_n\xto{}
\mu_\infty$ or $\mu_n\xto{\GH}
\mu_\infty$) 
if the pushforward measures $\iota_n\#\mu_n$ weakly converge to the $\iota_\infty\#\mu_\infty$.

In other words, 
if for any continuous function $\phi\:\bm{X}\to\RR$ with a compact support, we have 
\[\int\limits_{\spc{X}_n} \phi\circ\iota_n(x)
\cdot
\dd_x\mu_n
\to 
\int\limits_{\spc{X}_\infty} \phi\circ\iota_\infty(x)
\cdot\dd_x\mu_\infty\]
as $n\to\infty$.
\end{subthm}
\end{thm}

\parbf{Liftings.}
Given a Gromov--Hausdorff convergence 
$\spc{X}_n\GHto \spc{X}_\infty$
and a point $p_\infty\in\spc{X}_\infty$ any sequence of points $p_n\in\spc{X}_n$ such that $p_n\GHto p$  will be called \emph{lifting}\index{lifting of a point} of $p_\infty$.
In this case the point $p_n\in \spc{X}_n$ will be called a lifting of $p_\infty$ in $\spc{X}_n$.
In this case we say that $\distfun{p_n}{}{}\:\spc{X}_n\to \RR$ 
is a  \emph{lifting}\index{lifting of a distance function} 
of the distance function $\distfun{p}{}{}\:\spc{X}\to \RR$.
Obviousely $\distfun{p_n}{}{}\GHto\distfun{p}{}{}$.

Note that liftings are not uniquely defined.
We will be interested in the properties of liftings for sufficiently large $n$.

The same way we can talk about liftings of a point array
$\bm{p}_\infty\z =(p_\infty^1,p_\infty^2,\dots,p_\infty^\kay)$ in $\spc{X}_\infty$
and about the corresponding distance map 
$\distfun{\bm{p}_\infty}{}{}\:\spc{X}_\infty\to\RR^\kay$
$$\distfun{\bm{p}_\infty}{}{}\:x\mapsto(\dist{p_\infty^1}{x}{},\dist{p_\infty^2}{x}{},\dots,\dist{p_\infty^k}{x}{})$$

\section{Gromov's selection theorem}

\begin{thm}{Gromov's selection theorem}\label{thm:gromov-selection}
Let $\spc{X}_n$ be a sequence of proper metric spaces 
with marked points $x_n\in \spc{X}_n$.
Assume that for any fixed $R,\eps>0$ there is $N=N(R,\eps)\in\ZZ_>$ 
such that for each $n$
the ball $\cBall[x_n,R]\subset \spc{X}_n$ admits a finite $\eps$-net with at most $N$ points.
Then there is a subsequence of $\spc{X}_n$ that admit a Gromov--Hausdorff convergence 
such that the sequence of marked points $x_n\in\spc{X}_n$ converges.
\end{thm}

\parit{Proof.}
Note that there is a sequence of integers $M_1<M_2<\dots$
such that in each space $\spc{X}_n$
there is a sequence of points $z_{i,n}\in\spc{X}_n$ such that
\[\dist{z_{i,n}}{x_n}{\spc{X}_n}\le \kay+1\quad \text{if}\quad i\le M_\kay\]
and
the points $z_{1,n},\dots,z_{M_\kay,n}$ form an $\tfrac1\kay$-net in $\cBall[x_n,\kay]_{\spc{X}_n}$.
The existence of the sequence $M_1,M_2,\dots$ follows from the main assumption in the theorem.

Passing to a subsequence, we can assume that the sequence \[\ell_n=\dist{z_{i,n}}{z_{j,n}}{\spc{X}_n}\] 
converges for any $i$ and $j$.

Let us consider countable set of points $\spc{W}=\{w_1,w_2,\dots\}$
equipped with the pseudometric defined as 
\[\dist{w_i}{w_j}{\spc{W}}
=
\lim_{n\to\infty}\dist{z_{i,n}}{z_{j,n}}{\spc{X}_n}.\]
Let $\hat{\spc{W}}$ be the metric space corresponding to $\spc{W}$.
Denote by
$\spc{X}_\infty$ the completion of $\hat{\spc{W}}$.

It remains to show that there is a Gromov--Hausdorff convergence 
$\spc{X}_n\GHto\spc{X}_\infty$ such that the sequence $x_n\in\spc{X}_n$ converges.
To prove it, we need to construct a metric on the disjoint union of \[\bm{X}=\spc{X}_\infty\sqcup\spc{X}_1\sqcup\spc{X}_2\sqcup\dots\] 
satisfies definitions \ref{def:comp-metr} and \ref{def:GH}.

The metric can be defined as follows.
Fix a sequence $\eps_\kay\to0+$
and let $N_\kay$ be the minimal integer such that
\[\left|\dist{w_i}{w_j}{\spc{W}}
-
\dist{z_{i,n}}{z_{j,n}}{\spc{X}_n}\right|<\eps_\kay
\]
if $i,j\le N_\kay$ and $n\ge N_\kay$.
\[\dist{x}{y}{\bm{X}}=
\begin{cases}
\dist{x}{y}{\spc{X}_n}& x,y\in \spc{X}_n
\\
\dist{x}{y}{\spc{X}_\infty}& x,y\in \spc{X}_\infty
\end{cases}
\]

Set,
\[
\delta(z_{i,n},z_{i,\infty})
=
\sup\set{\left|\dist{z_{i,m}}{z_{j,m}}{\spc{X}_m}
-
\dist{z_{i,\infty}}{z_{j,\infty}}{\spc{X}_\infty}\right|}
{j\le i,\, m\ge n}.\]
Let us equip $\bm{X}$ with the maximal metric such that each inclusion $\iota_n\:\spc{X}_n$ 
is short 
and the $\dist{z_{i,n}}{z_{i,\infty}}{\bm{X}}\le \delta(z_{i,n},z_{i,\infty})$ for any $i$ and $n$.
In other words, let us define 
\qeds

\section{Convergence of compact spaces.}

\begin{thm}{Definition}
Let $\spc{X}$ and $\spc{Y}$ be metric space,
a map $f\:\spc{X}\to\spc{Y}$
is called \emph{$\eps$-isometry}\index{isometry!$\eps$-siometry}
if the following two condition hold:
\begin{subthm}{}
$\Im f$ is an $\eps$-net of $\spc{Y}$
\end{subthm}

\begin{subthm}{}
$\dist{f(x)}{f(x')}{\spc{Y}}\lege\dist{x}{x'}{\spc{X}}\pm\eps$ for any $x,x'\in\spc{X}$
\end{subthm}

\end{thm}

\begin{thm}{Lemma}\label{lem:almost-isom}
Let $\spc{X}_1,\spc{X}_2,\dots$ and $\spc{X}_\infty$ be metric spaces and $\eps_n\to\0+$ as $n\to\infty$.
Suppose that either 

\begin{subthm}{}
for each $n$ there is an $\eps_n$-isometry $f_n\:\spc{X}_n\to\spc{X}_\infty$, or
\end{subthm}

\begin{subthm}{}
for each $n$ there is an $\eps_n$-isometry $h_n\:\spc{X}_\infty\to\spc{X}_n$.
\end{subthm}

Then there is a Gromov--Hausdorff convergence $\spc{X}_n\GHto \spc{X}_\infty$.
\end{thm}


\parit{Proof.}
Let us construct a common space $\bm{X}$ for the spaces $\spc{X}_1,\spc{X}_2,\dots$ and $\spc{X}_\infty$
by taking the metric $\rho$ on the disjoint union $\spc{X}_\infty\sqcup\spc{X}_1\sqcup\spc{X}_2\sqcup\dots$ that is defined the following way:
\begin{align*}
\rho(x_n,y_n)&=\dist{x_n}{y_n}{\spc{X}_n},\quad \rho(x_\infty,y_\infty)=\dist{x_\infty}{y_\infty}{\spc{X}_\infty},
\\
\rho(x_n,x_\infty)&=\inf\set{\dist{x_n}{y_n}{\spc{X}_n}+\eps_n+\dist{x_\infty}{f(y_n)}{\spc{X}_\infty}}{{y_n}\in \spc{X}_n},
\\
\rho(x_n,x_m)&=\inf\set{\rho(x_n,y_\infty)+\rho(x_m,y_\infty)}{y_\infty\in\spc{X}_\infty},
\end{align*}
where we assume that $x_m\in \spc{X}_m$, $x_n\in \spc{X}_n$, and $x_\infty\in \spc{X}_\infty$. 

It remains to observe that $\rho$ is indeed a metric and 
$\spc{X}_n\Hto \spc{X}_\infty$ in~$\bm{X}$.

The proof of the second part is analogous; one only need to change one line in the definition of $\rho$ to the following:
\[\rho(x_n,x_\infty)=\inf\set{\dist{x_n}{h(y_\infty)}{\spc{X}_n}+\eps_n+\dist{x_\infty}{y_\infty}{\spc{X}_\infty}}{{y_\infty}\in \spc{X}_\infty}.\]
\qedsf

\begin{thm}{Definition}
 Given two compact spaces $\spc{X}$ and $\spc{Y}$, we will write 
\begin{itemize}
\item $\spc{X}\le \spc{Y}$ if there is a non-contracting map $\map\:\spc{X}\to \spc{Y}$.
\item $\spc{X}\le \spc{Y}+\eps$ if there is a map $\map\:\spc{X}\to \spc{Y}$ such that for any $x,x'\in \spc{X}$ we have
\[\dist{x}{x'}{}\le \dist{\map(x)}{\map(x')}{}+\eps.\]
\end{itemize}

\end{thm}

\begin{thm}{Lemma}\label{lem:>=-isometry}
Let $\spc{X}$ and $\spc{Y}$ be two metric spaces and $\spc{X}$ is compact, then
\[
\spc{X}\ge\spc{Y}\ge\spc{X}
\quad \iff\quad 
\spc{X}\iso\spc{Y}.
\]

\end{thm}

The following proof was suggested by Travis Morrison.

\parit{Proof.}
Let $f\: \spc{X} \to \spc{Y}$ 
and $g\: \spc{Y} \to \spc{X}$ be non contracting mappings.
It is sufficient to prove that $h  = g\circ f\:\spc{X}\to \spc{X}$ is an isometry. 

Given any pair of points $x,y\in \spc{X}$, 
set $x_n\z=h^{\circ n}(x)$ and $y_n\z=h^{\circ n}(y)$.
Since $\spc{X}$ is compact, one can choose an incresing sequence of integers $n_\kay$
such that both sequences $(x_{n_i})_{i=1}^\infty$ and $(y_{n_i})_{i=1}^\infty$
converge.
In particular, both of these sequences  are converging in itself;
that is,
\[
\dist{x_{n_i}}{x_{n_j}}{},\dist{y_{n_i}}{y_{n_j}}{}\to 0
\]
as $\min\{i,j\}\to\infty$.
Sinse $h$ is nonconotracting, we get
\[
\dist{x}{x_{|n_i-n_j|}}{}\le \dist{x_{n_i}}{x_{n_j}}{}.
\]
It follows that  
there is a sequence $m_i\to\infty$ such that
\[
x_{m_i}\to x\quad \text{and}\quad y_{m_i}\to y\quad \text{as}\quad \kay\to\infty.
\eqlbl{eq:x_l->x}
\]

Let $\ell_n=\dist{x_n}{y_n}{}$.
Since $h$ is noncontracting, $(\ell_n)$ is a nondecreasing sequence.
On the other hand, 
from \ref{eq:x_l->x}, it follows that $\ell_{m_i}\to\dist{x}{y}{}=\ell_0$ as $m_i\to\infty$;
that is, $(\ell_n)$ is a constant sequece.
In particular $\ell_0=\ell_1$ for any $x$ and $y\in \spc{X}$;
that is, $h$ is distance preserving map.

Therefore $h(X)$ is isometric to $\spc{X}$.
From \ref{eq:x_l->x}, we get that $h(\spc{X})$ is everywhere dense.
Since $X$ is compact, we get that $h(\spc{X})=\spc{X}$.
\qeds




The \emph{Gromov--Hausdorff distance}\index{Gromov--Hausdorff distance} between isometry classes of compact metric spaces $\spc{X}$ and $\spc{Y}$, is defined by
\[\GHdist(\spc{X},\spc{Y})
\df
\inf\set{\eps>0}{\spc{X}\le \spc{Y}+\eps\ \text{and}\ \spc{Y}\le \spc{X}+\eps}.
\]
The Gromov--Hausdorff distance turns the set of all isometry classes of compact metric spaces into a metric space.
The following theorem shows that convergence in this space coinsides with the Gromov--Hausdorff convergence defined above.

\begin{thm}{Theorem} Let $\spc{X}_1,\spc{X}_2,\dots$ and $\spc{X}_\infty$ be compact metric spaces.
Then there is a convergence $\spc{X}_n\GHto \spc{X}_\infty$ if and only if
$\GHdist(\spc{X}_n,\spc{X}_\infty)\to 0$ as $n\to\infty$.

\end{thm}

\parit{Proof; if part.}
Suppose $a_n\:\spc{X}_\infty\to \spc{X}_n$
and $b_n\:\spc{X}_n\to \spc{X}_\infty$ are sequences of maps such that
\[\dist{a_n(x)}{a_n(y)}{\spc{X}_\infty}
\ge
\dist{x}{y}{\spc{X}_n}-\delta_n,\]
\[\dist{b_n(v)}{b_n(w)}{\spc{X}_n}
\ge
\dist{v}{w}{\spc{X}_\infty}-\delta_n\]
for any $x,y\in \spc{X}_n$ and $v,w\in\spc{X}_\infty$ and some sequence $\delta_n\to0+$.
Let us show that $\spc{X}_n\xto{\GH}\spc{X}_\infty$. 

Fix $\eps>0$ and choose a maximal $\eps$-packing $\{x^1,x^2,\dots,x^\kay\}$ in $\spc{X}_\infty$ such that the value $\sum_{i<j}\dist{x^i}{x^j}{}$ is maximal.
Note that 
\[\dist{a_n\circ b_n(x^i)}{a_n\circ b_n(x^j)}{}\ge\dist{x^i}{x^j}{}-2\cdot\delta_n.\]
Since the value $\sum_{i<j}\dist{x^i}{x^j}{}$ is maximal, 
\[\dist{a_n\circ b_n(x^i)}{a_n\circ b_n(x^j)}{}\to\dist{x^i}{x^j}{}\]
for all $i$ and $j$ as $n\to\infty$.
For all large $n$,
we have $2\cdot\delta_n<\dist{x^i}{x^j}{}-\eps$;
thus 
\[\dist{b_n(x^i)}{b_n(x^j)}{\spc{X}_n}>\eps
\quad\text{and}\quad\dist{a_n\circ b_n(x^i)}{a_n\circ b_n(x^j)}{\spc{X}_n}>\eps\] 
for all $i\not=j$.
Therefore for each large $n$, 
the set $\{a_n\circ b_n(x^i)\}$ forms a maximal $\eps$-packing and therefore an $\eps$-net in $\spc{X}_\infty$.

Since $\{a_n\circ b_n(x^i)\}$ is an  $\eps$-net in $\spc{X}_\infty$,
for any $y_n\in\spc{X}_n$, there is $x^i$ such that $\dist{a_n\circ b_n(x^i)}{a_n(y_n)}{}<\eps$.
Thus, $\dist{b_n(x^i)}{y_n}{}<\eps+\delta_n$;
that is, $\{b_n(x^i)\}$ is a $(\eps+\delta_n)$-net in $\spc{X}_n$.

Given $y\in \spc{X}_n$ choose $x^i$ so that $\dist{b_n(x^i)}{y_n}{}<\eps+\delta_n$ and define $h_n(y)=a_n\circ b_n(x^i)$.
Observe that $h_n$ is a $3\cdot\eps$-isometry for all large $n$.
Since $\eps>0$ is arbitrary, there is a sequence of $\eps_n$-isometries $\spc{X}_n\to\spc{X}_\infty$ such that $\eps_n\to\0+$ as $n\to\infty$.
It remains to apply \ref{lem:almost-isom}.

\parit{Only-if part.}
Assume $\spc{X}_n\xto{\GH}\spc{X}_\infty$.
Fix $\eps>0$ and choose a maximal $\eps$-packing $\{x^1,x^2,\dots,x^\kay\}$ in $\spc{X}_\infty$.
For each $x^i$, 
choose a sequence $x^i_n\in\spc{X}_n$ such that $a_n(x^i_n)\to x^i$.
Note that for all large $n$, we have $\dist{x^i_n}{x^j_n}{}>\eps$.
For each point $z\in \spc{X}_\infty$, choose $x^i$ so that $\dist{z}{x^i}{}<\eps$ and define map $b_n\:\spc{X}_\infty\to\spc{X}_n$ such that 
$b_n(z)=x^i_n$.
Observe that 
\[\dist{b_n(y)}{b_n(z)}{\spc{X}_n}+3\cdot\eps>\dist{y}{z}{\spc{X}_\infty}\]
for all large $n$.

The same way we can construct a map $a_n\:\spc{X}_n\to\spc{X}_\infty$ such that 
\[\dist{a_n(y)}{a_n(z)}{\spc{X}_\infty}+3\cdot\eps>\dist{y}{z}{\spc{X}_n}\]
Whence $\GHdist(\spc{X}_n,\spc{X}_\infty)\to 0$ as $n\to \infty$.
\qeds

The following theorem roughly states that isometry class of Gromov--Hausdorff limit is uniquely defined once it is compact. 

\begin{thm}{Theorem} Let $\spc{X}_1,\spc{X}_2,\dots$ and $\spc{X}_\infty$ and $\bar{\spc{X}}_\infty$ be metric spaces
such that $\spc{X}_n\xto{\GH}\spc{X}_\infty$, 
$\spc{X}_n\xto{\bar\GH}\bar{\spc{X}}_\infty$.

Assume that $\bar{\spc{X}}_\infty$ is compact.
Then $\spc{X}_\infty\iso \bar{\spc{X}}_\infty$.
\end{thm}


\parit{Proof.}
For each point $x_\infty\in\spc{X}_\infty$,
choose its liftings $x_n\in\spc{X}_n$.

Choose a nonprincipal ultrafilter $\o$.
Define $\bar x_\infty\in \bar{\spc{X}}_\infty$ as the $\o$-limit of $x_n$ with respect to $\bar \tau$.
We claim that the map $x_\infty\to \bar x_\infty$ is an isometry.

Indeed, by to the definition of Gromov--Hausdorff convergence, 
\[\dist{\bar x_\infty}{\bar y_\infty}{\bar{\spc{X}}_\infty}
=
\lim_{n\to\o}\dist{x_n}{y_n}{\spc{X}_n}
=
\dist{x_\infty}{y_\infty}{\spc{X}_\infty}.
\]
Thus, the map $x_\infty\to\bar x_\infty$ gives a distance preserving map
$\map\:\spc{X}_\infty\hookrightarrow\bar{\spc{X}}_\infty$.
In particular,  
$\spc{X}_\infty$ is compact.
Switching $\spc{X}_\infty$ and $\bar{\spc{X}}_\infty$ and applying the same argument, 
we get an isometric embedding 
$\bar{\spc{X}}_\infty\hookrightarrow\spc{X}_\infty$.
Thus, the result follows from Lemma~\ref{lem:>=-isometry}.
\qeds

\begin{thm}{Exercise}
Let $\spc{X}_n$ be a sequence of metric spaces that admit 
two Gromov--Hausdorff convergences
$\GH$ and $\GH'$.
Assume 
$\spc{X}_n\xGHto{\GH}\spc{X}_\infty$ and $\spc{X}_n\xGHto{\GH'}\spc{X}_\infty'$.
Show that if $\spc{X}_\infty$ is proper and there is a sequence of points $x_n\in \spc{X}_n$ 
that converges in both
$\GH$ and $\GH'$, then $\spc{X}_\infty\iso\spc{X}_\infty'$.
\end{thm}






