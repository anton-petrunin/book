%%!TEX root = all.tex
\chapter{Convergence of metric spaces}

In this section we discuss the
Gromov--Hausdorff convergence of metric spaces.

It seems that Hausdorff convergence was first introduced by Hausdorff in \cite{hausdorff}
and couple of years later an equivalent definition was given by Blaschke in \cite{blaschke}.
Further a refinement of this definition was introduced by Frol\'{\i}k in \cite{frolik}
and then rediscovered by Wijsman in \cite{wijsman},
however this refinement was a step back to so called \emph{closed convergence} introduced by Hausdorff in the original book. 
By that reason we call it Hausdorff convergence
instead of
\emph{Hausdorff--Blascke--Frol\'{\i}k--Wijsman convergence}.

The definition of Gromov--Hausdorff convergence or metric spaces use 
the Hausdorff convergence.
It means that the sequence of metric spaces admit a sequence of distance preserving embedding into huge metric space so that the their
images converge in Hausdorff sense.
The Gromov--Hausdorff convergence is also defined in a nonstandard way.


\section{Convergence of subsets}


Let $\spc{X}$ be a metric space and $A\subset \spc{X}$.
We will denote by $\dist{A}{x}{}=\dist{A}{x}{\spc{X}}$ the distance from $A$ to a point $x$ in $\spc{X}$;
i.e.,
$$\dist{A}{x}{}\df\inf\set{\dist{a}{x}{}}{a\in A}.$$

\begin{thm}{Definition of Hausdorff convergence}\label{def:hausdorff-coverge}
Given a sequence of closed sets $(A_n)_{n=1}^\infty$ in a metric space $\spc{X}$ 
a closed set $A_\infty\subset \spc{X}$ is called Hausdorff limit of $(A_n)_{n=1}^\infty$,
briefly $A_n\Hto A_\infty$ if 
$$\dist{A_n}{x}{}\to\dist{A_\infty}{x}{}\ \ \t{as}\ \ n\to\infty$$
for any fixed $x\in \spc{X}$.

In this case the sequence of closed sets $(A_n)_{n=1}^\infty$ is called \emph{converging} or \emph{converging in the sense of Hausdorff}.
\end{thm}

\begin{thm}{Selection theorem}
Let $\spc{X}$ be a proper space
and $(A_n)_{n=1}^\infty$ be a sequence of closed sets in $\spc{X}$.
Assume that for some (and therefore any) point  $x\in\spc{X}$ 
the sequence $\dist{A_n}{x}{}$ is bounded.
Then  $(A_n)_{n=1}^\infty$ has a converging subsequence in the sense of Hausdorff.
\end{thm}

\parit{Proof.}
Since $X$ is proper,
we can choose a countable dense set $\{x_1,x_2,\dots\}$ in $\spc{X}$.
Note that the sequence $a_n=\dist{A_n}{x_\kay}{}$ is bounded for each $\kay$. 
Therefore, passing to a subsequence of $(A_n)_{n=1}^\infty$,
we can assume that $\dist{A_n}{x_\kay}{}$ is converging as $n\to\infty$ for any fixed $\kay$.

Note that for each $n$, the function $\dist{A_n}{}{}\:\spc{X}\to\RR$ is 1-Lipschitz and nonnegative.
Therefore the sequence $\dist{A_n}{}{}$ converges pointwise to a 1-Lipschitz nonnegative function $f\:\spc{X}\to\RR$.

Set $A_\infty=f^{-1}(0)$.
Since $f$ is 1-Lipschitz, 
$\dist{A_\infty}{y}{}\ge f(y)$ for any $y\in \spc{X}$.
It remains to show that $\dist{A_\infty}{y}{}\le f(y)$ for any $y$.

Assume contrary;
i.e., $f(z)<R<\dist{A_\infty}{z}{}$ for some $z\in \spc{X}$ and $R>0$.
Then for any large enough $n$ there is a point $z_n\in A_n$ such that
$\dist{x}{z_n}{}\le R$.
Since $\spc{X}$ is proper, we can pass to a partial limit $z_\infty$ of $z_n$ as $n\to\infty$.

Clearly that $f(z_\infty)=0$, i.e., $z_\infty\in A_\infty$.
On the other hand, 
\[\dist{A_\infty}{y}{}\le\dist{z_\infty}{y}{}\le R<\dist{A_\infty}{y}{},\] 
a contradiction.
\qeds

\section{Gromov--Hausdorff convergence}

\begin{thm}{Definition}\label{def:comp-metr}
Let $\set{\spc{X}_\alpha}{\alpha\in\IndexSet}$ be a set of metric spaces.
A metric space $\bm{X}$
is called \emph{common space}\index{common space} of $\set{\spc{X}_\alpha}{\alpha\in\IndexSet}$ if its underlying set is formed by the disjoint union $$\bigsqcup_{\alpha\in\IndexSet} \spc{X}_\alpha$$ 
and each inclusion $\iota_\alpha\:\spc{X}_\alpha\hookrightarrow\bm{X}$
is distance preserving.
\end{thm}

\begin{thm}{Definition}\label{def:GH}
Let $\bm{X}$ be a common space for proper metric spaces
$\spc{X}_1,\spc{X}_2,\dots$ and $\spc{X}_\infty$.
Assume that $\spc{X}_n$ forms an open set in $\bm{X}$ for each $n<\infty$ and 
$\spc{X}_n\Hto \spc{X}_\infty$ in $\bm{X}$ as $n\to\infty$.

In this case the topology $\GH$ of $\bm{X}$ is called \emph{Gromov--Hausdorff convergence}\index{Gromov--Hausdorff convergence}
and we write $\spc{X}_n\GHto \spc{X}_\infty$ or $\spc{X}_n\xGHto{\GH} \spc{X}_\infty$;
the later notation is used if we need to consider specific Gromov--Hausdorff convergene $\GH$.
The space $\spc{X}_\infty$ is called the limit space of the sequence $(\spc{X}_n)$ along $\GH$.
\end{thm}

Once we write $\spc{X}_n\GHto \spc{X}_\infty$ we mean that we made a choice of Gromov--Hausdorff convergence.

Note that for a fixed sequence of metric spaces $\spc{X}_1,\spc{X}_2,\dots$ one may construct different Gromov--Hausdorff convergences, say $\spc{X}_n\xGHto{\GH} \spc{X}_\infty$ and $\spc{X}_n\xGHto{\GH'} \spc{X}_\infty'$  and their limit spaces $\spc{X}_\infty$ and $\spc{X}_\infty'$ need not to be isometric to each other. 
For example, for the constant sequence $\spc{X}_n\iso\RR_{\ge0}$, 
one may take $\spc{X}_\infty\iso\RR_{\ge0}$.
In this case a point in the disjont space $\bm{X}$ can be thought as a pair $(x,n)\in \RR_{\ge}\times (\ZZ_>\cup \{\infty\})$ 
and the metric on $\bm{X}$ can be defined the following way
$$\dist{(x,n)}{(y,m)}{\bm{X}}\df|\tfrac1n-\tfrac1m|+|x-y|,$$
where we assume that $0=\tfrac1\infty$.
On the other hand, one can take $\spc{X}_\infty'\iso\RR$,
and consider the metric
\begin{align*}
\dist{(x,n)}{(y,m)}{\bm{X}'}
&=|\tfrac1n-\tfrac1m|+|(x-n)-(y-m)|,
\\
\dist{(x,n)}{(y,\infty)}{\bm{X}'}
&=\tfrac1n+|(x-n)-y|,
\\
\dist{(x,\infty)}{(y,\infty)}{\bm{X}'}
&=|x-y|.
\end{align*}
where $n, m<\infty$.

\begin{thm}{Induced convergences}
Assume $\spc{X}_n\xGHto{\GH}\spc{X}_\infty$,
and $\bm{X}$ as in the definition \ref{def:GH}
and $\iota_n\:\spc{X}_n\hookrightarrow\bm{X}$, $\iota_\infty\:\spc{X}_\infty\hookrightarrow\bm{X}$ are corresponding inclusions.

\begin{subthm}{}
A sequence of points $x_n\in\spc{X}_n$ converges to $x_\infty\in\spc{X}_\infty$ (briefly, $x_n\to x_\infty$ or $x_n\xto{\GH} x_\infty$) 
if $\dist{x_n}{x_\infty}{\bm{X}}\to 0$.
\end{subthm}

\begin{subthm}{}
A sequence of closed sets 
$\mathfrak{C}_n\subset \spc{X}_n$ 
converges to a closed  set 
$\mathfrak{C}_\infty\subset \spc{X}_\infty$ (briefly, $\mathfrak{C}_n\to \mathfrak{C}_\infty$ or $\mathfrak{C}_n\xto{\GH} \mathfrak{C}_\infty$)
if $\mathfrak{C}_n\Hto\mathfrak{C}_\infty$ as subsets of $\bm{X}$.
\end{subthm}

\begin{subthm}{}
A sequence of open sets $\Omega_n\subset \spc{X}_n$ 
converges to an open set $\Omega_\infty\subset \spc{X}_\infty$
(briefly, $\Omega_n\to \Omega_\infty$ 
or $\Omega_n\xto{\GH} \Omega_\infty$)
if the complements $\spc{X}_n\backslash \Omega_n$ converge to the complement $\spc{X}_\infty\backslash \Omega_\infty$ as closed sets.
\end{subthm}


\begin{subthm}{} Let $\spc{X}_n\xGHto{\GH} \spc{X}_\infty$ and $\spc{Y}_n\xGHto{\theta} \spc{Y}_\infty$. 
A sequence of submaps, $\map_n\:\spc{X}_n\subto \spc{Y}_n$ converges to a submap $\map_\infty\:\spc{X}_\infty\subto \spc{Y}_\infty$ if the following conditions holds
\begin{itemize}
\item $\Dom\map_n\to \Dom\map_\infty$ as a sequence of open sets.

\item For any $x_\infty\in \Dom \map_\infty$ and any sequence $x_n\in \spc{X}_n$ such that $x_n\to x_\infty$
\[\spc{Y}_n\ni \map _n(x_n)\xto\theta \map_\infty(x_\infty)\in\spc{Y}_\infty\] 
as $n\to\infty$.
\end{itemize}
\end{subthm}

\begin{subthm}{} Given a sequence of measures $\mu_n$ on $\spc{X}_n$
denote by $\iota_n\#\mu_n$ the pushforward measures on $\bm{X}$.
We say that $(\mu_n)$ weakly converges to a measure $\mu_\infty$ on $\spc{X}_\infty$ 
(brefly, $\mu_n\xto
\mu_\infty$ or $\mu_n\xto{\GH}
\mu_\infty$) 
if the pushforward measures $\iota_n\#\mu_n$ weakly converge to the $\iota_\infty\#\mu_\infty$.

In other words, 
if for any continuous function $\phi\:\bm{X}\to\RR$ with a compact support, we have 
\[\int\limits_{\spc{X}_n} \phi\circ\iota_n(x)
\cdot
\d_x\mu_n
\to 
\int\limits_{\spc{X}_\infty} \phi\circ\iota_\infty(x)
\cdot\d_x\mu_\infty\]
as $n\to\infty$.
\end{subthm}
\end{thm}

\parbf{Liftings.}
Given a Gromov--Hausdorff convergence 
$\spc{X}_n\GHto \spc{X}_\infty$
and a point $p_\infty\in\spc{X}_\infty$ any sequence of points $p_n\in\spc{X}_n$ such that $p_n\GHto p$  will be called \emph{lifting}\index{lifting of a point} of $p_\infty$.
In this case the point $p_n\in \spc{X}_n$ will be called a lifting of $p_\infty$ in $\spc{X}_n$.
In this case we say that $\dist{p_n}{}{}\:\spc{X}_n\to \RR$ 
is a  \emph{lifting}\index{lifting of a distance function} 
of the distance function $\dist{p}{}{}\:\spc{X}\to \RR$.
Obviousely $\dist{p_n}{}{}\GHto\dist{p}{}{}$.

Note that liftings are not uniquely defined.
We will be interested in the properties of liftings for sufficiently large $n$.

The same way we can talk about liftings of a point array
$\bm{p}_\infty=(p_\infty^1,p_\infty^2,\dots,p_\infty^\kay)$ in $\spc{X}_\infty$
and about the corresponding distance map 
$\dist{\bm{p}_\infty}{}{}\:\spc{X}_\infty\to\RR^\kay$
$$\dist{\bm{p}_\infty}{}{}\:x\mapsto(\dist{p_\infty^1}{x}{},\dist{p_\infty^2}{x}{},\dots,\dist{p_\infty^k}{x}{})$$

\section{Gromov's selection theorem}

\begin{thm}{Gromov's selection theorem}\label{thm:gromov-selection}
Let $\spc{X}_n$ be a sequence of proper metric spaces 
with marked points $x_n\in \spc{X}_n$.
Assume that for any fixed $R,\eps>0$ there is $N=N(R,\eps)\in\ZZ_>$ 
such that for each $n$
the ball $\cBall[x_n,R]\subset \spc{X}_n$ admits a finite $\eps$-net with at most $N$ points.
Then there is a subsequence of $\spc{X}_n$ which admit a Gromov--Hausdorff convergence 
such that the sequence of marked points $x_n\in\spc{X}_n$ converges.
\end{thm}

\parit{Proof.}
Note that there is a sequence of integers $M_1<M_2<\dots$
such that in each space $\spc{X}_n$
there is a sequence of points $z_{i,n}\in\spc{X}_n$ such that
\[\dist{z_{i,n}}{x_n}{\spc{X}_n}\le \kay+1\ \ \text{if}\ \ i\le M_\kay\]
and
the points $z_{1,n},\dots,z_{M_\kay,n}$ form an $\tfrac1\kay$-net in $\cBall[x_n,\kay]_{\spc{X}_n}$.
The existence of the sequence $M_1,M_2,\dots$ follows from the main assumption in the theorem.

Passing to a subsequence, we can assume that the sequence \[\ell_n=\dist{z_{i,n}}{z_{j,n}}{\spc{X}_n}\] 
converges for any $i$ and $j$.

Let us consider countable set of points $\spc{W}=\{w_1,w_2,\dots\}$
equipped with the pseudometric defined as 
\[|w_i,w_j|_{\spc{W}}
=
\lim_{n\to\infty}\dist{z_{i,n}}{z_{j,n}}{\spc{X}_n}.\]
Let $\hat{\spc{W}}$ be the metric space corresponding to $\spc{W}$.
Denote by
$\spc{X}_\infty$ the completion of $\hat{\spc{W}}$.

It remains to show that there is a Gromov--Hausdorff convergence 
$\spc{X}_n\GHto\spc{X}_\infty$ such that the sequence $x_n\in\spc{X}_n$ converges.
To prove it, we need to construct a metric on the disjoint union of \[\bm{X}=\spc{X}_\infty\sqcup\spc{X}_1\sqcup\spc{X}_2\sqcup\dots\] 
satisfies definitions \ref{def:comp-metr} and \ref{def:GH}.

The metric can be defined as follows.
Fix a sequence $\eps_\kay\to0+$
and let $N_\kay$ be the minimal integer such that
\[\left|\dist{w_i}{w_j}{\spc{W}}
-
\dist{z_{i,n}}{z_{j,n}}{\spc{X}_n}\right|<\eps_\kay
\]
if $i,j\le N_\kay$ and $n\ge N_\kay$.
\[\dist{x}{y}{\bm{X}}=
\begin{cases}
\dist{x}{y}{\spc{X}_n}& x,y\in \spc{X}_n
\\
\dist{x}{y}{\spc{X}_\infty}& x,y\in \spc{X}_\infty
\end{cases}
\]

Set,
\[
\delta(z_{i,n},z_{i,\infty})
=
\sup\set{\left|\dist{z_{i,m}}{z_{j,m}}{\spc{X}_m}
-
\dist{z_{i,\infty}}{z_{j,\infty}}{\spc{X}_\infty}\right|}
{j\le i,\, m\ge n}.\]
Let us equip $\bm{X}$ with the maximal metric such that each inclusion $\iota_n\:\spc{X}_n$ 
is short 
and the $\dist{z_{i,n}}{z_{i,\infty}}{\bm{X}}\le \delta(z_{i,n},z_{i,\infty})$ for any $i$ and $n$.
In other words, let us define 
\qeds

\begin{thm}{Lemma}
Let $\spc{X}$ and $\spc{Y}$ be two metric spaces with two sequences of points $x_1,x_2,\dots\in\spc{X}$ and $y_1,y_2,\dots\in\spc{Y}$.
Set
\[\eps_n
=
\max
\set{\left|\dist{x_n}{x_i}{\spc{X}}-\dist{y_n}{y_i}{\spc{Y}}\right|}{i<n}.\]
Then for any $\eps>0$,
the function $\rho\:((\spc{X}\sqcup\spc{Y})\times (\spc{X}\sqcup\spc{Y}))$ defined below is a metric on the disjoint union $\spc{X}\sqcup\spc{Y}$.
\begin{align*}
\rho(x,x')&=\dist{x}{x'}{\spc{X}}&&\text{for any}\ \ x,x'\in \spc{X},
\\
\rho(y,y')&=\dist{y}{y'}{\spc{Y}}&&\text{for any}\ \ y,y'\in \spc{Y},
\\
\rho(x,y)=\rho(y,x)&=\inf_i\{\dist{x}{x_n}{\spc{X}}+\eps+\eps_n+\dist{y_n}{y}{\spc{Y}}\}&&\text{for any}\ \ x\in\spc{X}\ \text{and}\ y\in \spc{Y}.
\end{align*}


In particular, both inculsions $\iota_{\spc{X}}\:\spc{X}\hookrightarrow(\spc{X}\sqcup\spc{Y},\rho)$ and $\iota_{\spc{Y}}\:\spc{Y}\hookrightarrow(\spc{X}\sqcup\spc{Y},\rho)$ are distance preserving.

\end{thm}


\begin{thm}{Theorem}
Let $\spc{X}_n$ be a sequence of metric spaces which admit 
two Gromov--Hausdorff convergences
$\GH$ and $\GH'$.
Assume 
$\spc{X}_n\xGHto{\GH}\spc{X}_\infty$ and $\spc{X}_n\xGHto{\GH'}\spc{X}_\infty'$.
\begin{subthm}{}
If  $\spc{X}_\infty$ is compact then $\spc{X}_\infty\iso\spc{X}_\infty'$.
\end{subthm}

\begin{subthm}{}
If  $\spc{X}_\infty$ is proper and there is a sequence of points $x_n\in \spc{X}_n$ 
which converges in both
$\GH$ and $\GH'$ then $\spc{X}_\infty\iso\spc{X}_\infty'$.
\end{subthm}
\end{thm}











\section{Convergence of compact spaces.}

\begin{thm}{Definition}
Let $\spc{X}$ and $\spc{Y}$ be metric space,
a map $f\:\spc{X}\to\spc{Y}$
is called \emph{$\eps$-isometry}\index{isometry!$\eps$-siometry}
if the following two condition hold:
\begin{subthm}{}
$\Im f$ is an $\eps$-net of $\spc{Y}$
\end{subthm}

\begin{subthm}{}
$\dist{f(x)}{f(x')}{\spc{Y}}\lege\dist{x}{x'}{\spc{X}}\pm\eps$ for any $x,x'\in\spc{X}$
\end{subthm}

\end{thm}

The following theorem roughly states that isometry class of Gromov--Hausdorff limit is uniquely defined once it is compact. 

\begin{thm}{Theorem} Let $\spc{X}_1,\spc{X}_2,\dots$ and $\spc{X}_\infty$ and $\bar{\spc{X}}_\infty$ be metric spaces
such that $\spc{X}_n\xto{\GH}\spc{X}_\infty$, 
$\spc{X}_n\xto{\bar\GH}\bar{\spc{X}}_\infty$.

Assume that $\bar{\spc{X}}_\infty$ is compact.
Then $\spc{X}_\infty\iso \bar{\spc{X}}_\infty$.
\end{thm}


\parit{Proof.}
For each $x_\infty\in\spc{X}_\infty$,
fix its liftings $x_n\in\spc{X}_n$.

Choose some ultrafilter $\o$.
Define $\bar x_\infty\in \bar{\spc{X}}_\infty$ such that $\bar x_n\xGHto{\bar\GH}\bar x_\infty$;
since $\bar{\spc{X}}_\infty$ is compact, the $\o$-limit is defined.
We claim that the map $x_\infty\to \bar x_\infty$ is an isometry.

Indeed, according to the definition of Gromov--Hausdorff convergence, 
\[\dist{\bar x_\infty}{\bar y_\infty}{\bar{\spc{X}}_\infty}
=
\lim_{n\to\o}\dist{x_n}{y_n}{\spc{X}_n}
=
\dist{x_\infty}{y_\infty}{\spc{X}_\infty}.
\]
Thus, the map $x_\infty\to\bar x_\infty$ gives a distance preserving map
$\map\:\spc{X}_\infty\hookrightarrow\bar{\spc{X}}_\infty$.
In particular,  
$\spc{X}_\infty$ is compact.
Switching $\spc{X}_\infty$ and $\bar{\spc{X}}_\infty$ and applying the same argument, 
we get an isometiric embedding 
$\bar{\spc{X}}_\infty\hookrightarrow\spc{X}_\infty$.
Thus, the result follows from Lemma~\ref{lem:>=-isometry}.
\qeds




\begin{thm}{Definition}
 Given two compact spaces $\spc{X}$ and $\spc{Y}$, we will write 
\begin{itemize}
\item $\spc{X}\le \spc{Y}$ if there is a non-contracting map $\map\:\spc{X}\to \spc{Y}$.
\item $\spc{X}\le \spc{Y}+\eps$ if there is a map $\map\:\spc{X}\to \spc{Y}$ such that for any $x,x'\in \spc{X}$ we have
\[\dist{x}{x'}{}\le \dist{\map(x)}{\map(x')}{}+\eps.\]
\end{itemize}

Further, we define \emph{Gromov--Hausdorff distance}\index{Gromov--Hausdorff distance} between $\spc{X}$ and $\spc{Y}$, $\GHdist(\spc{X},\spc{Y})$ as infimum of all $\eps>0$ such that
$\spc{X}\le \spc{Y}+\eps$ and $\spc{Y}\le \spc{X}+\eps$.
\end{thm}

\parbf{Remark for AKP.}
This notation is a bit slang-like, maybe better to use $\succcurlyeq$ instead of $\ge$.
Yet $\spc{Y}+\eps$ migh be needed to be changed;
one might think that $\spc{Y}+\eps$ is a metric space $(\ushort{\spc{Y}},\dist{}{}{}+\eps \rho)$, where $\rho$ is a discreate metric on $\ushort{\spc{Y}}$ (i.e. $\rho(x,y)=1$ if $x\not=y$). 
It will be totally correct to introduce some operator say $\op{Add}_\eps\spc{Y}$, but it is also less intuitive...
One may also write something like 
$\spc{X}\succcurlyeq \spc{Y}^{+\eps}$ or $\spc{X}\succcurlyeq^{\eps+} \spc{Y}$...

\medskip

The Gromov--Hausdorff distance turns the set of all isometry classes of compact metric spaces into a metric space.
The following theorem shows that convergence in this space coinsides with the Gromov--Hausdorff convergence defined above.

\begin{thm}{Theorem} Let $\spc{X}_1,\spc{X}_2,\dots$ and $\spc{X}_\infty$ be compact metric spaces.
Then 
\[\spc{X}_n\xto{}\spc{X}_\infty
\ \ \Leftrightarrow\ \ 
\GHdist(\spc{X}_n,\spc{X}_\infty)\to 0.\]

\end{thm}

\parit{Proof; $(\Leftarrow)$.}
Let $\spc{X}_n\xto{\GH} \spc{X}_\infty$
and $b_n\:\spc{X}_n\to \spc{X}_\infty$ be a sequence of maps such that
\[\dist{a_n(x)}{a_n(y)}{\spc{X}_\infty}
\ge
\dist{x}{y}{\spc{X}_n}-\eps_n,\]
\[\dist{b_n(v)}{b_n(w)}{\spc{X}_n}
\ge
\dist{v}{w}{\spc{X}_\infty}-\eps_n\]
for any $x,y\in \spc{X}_n$ and $v,w\in\spc{X}_\infty$ and some sequence $\eps_n\to0+$.

Let us show that $\spc{X}_n\xto{\GH}\spc{X}_\infty$. 

Fix $\eps>0$ and choose a maximal $\eps$-packing $\{x^1,x^2,\dots,x^\kay\}$ in $\spc{X}_\infty$ such that the value $\sum_{i<j}\dist{x^i}{x^j}{}$ is maximal.
Note that 
\[\dist{a_n\circ b_n(x^i)}{a_n\circ b_n(x^j)}{}>\dist{x^i}{x^j}{}-2\cdot\eps_n.\]
Thus, 
$\dist{a_n\circ b_n(x^i)}{a_n\circ b_n(x^j)}{}-\dist{x^i}{x^j}{}
\to_{n\to\infty} 
0$ for all $i$ and $j$.
For all large $n$,
we have $2\cdot\eps_n<\dist{x^i}{x^j}{}-\eps$.
Thus $\dist{b_n(x^i)}{b_n(x^j)}{\spc{X}_n}>\eps$ 
and \[\dist{a_n\circ b_n(x^i)}{a_n\circ b_n(x^j)}{\spc{X}_n}>\eps\] 
for all $i\not=j$.
Thus, for each large $n$, 
the set $\{a_n\circ b_n(x^i)\}$ forms a maximal $\eps$-packing and therefore an $\eps$-net in $\spc{X}_\infty$.
Since $\eps>0$ is arbitrary, we get \ref{def:GH:appr1}.

Since $\{a_n\circ b_n(x^i)\}$ is an  $\eps$-net in $\spc{X}_\infty$,
for any $y_n\in\spc{X}_n$, there is $x^i$ such that $\dist{a_n\circ b_n(x^i)}{a_n(y_n)}{}<\eps$.
Thus, $\dist{b_n(x^i)}{y_n}{}<\eps+\eps_n$.

Further, note that the set $\{b_n(x^i)\}$ forms an $(\eps+\eps_n)$-net in $\spc{X}_n$.
Otherwise, if for some $y_n\in \spc{X}_n$, we have $\dist{y_n}{b_n(x^i)}{}$ for all $i$,
then $\dist{a_n\circ b_n(x^i)}{a_n(y_n)}{}>\eps$; 
i.e. $\kay<\pack_\eps{X}_\infty$, a contradiction.

\parit{$(\Rightarrow)$.}
Assume $\spc{X}_n\xto{\GH}\spc{X}_\infty$.
Fix $\eps>0$ and choose a maximal $\eps$-packing $\{x^1,x^2,\dots,x^\kay\}$ in $\spc{X}_\infty$.
For each $x^i$, 
choose a sequence $x^i_n\in\spc{X}_n$ such that $a_n(x^i_n)\to x^i$.
Note that for all large $n$, we have $\dist{x^i_n}{x^j_n}{}>\eps$.
For each point $z\in \spc{X}_\infty$, choose $x^i$ so that $\dist{z}{x^i}{}<\eps$ and define map $b_n\:\spc{X}_\infty\to\spc{X}_n$ such that 
$b_n(z)=x^i_n$.
\qeds



\begin{thm}{Lemma}\label{lem:>=-isometry}
Let $\spc{X}$ and $\spc{Y}$ be two metric spaces 
and $\spc{X}$ is compact then
\[
\spc{X}\ge\spc{Y}\ge\spc{X}
\ \ \Leftrightarrow\ \ 
\spc{X}\iso\spc{Y}.
\]

\end{thm}

The following proof was suggested by Travis Morrison.

\parit{Proof.}
Let $f\: \spc{X} \to \spc{Y}$ 
and $g\: \spc{Y} \to \spc{X}$ be non contracting mappings.
It is sufficient to prove that $h  = g\circ f\:\spc{X}\to \spc{X}$ is an isometry. 

Given any pair of points $x,y\in \spc{X}$, 
set $x_n=h^{\circ n}(x)$ and $y_n=h^{\circ n}(y)$.
Since $\spc{X}$ is compact, one can choose an incresing sequence of integers $n_\kay$
such that both sequences $(x_{n_i})_{i=1}^\infty$ and $(y_{n_i})_{i=1}^\infty$
converge.
In particular, both of these sequences  are Cauchy;
i.e.,
\[
\dist{x_{n_i}}{x_{n_j}}{},\dist{y_{n_i}}{y_{n_j}}{}\to 0
\]
as $\min\{i,j\}\to\infty$.
Sinse $h$ is nonconotracting, we get
\[
\dist{x}{x_{|n_i-n_j|}}{}\le \dist{x_{n_i}}{x_{n_j}}{}.
\]
It follows that  
there is a sequence $m_i\to\infty$ such that
\[
x_{m_i}\to x\ \ \text{and}\ \ y_{m_i}\to y\ \ \text{as}\ \ \kay\to\infty.
\eqlbl{eq:x_l->x}
\]

Let $\ell_n=\dist{x_n}{y_n}{}$.
Since $h$ is noncontracting, $(\ell_n)$ is a nondecreasing sequence.
On the other hand, 
from \ref{eq:x_l->x}, it follows that $\ell_{m_i}\to\dist{x}{y}{}=\ell_0$ as $m_i\to\infty$;
i.e., $(\ell_n)$ is a constant sequece.
In particular $\ell_0=\ell_1$ for any $x$ and $y\in \spc{X}$;
i.e., $h$ is distance preserving map.

Therefore $h(X)$ is isometric to $\spc{X}$.
From \ref{eq:x_l->x}, we get that $h(\spc{X})$ is everywhere dense.
Sinse $X$ is compact, we get $h(\spc{X})=\spc{X}$.
%Let  $\{x^i\}_{i=1}^{n}$ be a maximal $\eps$-packing in $\spc{X}$ such that  $\sum_{i<j}\dist{x^i}{x^j}{}$ attends maximum value. The set $\{\map (x^i)\}$ forms a maximal $\eps$-packing in $\spc{X}$. Sinse $\sum_{i<j}\dist{x^i}{x^j}{}$ is maximal, we have \[\dist{\map (x^i)}{\map (x^j)}{}=\dist{x^i}{x^j}{}\] for all $i,j$.
%
%Recal that maximal $\eps$-packing is an $\eps$-net (see ???).
%Thus, the set $\{\map(x^i)\}$ forms an $\eps$-net in $\spc{X}$.
%Thus for any two points $p,q$ one can find points $x^i,x^j$ such that
%\[\dist{\map (p)}{\map (x^i)}{}\le\eps\ \ \text{and}\ \ \dist{\map (q)}{\map (x^j)}{}\le\eps\]
%Therefore, since $\map $ is non contracting,
%\[\dist{p}{q}{}
%\le
%\dist{\map (p)}{\map (q)}{}
%\le
%\dist{\map (x^i)}{\map (x^j)}{}+2\cdot\eps
%=
%\dist{x^i}{x^j}{}+2\cdot\eps
%\le
%\dist{p}{q}{}+4\cdot\eps.\]
%Sinse $\eps>0$ can be chousen arbitrary, we have
%\[\dist{p}{q}{}
%=
%\dist{\map (p)}{\map (q)}{}.\]
\qeds

\section{Comments}

One of the most remarcble revolution in geometry was made by seemingly very simple observation made by Gromov.
He gave a transparent intuitive notion of convergence for metric spaces.
Although some types of convergences of metric spaces
were considered in geometry before Gromov,
all the considered metric were designed to be used in very speacific problems.
A very good account for different types of convergences used in metric geometry can be found in ???.

The Gromov--Hausdorff convergence (under the name Hausdorff convergence)
was introduced by Gromov in \cite{gromov-polynomial-growth}.
It was essential step in the proof that any group of polynomial growth has is a nilpontant subgroup of finite index.
It is remarkable that the ultra-limits,
an other important type of convergence,
also appear first in the context of  groups of polynomial growth;
they were introduced by ??? in ???, where generalization of Gromov's theorem
Both concepts quikly sprad into all branches of geometry.


The definition \ref{def:hausdorff-coverge} was given by Frol\'{\i}k in \cite{frolik} %CHECK
and was rediscovered by Wijsman in \cite{wijsman}.

This definition is a slight modification of the convergence for subsets of metric space,
first introduced by Hausdorff in \cite{hausdorff}%CHECK
, two years later the same convergence was introduced by Baschke in \cite{blaschke}%independently???
. 
Blaschke does not refer to Hausdorff; it might mean that his discovery was independent 
or yet more likely that this type of convergence was used in mathematical folklore before these publications. 

If in this definition, 
you exachange the pointwise convergence to uniform convergence,
then you get the usual Hausdorff convergence.
These two convergences, coinside on the 
set of bounded subsets
and Frol\'{\i}k--Wijsman modification make it more sutable for working with unbounded sets.

Originaly, the Gromov--Hausdorff convergence was defined differently, see ???.
Instead one consider two cases: 
\begin{enumerate}
\item If the limit is compact, then one require existance of one GH-convergence.
\item In the non-compact case one makes a choice of point $x_\infty\in\spc{X}_\infty$ and a sequence $x_n\in\spc{X}_n$ and then to define pointed Gromov--Hausdorff convergence $(\spc{X}_n,x_n)\to (\spc{X}_\infty, x_\infty)$ one requires existance of GH-convergence $\GH$, such that $x_n\xto{\GH}x_\infty$.
\end{enumerate}
The later is called \emph{pointed Gromov--Hausdorff convergence}.
It is easy to see that if $\spc{X}_n\xto{\GH}\spc{X}_\infty$ then for any sequence $x_n\in\spc{X}_n$ 
such that $x_n\xto{\GH} x_\infty\in\spc{X}_\infty$, 
the seqence $(\spc{X}_n,x_n)$ 
converges to $(\spc{X}_\infty,x_\infty)$ 
in the sense of pointed Gromov--Hausdorff convergence.

The Gromov--Hausdorff convergence is defined only for proper spaces;
thus, once we write $\spc{X}_n\xto{}\spc{X}_\infty$, 
we implicitly assume that all $\spc{X}_1,\spc{X}_2,\dots$ and $\spc{X}_\infty$ are proper spaces.


