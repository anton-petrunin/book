%%!TEX root = all.tex
\chapter{Boundary}

\section{Definition}\label{sec:bry}

The boundary of complete length $\Alex{}$ space is defined only if its dimension is finite;
the definition uses induction on the dimension.

First, recall that if $\spc{L}$ is one-dimensional complete length $\Alex\kappa$ space, then $\spc{L}$ isometric to a one dimensional Riemannian manifold possibly with boundary  \akp{thm:dim=1.CBB}.
In this case, we define boundary of $\spc{L}$ (denoted as $\partial \spc{L}$)\index{$\partial\spc{L}$} to be boundary of corresponding manifold.

Now, assume that the notion of boundary is already defined in all dimensions less than $m$.
Recall that if $\spc{L}$ is an $m$-dimensional complete length $\Alex\kappa$ space, then $\Sigma_p$ is an $(m-1)$-dimensional complete length $\Alex1$ space for any $p\in \spc{L}$ \akp{thm:tan4finite};
thus $\partial\Sigma_p$ is already defined.
Then define 
\[\partial \spc{L}=\set{p\in \spc{L}}{\partial\Sigma_p\not=\emptyset};\]
that is, a point $p\in \spc{L}$ lies on the boundary if and only if its space of directions $\Sigma_p$ has nonempty boundary.


\begin{thm}{Proposition}
Let $\spc{L}$ be an $m$-dimensional complete length $\Alex\kappa$ space.
Then for any $p\in\partial\spc{L}$, $\rank_p\le m-1$.
\end{thm}

\parit{Proof.}???\qeds


The following statement has been proven in \cite{perelman:spaces2}, then its
formulation was made more exact in \cite{alexander-bishop:fk}. 
Here we give a simplified proof with the use of a gradient exponent.

\begin{thm}{Theorem} \label{thm:dist-to-bry} 
Let $\spc{L}$ be complete length $\Alex\kappa$ space and
$\partial \spc{L}\not=\emptyset$.
Suppose that $2\le\dim \spc{L}<\infty$.
Then the function 
\[f=\sn\kappa\circ\distfun{\partial \spc{L}}{}{}\] 
satisfies $f''+\kappa\cdot  f\le 0$.

In particular,
\begin{subthm}{} if $\kappa=0$, the function $\distfun{\partial \spc{L}}{}{}$ is concave;
\end{subthm}

\begin{subthm}{} if $\kappa>0$, the level sets $L_x=\distfun[-1]{\partial \spc{L}}{}{}(x)\subset \spc{L}$, $x>0$
are strictly concave hypersurfaces.
\end{subthm}

Moreover, the same conclusions hold for $f=\sn\kappa\circ\distfun{E}{}{}$, where $E$ is an $(m-1)$-dimensional extremal subset???.
\end{thm}

\parit{Proof.}  
We have to show that for any unit-speed geodesic $\gamma$, 
\[(f\circ\gamma)''(t_0)\le -\kappa\cdot  (f\circ\gamma)(t_0).\]
for and any $t_0$ in the interior of interval of definition.
Without loss of generality we can assume $t_0=0$.

First consider the space $\Lob[+]2\kappa$
 --- the halfplane of the model plane $\Lob2\kappa$.
Note that $\Lob[+]2\kappa$ is a two-dimensional complete length $\Alex\kappa$ space
and its boundary $\partial\Lob[+]2\kappa$ is the relative boundary of $\Lob[+]2\kappa$ in $\Lob2\kappa$.
Direct calculations show that the statement is true for $\Lob[+]2\kappa$.

Let $p\in \partial \spc{L}$ be a closest point to $\gamma(0)$ and
$\alpha=\mangle(\gamma^+(0),\dir{\gamma(0)}p)$.

\begin{wrapfigure}{r}{40mm}%???REDO
\begin{lpic}[t(0mm),b(10mm),r(0mm),l(0mm)]{pics/dist-to-bry(0.4)}
\lbl[b]{14,144;$\tilde \gamma(0)$}
\lbl[b]{63,158;$\tilde \gamma(\tau)$}
\lbl[lt]{22,134;$\alpha$}
\lbl[b]{18,39;$\tilde \beta$}
\lbl[t]{15,15;$\tilde p$}
\lbl[t]{63,15;$\tilde q$}
\lbl[lb]{75,17;$\partial\Lob[+]{2}{\kappa}$}
\end{lpic}
\end{wrapfigure}

Consider the following configuration in the model halfplane $\Lob[+]2\kappa$: 
Take a point $\tilde p\in\partial\Lob[+]2{\kappa}$ and consider the geodesic $\tilde \gamma$ in
$\Lob[+]2\kappa$ such that 
\[\dist{\gamma(0)}{p}{}=\dist{\tilde \gamma(0)}{\tilde p}{}=\dist{{\partial\Lob[+]2\kappa}}{\tilde \gamma(0)}{},\] 
so $\tilde p$ is the closest point to $\tilde \gamma(0)$ on the
boundary
and
\[\mangle(\tilde \gamma^+(0),\dir{\tilde \gamma(0)}{\tilde p})=\alpha.\]

If $\kappa>0$, then $\tilde p$ exists only if $\dist{\gamma(0)}{p}{}\le
\frac{\pi}{2\cdot\sqrt\kappa}$. 
But this is always the case since otherwise any small
variation of $p$ in $\partial \spc{L}$ decreases distance $\dist{\gamma(0)}{p}{}$.
Indeed, take $p'\in \partial \spc{L}$ near $p$. %???why it exists???
Since $\partial \spc{L}$ is an extremal subset, we have $\mangle\hinge{p}{p'}{\gamma(0)}\le \tfrac\pi2$ and if 
$\dist{\gamma(0)}{p}{}>
\frac{\pi}{2\cdot\sqrt\kappa}$, angle comparison  implies that $\dist{\gamma(0)}{p'}{}<\dist{\gamma(0)}{p}{}$.

Then it is sufficient to show that 
\[\dist{{\partial \spc{L}}}{\gamma(\tau)}{} \le
\dist{{\partial\Lob[+]2\kappa}}{\tilde \gamma(\tau)}{}+o(\tau^2).\eqlbl{eq:thm:dist-to-bry*}\]
Set 
$\beta(\tau)=\mangle\hinge p{\gamma(0)}{\gamma(\tau)}$
and
$\tilde \beta(\tau)=\mangle\hinge{\tilde p}{\tilde \gamma(0)}{\tilde \gamma(\tau)}$.
From the angle comparison,
\[\dist{p}{\gamma(\tau)}{}
\le
\dist{\tilde p}{\tilde \gamma(\tau)}{}\]
and
\[\theta(\tau)=\max\left\{0,\,\tilde \beta(\tau)-\beta(\tau)\right\}=o(\tau).\eqlbl{eq:thm:dist-to-bry**}\]
Note that the tangent cone at $p$ splits
: $\T_p \spc{L}\iso\RR_{\ge0}\times \T_p\partial
\spc{L}$;
this follows from the fact that $p$ lies on a shortest path between
two preimages of $\gamma(0)$ in the doubling??? $\tilde {\spc{L}}$ of $\spc{L}$, see
???.
Therefore there is a direction $\xi\in \T_p\partial
\spc{L}$ such that $\mangle(\dir p{\gamma(\tau)},\xi)+\beta(\tau)=\tfrac\pi2$.

Set $q=\gexp\mc\kappa_p\left(\dist[{{}}]{\tilde p}{\tilde q}{}\cdot\xi\right)$.%
\footnote{\label{qg-grad} 
Alternatively, one can set $q=\gamma(\dist{\tilde p}{\tilde q}{})$, where $\gamma$ is a
quasigeodesic in $\partial \spc{L}$ starting at $p$ in direction $\frac{w}{|w|}\in
\Sigma_p$; 
its existance follows from Theorem~\ref{thm:exist-qg-ext}.} 
Since gradient curves preserve extremal subsets $q\in \partial \spc{L}$ (see
Property~\ref{grad-in-extr} on page~\pageref{grad-in-extr}).
Applying the comparison from Section~\ref{sph-hyp-exp}, we get
\begin{align*}
\dist{{\partial \spc{L}}}{\gamma(\tau)}{}
&\le\dist{q}{\gamma(\tau)}{}
\le
\\
&\le
\side\kappa \hinge{p}{q}{\gamma(\tau)}
\le
\\
&\le\dist{\tilde q}{\tilde \gamma(\tau)}{}
+O\left(\dist[{{}}]{\tilde p}{\tilde q}{}\cdot\theta(\tau)\right).
\end{align*}
Clearly $\dist{\tilde p}{\tilde q}{}=O(\tau)$;
thus, applying \ref{eq:thm:dist-to-bry**}, we get \ref{eq:thm:dist-to-bry*}.
\qeds

For a subset $A$ of metric space $\spc{X}$ 
we will denote by $\Fr A=\Fr_\spc{X} A$\index{$\Fr$} the \emph{relative boundary}\index{relative boundary} of $A$ in $\spc{X}$ 
and by $\Int A=\Int_\spc{X} A$\index{$\Int$} the \emph{interior}\index{interior} of $A$ in $\spc{X}$ 

\begin{thm}{Theorem}\label{thm:fr-bry}
Let $\spc{L}$ be an $m$-dimensional complete length $\Alex\kappa$ space and $K\subset \spc{L}$ be a convex closed subset with nonempty interior.
Then $K$ be an $m$-dimensional complete length $\Alex\kappa$ space, 
moreover 
\[\partial K=\Fr_\spc{L} K\cup(\partial \spc{L}\cap K).\]

In particular, if $\partial \spc{L}=\emptyset$, then $\partial K=\Fr_\spc{L} K$.
\end{thm}

\parit{Proof.}
Evidently $K$ is a complete $m$-dimensional length $\Alex{}$ space.

To prove that $\Fr_\spc{L} K\subset \partial K$ we use induction by $m$; the base $m=1$ is trivial.

Let $p\in\Fr_\spc{L} K$ and $\eps>0$,
choose a point $x\in \spc{L}\setminus K$ which is $\tfrac\eps2$-close to $p$ 
and let $q\in K$ be a closest point in $K$ to $x$.
Clearly $\dist{x}{q}{}\le\dist{x}{p}{}$, therefore $q$ is $\eps$-colse to $p$.
Clearly, for any $\xi\in\Sigma_{q}K\subset \Sigma_{q}\spc{L}$ we have $\mangle(\xi,\dir {q}x)\ge \tfrac\pi2$.
Therefore, $\Fr_{\Sigma_{q}\spc{L}}\Sigma_{q}K\not=\emptyset$.
From the induction hypothesis, we have $\partial\Sigma_{q}\not=\emptyset$;
that is, $q\in\partial K$.
Since $\partial K$ is closed (???) and $\eps$ is arbitrary positive number, 
we get that $p\in \partial K$, or $\Fr_\spc{L} K\subset\partial K$.

Finally, if $p\in\Int_\spc{L} K$, then $\Sigma_pK=\Sigma_p\spc{L}$.
Thus, in this case, $p\in\partial K\ \z\Leftrightarrow\ p\in\partial \spc{L}$ and result follows.
\qeds

\section{Boundary and rank}

If in addition $\kay=m-1$ and $p\notin\partial\spc{L}$, then
$w_p^{-1}(q)$ is  a convex set in $\Delta^\kay$ which contains at least two distinct points. 

???
It remains to prove the second part of the theorem.
According to ???,
$\partial\spc{L}$ is a closed set in $\spc{L}$.
If $p\notin\partial\spc{L}$, then we can assume that $\Omega\cap\partial\spc{L}=\emptyset$.

\begin{clm}{}
Assume $w_p^{-1}(q)$ is a one point set
for some $q\in \Omega$.
Then $\T_q\iso \EE^m$.
\end{clm}

\parit{Proof of the claim.}
Let $\{\bm{x}\}=w_p^{-1}(q)$.
Consider function 
\[f_{\bm{x}}=\sum_ix^if^i.\]
Note that $\dd_qf_{\bm{x}}\le 0$.
In particular the set $E=\set{v\in\T_q}{\dd_qf_{\bm{x}}(v)=0}$ is convex.

Let us prove that $E$ contains an isometric copy of $\EE^{m-1}$. 
???

From the Toponogov splitting theorem $\T_p\iso \EE^{m-1}\oplus \spc{R}$,
where $\spc{R}$ is a one-dimensional $\Alex0$ space with a cone structure.
From Theorem~\ref{thm:dim=1.CBB}, $\spc{R}$ is isometric to $\RR$ or $\RR_{\ge0}$;
in the later case $q\in\partial\spc{L}$, which is impossible.
Threfore $\spc{R}\iso \RR$ and $\T_q\iso \EE^{m-1}$.

\section{Remarks and open problems}

The following is the oldest conjecure in Alexandrov geometry.

\begin{thm}{Conjecture}\label{conj:bry}
The boundary of an Alexandrov's space equipped with length-metric is an
Alexandrov's space with the same lower curvature bound.
\end{thm}

This is equivalent to the following:

\begin{thm}{Conjecture}\label{conj:bry'} 
Let $\spc{L}$ be an Alexandrov space without boundary. 
Then a convex hypersurface in $\spc{L}$ equipped with length-metric is an Alexandrov's space with the same lower curvature bound.
\end{thm}

This conjecture, if true, would give affirmative ansewer to Problem~\ref{open:varray}.
???
Indeed, according to ???,
parabolic cone $\RR\warp{t\mapsto e^{\lambda\cdot t}}\spc{L}$
For example if $\spc{L}$ is a non-negatively curved Alexandrov's space and $f:\spc{L}\to\RR$ is concave (so $\spc{L}$ is necessarily open) then for any $t$ the graph
\[\spc{L}_t=\{\,(x,t\cdot f(x))\in \spc{L}\times\RR\,\}\]
with length-metric would be an Alexandrov's space. 
Clearly $\spc{L}_t\GHto \spc{L}$ as $t\to0$. 
An analogous construction exists for semiconcave functions on closed manifolds, but
one has to take a \emph{parabolic cone}\footnote{see footnote~\ref{par-cone} on page~\pageref{par-cone}} instead of the product.




If in addition $\spc{L}$ is Riemannian manifold
then, accoording to Buyalo's theorem (\ref{thm:buyalo}),
the conjecture \ref{conj:bry'} is true.
At the moment all known proofs of Buyalo's theorem using smoothing and Gauss formula.
Thus at the moment it is problemattic to generalize such proof.
There is yet one beautiful synthetic proof in \cite{milka-conv}, which works only for convex hypersurfaces in the Euclidian space.

There is a chance of attacking this problem by proving a type of the Gauss formula for
Alexandrov's spaces. 
One has to start with defining a curvature tensor of Alexandrov's spaces (it
should be a measure-valued tensor field), then prove that the constructed tensor is really responsible for the geometry of the space. 
Such things were already done in the two-dimensional case and for spaces with bilaterly bounded curvature, see \cite{reshetnyak:curvature} and
\cite{nikolaev:curvature} respectively.
So far the best results in this direction are given in \cite{perelman:DC}, 
see also Section~\ref{app-tight} for more details.
Some further steps in this directions were also done in \cite{lebedeva-curv}. 

Almost everything that is known so far about the length-metric of a boundary is also known for the length-metric of a general extremal subset.
In \cite{perelman-petrunin:extremal}, it was conjectured  that an analog of Conjecture~\ref{conj:bry} is true for any \emph{primitive extremal subset}, but it turned out to be wrong; a simple example was constructed in \cite{petrunin:extremal}. 
All such examples appear when codimension of extremal subset is $\ge 3$.
So it still might be true that

\begin{thm}{Conjecture}\label{conj:codim=2}
Let $\spc{L}$ be an $m$-dimensional complete length $\Alex\kappa$ space, $E\subset \spc{L}$ be a primitive extremal subset and $\codim E=2$.
Then $E$ equipped with length-metric is an $m$-dimensional $\Alex\kappa$ space
\end{thm}


\section{Exercises}

The following gives a converse to Theorem~\ref{thm:dist-to-bry}.

\begin{thm}{Exercise} 
Let $\spc{L}$ be an $m$-dimensional complete length $\Alex0$ space and $\mathfrak{C}\subset  \spc{L}$ be a closed subset such that its distance function 
$\distfun{\mathfrak{C}}{}{}\:\spc{L}\to \RR$ is a concave.
Prove that $\mathfrak{C}$ is a $(m-1)$-dimensional extremal subset in $\spc{L}$.
\end{thm}

\begin{thm}{Exercise}\label{ex:supporting-vector}
Let $\spc{L}$ be an $m$-dimensional complete length $\Alex0$ space
and $f\:\spc{L}\subto\RR$ is locally Lipschitz semiconcave subfunction and $p\in\Dom f$.
A vector $w\in\T_p$ is called \emph{supporting}\index{supporting vectore} 
for $f$ at $p$ if 
\[\dd_pf(x)+\<w,x\>\le 0\]
for any $x\in \T_p$.

\begin{subthm}{}
Show that if $p\notin\partial\spc{L}$ there is a supporting vector for $f$ at $p$.
\end{subthm}

\begin{subthm}{}
Show that set of supporting vectors of $f$ at $p$
forms a convex closed set in $\T_p$.
\end{subthm}

\begin{subthm}{}
Assume $\partial{\spc{L}}\ne\emptyset$;
denote by $\spc{W}$ the doubling of $\spc{L}$ 
in its boundary and let $\proj\:\spc{W}\to\spc{L}$
be the natural projection.

Show that $f$ admits a supporting vector at any point $p\in\Dom f$ 
if and only if $f\circ\proj\:\spc{W}\to \RR$ is a semiconcave subfunction.
\end{subthm}

\end{thm}









