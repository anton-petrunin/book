%%!TEX root = all.tex
\chapter{Parallel transport and second variation}

\section{Burago's conjecture}

The statement of the following theorem was conjectured by Yu. Burago in the end of 80's; it was proved in \cite{petrunin:parallel}.
Earlier, in \cite[7.16]{BGP} a weaker statement was proved.
Yet earlier in \cite{milka-parallel-conv}, it was shown that the statement is true for convex hypersurfaces in Euclidean space.

\begin{thm}{Theorem}\label{thm:parallel}
Let $\spc{L}$ be an $m$-dimensional complete length $\Alex\kappa$ space
and $[pq]$ be a geodesic in $\spc{L}$. 
Then for any two points $x,y\in \l] p q \r[$, we have
$\T_x\iso \T_y$.
\end{thm}

In ???, we will show existance of a cone isometry $???:\T_x\to \T_y$ for which an analog of second variation formula holds.

\begin{thm}{Corollary}\label{cor:reg-end}
If $p\in \spc{L}$ is a then for any geodesic $[pq]$ and any point $\bar p\in \l[ p q \r[$ we have $\Sigma_{\bar p}\ge \Sigma_p$.

In particular, if $p\in \spc{L}$ is a Euclidean point, then so is $\bar p$.
\end{thm}

\begin{thm}{Corollary}\label{cor:reg-conv}
Let $\spc{L}$ be an $m$-dimensional complete length $\Alex\kappa$ space and 
$Q\subset \spc{L}$ is a subset of points such that $p\in Q$ 
if $\Sigma_p\ge \Sigma_q$ for some point $q$.
Then $Q$ is convex.

In particular, the set of Euclidean points in $\spc{L}$ forms a convex set.
\end{thm}

Note that according to \ref{LinDim+-f}, the set of Euclidean points 
is dense G-delta set and according to ??? its complement has zero volume.

\parit{Proof of \ref{cor:reg-conv}.} 
It is sufficient to prove only the second part of corollary.

Sinse $p$ is Euclidean, $\SS^{m-1}\iso\Sigma_p$.
From \ref{cor:simicont-Sigma}, $\SS^{m-1}\ge\Sigma_x$ for any $x\in \spc{L}$.

Let $x\in \l] p q \r[$. 
Take a sequence of points $p_n\in \l] p q \r[$ such that $p_n\to p$ as $n\to\infty$.
According to \ref{thm:parallel}, $\Sigma_x\iso\Sigma_{p_n}$. 
Thus, from \ref{cor:simicont-Sigma}, $\Sigma_x\ge\Sigma_p$.
I.e. $\SS^{m-1}\ge \Sigma_x\ge\SS^{m-1}$ and
 acoording to \ref{lem:>=-isometry}, $\Sigma_x\iso\SS^{m-1}$.
\qeds

The proof below combains idea from \cite{BGP} with an analog of geometric interpretation of parallel transport given in \cite{nikolaev}.

\parit{Proof of \ref{thm:parallel}.} 
First let us describe plan of the proof.
According to ???, for any $x\in \l] p q \r[$
\[\T_{x}\iso \T_{x}^\top \oplus \T_{x}^\perp,\] 
where $\T_{x}^\top$ and $\T_{x}^\perp$ are subcones of $\T_x$ correspondingly tangent and perependicular to $[pq]$.
Clearly, $\T_x^\top$ is isometric to $\RR$.
Given a vector $v\in \T_x$, let us denote by $v^\top$ and $v^\perp$ its orthogonal projection to $\T_x^\top$ and $\T_x^\perp$ correspondingly.

Given two points $x,y\in \l] p q \r[$,
we will construct a noncontracting map $\Upsilon_{x,y}:\T_x^\perp\to \T_y^\perp$
such that $|\Upsilon_{x,y}(v)|\le|v|$ for any $v\in \T_x^\perp$. 
Once it is done, applying \ref{lem:>=-isometry} to restrictions of $\Upsilon_{x,y}$ and $\Upsilon_{y,x}$ to unit balls, we get that $\T_x^\perp\iso \T_y^\perp$ and therefore $\T_x\iso \T_y$. 

In order to construct $\Upsilon_{x,y}$ we first construct an other type of maps  $\Pi_{x,y}:\T_x^\perp\to \T_y^\perp$ and then obtain $\Upsilon_{x,y}$ as a limit of compositions of $\Pi_{x_{i-1},x_i}$ for partition $x=x_0,x_1,\dots,x_n=y$ of $[xy]$.

\smallskip

Now we follow the plan. 
We will give a proof in case $\kappa=0$.
In the general case calculations more complicated;
we include only key formulas in the footnotes.

Without loss of generality, we can assume that on the geodesic $[p q]$ the points appear in the following order $p,x,y,q$.

Let us define map $\Pi_{x,y}:\T_x^\perp\to \T_y^\perp$ by\footnote{???}
\[\Pi_{x,y}(v)
=
\lim_{n\to\o}n\cdot{\ddir{y}{\gexp\mc0_{x}(\eps\cdot v)}^\perp}.
\eqlbl{eq:thm:parallel-2}\]
In the following claim, we list properties of $\Pi_{x,y}$, which will be proved later.

\begin{thm}{Claim}\label{clm:prop-of-Pi}

\begin{subthm}{<Const v} There is $\Const\in\RR$ such that 
\[\l|\ddir{y}{\gexp\mc0_{x}v}^\perp\r|\le \Const\cdot|v|.\]
In particular, $\Pi_{x,y}(v)$ is defined for any $v\in \T_x^\perp$.
\end{subthm}

\begin{subthm}{noncontracting} $\Pi_{x,y}$ is a non contracting map%
\footnote{For general $\kappa$, the map ???
$v\mapsto \frac{\sn\kappa\dist[{{}}]{x}{y}{}}{\dist{x}{y}{}}\cdot\Pi_{x,y}(v)$
is non contracting.}%
.
\end{subthm}

\begin{subthm}{2nd-var}
(second variation)
For any $v\in\T_x$ and $w\in\T_y$ we have%
%%%%%%%%%%%%%%%%%%%%%%%%%%%%%%%%%%
\footnote{For general $\kappa$,??? \vskip-6mm
\[\dist{\gexp\mc\kappa_{x}(\eps\cdot v)}{\gexp\mc\kappa_{y}(\eps\cdot w)}{}
\le
\dist{x}{y}{}
+\frac{\eps^2\cdot\cs\kappa\dist[{{}}]{x}{y}{}}{2\cdot\dist[{{}}]{x}{y}{}}
\cdot\l[|v|^2-|\Pi_{x,y}(v)|^2+\dist[2]{\Pi_{x,y}(v)}{w}{}\r]
+o(\eps^2).\]}%
%%%%%%%%%%%%%%%%%%%%%%%%%%%%%%%%%%
\begin{multline*}
\dist{\gexp\mc\kappa_x(\eps\cdot v)}{\gexp\mc\kappa_{y}(\eps\cdot w)}{}
\le
\\
\le \dist{x}{y}{}+\frac{\eps^2}{2\cdot\dist[{{}}]{x}{y}{}}\cdot\l[|v|^2-|\Pi_{x,y}(v)|^2
+\dist[2]{\Pi_{x,y}(v)}{w}{}\r]+o(\eps^2).
\end{multline*}
for $\o$-almost all $\eps>0$.
\end{subthm}
\end{thm}

Let us divide $[xy]$ into $n$ equal parts by points $x=x_0,x_1,\dots,x_n=y$.
Set $\Pi_i=\Pi_{x_{i-1},x_i}$ and consider composition
\[\Phi_n=\Pi_n\circ\Pi_{n-1}\circ\dots\circ\Pi_1:\T_x^\perp\to \T_y^\perp.\]

\begin{thm}{Claim}\label{clm:prop-of-Phi}
There is $\Const\in\RR$ such that for any $v\in \T_x^\perp$
\[|\Phi_n(v)|\le\l(1+\frac{\Const}{n}\r)\cdot|v|.\]
\end{thm}

Thus passing to a partial limit $\Phi_n\to\Upsilon_{x,y}$ as $n\to\infty$, we get the required map.
It only remains to prove claims \ref{clm:prop-of-Pi} and \ref{clm:prop-of-Phi}.
\qeds

\begin{wrapfigure}{r}{70mm}
\begin{lpic}[t(5mm),b(15mm),r(0mm),l(0mm)]{pics/Pi-const(0.6)}
\lbl[tr]{0,0;$p$}
\lbl[tl]{105,0;$q$}
\lbl[t]{31,-1;$x$}
\lbl[t]{76,-1;$y$}
\lbl[lb]{38,27;$x'=\gexp_{x}v$}
\lbl[r]{30,28;$v$}
\lbl[lb]{78,5;$\alpha$}
\end{lpic}
\end{wrapfigure}

\parit{Proof of \ref{clm:prop-of-Pi}.}\\
\noi{\it (\ref{SHORT.<Const v}).}
Set $x'=\gexp\mc\kappa_x v$.\\
From hinge comparison,
\[\dist{p}{x'}{}\le \dist{p}{x}{}+O(|v|^2),\]
\[\dist{y}{x'}{}\le\dist{y}{x}{}+O(|v|^2),\]
\[\dist{q}{x'}{}\le\dist{q}{x}{}+O(|v|^2).\]
By triangle inequalities $\dist{q}{x'}{}\ge\dist{p}{q}{}-\dist{p}{x'}{}$ and $\dist{y}{x'}{}\ge\dist{p}{y}{}-\dist{y}{x'}{}$;
thus,
\[\dist{q}{x'}{}=\dist{q}{x}{}+O(|v|^2)\ \ \text{and} \ \ \dist{y}{x'}{}=\dist{y}{x}{}+O(|v|^2).\]
Therefore
\[\alpha
=
\alpha(v)
\df
\mangle\hinge y{x'}q
\ge
\angk\kappa y{x'}q\ge\pi-O(|v|).\]
Thus 
\[
\l|\ddir{y}{\circ\gexp\mc\kappa_{x}v}^\perp\r|
=
\l|\ddir{y}{x'}^\perp\r|
=
\dist[{{}}]{x'}{y}{}\cdot\sin\alpha=O(|v|).\]

\parit{(\ref{SHORT.noncontracting}).} 
Note that from above we also have
\[\l|\ddir{y}{\circ\gexp\mc\kappa_x v}^\top\r|
=
\dist[{{}}]{x'}{y}{}\cdot\cos\alpha
=\dist{x}{y}{}+O(|v|^2).
\eqlbl{eq:clm:prop-of-Pi-1}\]
Given $v\in T^\perp_x$ and $\eps>0$, set 
$\hat v_\eps=\ddir y{\gexp_x (\eps\cdot v)}\in \T_y$.
It is sufficient  to show that for any $u,v\in \T_x^\perp$, we have%
\footnote{For general $\kappa$, 
\[\dist{\hat u_\eps^\perp}{\hat v_\eps^\perp}{}
\ge
\eps\cdot\frac{\sn\kappa\dist[{{}}]{x}{y}{}}{\dist{x}{y}{}}
\cdot 
\dist[{{}}]{u}{v}{}+o(\eps)\leqno\t{\ref{eq:clm:prop-of-Pi-2}}\mc\kappa\]
}%
\[\dist{\hat u_\eps^\perp}{\hat v_\eps^\perp}{}
\ge
\eps\cdot\dist[{{}}]{u}{v}{}+o(\eps).\eqlbl{eq:clm:prop-of-Pi-2}\mc0\]

Since $\kappa=0$, the map $\log y{???}$ is non-contracting; therefore%
\footnote{For general $\kappa$, we have ???
\[\l(1+O(\dist{x}{y}{})\r)\cdot\dist[{{}}]{\hat v_\eps}{\hat w_\eps}{}
\ge
\side\kappa \hinge{\0}{\hat v_\eps}{\hat w_\eps}
\ge
\dist{\gexp\mc\kappa_x(\eps\cdot v)}{\gexp\mc\kappa_x(\eps\cdot  w)}{}
=
\eps\cdot \dist[{{}}]{v}{w}{}+o(\eps).\]
}%
\[\dist{\hat v_\eps}{\hat w_\eps}{}
\ge
\dist{\gexp\mc0_x(\eps\cdot v)}{\gexp\mc0_x(\eps\cdot w)}{}
=
\eps\cdot\dist[{{}}]{v}{w}{}+o(\eps).\]
From \ref{eq:clm:prop-of-Pi-1}, we get 
\[\dist{\hat v_\eps^\top}{\hat w_\eps^\top}{}
=
O(\eps^2).\]
We obtain \ref{eq:clm:prop-of-Pi-2}, since 
\[\dist[2]{\hat v_\eps^\top}{\hat w_\eps^\top}{}
+
\dist[2]{\hat v_\eps^\perp}{\hat w_\eps^\perp}{}
=
\dist[2]{\hat v_\eps}{\hat w_\eps}{}.\]

\parit{(\ref{SHORT.2nd-var})}
As above, for $v\in \T^\perp_x$ and $\eps>0$, 
set $\hat v_\eps=\ddir y{\,\gexp\mc0_x (\eps\cdot v)}\in \T_y$
\begin{center}
\begin{lpic}[t(-0mm),b(0mm),r(0mm),l(0mm)]{pics/2nd-var(0.4)}
\lbl[r]{0,47;$\hat v_\eps$}
\lbl[tl]{138,47;$\hat v_\eps^\perp$}
\lbl[tl]{138,5;$\0$}
\lbl[b]{162,76;$\eps\cdot w$}
\lbl[r]{120,54;$\eps\cdot\Pi_{x,y}(v)$}
\lbl{160,10;{\Large $\T_y$}}
\end{lpic}
\end{center}

From comparison, we have%
\footnote{For general $\kappa$, ???
}%
\[|\hat v_\eps|^2=\dist[2]{y}{\gexp\mc0_x (\eps\cdot v)}{}
\le 
\dist[2]{x}{y}{}+\eps^2\cdot|v|^2.\]
Clearly, 
$\angk0 {\hat v_\eps^\perp}{\0}{v_\eps}=\mangle\hinge{\hat v_\eps^\perp}{\0}{v_\eps}=\tfrac\pi2$
and 
$\angk0 {\hat v_\eps^\perp}{w}{v_\eps}=\mangle\hinge{\hat v_\eps^\perp}{w}{v_\eps}=\tfrac\pi2$.
Thus, applying Pythegorian theorem, 
\begin{align*}
\dist[2]{\hat v_\eps}{\eps\cdot w}{}
&=
\dist[2]{\hat v_\eps}{\hat v_\eps^\perp}{}
+
\dist[2]{\hat v_\eps^\perp}{\eps\cdot w}{}=
\\
&=|\hat v_\eps|^2-|\hat v_\eps^\perp|^2
+
\dist[2]{\hat v_\eps^\perp}{\eps\cdot w}{}
\le
\\
&\le 
\dist[2]{x}{y}{}+\eps^2{|v|^2}-|\hat v_\eps^\perp|^2
+
\dist[2]{\hat v_\eps^\perp}{\eps\cdot w}{}.
\end{align*}
Sinse $\frac{1}{\eps}\cdot\hat v_\eps^\perp\to \Pi_{x,y}(v)$ as $n\to\o$, we have
\[\dist[2]{\hat v_\eps}{\eps\cdot w}{}
\le 
\dist[2]{x}{y}{}+\eps^2\cdot\biggl[{|v|^2}-|\Pi_{x,y}(v)|^2+
\dist[2]{\Pi_{x,y}(v)}{w}{}\biggr]+o(\eps^2)\]
for $\o$-almost all $\eps>0$. 
\qeds

\parit{Proof of \ref{clm:prop-of-Phi}.}
For $v\in \T_x^\perp$, construct recurcevly a sequence vectors 
$v_i\in \T_{x_i}^\perp$, 
$v_0=v$ and $v_i=\Pi_i(v_{i-1})$ for all $i$.

By triangle inequality, we have
\begin{multline*}
\dist{p}{q}{}
\le 
\dist{p}{\gexp_{x_0}(\eps\cdot v_0)}{}
+
\dist{\gexp_{x_0}(\eps\cdot v_0)}{\gexp_{x_1}(\eps\cdot v_1)}{}
+\dots\\
\dots+\dist{\gexp_{x_{n-1}}(\eps\cdot v_{n-1})}{\gexp_{x_n}(\eps\cdot v_n)}{}
+\dist{\gexp_{x_n}(\eps\cdot v_n)}{q}{}.
\end{multline*}
Clealrly
\[\dist{p}{\gexp_{x_0}(\eps\cdot v_0)}{}
\le 
\dist{p}{x}{}+\tfrac{\eps^2}2\cdot\frac{|v_0|^2}{\dist{p}{x}{}}+o(\eps^2),\]
\[\dist{\gexp_{x_n}(\eps\cdot v_n)}{q}{}
\le 
\dist{y}{q}{}+\tfrac{\eps^2}2\cdot\frac{|v_n|^2}{\dist{y}{q}{}}+o(\eps^2).\]
Therefore, according to \ref{2nd-var}, for $\o$-almost all $\eps>0$ we have
\begin{align*}\dist{p}{q}{}
&\le\dist{p}{q}{}+\tfrac{\eps^2}2\cdot\biggl\{\biggr.\frac{|v_0|^2}{\dist{p}{x}{}}+\\
&{}\ \ \ \ +
\frac{n}{\dist{x}{y}{}}\cdot\biggl[(|v_0|^2-|v_1|^2)+(|v_1|^2-|v_2|^2)+\dots
+(|v_{n-1}|^2-|v_n|^2)\biggr]+\\
&{}\ \ \ \ +
\frac{|v_n|^2}{\dist{y}{q}{}}\biggl.\biggr\}+o(\eps^2)=\\
&=
\dist{p}{q}{}
+
\tfrac{\eps^2}2\cdot\l\{\l(\frac{1}{\dist{p}{x}{}}+
\frac{n}{\dist{x}{y}{}}\r)\cdot|v_0|^2
+\l(\frac{1}{\dist{y}{q}{}}-\frac{n}{\dist{x}{y}{}}\r)\cdot|v_n|^2\r\}
+o(\eps^2).
\end{align*}
Thus,
\[\l(
\frac{1}{\dist{p}{x}{}}
+\frac{n}{\dist{x}{y}{}}\r)\cdot|v_0|^2
+\l(\frac{1}{\dist{y}{q}{}}
-\frac{n}{\dist{x}{y}{}}
\r)\cdot|v_n|^2\ge 0.\]
Since $v_0=v$ and $\Phi_n(v)=v_n$, the result follows.\qeds







\section{Second variation}

In the formulation of second variation formula,
we will use certain tangent $\kappa$-cone as a comparison space.
Let us first dicuss basic geometry behind it.

Given a Euclidean cone $K$, let us denote by $(K)\mc\kappa$ the corresponding $\kappa$-cone;
that is, if $K=\Cone\Sigma$, then $(K)\mc\kappa=\Cone\mc\kappa\Sigma$.
We will denote by $\tau\mc\kappa\:K\to(K)\mc\kappa$ the \textbf{tau}tological submap;
that is, the point with polar coordinates $(x,\xi)\in\RR\times\Sigma$ mapped to the point with the same polar coordinates in $(K)\mc\kappa$.
Clearly $\Dom\tau\mc\kappa\z=\cBall[\0,\varpi\kappa]\subset K$.
If $K$ is $\Alex0$, then $(K)\mc\kappa$ is $\Alex\kappa$.
It is easy to see that given cones $K$, $K_1$ and $K_2$, we have
\[K\iso K_1\times K_2\ \ \Longleftrightarrow\ \ (K)\mc\kappa\iso(K_1)\mc\kappa\times_{\cs\kappa\circ\distfun{o}{}{}} (K_2)\mc\kappa.\]

Let us remind that if $x\in \l] p q \r[$ 
then tangent space $\T_x$ at $x$ splits as $\T_x=\T_x^\perp\oplus\T_x^\top$ where $\T_p^\perp$ and $\T_x^\top$ are normal and tangent subcones to $[pq]$ (here $\T_x^\top$ is isometric to $\RR$).
Thus, we have 
\[(\T_x)\mc\kappa
=
(\T_x^\perp)\mc\kappa\times_{\cs\kappa\circ\distfun{o}{}{}}(\T_x^\top)\mc\kappa.\]
Set $\~p=\tau\mc\kappa\cdot\ddir x p$ and $\~q=\tau\mc\kappa\cdot\ddir x q$.
Clearly $\~p$ and $\~q$ are uniquely defined, 
both $\~p$ and $\~q$ lie in the fiber over $o\in \T_x^\perp$ 
and $\dist{\~p}{\~q}{}=\dist{p}{q}{}$.


\begin{thm}{???Theorem}\label{thm:2nd-var}
Let $\o$ be a ???,
$\spc{L}$ be an $m$-dimensional complete length $\Alex\kappa$ space,
$[pq]$ be a geodesic in $\spc{L}$ and $x\in \l] p q \r[$.
Set $\~p=\tau\mc\kappa\ddir{x}p$ and $\~q=\tau\mc\kappa\ddir {x} q$, so $\~p,\~q\in(\T_x)\mc\kappa$.

Then there is two norm-preserving noncontracting maps 
$\map_p\:\T_p\to\T_{\~p}$ and $\map_q\:\T_q\to\T_{\~q}$, 
such that for any two vectors $v\in\T_p$ and $w\in\T_q$ we have
\[\dist{\gexp(\eps\cdot v)}{\gexp(\eps\cdot w)}{}
\le
\dist{\exp(\eps\cdot\map_p v)}{\exp(\eps\cdot\map_qw)}{}
+o(\eps^2)
\eqlbl{eq:thm:2nd-var*}\]
for $\o$-almost all $\eps$.
\end{thm}

One can the right part of \ref{eq:thm:2nd-var*} competely agebrically, but in general the formulas become bit monster-like.

\begin{thm}{Corollaries} 
In the assumtions of \ref{thm:2nd-var},
\begin{enumerate}
\item If in addition $\T_p\iso\T_x$ ($\T_q\iso\T_x$) then $\map_p$ (correspondingly $\map_q$) is an isometry;
\item The restriction of $\map_p$ to $\Lin_p$ (correspondingly $\map_q$ to $\Lin_q$) is an isometric map.
\item If $\T_p\iso \EE^m$ (correspondingly $\T_q\iso \EE^m$) then $\map_p$ (correspondingly $\map_q$) is isometry.
\item For almost all $(p,q)\in \spc{L}\times \spc{L}$, 
the maps $\map_p$ and $\map_q$ can be chosen independently form the $\o$.
\end{enumerate}
\end{thm}

\parit{Proofs of corollaries.} ???
\qeds

The proof of second variation formula is build on the same ideas as parallel transportation, but it uses extra technical steps.
The key lemma in the proof is the following, it may regarded as a partial case of corollary ???.

\begin{thm}{Lemma}
Let $\o$ be a ???,
$\spc{L}$ be an $m$-dimensional complete length $\Alex\kappa$ space,
$[pq]$ be a geodesic in $\spc{L}$ and $x,y$ be two distinct points $\l] p q \r[$.

???

Set $\~p=???\log_{\kappa,x} p$ and $\~q=???\log_{\kappa,x} q$.

Then there is two norm-preserving noncontracting maps 
$\map_p\:\T_p\to\T_{\~p}$ and $\map_q\:\T_q\to\T_{\~q}$, 
such that for any two vectors $v\in\T_p$ and $w\in\T_q$ we have
\[\dist{\gexp_p\eps\cdot v}{\gexp_q\eps\cdot w}{}
\le
\dist{\exp_{\~p}\eps\cdot\map_p v}{\exp_{\~q}\eps\cdot\map_q w}{}
+o(\eps^2)\]
for $\o$-almost all $\eps$.
\end{thm}

\section{Remarks and open problems}

\begin{thm}{Open question}
Let $\spc{L}$ be an $m$-dimensional complete length $\Alex{}$ space, 
$[pq]$ be a geodesics in $\spc{L}$,
$\xi=\dir p q$ 
and $x\in \l] p q \r[$.
Is it true that $\T_{\xi}\T_p\iso\T_x$?
\end{thm}
 
