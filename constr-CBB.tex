%%!TEX root = the-constr-CBB.tex
\chapter{Constructions}


\section{Busemann functions}\label{sec:busemann}

Let $\spc{X}$ be a metric space 
and $\gamma:[0,\infty)\to \spc{X}$ be a \emph{ray} (i.e. a distance preserving map of $[0,\infty)$ in $\spc{X}$).
The function 
\[\bus_\gamma(x)=\lim_{t\to\infty}\dist{\gamma(t)}{x}{}- t\eqlbl{eq:def:busemann*}\]
is called the  \emph{Busemann function}\index{Busemann function} of $\gamma$. 

If $\spc{X}=\Cone\Sigma$, 
then for each element $\xi\in\Sigma$,
$\gamma(t)=t\cdot\xi$ forms a ray.
In this case, 
$\bus_\gamma(x)=-\<x,\xi\>$,
where $\<{*},{*}\>$ denotes scalar product (\ref{def:scalar-product}). 


\begin{thm}{Theorem}\label{thm:busemann}
Suppose $\spc{X}$ is a metric space and $\gamma\:[0,\infty)\to \spc{X}$ is a ray. 
Then the Buseman function $\bus_\gamma\:\spc{X}\to \RR$ is defined
and $1$-Lipschitz.

Moreover:
\begin{subthm}{}
 If  $\spc{X}\in \Cat{}{\kappa}$ (respectively $\spc{X}\in \CBB{}{\kappa}$) for some $\kappa<0$, then the function  $f=\exp(\sqrt{-\kappa}\cdot\bus_\gamma)$ satisfies
\[f''+\kappa\cdot f\ge 0\ \ \ (\t{respectively}\ \ f''+\kappa\cdot f\le 0).\eqlbl{eq:sec:busemann*}\]
\end{subthm}

\begin{subthm}{} If $\spc{X}\in \Cat{}{0}$ (respectively $\spc{X}\in \CBB{}{0}$), then $\bus_\gamma$ is convex (respectively concave).
\end{subthm}
\end{thm}

\parit{Proof.}
As  follows from the triangle inequality, the function $t\mapsto\dist{\gamma(t)}{x}{}- t$ is decreasing in $t$.  Clearly $\dist{\gamma(t)}{x}{}- t\ge-\dist{\gamma(0)}{x}{}$.
Thus the limit in \ref{eq:def:busemann*} is defined.

%!!! I inserted a proof.
By the definition of Busemann function,
\begin{align*}
\exp(\sqrt{-\kappa}\cdot\bus_\gamma) 
&= \exp \lim_{t\to \infty} \sqrt{-\kappa}\cdot(\dist{{\gamma (t)}}{}{} - t) 
\\
&= \lim_{t\to \infty} \left(\exp \sqrt{-\kappa}(d_{\gamma (t)} -t) + \exp
\sqrt{-\kappa}\cdot(-d_{\gamma (t)}-t)\right)\\
&=  \lim_{t\to \infty} 2 \cosh \sqrt{-\kappa}d_{\gamma (t)} \exp\sqrt{-\kappa}\cdot(-t).
\end{align*}

By the function comparison definitions of $\Cat{}{\kappa}$ (\ref{function-comp}) or $\CBB{}{\kappa}$ (\ref{comp-kappa}),  for any $p\in \spc{U}$ the function $f=\cosh \sqrt{-\kappa}\circ\dist{p}{}{}$ satisfies $f''+\kappa \cdot f\ge 0$ (respectively  $f''+\kappa \cdot f\le 0$). The result follows.
\qeds

\begin{thm}{Corollary}
Assume $\spc{K}$ be a cone.
Given $w\in  \spc{K}$ consider 
the function $f_w\:\spc{K}\to\RR$
\[f_w(x)=\<w,x\>.\]
Then
\begin{subthm}{}
If $\spc{K}\in\CBB{}{0}$ then $f_w$ is convex.
\end{subthm}
\begin{subthm}{}
If $\spc{K}\in\cCat{}{0}$ then $f_w$ is concave.
\end{subthm}
\end{thm}

\parit{Proof.}
In both cases, the statement are trivial if $w=0$.

If $w\ne 0$, consider the ray $\gamma(t)=\tfrac1{|w|}\cdot t$

\section{Sumbetry}\label{sec:quotient-CBB}

Recall that map $\sigma\:\spc{L}\to\spc{M}$ between the metric spaces $\spc{L}$ and $\spc{M}$
is called 
\emph{submetry}\index{submetry} if 
\[\sigma(\oBall(p,r)_\spc{L})=\oBall(\sigma(p),r)_{\spc{M}}\]
for any $p\in \spc{L}$ and $r\ge 0$.

Equivalently, a map $\sigma\:\spc{L}\to\spc{M}$ is called sumbetry if it is 1-Lipshitz and 1-co-Lipschitz at the same time.

Note that any submetry is an onto map.

The main souse of examples of submetries comes from isometric group actions.

Nemely, assume $\spc{L}$ is a metric space and the group $G$ acts on $\spc{L}$.
Denote by $\spc{L}/G$ the set of $G$-orbits;
let us equip it with the Hausdorff metric. 

In general, $\spc{L}/G$ is a premetric space,
but if all the $G$-orbits form closed sets in $\spc{L}$
then $\spc{L}/G$ is a metric space.

\begin{thm}{Proposition}\label{prop:submet/G}
Let $\spc{L}$ be a metric space and group $G$  acts on $\spc{L}$ by isometries  
and in such a way that every $G$-orbit is closed.
Then the projection map $\spc{L}\to \spc{L}/G$ is a sumbetry.
\end{thm}


\begin{thm}{Proposition}
\label{prop:submet-length}
Let  $\spc{L}$ be a length space 
and $\sigma\:\spc{L}\to \spc{M}$ is a submetry.
Then $\spc{M}$ is a length space.
\end{thm}

\parit{Proof.}
Fix $\eps>0$ and a pair of points $x,y\in \spc{M}$.

Since $\sigma$ is $1$-co-Lipschitz, there are points $\hat x,\hat y\in \spc{L}$
such that $\sigma(\hat x)=x$, $\sigma(\hat y)=y$ 
and $\dist{\hat x}{\hat y}{\spc{L}}<\dist{x}{y}{\spc{M}}+\eps$.

Since ${\spc{L}}$ is a length space, 
there is a curve $\gamma$ 
joining $\hat x$ to $\hat y$ in ${\spc{L}}$
such that
\[\length\gamma\le \dist{x}{y}{\spc{M}}+\eps.\]

Since $\sigma$ is $1$-Lipschitz,
there 
\[\length\sigma\circ\gamma\le \length\gamma.\]

Note that the curve $\sigma\circ\gamma$ joins $x$ to $y$
and from above
\[\length\sigma\circ\gamma<\dist{x}{y}{\spc{M}}+\eps.\]
Since $\eps>0$ is arbitrary,
we have that $\spc{M}$ is a length space.
\qeds


\begin{thm}{Theorem}\label{thm:submetry-CBB}
Let $\spc{L}\in\CBB{}{\kappa}$ and $\spc{M}$ be a metric space.
Assume there is a sumbetry $\sigma\:\spc{L}\to\spc{M}$.
Then $\spc{M}\in \CBB{}{\kappa}$.
\end{thm}

The theorem above together with Proposition~\ref{prop:submet/G}
imply the following.

\begin{thm}{Corollary}\label{thm:CBB/G}
Let $\spc{L}\in\CBB{}{\kappa}$ and the group $G$ acts on $\spc{L}$ by isometries 
and it has closed orbits.
Then $\spc{L}/G\in\CBB{}{\kappa}$. 
\end{thm}

\parit{Proof.}
Fix $\delta>0$ and a quadruple of points $p,x^1,x^2,x^3\in \spc{M}$.
Choose arbitrary $\hat p\in \spc{L}$ such that $\sigma(\hat{p})=p$.
Next $\sigma$ is submetry, we can choose the points $\hat{x}^1,\hat{x}^2,\hat{x}^3\in \spc{M}$.
such that $\sigma(\hat x_i)=x_i$ and
\[\dist{\hat{p}}{\hat{x}^i}{\spc{L}}
\lege
\dist{p}{x^i}{\spc{M}}\pm\delta\]
for all $i$.
Note that 
\[\dist{\hat{x}^i}{\hat{x}^j}{\spc{L}}
>
\dist{x^i}{x^j}{\spc{M}}-2\cdot\delta\]
for all $i$ and $j$.
Therefore given $\eps>0$, the value $\delta$ above can be chousen in such a way that the inequality
\[\angk\kappa {\hat{p}}{\hat{x}^1}{\hat{x}^2}
<
\angk\kappa p{x^1}{x^2}+\eps
\eqlbl{eq:angles-M-L}\]
holds for all $i$ and $j$.

By (1+3)-point comparison in $\spc{L}$,
we have
\[\angk\kappa {\hat{p}}{\hat{x}^1}{\hat{x}^2}
+\angk\kappa {\hat{p}}{\hat{x}^2}{\hat{x}^3}
+\angk\kappa {\hat{p}}{\hat{x}^3}{\hat{x}^1}
\le 
2\cdot\pi\]
assuming all the angles on the left hand side are defined.
Applying  \ref{eq:angles-M-L}, 
we get 
\[\angk\kappa p{x^1}{x^2}
+\angk\kappa p{x^2}{x^3}
+\angk\kappa p{x^3}{x^1}< 2\cdot\pi+3\cdot\eps.\]

Since $\eps>0$ is arbitrary the (1+3)-point comparison holds in $\spc{M}$.
It remains to apply Proposition~\ref{prop:submet-length}.
\qeds


\section{Doubling theorem}\label{sec:doubling}

\parbf{Doubling.} Let $\spc{V}$ be a metric space 
and $A\subset \spc{V}$ be a closed subset.
A metric space $\spc{W}$ glued from two copies of $\spc{V}$ along $A$ is called \emph{doubling of $\spc{V}$ in $A$}\index{doubling}.

More formally, 
there is an isometric involution of $\spc{W}$ which is called \index{reflection of doubling}\emph{reflection}
denoted as $x\mapsto x'$
and 
$\spc{W}$ contains $\spc{V}$ as a subspace 
in such a way that
for any $x\in \spc{W}$ we have $x\in \spc{V}$ or $x'\in \spc{V}$ and 
\[
\dist{x'}{y}{\spc{W}}
=\dist{x}{y'}{\spc{W}}
=\inf_{a\in A}\{\dist{x}{a}{\spc{V}}+\dist{a}{y}{\spc{V}}\}.
\]
for any $x,y\in \spc{V}$.

The image of $\spc{V}$ under the isometry $x\mapsto x'$ will be denoted by $\spc{V}'$;
it is an isometric copy of $\spc{V}$ and $\spc{V}\cap\spc{V}'=A$.
Moreover $a=a'$ $\Leftrightarrow$ $a\in A$.


Note that from above it follows that
\begin{itemize}
\item If $\spc{V}$ is a complete length space then so is $\spc{W}$.
\item If $\spc{V}$ is proper then so is $\spc{W}$.
In this case, for any $x,y\in\spc{V}$ there is $a\in A$ such that 
\[\dist{x}{a}{\spc{V}}+\dist{a}{y}{\spc{V}}=\dist{x}{y'}{\spc{W}}.\]
\end{itemize}

\parbf{Boundary strata.}
Let $\spc{L}\in\CBB{}{}$.
A closed subset $A\subset \spc{L}$ 
is called a 
\index{boundary strata}\emph{boundary strata} if the function $\dist{A}{}{}$
is semiconcave.

The meaning of the term boundary strata,
will become clear once we define boundary of finite dimensional $\CBB{}{}$ spaces.


\begin{thm}{Doubling theorem}\label{thm:doubling}
Let $\spc{L}\in \CBB{}{\kappa}$ 
and $A$ be a boundary strata in $\spc{L}$.
Then the doubling of $\spc{L}$ in $A$ 
is a $\CBB{}{\kappa}$ space.
\end{thm}

In the proof we will use the following statements.


\begin{thm}{Proposition}\label{prop:A-extremal}
Let $\spc{L}\in\CBB{}{}$ and $A\subset \spc{L}$ 
be a boundary strata.
Assume $\alpha(t)$ is a radial curve in $\spc{L}$ 
and $\alpha(t_0)\in A$.
Then $\alpha(t)\in A$ for any $t\ge t_0$. 
\end{thm}


\begin{thm}{Lemma}\label{lem:ultra-doubling}
Let $\spc{V}_n$ be a sequence of metric spaces 
and $A_n\subset \spc{V}_n$ be a closed subsets
and $a_n\in A_n$ for each $n$.
Denote by $(\spc{W}_n)$ the doubling of $\spc{V}_n$ in $A_n$;

Then 
\begin{subthm}{}
$\spc{W}_\o$ is isometric to the doubling of $\spc{V}_\o$ in $A_\o$.
\end{subthm}



\begin{subthm}{}
If $\spc{V}_n\in\CBB{}\kappa$ and $A_n$ is a boundary strata of $\spc{V}_n$
then 
$A_\o$ is a boundary strata of $\spc{L}_\o$.
\end{subthm}

\end{thm}


\begin{thm}{Splitting Lemma}\label{lem:split}
Let $\spc{L}\in\CBB{}{}$ 
and $A\subset \spc{L}$ 
be a closed subset such that
$\dist{A}{}{}$ is semiconcave.
Assume for some $x,y\in\spc{L}\backslash A$ 
and $a\in A$ be points such that
$\dist{a}{x}{}+\dist{a}{y}{}$ takes the minimal value for $a\in A$.

Then 

\begin{subthm}{lem:split:split}
$\o\cdot\spc{L}_a$ splits with factor $\o\cdot A$ and an other factor isometric to $\RR_{\ge0}$.
Moreover $\o\cdot A\times\RR$ splits in the directions of of $[ax]$ and $[ay]$.
That is, assume $X$ and $Y$ are the $\o$-limits of $[ax]$ and $[ay]$ in $\o\cdot\spc{L}_a$.
Let us identify $\o\cdot\spc{L}_a$ 
with the subset $\o\cdot A\times \RR_{\ge0}$ in $\o\cdot A\times \RR$.
Then  $\o\cdot A\times \RR$ admits two splitting with the factor isometric to $\RR$ containing $X$ and $Y$ correspondingly.
\end{subthm}

\begin{subthm}{lem:split:angle}
For any $z\in \spc{L}$ there is a sequence of points $b_n\in A$ 
such that 
$b_n\to a$ as $n\to\o$ and
\[\lim_{n\to\o}\left[\angk\kappa a{z}{b_n}+\angk\kappa ay{b_n}\right]\le \pi-\angk\kappa axz.\]
 
\end{subthm}

\end{thm}



\parit{Proof.}
Consider the function
\[f(z)=\dist{x'}{}{}+\dist{y}{}{}.\]
Note that $f$ is semiconvex in the neighborhood of $a$.
Therefore $\nabla_af$ is well defined.
\qeds

\parit{Proof of Doubling theorem (\ref{thm:doubling}).}
Denote by $\spc{W}$ the doubling of $\spc{L}$ in $A$.

\parit{Reduction to geodesic case.}
Let us pass to the $\o$-powers $A^\o\subset \spc{L}^\o\subset\spc{W}^\o$.
By Corollary \ref{cor:ulara-geod}, $\spc{W}^\o$ is geodesic.
By Lemma~\ref{lem:ultra-doubling},
 $\spc{W}^\o$ is doubling of $\spc{L}^\o$ in $A^\o$ and $A^\o$ is a boundary strata in $\spc{L}^\o$.
That is, assuming if the theorem holds if the doubling space is geodesic, we have that $\spc{W}^\o\in\CBB{}{\kappa}$ and by Proposition~\ref{prp:A^omega}, we get $\spc{W}\in\CBB{}{\kappa}$.


\parit{Geodesic case.} From now we assume that $\spc{W}$ is geodesic.

Let us show the adjacent angle comparison \ref{2-sum} 
holds in $\spc{W}$.
That is, we need to show that for any geodesic $[x y]$ and $z\in \l]x y\r[$, $z\not=p$ we have
\[\angk\kappa z p x
+\angk\kappa z p y\le \pi.\]

\parit{Case 1.}
Assume $p,x,y\in \spc{L}$. 

If $z\notin \spc{L}$ then $z'\in \spc{L}$
and 
$\dist{p}{z}{\spc{W}}
\ge\dist{p}{z'}{\spc{W}}
=\dist{p}{z'}{\spc{L}}.$
By Alexandrov's lemma \ref{lem:alex},
\[\angk\kappa{z'}p x+\angk\kappa{z'}py\ge\angk\kappa z p x+\angk\kappa z p y.\]

Note that  $[xy]\cap\spc{L}$ 
together with the reflection of $[xy]\backslash\spc{L}$
forms another geodesic from $x$ to $y$ in $\spc{L}$.

Therefore we can assume that $[xy]\subset \spc{L}$. 
In this case adjacent angle comparison  
\[\angk\kappa z p x
+\angk\kappa z p y\le \pi.\]
follows from the comparison in $\spc{L}$. 

The same way we can do the case $p,x,y\in \spc{L}'$.

\parit{Case 2.} 
The points $x$ and $y$ lie on opposite sides of $A$ and $z\in A$.

Without loss of generality we can assume that $p,x\in\spc{L}$
and $y\not\in \spc{L}$ so $y'\in\spc{L}$.

Let $b_n\in A$ be the sequence of points 
provided by Lemma~\ref{lem:split} for the quadruple $p, x, y', z\in  \spc{L}$.

???

Let $\gamma$ be a geodesic in $\oBall(p,\varpi\kappa)_{\spc{W}}$. 

By function comparison (\ref{comp-kappa}),
it is sufficient to show that
\[\dist{p}{\gamma(t)}{\spc{W}}
\le 
\side\kappa\{\phi;\dist{p}{\gamma(0)}{\spc{W}},t\}
\eqlbl{eq:doubling-comp}\]
for fixed $\phi\in [0,\pi]$ and all $t$ sufficiently close to $0$.

If $p$ and $\gamma(0)$ lie on one side from $A$ in $\spc{W}$
then 
he inequality follows from the point-on-side comparison (\ref{point-on-side})
in $\spc{L}$.
The same holds if $p\in A$.

It remains to consider two cases:

\parbf{Case 1.} Assume $p$ and $\gamma(0)$ lie on the opposite sides from $A$ in $\spc{W}$.
Set $q=\gamma(0)$, consider a geodesic $[pq]_{\spc{W}}$.
Denote by $a$ the intersection point of $[pq]_{\spc{W}}$ and $A$.

We may assume that $[pa]\in \spc{L}'$ and $[qa]\in \spc{L}$.
It follows since any geodesic $[vw]$ which lies completely in $\spc{L}$  
and with ends at $A$ 
can be reflected by the isometric involution in $\spc{L}'$
and the other way around.
 

Without loss of generality we may assume that $p\notin\spc{L}$
and therefore $p',q\in \spc{L}$. 
Set $\phi=\mangle(\dir qa,\gamma^+(0))$.
Fix $x=\gamma(t)$ for $t\approx0$.
In order to prove \ref{eq:doubling-comp},
we need to show that 
\begin{align*}
\dist{p}{x}{\spc{W}}
&=
\inf\set{\dist{p'}{z}{\spc{L}}+\dist{x}{z}{\spc{L}}}{z\in A}\le
\\
&\le \side\kappa\{\phi;\dist{p}{q}{\spc{W}},t\}.
\end{align*}


Note that $q'\in \spc{L}$ 
and $a,p,q' z'$satisfy the assumptions in Lemma \ref{lem:split}.
Denote by $b_n$ the sequence of points provided by \ref{lem:split:angle}.

Set 
\[
\alpha=\lim_{n\to\o}\angk\kappa a{p'}{b_n}.
\]
Let $[\tilde x\tilde x'\tilde y]$ 
be a triangle in the model plane $\Lob{}\kappa$
such that 
\begin{align*}
\dist{\tilde x}{\tilde x'}{}&=\dist{x}{x'}{},
\\
\dist{\tilde x}{\tilde y}{}&=\dist{x}{a}{}+\dist{y}{a}{}
\\
\mangle\hinge x{x'}y&=\alpha
\end{align*}
Choose $\tilde a\in[\tilde x\tilde y]$ such that $\dist{\tilde a}{\tilde y}{}=\dist{a}{y}{}$.
Assumeing $\dist{x}{x'}{}$ is small, 
we can choose $\tilde b\in [\tilde x'\tilde y]$
such that $\mangle \tilde y\tilde a\tilde b=\phi$.
Set $r=\dist{\tilde a}{\tilde b}{}$.

%???PIC

For each $n$ denote by $\beta_n$ the $\kappa$-radial curve with respect to $a$ which starts at $b_n$.
Set $c_n=\beta_n(r)$;
it is defined for all large $n$ and
by Proposition~\ref{prop:A-extremal}, $c_n\in A$.
By Radial monotonicity~\ref{rad-mon},
\begin{align*}
\dist{x'}{y}{\spc{W}}
&\le\lim_{n\to\o}\dist{x'}{c_n}{\spc{L}}+\dist{y}{c_n}{\spc{L}}\le
\\
&\le \dist{\tilde x'}{\tilde y}{\Lob2\kappa}
\end{align*}
Hence \ref{eq:doubling-comp} follows.
 

\parbf{Case 2.} Assume $\gamma(0)\in A$. 
Fix $t>0$ and set $a=\gamma(0)$, $x=\gamma(t)$, $y=\gamma(-t)$ and $p$.

Without loss of generality we can assume that $x$ and $p$ lie on one side from $A$.
In this case $a,p,x\in\spc{L}$; set  $\phi=\mangle{\hinge apx}_{\spc{L}}$.

In order to prove \ref{eq:doubling-comp},
it is sufficient to show that 
\begin{align*}
\dist{p}{y}{\spc{W}}
&=\inf\set{\dist{p}{z}{\spc{L}}+\dist{y'}{z}{\spc{L}}}{z\in A}\le
\\
&\le \side\kappa\{\phi;\dist{p}{a}{\spc{L}},-t\}
\end{align*}

Denote by $b_n\in A$ the sequence of points provided by lemma \ref{lem:split}
for the points $a,p,x,y'$.
Set $\alpha=\lim_{n\to \o}\angk\kappa a{y'}{b_n}$.
Consider the


\qeds









\section{Exercises}

\begin{thm}{Exercise}
 Assume $\spc{X}\in\cCat{}{0}$ and $A\subset \spc{X}$ is a closed subset.
Assume that  $\Doubling{A}{\spc{X}}\in\cCat{}{0}$. 
Show that $A$ is a totally convex set of $\spc{X}$.
\end{thm}

\begin{thm}{Exercise}
Assume $\spc{X}\in\CBB{m}{0}$ and $A\subset \spc{X}$ is a closed subset.
Assume that  $\Doubling{A}{\spc{X}}\in\CBB{m}{0}$. 
Show that $A$ is formed by union of primitive extremal subsets of dimension $m-1$.
\end{thm}

\begin{thm}{Exercise}
Let $\spc{U}\in\cCat30$ and 
$\proj\:\~U\to U$ be a covering map which is branching along a simple curve $\gamma$.
Define the length-metric on $\~U$ by setting 
\[\length\alpha=\length(\proj\circ\alpha).\]
Prove that $\~U\in\cCat30$ if and only if $\gamma$ is a line.
\end{thm}

\begin{thm}{Exercise}
Let $\spc{X}$ be a double cover of $\EE^3$ which is branching along two distinct lines $\ell^1$ and $\ell^2$.
Show that  $\spc{X}\in\cCat{}{0}$ if and only if $\ell^1$ intesect $\ell^2$ at right angle.
\end{thm}

\begin{thm}{Exercise}

\begin{subthm}{}
Prove the converse to Doubling theorem~\ref{thm:doubling}.
That is, 
show that the doubling of $\spc{V}$ 
in a closed set $A\subset \spc{V}$ 
is a $\CBB{}{\kappa}$ space 
then $\spc{V}\in\CBB{}{\kappa}$ and $A$ is a boundary strata in $\spc{V}$.
\end{subthm}


\begin{subthm}{}
Show that any $\Cat{}{0}$ space is isometric to a convex subset in a geometrically complete $\cCat{}{0}$ space.
\end{subthm}

\begin{subthm}{}
Construct a compat  $\cCat{}{0}$ space 
which is not isometric to any convex subset in a geodesically complete locally compact $\cCat{}{0}$ space.
\end{subthm}


\end{thm}

\begin{thm}{Exercise}
Construct a $\spc{L}\in \CBB{2}{-1}$, 
such that $\curv_p \spc{L}\ge 1$ for all $p$ in an open everywhere dense set of $\spc{L}$, but $\spc{L}\notin \CBB{2}{1}$.
\end{thm}

\begin{thm}{Exercise}
Construct a space $\spc{L}\in\CBB{}{}$
which contains an everywhere dense G-delta set $A$
such that 
$A\cap\l]xy\r[=\emptyset$
for any geodesic $[xy]$ in $\spc{L}$. 

Compare with Plaut's theorem (\ref{thm:almost.geod}).
\end{thm}

\begin{thm}{Exercise}\label{ex:no-convex-nbhd-CBB}
Construct a space $\spc{L}\in\CBB{}{0}$
with a point $p\in \spc{L}$ which does not admit arbitrary small closed convex neighborhood. 
\end{thm}


\begin{thm}{Exercise}\label{ex:nan-li}
Let $\spc{L}\in\CBB{m}{\kappa}$, $\partial \spc{L}\ne \emptyset$
and $\iota\: \partial \spc{L}\to \partial \spc{L}$ is length preserving involution.
Show that $\spc{L}/\iota\in \CBB{m}{\kappa}$.
\end{thm}

\begin{thm}{Exercise}\label{ex:fixed-point}
Let $m\ge 2$,
$\spc{L}\in\CBB{1}{m}$ and the group $G$ acts on  $\spc{L}$ by isometries.
Assume $\diam(\spc{L}/G)>\tfrac\pi2$.
Show that the action of $G$ has a fixed point on $\spc{L}$.
\end{thm}









