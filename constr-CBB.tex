%%!TEX root = the-constr-CBB.tex
\chapter{Constructions}


\section{Busemann functions}\label{sec:busemann}

Let $\spc{X}$ be a metric space 
and $\gamma:[0,\infty)\to \spc{X}$ be a \emph{ray} (i.e. a distance preserving map of $[0,\infty)$ in $\spc{X}$).
The function 
\[\bus_\gamma(x)=\lim_{t\to\infty}\dist{\gamma(t)}{x}{}- t\eqlbl{eq:def:busemann*}\]
is called the  \emph{Busemann function}\index{Busemann function} of $\gamma$. 

If $\spc{X}=\Cone\Sigma$, 
then for each element $\xi\in\Sigma$,
$\gamma(t)=t\cdot\xi$ forms a ray.
In this case, 
$\bus_\gamma(x)=-\<x,\xi\>$,
where $\<{*},{*}\>$ denotes scalar product (\ref{def:scalar-product}). 


\begin{thm}{Theorem}\label{thm:busemann}
Suppose $\spc{X}$ is a metric space and $\gamma\:[0,\infty)\to \spc{X}$ is a ray. 
Then the Buseman function $\bus_\gamma\:\spc{X}\to \RR$ is defined
and $1$-Lipschitz.

Moreover:
\begin{subthm}{}
 If  $\spc{X}\in \Cat{}{\kappa}$ (respectively $\spc{X}\in \CBB{}{\kappa}$) for some $\kappa<0$, then the function  $f=\exp(\sqrt{-\kappa}\cdot\bus_\gamma)$ satisfies
\[f''+\kappa\cdot f\ge 0\ \ \ (\t{respectively}\ \ f''+\kappa\cdot f\le 0).\eqlbl{eq:sec:busemann*}\]
\end{subthm}

\begin{subthm}{} If $\spc{X}\in \Cat{}{0}$ (respectively $\spc{X}\in \CBB{}{0}$), then $\bus_\gamma$ is convex (respectively concave).
\end{subthm}
\end{thm}

\parit{Proof.}
As  follows from the triangle inequality, the function $t\mapsto\dist{\gamma(t)}{x}{}- t$ is decreasing in $t$.  Clearly $\dist{\gamma(t)}{x}{}- t\ge-\dist{\gamma(0)}{x}{}$.
Thus the limit in \ref{eq:def:busemann*} is defined.

%!!! I inserted a proof.
By the definition of Busemann function,
\begin{eqnarray*}
\exp \sqrt{-\kappa}\bus_\gamma &= &\exp \lim_{t\to \infty} \sqrt{-\kappa}(d_{\gamma (t)} - t) \\
&= & \lim_{t\to \infty} \left(\exp \sqrt{-\kappa}(d_{\gamma (t)} -t) + \exp
\sqrt{-\kappa}(-d_{\gamma (t)}-t)\right)\\
&= & \lim_{t\to \infty} 2 \cosh \sqrt{-\kappa}d_{\gamma (t)} \exp\sqrt{-\kappa}(-t).
\end{eqnarray*}

By the function comparison definitions of $\Cat{}{\kappa}$ (\ref{function-comp}) or $\CBB{}{\kappa}$ (\ref{comp-kappa}),  for any $p\in \spc{U}$ the function $f=\cosh \sqrt{-\kappa}\circ\dist{p}{}{}$ satisfies $f''+\kappa \cdot f\ge 0$ (respectively  $f''+\kappa \cdot f\le 0$). The result follows.
\qeds

\begin{thm}{Corollary}
Assume $\spc{K}$ be a cone.
Given $w\in  \spc{K}$ consider 
the function $f_w\:\spc{K}\to\RR$
\[f_w(x)=\<w,x\>.\]
Then
\begin{subthm}{}
If $\spc{K}\in\CBB{}{0}$ then $f_w$ is convex.
\end{subthm}
\begin{subthm}{}
If $\spc{K}\in\cCat{}{0}$ then $f_w$ is concave.
\end{subthm}
\end{thm}

\parit{Proof.}
In both cases, the statement are trivial if $w=0$.

If $w\ne 0$, consider the ray $\gamma(t)=\tfrac1{|w|}\cdot t$

\section{Sumbetry}\label{sec:quotient-CBB}

Recall that map $\sigma\:\spc{L}\to\spc{M}$ between the metric spaces $\spc{L}$ and $\spc{M}$
is called 
\emph{submetry}\index{submetry} if 
\[\sigma(\oBall[p,R]_\spc{L})=\oBall[\sigma(p),R]_{\spc{M}}\]
for any $p\in \spc{L}$ and $R\ge 0$.

Equivalently, a map $\sigma\:\spc{L}\to\spc{M}$ is called sumbetry if it is 1-Lipshitz and 1-co-Lipschitz at the same time.

A big sourse of examples of sumbetries provided by isometric group actions. 

\begin{thm}{Proposition}
Let $\spc{X}$ be a metric space.
Assume that group $G$ acts on $\spc{X}$ by isometries 
and each $G$ orbit forms is proper.
Then the projection map $\spc{X}\to \spc{X}/G$ is a sumbetry.
\end{thm}

\begin{thm}{Theorem}
Let $\spc{L}\in\CBB{}{\kappa}$ and $\spc{M}$ is a metric space.
Assume there is a sumbetry $\sigma\:\spc{L}\to\spc{M}$.
Then $\spc{M}\in \CBB{}{\kappa}$.
\end{thm}







\section{Quotient spaces}\label{sec:quotient-CBB}

Let $\spc{L}$ be a metric space.
Assume that the group $G$ acts on $\spc{L}$ by isometries.
The set $\spc{L}/G$ 
of $G$-orbits in $\spc{L}$
comes with the natural metric ??? 


\begin{thm}{Theorem}\label{thm:CBB/G}
Let $\spc{L}\in\CBB{}{\kappa}$ and the group $G$ acts on $\spc{L}$ by isometries 
and it has closed orbits.
Then $\spc{L}/G\in\CBB{}{\kappa}$. 
\end{thm}

\parit{Proof.}
Applying Proposition~\ref{prop:length-X/G},
it is sufficient to check that (1+3)-point comparison holds in $\spc{L}/G\in\CBB{}{\kappa}$.

Denote by $\proj\:\spc{L}\to \spc{L}/G$ the projection,
$\proj(x)=G\cdot x$

Choose arbitrary point $\hat p$, $\hat x^1$, $\hat x^2$ and $\hat x^3$ in the quotient space $\spc{L}/G$.
Choose points $p$, $x^1$, $x^2$ and $x^3$ in $\spc{L}$ 
so that $\proj(p)=\hat p$ and $\proj(x^i)=\hat x^i$ for each $i$.

Note that 
\begin{align*}
\dist{p}{x^i}{\spc{X}}&\ge\dist{\hat p}{\hat x^i}{\spc{X}/G}
&
\dist{x^i}{x^j}{\spc{X}}&\ge\dist{\hat x^i}{\hat x^j}{\spc{X}/G}
\end{align*}
for all $i$ and $j$.

Given $\delta>0$,
the points $x^i$ can be chosen so that 
\[\dist{p}{x^i}{\spc{X}}<
\dist{\hat p}{\hat x^i}{\spc{X}/G}+\delta.\]
for each $i$.%
\footnote{If $\spc{L}$ is proper, then we can assume 
\[\dist{p}{x^i}{\spc{X}}=
\dist{\hat p}{\hat x^i}{\spc{X}/G}.\]}

Given $\eps>0$ we can choose the value $\delta>0$ above so that 
\[\angk\kappa{p}{x^i}{x^j}_{\spc{L}}+\eps > \angk\kappa{\hat p}{\hat x^i}{\hat x^j}_{\spc{L}/G}\]
for all $i$ and $j$, assuming the left hand side is defined.
From (1+3)-point comparison in $\spc{L}$
\[\angk\kappa{p}{x^1}{x^2}+\angk\kappa{p}{x^2}{x^3}+\angk\kappa{p}{x^3}{x^1}\le 2\cdot\pi\]
we get
\[\angk\kappa{\hat p}{\hat x^1}{\hat x^2}+\angk\kappa{\hat p}{\hat x^2}{\hat x^3}+\angk\kappa{\hat p}{\hat x^3}{\hat x^1}\le 2\cdot\pi-3\cdot\eps.\]
Since $\eps>0$ is arbitrary,
the later implies (1+3)-point comparison in $\spc{L}/G$.
\qeds


\section{Doubling theorem}\label{sec:doubling}

Let $\spc{X}$ be a metric space 
and $A\subset \spc{X}$ be a closed subset.
A metric space $\spc{W}$ glued from two copies of $\spc{X}$ along $A$ is called \emph{doubling of $\spc{X}$ with respect to $A$}\index{doubling} (briefly, $\spc{W}=\Doubling{A}{\spc{X}}$).

More formally,
let us think of disjoint union of two copies of $\spc{X}$
as of the product $\spc{X}\times\{-1,1\}$.
Consider on $\spc{X}\times\{-1,1\}$ the equivalence realation 
$\sim$ defined as $(a,-1)\sim (a,+1)$ for any $a\in A$.
Now set $\spc{W}=(\spc{X}\times\{-1,1\})_\sim$ and
define metric on $\spc{W}$ the following way:
\[
\dist{(x,-1)}{(y,-1)}{\spc{W}}
=\dist{(x,1)}{(y,1)}{\spc{W}}
=\dist{x}{y}{\spc{X}},
\]
\[
\dist{(x,-1)}{(y,1)}{\spc{W}}
=\dist{(x,1)}{(y,-1)}{\spc{W}}
=\inf_{z\in A}\{\dist{x}{z}{\spc{X}}+\dist{z}{y}{\spc{X}}\}.
\]
for any $x,y\in \spc{X}$.

Let us think of $\spc{X}$ as the subset of $\spc{W}$ formed by 
pairs $(x,1)\in \spc{W}$, $x\in \spc{X}$.
Given $w=(x,\pm1)\in \spc{W}$ set $w'=(x,\mp1)$, so $w''=w$ for any $w\in \spc{W}$. 

Note that 
\begin{itemize}
\item If $\spc{X}$ is a complete length space then so is $\Doubling{A}{\spc{X}}$.
\item If $\spc{X}$ is proper then so is $\Doubling{A}{\spc{X}}$.
Moreover, in this case, for any $x,y\in\spc{X}$ there is $a\in A$ such that 
\[\dist{x}{a}{}+\dist{a}{y'}{}=\dist{x}{y'}{}.\]
\end{itemize}
 
In particular, $(p,1)$ can be connected to $(q,-1)$ by a geodesic 
can be connected by a geodesic $[(p,1)(q,-1)]$ which passes through $(a,\pm1)$ and before $(a,\pm1)$ it lies in $\spc{X}\times 1$ and after in $\spc{X}\times(-1)$.

There is a natural projection map $\Doubling{A}{\spc{X}}\to \spc{X}$, defined by $(x,i)\mapsto x$.
Also there is a reflection map $\Doubling{A}{\spc{X}}\to \Doubling{A}{\spc{X}}$, defined by  $(x,i)\mapsto (x,-i)$ which has $A$ as a fixed point set.
There are two distance preserving maps $\spc{X}\to \Doubling{A}{\spc{X}}$, defined by $x\mapsto(x,i)$, $i=\pm1$. 


\begin{thm}{Theorem}\label{thm:doubling}
Let $\spc{L}\in \CBB{m}\kappa$, 
$\partial \spc{L}\not=\emptyset$. 
Then $\Doubling{\partial \spc{L}}{\spc{L}}\in \CBB{m}\kappa$ and if $\proj(p)\in\partial \spc{L}$ then $\T_p\iso\Doubling{\partial\T_{\proj(p)}}{\T_{\proj(p)}}$.

Moreover, the same statement holds for $\Doubling{E}{\spc{L}}$, where $E$ is a union of primitive $(m-1)$-dimensional extremal subset.
\end{thm}

\parit{Proof.}
From \ref{thm:dist-to-bry}, 
we have that if for some point $x\in \l] p q \r[$ lies on the boundary of $\spc{L}$ then $[pq]\subset \partial \spc{L}$.

Set $\spc{W}=\Doubling{\partial \spc{L}}{\spc{L}}$, $N=\proj^{-1}(\partial \spc{L})$ and $\spc{L}_+=e_+(\spc{L})$ $\spc{L}_-=e_-(\spc{L})$.
From above, it follows that if $p,q\in \spc{L}_+$ then any geodesic $[pq]$ lies completely in $\spc{L}_+$, otherwise one could reflect the part of $[pq]$ which lie in $\spc{L}_-$ and obtain a geodesic in $\spc{L}_+$ which contradicts above ???

Let $p\in N$, 
$q_+\in \spc{L}_+$ 
and $q_-\in \spc{L}_-$.
Both subsets $\spc{L}_\pm\subset W$ with inherited metric form an $\CBB{m}{\kappa}$-space.
Since $[pq_\pm]\in \spc{L}_\pm$, $\dir p{q_\pm}\in\Sigma_p\spc{L}_\pm$.
Note that $\Sigma_p\spc{L}_\pm$ are isometric and they admit natuarl map.



It is sufficient to show that comparison ??? is satisfied in a neigborhood of any $p\in N$.
Note that if $p\in N$, then for any $q\in W$ the geodesic $[pq]$ lies completely in $\spc{L}_+$ or $\spc{L}_-$ (otherwise ???).
Thus the set of geodesic directions $\Sigma'_pW$ can be identified with???


We proceed by induction on $m$, the base $m=1$ is trivial.
By induction hypothesis, for any $p\in\partial \spc{L}$ we have 
$\Doubling{\partial\Sigma_p}{\Sigma_p}\in\CBB{m-1}1$.

\begin{thm}{Claim}
Assume $z\in N$ be an interior point of geodesic $[pq]$ in $W$.
Then each of subsegments $[pz]$ and $[qz]$ of $[pq]$ lie in $\spc{L}_+$ or $\spc{L}_-$.
Moreover $\dir z q$ and $\dir z p$
\end{thm}

Assume $???$ lies on a geodesic $[xy]$ in $\Doubling{\partial \spc{L}}{\spc{L}}$.
Then geodesics $[px]$ and $[py]$ each lie in one of ???.





Set $W=\Doubling{\partial \spc{L}}{\spc{L}}$.
It is sufficient to show that any two points $p,q\in W$ can be connected by a geodesic $[p q]$ ??? such that for any $s\in W$ the function
$y(t)=\md\kappa\dist[{{}}]{s}{\geod_{[pq]}}{}$ satisfies differential inequality $y''+\kappa\cdot  y\le 1$.
\qeds

\section{Gluing theorem}

In the two dimensional case, the gluing theorem was proved first by Alexandrov;
see \cite[\S 11]{pogorelov:book} for the proof. 
Later Perelman proved
the Doubling theorem \ref{thm:doubling}
for multidimensional Alexandrov spaces; 
this is a special
case of the theorem formulated below. 
The original Alexandrov’s theorem had
a lot of applications to the bending of convex surfaces with boundary, which are
currently impossible to generalize to the multidimensional case, 
because they are
supported by the Theorem about convex embeddings 
\cite[\S 6--7]{pogorelov:book}%??? \S???.
Formally speaking, the following theorem gives new examples of Alexandrov spaces,
but we have very limited souse of examples 
of pairs of Alexandrov spaces with
isometric boundaries.

\begin{thm}{Glueing theorem}\label{thm:gluing-cbb}
Let $\spc{L}_1,\spc{L}_2\in\CBB m\kappa$ be two spaces with nonempty boundaries.
Assume there is an isometry $\iota:\partial \spc{L}_1\to \partial \spc{L}_2$ and 
\[\spc{L}=\spc{L}_1\sqcup_\iota \spc{L}_2\]
is the space glued from $\spc{L}_1$ and $\spc{L}_2$ along $\iota$.
Then $\spc{L}\in\CBB m\kappa$.
\end{thm}

The proof is build the following technical statement.

\begin{thm}{Proposition}
Let $\spc{L}\in\CBB m\kappa$, 
$\partial \spc{L}\not=\emptyset$.

Assume $f\:\partial \spc{L}\to\RR$ satisfies\footnote{In case $\kappa=0$ this inequality can be rewritten 
as $(f^2)''\le 2$.}
\[(\md\kappa\circ f)''\le 1-\kappa\cdot (\md\kappa\circ f)\]
then the envelope function $g\:\spc{L}\to\RR$ defined by 
\[g(x)=\inf_{y\in\partial \spc{L}} \{\dist{x}{y}{}+f(y)\}\] 
satisfies the same inequality
\[(\md\kappa\circ g)''\le 1-\kappa\cdot (\md\kappa\circ g)\]
in $\spc{L}$.
\end{thm}

By Function comparison, the inequality in the proposition holds for any distance function in $\CBB{}{\kappa}$ space; see Theorem~\ref{thm:conc}.

In order to prove the proposition we will need the following Lemma.

\begin{thm}{Lemma}\label{lem:dir-bry}
Let $\spc{L}\in\CBB{m}{}$ and $p\in\partial\spc{L}$.
We will consider $\partial\spc{L}$ with the length-metric which is denoted as $\dist{}{}{\partial L}$.

Then for any direction $\eta\in\partial\Sigma_p$
there exists a geodesic $[pq]_{\partial \spc{L}}$ in $\partial \spc{L}$ which runs from $p$ in a direction arbitrarily close to $\eta$.
\end{thm}
 


\parit{Proof.} 
Recall that the boundary $\partial \spc{L}$ 
is an extremal subset of $\spc{L}$;
see ???.
By Lieberman's lemma (???)
any geodesic $[pq]_{\partial\spc{L}}$ in $\spc{L}$
is a quasigeodesic in the ambient space $\spc{L}$.
By ???, any such geodesic  $[pq]_{\partial\spc{L}}$ in $\spc{L}$
has defined direction $\dir{p}{q}_{\spc{L}}\in \Sigma_p$.

Consider the sequence $q_n=\gexp_p(\tfrac1n\cdot \eta)$.
According to ???, $q_n\in \partial \spc{L}$ and according to ???
$\dir{p}{q_n}\to \eta$ as $n\to\infty$.

Pass to a subsequence of $(q_n)$ so that the sequence of directions $\xi_n=\dir{p}{q_n}_{\partial\spc{L}}$ converges;
denote by $\xi\in\Sigma_p$ its limit.

It is sufficient to show that $\xi=\eta$.
Assume contrary, so $\eps=\mangle(\xi,\eta)>0$.

Note that there is a point $z \in\spc{L}$ 
such that $\mangle(\xi,\dir{p}{z})<\tfrac\eps3$.

Since $[pq_n]_{\partial\spc{L}}$ is a quasigeodesic in $\spc{L}$,
we get $\dist{z}{q_n}{}\le ???$
On the other hand 

\qeds





\parit{Proof.}
We have to show that for any $p\in \spc{L}$, the function $f=\dist{p}{}{}$ satisfies $f''\le\frac{\cs\kappa f}{\sn\kappa f}(1-(f')^2)$.
Let us define a seqence of functions $f_n\:\spc{L}\to \RR$ the following way:???
\qeds

\parit{Proof of theorem \ref{thm:gluing-cbb}.}
Fix point $p\in \spc{L}_1$.
Denote by $f_1$ the restriction $\dist{p}{}{}|\partial \spc{L}_1$.



\qeds


\begin{thm}{Lemma} 
???

Given $\eps>0$, for any direction $\xi\in\partial \Sigma_p$ there is a geodesic $\gamma$ in $\partial \spc{L}$ such that
$\gamma^+\approx\xi$. 
\end{thm}





\parit{Proof.}
Let $N = \spc{L}_1 \cap \spc{L}_2 \subset ??? $.
Definition 2.3. The m-predistance |pq|m between points p and q in X is the
minimal length of broken geodesics with vertices p = p0 , p 1 , . . . , pk+1 = q, where
k ≤ m, plpl±1 is a shortest path that lies completely in one of Mi for every
l ∈ {1, 2, . . ., k}, and pl lies in N . A broken geodesic that realizes this minimum
is called an m-shortest path.
Remark 2.4. It is easy to see that |pq| m ≥ |pq|m+1 ≥ |pq|, limm→∞ |pq|m = |pq|,
|pq|m + |qr|l ≥ |pr|m+l
 if q ∈ X \ N ,
(2.1)
|pq|m + |qr|l ≥ |pr|m+l+1
 if q ∈ N .
For every interior vertex p = pl , l ∈ {1, 2, . . . , k}, of an m-shortest path, we can
define directions of exit and entrance ξi as directions in Σp(M i ) of shortest paths
in Mi.
By Theorem 1.2 the isometry is : ∂M1 → ∂M 2 = N gives an isometry isp :
∂Σp(M1) → ∂Σp(M2) = Σp(N ) and isp : ∂Cp(M 1 ) → ∂Cp(M 2 ) = Cp (N ). 
Set
Σp 
 (X) := Σp(M1) ∪isp (x)=x Σp (M2 ),

Cp (X) := C(Σp (X)) = Cp(M1) ∪isp (x)=x Cp(M2).
From the induction hypothesis, Σ
(X) will be an Alexandrov space with curvature ≥ 1, and therefore Cp (X) will be a cone with curvature ≥ 0.
Notation. 
If K1 and K2 are two compact metric spaces, we say that K1 ≤ K2
if there is a noncontracting map m : K1 → K2 . If (L1 , p1 ) and (L2 , p2 ) are two
locally compact metric spaces with base points, we say that (L1, p1) ≤ (L2 , p2)
if for any R > 0 there is a noncontracting map m : BR (p1 ) → BR(p2 ).

Proof. Let N = ∂M . The boundary is an extremal subset and therefore we
can use notation q ◦(= qp ◦ ) for the set of all directions of entrance in Σp(N ) of
shortest paths between p and q in the length metric of N .
Choose a sequence of points qn ∈ N such that qn → p and (qn , η) → 0
(where qn = (qn )p is the direction at p of the shortest path pq). Assume that
◦ ◦(ηq n) ≥ ε for all n. Pass to a subsequence such that limn→∞ (θ qn) → 0 for
some direction θ.
Find a point r ∈ M such that (r , θ) < ε/6. Let {rn } be points on the
shortest path pr such that |prn| = |pqn|. Since the shortest path from p to qn
in N is a quasigeodesic (see Theorem 1.1), we conclude by using [Perelman and
Petrunin 1994, 1.4(G2), 1.5] that |rn qn | < (ε/5)|pq n | for n sufficiently large,
hence that limn→∞ (qn , r ) < ε/3 for ε ≤ π/4. Therefore
lim
 (qn , θ) < ε/2.
n→∞
We obtain a contradiction because lim n→∞ qn = η and
 (η, θ) ≥ ε.

The rest of this section will be devoted to the proof of Theorem 2.1. Let N =
M1 ∩ M2 = ∂Mi ⊂ X.
Definition 2.3. The m-predistance |pq|m between points p and q in X is the
minimal length of broken geodesics with vertices p = p0 , p 1 , . . . , pk+1 = q, where
k ≤ m, plpl±1 is a shortest path that lies completely in one of Mi for every
l ∈ {1, 2, . . ., k}, and pl lies in N . A broken geodesic that realizes this minimum
is called an m-shortest path.
Remark 2.4. It is easy to see that |pq| m ≥ |pq|m+1 ≥ |pq|, limm→∞ |pq|m = |pq|,
|pq|m + |qr|l ≥ |pr|m+l
 if q ∈ X \ N ,
(2.1)
|pq|m + |qr|l ≥ |pr|m+l+1
 if q ∈ N .
For every interior vertex p = pl , l ∈ {1, 2, . . . , k}, of an m-shortest path, we can
define directions of exit and entrance ξi as directions in Σp(M i ) of shortest paths
in Mi.
By Theorem 1.2 the isometry is : ∂M1 → ∂M 2 = N gives an isometry isp :
∂Σp(M1) → ∂Σp(M2) = Σp(N ) and isp : ∂Cp(M 1 ) → ∂Cp(M 2 ) = Cp (N ). Set
Σp 
 (X) := Σp(M1) ∪isp (x)=x Σp (M2 ),
 
Cp (X) := C(Σp (X)) = Cp(M1) ∪isp (x)=x Cp(M2).
From the induction hypothesis, Σ
(X) will be an Alexandrov space with curva-
pture ≥ 1, and therefore Cp (X) will be a cone with curvature ≥ 0.
Notation. If K1 and K2 are two compact metric spaces, we say that K1 ≤ K2
if there is a noncontracting map m : K1 → K2 . If (L1 , p1 ) and (L2 , p2 ) are two
locally compact metric spaces with base points, we say that (L1, p1) ≤ (L2 , p2)
if for any R > 0 there is a noncontracting map m : BR (p1 ) → BR(p2 ).


(because a shortest path expδn (x)yn completely lies in one of the Mi and because
|yn exp δn (x∗ )| = o(δn)/δn ). Therefore |p exp δn (x∗ )| ≤ (1−ε) |x∗| for n sufficiently
∗large, By Lemma 1.5, a limit of shortest paths in N/δn between p and expδn (x )
(which is a quasigeodesic by the generalized Lieberman lemma, Theorem 1.1) is
∗a shortest path ox in Cp(Mi). Because limits preserve lengths of quasigeodesics
[Perelman and Petrunin 1994, 2.3(3)], we have
lim |p expδn (x∗)|N/δ = |x∗|.
n→∞
Hence for n sufficiently large we get
∗ ∗|p expδn (x )| ≤ (1 − ε) |p expδn (x )|N/δ .
Therefore we can find a segment sn rn on a shortest path p expδn (x∗) that com-
pletely lies in one of the Mi /δn , such that sn , rn ∈ N/δn and
|s n r n |Mi ≤ (1 − ε)(|prn|N − |psn |N )
 (2.2)
(where we use the same notation for points in N and N/δ).
We can easily pass to a subsequence such that limn→∞ |psn |N /|prn|N = c. for
some 0 ≤ c ≤ 1.
Now we consider two cases, c = 1 and c = 1.
Suppose c = 1, and consider limit (Mi /|prn|N , p) −→ GH
 Cp(Mi). Pass to a
subsequence such that sn → s and rn → r. The boundary N is an extremal
GH
subset; therefore, by Theorem 1.2, (N/|prn|N , p) −→ Cp (N ) as length-metric
spaces. Hence
|snrn|Mi
 |ps n | N
lim
 = |sr| ≥ |r| − |s| = |r|C(N) − |s|C(N) = 1 − lim
 ,
n→∞
 |prn |N
 n→∞ |prn | N
contradicting (2.2).
Suppose instead that c = 1. Pass to a subsequence such that there exists
a limit (Mi /|sn rn|Mi, s n ) −→ GH
 (Ms , s). (We remark that Ms need not be the
tangent cone.) Set Ns = ∂Ms . By Theorem 1.2 we have
(N/|snrn|M i , sn ) −→ GH
 (Ns , s).
Let fn : N/|snrn|M i → R be functions defined by
f n (x) = |px|N/|snrn |Mi − |psn|N/|snrn |Mi .
Pass to a subsequence such that there exists a limit f : Ns → R, f = limn→∞ fn .
It is easy to see that Ms can be represented as a product R × Ms such that
f(x) ≤ prR(x), where prR is the projection Ms → R. Indeed a sequence of
quasigeodesics that prolong shortest paths psn in N easily goes to a straight line
in Ms , so by the Toponogov splitting theorem we have such a representation.
Therefore Ns is split as well, Ns = R × Ns.
Let σn be a shortest path in N between p and sn , parametrized by distance
from sn, and let σ be a limit of {σn/|rnsn|Mi }. By the triangle inequality, for any

T > 0 we have |xp|N − |snp| ≤ |xσn(|sn rn |T )| − |s n rn |T . As a limit we obtain
that f(x) ≤ |xσ(T )| − T . For T → ∞ the right side goes to the Busemann
function of σ which coincides with prR .
Pass to a subsequence such that there is a limit as rn → r. We obtain
1 = |rs| ≥ prR (r) ≥ f(r) = lim (|prn | N − |psn|N )/|rns n |M i ,
n→∞
again contradicting (2.2). This concludes the proof of the lemma.

Lemma 2.6. The directions of exit and entrance (ξi ) of any m-shortest path at
every interior vertex p = pl , for l ∈ {1, 2, . . ., k ≤ m} (see Definition 2.3), are
opposite in Cp (X) (that is, |ξ1 ξ2 | = 2|ξ1 | = 2|ξ2|; see [Perelman and Petrunin
1994, 2.1]).
Proof. Let ξi ∈ Σp(Mi ) be directions of exit/entrance of the m-shortest path
at the interior vertex p. We first prove that |ξ1 ν|0 + |ξ2 ν|0 = π for any ν ∈
Σp(N ) ⊂ Σp 
(X). Here the left side is the sum of two 0-distances in the glued

space Σp (X), each of which, by Definition 2.3, is measured in one of the Σp (Mi ).
Assume we have proved the lemma for dim < n, and let dim Σ
 (X) = n. From
pthe first variation formula we obtain
f(ν) := |ξ1ν|0 + |νξ2|0 ≥ π
for any ν ∈ Σp(N ). Assume ν  ̄ is the minimum point in Σp (N ) of the last
function. Thus, ξ1ν  ̄ξ2 is a 1-shortest path. Let γ be a shortest path in Σp(N )
such that γ(0) = ν  ̄ with arbitrary initial data γ +(0) = η. Assume f( ̄
 ν ) > π.
By the induction assumption, |(ξ 1 )ν  ̄ η|0 + |η(ξ2)ν  ̄|0 = π. By the generalized
Lieberman lemma, Theorem 1.1, γ is a quasigeodesic as a curve in Σp(M 1 ) and
Σp(M2). By [Perelman and Petrunin 1994, 1.4(G1)], the condition f( ̄
 ν ) > π
implies (f ◦ γ)(x) < (f ◦ γ)(0) = f( ̄
 ν ) for sufficiently small x. This contradicts
the assumption that f has a minimum at ν  ̄.
Therefore f( ̄
 ν ) = π. Take any shortest path γ in Σp (N ) such that γ(0) = ν  ̄.
Then γ is a quasigeodesic for Σp (M1 ) and Σp(M2 ). Set
g(ν) := cos |ξ1ν|0 + cos |νξ2|0
for ν ∈ Σp(N ). By the preceding arguments, g( ̄
 ν) = g ◦ γ(0) = 0, (g ◦ γ) (0) = 0
and g ◦ γ ≤ 0. By [Perelman and Petrunin 1994, 1.3(L2)], (g ◦ γ) + g ◦ γ ≥ 0.
Therefore (g ◦ γ) ≥ 0 and so g ◦ γ ≡ 0; in particular for any ν, g(ν) = 0.
Therefore f ≡ π, that is, |ξ1ν|0 + |ξ2 ν|0 = π as claimed.
In order to prove that ξ1 and ξ2 are opposite, it is enough to show that
2|ξ1| = 2|ξ2 | = |ξ1ξ2 holds in Cp (X), or equivalently that |ξ1 ξ2 = π holds in
| |Σ
 p (X). If this is false, there is m such that |ξ1 ξ2 | m < π in Σp 
 (X). Let θ be the
closest vertex to ξ1 of the m-shortest path ξ1ξ2 . By the preceding discussion,
there is a 1-shortest path through θ of length π. Therefore we have two distinct

directions at θ which are opposite to (ξ1 )θ , a contradiction to the fact that Σp
is an Alexandrov space. This completes the proof of the lemma.

Corollary 2.7. Let ξi ∈ Σp(Mi) be directions of exit/entrance of an m-shortest
path at an interior vertex . For any η ∈ Σp(Mi ) there is a unique η∗ ∈ Σp(N )
such that
|ξ1 η|0 + |ηη∗ |0 + |η∗ξ2|0 = π
or
|ξ1η ∗|0 + |η ∗η| 0 + |ηξ2|0 = π.
Proof. Suppose η ∈ Σp(M1). Consider the 1-shortest path ηξ2 . Applying
Lemma 2.6 to Σ
 (X) we see that the directions at the vertex are opposite;
therefore this 1-shortest p path is a part of a 1-shortest path ξ 1 ξ2 .

Lemma 2.8. Let γ : [a, b] → X be a quasigeodesic in one of the int Mi or a
shortest path in the length metric of N . Then
ρk (|pγ(t)|m ) + kρk (|pγ(t)|m) ≤ 1
for any p ∈ X
For the definition of ρk see [Perelman and Petrunin 1994, 1.4(L2)].
Proof. We consider the case k = 0; we must show that (|pγ(t)|2 m ) ≤ 2.
This is true for m = 0 because

 |pq|Mi
 if p ∈ M i , q ∈ int Mi or q ∈ Mi, p ∈ int Mi,
|pq|0 = mini |pq| Mi if p, q ∈ N ,

∞
 otherwise.
(Recall that a shortest path in N is a quasigeodesic in both Mi by the generalized
Lieberman Lemma).
Suppose the claim is true for all l < m and false for m. Then the standard
idea shows that in this case there exists t0 ∈ (a, b) and ε > 0 such that for
|t − t0| < ε
|pγ(t)|2 m ≥ |pγ(t0)|2 m − A(t − t0 ) + (t − t0 )2 + ε(t − t0)2 ,
for some constant A.
Assume t0 = 0. Set q = γ(0) and let p = p0p1 . . . pk pk+1 = q be an m-shortest
path. Take a sequence tj → 0 such that the sequence ((γ(tj )pk )∗ (as in Corollary
2.5) goes to some direction ν ∈ Σpk (N ). Using Lemma 2.2 we can find a shortest
path γk in N which goes from pk in a direction arbitrarily close to ν.
In the following proof one might get lost in calculations and lose the main
∗idea. If we assume that all ((γ(tj )pk ) coincide with ν and there is a shortest
path (in the length-metric of N ) that goes in this direction, one can ignore the
residue terms below.

Set α = ((pk )q , γ+ (0)), β = (qpk , γk +(0)), βj = (γk + (0) (γ(tj )) pk ), θj =
+((γ(tj ))pk qpk ), and δ = (γk , ν), as in the figure below.

It is easy to see that
tj sin α
θj ≥
 + o(tj ).
|pk q|0
We can assume that qpk ∈ Σpk (N ); otherwise our m-shortest path lies completely
+in N . By the cosine rule applied to the triangle qpk (γ(tj ))pk γk (0), we have
sin α
β − βj ≥ 1 + o(δ) + o(tj )/tj θj ≥ tj
 + o(δ)
 + o(tj ).
|pkq|0
Hence
t2 sin2 α
cos(β − βj ) ≤ 1 −
 j + o(δ)t2 j + o(t2 j ).
2 |pq|20
From the induction assumption and Lemma 2.6 we have
2 2 2|p γ k(τ )|m−1 ≤ |p pk | m−1 + 2τ |p p k|m−1 cos β + τ .
Because γk is a quasigeodesic for both of the Mi, we obtain
|γ(tj )γk (τ )|20 ≤ |γ(tj ) pk |20 − 2 cos βj τ |γ(tj ) pk|0 + τ 2 ,
where these distances are measured in a fixed Mi .
Therefore, using (2.1) and the previous two inequalities, we have
2 2
|p γ(tj )|m ≤ min (|p γk (τ )|m−1 + |γ(tj )γk (τ )|0)τ
≤ min (|AB(τ )| + |B(τ )C|) 2
τ
= |AC|2 = |p pk | m−1 2 + |γ(tj ) pk |20 + 2 |p pk|m−1|γ(tj ) pk | 0 cos(β − βj ),

where A, B(τ ) and C are as shown in the following diagram in the plane:
Because γ is either a quasigeodesic in one of the Mi , or a shortest path in N
and therefore a quasigeodesic in both of the Mi (see Theorem 1.1), we conclude
that
|pk γ(tj )| 20 ≤ |pk q|20 + tj 2 − 2tj |pk q| 0 cos α
and so
t2 sin2 α
j 2|pk γ(tj )|0 ≤ |pkq|0 − tj cos α +
 + o(tj ).
2|pk q|0
Hence
|p γ(tj)|2 m ≤ |p pk|2 m−1 + |q pk |20 + tj 2 − 2tj |q pk |0 cos α
t2 sin2 α
+ 2|p pk |m−1 |pk q|0 − tj cos α +
 j + o(t2 j )
2|pkq|0
t2 sin2 α
j 2 2×
 1 −
 + tj o(δ) + o(tj )
2|pkq|20
2 2 2 2≤ (|ppk |m−1 +|pk q|0) −2tj (|ppk|m−1 +|pk q| 0 ) cos α +tj +tj o(δ)+o(tj )
2 2 2 2= |p q|m − 2tj|pq|m cos α + tj + tj o(δ) + o(tj ).
+ −This inequality for two sequences tj → 0 and tj → 0 contradicts our assump-
tion for sufficiently small δ.

We continue the proof of Theorem 2.1, showing that every m-shortest path is
a k-quasigeodesic. Indeed, using [Perelman and Petrunin 1994, 1.4(L2), 1.5],
we only need to verify that ρk(|γ(t)p|) ≤ 1 − kρk (|γ(t)p|). Now |γ(t)p| =
lim n→∞ |γ(t)p|n , and using Lemma 2.8 and [Perelman and Petrunin 1994, 1.3(4)]
we obtain the needed inequality for all t = tl (where γ(tl ) = pl).
Let σ be a shortest path between an arbitrary point x and γ(tl ), parametrized
by distance from γ(tl ). By Lemmas 2.5 and 2.6 we conclude that, for fixed ε,
|σ(T )γ(tl + T ε)| + |σ(T )γ(tl − T ε)| ≤ 2T + CT ε2 + o(T ).
Therefore
distx ◦ γ(tl + T ε) + distx ◦ γ(tl − T ε) ≤ 2distx ◦ γ(tl ) + CT ε2 + o(T ).
Therefore, for T → 0,
+ −(distx ◦ γ) (tl ) ≤ (distp ◦ γ) (tl ) + Cε.

Hence, for ε → 0, we obtain (distx ◦ γ)+ (tl) ≤ (distx ◦ γ)−(tl ). From this, using
[Perelman and Petrunin 1994, 1.3(2)], we obtain the needed inequality for any t.
Let γm be an m-shortest path between p, q ∈ X. Then γ = limm→∞ γm is
a shortest path between p and q. It is easy to see that γ is convex (as a limit
of convex curves) and parametrized by the arclength (because length(γm ) →
length(γ)); hence γ is a quasigeodesic. Therefore by [Perelman and Petrunin
1994, 1.6] we obtain that X is an Alexandrov space of curvature ≥ k. This
completes the proof of the Gluing Theorem.




\section{Exercises}

\begin{thm}{Exercise}
 Assume $\spc{X}\in\cCat{}{0}$ and $A\subset \spc{X}$ is a closed subset.
Assume that  $\Doubling{A}{\spc{X}}\in\cCat{}{0}$. 
Show that $A$ is a totally convex set of $\spc{X}$.
\end{thm}

\begin{thm}{Exercise}
Assume $\spc{X}\in\CBB{m}{0}$ and $A\subset \spc{X}$ is a closed subset.
Assume that  $\Doubling{A}{\spc{X}}\in\CBB{m}{0}$. 
Show that $A$ is formed by union of primitive extremal subsets of dimension $m-1$.
\end{thm}

\begin{thm}{Exercise}
Let $\spc{U}\in\cCat30$ and 
$\proj\:\~U\to U$ be a covering map which is branching along a simple curve $\gamma$.
Define the length-metric on $\~U$ by setting 
\[\length\alpha=\length(\proj\circ\alpha).\]
Prove that $\~U\in\cCat30$ if and only if $\gamma$ is a line.
\end{thm}

\begin{thm}{Exercise}
Let $\spc{X}$ be a double cover of $\EE^3$ which is branching along two distinct lines $\ell^1$ and $\ell^2$.
Show that  $\spc{X}\in\cCat{}{0}$ if and only if $\ell^1$ intesect $\ell^2$ at right angle.
\end{thm}

\begin{thm}{Exercise}

\begin{subthm}{}
Show that any $\Cat{}{0}$ space is isometric to a convex subset in a geometrically complete $\cCat{}{0}$ space.
\end{subthm}

\begin{subthm}{}
Construct a compat  $\cCat{}{0}$ space 
which is not isometric to any convex subset in a geodesically complete locally compact $\cCat{}{0}$ space.
\end{subthm}


\end{thm}

\begin{thm}{Exercise}
Construct a $\spc{L}\in \CBB{2}{-1}$, 
such that $\curv_p \spc{L}\ge 1$ for all $p$ in an open everywhere dense set of $\spc{L}$, but $\spc{L}\notin \CBB{2}{1}$.
\end{thm}

\begin{thm}{Exercise}
Construct a space $\spc{L}\in\CBB{}{}$
which contains an everywhere dense G-delta set $A$
such that 
$A\cap\l]xy\r[=\emptyset$
for any geodesic $[xy]$ in $\spc{L}$. 

Compare with Plaut's theorem (\ref{thm:almost.geod}).
\end{thm}

\begin{thm}{Exercise}\label{ex:no-convex-nbhd-CBB}
Construct a space $\spc{L}\in\CBB{}{0}$
with a point $p\in \spc{L}$ which does not admit arbitrary small closed convex neighborhood. 
\end{thm}


\begin{thm}{Exercise}\label{ex:nan-li}
Let $\spc{L}\in\CBB{m}{\kappa}$, $\partial \spc{L}\ne \emptyset$
and $\iota\: \partial \spc{L}\to \partial \spc{L}$ is length preserving involution.
Show that $\spc{L}/\iota\in \CBB{m}{\kappa}$.
\end{thm}

\begin{thm}{Exercise}\label{ex:fixed-point}
Let $m\ge 2$,
$\spc{L}\in\CBB{1}{m}$ and the group $G$ acts on  $\spc{L}$ by isometries.
Assume $\diam(\spc{L}/G)>\tfrac\pi2$.
Show that the action of $G$ has a fixed point on $\spc{L}$.
\end{thm}









