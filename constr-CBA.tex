­%%!TEX root = the-constr-CBA.tex
\chapter{Convexity}

\section{Local convexity implies global}

Assume $\spc{U}\in\Cat{}{\kappa}$. 
Recall that by Corollary \ref{cor:loc-geod-are-min} any two points in  $\spc{U}$ on the distance smaller that $\varpi\kappa$ are joined by unique minimizing geodesic.
It follows that any weakly $\varpi\kappa$-convex set in $\spc{U}$ is convex.

\begin{thm}{Lemma}
Let $\spc{U}\in \cCat{}{\kappa}$ 
and $A\subset \spc{U}$ be a closed $\varpi\kappa$-convex set.
Then for the function 
\[f=\sn\kappa\circ\dist{A}{}{}\]
we have
\[f''+\kappa f\ge 0\]
in $\oBall(A,\tfrac{\varpi\kappa}2)$.
\end{thm}

\parit{Proof.}
???
\qeds

\begin{thm}{Corollary}
Let $\spc{U}\in \cCat{}{\kappa}$ 
and $A\subset \spc{U}$ be a closed  locally convex set.
Then there is an open set $\Omega\supset A$
such that the function 
\[f=\sn\kappa\circ\dist{A}{}{}\]
sutisfies 
\[f''+\kappa f\ge 0\]
in $\Omega$.
\end{thm}

\parit{Proof.}
???
\qeds



\begin{thm}{Theorem}
Assume $\spc{U}\in \cCat{}{\kappa}$ and $A\subset \spc{U}$ is a closed connected locally convex set.
Assume $\dist{x}{y}{}<\varpi\kappa$ for any $x,y\in A$.
Then $A$ is $\varpi\kappa$-convex.
\end{thm}

The following proof is given by Sergei Ivanov, see \cite{ivanov:local-global-convexity}

\parit{Proof.}
Note that $A$ is path connected.
Indeed, local convexity implies that $A$ is locally path connected.
Since $A$ is connected, the later implies that $A$ is path connected.

Fix two points $x,y\in A$. 
Let us connect $x$ to $y$ by a path $\alpha\:[0,1]\to A$.
Denote by $\gamma_t$ the geodesic path from $x$ to $\alpha(t)$.
By ??? it depends continuosly from ...

According to ??? there is an open set $\Omega\supset A$
such that for the function $f(x)=\sn\kappa\dist{A}{x}{}$ 
we have
\[f''+\kappa f\ge 0\]
everywhere in $\Omega$.
It follows that if $\gamma$ is a geodesic in $\Omega$ 
of length $\ell<\varpi\kappa$ 
has its ends in $A$ then it lies in $A$ completely.



Connect them by a broken geodesic in $A$;
let $x=x^0,x^1,\dots,x^\kay=y$ be its vertices.
Choose consecutive geodesics $[x^0x^1]$ and $[x^1x^2]$ and move $x^1=x^1(t)$ to $x^2$ along $[x^1x^2]$. 
Since distance between any pair of points in $A$ is less than $\varpi\kappa$,
the geodesic $[x^0,x^1(t)]$ uniquely defined fro any $t$.
Let $t_0$ be the moment  touches the boundary at some moment $t$, observe a contradiction. If not, then the final segment $[a,c]$ is also contained in the interior, so we get a polygonal line with fewer edges. Repeat this procedure until it becomes a single segment.


Take $x,y\in A$, there is a polygonal path from $a$ to $b$ in $A$. Suppose this path has $N$ edges. Then there is a shortest polygonal path of at most $N$ edges in the closure of $A$. If it is not a straight segment, the first edge must touch the boundary of $A$ (otherwise one can shorten the path). The first point where it meets the boundary has obvious problems with local supporting hyperplane.
\qeds


\section{Reshetnyak's gluing theorem.}\label{sec:cba-gluing}

The following theorem was proved by Reshetnyak \cite{reshetnyak:glue},  assuming $\spc{U}_1, \spc{U}_2$ are proper.

\begin{thm}{Reshetnyak gluing}\label{thm:gluing}
Suppose 
$\spc{U}_1$ and  $\spc{U}_2$ are two $\varpi\kappa$-geodesic spaces 
with isometric complete $\varpi\kappa$-convex sets $A_i\subset\spc{U}_i$.  Let $\iota\:A_1\to A_2$ be an isometry.
Let $\spc{W}=\spc{U}_1\sqcup_{\iota}\spc{U}_2$;
i.e., $\spc{W}$ is gluing of $\spc{U}_1$ and  $\spc{U}_2$ along $\iota$;see Section~\ref{sec:quotient}.

Then 
\begin{subthm}{gluing0}
Both canonical mappings $\jmath_i\:\spc{U}_i\to\spc{W}$ are isometric 
and the images $\jmath_i(\spc{U}_i)$ are $\varpi\kappa$-convex subsets in $\spc{W}$.
\end{subthm}

\begin{subthm}{gluing2}
If $\spc{U}_1, \spc{U}_2\in\cCat{}{\kappa}$  
then $\spc{W}\in\cCat{}{\kappa}$.
\end{subthm} 
\end{thm}

\parit{Proof.} 
Part (\ref{SHORT.gluing0})
follows directly from $\varpi\kappa$-convexity of $A_i$.

\parit{(\ref{SHORT.gluing2}).} 
According to (\ref{SHORT.gluing0}),
we can identify $\spc{U}_i$ with its image $\jmath_i(\spc{U}_i)$ in $\spc{W}$;
this way both subsets $A_i\subset \spc{U}_i$ will be identified and denoted further by $A$.
Note that $A=\spc{U}_1\cap \spc{U}_2\subset \spc{W}$,
therefore $A$ is also $\varpi\kappa$-convex set in $\spc{W}$.

Further note that completeness of $\spc{U}_1$ and $\spc{U}_2$ implies completeness of $\spc{W}$.
Thus part (\ref{SHORT.gluing2}) can be reformulated the following way:

\begin{thm}{Reformulation of \ref{gluing2}}
Let $\spc{W}$ be a complete length space which has two closed $\varpi\kappa$-convex sets $\spc{U}_1,\spc{U}_2\subset\spc{W}$ such that
$\spc{U}_1\cup\spc{U}_1=\spc{W}$ and $\spc{U}_1,\spc{U}_2\in\cCat{}{\kappa}$.
Then $\spc{W}\in\cCat{}{\kappa}$.
\end{thm}

Set $A=\spc{U}_1\cap\spc{U}_1$.
First we prove the following:

\begin{clm}{}\label{clm:geod-gluing}
If our space $\spc{W}$ happened to be $\varpi\kappa$-geodesic then $\spc{W}\in\cCat{}{\kappa}$
\end{clm}



\begin{wrapfigure}{r}{20mm}
\begin{lpic}[t(0mm),b(0mm),r(0mm),l(0mm)]
{pics/resh-glue(1)}
\lbl[l]{6.5,21;$x^0$}
\lbl[r]{1.5,1.5;$x^1$}
\lbl[tl]{16,0.5;$x^2$}
\lbl[rb]{4,11;$z^1$}
\lbl[lb]{13,8;$z^2$}
\end{lpic}
\end{wrapfigure}

\parit{Proof of the claim.}
According to \ref{prop:k-thin},
it is sufficient to show that any triangle $\trig {x^0}{x^1}{x^2}$ of perimeter $<2\cdot \varpi\kappa$ 
in $\spc{W}$ is $\kappa$-thin.
This is obviousely true if all three points $x^0$, $x^1$, $x^2$ lie in one of $\spc{U}_i$.
Thus, without loss of generality, we may assume that $x^0\in\spc{U}_1$ and $x^1,x^2\in\spc{U}_2$.

Choose points $z^1,z^2\in A$ 
which lie correspondingly on the sides $[x^0x^1], [x^0x^2]$.
Note that all distances between any pair of points from $x^0$, $x^1$, $x^2$, $z^1$, $z^2$ are less than $\varpi\kappa$.
Therefore
\begin{itemize}
\item triangle $\trig{x^0}{z^1}{z^2}$ lies in $\spc{U}_1$,
\item both triangles $\trig{x^1}{z^1}{z^2}$ and $\trig{x^1}{z^2}{x^2}$ lie in $\spc{U}_2$.
\end{itemize}
In particular each triangle $\trig{x^0}{z^1}{z^2}$,
$\trig{x^1}{z^1}{z^2}$ and $\trig{x^1}{z^2}{x^2}$ is $\kappa$-thin.

Applying the inheritance lemma for thin triangles (\ref{lem:inherit-angle}) twice 
we get that $\trig {x^0}{x^1}{z^2}$ 
and consequently $\trig {x^0}{x^1}{x^2}$ is $\kappa$-thin.
\claimqeds

Now we come back to the general case;
i.e., $\spc{W}$ is not necessury $\varpi\kappa$-geodesic.
Note that the ultrapower $\spc{W}^\o$ is geodesic, see \ref{cor:ulara-geod}.
It is easy to see that 
\begin{itemize}
\item $\spc{U}_i^\o$ form weak $\varpi\kappa$-convex sets of $\spc{W}^\o$,
\item $A^\o=\spc{U}_1^\o\cap\spc{U}_2^\o$ is also a weakly $\varpi\kappa$-convex set.
\end{itemize}
%???IT SEEMS THAT ULTRAPRODUCT FOR SUBSETS ARE NOT REALLY DEFINED YET
From Proposition~\ref{prop:CAT^omega}, $\spc{U}_i^\o\in \cCat{}{\kappa}$.
According to Corollary~\ref{cor:weak>convex}, 
we get that $A^\o$ is $\varpi\kappa$-convex in both $\spc{U}_i$ and
concequently both $\spc{U}_i^\o$ are $\varpi\kappa$-convex subsets in $\spc{W}^\o$.

From Claim~\ref{clm:geod-gluing}, 
we get $\spc{W}^\o\in\cCat{}{\kappa}$.
Applying the Proposition~\ref{prop:CAT^omega} once more, 
we get $\spc{W}\in\cCat{}{\kappa}$.
\qeds 

\begin{thm}{Exercise}
Let $Q$ be the nonconvex subset of the plane 
bounded by two rays $\gamma_1$ and $\gamma_2$
with common starting point and angle $\alpha$ between them.
Assume $\spc{U}\in\cCat{}{0}$
and $\gamma_1',\gamma_2'$ be two rays in $\spc{U}$
starting point and angle $\alpha$ between them.
Show that the space glued from $Q$ and $\spc{U}$ along the corresponding rays is a $\cCat{}{0}$ space.
\end{thm}

\begin{thm}{Exercise}\label{ex:reshetnyak-doubling}
Assume $\spc{U}\in\cCat{}{0}$ and $A\subset \spc{U}$ is a closed subset.
Assume that the doubling of $\spc{U}$ in $A$ is $\cCat{}{0}$. 
Show that $A$ is a totally convex set of $\spc{U}$.
\end{thm}



%%%%%%%%%%%%%%%%%%%%%%%%%%%%%%%%%%%%%%%%%%%%%%%%%


\section{Convex sets and projection}

\begin{thm}{Lemma}\label{lem:model-d-seg}
Let $A\subset \Lob2\kappa$ 
be a geodesic segment. 
Then 
$f=\sn{\kappa}\circ\dist{A}{}{}.$
 satisfies \[f''+\kappa\cdot f \ge  0.\]
\end{thm}
\parit{Proof.} The restriction of
$\dist{A}{}{}$ to a sufficiently short geodesic coincides either with
distance to the image of a complete geodesic, or distance to an endpoint of
$A$, or a join of these two. The first  satisfies $f''+\kappa\cdot f =  0$ by \ref{sn-diff-eq}.  The second satisfies $f''+\kappa\cdot f \ge 0$ by direct calculation from \ref{md-diff-eq}. This suffices, since the derivative at the
join is two-sided, by the first variation formula.
\qeds


Part (\ref{SHORT.distance-to-convex}) of  the next theorem should be compared to the theorem on distance to the boundary of a $\CBB{}{}$-space (\ref{thm:dist-to-bry}).


%Recall that $\cBall[K,\tfrac{\varpi\kappa}{2}]$ denotes the closed tubular neighborhood of radius $R$ about $K$ (\ref{sec:metric spaces}).



\begin{thm}{Theorem on distance to a convex set} 
\label{thm:retract-to-convex} 
Suppose $\spc{U}$ is  a complete  length space satisfying $\curv\spc{U}\le\kappa$.  Let  $A\subset\spc{U} $  be a closed totally $\varpi\kappa$-convex subset. 
Then:

\begin{subthm}{footpoint}
There is a continuous map 
$$f_A\: \oBall[A,\tfrac{\varpi\kappa}{2}]\to A$$
	such that  $f_A(x)$ is the unique closest  point of $x$ to $A$. 
%For any $x\in \cBall[K,\tfrac{\varpi\kappa}{2}]$, there is a unique closest  point of $x$ to $K$ (called the \emph{footpoint of $x$ on $K$}, 
%denoted here by $f_K(x))$.  
Moreover, there is a unique geodesic from $x$ to $f_A(x)$, and this geodesic depends continuously on $x$.  \end{subthm}

%\begin{subthm}{retract-to-convex}
%If $\kappa\le 0$, $K$ is a short retract under $f_K$  of $\cBall[K,\tfrac{\varpi\kappa}{2}]$; i.e. the footpoint map  
%\[f_K\:\cBall[K,\tfrac{\varpi\kappa}{2}]\to K
%\]
%is a short map that is identical on $K$.

%Moreover $K$  is a strictly short retract under $f_K$ of $\oBall[K,\tfrac{\varpi\kappa}{2}]$; i.e. 
%for any $x\in K$ and $y\notin K\cap\oBall[K,\tfrac{\varpi\kappa}{2}] $, we have 
%\[\dist{f_K(y)}{x}{}<\dist{y}{x}{}.\]
%\end{subthm}

\begin{subthm}{distance-to-convex}
Set
$f=\sn{\kappa}\circ\dist{A}{}{}.$
Then the  restriction of $f$ to  $\oBall[A,\tfrac{\varpi\kappa}{2}]$  satisfies \[f''+\kappa\cdot f\ge 0%\eqlbl{eq:dist-convex}
 .\]
In particular, if $\spc{U}$ is a complete  length space  with  $\curv\spc{U}\le0$, and $A\subset\spc{U}$  is  totally convex, then the  function
$f= \dist{A}{}{}$
is convex on $\spc{U}$.
\end{subthm} 

\begin{subthm}{short-retract} 
If $\kappa\le 0$,
%$A\subset \oBall(p,\tfrac{\varpi\kappa}{2})$, 
then $f_A$ is a strictly short retraction of $ \spc{U}$ onto $A$,
i.e. $f_A$ is a short map
that is identical on $A$, such that for any $x\in A$ and $y\notin A$ we have 
\[\dist{x}{f_A(y)}{}<\dist{x}{y}{}.\]
 % If $\kappa \le 0$, then $f_A$ is a short retraction of $\spc{U}$ onto $A$.
 \end{subthm}
\end{thm}
\parit{Proof; (\ref{SHORT.footpoint}).}
By the Lifting globalization theorem  \ref{thm:globalization-lift}, for any  $p\in\spc{U}$
 there is a space $\spc{B}_p\in\Cat{}{\kappa}$, where $\spc{B}_p=\oBall[\hat p,\varpi\kappa/2]$ for some $\hat p \in \spc{B}$,
and a locally isometric map $\map_p\:\spc{B}_p\to\spc{U}$
with $\map(\hat p)=p$ and the following lifting property: 
for any curve $\alpha\:[0,a]\to\spc{U}$ with $\alpha(0)=p$ and $\length\alpha\le\varpi\kappa/2$,
there is a unique curve $\hat\alpha \:[0,a]\to \spc{B}_p$ such that $\hat\alpha (0) = \hat p$ and $\map_p\circ\hat\alpha=\alpha$.

%Let $\map\: \hat {\spc{U}}\to\spc{U}$
%be the simply connected metric covering.
%By the Hadamard--Cartan theorem
%(\ref{thm:hadamard-cartan}), $\hat {\spc{U}}\in\Cat{}{0}$.  The map $\map$ is a local isometry and has  the lifting property, i.e.
%for any $x \in \spc{U}$ and $\hat x \in \hat {\spc{U}}$ such that $\map (\hat x) = x$, and any path $\alpha\:[0,1]\to\spc{U}$ such that  $\alpha(0)=x$, 
%there is a unique path $\hat\alpha \:[0,1]\to \hat{\spc{U}}$ such that $\hat\alpha(0)=\hat x$ 
%and $\map\circ\hat\alpha=\alpha$.

Fix $p\in \oBall[A,\tfrac{\varpi\kappa}{2}]$. 
Set $\hat A_p= (\map_p)^{-1}(A)$.

If $\hat\gamma$ is a geodesic of $\spc{B}_p$  with endpoints in $\hat A_p$, then $\length \hat\gamma < \varpi\kappa$, and $\map_p\circ\hat\gamma$ is a local geodesic of the same length in $\spc{U}$ with endpoints in $A$.  Since  $A\subset\spc{U} $  is totally $\varpi\kappa$-convex,  $\map_p\circ\hat\gamma$ lies in $A$.  Therefore $\hat\gamma$ lies in $\hat A_p$, and   $\hat A_p$ is a convex subset of  $\spc{B}_p$.  

Since $\map_p$ is a local isometry,  $\dist{\hat A_p}{\hat p}{\spc{B}_p} \ge \dist{A}{p}{\spc{U}}$.  Then the lifting property of $\map_p$ implies $\dist{\hat A_p}{\hat p}{\spc{B}_p} = \dist{A}{p}{\spc{U}}$.  Moreover, a curve $\hat\alpha$ realizes distance from $\hat p$ to $\hat A_p$ if and only if  $\alpha = \map_p\circ\hat\alpha$
 realizes distance from $p$ to $A$.

Let  $\spc{W}=\spc{B}_p
\sqcup_{\iota}\spc{B}_p$ be the doubling of $\spc{B}_p$ with respect to  $\hat A_p$, where  $\iota\:\hat A_p\to\spc{B}_p$ is the inclusion map, i.e. $\spc{W}$ is the result of gluing two copies of $\spc{B}_p$ along $\hat A_p$  (\ref{sec:doubling}).  Then $\spc{W}\in\Cat{}{\kappa}$ by Reshetnyak gluing (\ref{thm:gluing}).


Given $w=(\hat p,1)\in \spc{W}$, set $w'=(\hat p,- 1)$.  Then there is a unique geodesic $\ddot\gamma_p$ in $\spc{W}$ joining $w$ and $w'$.  This geodesic must intersect $\hat A_p\subset \spc{W}$, and since there is no shorter curve in $\spc{W}$ that joins $w$ and $w'$, must consist of two geodesic segments $(\hat \gamma_p,1)$ and $(\hat\gamma_p, -1)$ where $\hat \gamma_p$ is a geodesic in $\spc{B}_p$  that realizes distance from $\hat p$ to  $\hat A_p$.  Uniqueness of $\ddot\gamma_p$ implies uniqueness of $\hat \gamma_p$.  It follows from the lifting property  that there is a unique closest point of $A$ to $p$, and  $\gamma_p=\map_p\circ\hat\gamma_p$ is the unique geodesic of $\spc{U}$ that realizes distance from $p$ to $A$.

Suppose $x\in \oBall[A,\tfrac{\varpi\kappa}{2}]$ satisfies $ \dist{p}{x}{\spc{U}}=\eps>0$ and lies in a $\Cat{}{\kappa}$ 
 neighborhood $\Omega$ of $p$ for which $\map_p|\Omega$ is an isometry. For sufficiently small $\eps $,  the unique minimizer $\gamma_x$ from $x$ to $A$ lies in $ \oBall[p,\tfrac{\varpi\kappa}{2}]$.    Let $\alpha\:[0,a]\to\spc{U}$ be the unitspeed curve that runs first  along  the geodesic in $\Omega$  from $p$ to $x$ and then along $\gamma_x$ to $A$. Let $\hat\alpha_{p,x}$ be the curve in $\spc{B}_p$ such that  $\hat\alpha (0) = \hat p$ and $\map_p\circ\hat\alpha=\alpha$.  Set $\hat x =\hat\alpha_{p,x} (\eps)$.  Then the curve
\[
\hat\gamma_{p,x} =\hat\alpha_{p,x} |[\eps, a]
%\eqlbl{eq:d-realizer}
\]
 is a geodesic of $\spc{B}_p$ that  realizes distance from $\hat x$ to  $\hat A_p$.  
Since  $\map_p\circ\hat\gamma_{p,x}$ is a reparametrization of $\gamma_x$,
\[
\dist{\hat A_p}{\hat x}{\spc{B}_p}= \dist{A}{ x}{\spc{U}}. 
\eqlbl{eq:d-realizer}
\]
 
Since geodesics of  $\spc{W}$ vary continuously with their endpoints (\ref{lem:cat-unique}), then $\hat\gamma_{p,x}$ varies continuously with $x$.
Hence (\ref{SHORT.footpoint}).

%Suppose there are two closest points $z_1,z_2$ of $\hat A_x$ to $\hat x$.  By point-on-side comparison (\ref{cat-monoton}), the distance from $\hat x$  to an interior point of $[z_1z_2]$ is $<\dist{\hat x}{z_i}{}$. This contradiction shows there is a unique closest point of $\hat A_x$ to $\hat x$.  It follows that there is a unique closest point of $A$ to $x$. 

\parit{(\ref{SHORT.distance-to-convex}).} 
%It suffices to verify the differential inequality $(f\circ\gamma)''+\kappa\cdot (f\circ\gamma)\ge 0$ for any geodesic 
For any $p\in \oBall[A,\tfrac{\varpi\kappa}{2}]$, set $ p^*=f_A( p )$. 


Suppose  $\beta\:[0,\eps]\to\spc{U}$ is a unit-speed geodesic  in $\oBall[A,\tfrac{\varpi\kappa}{2}]$ from $p$ to $x$.  
We wish to examine the distances $\dist{\beta (t)}{ A}{\spc{U}}$.
By \ref{eq:d-realizer}, we may assume $\spc{U}\in\Cat{}{\kappa}.$

Extend $\beta$ to $\alpha\:[0,a]\to\spc{U}$, so that $\alpha[0,\eps]=\beta$  and $\alpha$  is a closed unit-speed curve that parametrizes 
the  quadrilateral $Q$ with sides   $\alpha([0,\eps]) =[px]$, $[xx^*]$, $[x^* p^*]$, $[p^*p]$.
 For $\eps$ sufficiently small, $\length Q < 2\varpi\kappa$ by continuity of $f_A$, and we may  apply Reshetnyak majorization (\ref{thm:major}) to  $\alpha$.  

Thus $\alpha$ is majorized by a closed convex region $D$  lying in a hemisphere $\Lob2\kappa$, where $D$ is bounded by a simple closed unit-speed curve $\tilde\alpha\:[0,a]\to\Lob2\kappa$.  By definition the majorizing map $F\:D\to \spc{U}$ is length-nonincreasing and  we may take $F\circ\tilde\alpha=\alpha$. It follows that  $\tilde\alpha$ parametrizes a convex quadrilateral $\tilde Q$ having  the same sidelengths as $Q$.  Let $\tilde\sigma$ denote the side of $\tilde Q$ opposite $\tilde\alpha([0,\eps])$.

By majorization, for $0\le t\le\eps$,
$$\dist{\tilde\alpha (t)}{ \tilde\sigma}{\Lob2\kappa}\ge  \dist{\beta (t)}{ y_t}{\spc{U}}\ge \dist{\beta (t)}{ A}{\spc{U}},$$
 where $y_t\in A$ is a point on $[p^* x^*]$.  


By construction, these inequalities are equalities  at
$t=0$ and $\eps$. 
Since $\dist{\tilde\alpha (t)}{ \tilde\sigma}{\Lob2\kappa}$, $0\le t\le \eps$, satisfies the desired convexity condition by Lemma \ref{lem:model-d-seg},
then
(\ref{SHORT.distance-to-convex}) follows.

\parit{(\ref{SHORT.short-retract}).}
For any $x\in \spc{U}=\oBall[A,\tfrac{\varpi\kappa}{2}]$, set  $x^*=f_A( x )$. 
Suppose  $x,y\in\spc{U}$.

Clearly $x=x^*$ if and only if $x\in A$.
Further, if $x\notin A$ then 
\[\mangle\hinge{x^*}{x}{p}\ge\tfrac\pi2\eqlbl{eq:<x*xp>=pi/2}\] 
for any $p\in A$;
otherwise there would be a point on $[x^*p]\subset A$ which is closer to $x$ than $x^*$.

Let us show that $\dist{x^*}{y^*}{}\le\dist{x}{y}{}$ for any $x,y\in \spc{U}$.
Without loss of generality we assume $x^*\not=y^*$ and $x\notin A$.

\parit{Case 1:} $y\in A$, so $y=y^*$.
From \ref{eq:<x*xp>=pi/2}, we have $\mangle\hinge{x^*}{x}{y^*}\ge\tfrac\pi2$.
From ??? comparison it follows that $\dist{x^*}{y^*}{}=\dist{x^*}{y}{}\le \dist{x}{y}{}$.

%%%ADD PIC

\parit{Case 2:} $x,y\notin A$.
In this case, \ref{eq:<x*xp>=pi/2} implies $\mangle\hinge{x^*}{x}{y^*}$, $\mangle\hinge{y^*}{x}{x^*}\ge\tfrac\pi2$.
Apply Reshetnyak majorization to the quadrilateral $Q$ with sides   $[xy]$, $[yy^*]$, $[y^* x^*]$, $[x^*x]$. Since a majorizing map cannot increase angles, there is a quadrilateral in  $\Lob2\kappa$ with the same sidelengths as $Q$, and whose angles at the vertices corresponding to $x^*$ and $y^*$ are $\ge\tfrac\pi2$.  It follows that 
$\dist{x^*}{y^*}{}\le \dist{x}{y}{}$.\qeds

  
\begin{thm}{Theorem on short retract, $\kappa>0$}
\label{strictly-short-retract} 
Suppose $\spc{U}\in\Cat{}{\kappa}$ where $\kappa>0$.  
Let $A\subset\spc{U} $  be a closed convex subset.
Assume $A\subset \cBall[p,\varpi\kappa]$ for some $p\in \spc{U}$.
Then $A$ is a \emph{short retract}\index{short retract} of $\spc{U}$;
i.e. there is a short map $\map[2]\:\spc{U}\to A$ which is identical on $A$.

More over if $A\subset \oBall(p,\tfrac{\varpi\kappa}{2})$, 
then the map $\map[2]$ can be chousen so that in addition 
\[\dist{\map[2](y)}{x}{}<\dist{y}{x}{}.\]
 for any $x\in A$ and $y\notin A$.
\end{thm}


\parit{Proof.} 
Applying rescaling, we can assume  $\kappa=1$.
Without loss of generality, we may assume that $p\in A$.

If $\dist{A}{x}{}\ge\tfrac\pi2$ then set $\map[2](x)=p$.

Otherwise, if $\dist{A}{x}{}<\tfrac\pi2$, by ???, 
there is unique point $x^*\in A$ that minimizes distance to $x$;
i.e. $\dist{x^*}{x}{}=\dist{A}{x}{}$.
In this case set 
\begin{align*}
\ell_x&=\dist{p}{x^*}{},
\\
\phi_x&=\tfrac\pi2-\dist[{{}}]{x^*}{x}{},
\\
\sin\psi_x&=\sin\phi_x\cdot\sin\ell_x, 
\ \ 0\le \psi_x\le \tfrac\pi2
\intertext{and define}
\map[2](x)&=\geod_{[px^*]}(\psi_x).
\end{align*}

Note that $\map[2]$ is a retraction to $A$; 
i.e.,
$\map[2](x)\in A$ for any $x\in \spc{U}$
and 
$\map[2](a)=a$ for any $a\in A$.

Let us show that $\map[2]$ is short.
Assume $x,y\in\oBall(A,\tfrac\pi2)$,
\begin{align*}
x'&=\map[2](x)
&
y'&=\map[2](y)
\\
r&=\dist{x}{y}{}
&
r'&=\dist{x'}{y'}{}
\\
d&=\dist{x^*}{y^*}{}
&
\alpha&=\angk1{p}{x^*}{y^*}
\end{align*}

Note that 
\[\cos r\le 
\cos\phi_x\cdot\cos\phi_y
-
\cos d\cdot\sin\phi_x\cdot\sin\phi_y.\eqlbl{eq:cos(r)}\]

Indeed, if $x,y\notin A$,
then 
$\mangle\hinge{x^*}{x}{y*}, 
\mangle\hinge{y^*}{y}{x*}
\ge 
\tfrac\pi2$
and
the inequality~\ref{eq:cos(r)} follows from the Arm lemma (\ref{lem:arm}).
If $x\in A$ and $y\notin A$, we get \ref{eq:cos(r)}, by angle comparison (\ref{cat-hinge}) 
since $\mangle\hinge{y^*}{y}{x*}\ge \tfrac\pi2$.
The same way \ref{eq:cos(r)} is proved 
in case $x\notin A$ and $y\in A$.
Finally, if $x,y\in A$, $\phi_x=\phi_y=\tfrac\pi2$ and $r=d$;
i.e., the inequality trivially holds.

Further note that
\[\cos\alpha
=
\frac{\cos d-\cos \ell_x\cdot\cos\ell_y}{\sin\ell_x\cdot\sin\ell_y}.\]
Applying angle-sidelength  monotonicity (\ref{cor:monoton-cba}) we get
\begin{align*}
\cos r'&\ge
\cos\psi_x\cdot\cos\psi_y
-
\cos \alpha \cdot\sin\psi_x\cdot\sin\psi_y=
\\
&=
\cos\psi_x\cdot\cos\psi_y
-(\cos d-\cos \ell_x\cdot\cos\ell_y)\cdot\sin\phi_x\cdot\sin\phi_y\ge
\\
&\ge \cos\psi_x\cdot\cos\psi_y
-\cos d\cdot\sin\phi_x\cdot\sin\phi_y
\end{align*}


Note that 
$\psi_x\le \phi_x$
and
$\psi_y\le \phi_y$;
in particular,
\[
\cos\phi_x\cdot\cos\phi_y\le \cos\psi_x\cdot\cos\psi_y.
\]
Hence 
\[\cos r'\ge \cos r;\]
i.e., the restriction $\map[2]|\oBall(A,\tfrac\pi2)$ is short.
Clearly $\map[2]$ is continuous,
since the complement of $\oBall(A,\tfrac\pi2)$ is mapped to $p$,
we get that $\map[2]$ is short; i.e.,
\[r'\le r \eqlbl{eq:cos=<cos}\]
for any $x,y\in\spc{U}$.

If we have equality in \ref{eq:cos=<cos}
then 
\[\cos\ell_x\cdot\cos\ell_y\cdot\sin\phi_x\cdot\sin\phi_y=0.\]
If $A\subset \oBall(p,\tfrac\pi2)$ then $\ell_x,\ell_y<\tfrac\pi2$;
which implies that $x\in A$ or $y\in A$.
Without loss of generality we may assume that $x\in A$.

It remains to show that if $y\notin A$ 
then the inequality~\ref{eq:cos=<cos}
is strict.
If $\dist{A}{y}{}\ge\tfrac\pi2$ then \ref{eq:cos=<cos} holds since 
the left hand side is $<\tfrac\pi2$,
while right hand side is $\ge \tfrac\pi2$.
If $\dist{A}{y}{}<\tfrac\pi2$ then $\phi_y>0$ and clearly $\psi_y<\phi_y$,
hence the inequality~\ref{eq:cos=<cos} is strict.
\qeds

We fail to find a transparent geometric proof of the statement above.
Below you will find a geometric way to think about the construction; 
%%%DOWN
compare to the construction 
in the proof of Kirszbraun's theorem (\ref{thm:kirsz+}).
%%%UP

\parit{Geometric interpretation of the map $\map[2]$.}
Set $\mathring{\spc{U}}=\Cone \spc{U}$;
denote by $\mathring{A}$ the subcone of $\mathring{\spc{U}}$ spanned by $A$.
The space $\spc{U}$ can be naturally identified with the unit sphere in $\mathring{\spc{U}}$;
i.e., the set 
\[\set{z\in \mathring{\spc{U}}}{|z|=1}.\]

According to ??? $\mathring{\spc{U}}\in\cCat{}{0}$.
Note that $\mathring{A}$ forms a convex closed subset of $\mathring{\spc{U}}$.
According to ???, for any point $x$ there is unique point $\hat x\in \mathring{A}$
which minimize the distance to $x$;
i.e., $\dist{\hat x}{x}{}=\dist{A}{x}{}$.
(If $|\hat x|\ne0$ then in the notations above we have
$x^*=\tfrac1{|\hat x|}\cdot\hat x$.)

Consider the ray $t\mapsto t\cdot p$ in  $\mathring{\spc{U}}$.
According to ???, %ASK Stephanie???
for given $s\in \mathring{\spc{U}}$
the geodesics $\geod_{[s\ t\cdot p]}$ converge as $t\to\infty$ to a ray, 
say $\alpha_s\:[0,\infty)\to \mathring{\spc{U}}$.



Note that if $|x|=1$ then $|\hat x|\le 1$.
By assumption for any $a\in A$ the function $t\mapsto |\alpha_a(t)|$ is monotonicity increasing.
Therefore there is unique value $t_x\ge 0$ such that
$|\alpha_{\hat x}(t_x)|=1$.
Consider the map $\map[2]\:\spc{U}\to A$
defined as 
\[\map[2](x)=\alpha_{\hat x}(t_x).\]

\section{Exercises}

\begin{thm}{Exercise} 
(Gluing with short maps)
 Let $X,X' \in\Cat{}{0}$, $K \subset X, K' \subset X'$ be closed convex subsets, 
and suppose the length-metrics on $\Fr_X K$ and $\Fr_{X'} K'$ are finite (???). 
Suppose $\phi\: \Fr_X K \to \Fr_{X'} K'$ is length-preserving and 
short with respect to the inherited
metrics. 
To $X'\backslash\Int K'$ in its length metric, 
glue $K$ along $\phi$, to obtain the space $Y$. 
Show $Y \in \Cat{}{0}$.

For example, take $K= \cBall[x,R]  \subset \Lob{m}0$ and $K' = \cBall(x',\sinh R)  \subset \Lob{m}{-1}$.
\end{thm}


