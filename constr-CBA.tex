­%%!TEX root = the-constr-CBA.tex
\chapter{Constructions in CBA spaces}


\section{Cones} 
\label{sec:cones}

Now we consider warped products (as defined  \ref{sec:wp-def}) for which the base $\spc{B}$ is an interval   and the warping function $f$ satisfies $f''=-\kappa\cdot f$.  We write  $\II_\kappa= [0,\varpi\kappa]$ if $\kappa > 0$, $\II_\kappa= [0,\infty)$ if $\kappa\le 0$.

\begin{thm}{Definition}
\label{def:cones}
The  \emph{$\kappa$-cone} $\Cone_\kappa \spc{X}$ over a metric space $\spc{X}$ is defined by
\[\Cone_\kappa \spc{X}= \II_\kappa\times_{\sn{\kappa} } \spc{X}.\]  
$\Cone_{\,0}  \spc{X}$ is called the  \emph{Euclidean cone} over $\spc{X}$, and written $\Cone \spc{X}$. $\Cone_{\,1} \spc{X}$ is called the
 \emph{spherical suspension} of $\spc{X}$. 
 
 For $\kappa <0$, we  also define  \emph{$\kappa$-cylinders} (respectively, \emph{ideal $\kappa$-cones}) over $\spc{X}$ as 
 \[\RR\times_{\cs{\kappa}} \spc{X} 
\quad (\text{respectively }\RR\times_{\exp (t\cdot\sqrt{-\kappa})} \spc{X}).\]

For $\kappa\le 0$, the \emph{vertex} of $\Cone_\kappa \spc{X}$ is the equivalence class $\{0\}\times \spc{X}$. If $\kappa>0$, there are two vertices, $\{0\}\times \spc{X}$ and $\{\varpi\kappa\}\times \spc{X}$.  We denote a vertex by  $o$.                             \end{thm} 

\parbf{Remark.} In these constructions, the leaves $\II_\kappa\times\{x\}$ or $\RR\times\{x\}$ in the restricted  metric are isometric copies of $\II_\kappa$ or $\RR$,  and are clearly uniquely minimizing on proper subintervals.
The leaf $\{t\}\times \spc{X}$ is length-isometric to $\spc{X}$ rescaled by the value of the warping function at $t$.  In a $\kappa$-cone, this leaf is the sphere of radius $t$ about $o$.
\ms

$\Cone_\kappa \spc{X}$ inherits the cosine law of the model space $\Lob{}\kappa$, as follows.  
We denote the points of $\Cone_\kappa \spc{X}$ by $t\cdot x\df (t,x)$ where $ t\in \II_\kappa, x\in \spc{X}$, and $t\cdot x = o$ if $t=0$. 

\begin{thm}{Lemma}\label{lem:cos-law}
For $u=s\cdot x,v=t\cdot y$, and  $\alpha=\mangle\hinge{o}u v$ (taken to be $0$ if $u$ or $v$ is a cone point), distance in $\Cone_\kappa \spc{X}$ is given by:
\[\dist{u}{v}{}=\side\kappa\{\alpha;t,s\}.\]

\end{thm}


\parit{Proof.}
If $\dist{x}{y}{}<\pi$, apply  Lemma \ref{lem:fiber-independence} to $\Cone_\kappa \spc{X}$ and $\Lob{}\kappa= \Cone_\kappa(\Lob11)$, where $\Lob11$ is the circle of length $2\pi$. Since $\dist{x}{y}{}=\angk{\kappa}{o}{u}{v} = \alpha$, the formula follows from the cosine law of $\Lob{}\kappa$. If $\dist{x}{y}{}\ge\pi$, apply  Lemma \ref{lem:fiber-independence} to $\Cone_\kappa \spc{X}$ and $ \Cone_\kappa(\RR-)$.  
Then $\alpha=\pi$ and 
\[\dist{u}{v}{}=\min\{s+t,2\varpi\kappa-s-t\},\] 
so the formula holds.
\qeds


We sometimes refer to the elements of $\Cone \spc{X}$ as \emph{vectors}.  The vertex $o$ plays the role of a ``zero vector''. 
\begin{thm}{Definition}
\label{def:scalar-product}
The \emph{absolute value}\index{absolute value} $|v|$ of $v=t\cdot x\in \Cone \spc{X}$ is defined by
\[|v|= t =\dist{o}{v}{}.\]
\index{$\vert{*}\vert$}
The \emph{scalar product}\index{scalar product} $\langle u, v\rangle$ of $u,v\in \Cone \spc{X}$ is defined by
\[\langle u, v\rangle
=
(|u|^2+|v|^2-\dist[2]{u}{v}{})/2=|u|\cdot |v|\cos\alpha,\] 
\index{$\<*,*\>$}where $\alpha=\mangle\hinge o u v=\angk{0}{o}{u}{v}$ will be also denoted as $\mangle(u,v)$.
\end{thm}
Note that in general, the sum of ``tangent vectors'' is undefined, so we do abuse the term ``vector'' here.

When it is convenient, we treat $\spc{X}$ as a ``unit sphere'' in $\Cone \spc{X}$, 
i.e., the set of unit vectors. 
Thus, $x\in \spc{X}$ might also refer to the corresponding unit vector in $\Cone \spc{X}$.
In this case for $x,y\in \spc{X}$ we will use mostly $\mangle(x,y)$ for $\dist{x}{y}{\spc{X}}$ 
and $\dist{x}{y}{}$ for $\dist{x}{y}{\Cone \spc{X}}$.

\ms

\begin{thm}{Definition}
\label{def:kappa-join}
The \emph{$\kappa$-join} of metric spaces $\spc{X}$ and $\spc{Y}$ is defined by
\[ \Join_\kappa \l(\spc{X},\spc{Y}\r) 
= 
[0,\varpi\kappa/2]\times_{(\sn{\kappa},\,\cs{\kappa})}\spc{X}\times \spc{Y}).\]
We call $\Join_1 \l(\spc{X},\spc{Y}\r)$
 the \emph{spherical join} of $\spc{X}$ and $\spc{Y}$, and $\Join_{-1} \l(\spc{X},\spc{Y}\r)$  the \emph{hyperbolic join} of $\spc{X}$ and $\spc{Y}$.
\end{thm}

We may express the  model spaces $\Lob{n}\kappa$ as $\kappa$-cones ands $\kappa$-joins. 

\begin{thm}{Lemma}\label{lem:model-as-join}
 Let $\Lob{0}\kappa$ be a point if $\kappa \le 0$ , and two points at distance  $\pi/\sqrt{\kappa}$ apart if $\kappa >0$. Then for $n\ge 1$,
 \begin{align*}
\Lob{n}\kappa &= 
\Cone_\kappa \Lob{n-1}1\\
&=
\RR\times_{\cs{\kappa}} \Lob{n-1}1 \\
&= \RR\times_{\exp \sqrt{-\kappa}\,t} \Lob{n-1}1,
\end{align*}

and

 \begin{align*}
\Lob{m+n+1}\kappa=\Join_\kappa\l(\Lob{m}1,\Lob{n}{\sgn\kappa}\r).
\end{align*}
\end{thm}

\parit{Proof.} Immediate from the definitions.
\qedsf


\begin{thm}{Lemma}\label{lem:cone-of-join}
The spherical join, $\Join_1 \l(\spc{X},\spc{Y}\r)$,
 of metric spaces $\spc{X}$ and $\spc{Y}$  satisfies 
\[ \Cone \Join_1 \l(\spc{X},\spc{Y}\r)
=\Cone \spc{X}\times\Cone \spc{Y}.\]
\end{thm}
\parit{Proof.} By definition, 
\begin{align*}
\Cone\Join_1 \l(\spc{X},\spc{Y}\r)
&= \RR_{\ge 0}\times_r\l(([0,\tfrac\pi2]\times_{\sin t}\spc{X})\times_{\cos t}\spc{Y}\r)
\\&= \l(( \RR_{\ge 0}\times_r[0,\tfrac\pi2])\times_{r\sin t}\spc{X}\r)\times_{r\cos t}\spc{Y}
\\&=\l((\RR_{\ge 0}\times\RR_{\ge 0})\times_x\spc{X}\r)\times_y\spc{Y}
\\&=\Cone \spc{X}\times\Cone \spc{Y}.
\end{align*}
\qedsf



For a function $f:\spc{X}\to\RR$ on a metric space $\spc{X}$, let $\Cone f:\Cone \spc{X}\to\RR$ be the linear homogeneous extension of $f$,
defined by 
\[(\Cone f)(t\cdot x)=tf(x).\]

\begin{thm}{Lemma}\label{lem:cone_f}  Let $f:\spc{X}\to\RR$ be a function on a geodesic space $\spc{X}$.  Then $\Cone f$ is convex  if and only if $f''\ge -f$ on $\spc{X}$ and $f(x)+f(y)\ge 0$ whenever
$\dist{x}{y}{}\ge\pi$.
\end{thm}

\parit{Proof.} 
Let $[xy]$ be a geodesic in $\spc{X}$, and $m$ be its midpoint.  
By lemma \ref{lem:cos-law}, if $\dist{x}{y}{}\le\pi$ then $\Cone [xy]$ in both its length-metric and external metric is isometric to the sector of $\Lob{}0$ whose angle at the origin is $\dist{x}{y}{}$;  
if $\dist{x}{y}{}\ge\pi$,  the geodesic $[s\cdot x\,\,t\cdot y]$ is the union of two segments from the vertex $o$ of lengths $s$ and $t$ respectively.

The inequality  $f''\ge -f$ on $\spc{X}$ is equivalent to the following midpoint inequality for all geodesics $[xy]$ with $\dist{x}{y}{}\le\pi$:
\[f(m)
\le 
\bigl(f(y)+f(x)\bigr)\bigl/\bigl(2\cos\frac{\dist{x}{y}{}}{2}\bigr),\]
since the solutions of $f''(t)=-f(t)$ satisfy the corresponding equation on any interval of length $<\pi$.  The convexity
inequality for $\Cone f$ on a geodesic $[s\cdot x\,\,t\cdot y]$ with $\dist{x}{y}{}\le\pi$, evaluated at the point that divides this geodesic in the
ratio $s:t$, is 
\[2st(s+t)^{-1} f(m)\cos\frac{\dist{x}{y}{}}{2} 
\le 
s(s+t)^{-1}tf(y)+t(s+t)^{-1}sf(x),\]
since linear functions satisfy the corresponding equation.  By sequential application of this process, $f''\ge-f$ holds on $\spc{X}$ if and only if $\Cone f$ is convex on any geodesic whose projection to $\spc{X}$ has length $\le\pi$.

When $\dist{x}{y}{}\ge\pi$, $\Cone f$ on $[s\cdot x\,\,t\cdot y]$ is linear on each segment from $o$, and hence is convex if and only if $f(x)+f(y)\ge 0$.
\qeds

%%%%%%%%%%%%%%%%%%%%%%%%%%%%%%%%%%%%%%%%%%%%%%%%%%%%%%%%


\section{Gluing.}\label{sec:cba-gluing}

The following theorem was proved by Reshetnyak \cite{reshetnyak:glue},  assuming $\spc{U}_1, \spc{U}_2$ are proper.

\begin{thm}{Reshetnyak gluing}\label{thm:gluing}
Suppose 
$\spc{U}_1$ and  $\spc{U}_2$ are two $\varpi\kappa$-geodesic spaces 
with isometric complete $\varpi\kappa$-convex sets $A_i\subset\spc{U}_i$.  Let $\iota\:A_1\to A_2$ be an isometry.
Let $\spc{W}=\spc{U}_1\sqcup_{\iota}\spc{U}_2$;
i.e., $\spc{W}$ is gluing of $\spc{U}_1$ and  $\spc{U}_2$ along $\iota$;see Section~\ref{sec:quotient}.

Then 
\begin{subthm}{gluing0}
Both canonical mappings $\jmath_i\:\spc{U}_i\to\spc{W}$ are isometric 
and the images $\jmath_i(\spc{U}_i)$ are $\varpi\kappa$-convex subsets in $\spc{W}$.
\end{subthm}

\begin{subthm}{gluing2}
If $\spc{U}_1, \spc{U}_2\in\cCat{}{\kappa}$  
then $\spc{W}\in\cCat{}{\kappa}$.
\end{subthm} 
\end{thm}

\parit{Proof.} 
Part (\ref{SHORT.gluing0})
follows directly from $\varpi\kappa$-convexity of $A_i$.

\parit{(\ref{SHORT.gluing2}).} 
According to (\ref{SHORT.gluing0}),
we can identify $\spc{U}_i$ with its image $\jmath_i(\spc{U}_i)$ in $\spc{W}$;
this way both subsets $A_i\subset \spc{U}_i$ will be identified and denoted further by $A$.
Note that $A=\spc{U}_1\cap \spc{U}_2\subset \spc{W}$,
therefore $A$ is also $\varpi\kappa$-convex set in $\spc{W}$.

Further note that completeness of $\spc{U}_1$ and $\spc{U}_2$ implies completeness of $\spc{W}$.
Thus part (\ref{SHORT.gluing2}) can be reformulated the following way:

\begin{thm}{Reformulation of \ref{gluing2}}
Let $\spc{W}$ be a complete length space which has two closed $\varpi\kappa$-convex sets $\spc{U}_1,\spc{U}_2\subset\spc{W}$ such that
$\spc{U}_1\cup\spc{U}_1=\spc{W}$ and $\spc{U}_1,\spc{U}_2\in\cCat{}{\kappa}$.
Then $\spc{W}\in\cCat{}{\kappa}$.
\end{thm}

Set $A=\spc{U}_1\cap\spc{U}_1$.
First we prove the following:

\begin{clm}{}\label{clm:geod-gluing}
If our space $\spc{W}$ happened to be $\varpi\kappa$-geodesic then $\spc{W}\in\cCat{}{\kappa}$
\end{clm}



\begin{wrapfigure}{r}{20mm}
\begin{lpic}[t(0mm),b(0mm),r(0mm),l(0mm)]
{pics/resh-glue(1)}
\lbl[l]{6.5,21;$x^0$}
\lbl[r]{1.5,1.5;$x^1$}
\lbl[tl]{16,0.5;$x^2$}
\lbl[rb]{4,11;$z^1$}
\lbl[lb]{13,8;$z^2$}
\end{lpic}
\end{wrapfigure}

\parit{Proof of the claim.}
According to \ref{prop:k-thin},
it is sufficient to show that any triangle $\trig {x^0}{x^1}{x^2}$ of perimeter $<2\cdot \varpi\kappa$ 
in $\spc{W}$ is $\kappa$-thin.
This is obviousely true if all three points $x^0$, $x^1$, $x^2$ lie in one of $\spc{U}_i$.
Thus, without loss of generality, we may assume that $x^0\in\spc{U}_1$ and $x^1,x^2\in\spc{U}_2$.

Choose points $z^1,z^2\in A$ 
which lie correspondingly on the sides $[x^0x^1], [x^0x^2]$.
Note that all distances between any pair of points from $x^0$, $x^1$, $x^2$, $z^1$, $z^2$ are less than $\varpi\kappa$.
Therefore
\begin{itemize}
\item triangle $\trig{x^0}{z^1}{z^2}$ lies in $\spc{U}_1$,
\item both triangles $\trig{x^1}{z^1}{z^2}$ and $\trig{x^1}{z^2}{x^2}$ lie in $\spc{U}_2$.
\end{itemize}
In particular each triangle $\trig{x^0}{z^1}{z^2}$,
$\trig{x^1}{z^1}{z^2}$ and $\trig{x^1}{z^2}{x^2}$ is $\kappa$-thin.

Applying the inheritance lemma for thin triangles (\ref{lem:inherit-angle}) twice 
we get that $\trig {x^0}{x^1}{z^2}$ 
and consequently $\trig {x^0}{x^1}{x^2}$ is $\kappa$-thin.
\claimqeds

Now we come back to the general case;
i.e., $\spc{W}$ is not necessury $\varpi\kappa$-geodesic.
Note that the ultrapower $\spc{W}^\o$ is geodesic, see \ref{cor:ulara-geod}.
It is easy to see that 
\begin{itemize}
\item $\spc{U}_i^\o$ form weak $\varpi\kappa$-convex sets of $\spc{W}^\o$,
\item $A^\o=\spc{U}_1^\o\cap\spc{U}_2^\o$ is also a weakly $\varpi\kappa$-convex set.
\end{itemize}
%???IT SEEMS THAT ULTRAPRODUCT FOR SUBSETS ARE NOT REALLY DEFINED YET
From Proposition~\ref{prop:CAT^omega}, $\spc{U}_i^\o\in \cCat{}{\kappa}$.
According to Corollary~\ref{cor:weak>convex}, 
we get that $A^\o$ is $\varpi\kappa$-convex in both $\spc{U}_i$ and
concequently both $\spc{U}_i^\o$ are $\varpi\kappa$-convex subsets in $\spc{W}^\o$.

From Claim~\ref{clm:geod-gluing}, 
we get $\spc{W}^\o\in\cCat{}{\kappa}$.
Applying the Proposition~\ref{prop:CAT^omega} once more, 
we get $\spc{W}\in\cCat{}{\kappa}$.
\qeds 
%%%%%%%%%%%%%%%%%%%%%%%%%%%%%%%%%%%%%%%%%%%%%%%%%


\section{Convex sets and projection}

\begin{thm}{Lemma}\label{lem:model-d-seg}
Let $A\subset \Lob2\kappa$ 
be a geodesic segment. 
Then 
$f=\sn{\kappa}\circ\dist{A}{}{}.$
 satisfies \[f''+\kappa\cdot f \ge  0.\]
\end{thm}
\parit{Proof.} The restriction of
$\dist{A}{}{}$ to a sufficiently short geodesic coincides either with
distance to the image of a complete geodesic, or distance to an endpoint of
$A$, or a join of these two. The first  satisfies $f''+\kappa\cdot f =  0$ by \ref{sn-diff-eq}.  The second satisfies $f''+\kappa\cdot f \ge 0$ by direct calculation from \ref{md-diff-eq}. This suffices, since the derivative at the
join is two-sided, by the first variation formula.
\qeds

\begin{thm}{Definition} 
\label{def:convex-set}
Let $\spc{X}$ be a metric space. 
A set $A\subset\spc{X}$ is called 
\emph{totally $R$-convex}%
\index{convex set!totally convex set}\index{totally convex set}
if for every two points $p,q\in A$, every local geodesic with endpoints $p$ and $q$ and length $< R$  lies in $A$.
\end{thm}

Note that if $\spc{X}\in\Cat{}{\kappa}$, then by Corollary \ref{cor:loc-geod-are-min},  $A\subset\spc{X}$ is totally $\varpi\kappa$-convex if and only if $A$ is $\varpi\kappa$-convex.


Part (\ref{SHORT.distance-to-convex}) of  the next theorem should be compared to the theorem on distance to the boundary of a $\CBB{}{}$-space (\ref{thm:dist-to-bry}).


%Recall that $\cBall[K,\tfrac{\varpi\kappa}{2}]$ denotes the closed tubular neighborhood of radius $R$ about $K$ (\ref{sec:metric spaces}).



\begin{thm}{Theorem on distance to a convex set} 
\label{thm:retract-to-convex} 
Suppose $\spc{U}$ is  a complete  length space satisfying $\curv\spc{U}\le\kappa$.  Let  $A\subset\spc{U} $  be a closed totally $\varpi\kappa$-convex subset. 
Then:

\begin{subthm}{footpoint}
There is a continuous map 
$$f_A\: \oBall[A,\tfrac{\varpi\kappa}{2}]\to A$$
	such that  $f_A(x)$ is the unique closest  point of $x$ to $A$. 
%For any $x\in \cBall[K,\tfrac{\varpi\kappa}{2}]$, there is a unique closest  point of $x$ to $K$ (called the \emph{footpoint of $x$ on $K$}, 
%denoted here by $f_K(x))$.  
Moreover, there is a unique geodesic from $x$ to $f_A(x)$, and this geodesic depends continuously on $x$.  \end{subthm}

%\begin{subthm}{retract-to-convex}
%If $\kappa\le 0$, $K$ is a short retract under $f_K$  of $\cBall[K,\tfrac{\varpi\kappa}{2}]$; i.e. the footpoint map  
%\[f_K\:\cBall[K,\tfrac{\varpi\kappa}{2}]\to K
%\]
%is a short map that is identical on $K$.

%Moreover $K$  is a strictly short retract under $f_K$ of $\oBall[K,\tfrac{\varpi\kappa}{2}]$; i.e. 
%for any $x\in K$ and $y\notin K\cap\oBall[K,\tfrac{\varpi\kappa}{2}] $, we have 
%\[\dist{f_K(y)}{x}{}<\dist{y}{x}{}.\]
%\end{subthm}

\begin{subthm}{distance-to-convex}
Set
$f=\sn{\kappa}\circ\dist{A}{}{}.$
Then the  restriction of $f$ to  $\oBall[A,\tfrac{\varpi\kappa}{2}]$  satisfies \[f''+\kappa\cdot f\ge 0%\eqlbl{eq:dist-convex}
 .\]
In particular, if $\spc{U}$ is a complete  length space  with  $\curv\spc{U}\le0$, and $A\subset\spc{U}$  is  totally convex, then the  function
$f= \dist{A}{}{}$
is convex on $\spc{U}$.
\end{subthm} 

\begin{subthm}{short-retract} 
If $\kappa\le 0$,
%$A\subset \oBall(p,\tfrac{\varpi\kappa}{2})$, 
then $f_A$ is a strictly short retraction of $ \spc{U}$ onto $A$,
i.e. $f_A$ is a short map
that is identical on $A$, such that for any $x\in A$ and $y\notin A$ we have 
\[\dist{x}{f_A(y)}{}<\dist{x}{y}{}.\]
 % If $\kappa \le 0$, then $f_A$ is a short retraction of $\spc{U}$ onto $A$.
 \end{subthm}
\end{thm}
\parit{Proof; (\ref{SHORT.footpoint}).}
By the Lifting globalization theorem  \ref{thm:globalization-lift}, for any  $p\in\spc{U}$
 there is a space $\spc{B}_p\in\Cat{}{\kappa}$, where $\spc{B}_p=\oBall[\hat p,\varpi\kappa/2]$ for some $\hat p \in \spc{B}$,
and a locally isometric map $\map_p\:\spc{B}_p\to\spc{U}$
with $\map(\hat p)=p$ and the following lifting property: 
for any curve $\alpha\:[0,a]\to\spc{U}$ with $\alpha(0)=p$ and $\length\alpha\le\varpi\kappa/2$,
there is a unique curve $\hat\alpha \:[0,a]\to \spc{B}_p$ such that $\hat\alpha (0) = \hat p$ and $\map_p\circ\hat\alpha=\alpha$.

%Let $\map\: \hat {\spc{U}}\to\spc{U}$
%be the simply connected metric covering.
%By the Hadamard--Cartan theorem
%(\ref{thm:hadamard-cartan}), $\hat {\spc{U}}\in\Cat{}{0}$.  The map $\map$ is a local isometry and has  the lifting property, i.e.
%for any $x \in \spc{U}$ and $\hat x \in \hat {\spc{U}}$ such that $\map (\hat x) = x$, and any path $\alpha\:[0,1]\to\spc{U}$ such that  $\alpha(0)=x$, 
%there is a unique path $\hat\alpha \:[0,1]\to \hat{\spc{U}}$ such that $\hat\alpha(0)=\hat x$ 
%and $\map\circ\hat\alpha=\alpha$.

Fix $p\in \oBall[A,\tfrac{\varpi\kappa}{2}]$. 
Set $\hat A_p= (\map_p)^{-1}(A)$.

If $\hat\gamma$ is a geodesic of $\spc{B}_p$  with endpoints in $\hat A_p$, then $\length \hat\gamma < \varpi\kappa$, and $\map_p\circ\hat\gamma$ is a local geodesic of the same length in $\spc{U}$ with endpoints in $A$.  Since  $A\subset\spc{U} $  is totally $\varpi\kappa$-convex,  $\map_p\circ\hat\gamma$ lies in $A$.  Therefore $\hat\gamma$ lies in $\hat A_p$, and   $\hat A_p$ is a convex subset of  $\spc{B}_p$.  

Since $\map_p$ is a local isometry,  $\dist{\hat A_p}{\hat p}{\spc{B}_p} \ge \dist{A}{p}{\spc{U}}$.  Then the lifting property of $\map_p$ implies $\dist{\hat A_p}{\hat p}{\spc{B}_p} = \dist{A}{p}{\spc{U}}$.  Moreover, a curve $\hat\alpha$ realizes distance from $\hat p$ to $\hat A_p$ if and only if  $\alpha = \map_p\circ\hat\alpha$
 realizes distance from $p$ to $A$.

Let  $\spc{W}=\spc{B}_p
\sqcup_{\iota}\spc{B}_p$ be the doubling of $\spc{B}_p$ with respect to  $\hat A_p$, where  $\iota\:\hat A_p\to\spc{B}_p$ is the inclusion map, i.e. $\spc{W}$ is the result of gluing two copies of $\spc{B}_p$ along $\hat A_p$  (\ref{sec:doubling}).  Then $\spc{W}\in\Cat{}{\kappa}$ by Reshetnyak gluing (\ref{thm:gluing}).


Given $w=(\hat p,1)\in \spc{W}$, set $w'=(\hat p,- 1)$.  Then there is a unique geodesic $\ddot\gamma_p$ in $\spc{W}$ joining $w$ and $w'$.  This geodesic must intersect $\hat A_p\subset \spc{W}$, and since there is no shorter curve in $\spc{W}$ that joins $w$ and $w'$, must consist of two geodesic segments $(\hat \gamma_p,1)$ and $(\hat\gamma_p, -1)$ where $\hat \gamma_p$ is a geodesic in $\spc{B}_p$  that realizes distance from $\hat p$ to  $\hat A_p$.  Uniqueness of $\ddot\gamma_p$ implies uniqueness of $\hat \gamma_p$.  It follows from the lifting property  that there is a unique closest point of $A$ to $p$, and  $\gamma_p=\map_p\circ\hat\gamma_p$ is the unique geodesic of $\spc{U}$ that realizes distance from $p$ to $A$.

Suppose $x\in \oBall[A,\tfrac{\varpi\kappa}{2}]$ satisfies $ \dist{p}{x}{\spc{U}}=\eps>0$ and lies in a $\Cat{}{\kappa}$ 
 neighborhood $\Omega$ of $p$ for which $\map_p|\Omega$ is an isometry. For sufficiently small $\eps $,  the unique minimizer $\gamma_x$ from $x$ to $A$ lies in $ \oBall[p,\tfrac{\varpi\kappa}{2}]$.    Let $\alpha\:[0,a]\to\spc{U}$ be the unitspeed curve that runs first  along  the geodesic in $\Omega$  from $p$ to $x$ and then along $\gamma_x$ to $A$. Let $\hat\alpha_{p,x}$ be the curve in $\spc{B}_p$ such that  $\hat\alpha (0) = \hat p$ and $\map_p\circ\hat\alpha=\alpha$.  Set $\hat x =\hat\alpha_{p,x} (\eps)$.  Then the curve
\[
\hat\gamma_{p,x} =\hat\alpha_{p,x} |[\eps, a]
%\eqlbl{eq:d-realizer}
\]
 is a geodesic of $\spc{B}_p$ that  realizes distance from $\hat x$ to  $\hat A_p$.  
Since  $\map_p\circ\hat\gamma_{p,x}$ is a reparametrization of $\gamma_x$,
\[
\dist{\hat A_p}{\hat x}{\spc{B}_p}= \dist{A}{ x}{\spc{U}}. 
\eqlbl{eq:d-realizer}
\]
 
Since geodesics of  $\spc{W}$ vary continuously with their endpoints (\ref{lem:cat-unique}), then $\hat\gamma_{p,x}$ varies continuously with $x$.
Hence (\ref{SHORT.footpoint}).\qeds

%Suppose there are two closest points $z_1,z_2$ of $\hat A_x$ to $\hat x$.  By point-on-side comparison (\ref{cat-monoton}), the distance from $\hat x$  to an interior point of $[z_1z_2]$ is $<\dist{\hat x}{z_i}{}$. This contradiction shows there is a unique closest point of $\hat A_x$ to $\hat x$.  It follows that there is a unique closest point of $A$ to $x$. 

\parit{(\ref{SHORT.distance-to-convex}).} 
%It suffices to verify the differential inequality $(f\circ\gamma)''+\kappa\cdot (f\circ\gamma)\ge 0$ for any geodesic 
For any $p\in \oBall[A,\tfrac{\varpi\kappa}{2}]$, set $ p^*=f_A( p )$. 


Suppose  $\beta\:[0,\eps]\to\spc{U}$ is a unit-speed geodesic  in $\oBall[A,\tfrac{\varpi\kappa}{2}]$ from $p$ to $x$.  
We wish to examine the distances $\dist{\beta (t)}{ A}{\spc{U}}$.
By \ref{eq:d-realizer}, we may assume $\spc{U}\in\Cat{}{\kappa}.$

Extend $\beta$ to $\alpha\:[0,a]\to\spc{U}$, so that $\alpha[0,\eps]=\beta$  and $\alpha$  is a closed unit-speed curve that parametrizes 
the  quadrilateral $Q$ with sides   $\alpha([0,\eps]) =[px]$, $[xx^*]$, $[x^* p^*]$, $[p^*p]$.
 For $\eps$ sufficiently small, $\length Q < 2\varpi\kappa$ by continuity of $f_A$, and we may  apply Reshetnyak majorization (\ref{thm:major}) to  $\alpha$.  

Thus $\alpha$ is majorized by a closed convex region $D$  lying in a hemisphere $\Lob2\kappa$, where $D$ is bounded by a simple closed unit-speed curve $\tilde\alpha\:[0,a]\to\Lob2\kappa$.  By definition the majorizing map $F\:D\to \spc{U}$ is length-nonincreasing and  we may take $F\circ\tilde\alpha=\alpha$. It follows that  $\tilde\alpha$ parametrizes a convex quadrilateral $\tilde Q$ having  the same sidelengths as $Q$.  Let $\tilde\sigma$ denote the side of $\tilde Q$ opposite $\tilde\alpha([0,\eps])$.

By majorization, for $0\le t\le\eps$,
$$\dist{\tilde\alpha (t)}{ \tilde\sigma}{\Lob2\kappa}\ge  \dist{\beta (t)}{ y_t}{\spc{U}}\ge \dist{\beta (t)}{ A}{\spc{U}},$$
 where $y_t\in A$ is a point on $[p^* x^*]$.  


By construction, these inequalities are equalities  at
$t=0$ and $\eps$. 
Since $\dist{\tilde\alpha (t)}{ \tilde\sigma}{\Lob2\kappa}$, $0\le t\le \eps$, satisfies the desired convexity condition by Lemma \ref{lem:model-d-seg},
then
(\ref{SHORT.distance-to-convex}) follows.\qeds

\parit{(\ref{SHORT.short-retract}).}
For any $x\in \spc{U}=\oBall[A,\tfrac{\varpi\kappa}{2}]$, set  $x^*=f_A( x )$. 
Suppose  $x,y\in\spc{U}$.

Clearly $x=x^*$ if and only if $x\in A$.
Further, if $x\notin A$ then 
\[\mangle\hinge{x^*}{x}{p}\ge\tfrac\pi2\eqlbl{eq:<x*xp>=pi/2}\] 
for any $p\in A$;
otherwise there would be a point on $[x^*p]\subset A$ which is closer to $x$ than $x^*$.

Let us show that $\dist{x^*}{y^*}{}\le\dist{x}{y}{}$ for any $x,y\in \spc{U}$.
Without loss of generality we assume $x^*\not=y^*$ and $x\notin A$.

\parit{Case 1:} $y\in A$, so $y=y^*$.
From \ref{eq:<x*xp>=pi/2}, we have $\mangle\hinge{x^*}{x}{y^*}\ge\tfrac\pi2$.
From ??? comparison it follows that $\dist{x^*}{y^*}{}=\dist{x^*}{y}{}\le \dist{x}{y}{}$.

%%%ADD PIC

\parit{Case 2:} $x,y\notin A$.
In this case, \ref{eq:<x*xp>=pi/2} implies $\mangle\hinge{x^*}{x}{y^*}$, $\mangle\hinge{y^*}{x}{x^*}\ge\tfrac\pi2$.
Apply Reshetnyak majorization to the quadrilateral $Q$ with sides   $[xy]$, $[yy^*]$, $[y^* x^*]$, $[x^*x]$. Since a majorizing map cannot increase angles, there is a quadrilateral in  $\Lob2\kappa$ with the same sidelengths as $Q$, and whose angles at the vertices corresponding to $x^*$ and $y^*$ are $\ge\tfrac\pi2$.  It follows that 
$\dist{x^*}{y^*}{}\le \dist{x}{y}{}$.\qeds

  
\begin{thm}{Theorem on short retract, $\kappa>0$}
\label{strictly-short-retract} 
Suppose $\spc{U}\in\Cat{}{\kappa}$ where $\kappa>0$.  
Let $A\subset\spc{U} $  be a closed convex subset.
Assume $A\subset \cBall[p,\varpi\kappa]$ for some $p\in \spc{U}$.
Then $A$ is a \emph{short retract}\index{short retract} of $\spc{U}$;
i.e. there is a short map $\map[2]\:\spc{U}\to A$ which is identical on $A$.

More over if $A\subset \oBall(p,\tfrac{\varpi\kappa}{2})$, 
then the map $\map[2]$ can be chousen so that in addition 
\[\dist{\map[2](y)}{x}{}<\dist{y}{x}{}.\]
 for any $x\in A$ and $y\notin A$.
\end{thm}


\parit{Proof.} 
Applying rescaling, we can assume  $\kappa=1$.
Without loss of generality, we may assume that $p\in A$.

If $\dist{A}{x}{}\ge\tfrac\pi2$ then set $\map[2](x)=p$.

Otherwise, if $\dist{A}{x}{}<\tfrac\pi2$, by ???, 
there is unique point $x^*\in A$ that minimizes distance to $x$;
i.e. $\dist{x^*}{x}{}=\dist{A}{x}{}$.
In this case set 
\begin{align*}
\ell_x&=\dist{p}{x^*}{},
\\
\phi_x&=\tfrac\pi2-\dist[{{}}]{x^*}{x}{},
\\
\sin\psi_x&=\sin\phi_x\cdot\sin\ell_x, 
\ \ 0\le \psi_x\le \tfrac\pi2
\intertext{and define}
\map[2](x)&=\geod_{[px^*]}(\psi_x).
\end{align*}

Note that $\map[2]$ is a retraction to $A$; 
i.e.,
$\map[2](x)\in A$ for any $x\in \spc{U}$
and 
$\map[2](a)=a$ for any $a\in A$.

Let us show that $\map[2]$ is short.
Assume $x,y\in\oBall(A,\tfrac\pi2)$,
\begin{align*}
x'&=\map[2](x)
&
y'&=\map[2](y)
\\
r&=\dist{x}{y}{}
&
r'&=\dist{x'}{y'}{}
\\
d&=\dist{x^*}{y^*}{}
&
\alpha&=\angk1{p}{x^*}{y^*}
\end{align*}

Note that 
\[\cos r\le 
\cos\phi_x\cdot\cos\phi_y
-
\cos d\cdot\sin\phi_x\cdot\sin\phi_y.\eqlbl{eq:cos(r)}\]

Indeed, if $x,y\notin A$,
then 
$\mangle\hinge{x^*}{x}{y*}, 
\mangle\hinge{y^*}{y}{x*}
\ge 
\tfrac\pi2$
and
the inequality~\ref{eq:cos(r)} follows from the Arm lemma (\ref{lem:arm}).
If $x\in A$ and $y\notin A$, we get \ref{eq:cos(r)}, by angle comparison (\ref{cat-hinge}) 
since $\mangle\hinge{y^*}{y}{x*}\ge \tfrac\pi2$.
The same way \ref{eq:cos(r)} is proved 
in case $x\notin A$ and $y\in A$.
Finally, if $x,y\in A$, $\phi_x=\phi_y=\tfrac\pi2$ and $r=d$;
i.e., the inequality trivially holds.

Further note that
\[\cos\alpha
=
\frac{\cos d-\cos \ell_x\cdot\cos\ell_y}{\sin\ell_x\cdot\sin\ell_y}.\]
Applying angle-sidelength  monotonicity (\ref{cor:monoton-cba}) we get
\begin{align*}
\cos r'&\ge
\cos\psi_x\cdot\cos\psi_y
-
\cos \alpha \cdot\sin\psi_x\cdot\sin\psi_y=
\\
&=
\cos\psi_x\cdot\cos\psi_y
-(\cos d-\cos \ell_x\cdot\cos\ell_y)\cdot\sin\phi_x\cdot\sin\phi_y\ge
\\
&\ge \cos\psi_x\cdot\cos\psi_y
-\cos d\cdot\sin\phi_x\cdot\sin\phi_y
\end{align*}


Note that 
$\psi_x\le \phi_x$
and
$\psi_y\le \phi_y$;
in particular,
\[
\cos\phi_x\cdot\cos\phi_y\le \cos\psi_x\cdot\cos\psi_y.
\]
Hence 
\[\cos r'\ge \cos r;\]
i.e., the restriction $\map[2]|\oBall(A,\tfrac\pi2)$ is short.
Clearly $\map[2]$ is continuous,
since the complement of $\oBall(A,\tfrac\pi2)$ is mapped to $p$,
we get that $\map[2]$ is short; i.e.,
\[r'\le r \eqlbl{eq:cos=<cos}\]
for any $x,y\in\spc{U}$.

If we have equality in \ref{eq:cos=<cos}
then 
\[\cos\ell_x\cdot\cos\ell_y\cdot\sin\phi_x\cdot\sin\phi_y=0.\]
If $A\subset \oBall(p,\tfrac\pi2)$ then $\ell_x,\ell_y<\tfrac\pi2$;
which implies that $x\in A$ or $y\in A$.
Without loss of generality we may assume that $x\in A$.

It remains to show that if $y\notin A$ 
then the inequality~\ref{eq:cos=<cos}
is strict.
If $\dist{A}{y}{}\ge\tfrac\pi2$ then \ref{eq:cos=<cos} holds since 
the left hand side is $<\tfrac\pi2$,
while right hand side is $\ge \tfrac\pi2$.
If $\dist{A}{y}{}<\tfrac\pi2$ then $\phi_y>0$ and clearly $\psi_y<\phi_y$,
hence the inequality~\ref{eq:cos=<cos} is strict.
\qeds

We fail to find a transparent geometric proof of the statement above.
Below you will find a geometric way to think about the construction;
compare to the construction 
in the proof of Kirszbraun's theorem (\ref{thm:kirsz+}).

\parit{Geometric interpretation of the map $\map[2]$.}
Set $\mathring{\spc{U}}=\Cone \spc{U}$;
denote by $\mathring{A}$ the subcone of $\mathring{\spc{U}}$ spanned by $A$.
The space $\spc{U}$ can be naturally identified with the unit sphere in $\mathring{\spc{U}}$;
i.e., the set 
\[\set{z\in \mathring{\spc{U}}}{|z|=1}.\]

According to ??? $\mathring{\spc{U}}\in\cCat{}{0}$.
Note that $\mathring{A}$ forms a convex closed subset of $\mathring{\spc{U}}$.
According to ???, for any point $x$ there is unique point $\hat x\in \mathring{A}$
which minimize the distance to $x$;
i.e., $\dist{\hat x}{x}{}=\dist{A}{x}{}$.
(If $|\hat x|\ne0$ then in the notations above we have
$x^*=\tfrac1{|\hat x|}\cdot\hat x$.)

Consider the ray $t\mapsto t\cdot p$ in  $\mathring{\spc{U}}$.
According to ???, %ASK Stephanie???
for given $s\in \mathring{\spc{U}}$
the geodesics $\geod_{[s\ t\cdot p]}$ converge as $t\to\infty$ to a ray, 
say $\alpha_s\:[0,\infty)\to \mathring{\spc{U}}$.



Note that if $|x|=1$ then $|\hat x|\le 1$.
By assumption for any $a\in A$ the function $t\mapsto |\alpha_a(t)|$ is monotonicity increasing.
Therefore there is unique value $t_x\ge 0$ such that
$|\alpha_{\hat x}(t_x)|=1$.
Consider the map $\map[2]\:\spc{U}\to A$
defined as 
\[\map[2](x)=\alpha_{\hat x}(t_x).\]

%%%%%%%%%%%%%%%%%%%%%%%%%%%%%%%%%%%%%%%%%%%%%%%%%%%%%%%%

%%!TEX root = the-defs-CBA.tex
\section{Jacobi lengths}
\label{sec:jacobi-length}

This section gives a characterization of  locally $\CAT\kappa$ spaces that captures the notion in a Riemannian manifold of lengths of normal and tangential Jacobi fields.
Tangential lengths, which are linear, must be discarded in order to identify specific curvature bounds.

This material is from \cite{???} and \cite{???}.  
It is used in \cite{???} to prove a sharp Alexandrov curvature bound above for Riemannian manifolds-with-boundary (\ref{thm:example-mnflds-with-bry:CBA}), and in \cite{} to obtain sharp curvature bounds for subspaces of spaces with curvature $\le\kappa$ (\ref{sec:gauss-equation}).

\begin{thm}{Definition}
Let $[pq]$ 
be a geodesic in an intrinsic space $\spc{X}$
and $\ell=\dist{p}{q}{}$.
A function $f\:[0,\ell)\to\RR_{\ge0}$ is called \emph{normal Jacobi length} along $[pq]$
if there is a sequence of geodesics $[pq_n]$ in $\spc{X}$
such that $q_n\to q$ as $n\to\infty$ and
\[f(t)
=
\lim_{n\to\infty}\frac{\dist{\geod_{[pq_n]}(t)}{\geod_{[pq]}(t)}{}}{\dist{q_n}{q}{}}.\]

The geodesic $[pq]$ 
is said to satisfy \emph{Jacobi splitting} if 
for any normal Jacobi length $f$ along $[pq]$, 
the sequence of geodesics $[pq_n]$ as above we have
\[\dist[2]{\gamma(t)}{\gamma_n(t_n)}{}=
f_n(t)^2\cdot\dist[2]{q}{q_n}{}
+
|t-t_n|^2
+
o(\dist[2]{q}{q_n}{}+|t-t_n|^2).\]
\end{thm}


\begin{thm}{Theorem}\label{thm:jacobi-length}
Let $\spc{U}$ be a locally geodesic metric space. 
Then $\spc{U}$ is locally $\CAT\kappa$ if and only if for any $p\in\spc{U}$
there is a neighborhood $N\ni p$
such that for any geodesic $[pq]$ in $N$ the following two conditions hold

\begin{subthm}{jacobi-split} 
The geodesic $[pq]$ satisfies Jacobi splitting;
\end{subthm}

\begin{subthm}{jacobi-convex}
Any normal Jacobi length $f$ along $[pq]$ 
satisfies the following differential inequality
\[f''+\kappa\cdot f\ge 0\]
in the sense deascribed in ???
 \end{subthm}
\end {thm}

\begin{thm}{Lemma}\label{lem:model-jacobi}
Normal Jacobi length $f$ for any geodesic with lenght $<\varpi\kappa$ in the model space $\Lob2\kappa$
satisfies
$$f'' +\kappa\cdot f=0.$$
  
\end{thm}

\begin{thm}{Lemma}
Let $\gamma_0,\gamma_1\:[a,b]\to\Lob{}\kappa$, 
be two geodesics such that
\[
\dist{\gamma_0(a)}{\gamma_0(b)}{}+ \dist{\gamma_0(b)}{\gamma_1(b)}{}+ \dist{\gamma_1(b)}{\gamma_1(a)}{} +  \dist{\gamma_1(a)}{\gamma_0(a)}{} <2 \varpi\kappa.
\]
Set $f(t)=\dist{\gamma_0(t)}{\gamma_1(t)}{}$.
Then 
\[f''+K\cdot f\ge 0\]
where
\[
K=\min\{0,-\kappa\cdot\bigl(\varpi\kappa/(b-a)\bigr)^2\}.
\]
 \end{thm} 

\begin{thm}{Definition (Jacobi length)}\label{def:jac}
Let $\spc{X}$ be an intrinsic metric space, and $\gamma:\II\to\spc{X}$ be a geodesic, where $\II$ is an interval. Then $f:\II\to [0,\infty)$  is called a \emph{Jacobi  length along $\gamma$} if there is a sequence of 
geodesics $\gamma^i:\II\to\spc{X}$ and a sequence of positive numbers $u_i\to 0$ such that for all $s\in \II$,
\[
f(s) = \lim_{i\to\infty} \  \dist{\gamma(s)}{\gamma^i(s)}{}/u_i.
\eqlbl{eq:jac-length}
\]
If moreover
\[
\dist{\gamma^i(t)}{\gamma(t)}{}= \dist{\gamma^i(t)}{\gamma(\II)}{} + o(u^i),\]
then $f$ is called a \emph{normal Jacobi  length along $\gamma$}.
\end{thm}

Given  a point  $p$ and a curve $\alpha\:[0,1]\to \spc{X}$ of constant speed $A$, recall that a map
\[
[0,1]\times[0,1]\to\spc{X}\:(t,s)\mapsto\gamma_t(s)
\] 
is a \emph{line-of-sight map from $p$ to $\alpha$} 
if each curve $\gamma_t$ is a geodesic from $p$ to  $\alpha(t)$ (\ref{def:sight}). 

\parbf{Example.}  
A sinusoidally $K$-convex function $f:[a,b]\to [0,\infty)$ must be continuous on $(a,b)$.
However, a sinusoidally $\kappa$-convex Jacobi length $f:[a,b]\to [0,\infty)$  in a $\CAT\kappa$ space may be discontinuous at the endpoints. 
Consider the complement $\spc{U}$  in  $ \EE^2$ of an open disk, and  a 
geodesic $\gamma\: [0,1]\to\spc{U}$  which lies on the bounding circle.  
A line-of-sight  map is defined by lifting the endpoint $\gamma(1)$ along an ``evolute''  curve of speed $1$ running at constant \,$\spc{U}$-distance from $\gamma(0)$. 
There is a  corresponding Jacobi length $f$ along $\gamma$, where   $f(s)=0$ for $0\le s <1$ \,and\,  $f(1)=1$.

%\parit{Proof of Lemma \ref{lem:jac-exist}.} 
%By assumption, 
%$$f_i(s)\,\le\,(\,\dist{\alpha(t)}{\alpha(t_i)}{}   /\sn{K}1\,)\,\cdot\, \sn{K}s\,+\, o(t-t_i).$$
% Therefore for fixed $s$, any limiting value
%of $f_i(s)/(t-t_i)$  as $i\to\infty$ is  at most equal to\, 
%$ A\cdot(\sn{K}s) /\sn{K}1).$
%Now that infinite limits are ruled
%out, we can use a diagonal process to pass to a subsequence of the $\gamma_i$ such that $f(s)<\infty$ is defined on a countable dense subset $S$ of $[0,1]$.

%To show that $f$ exists and is continuous on $[0, 1)$, it suffices to show, for
%fixed $s_0\in [0, 1)$, that any limiting value \,$C$\, of $f_i(s_0)/(t-t_i)$ is equal to any limiting value $D$ of $f(s_i)$, as $s_i\in S$ converges  monotonically to $s_0$.

%Since the $K$-sinusoid  with values
%$f(s_1)$ and $C$ at $s_1$ and $s_0$, respectively, dominates $f(s_i)$ for all $i$, then $D\le C$. 

%When $s_0=0$, we have $D=C$ since  $C=0$.  
%Suppose $s_0\in (0,1)$.  Take $ \bar s\in S$ on the opposite side of $s_0$ from the monotonically converging $s_i$. The $K$-sinusoid ${\bar f}_i$ with values $f(s_i)$ and $f(\bar s)$ at $s_i$ and $\bar s$, respectively, satisfies ${\bar f}_i(s_0) \ge C$ for all $i$.
%Hence $D \ge C$.\qeds

\begin{thm}{Definition}\label{def:line-of-sight}
Let $\spc{X}$ be an intrinsic metric space.
A line-of-sight map 
is said to \,\emph{satisfy  Jacobi splitting}\,  if
the following holds:

For each $t\in[0,1)$ there is a sequence  $t_i\to t^+$, and   a  Jacobi length $f$  along $\gamma$ given by \ref{eq:jac-length}, 
%and continuous on $[0,1)$, 
where $\gamma=\gamma_t$ , $\gamma_i=\gamma_{t_i}$ and $u_i=t_i-t$. In addition, setting \,$\ell=\length\gamma$\, and\, $\ell_i=\length\gamma_i$, for $s\in (0,1)$  set
\[
g_i(s)= 
\dist{\gamma(s)}{\gamma_i(
\ell\cdot
{\ell_i}^{-1}
\cdot s)}{}.
\]
Then, passing to a subsequence, 
the \emph{normal Jacobi length} $f_N$ and \emph{tangential Jacobi length} $f_T$  defined on $(0,1)$ by 
\[
f_N(s)\, = \,\lim_{i\to \infty} \bigl( g_i(s)/(t-t_i)\bigr),\ \ \ 
f_T(s) \,= \, \lim_{i\to\infty}\bigl(|\ell - \ell_i|\cdot s\,/(t-t_i)\bigr)
\eqlbl{eq:normal-tangent-jacobi-length}
\]
exist and satisfy
\[
f^2 = {f_N}^2 + {f_T}^2.\\
\eqlbl{eq:jacobi-split}
\]
%If  there is no sequence $t_i\to t^+$ such that $\ell_i\ge\ell$, then $g_i(1)$ is undefined, and we define $f_N(1)$ by \,\ref{eq:jacobi-split}; that is, $f_N(1)^2=f(1)^2-f_T(1)^2$.
\end{thm}

Recall that $\geodpath_{[xy]}$ denote a geodesic path from $x$ to $y$.



%\begin{thm}{Lemma}\label{lem:model-jacobi}
%For any $p\in\Lob{}\kappa$ and any triangle $\trig {\tilde x}{\tilde y}{\tilde z}$ in $\oBall(p,\tfrac{\varpi\kappa}{2})$, the line-of-sight map from $\tilde x$ to $\geodpath_{[\tilde y\tilde z]}$ satisfies Jacobi splitting.  Moreover,   for each $t\in [0,1)$ the corresponding  normal Jacobi length  is a $(\kappa\cdot\ell^2)$-sinusoid.
%let $\gamma$ and $\sigma$ be geodesics of length $\le\varpi\kappa$ and  speed $\le 1$.  Then $\dist{\gamma(s)}{\sigma_i(s)}{}$ is sinusoidally $\hat\kappa$-convex where  $\hat\kappa=\max\{0,\kappa\}$.


\begin{thm}{Theorem}\label{thm:jacobi-length}
Let $\spc{U}$ be a locally geodesic metric space.
Then $\spc{U}$ is locally $\CAT\kappa$ if and only if for any $p\in\spc{U}$ and any triangle $\trig {x}{y}{z}$ in some open ball about $p$:

%\begin{subthm}{jacobi-lip}
%line-of-sight maps are Lipschitz; 
%\end{subthm}

%\begin{subthm}{first-var}
%first variation equation holds from $x$ to $\geodpath_{[yz]}$;\end{subthm}

\begin{subthm}{jacobi-split} 
the line-of-sight map from $x$ to $\geodpath_{[yz]}$
%is Lipschitz and 
satisfies   Jacobi splitting;
\end{subthm}

\begin{subthm}{jacobi-convex}
each corresponding normal Jacobi length is  sinusoidally $(\kappa\cdot\ell^2)$-convex,
where $\ell$ denotes ???.
 \end{subthm}

\end {thm}

\parit{Proof of ``only if''.} 
%\parit{($\Rightarrow$)} 
Suppose $\spc{U}$ is locally $\CAT\kappa$. 
For any $p\in\spc{U}$, there is a convex ball 
$\oBall(p,R)$ which is $\CAT\kappa$ and $R<\tfrac{\varpi\kappa}{2}$ (\ref{cor:convex-balls}). 
Choose the triangle $\trig {x}{y}{z}$ to lie in $\oBall(p,R)$.

%\parit{(\ref{SHORT.first-var}).}  (\ref{SHORT.first-var}) holds  by \ref{thm:1st-var=cba}. 
 
\parit{(\ref{SHORT.jacobi-split}).}
We use the notation of Definition  \ref{def:line-of-sight}. The line-of-sight map from $x$ to $\geodpath_{[yz]}$ is Lipschitz by  triangle comparison. Therefore for each $t\in[0,1]$ there is a sequence  $t_i\to t^+$, and   a  Jacobi length $f$  along $\gamma$ given by \ref{eq:jac-length}.

\begin{clm}{}\label{right-angles}
For $s\in (0,1)$
% and also for $s=1$ in case we may choose $t_i\to t^+$ such that $\ell_i\ge\ell$
and $i$ sufficiently large, the angles
%set 
%\[
% \mangle^\gamma_i(s) =\mangle\hinge {\gamma( s) }{p}{ \gamma_i(\ell\cdot
%{\ell_i}^{-1}\cdot s)}\,, \ \ \ 
% \mangle_i(s) =\mangle\hinge {\gamma_i ( \ell\cdot
%{\ell_i}^{-1}\cdot s)}{p}{ \gamma(s)}.
%%\eqlbl{eq:jacfd2}
%\]
\[
\mangle\hinge {\gamma( s) }{\gamma(1)}{ \gamma_i(\ell\cdot
{\ell_i}^{-1}\cdot s)}\,,  \ \ \ 
 \mangle\hinge {\gamma_i ( \ell\cdot
{\ell_i}^{-1}\cdot s)}{\gamma_i(1)}{ \gamma(s)}.
\eqlbl{eq:jacfd}
\]
converge to $\pi/2$, as do the angles 
\[
  \mangle\hinge {\gamma( s) }{\gamma(1)}{ \gamma_i(\ell\cdot
{\ell_i}^{-1}\cdot s)}\,,  \ \ \ 
 \mangle\hinge {\gamma_i ( \ell\cdot
{\ell_i}^{-1}\cdot s)}{\gamma_i(1)}{ \gamma(s)}.
\eqlbl{eq:jacfd1}
\]
%
%\[
%\mangle\hinge {\gamma( s) }{p}{ \gamma_i(\ell\cdot
%{\ell_i}^{-1}\cdot s)}\,, \ \ \ 
% \mangle\hinge {\gamma_i ( \ell\cdot
%{\ell_i}^{-1}\cdot s)}{p}{ \gamma(s)}.
%\eqlbl{eq:jacfd2}
%\]
(If $g_i$ vanishes on an
initial interval, we set the undefined angles to be $\pi/2$.)

\end{clm}

By majorization (\ref{thm:major}), since $g_i$ is  sinusoidally $K$-convex for some $K$, hence  semiconvex.  Therefore $g_i$  has derivatives almost everywhere and one-sided derivatives everywhere.   Since $g_i\to 0$, it follows that $g_i^{\pm}\to 0$ 
%converge uniformly to $0$ on $[0,1-\epsilon)$ 
(Corollary \ref{lem:der-conv-lim}).

 By the first variation equation \ref{cor:both-end-first-var-cba}, 
 \[g_i^-
= -\cos \mangle^{\gamma}_i - \cos \mangle_i.
\]
By convexity of balls, the angles  \ref{eq:jacfd} are $\le\pi/2$.  
Hence $\mangle^{\gamma}_i (s)$ and $\mangle_i(s)$ approach $\pi/2$ as $i\to\infty$. 

Similarly, since
the angles \ref{eq:jacfd1} are $
\ge \pi/2$ by  the triangle inequality for angles, both angles
approach
$\pi/2$. Hence claim \ref{right-angles}.

\begin{clm}{}\label{curv-jacobi-split}
The line-of-sight map from $x$ to  $\geodpath_{[yz]}$ 
%is Lipschitz and 
satisfies Jacobi splitting.\end{clm}
  
We continue in
 the notation of Definition  \ref{def:line-of-sight}. 
% Passing to a subsequence, we obtain a    Jacobi length $f_N$ along $\gamma|[0,1)$  given by \ref{eq:normal-tangent-jacobi-length}.
Let $s\in (0,1)$.
 
We must prove
\[
f(s)^2 = f_N(s)^2 + s^2\cdot \lim_{i\to\infty} \bigl((\ell -\ell_i)/(t-t_i)\bigr)^2,
%= F(s)^2 + s^2\cos^2\theta,
\]
or equivalently, 
\[
g_i(s)^2 - f_i(s)^2 + s^2\cdot(\ell -\ell_i)^2 = o((t-t_i)^2).
\]

Let $\triangle_i=[{\gamma(s)}\ {\gamma_i( \ell\cdot{\ell_i}^{-1}\cdot s)} \ \gamma_i(s)]$. Then  $\triangle_i$ has sidelengths $g_i(s)$,
$s\cdot (\ell_i -\ell)$,  $f_i(s)$. 
Let $\triangle_i^0=\modtrig 0({\gamma(s)}\,{\gamma_i( \ell\cdot{\ell_i}^{-1}\cdot s)}\,{\gamma_i(s)})$ be the Euclidean model triangle for $\triangle_i$. Let $\mangle_i(s)$ be the angle of $\triangle_i$ at the vertex $\gamma_i( \ell\cdot{\ell_i}^{-1}\cdot s)$ and  $\mangle_i^0(s)$ be the corresponding angle of ${\triangle}_i^0$. Let $\mangle_i^\kappa(s)$ be the corresponding comparison angle in $\Lob{}\kappa$.  
By triangle comparison,
${\mangle}_i^\kappa(s)\ge \mangle_i(s)$. 

By the law of cosines,
\[
g_i(s)^2 - f_i(s)^2 + s^2\cdot(\ell -\ell_i)^2=
g_i(s)\cdot s \cdot (\ell -\ell_i)\cdot\cos\mangle_i^0(s).
\eqlbl{eq:jacfd3}
\]
  

Consider  the quadrilateral $\Box_i= [{\gamma(s)}\,{\gamma_i( \ell\cdot{\ell_i}^{-1}\cdot s)}\,{\gamma_i(s)} \gamma(\ell_i\cdot{\ell}^{-1}\cdot s))]$. 
By majorization (\ref{thm:major}), $\Box_i$ is majorized by a quadrilateral ${\Box}_i^\kappa$ in $\Lob{}\kappa$.
${\Box}_i^\kappa$ has two
adjacent sidelengths agreeing with those of ${\triangle}_i^\kappa$,
and the diagonal joining their endpoints  is $\ge f_i(s)$, which is the third sidelength of
${\triangle}_i^\kappa$. Therefore the included angle $\mangle_i^{\Box^\kappa}$ satisfies 
\[
\mangle_i^{\Box^\kappa}\ge \mangle_i^\kappa(s) \ge \mangle_i(s).
\eqlbl{eq:jacfd4}
\]

All four angles of $\Box_i^\kappa$
majorize angles that become, by claim \ref{right-angles}, arbitrarily close to $\pi/2$ as $i\to\infty$.   Therefore all
angles of $\Box_i^\kappa$, including $\mangle_i^{\Box^\kappa}$, become
arbitrarily close to $\pi/2$ as $i\to\infty$.  By \ref{eq:jacfd4}, the same is true of  $\mangle_i^\kappa(s)$.  

Thus the expression \ref{eq:jacfd3}, when divided by $(t-t_i)^2$, approaches $0$, since  
\[
 \lim_{i\to\infty} \cos\mangle_i^0(s)=\lim_{i\to\infty} \cos\mangle_i^\kappa(s)=0,
\] 
$\lim_{i\to \infty}\bigl( g_i(s)/(t-t_i)\bigr)=f_N(s)\,$ and \,$\lim_{i\to \infty}\bigl((\ell_i - \ell)\,/(t-t_i)\bigr)=\cos\mangle\hinge {\alpha(t)} {\alpha(1)} {p}$.  This completes the verification of
claim \ref{curv-jacobi-split} and hence of (\ref{SHORT.jacobi-split}).

\parit{(\ref{SHORT.jacobi-convex}).}
As in the proof of (\ref{SHORT.jacobi-split}), the angles of the 
quadrilateral ${\Box}_i^\kappa$ in $\Lob{}\kappa$ become arbitrarily close to $\pi/2$ as $i\to\infty$.   
By Reshetnyak majorization and Lemma \ref{lem:model-jacobi}, it follows that
$f_N|[0,1)$ is sinusoidally $(\kappa\cdot\ell^2)$-convex.  
\qeds

\parit{Proof of ``if''.} 
Suppose
% (\ref{SHORT.first-var}), 
 (\ref{SHORT.jacobi-split}) and (\ref{SHORT.jacobi-convex}) hold for any triangle $\trig {x}{y}{z}$ in some open ball about each point of $\spc{U}$.
We must show that $\spc{U}$ is locally $\CAT\kappa$.

Let  $\tilde \alpha$ be  the $\kappa$-development based at $\tilde x$ in $\Lob2\kappa$ of $\geodpath_{[yz]}$ with respect to $x$.  We thus obtain a comparison figure in $\Lob2\kappa$, two of whose sides, $[\tilde x\tilde y]$ and $[\bar x \bar z]$, are geodesics of the same length as $[xy]$ and $[xz]$ respectively, and whose third side is $\tilde \alpha$. Since a model triangle for $\trig {x}{y}{z}$  may be obtained by replacing  $\tilde \alpha$ with a geodesic of the same length, thereby increasing the angle at $\tilde x$,  it suffices to show \,$\mangle\hinge {x}{y}
{z}\le\mangle\hinge {\tilde x}{\tilde y}{\tilde z}$.

Let $y^\prime$ and $z^\prime$ be two points  different from $x$ and lying on $[xy]$  and $[ x z]$  respectively.  Choose $y^\prime$ and $z^\prime$  sufficiently close to $x$ that a geodesic $\tilde \beta$ joining the corresponding points $\tilde  y^\prime$ and $\tilde  z^\prime$ on $[\tilde x\tilde y]$ and $[\bar x \bar z]$  respectively remains within the subgraph of $\tilde \alpha$. 

Consider a Lipschitz line-of-sight map from $x$ to $\geodpath_{[yz]}$ as in (\ref{SHORT.jacobi-split}) and  (\ref{SHORT.jacobi-convex}). By  definition of development, there is a corresponding line-of-sight map from $\tilde x$ to $\tilde \alpha$, whose varying sight-geodesics we denote by $\tilde \gamma_t$.    Then $\tilde \beta$  intersects transversely each $\tilde \gamma_t$ at a point $\tilde \gamma_t(s)$, and $s=s(t)$ is $C^1$. It follows that the corresponding curve  $\beta(t)=\gamma_t(s(t))$   in $\spc{U}$ is Lipschitz. Therefore $\beta$ has speed almost everywhere  and its length is obtained by integrating its speed. 
%Now fix a value of $t$ at which $\beta$ is differentiable. 

By (\ref{SHORT.jacobi-split}), there is a Jacobi length $f$ along \,$\gamma_t$\, 
that arises from varying $t$ in the positive direction, and satisfies Jacobi splitting $f^2 = {f_N}^2 + {f_T}^2$.  Let $\tilde f$ be the corresponding Jacobi length.  By Lemma \ref{lem:model-jacobi}, \,$\tilde f^2 = {\tilde f_N}^2 + {\tilde f_T}^2$, where $\tilde f_N$ is a  $(\kappa\cdot\ell^2)$-sinusoid. By (\ref{SHORT.jacobi-convex}), $f_N$ is sinusoidally $(\kappa\cdot\ell^2)$-convex. By definition of development, it follows that $f_N\le\tilde f_N$, and moreover $f_T=\tilde f_T$.  Thus the speed of $\beta$ at $t$  is no greater than the corresponding speed of $\tilde \beta$.   Therefore $\length \beta\le\length\tilde \beta$, hence $\dist{y^\prime}{z^\prime}{}\le \dist{\tilde y^\prime}{\tilde z^\prime}{}$.  By definition of angle, $\mangle\hinge {x}{y}
{z}\le\mangle\hinge {\tilde x}{\tilde y}{\tilde z}$, as required.\qeds

\begin{thm}{Corollary}
Suppose  $\spc{X}$ is locally geodesic locally $\CAT{\hat\kappa}$ space for some $\hat\kappa$.
If in some open ball about each point, normal Jacobi lengths satisfy $f''+\kappa f\ge 0$, then $\spc{X}$ is locally $\CAT\kappa$.
\end{thm}

 

%%%%%%%%%%%%%%%%%%%%%%%%%%%%%%%%%%%%%%%%%%%%%%%%%%%%%
%%!TEX root = the-defs-CBA.tex
%%array^

\section{Dual-(1+3)-point comparison.}\label{sec:dual-1+3}



The idea that ``the essence of curvature lies in the general notion of a convex metric space and of a quadruple of points in such a space'' apparently originated with Wald \cite[p.17]{menger:wald}. A metric space has \emph{Wald curvature} $\kappa(p)$ at $p$  if  any quadruple in a sufficiently small neighborhood of $p$ embeds in $\Lob3\kappa$ where $\kappa$ is arbitrarily close to $\kappa(p)$.  Wald proved that any compact geodesic metric space which at every point admits a Wald curvature $\kappa(p)$ is a surface for which $\kappa(p)$ is the Gaussian curvature \cite{wald}. %??? I do not think so, A. 
  
This section gives a $4$-point definition of  $\Cat{}{\kappa}$ (Theorem \ref{thm:alternate-cat-def}) that plays a dual role to our definition of $\Alex{\kappa}$ spaces (Definition \ref{df:cbb1+3}).  
The duality relates to Wald's idea of embedding quadruples in model spaces. 


\begin{thm}{Definition}\label{def:alt-cba}
Let $\spc{X}$ be a metric space.
A quadruple $p,x^1,x^2,x^3$ of points in $\spc{X}$ 
is said to satisfy \emph{dual-(1+3)-point $\kappa$-comparison at the vertex $p$} if
\[
\angk\kappa p{x^i}{x^j}\le \angk\kappa p{x^\kay}{x^i} + \angk\kappa p{x^\kay}{x^j}, 
\]
for any permutation $(i,j,\kay)$ of $(1,2,3)$
or at least one of the model angles $\angk\kappa p{x^i}{x^j}$  is not defined.
\end{thm}

We say the inequality \ref{def:alt-cba} \emph{holds as an equality} if in addition, equality holds for some choice of $i,j$.

Recall the (1+3)-point comparison (\ref{df:1+3}) that we used to define $\Alex{\kappa}$ spaces:
%{df:cbb1+3}

\begin{thm}{Definition}\label{def:cbb} 
In a metric space, a quadruple $Q\:p,x^1,x^2,x^3$ is said to satisfy \emph{(1+3)-point $\kappa$-comparison at   $p$} if 
\[\angk{\kappa} p{x^1}{x^2}+\angk{\kappa} p{x^2}{x^3}+\angk{\kappa}p{x^3}{x^1}\le 2\cdot \pi,
\]
or at least one of the model angles $\angk\kappa p{x^i}{x^j}$  is not defined.
\end{thm}

Conditions \ref{def:alt-cba} and \ref{def:cbb} are dual in the sense that together they force the three model angles at $p$ to be the sidelengths of a triangle in $\Lob3\kappa$, and hence  govern whether a quadruple has an isometric embedding (briefly, \emph{embeds}) in $\Lob3\kappa$:


\begin{thm}{Lemma}
\label{lem:embedding-angles} 
In a metric space, let $Q$ be a  quadruple all of whose model angles are defined.
\begin{subthm}{embed}
If both  (1+3)-point $\kappa$-comparison (\ref{def:cbb}) and dual-(1+3)-point $\kappa$-comparison (\ref{def:alt-cba}) hold at some $p\in Q$, then $Q$ embeds in $\Lob3\kappa$. 
Conversely, if such an embedding exists, then both \ref{def:alt-cba} and \ref{def:cbb} hold at every $p\in Q$. 
\end{subthm}
\begin{subthm}{planar-embed}
If either \ref{def:alt-cba} or \ref{def:cbb} holds as an equality at some $p\in Q$, then 
$Q$ embeds in  $\Lob2\kappa$.  Conversely, if such an embedding exists, then either \ref{def:alt-cba} or \ref{def:cbb} holds as an equality at every $p\in Q$.
\end{subthm}
\end{thm}


\parit{Proof. (\ref{SHORT.embed}).} Suppose conditions  \ref{def:alt-cba} and \ref{def:cbb} hold at $p\in Q$.  Then the three model angles at $p$ are the sidelengths of a triangle $\triangle$ in $\SS^2$.  If  \ref{def:cbb} holds as an equality and one of the model angles at $p$ is $\pi$, there are infinitely many choices of $\triangle$;  in this case, let us choose $\triangle$ to be a great circle.   In any case, for each subtriple of $Q$ containing $p$, we may construct a model triangle in the $\kappa$-cone over the corresponding side of  $\triangle$, where $p$ corresponds to the cone vertex.  The result is an embedding of $Q$ in $\Cone\mc\kappa(\SS^2) = \Lob3\kappa$ (lemma \ref{lem:cos-law}).

Conversely, a quadruple $Q$ in $\Lob3\kappa$ satisfies \ref{def:alt-cba} by the triangle inequality for angles, and  satisfies \ref{def:cbb} because the maximum perimeter of a triangle in $\SS^2$ is $2\cdot\pi$.
 

\parit{(\ref{SHORT.planar-embed}).} Suppose either of the conditions \ref{def:alt-cba} or \ref{def:cbb} holds as an equality at $p\in Q$. Then the other condition also holds since  the maximum value of any angle is $\pi$.  
Then the triangle $\triangle$ of part (\ref{SHORT.embed}) is contained in a great circle $\SS^1\subset\SS^2$, and $Q$ embeds in $\Cone\mc\kappa(\SS^1)= \Lob2\kappa$. 

Conversely, if  $Q$  lies in  $\Lob2\kappa$, then at any $p\in Q$  the three model angles are the sidelengths of a triangle in $\SS^1$.  It follows that one of the two conditions  \ref{def:alt-cba} or \ref{def:cbb} holds as an equality.
\qeds


%Instead of defining $\Cat{}{\kappa}$ in terms of (2+2)-point $\kappa$-comparison, as in definition \ref{def:cat}, the following theorem shows  we could  have used dual-(1+3)-point $\kappa$-comparison instead.  
The following theorem is the dual formulation for $\Cat{}{\kappa}$ of definition \ref{df:1+3} of $\Alex{\kappa}$ spaces. 
The defining condition in  \ref{df:1+3} is that every quadruple  satisfies (1+3)-point $\kappa$-comparison (\ref{def:cbb})  at every vertex $p$.

\begin{thm}{Theorem}\label{thm:alternate-cat-def}
A $\varpi\kappa$-geodesic metric space is $\Cat{}{\kappa}$ if and only if every quadruple  satisfies dual-(1+3)-point $\kappa$-comparison (\ref{def:alt-cba}) at some vertex $p$.
\end{thm}


\begin{thm}{Definition}\label{def:Sigma-Q}
In a metric space, let $Q$ be a  quadruple all of whose model angles are defined.
%, and no pair  $x,y\in Q$ satisfies $\dist{x}{y}{} = \varpi\kappa$.  
Writing $Q:p,x^1,x^2,x^3$, set $\~\Sigma^\kappa(p)= \angk{\kappa} p{x^1}{x^2}+\angk{\kappa} p{x^2}{x^3}+\angk{\kappa}p{x^3}{x^1}$. Set 
\[\~\Sigma^\kappa(Q) = \max_{p\in Q}\~\Sigma^\kappa(p).\]
\end{thm}


%
%In the proof of  theorem \ref{thm:alternate-def}, we say that two triangles in $\Lob2{\kappa}$ with a common vertex $x$  \emph{do not overlap} if the intersection of their convex hulls is $x$.  Our proof  will use the Overlap Lemma (\ref{lem:extend-overlap}).

\parit{Proof of Theorem \ref{thm:alternate-cat-def}.}  
A quadruple with an undefined model angle satisfies dual-(1+3)-point $\kappa$-comparison for some vertex, and  (2+2)-point $\kappa$-comparison (\ref{def:cba}) for all  vertex pairs. Thus it suffices to prove:
\begin{clm}{}\label{2+2-equiv}
In a metric space, let $Q$ be a  quadruple all of whose model angles are defined.\\
$Q$ satisfies dual-(1+3)-point $\kappa$-comparison (\ref{def:alt-cba})
 for some $p\in Q$ 
$\Longleftrightarrow$ \ \ \ \ \\
$Q$ satisfies (2+2)-point $\kappa$-comparison (\ref{def:cba}) for all  $p^1,p^2\in Q$. \end{clm}  

 \parit{ ($\Rightarrow$).}
Suppose dual-(1+3)-point $\kappa$-comparison \ref{def:alt-cba} holds at some $p\in Q$. 
%We must show that (2+2)-point comparison (\ref{def:cba}) holds for all $p^1,p^2\in Q$.  
Write $Q\:p,x^1,x^2,x^3$.  

If $\~\Sigma^{\kappa}(p)\le 2\cdot \pi$, then $Q$ embeds in $\Lob3\kappa$ by lemma \ref{lem:embedding-angles}, and the triangle inequality for angles in $\Lob3\kappa$ implies (2+2)-point comparison.

Suppose $\~\Sigma^{\kappa}(p)> 2\cdot \pi$.  
Since the sum of any two model angles at $p$ exceeds $\pi$, then (2+2)-point comparison holds for any pair $p^1,p^2\in Q$ that includes $p$. 
Now construct on each side $[\~x^j\~x^\kay]$ of the model triangle $\trig{\~x^1}{ \~x^2}{ \~x^3} = \modtrig\kappa(x^1 x^2 x^3 )$, the model triangle $\trig{\~p^i}{\~x^j}{\~x^\kay}=\modtrig{\kappa}{p}{x^j}{x^\kay}$ with $\~p^i$ and $\~x^i$  in the same closed halfspace determined by $[\~x^j\~x^\kay]$.  
By the Overlap Lemma \ref{no-overlap}, no two triangles $\trig{\~p^i}{\~x^j}{\~x^\kay}$ overlap. 
It follows that (2+2)-point comparison  (\ref{def:cba}) holds for any pair $p^1,p^2\in Q$ that does not include $p$.  
For instance, since $\angk\kappa {x^1}{x^3}p<\angk\kappa {x^1}{x^2}{x^3}$, then  \ref{def:cba} holds for $p^1=x^1$, $p^2=x^2$.

\parit{($\Leftarrow$).}
Suppose dual-(1+3)-point $\kappa$-comparison \ref{def:alt-cba}  fails for all $p\in Q$. Then  $\~\Sigma^{\kappa}(Q)\le 2\cdot \pi$. 
Moreover, for each $p\in Q$ we have a labeling $p, x^1,x^2,x^3$ of $Q$ such that 
\[\angk{\kappa} {x^1}{x^3}{x^2}
>
\angk{\kappa} {x^1}{p}{x^2}+\angk{\kappa} {x^1}{p}{x^3}.
\eqlbl{eq:dual-1+3-pf1}\]

Again construct on each side  $[\~x^j\~x^\kay]$ of the model triangle $\trig{\~x^1}{ \~x^2}{ \~x^3} = \modtrig\kappa(x^1 x^2 x^3 )$, a model triangle $\trig{\~p^i}{\~x^j}{\~x^\kay}=\modtrig{\kappa}{p}{x^j}{x^\kay}$, with $\~p^i$ and $\~x^i$  in the same closed halfspace determined by $[\~x^j\~x^\kay]$.  
 By \ref{eq:dual-1+3-pf1}, $\trig{\~p^3}{\~x^1}{\~x^2}$ and $\trig{\~p^2}{\~x^3}{\~x^1}$ do not overlap.
Since $\~\Sigma^{\kappa}(Q)\le 2\cdot \pi$, then by lemma \ref{two-overlap},
\[\angk{\kappa} {p}{x^3}{x^2}>\angk{\kappa} {p}{x^1}{x^2}+\angk{\kappa} {p}{x^1}{x^3}.\eqlbl{eq:dual-1+3-pf2}\]
By \ref{eq:dual-1+3-pf1} and \ref{eq:dual-1+3-pf2},  (2+2)-point comparison fails  for $p^1=p, \,p^2=x^1$.  
 \qeds




%%%%%%%%%%%%%%%%%%%%%%%%%%%%%%%%%%%%%%%%%%%%%%%%%%%%%%%%%%%%%%%%%%%%%%%%%%%%%%%%%%%%%%%%%%%


 
\section{Exercises}

\begin{thm}{Exercise} 
(Gluing with short maps)
 Let $X,X' \in\Cat{}{0}$, $K \subset X, K' \subset X'$ be closed convex subsets, 
and suppose the length-metrics on $\Fr_X K$ and $\Fr_{X'} K'$ are finite (???). 
Suppose $\phi\: \Fr_X K \to \Fr_{X'} K'$ is length-preserving and 
short with respect to the inherited
metrics. 
To $X'\backslash\Int K'$ in its length metric, 
glue $K$ along $\phi$, to obtain the space $Y$. 
Show $Y \in \Cat{}{0}$.

For example, take $K= \cBall[x,R]  \subset \Lob{m}0$ and $K' = \cBall(x',\sinh R)  \subset \Lob{m}{-1}$.
\end{thm}


