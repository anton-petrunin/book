%%!TEX root = all.tex
%array^
\chapter*{Introduction}

The goal of this book is to give a comprehensive description of theory of Alexandrov spaces 
with curvature bounded above and below.
We plan to include all the basic material as well as selected topics inspired by considering the two contexts simultaneously.

Alexandrov geometry can use ``back to Euclid'' as a slogan.
Alexandrov spaces are defined via axioms similar to the one given by Euclid,
but certain  equalities exchanged to inequalities. 
Depending on the sign of inequalities we get Alexandrov spaces with \emph{curvature bounded above} and \emph{curvature bounded below}.
Although the definitions of two classes of spaces are similar, their properties and known applications are qute different.
Our approach is novel in its attention to the interrelatedness of the two fields, and its emphasis on the way each illuminates the other.

We only consider inner theory, leaving most applications aside.


\section*{Brief history}

The first synthetic description of curvature is due to Abraham Wald; 
it was given in a lonely publication on a ``coordinateless description of Gauss surfaces'' published in 1936 \cite{wald}.
In 1941, similar definitions were rediscovered independently by Aleksandr Alexandrov \cite{alexandrov:def}.

In Alexandrov's work the first fruitful applications of this approach were given.
Mainly: \emph{Alexandrov's embedding theorem} which describes closed convex surfaces in Euclidean 3-space,
and the \emph{gluing theorem}, which gave a flexible tool to modify non-negativly curved metrics on a sphere.
These two results together gave  a very intuitive geometric tool to study embeddings and bending of surfaces in Euclidean space, and changed this subject dramatically.
They formed the foundation of branch of geometry now called \emph{Alexandrov geometry}.


\parbf{Curvature bounded below.}
The theory was developed mostly in the two-dimensional case 
by Aleksandr Alexandrov
and his students  
Yurii  Borisov,
Yurii  Burago,
Iosif  Liberman,
Sergey  Olovyanishnikov,
Aleksey  Pogorelov,
Yurii  Reshetnyak,
Yurii  Volkov,
Viktor  Zalgaller and others.
%WE SHOULD WRIE A BIT ABOUT EVERYONE???

Before the 90's  there were only few papers on the higher dimensional Alexandrov spaces.
Namely, Anatoliy Milka started to consider multidimensional Alexandrov's spaces with lower curvature bound in 60's;
in particular, he proved a splitting theorem for such spaces \cite{milka-line}.
His definition was global; it is similar to what we do in this book.
An implicit applications of higher dimensional Alexandrov geometry wes given by Michael Gromov in his bound on Betti numbers \cite{gromov:betti};.
Another implicit application was given by Wu-Yi Hsiang and Bruce Kleiner in their paper on non-negatively curved manifolds with infinite group of symmetries \cite{hsiang-kleiner}.


In the 80's the interest in convergence of Riemannian manifolds created by Gromov's compactness theorem \cite{gomov-precompactness} turned attention toward the singular spaces that can occur as limits of Riemannian manifolds. 
Immediately it was recognized that if the manifolds have a uniform lower sectional curvature bound, then the limit spaces have a lower curvature bound in the sense of Alexandrov. 
There followed throughout the 90's an explosion of work starting with papers of Yurij Burago, Grigori Perelman and Mikhail Gromov  \cite{BGP,perelman:spaces2}.

The above activity was very much related to comparison geometry.
The latter approach is not that maximalistic --- it is a branch of differential geometry which compares Riemannian manifolds  to  spaces of constant curvature.
In addition to the already-mentioned \emph{Gromov's compactness theorem},
the following results had a big influence on the development of Alexandrov's geometry:
\emph{Toponogov comparison theorem} \cite{toponogov-globalization+splitting}, which is a generalization of theorem of Alexandrov \cite{alexandrov-comparison},
\emph{Toponogov splitting theorem} \cite{toponogov-globalization+splitting}, which is a generalizations of Cohn-Vossen's theorem \cite{cohn-vossen_line},
\emph{Finiteness theorems} of
Cheeger and
Grove--Petersen \cite{cheeger-finiteness,grove-petersen:finiteness},
and 
\emph{Yamaguchi fibration theorem} \cite{yamaguchi-fibration}.


Let us give a list of available introductions to  Alexandrov spaces with curvature bounded below: 
\begin{itemize}
\item The first introduction to Alexandrov's geometry is given in the original paper of Yurii Burago, Michael Gromov and Grigori Perelman \cite{BGP} 
and its extension \cite{perelman:spaces2} written by Perelman.
\item A more reader-friendly  introduction was written by Katsuhiro Shiohama \cite{shiohama}.
It is designed to be reader-friendly.  
The first 8 sections can be used as an introduction, but it is a bit brief.
\item \cite[Chapter 10]{BBI} gives another reader-friendly introduction written by Dmiti Burago, Yurii Burago and Sergei Ivanov.
\end{itemize}
In addition, let us mention two surveys, one by Conrad Plaut \cite{plaut:survey} and the other by the third author \cite{petrunin:survey}.

\parbf{Curvature bounded above.}
The study of the spaces with curvature bounded above started later.
The first paper on the subject was written by Alexandrov, appearing in 1951 \cite{alexandrov:strong-angle}.
An analogous weaker definition was considered earlier by Busemann in \cite{busemann-CBA}.

Fundamental results in this direction were obtained by Yurii Reshetnyak.
It includes his majorization theorem and gluing theorem.
The gluing theorem states that if two non-positively curved spaces have isometric convex sets, then the space obtained by gluing these sets along the isometry is also non-positively curved.

Here is a list of available introductions to the Alexandrov spaces with curvature bounded above: 
\begin{itemize}
\item The book of Martin Bridson and André Haefliger \cite{BH} gives the most comprehensive introduction to the geometry of Alexandrov spaces with lower curvature bound which is available today. 
\item The lecture notes of Werner Ballmann \cite{ballmann:lectures} include a brief and ??? introduction to the geometry of Alexandrov spaces with lower curvature bound.
\item \cite[Chapter 9]{BBI} another reader-friendly introduction by Yurii Burago, Dima Burago and Sergei Ivanov.
\item A book of all three authors \cite{AKP-CAT} gives an introduction to $\CAT{0}$ spaces which suppose to demonstrate the beauty and power of Alexandrov geometry by reaching interesting applications and theorems with a minimum of preparation.
\item A book of J\"{u}rgen Jost \cite{jost} gives a more analytic viewpoint on $\CAT{}$ spaces
\end{itemize}

One of the most striking applications was given by Dmitry Burago, Sergei Ferleger and Alexey Kanonenko \cite{BFK};
it use $\CAT0$ spaces to study billiards; this idea was developed further in ???. 
Another beautiful application is the construction of exotic aspherical manifolds by Michael Davis \cite{davis:aspherical}; this idea was developed further in ???.
Both of these topics are discussed in \cite{AKP-CAT}.

\parbf{Satellites and successors.}
Surfaces with bounded integral curvature were studied in Alexandrov's school;
a book on the subjcet was written by Aleksandr Alexandrov and Viktor Zalgaller \cite{aleksandrov-zalgaller}.

A survey on spaces with two-sided bound on curvature in the sense of Alexandrov is written by Valerij Berestovskij, Igor Nikolaev \cite{berestovskii-nikolaev}.

A spin-off of the idea of synthetically defining upper curvature bounds 
was given by Gromov \cite{gromov:hyp-groups}. 
He gave an analogous definition for discrete metric spaces, so called $\delta$-hyperbolic spaces. 
It provided an intuitive geometric tool for studying small cancellation groups.

The so called \emph{curvature dimension condition} introduced by John Lott, Cédric Villani and Karl-Theodor Sturm give a synthetic description of Ricci curvature bounded below; see the book of Villani \cite{villani} and references there
in.

Alexandrov geometry influenced the development of analysis on metric spaces. 
An excellent book on the subject was written by Juha Heinonen, Pekka Koskela, Nageswari Shanmugalingam, and Jeremy Tyson \cite{HKST}.

\section*{Acknowledgment}
We want to thank 
Semyon Alesker,
I. David Berg,
Richard Bishop, 
Yuri Burago, 
Sergei Ivanov,
Bernd Kirchheim, 
Bruce Kleiner, 
John Lott,
Alexander Lytchak, 
Greg Kuperberg, 
Nikolai Kosovsky, 
Nina Lebedeva,
Wilderich Tuschmann and
Sergio Zamora.
%who else?

%Yet special thanks to our non-mathematicician friends and relatives M.~Prelovskaya, J.~Tuschamnn, F.~Champong???; they made for us food, provide place to stay and did not ask stupid questions while this book was written.

We want to thank the mathematical institutions where we worked on this book:
BIRS, 
MFO, 
Henri Poincar\'{e} Institute,
University of Cologne, 
Max Planck Institute for Mathematics.
%what else????



