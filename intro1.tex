%%!TEX root = arXiv.tex
%array^
\chapter*{Preface}


Alexandrov spaces are defined via axioms similar to those given by Euclid.
The Alexandrov axioms replace certain  equalities by inequalities. 
Depending on the signs of the inequalities, we get Alexandrov spaces with {}\emph{curvature bounded above} and {}\emph{curvature bounded below}.
Although the definitions of the two classes of spaces are similar, their properties and known applications are quite different.
Our approach is novel in its attention to the interrelatedness of the two fields, and its emphasis on the way each illuminates the other.


The goal of this book is to give a comprehensive exposition of the structure theory of Alexandrov spaces 
with curvature bounded above and below.
It includes all the basic material as well as selected topics inspired by considering the two contexts simultaneously.
We only consider  the intrinsic theory, leaving applications aside.

Let us note that our presentation is not linear;
sometimes proofs and topics are deferred for later chapters to streamline the exposition and make it more natural.

This book includes material up to the definition of dimension.
Another volume still in preparation will cover further topics.
%Altho we do not follow this rule all the time. %!!! V:I don't understand what this phrase means so I left it out.
%The other book in preparation should cover finite dimensional spaces.


\section*{Brief history}

The first synthetic description of curvature is due to Abraham Wald \cite{wald}; 
it was given in a lone publication on a ``coordinateless description of Gauss surfaces'' published in 1936.
In 1941, similar definitions were rediscovered independently by Aleksandr Alexandrov \cite{alexandrov:def}.

In Alexandrov's work the first fruitful applications of this approach were given.
Mainly: {}\emph{Alexandrov's embedding theorem}, which describes closed convex surfaces in Euclidean 3-space,
and the {}\emph{gluing theorem}, which gave a flexible tool to modify non-negativly curved metrics on a sphere.
These two results together gave  a very intuitive geometric tool to study embeddings and bending of surfaces in Euclidean space, and changed the subject dramatically.
They formed the foundation of the branch of geometry now called {}\emph{Alexandrov geometry}.


\parbf{Curvature bounded below.}
The theory was developed mostly in the two-dimensional case 
by Aleksandr Alexandrov
and his school:  
Yurii  Borisov,
Yurii  Burago,
Iosif  Liberman,
Sergey  Olovyanishnikov,
Aleksey  Pogorelov,
Yurii  Reshetnyak,
Yurii  Volkov,
Viktor  Zalgaller and others.
%WE SHOULD WRIE A BIT ABOUT EVERYONE???

The first result in higher dimensional Alexandov spaces was the splitting theorem.
It was proved by Anatoliy Milka \cite{milka-line}, the paper appeared in 1967;
he used a global definition similar to the one used in this book. %what we do in this book.

In the 80's the interest in convergence of Riemannian manifolds spurred by Gromov's compactness theorem \cite{gromov-MS} turned attention toward the singular spaces that can occur as limits of Riemannian manifolds.
Immediately it was recognized that if the manifolds have a uniform lower sectional curvature bound, then the limit spaces have a lower curvature bound in the sense of Alexandrov. 
There followed throughout the 90's an explosion of work on intrinsic theory of Alexandrov spaces starting with papers of Yurij Burago, Grigori Perelman and Mikhail Gromov  \cite{burago-gromov-perelman,perelman:spaces2}.
Similar ideas were developed independently by Karsten Grove and Peter Petersen, but they were not converted into into a publication.

Around the same time an implicit application of higher-dimensional Alexandrov geometry was given by Mikhail Gromov in his bound on Betti numbers \cite{gromov:betti}.
%{\color{red} V:  should we also mention Gromov's bound on the number of generators of the fundamental group?}
Another implicit application was given latter by Wu-Yi Hsiang and Bruce Kleiner in their paper on non-negatively curved manifolds with infinite symmetry group \cite{hsiang-kleiner}.
The work of Hsiang and Kleiner and its extension by Karsten Grove and Burkhard Wilking \cite{grove-wilking} are some of the most beautiful applications of this branch of Alexandrov geometry.

The above activity was very much related to the so called {}\emph{comparison geometry},
a branch of differential geometry that compares Riemannian manifolds  to  spaces of constant curvature.
In addition to the already-mentioned {}\emph{Gromov's compactness theorem},
the following results had a big influence on the development of Alexandrov's geometry:
{}\emph{Toponogov comparison theorem} \cite{toponogov-globalization+splitting}, which is a generalization of the theorem of Alexandrov \cite{alexandrov-comparison};
{}\emph{Toponogov splitting theorem} \cite{toponogov-globalization+splitting}, which is a generalization of Cohn-Vossen's theorem \cite{cohn-vossen_line};
{}\emph{Finiteness theorems} of
Cheeger
and
Grove--Petersen \cite{cheeger-finiteness,grove-petersen:finiteness};
Gromov's bound on the number of generators of the fundamental group 
\cite[1.5]{gromov:almost-flat};
and 
{}\emph{Yamaguchi fibration theorem} \cite{yamaguchi-fibration}.

%
Let us give a list of available introductory texts on Alexandrov spaces with curvature bounded below: 
\begin{itemize}
\item The first introduction to Alexandrov's geometry is given in the original paper of Yurii Burago, Michael Gromov and Grigori Perelman \cite{burago-gromov-perelman} 
and its extension \cite{perelman:spaces2} written by Perelman.
\item A brief and reader-friendly introduction was written by Katsuhiro Shiohama \cite[Sections 1--8]{shiohama}.
\item \cite[Chapter 10]{burago-burago-ivanov} gives another reader-friendly introduction, written by Dmiti Burago, Yurii Burago and Sergei Ivanov.
\end{itemize}
In addition, let us mention two surveys, one by Conrad Plaut \cite{plaut:survey} and the other by the third author \cite{petrunin:survey}.

\parbf{Curvature bounded above.}
The study of  spaces with curvature bounded above started later,
inspired by analogy with the theory of curvature bounded below.
The first paper on the subject was written by Alexandrov \cite{alexandrov:strong-angle}, appearing in 1951.
An analogous weaker definition was considered earlier by Herbert Busemann \cite{busemann-CBA}.

Fundamental results in this direction were obtained by Yurii Reshetnyak.
This includes his {}\emph{majorization theorem} and {}\emph{gluing theorem}.
The gluing theorem states that if two non-positively curved spaces have isometric convex sets, then the space obtained by gluing these sets along an isometry is also non-positively curved.

The development of Alexandrov geometry was greatly influenced by the {}\emph{Hadamard--Cartan theorem}.
Its original formulation states that the exponential map at any point of a complete Riemannian manifold with nonpositive sectional curvature is a covering.
In particular it implies that the universal cover is diffeomorphic to Euclidean space of the same dimension. 
See further discussion below (\ref{thm:hadamard-cartan}).

Here is a list of available texts covering the basics of Alexandrov spaces with curvature bounded above: 
\begin{itemize}
\item The book of Martin Bridson and Andr\'e Haefliger \cite{bridson-haefliger} gives the most comprehensive introduction available today. 
\item The lecture notes of Werner Ballmann \cite{ballmann:lectures} include a brief 
and clear
introduction.
\item \cite[Chapter 9]{burago-burago-ivanov} another reader-friendly introduction by Yurii Burago, Dima Burago and Sergei Ivanov.
\item A book  by the three authors of the present volume  \cite{alexander-kapovitch-petrunin-CAT} gives an introduction aiming at reaching interesting applications and theorems with a minimum of preparation.
\item The book of J\"{u}rgen Jost \cite{jost} gives a more analytic viewpoint to the subject.
\end{itemize}

One of the most striking applications of $\CAT0$ spaces was given by Dmitry Burago, Sergei Ferleger and Alexey Kanonenko \cite{burago-ferleger-kononenko1998-1},
who used them to study billiards; this idea was developed further in \cite{burago-ferleger-kononenko1998-2,burago-ferleger-kononenko1998-3,burago-ferleger-kononenko1998-4,burago-ferleger-kononenko2000,burago-ferleger-kononenko2001}. 
Another beautiful application is the construction of exotic aspherical manifolds by Michael Davis \cite{davis:aspherical}; related results are surveyed in \cite{davis:exotic,charney-davis-1995}.
Both of these topics are discussed in \cite{alexander-kapovitch-petrunin-CAT}.

\parbf{Satellites and successors.}
Surfaces with bounded integral curvature were studied by Alexandrov's school.
An excelent book on the subject was written by Aleksandr Alexandrov and Viktor Zalgaller \cite{aleksandrov-zalgaller}; see also a more up to date survey by Yuri Reshetnyak \cite{reshetnyak:2D}.

Spaces with two-sided bounds on curvature is another subject already studied  by Alexandrov's school;
a good survey is written by Valerij Berestovskij and Igor Nikolaev \cite{berestovskii-nikolaev}.

A spin-off of the idea of synthetically defining upper curvature bounds 
was given by Gromov \cite{gromov:hyp-groups}. 
He  defined the so called  $\delta$-hyperbolic spaces which satisfy   a coarse version of the  negative curvature condition which applies in particular to discrete metric spaces.
This notion and its various generalizations such as semi-hyperbolicity (a coarse version of non-positive curvature) and relative hyperbolicity have  led to the emergence of the subject of geometric group theory which relates geometric properties of groups to their algebraic ones.
This is a well developed subject with a large number of subfields and applications such as to the theory of small cancellation groups, automatic groups,  mapping class groups, automorphisms of free groups, isoperimetric inequalities on groups, actions on $\R$-trees, study of Gromov's boundaries of groups.

The study of group actions on $\CAT 0$ spaces and $\CAT 0$ cube complexes played a key role in the proof of the virtually fibered conjecture that a finite cover of  every closed hyperbolic 3-manifold fibers over the circle.

The so called {}\emph{curvature dimension condition} introduced by John Lott, C\'edric Villani and Karl-Theodor Sturm gives a synthetic description of Ricci curvature bounded below; see the book of Villani \cite{villani} and references therein.

Alexandrov geometry influenced the development of analysis on metric spaces. 
An excellent book on the subject was written by Juha Heinonen, Pekka Koskela, Nageswari Shanmugalingam, and Jeremy Tyson \cite{heinonen-koskela-shanmugalingam-tyson}.

\section*{Acknowledgment}
We want to thank 
Semyon Alesker,
I. David Berg,
Richard Bishop, 
Yuri Burago, 
Nicola Gigli,
Sergei Ivanov,
Bernd Kirchheim, 
Bruce Kleiner, 
John Lott,
Alexander Lytchak, 
Greg Kuperberg, 
Nikolai Kosovsky, 
Nina Lebedeva,
Wilderich Tuschmann, and
Sergio Zamora Barrera.
%who else?

%Yet special thanks to our non-mathematicician friends and relatives M.~Prelovskaya, J.~Tuschamnn, F.~Champong???; they made for us food, provide place to stay and did not ask stupid questions while this book was written.

We want to thank the mathematical institutions which hosted various authors during the preparation of this book: %where we worked on this book:
BIRS, 
MFO, 
Henri Poincar\'{e} Institute,
University of Cologne, 
Max Planck Institute for Mathematics.
%what else????

%???Grants

During the long writing of this book, we were partially supported by the following grants:
Stephanie Alexander --- 
Simons Foundation grant 209053;
Vitali Kapovitch ---  NSF grant DMS-0204187, NSERC Discovery grants and Simons Foundation grant 390117;
Anton Petrunin --- 
NSF grants
%DMS-0103957,
DMS-0406482,
DMS-0905138,
DMS-1309340,
DMS-2005279,
and Simons Foundation grants 
245094 and 584781.

