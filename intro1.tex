%%!TEX root = arXiv.tex
%array^
\chapter*{Preface}

Alexandrov spaces are defined via axioms similar to those given by Euclid.
The Alexandrov axioms replace certain  equalities with inequalities. 
Depending on the signs of the inequalities, we obtain Alexandrov spaces with {}\emph{curvature bounded above} (CBA) and {}\emph{curvature bounded below} (CBB).
The definitions of the two classes of spaces are similar, but their properties and known applications are quite different.

The goal of this book is to give a comprehensive exposition of the structure theory of Alexandrov spaces 
with curvature bounded above and below.
It includes all the basic material as well as selected topics inspired by considering the two contexts simultaneously.
We only consider the intrinsic theory, leaving applications aside. 
Our presentation is linear,
with a few exceptions where topics are deferred to later chapters to streamline the exposition.
This book includes material \emph{up to the definition of dimension}.
Another volume still in preparation will cover further topics.


\section*{Brief history}

The first synthetic description of curvature is due to Abraham Wald \cite{wald}; 
it was given in a lone publication on a ``coordinateless description of Gauss surfaces'' published in 1936.
In 1941, similar definitions were rediscovered by Alexandr Alexandrov \cite{alexandrov:def}.

In Alexandrov's work the first fruitful applications of this approach were given.
Mainly: {}\emph{Alexandrov's embedding theorem} \cite{alexandrov-1941,alexandrov-1941convex}, which describes closed convex surfaces in Euclidean 3-space,
and the {}\emph{gluing theorem} \cite{alexandrov-1946}, which gave a flexible tool to modify non-negatively curved metrics on a sphere.
These two results together gave an intuitive geometric tool to study embeddings and bending of surfaces in Euclidean space and changed the subject dramatically.
They formed the foundation of the branch of geometry now called {}\emph{Alexandrov geometry}.


\parbf{Curvature bounded below.}
The theory grew out of studying intrinsic and extrinsic geometry of convex surfaces without the smoothness condition.
It was developed by Alexandr Alexandrov
and his school.
Here is a very incomplete list of contributors to the subject:
Yuriy  Borisov,
Yuriy  Burago,
Boris Dekster,
Iosif  Liberman,
Sergey  Olovyanishnikov,
Aleksey  Pogorelov,
Yuriy  Reshetnyak,
Yuriy  Volkov,
Viktor  Zalgaller.
%WE SHOULD WRIE A BIT ABOUT EVERYONE???

The first result in higher dimensional Alexandrov spaces was the splitting theorem.
It was proved by Anatoliy Milka \cite{milka-line} and appeared in 1967.
Milka used a global definition similar to the one used in this book. %what we do in this book.

{\sloppy 

In the 80's the interest in convergence of Riemannian manifolds spurred by \emph{Gromov's compactness theorem} \cite{gromov-MS} turned attention toward the singular spaces that can occur as limits of Riemannian manifolds.
Immediately it was recognized that if the manifolds have a uniform lower sectional curvature bound, then the limit spaces have a lower curvature bound in the sense of Alexandrov.
There followed during the 90's an explosion of work on intrinsic theory of Alexandrov spaces  starting with papers of Yuriy Burago, Grigory Perelman, and Michael Gromov  \cite{burago-gromov-perelman,perelman:spaces2}.
Similar ideas were developed independently by Karsten Grove
and Peter Petersen, whose work was not converted into a publication, and also by
Conrad Plaut~\cite{plaut-preprint}.

}

Around the same time an implicit application of higher-dimensional Alexandrov geometry was given by Michael Gromov in his bound on Betti numbers \cite{gromov:betti}.
%{\color{red} V:  should we also mention Gromov's bound on the number of generators of the fundamental group?}
Another implicit application, which essentially used Alexandrov geometry before it was was actually introduced, given later by Wu-Yi Hsiang and Bruce Kleiner in their paper on non-negatively curved 4-manifolds with infinite symmetry groups \cite{hsiang-kleiner}.
The work of Hsiang and Kleiner and its extension by Karsten Grove and Burkhard Wilking \cite{grove-wilking} are some of the most beautiful applications of this branch of Alexandrov geometry.

The above activity was very much related to so-called {}\emph{comparison geometry},
a branch of differential geometry that compares Riemannian manifolds  to  spaces of constant curvature.
In addition to the already-mentioned {}\emph{Gromov's compactness theorem},
the following results had a big influence on the development of Alexandrov geometry:
{}\emph{Toponogov comparison theorem} \cite{toponogov-globalization+splitting}, which is a generalization of the theorem of Alexandrov \cite{alexandrov-comparison};
{}\emph{Toponogov splitting theorem} \cite{toponogov-globalization+splitting}, which is a generalization of Cohn-Vossen's theorem \cite{cohn-vossen_line};
{}\emph{Finiteness theorems} of
Cheeger
and
Grove--Petersen \cite{cheeger-finiteness,grove-petersen:finiteness};
Gromov's bound on the number of generators of the fundamental group 
\cite[1.5]{gromov:almost-flat};
and 
{}\emph{Yamaguchi fibration theorem} \cite{yamaguchi-fibration}.

%
Let us give a list of available introductory texts on Alexandrov spaces with curvature bounded below: 
\begin{itemize}
\item The first introduction to Alexandrov geometry is given in the original paper of Yuriy Burago, Michael Gromov, and Grigory Perelman \cite{burago-gromov-perelman} 
and its extension \cite{perelman:spaces2} written by Perelman.
\item A brief and reader-friendly introduction was written by Katsuhiro Shiohama \cite[Sections 1--8]{shiohama}.
\item \cite[Chapter 10]{burago-burago-ivanov} gives another reader-friendly introduction, written by Dmiti Burago, Yuriy Burago, and Sergei Ivanov.
\end{itemize}
In addition, let us mention two surveys, one by Conrad Plaut \cite{plaut:survey} and the other by the third author \cite{petrunin:survey}.

{\sloppy

\parbf{Curvature bounded above.}
The study of  spaces with curvature bounded above started later,
inspired by analogy with the theory of curvature bounded below.
The first paper on the subject was written by Alexandrov \cite{alexandrov:strong-angle}, appearing in 1951.
An analogous weaker definition was considered earlier by Herbert Busemann \cite{busemann-CBA}.

}

Contributions to the subject were made by
Valerii Berestovskii, 
Arne Beurling, 
Igor Nikolaev,
Dmitry Sokolov,
Yuriy Reshetnyak,
Samuel Shefel; this list is not complete as well.
The most fundamental results were obtained by Yuriy Reshetnyak.
They include his {}\emph{majorization theorem} and {}\emph{gluing theorem}.
The gluing theorem states that if two non-positively curved spaces have isometric convex sets, then the space obtained by gluing these sets along an isometry is also non-positively curved.

The development of Alexandrov geometry was greatly influenced by the {}\emph{Hadamard--Cartan theorem}.
Its original formulation states that the exponential map at any point of a complete Riemannian manifold with nonpositive sectional curvature is a covering.
In particular, it implies that the universal cover is diffeomorphic to Euclidean space of the same dimension. 
See further discussion below (\ref{thm:hadamard-cartan}).

An influential implicit application of Alexandrov spaces with curvature bounded above can be seen in {}\emph{Euclidean buildings}, introduced by Jacques Tits as a means to study algebraic groups.


Here is a list of available texts covering the basics of Alexandrov spaces with curvature bounded above: 
\begin{itemize}
\item The book of Martin Bridson and Andr\'e Haefliger \cite{bridson-haefliger} gives the most comprehensive introduction available today. 
\item The lecture notes of Werner Ballmann \cite{ballmann:lectures, ballmann:notes} include a brief 
and clear
introduction.
\item \cite[Chapter 9]{burago-burago-ivanov} gives another reader-friendly introduction, by Yuriy Burago, Dmitry Burago, and Sergei Ivanov.
\item A book  by the three authors of the present volume  \cite{alexander-kapovitch-petrunin-CAT} gives an introduction aiming at reaching interesting applications and theorems with a minimum of preparation.
\item The book of Jürgen Jost \cite{jost:book} gives a more analytic viewpoint to the subject.
\end{itemize}

One of the most striking applications of $\CAT0$ spaces was given by Dmitry Burago, Sergei Ferleger, and Alexey Kononenko \cite{burago-ferleger-kononenko1998-1},
who used them to study {}\emph{billiards}; this idea was developed further in \cite{burago-ferleger-kononenko1998-2,burago-ferleger-kononenko1998-3,burago-ferleger-kononenko1998-4,burago-ferleger-kononenko2000,burago-ferleger-kononenko2001}. 
Another beautiful application is the construction of {}\emph{exotic aspherical manifolds} by Michael Davis \cite{davis:aspherical}; related results are surveyed in \cite{davis:exotic,charney-davis-1995}.
Both of these topics are discussed in \cite{alexander-kapovitch-petrunin-CAT}.
The study of group actions on $\CAT 0$ spaces and $\CAT 0$ cube complexes played a key role in the proof of the {}\emph{virtually fibered conjecture} that a finite cover of  every closed hyperbolic 3-manifold fibers over the circle.

\parbf{Satellites and successors.}
Surfaces with {}\emph{bounded integral curvature} were studied by Alexandrov's school.
An excellent book on the subject was written by Alexandr Alexandrov and Viktor Zalgaller \cite{alexandrov-zalgaller}; see also a more up-to-date survey by Yuriy Reshetnyak \cite{reshetnyak:2D}.

Spaces with {}\emph{two-sided bounded curvature} is another subject already studied  by Alexandrov's school;
a good survey is written by Valerij Berestovskij and Igor Nikolaev \cite{berestovskii-nikolaev}.

A spin-off of the idea of synthetically defining upper curvature bounds 
was given by Michael Gromov \cite{gromov:hyp-groups}. 
He  defined so-called  {}\emph{$\delta$-hyperbolic spaces}, which satisfy   a coarse version of the  negative curvature condition, applying  in particular to discrete metric spaces.
This notion and its various generalizations such as semi-hyperbolicity (a coarse version of non-positive curvature) and relative hyperbolicity have  led to the emergence of the subject of {}\emph{geometric group theory}, which relates geometric properties of groups to their algebraic ones.
This is a well-developed subject with a large number of subfields and applications, such as the theory of small cancellation groups, automatic groups,  mapping class groups, automorphisms of free groups, isoperimetric inequalities on groups, actions on $\R$-trees, Gromov's boundaries of groups.

{\sloppy

The so-called {}\emph{curvature dimension condition} introduced by John Lott, C\'edric Villani, and Karl-Theodor Sturm gives a synthetic description of Ricci curvature bounded below; see the book of Villani \cite{villani} and references therein.
A striking application of this theory to geodesic flow in $\Alex{}$ spaces was found recently by
Elia Bruè,
Andrea Mondino,
and Daniele Semola \cite{brue-mondino-semola}.

Alexandrov geometry influenced the development of {}\emph{analysis on metric spaces}. 
An excellent book on the subject was written by Juha Heinonen, Pekka Koskela, Nageswari Shanmugalingam, and Jeremy Tyson~\cite{heinonen-koskela-shanmugalingam-tyson}.

}

\section*{Acknowledgment}
We thank 
Semyon Alesker,
Valerii Berestovskii,
I. David Berg,
Richard Bishop, 
Yuriy Burago, 
Alexander Christie,
Nicola Gigli,
Sergei Ivanov,
Bernd Kirchheim, 
Bruce Kleiner, 
Rostislav Matveyev,
John Lott, 
Greg Kuperberg, 
Nikolai Kosovsky, 
Nina Lebedeva,
Wilderich Tuschmann, 
and
Sergio Zamora Barrera.
A special thanks to Alexander Lytchak, who contributed deeply to this book during the long process of writing, but refused to be one of us.
%who else?

We thank the mathematical institutions where we worked on this book: 
%where we worked on this book:
BIRS, 
MFO,
Henri Poincar\'{e} Institute,
IHES,
University of Cologne, 
Max Planck Institute for Mathematics,
EIMI.
%what else????

%???Grants

During the long writing of this book, we were partially supported by the following grants:
Stephanie Alexander --- 
Simons Foundation grant 209053;
Vitali Kapovitch ---  NSF grant DMS-0204187, NSERC Discovery grants, and Simons Foundation grant 390117;
Anton Petrunin --- 
NSF grants
%DMS-0103957,
DMS-0406482,
DMS-0905138,
DMS-1309340,
DMS-2005279,
Simons Foundation grants 
245094 and 584781,
and
Minobrnauki, grant 075-15-2022-289.
