%%!TEX root = all.tex
%array^
\chapter*{Introduction}

In this book, the term Alexandrov space will refer to one of the following types of %complete 
metric spaces:
\begin{itemize}
\item spaces with curvature bounded below;
\item spaces with curvature bounded above;
\item spaces with two-sided bounded curvature.
\end{itemize}
All these classes of spaces are defined using metric comparison to a model space, without using any analytical tools. 

The goal of this book is to give a comprehensive description of theory of Alexandrov spaces with curvature bounded below (CBB???);  and in the theory of Alexandrov spaces with curvature bounded above (CBA???), about which more is currently available in book form, to cover all the basic material as well as selected topics inspired by considering the two contexts simultaneously.  
Although the two fields developed quite independently, they have many similar guiding intuitions and technical tools.  Our approach is novel in its attention to the interrelatedness of the two fields, and its emphasis on the way each illuminates the other.  We only consider inner theory, leaving most applications aside.


\section*{History}

The first synthetic description of curvature is due to Abraham Wald; 
it was given in a lonely publication on a ``coordinateless description of Gauss surfaces'' published in 1936, see \cite{wald}.
In 1941, similar definitions were rediscovered independently by Aleksandr Danilovich Aleksandrov,
see \cite{alexandrov:def}.
In Alexandrov's work the first fruitful applications of this approach were given.
Mainly:
\begin{itemize}
\item Alexandrov's embedding theorem  --- 
\textit{metrics of non-negative curvature on the sphere, and only they, are isometric to closed convex surfaces in Euclidean 3-space}. This theorem appears first in \cite{alexandrov:def} and later in the book \cite{alexandrov-embedding}.
\item Gluing theorem, which gave a flexible tool to modify non-negativly curved metrics on a sphere, see \cite{alexandrov-glueing}.
\end{itemize}
These two results together gave  a very intuitive geometric tool to study embeddings and bending of surfaces in Euclidean space, and changed this subject dramatically.
They formed the foundation of branch of geometry now called \emph{Alexandrov geometry}.

Alexandrov geometry can use ``back to Euclid'' as a slogan.
Alexandrov spaces are defined via axioms similar to the one given by Euclid,
but certain  equalities exchanged to inequalities. 
Depending on the sign of inequalities we get Alexandrov spaces with \emph{curvature bounded above} and \emph{curvature bounded below}.
Althogh the definitions of two classes of spaces are similar, their properties and known applications are qute different.

\parbf{Curvature bounded below.}
The theory was developed mostly in the two-dimensional case 
by Alexandrov
and his students  
Yurii Fedorovich Borisov,
Yurii Dmitrievich Burago,
Iosif Meerovich Liberman,
Sergey Pantelemonovich Olovyanishnikov,
Aleksey Vasilyevich Pogorelov,
Yurii Grigorievich Reshetnyak,
Yurii Alexandrovich Volkov,
Viktor Abramovich Zalgaller and others.
%WE SHOULD WRIE A BIT ABOUT EVERYONE???

Before the 80's 
there were only few papers on the higher dimensional analog.
\begin{itemize}
\item The work of Anatoliy Dmitrievich Milka, 
who started to consider multidimensional Alexandrov's spaces with lower curvature bound in 60's;
Milka used the global definition similarly to what we do in this book.
In particular he proved the splitting theorem for such spaces.
Milka was developing the theory of polyhedral spaces with curvature bounded below.
\item The work of Bruce Kleiner on non-negatively curved manifolds with infinite group of symmetries.
This work can be considered as the first application of Alexandrov geometry in dimension grater than 2,
but Kleiner used Alexandrov geometry only implicitly, 
\end{itemize}

In the 80's the interest in convergence of Riemannian manifolds created by Gromov's compactness theorem (\cite{gomov-precompactness}) turned attention toward the singular spaces that can occur as limits of Riemannian manifolds. 
Immediately it was recognized that if the manifolds have a uniform lower sectional curvature bound, then the limit spaces have a lower curvature bound in the sense of Alexandrov. 
There followed throughout the 90's an explosion of work starting with papers of Yurij Dmitrievich Burago, Grigori Yakovlevich Perelman and Mikhail Leonidovich Gromov  \cite{BGP} and \cite{perelman:spaces2}.

The above activity was very much related to comparison geometry.
The latter approach is not that maximalistic --- it is a branch of differential geometry which compares Riemannian manifolds  to  spaces of constant curvature.
In addition to the already-mentioned Gromov's compactness theorem,
the following results had a big influence on the development of Alexandrov's geometry:
\begin{itemize}
\item Toponogov comparison and splitting theorems;  
these are generalizations of Cohn-Vossen's theorem,
\item Finiteness theorems of
Cheeger and
Grove--Petersen--Wu,
\item Yamaguchi fibration theorem.
\end{itemize}

Here is a list of available introductions to  Alexandrov spaces with curvature bounded below: 
\begin{itemize}
\item The original paper of Burago, Gromov and Perelman \cite{BGP} 
and its extension \cite{perelman:spaces2} written by Perelman.
It is the first introduction to Alexandrov's geometry. 
\item Shiohame's introduction to Alexandrov's geometry \cite{shiohama}.
It is designed to be reader-friendly. 
\item Plaut's survey in Alexandrov's geometry \cite{plaut:survey}
written mostly for topologists. 
The first 8 sections can be used as an introduction, but it is a bit brief.
\item \cite[Chapter 10]{BBI} is yet another reader-friendly introduction.
\end{itemize}

\parbf{Curvature bounded above.}
The study of the spaces with curvature bounded above started later.
The first paper on the subject was written by Alexandrov, appearing in 1951, see \cite{alexandrov:strong-angle}.
An analogous weaker definition was considered earlier by Busemann in \cite{busemann-CBA}.

The fundamental results in this direction were obtained by Yurii Grigorievich Reshetnyak.
It includes his majorization theorem and gluing theorem.
The gluing theorem states that if two non-positively curved spaces have isometric convex sets, then the space obtained by gluing these sets along the isometry is also non-positively curved.

One of the most beautiful applications of the gluing theorem was given  recently by Dmitry Burago, Sergei Ferleger and Alexey Kanonenko in \cite{BFK};
in particular it gives an estimate for the number of collisions of a billiard without walls.

Here is a list of available introductions to the Alexandrov spaces with curvature bounded above: 
\begin{itemize}
\item \cite{BH}
\item \cite{ballmann:lectures} Lecture notes of Werner Ballmann
\item \cite[Chapter 9]{BBI} a reader-friendly introduction.
\end{itemize}

\parbf{Further development.}
The theory of spaces with two sided curvature bound was developed by V. N. Berestovskii and I. G. Nikolaev. It turns out that these are in fact non-smooth Riemannian manifolds. 

A spin-off of the idea of synthetically defining upper curvature bounds 
was given by Gromov. 
He gave an analogous definition for discrete metric spaces, so called $\delta$-hyperbolic spaces. 
It provided an intuitive geometric tool for studying small cancellation groups.

The curvature dimension condition by Lott, Villani and Sturm is a further generalization of Alexandrov spaces with curvature bounded below; see the book of Villani \cite{villani} and references there
in.

\section*{Acknowledgment}
We want to thank 
Semyon Alesker,
I. David Berg,
Richard Bishop, 
Yuri Burago, 
Sergei Ivanov,
Bernd Kirchheim, 
Bruce Kleiner, 
John Lott,
Alexander Lytchak, 
Greg Kuperberg, 
Nikolai Kosovsky, 
Nina Lebedeva and
Wilderich Tuschmann.
%who else?

%Yet special thanks to our non-mathematicician friends and relatives M.~Prelovskaya, J.~Tuschamnn, F.~Champong???; they made for us food, provide place to stay and did not ask stupid questions while this book was written.

We want to thank the mathematical institutions where we worked on this book:
BIRS, 
MFO, 
Henri Poincar\'{e} Institute,
University of Cologne, 
Max Planck Institute for Mathematics.
%what else????



