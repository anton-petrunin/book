\section{Helly's theorem}\label{sec:helly}

%???CHECK lang-schroeder HAlly THEOREM???

\begin{thm}{Helly's theorem}\label{thm:helly}
Let $\spc{U}$  be a complete length $\CAT0$ space
and $\{K_\alpha\}_{\alpha\in \IndexSet}$ be an arbitrary collection of closed bounded convex subsets of $\spc{U}$.

If 
\[\bigcap_{\alpha\in \IndexSet}K_\alpha=\emptyset\]
then there is an index array $\alpha_1,\alpha_2,\dots,\alpha_n\in \IndexSet$ such that
\[\bigcap_{i=1}^nK_{\alpha_i}=\emptyset.\]

%???Moreover, if $\dim \spc{U}\le m$ then one can assume above $n\le m+1$.
\end{thm}

\parbf{Remarks.}
\begin{enumerate}[(i)]
\item In general, none of $K_\alpha$ might be compact; 
otherwise the statement is trivial.
\item If $\spc{U}$ is a Hilbert space (not necessarily separable), 
then the above result is equivalent to the statement that a convex bounded set 
which is closed in ordinary topology forms a compact set in the weak topology.

In fact, one can define \emph{weak topology} on arbitrary metric space, by taking exteriors of closed ball as its prebase.
Then the result above implies for  a complete length $\CAT0$ space $\spc{U}$, any closed bounded convex set in $\spc{U}$ is compact in weak topology 
(compare to \cite{monod}).
\end{enumerate}

\medskip

We present the proof the original proof of Urs Lang and Viktor Schroeder from \cite{lang-schroeder}.

%\parit{Proof.}
%Let us first prove uniqueness. 
%Assume there are two points $y',y''\in K$ 
%so that $\dist{y'}{p}{}=\dist{y''}{p}{}=\dist{K}{p}{}$.
%Take $z$ to be midpoint of $[y'y'']$. 
%Since $K$ is convex, $z\in K$.
%From comparison, we have that $\dist{z}{p}{}<\dist{y'}{p}{}=\dist{K}{p}{}$, a contradiction
%
%The proof of existence is analogous.
%Take a sequence  of points $y_n\in K$ 
%such that $\dist{y_n}{p}{}\to \dist{K}{p}{}$.
%It is sufficient to show that $(y_n)$ converges in itself; 
%thus one could take $p^*=\lim_n y_n$.

%Assume $(y_n)$ does not converge in itself, then for some fixed $\eps>0$, 
%we can choose two subsequences $(y_n')$ and $(y_n'')$ of $(y_n)$ 
%such that 
%$\dist{y'_n}{y''_n}{}\ge\eps$ for each $n$.
%Set $z_n$ to be the midpoint of $[y'_ny''_n]$; from convexity we have $z_n\in K$.
%From point-on-side comparison \ref{cat-monoton}, there is $\delta>0$ 
%such that $\dist{p}{z_n}{}\le \max\{\dist{p}{y'_n}{},\dist{p}{y''_n}{}\}-\delta$. 
%Thus 
%\[\limsup_{n\to\infty}\dist{p}{z_n}{}<\dist{K}{x}{},\] 
%a contradiction\qeds

\parit{Proof of \ref{thm:helly}.} 
Assume the contrary. Then for any finite set $F\subset \IndexSet$
\[K_{F}\df \bigcap_{\alpha\in F}K_{\alpha}\not=\emptyset,\]
we will construct point $z$ such that $z\in K_\alpha$ for each $\alpha$.
Thus we will arrive to contradiction since
\[\bigcap_{\alpha\in \IndexSet}K_\alpha=\emptyset.\]

Choose a point $p\in \spc{U}$ and set $r=\sup\dist{K_{F}}{p}{}$ where $F$ runs all finite subsets of $\IndexSet$.
Set $p^*_F$ to be the closest point on $K_{F}$ from $p$; 
according to closest-point projection lemma (\ref{lem:closest point}), $p^*_F$ 
exits and is unique.

Take a nested sequence of finite subsets 
$F_1\subset F_2\subset \dots$ of $\IndexSet$, such that $\dist{K_{F_n}}{p}{}\to r$.

Let us show that the sequence $(p^*_{F_n})$ converges in itself. 
Indeed, if not then for some fixed $\eps>0$, 
we can choose two subsequences $(y'_n)$ and $(y''_n)$ of $(p^*_{F_n})$ 
such that $\dist{y'_n}{y''_n}{}\ge\eps$.
Set $z_n$ to be midpoint of $[y'_ny''_n]$. 
From point-on-side comparison (\ref{point-on-side}), 
there is $\delta>0$ such that 
\[\dist{p}{z_n}{}\le \max\{\dist{p}{y'_n}{},\dist{p}{y''_n}{}\}-\delta.\]
Thus 
\[\limsup_{n\to\infty}\dist{p}{z_n}{}<r.\]
On the other hand, from convexity, each $F_n$ 
contains all $z_\kay$ with sufficiently large $\kay$, a contradiction.

Thus, $p^*_{F_n}$ converges and we can set $z=\lim_n p^*_{F_n}$.
Clearly 
\[\dist{p}{z}{}=r.\]

Repeat the above arguments for  the sequence $F_n'=F_n\cup \{\alpha\}$.
As a result, we get another point $z'$ such that $\dist{p}{z}{}=\dist{p}{z'}{}=r$ and 
$z,z'\in K_{F_n}$ for all $n$.
Thus, if $z\not=z'$ the midpoint $\hat z$ of $[zz']$ would belong to all 
$K_{F_n}$ and from comparison we would have $\dist{p}{\hat z}{}<r$, a contradiction.

Thus, $z'=z$; in particular 
$z\in K_\alpha$ for each $\alpha\in\IndexSet$.
\qeds