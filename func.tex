%%!TEX root = the-func.tex

\chapter{Maps and functions}

Here we introduce some classes of maps between metric spaces and develop a language to describe various notions of convexity/concavity of real-valued functions on general metric space.

\section{Submaps}\label{sec:submaps}

We will often need maps and functions defined on subsets of a metric space.
We call them \index{submap}\emph{submaps} and \index{subfunction}\emph{subfunctions}.
Thus, given  metric spaces $\spc{X}$ and $\spc{Y}$, 
a submap $\map\:\spc{X}\subto \spc{Y}$ is a map defined on a subset $\Dom\map\subset \spc{X}$.

A submap is said to be \index{continuous submap}\emph{continuous} if the inverse image of any open set is open.
Note that for a continuous submap $\map$, the domain $\Dom\map$ is automatically open.
Indeed, if  submap $\map\:\spc{X}\subto \spc{Y}$ is continuous then
 $\Dom \map=\map^{-1}(\spc{Y})$ is open as the preimage of an open set.
The same is true for upper and lower semicontinuous functions $f\:\spc{X}\subto \RR$ since they are  continuous functions for a special topology on $\RR$.

(Continuous partially defined maps could be defined via closed sets; namely, one could require that inverse images of closed sets are closed.
While this condition is equivalent to continuity for functions defined on the whole space,
it is different for partially defined functions. 
In particular, with this definition the domain of a continuous submap would have to be closed.)

\section{Lipschitz conditions}


\begin{thm}{Lipschitz maps}
Suppose $\spc{X}$ and $\spc{Y}$ are metric spaces, 
$\map\:\spc{X}\subto\spc{Y}$ is a continuous submap,  
and $\Lip\in\RR$.

\begin{subthm}{}
The submap $\map$ is called \index{Lipschitz map}\emph{$\Lip$-Lipschitz} if
\[\dist{\map(x)}{\map(y)}{\spc{Y}}
\le
\Lip\cdot
\dist{x}{y}{\spc{X}}\]  
for any two points $x,y\in\Dom \map$.

\begin{itemize}
 \item $1$-Lipschitz maps will be also called \index{short map}\emph{short}.
\end{itemize}

\end{subthm}

\begin{subthm}{}
We say that $\map$ is \index{Lipschitz map}\emph{Lipschitz} if it is $\Lip$-Lipschitz for some constant $\Lip$.
The minimal such constant is denoted by $\lip\map$.
\end{subthm}

\begin{subthm}{}
We say that $\map$ is \emph{locally Lipschitz} 
if any point $x\in\Dom \map$ admits a neighborhood 
$\Omega\subset \Dom\map$ such that the restriction $\map|_\Omega$ is Lipschitz.
\end{subthm}

\begin{subthm}{}
Given $p\in\Dom \map$, we denote by $\lip_p\map$ the infimum of the real values $\Lip$ such that
$p$ admits  a neighborhood 
$\Omega\subset \Dom\map$ such that the restriction $\map|_\Omega$ is $\Lip$-Lipschitz.
\end{subthm}
\end{thm}

Note that $\map\:\spc{X}\to\spc{Y}$ is $\Lip$-Lipschitz if and only if
\[\map(\oBall(x, R)_{\spc{X}})\subset\oBall(\map(x),\Lip\cdot R)_{\spc{Y}}\]
for any $R\ge 0$ and $x\in \spc{X}$.
The following definition gives a dual version.

\begin{thm}{Definitions}
Let $\spc{X}$ and $\spc{Y}$ be metric spaces, 
$\map\:\spc{X}\to\spc{Y}$ be a map,  
and $\Lip\in\RR$.
\begin{subthm}{}
The map $\map$ is called \index{$\Lip$-co-Lipschitz map}\emph{$\Lip$-co-Lipshitz} if 
\[\map(\oBall(x,\Lip\cdot R)_{\spc{X}})\supset\oBall(\map(x),R)_{\spc{Y}}\]
for any $x\in \spc{X}$ and $R>0$.
\end{subthm}

\begin{subthm}{}
The map $\map$ is called \index{co-Lipschitz map}\emph{co-Lipschitz} if it is $\Lip$-co-Lipschitz
for some constant $\Lip$.
The minimal such constant is denoted by $\colip\map$.

\end{subthm}
\end{thm}

From the definition of co-Lipschitz maps we get the following:

\begin{thm}{Proposition}
Any co-Lipschitz map is open and surjective.
\end{thm}

In other words,  $\Lip$-co-Lipschitz maps 
can be considered as a quantitative version of open maps.
For that reason they are also called $\Lip$-open \cite{burago-gromov-perelman}.
Also, be aware that some authors
refer to our $\Lip$-co-Lipschitz maps
as $\tfrac1\Lip$-co-Lipschitz.

\begin{thm}{Proposition}\label{prop:colip=>complete}
Let $\spc{X}$ and $\spc{Y}$ be metric spaces such that $\spc{X}$ is complete, and let
$\map\: \spc{X}\to\spc{Y}$ be a continuous co-Lipschitz map. 
Then $\spc{Y}$ is complete.
\end{thm}

\parit{Proof.}
Choose a Cauchy sequence $y_n$ in $\spc{Y}$.
Passing to a subsequence if necessary, we may assume that $\dist{y_n}{y_{n+1}}{\spc{Y}}< \tfrac1{2^n}$ for each~$n$.

Denote by $\Lip$ a co-Lipschitz constant for $\map$.
Note that  there is a sequence $x_n$ in $\spc{X}$
such that
\[\map(x_n)=y_n\quad \text{and}\quad \dist{x_n}{x_{n+1}}{\spc{X}}< \tfrac{\Lip}{2^n}\eqlbl{eq:colip+1/2n}\]
for each $n$. 
Indeed, such a sequence can be constructed recursively. 
Assuming that the points $x_1,\dots,x_{n-1}$ are already constructed, 
the existence of a sequence $x_n$ satisfying \ref{eq:colip+1/2n}
follows since $\map$ is $\Lip$-co-Lipschitz.

Notice that the sequence $x_n$ is Cauchy.
Since $\spc{X}$ is complete, $x_n$ converges in $\spc{X}$; denote its limit by $x_\infty$ 
and set $y_\infty= \map(x_\infty)$.
Since $\map$ is continuous,
$y_n\to y_\infty$ as $n\to\infty$.
Hence the result.
\qeds

\begin{thm}{Lemma}\label{lem:lip-approx}
Let $\spc{X}$ be a metric space and $f\:\spc{X}\to\RR$ be a continuous function.
Then for any $\eps>0$ there is a locally Lipschitz function $f_\eps\:\spc{X}\to\RR$
such that $|f(x)-f_\eps(x)|<\eps$ for any $x\in \spc{X}$.
\end{thm}

\parit{Proof.}
Assume that $f\ge 1$.
Construct a continuous positive function $\rho\:\spc{X}\to \RR_{>0}$ such that 
\[\dist{x}{y}{}<\rho(x)\quad \Rightarrow\quad |f(x)-f(y)|<\eps.\]
Consider the function
\[
f_\eps(x)
=
\sup\set{f(x)
\cdot
\left(1-\tfrac{\dist{x}{y}{}}{\rho(x)}\right)}%
{x\in\spc{X}}.
\]
It is straightforward to check that each $f_\eps$ is locally Lipschitz and $0\le f_\eps-f<\eps$.

Since any continuous function can be presented as the difference of two continuous functions bounded below by $1$, the result follows.
\qeds

\section{Isometries and submetries}\label{sec:quotient-CBB}

\begin{thm}{Isometry}\label{def:isometry}
Let $\spc{X}$ and $\spc{Y}$ be metric spaces
and $\map\:\spc{X}\to \spc{Y}$ be a map
\begin{subthm}{}
The map $\map$ is \index{distance-preserving map}\emph{distance-preserving} if
$$\dist{\map(x)}{\map(x')}{\spc{Y}}=\dist{x}{x'}{\spc{X}}$$
for any $x,x'\in X$.
\end{subthm}

\begin{subthm}{}
A distance-preserving bijection $\map$ is called an \index{isometry}\emph{isometry}.
\end{subthm}

\begin{subthm}{}
The spaces $X$ and $Y$ are called \emph{isometric} (briefly $X\iso Y$)
 if there is an isometry  $\map\:X\to Y$.
\end{subthm}

\end{thm}

\begin{thm}{Submetry}\label{def:submetry}
A map $\sigma\:\spc{L}\to\spc{M}$ between the metric spaces $\spc{L}$ and $\spc{M}$
is called a \index{submetry}\emph{submetry} if 
\[\sigma(\oBall(p,r)_\spc{L})=\oBall(\sigma(p),r)_{\spc{M}}\]
for any $p\in \spc{L}$ and $r\ge 0$.
\end{thm}


Note $\sigma\:\spc{L}\to\spc{M}$ is a submetry if it is 1-Lipschitz and 1-co-Lipschitz at the same time.

Note also that any submetry is an onto map.

The main source of examples of submetries comes from isometric group actions.

Namely, assume $\spc{L}$ is a metric space and $G$ is a subgroup of isometries of $\spc{L}$.
Denote by $\spc{L}/G$ the set of $G$-orbits;
let us equip it with the pseudometric defined by
\[\dist{G\cdot x}{G\cdot y}{\spc{L}/G}=\inf\set{\dist{g\cdot x}{h\cdot y}{\spc{L}}}{g,h\in G}.\]
Note that if all the $G$-orbits form closed sets in $\spc{L}$,
then $\spc{L}/G$ is a genuine metric space.

\begin{thm}{Proposition}\label{prop:submet/G}
Let $\spc{L}$ be a metric space.
Assume that a group $G$  acts on $\spc{L}$ by isometries  
and in such a way that every $G$-orbit is closed.
Then the projection map $\spc{L}\to \spc{L}/G$ is a submetry.
\end{thm}

\parit{Proof.}
Denote by $\hat x$ the projection of $x\in \spc{L}$ in $\spc{L}/G$.
We need to show that the map $x\mapsto\hat x$ is $1$-Lipschitz and $1$-co-Lipschitz.
The co-Lipschitz part follows directly from the definitions of Hausdorff distance and co-Lipschitz maps.

Assume $\dist{x}{y}{\spc{L}}< r$; equivalently $\oBall(x,r)_{\spc{L}}\ni y$.
Since the action $G\acts \spc{L}$ is isometric, 
$\oBall(g\cdot x,r)_{\spc{L}}\ni g\cdot y$ for any $g\in G$.

In particular the orbit $G\cdot y$ lies in the open $r$-neighborhood of the orbit $G\cdot x$.
In the same way we can prove that the orbit $G\cdot x$ lies in the open $r$-neighborhood of the orbit $G\cdot y$. 
That is, the Hausdorff distance between the orbits $G\cdot x$ and $G\cdot y$ is  $<r$
or, equivalently, $\dist{\hat x}{\hat y}{\spc{L}/G}< r$.
Since $x$ and $y$ are arbitrary, the map $x\mapsto\hat x$ is $1$-Lipschitz.
\qeds


\begin{thm}{Proposition}
\label{prop:submet-length}
Let  $\spc{L}$ be a length space 
and $\sigma\:\spc{L}\to \spc{M}$ be a submetry.
Then $\spc{M}$ is a length space.
\end{thm}

\parit{Proof.}
Fix $\eps>0$ and a pair of points $x,y\in \spc{M}$.

Since $\sigma$ is $1$-co-Lipschitz, there are points $\hat x,\hat y\in \spc{L}$
such that 
$\sigma(\hat x)\z=x$,
$\sigma(\hat y)\z=y$, 
and $\dist{\hat x}{\hat y}{\spc{L}}<\dist{x}{y}{\spc{M}}+\eps$.

Since ${\spc{L}}$ is a length space, 
there is a curve $\gamma$ 
joining $\hat x$ to $\hat y$ in ${\spc{L}}$
such that
\[\length\gamma\le \dist{x}{y}{\spc{M}}+\eps.\]

Since $\sigma$ is $1$-Lipschitz,
\[\length\sigma\circ\gamma\le \length\gamma.\]

The curve $\sigma\circ\gamma$ joins $x$ to $y$,
and by the above,
\[\length\sigma\circ\gamma<\dist{x}{y}{\spc{M}}+\eps.\]
Since $\eps>0$ is arbitrary,
$\spc{M}$ is a length space.
\qeds

\section{Speed of curves}
\label{sec: speed}

Let $\spc{X}$ be a metric space.
Recall that a \emph{curve}  
in $\spc{X}$ is a continuous map $\alpha\:\II\to \spc{X}$, where $\II$ is a real interval. 
%(see Section~\ref\{sec:intrinsic}). 
A curve is called \index{Lipschitz map!Lipschitz curve}\emph{Lipschitz} or \emph{locally Lipschitz} if $\alpha$ is Lipschitz or locally Lipschitz. 
 Length of curves is defined in \ref{def:length}.

The following theorem follows from \cite[2.7]{burago-burago-ivanov}.

\begin{thm}{Theorem}\label{thm:speed}
Let $\spc{X}$ be a metric space  
and $\alpha\:\II \to \spc{X}$ be a locally Lipschitz
curve. 
Then the speed function
\[\speed_{t_0}\alpha
=
\lim_{\substack{t\to t_0+\\s\to t_0-}}\frac{\dist{\alpha(t)}{\alpha(s)}{}}{|t-s|}\] 
is defined for almost all $t_0 \in \II$, and 
\[\length\alpha=\int\limits_\II \speed_{t}\alpha\cdot\dd t,\]
where $\int$ denotes the Lebesgue integral.
\end{thm}

A curve $\alpha\:\II\to\spc{X}$ is called a  \index{unit-speed curve}\emph{unit-speed curve}
if for any subinterval $[a,b]\subset\II$, we have
\[b-a=\length(\alpha|_{[a,b]}).\]
According to the above theorem, this is equivalent to the condition that $\alpha$ is Lipschitz and $\speed\alpha\ae 1$.

The following generalization of the standard Rademacher theorem 
on differentiability almost everywhere of Lipschitz maps between smooth manifolds \cite[5.5.2]{burago-burago-ivanov} was proved by Bernd Kirchheim \cite{kirchheim}. 

The conclusion of the standard Rademacher theorem does not make sense for maps to a metric space since the target might have no linear structure.
But the theorem does not hold even if we assume that the target is a Banach space.
For example the map $[0,1]\to L^1[0,1]$ defined by $x\mapsto \chi_{[0,x]}$ is distance-preserving and in particular Lipschitz (here $\chi_A$ denotes the characteristic function of $A$).
However the differential $\dd_xf$ is not defined for any $x$.

\begin{thm}{Theorem}\label{thm:Rademacher-md}
Let $\spc{X}$ be a metric space 
and $f\:\RR^n \subto \spc{X}$ be $1$-Lipschitz. 
Then for almost all $x\in\Dom f$ there is a pseudonorm 
$\lVert*\rVert_x$ on $\RR^n$ such that
\[\dist{f(y)}{f(z)}{\spc{X}}=\lVert z-y\rVert_x+o(|y-x|+|z-x|).\]
\end{thm}


Given $f$, the (pseudo)norm $\lVert*\rVert_x$ in the above theorem 
will be called its \index{differential of a metric}\emph{differential of the induced metric} at $x$, or \index{metric differential}\emph{metric differential} at $x$.

\section{Convex real-to-real functions}\label{sec:conv-real}

We will be interested in  generalized solutions
of the following differential inequalities
\[y''+\kappa\cdot  y\ge \lambda
\quad \text{and respectively}
\quad y''+\kappa\cdot  y\le \lambda
\eqlbl{eq:sec:conv-real*}\]
for fixed $\kappa,\lambda\in\RR$.
The solutions $y\:\RR\subto\RR$ are only assumed to be upper (respectively lower) semicontinuous subfunctions.

The inequalities  \ref{eq:sec:conv-real*} are understood in the sense of distributions.
That is, for any smooth function $\phi$ with compact support $\supp\phi\subset\Dom y$ the following inequality should be satisfied:
\[\begin{aligned}
\int\limits_{\Dom y}\left[y(t)\cdot\phi''(t)+\kappa\cdot  y(t)\cdot\phi(t)-\lambda\right]\cdot\dd t
&\ge 0
\\
\text{respectively}\quad &\le0.
\end{aligned}
\eqlbl{eq:distr-conc}\]
The integral is understood in the sense of Lebesgue;
in particular the inequality \ref{eq:distr-conc}
makes sense for any Borel-measurable subfunction $y$.
The proofs of the following propositions are straightforward.

\begin{thm}{Proposition}
Let $\II\subset\RR$ be an open interval and $y_n\:\II\to\RR$ be a sequence of solutions of one of the inequalities in \ref{eq:sec:conv-real*}.
Assume $y_n(t)\to y_\infty(t)$ as $n\to\infty$ for any $t\in \II$.
Then $y_\infty$ is a solution of the same inequality in \ref{eq:sec:conv-real*}.
\end{thm}

Assume $y$ is a solution of one of the inequalities in \ref{eq:sec:conv-real*}.
For $t_0\in \Dom y$, let us define the \emph{right (left) derivative } $y^+(t_0)$ ($y^-(t_0)$) at $t_0$ by
\[y^\pm(t_0)=\lim_{t\to t_0\pm} \frac{y(t)-y(t_0)}{|t-t_0|}.\]
Note that our sign convention for $y^-$ is not standard --- for $y(t)=t$ we have
$y^+(t)=1$ and $y^-(t)=-1$.

\begin{thm}{Proposition}\label{prop:derivative-of-convex-function}
Let $\II\subset\RR$ be an open interval and $y\:\II\to\RR$ be a solution of an inequality in \ref{eq:sec:conv-real*}.
Then  $y$ is locally Lipschitz; its right and left derivatives $y^+(t_0)$ and $y^-(t_0)$ are defined
for any $t_0\in\II$.
Moreover 
\[y^+(t_0)+y^-(t_0)\ge 0
\quad \text{or respectively}
\quad y^+(t_0)+y^-(t_0)\le 0.\]
\end{thm}


The next theorem gives a  number of equivalent ways to define such  generalized solutions.

\begin{thm}{Theorem}\label{y''=<1-ky}
Let $\II$ be an open real interval and $y\:\II\to\RR$ be a locally Lipschitz function.
Then the following conditions are equivalent:
\begin{subthm}{}$y''\ge \lambda-\kappa\cdot  y$ (respectively $y''\le \lambda-\kappa\cdot  y).$
\end{subthm}

\begin{subthm}{barrier}(barrier inequality) For any $t_0\in \II$, 
there is a solution $\bar y$ 
of the ordinary differential equation $\bar y''=\lambda-\kappa\cdot  \bar y$ 
with $\bar y(t_0)\z= y(t_0)$ such that $\bar y\ge y$ (respectively $\bar y\le y$) for all $t\in [t_0-\varpi\kappa,t_0+\varpi\kappa]\cap \II$.

The function $\bar y$ is called a {}\emph{lower} (respectively {}\emph{upper}) \index{barrier}\emph{barrier} of $y$ at $t_0$.
\end{subthm}



\begin{subthm}{y''-mono} (Jensen's inequality)\index{Jensen's inequality}
For any pair of values $t_1<t_2$ in $\II$ such that $|t_2-t_1|<\varpi\kappa$,  the unique solution $z(t)$ of \[z''\z=\lambda-\kappa\cdot  z\] such that
\[z(t_1)=y(t_1),\quad z(t_2)=y(t_2)\] 
satisfies $y(t)\le z(t)$ (respectively $y(t)\ge z(t)$) for all $t\in[t_1,t_2]$.
\end{subthm}

Further, the following property holds:

\begin{subthm}{barrier'} 
Suppose $y''\le \lambda-\kappa\cdot  y$. Let $t_0\in\II$, and   $\bar y$ be a solution of  the
 ordinary differential equation $\bar y''=\lambda-\kappa\cdot  \bar y$ 
such that  $\bar y(t_0)= y(t_0)$ and 
$y^+(t_0)\le \bar y^+(t_0)\le -y^-(t_0)$.
%S:  Note I changed this from $y^+(t_0)\le y(t_0)\le -y^-(t_0)$
(Note that such a $\bar{y}$ is unique if $y$ is differentiable at $t_0$.) 

Then $\bar y\ge y$  for all $t\in [t_0-\varpi\kappa,t_0+\varpi\kappa]\cap \II$; that is, $\bar{y}$ is a barrier of $y$ at $t_0$. (Similarly, by reversing inequalities, for $y''\ge \lambda-\kappa\cdot  y$.) 
\end{subthm}
\end{thm}

The proof is left to the reader.


Note that Theorem ~\ref{y''=<1-ky} in particular implies that $y$ satisfies $y''\ge \lambda$ ( $y''\le \lambda$)  on an interval $\II\subset\RR$  if and only if $y(t)-\frac{\lambda}{2}t^2$ is convex (concave) on $\II$.

We will often need the following fact about convergence of derivatives of convex functions:

{\sloppy 

\begin{thm}{Two-shoulder lemma}\label{lem:der-conv-lim}
Let $\II$ be an open interval 
and $f_n\:\II\to\RR$ be a sequence of 
%concave
convex functions. 
Assume the functions $f_n$ pointwise converge to a function $f\:\II\to\RR$.
Then for any $t_0\in \II$,
\[f^\pm(t_0)\le \liminf_{n\to\infty}f^\pm_n(t_0).\]
\end{thm}

}

\parit{Proof.}
Since the $f_n$ are convex, we have $f^+_n(t_0)+f^-_n(t_0)\ge0$, and for any~$t$,
\[f_n(t)\ge f_n(t_0)\pm f^\pm(t_0)\cdot (t-t_0).\]
Passing to the limit, we get
\[f(t)\ge f(t_0)+\left[\limsup_{n\to\infty}f^+_n(t_0)\right]\cdot (t-t_0)\]
for $t\ge t_0$, and 
\[f(t)\ge f(t_0)-\left[\limsup_{n\to\infty}f^-_n(t_0)\right]\cdot (t-t_0)\]
for $t\le t_0$.
Hence the result.
\qeds

\begin{thm}{Corollary}
\label{cor:der-conv-lim}
Let $\II$ be an open interval 
and $f_n\:\II\to\RR$ be a sequence of functions such that $f_n''\le \lambda$ that converge pointwise to a function $f\:\II\to\RR$.
Then: 
\begin{subthm}{} If $f$ is differentiable at $t_0\in \II$, then
\[f'(t_0)=\pm\lim_{n\to\infty} f^\pm_n(t_0).\]
\end{subthm}

\begin{subthm}{} If all $f_n$ and $f$ are differentiable at $t_0\in \II$, then
\[f'(t_0)=\lim_{n\to\infty} f'_n(t_0).\]
\end{subthm}
\end{thm}

\parit{Proof.} Set $\hat f_n(t)=f_n(t)-\tfrac{\lambda}{2}\cdot t^2$ and $\hat f(t)=f(t)-\tfrac\lambda2\cdot t^2$.
Note that the $\hat f_n$ are concave and $\hat f_n\to \hat f$ pointwise.
Thus the theorem follows from the two-shoulder lemma (\ref{lem:der-conv-lim}).\qeds











\section{Convex functions on a metric space}\label{sec:conv-fun}

In this section we define different types of convexity/concavity
in the context of metric spaces; it will be mostly used for geodesic spaces.
The notation refers to the corresponding second order ordinary differential inequality. 

\begin{thm}{Definition}\label{def:lam-convex}
Let $\spc{X}$ be a metric space.
We say that an upper semicontinuous subfunction $f\:\spc{X}\subto(-\infty,\infty]$ 
satisfies the inequality
\[f''+\kappa\cdot  f\ge \lambda\]
if for any unit-speed geodesic $\gamma$ in $\Dom f$, 
the real-to-real function $y(t)\z= f\circ\gamma(t)$
satisfies 
\[y''+\kappa\cdot  y\ge \lambda\]
in the domain $\set{t}{y(t)<\infty}$;
see the definition in Section~\ref{sec:conv-real}.

We say that a lower semicontinuous subfunction $f\:\spc{X}\subto[-\infty,\infty)$ 
satisfies the inequality
\[f''+\kappa\cdot  f\le \lambda\]
if the subfunction $h=-f$ 
satisfies 
\[h''-\kappa\cdot  h\ge -\lambda.\]

Functions satisfying the inequalities
\[f''\ge \lambda\quad\text{and}\quad f''\le \lambda\]
are called 
\index{convex function}\index{concave function}\index{$\lambda$-convex function}\index{$\lambda$-concave function}\emph{$\lambda$-convex} and \emph{$\lambda$-concave} respectively.

$0$-convex and $0$-concave subfunctions will also be called \emph{convex} and \emph{concave} respectively.

If $f$ is $\lambda$-convex for some $\lambda>0$, $f$ will be called \index{strongly convex function}\emph{strongly convex};
correspondingly, if $f$ is $\lambda$-concave for some $\lambda<0$, $f$ will be called \index{strongly concave function}\emph{strongly concave}.

If for any point $p\in\Dom f$ 
there is a neighborhood $\Omega\ni p$ and a real number $\lambda$
such that the restriction $f|_\Omega$ is $\lambda$-convex (or $\lambda$-concave),
then $f$ is called \index{semiconvex function}\index{semiconcave function}\emph{semiconvex} (respectively \emph{semiconcave}).\end{thm}

Various authors define the class of $\lambda$-convex ($\lambda$-concave) functions differently. 
Their definitions may correspond to $\pm\lambda$-convex ($\pm\lambda$-concave) or $\pm\tfrac\lambda2$-convex ($\pm\tfrac\lambda2$-concave) functions in our definitions.

\begin{thm}{Proposition}\label{prop:conv-comp}
Let $\spc{X}$ be a metric space.
Assume that $f\:\spc{X}\subto \RR$ is a semiconvex subfunction
and $\phi\:\RR\to\RR$ is a nondecreasing semiconvex function.
Then the composition $\phi\circ f$ is a semiconvex subfunction.
\end{thm}

The proof is straightforward.




