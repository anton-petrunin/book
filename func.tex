%%!TEX root = all.tex
%array^
\chapter{Maps and functions}

Here we introduce some classes of maps between metric spaces and develop a language to describe different levels of convexity/concavity of real-valued functions on general metric space.

Changing sign changes a concave function into a convex function and other way around.
Concave functions are used mainly in the $\Alex{}$ spaces,
while convex ones are used  mainly in $\CAT{}$ spaces.



\section{Maps, functions and relatives}

We will often need maps and functions defined on subsets of a metric space. We call them \emph{submaps} and \emph{subfunction}\index{submap}\index{subfunction}.
Thus given a metric space $\spc{X}$ and $\spc{Y}$, 
a submap $\map\:\spc{X}\subto \spc{Y}$ is a map defined on a subset $\Dom\map\subset \spc{X}$.

Almost all subfunctions and sub maps we consider are continuous (or at least semicontinuous).
In this case the domain $\Dom\map$ is automatically open.
Indeed, the continuous maps are defined as those for which preimage of open set is open.
Assume $\map\:\spc{X}\subto \spc{Y}$ is a continuous submap.
Since $\spc{Y}$ is open then so is $\map^{-1}(\spc{Y})$ 
which is the same as $\Dom \map$.
The same is true for upper and lower semicontinuous functions $f\:\spc{X}\subto \RR$ since it is a continuous function for special topology on $\RR$.
\footnote{The continuous partially defined maps can be defined differently.
For example, one may request that inverse images of closed sets are closed.
While this condition is equivalent to the continuity for the functions defined on whole space,
it is different for partially defined functions. 
In particular with this definition domain of the closed-continuous map must be closed.}

\begin{thm}{Isometry}
Let $\spc{X}$ and $\spc{Y}$ be metric spaces
and $\map\:\spc{X}\to \spc{Y}$ be a map
\begin{subthm}{}
The map $\map$ is \emph{distance preserving}\index{distance preserving map} if
$$|\map(x)-\map(x')|_{\spc{Y}}=|x-x'|_{\spc{X}}$$
for any $x,x'\in X$.
\end{subthm}

\begin{subthm}{}
A distance preserving bijection $\map$ is called an \emph{isometry}\index{isometry}.
\end{subthm}

\begin{subthm}{}
The spaces $X$ and $Y$ are called \emph{isometric}\index{isometric spaces} (briefly $X\iso Y$)
 if there is an isometry  $\map\:X\to Y$.
\end{subthm}

\end{thm}


\begin{thm}{Lipschitz maps}
Let $\spc{X}$ and $\spc{Y}$ be metric spaces 
$\map\:\spc{X}\subto\spc{Y}$ be a submap 
and $\Lip\in\RR$.

\begin{subthm}{}
The submap $\map$ is called \emph{$\Lip$-Lipschitz}\index{$\Lip$-Lipschitz map} if
\[\dist{\map(x)}{\map(y)}{\spc{Y}}
\le
\Lip\cdot
\dist{x}{y}{\spc{X}}\]  
for any two points $x,y\in\Dom \map$.

\begin{itemize}
 \item The $1$-Lipschitz maps will be also called \emph{short}\index{short map}.
\end{itemize}

\end{subthm}

\begin{subthm}{}
We say that $\map$ is \emph{Lipschitz}\index{Lipschitz map} if it is $\Lip$-Lipschitz for some constant $\Lip$.
The minimal such constant is denoted by $\lip\map$.
\end{subthm}

\begin{subthm}{}
We say that $\map$ is \emph{locally Lipschitz}\index{locally Lipschitz map} 
if any point $x\in\Dom \map$ admits a neighborhood 
$\Omega\subset \Dom\map$ such that the restriction $\map|\Omega$ is Lipschitz.
\end{subthm}

\begin{subthm}{}
Given $p\in\Dom \map$, we denote by $\lip_p\map$ the infimum of the real values $\Lip$ such that
$p$ admits  a neighborhood 
$\Omega\subset \Dom\map$ such that the restriction $\map|\Omega$ is $\Lip$-Lipschitz.
\end{subthm}
\end{thm}

Note that $\map\:\spc{X}\to\spc{Y}$ is $\Lip$-Lipschitz if
\[\map(\oBall(x, R)_{\spc{X}})\subset\oBall(\map(x),\Lip\cdot R)_{\spc{Y}}\]
for any $R\ge 0$ and $x\in \spc{X}$.
The following definition gives a dual version of it.

\begin{thm}{Definitions}
Let $\spc{X}$ and $\spc{Y}$ be metric spaces, 
$\map\:\spc{X}\to\spc{Y}$ be a map 
and $\Lip\in\RR$.
\begin{subthm}{}
The map $\map$ is called \emph{$\Lip$-co-Lipshitz}\index{$\Lip$-co-Lipshitz map} if 
\[\map(\oBall(x,\Lip\cdot R)_{\spc{X}})\supset\oBall(\map(x),R)_{\spc{Y}}\]
for any $x\in \spc{X}$ and $R>0$.
\end{subthm}

\begin{subthm}{}
The map $\map$ is called \emph{co-Lipshitz}\index{co-Lipshitz map} if it is $\Lip$-co-Lipshitz
for some constant $\Lip$.
The minimal such constant is denoted by $\colip\map$.
\begin{itemize}
 \item The map $\map$ is called \emph{submetry}\index{submetry} if it is short and 1-co-Lipschitz.
\end{itemize}

\end{subthm}
\end{thm}

The $\Lip$-co-Lipschitz maps 
can be considered as a quantitative version of open maps.
By that reason they also called $\Lip$-open; see \cite{BGP}.
Also, be aware that some authors, 
refer to our $\Lip$-co-Lipschitz maps
as to $\tfrac1\Lip$-co-Lipschitz;
see ???.

\begin{thm}{Proposition}
Any co-Lipschits map is open and surjective.
\end{thm}

\parit{Proof.} Follows from the definition directly.
\qeds

\begin{thm}{Proposition}\label{prop:colip=>complete}
Let $\spc{X}$ and $\spc{Y}$ be metric spaces and
$\map\: \spc{X}\to\spc{Y}$ is a continuous co-Lipschitz map 
Then $\spc{Y}$ is complete.
\end{thm}

\parit{Proof.}
Choose a converging in itself sequence $y_1,y_2,\dots$ in $\spc{Y}$.
Passing to a subsequence if nececury, we may assume that $\dist{y_n}{y_{n+1}}{\spc{Y}}< \tfrac1{2^n}$ for each $n$.

Denote by $\Lip$ a co-Lipschitz constant of $\map$.
Note that  there is a sequence $x_1,x_2,\dots$ in $\spc{X}$
such that
\[\map(x_n)=y_n\ \ \text{and}\ \ \dist{x_n}{x_{n+1}}{\spc{X}}< \tfrac{\Lip}{2^n}.\eqlbl{eq:colip+1/2n}\]
for each $n$. 
Indeed, such a sequence $(x_n)$ can be constructed recursively. 
Assuming that the points $x_1,\dots,x_{n-1}$ are already constructed, 
the existence of $x_n$ satisfying \ref{eq:colip+1/2n}
follows since $\map$ is $\Lip$-co-Lipschitz.

Notice that $(x_n)$ converges in itself in $\spc{X}$.
Denote by $x_\infty$ its limit
and set $y_\infty= \map(x_\infty)$.
Since $\map$ is continuous,
$y_n\to y_\infty$ as $n\to\infty$.
Hence the result follows.
\qeds

\section{Sumbetry}\label{sec:quotient-CBB}

A map $\sigma\:\spc{L}\to\spc{M}$ between the metric spaces $\spc{L}$ and $\spc{M}$
is called 
\emph{submetry}\index{submetry} if 
\[\sigma(\oBall(p,r)_\spc{L})=\oBall(\sigma(p),r)_{\spc{M}}\]
for any $p\in \spc{L}$ and $r\ge 0$.

Equivalently, a map $\sigma\:\spc{L}\to\spc{M}$ is called sumbetry if it is 1-Lipshitz and 1-co-Lipschitz at the same time.

Note that any submetry is an onto map.

The main souse of examples of submetries comes from isometric group actions.

Nemely, assume $\spc{L}$ is a metric space and the group $G$ acts on $\spc{L}$.
Denote by $\spc{L}/G$ the set of $G$-orbits;
let us equip it with the Hausdorff metric. 

In general, $\spc{L}/G$ is a premetric space,
but if all the $G$-orbits form closed sets in $\spc{L}$
then $\spc{L}/G$ is a metric space.

\begin{thm}{Proposition}\label{prop:submet/G}
Let $\spc{L}$ be a metric space and group $G$  acts on $\spc{L}$ by isometries  
and in such a way that every $G$-orbit is closed.
Then the projection map $\spc{L}\to \spc{L}/G$ is a sumbetry.
\end{thm}

%???should it be proved???


\begin{thm}{Proposition}
\label{prop:submet-length}
Let  $\spc{L}$ be a length space 
and $\sigma\:\spc{L}\to \spc{M}$ is a submetry.
Then $\spc{M}$ is a length space.
\end{thm}

\parit{Proof.}
Fix $\eps>0$ and a pair of points $x,y\in \spc{M}$.

Since $\sigma$ is $1$-co-Lipschitz, there are points $\hat x,\hat y\in \spc{L}$
such that $\sigma(\hat x)=x$, $\sigma(\hat y)=y$ 
and $\dist{\hat x}{\hat y}{\spc{L}}<\dist{x}{y}{\spc{M}}+\eps$.

Since ${\spc{L}}$ is a length space, 
there is a curve $\gamma$ 
joining $\hat x$ to $\hat y$ in ${\spc{L}}$
such that
\[\length\gamma\le \dist{x}{y}{\spc{M}}+\eps.\]

Since $\sigma$ is $1$-Lipschitz,
there 
\[\length\sigma\circ\gamma\le \length\gamma.\]

Note that the curve $\sigma\circ\gamma$ joins $x$ to $y$
and from above
\[\length\sigma\circ\gamma<\dist{x}{y}{\spc{M}}+\eps.\]
Since $\eps>0$ is arbitrary,
we have that $\spc{M}$ is a length space.
\qeds

\section{Speed of curves}

Let $\spc{X}$ be a metric space.
A \emph{curve}\index{curve} in $\spc{X}$ is a continuous map $\alpha\:\II\to \spc{X}$, where $\II$ is a \emph{real interval}\index{real interval} (that is, an arbitrary convex subset of $\RR$);
the curve it called \emph{Lipschitz}\index{Lipschitz curve} or \emph{locally Lipschitz}\index{locally Lipschitz curve} if the map $\alpha\:\II\to \spc{X}$ is.

%??? Maybe we only need locally Lipschitz curves, A.???

The length of the curve $\alpha\:\II\to \spc{X}$, 
is defined by
\[\length\alpha=\sup\set{\sum_n\dist{\alpha(t_{n-1})}{\alpha(t_n)}{\spc{X}}}{t_0<t_1<\dots<t_\kay}.\]
The following lemma is an easy exercise.
%The same definition works for arbitrary map $\alpha\:\II\to \spc{X}$, not necessary continuous.

\begin{thm}{Lower semicontinuity of length}\label{thm:semicont-of-length}
Assume $\alpha_n\:\II\to \spc{X}$ is a sequence of curves which converges pointwise to a curve $\alpha_\infty\:\II\to \spc{X}$.
Then 
\[\length\alpha_\infty\le \liminf_{n\to\infty}\length\alpha_n.\]

\end{thm}



The following theorem follows from \cite[2.7.4]{BBI}.

\begin{thm}{Theorem}\label{thm:speed}
Let $\spc{X}$ be a metric space  
and $\alpha\:\II \to \spc{X}$ be a locally Lipschitz
curve. 
Then the speed function
\[\speed\alpha(t_0)=\lim_{t\to t_0}\frac{\dist{\alpha(t)}{\alpha(t_0)}{}}{|t-t_0|}\] 
is defined for almost all $t_0 \in \II$ and 
\[\length\alpha=\int\limits_\II \speed\alpha(t)\cdot\d t,\]
where $\int$ denotes the Lebesgue integral.
\end{thm}

A curve $\alpha\:\II\to\spc{X}$ is called \emph{unit-speed curve}\index{unit-speed curve}, 
if for any subinterval $[a,b]\subset\II$, we have
\[b-a=\length(\alpha|[a,b]).\]
According to the above theorem, this is equivalent to the condition that $\alpha$ is Lipschitz and $\speed\alpha\ae 1$.

The following theorem from \cite{kirchheim} %CHECK??? 
generalize the one above.


\begin{thm}{Theorem}
Let $\spc{X}$ be a metric space 
and $f\:\RR^n \subto \spc{X}$ be $1$-Lipschitz. 
Then for almost all $x\in\Dom f$ there is a prenorm $\lVert*\rVert_x$ on $\RR^n$ such that
we have
\[\dist{f(y)}{f(z)}{\spc{X}}=\lVert z-y\rVert_x+o(|y-x|+|z-x|).\]
Moreover if $s(x)=\sup\set{\tfrac{\lVert w\rVert_x}{|w|}}{w\not=0}$, then 
\[\HausMes_n[\Im f] \le \int\limits_{\Dom f}s(x).\]
\end{thm}


The norm $\lVert*\rVert_x$ in the above theorem 
will be called \emph{differential of induced metric} at $x$.

\section{Convexity of real-to-real functions}\label{sec:conv-real}

We will be interested in the solutions
of the following differential inequalities
\[y''+\kappa\cdot  y\ge \lambda
\ \ \t{(and respectively}
\ \ y''+\kappa\cdot  y\le \lambda\t{)}\eqlbl{eq:sec:conv-real*}\]
for fixed $\kappa,\lambda\in\RR$.
The solution $y\:\RR\subto\RR$ are only assumed to be lower%???
 (respectively upper%???
) semicontinuous subfunctions.

The inequalities  \ref{eq:sec:conv-real*} are understood in the sense of distributions.
That is, for any smooth function $\phi$ with compact support $\supp\phi\subset\Dom y$ the following inequality should be satisfied:
\[\int\limits_{\Dom y}\l[y(t)\cdot\phi''(t)+\kappa\cdot  y(t)\cdot\phi(t)-\lambda\r]\cdot\d t
\ge 0\ \ \t{(and respectively}\ \ \le0\t{)}.
\eqlbl{eq:distr-conc}\]
The integral is understood in the sense of Lebesgue,
in particular the inequality \ref{eq:distr-conc}
makes sense for any Borel-measurable subfunction $y$.


\begin{thm}{Proposition}
Let $\II\subset\RR$ is an open interval and $y_n\:\II\to\RR$ be a sequence of solutions of one of the inequality in \ref{eq:sec:conv-real*}.
Assume $y_n(t)\to y_\infty(t)$ as $n\to\infty$ for any $t\in \II$.
Then $y_\infty$ is a solution of the same inequality in \ref{eq:sec:conv-real*}.
\end{thm}

Assume $y$ is a solution of one of the inequality in \ref{eq:sec:conv-real*}.
For $t_0\in \Dom y$, let us define right (left) derivative $y^+(t_0)$ ($y^-(t_0)$) at $t_0$ by
\[y^\pm(t_0)=\lim_{t\to t_0\pm} \frac{y(t)-y(t_0)}{|t-t_0|}.\]
The sign convention we use for $y^-$ is not standard, thus for $y(t)=t$ we have
$y^\pm(t)=\pm 1$.

\begin{thm}{Proposition}\label{prop:derivative-of-convex-function}
Let $\II\subset\RR$ is an open interval and $y_n\:\II\to\RR$ be a sequence of solutions of one of the inequality in \ref{eq:sec:conv-real*}.
Then right  and left derivatives $y^+(t_0)$ and $y^-(t_0)$ are defined
for any $t_0\in\II$.
\end{thm}


The next theorem gives a  number of equivalent ways to define such a generalized solution.

\begin{thm}{Theorem}\label{y''=<1-ky}
Let $\II$ be an open real interval and $y\:\II\to\RR$ be a locally Lipschitz function
then the following conditions are equivalent:
\begin{subthm}{}$y''\ge \lambda-\kappa\cdot  y$ (respectively $y''\le \lambda-\kappa\cdot  y)$
\end{subthm}

\begin{subthm}{barrier}(barrier inequality) For any $t_0\in \II$, 
there is a solution $\bar y$ 
of ordinary differential equation $\bar y''=\lambda-\kappa\cdot  \bar y$ 
with $\bar y(t_0)= y(t_0)$ such that $\bar y\ge y$ (respectively $\bar y\le y$) for all $t\in [t_0-\varpi\kappa,t_0+\varpi\kappa]\cap \II$.

The function $\bar y$ is called \emph{lower} (respectively \emph{upper}) \emph{barrier} of $y$ at $t_0$\index{barrier}.
\end{subthm}

\begin{subthm}{barrier'} 
Suppose $y''\le \lambda-\kappa\cdot  y$ . Let $t_0\in\II$ and let  $\bar y$ be a solution of  the
 ordinary differential equation $\bar y''=\lambda-\kappa\cdot  \bar y$ 
such that  $\bar y(t_0)= y(t_0)$ and $y^+(t_0)\le y(t_0)\le -y^-(t_0)$. (note that such $\bar{y}$ is unique if $y$ is differentiable at $t_0$). 

Then $\bar y\ge y$  for all $t\in [t_0-\varpi\kappa,t_0+\varpi\kappa]\cap \II$; that is $\bar{y}$ is a barrier of $y$ at $t_0$.
\end{subthm}

\begin{subthm}{y''-mono} (Jensen's inequality)\index{Jensen's inequality}
For any pair of values $t_1<t_2$ in $\II$, such that $|t_2-t_1|<\varpi\kappa$ the unique solutions $z(t)$ of \[z''\z=\lambda-\kappa\cdot  z\] such that
\[z(t_1)=y(t_1),\ \ z(t_2)=y(t_2)\] 
satisfies $y(t)\le z(t)$ (respectively $y(t)\ge z(t)$) for all $t\in[t_1,t_2]$.
\end{subthm}
\end{thm}

\parit{Proof.} It is left to the reader.\qeds


We will often need the following simple fact about convergence of derivative of convex functions:

\begin{thm}{Lemma on equilibrium}\label{lem:der-conv-lim}
Let $\II$ be an open interval 
and $f_n\:\II\to\RR$ be a sequence of concave functions. 
Assume the functions $f_n$ pointwise converge to a function $f\:\II\to\RR$.
Then for any $t_0\in \II$,
\[f^\pm(t_0)\le \liminf_{n\to\infty}f^\pm_n(t_0).\]
\end{thm}

\parit{Proof.}
Since $f_n$ are convex, we have $f^+_n(t_0)+f^-_n(t_0)\ge0$, and for any $t$,
\[f_n(t)\ge f_n(t_0)\pm f^\pm(t_0)\cdot (t-t_0).\]
Passing to the limit, we get
\[f(t)\ge f(t_0)+\l[\limsup_{n\to\infty}f^+_n(t_0)\r]\cdot (t-t_0)\]
for $t\ge t_0$, and 
\[f(t)\ge f(t_0)-\l[\limsup_{n\to\infty}f^-_n(t_0)\r]\cdot (t-t_0)\]
for $t\le t_0$.
Hence the result.
\qeds

\begin{thm}{Corollary}
\label{cor:der-conv-lim}
Let $\II$ be an open interval 
and $f_n\:\II\to\RR$ be a sequence of functions which satisfy $f_n''\le \lambda$ and converge pointwise to a function $f\:\II\to\RR$.
Then: 
\begin{subthm}{} If $f$ is differentiable at $t_0\in \II$, then
\[f'(t_0)=\pm\lim_{n\to\infty} f^\pm_n(t_0).\]
\end{subthm}

\begin{subthm}{} If all $f_n$ and $f$ are differentiable at $t_0\in \II$, then
\[f'(t_0)=\lim_{n\to\infty} f'_n(t_0).\]
\end{subthm}
\end{thm}

\parit{Proof.} Set $\hat f_n(t)=f_n(t)-\tfrac{\lambda}{2}\cdot t^2$ and $\hat f(t)=f(t)-\tfrac\lambda2\cdot t^2$.
Note that $\hat f_n$ are concave and $\hat f_n\to \hat f$ pointwise.
Thus, theorem follows from the lemma on equilibrium (\ref{lem:der-conv-lim}).\qeds











\section{Convexity of functions on metric space.}\label{sec:conv-fun}

Although the following definitions has sense for general metric space,
-
\begin{thm}{Definition}\label{def:lam-convex}
Let $\spc{X}$ be a metric space.
A subfunction $f\:\spc{X}\subto(-\infty,\infty]$ is called \emph{semiconvex}\index{semiconvex}
if 
for any point $p\in \Dom f$ there is $\lambda\in\RR$ and a neighborhood $\Omega_p\subset \Dom f$ 
such for any unit-speed geodesic $\gamma$ in $\Omega_p$ 
the real-to-real function $t\mapsto f\circ\gamma(t)-\tfrac{\lambda}{2}\cdot t^2$ is convex.


A subfunction $f\:\spc{X}\subto[-\infty,\infty)$ is called \emph{semiconcave}\index{semiconcave} 
if the subfunction $(-f)$ is semiconvex.
\end{thm}

Further, we give a meaning of certain type of second order ordinary differential inequalities in context of metric spaces.

\begin{thm}{Definition}\label{def:f''}
Let $\spc{X}$ be a metric space 
and $\phi\:\RR^2\to\RR$ be a Lipschitz function.
Given a semiconvex (semiconcave) subfunction $f\:\spc{X}\subto\bar\RR$, we will write
\[f''\ge \phi(f,|f'|)\ \ \l(\t{correspondingly}\ \  f''\le \phi(f,|f'|)\ \r)\]
if for any geodesic $\gamma$ in $\Dom f$ the real-to-real function $y=f\circ\gamma$
satisfies the differential inequality
\[y''\ge \phi(y,|y'|)\ \ \l(\t{correspondingly}\ \  y''\le \phi(y,|y'|)\r)\]
in the sense of ???.
\end{thm}

%??? ADD strict inequalities???

For $\lambda\in\RR$, the solutions of differential inequalities 
\[f''\ge \lambda \ \ \t{and}\ \ f''\le\lambda\]
will be called correspondingly \emph{$\lambda$-convex}\index{$\lambda$-convex} and \emph{$\lambda$-concave}\index{$\lambda$-concave} subfunctions.
If $\lambda=0$,
$0$-convex and $0$-concave subfunctions will be also called \emph{convex} and \emph{concave} correspondingly.
If $f$ is $\lambda$-convex for some $\lambda>0$ it will be called \emph{strongly convex};
correspondingly if $f$ is $\lambda$-concave for some $\lambda<0$ it will be called \emph{strongly concave}.

\begin{thm}{Proposition}\label{prop:conv-comp}
Let $\spc{X}$ be a metric space.
Assume that $f\:\spc{X}\subto \RR$ is a semiconvex subfunction
and $\phi\:\RR\to\RR$ is a nondecreasing seimconvex function.
Then the composition $\phi\circ f$ is a semiconvex subfunction.
\end{thm}


\parbf{NB!}
Various authors define the class of $\lambda$-convex ($\lambda$-concave) function differently. 
It may correspond to $\pm\lambda$-convex ($\pm\lambda$-concave) or $\pm\tfrac\lambda2$-convex ($\pm\tfrac\lambda2$-concave) function in our definitions.

\medskip

Further, two classes of functions, satisfying differential inequalities
\[f''\ge \lambda-\kappa\cdot  f
\ \ \text{and}\ \ 
f''\le \lambda-\kappa\cdot  f\]
will play a very important role.
The solutions of these differential inequalities should be understood to possess a special form of convexity and concavity that is optimized to reflect properties of spaces with curvature $\le \kappa$ and $\ge\kappa$.  
If $\kappa\not=0$,
by adding the constant $-\lambda/\kappa$ to these solutions, we obtain the solutions of the differential inequalities
\[f''\ge -\kappa\cdot  f
\ \ \text{and}\ \ 
f''\le -\kappa\cdot  f,\]
which also arise naturally.


%%%%%%%%%%%%%%%%%%%%%%%%%%%%%%%%%%%%%%%%%%%%%%%%%%%
%%%%DOWN

\section{Miscellaneous}

\begin{thm}{Convexity of the limit}

 
\end{thm}


Tietze extension theorem???

\begin{thm}{Lemma}\label{lem:lip-approx}
Let $\spc{X}$ be a metric space and $f\:\spc{X}\to\RR$ be a continuous function.
Then for any $\eps>0$ there is a locally Lipschitz function $f_\eps\:\spc{X}\to\RR$
such that $|f(x)-f_\eps(x)|<\eps$ for any $x\in \spc{X}$.
\end{thm}

\parit{Proof.}
Assume that $f\ge 1$.
Construct a continous positive function $\rho\:\spc{X}\to \RR_{>0}$ such that 
\[\dist{x}{y}{}<\rho(x)\ \ \Rightarrow\ \ |f(x)-f(y)|<\eps.\]
Consider function
\[
f_\eps(x)
=
\sup\set{f(x)
\cdot
\l(1-\tfrac{\dist{x}{y}{}}{\rho(x)}\r)}%
{x\in\spc{X}}.
\]
It is straightforward to check that each $f_\eps$ is locally Lipschtz and $0\le f_\eps-f<\eps$.

Note that any continuous function can be presented as difference of two continuous functions bounded below by $1$.
Hence the result.
\qeds
