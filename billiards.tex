\chapter{Billiards}
\section{Piecewise concave table}

\begin{wrapfigure}[7]{r}{37mm}
\begin{lpic}[t(-0mm),b(-0mm),r(0mm),l(0mm)]{pics/table(.5)}
\end{lpic}
\end{wrapfigure}

In this section, we shall consider a billiard table 
formed by Euclidean space with a finite collection of convex sets removed.

Let $\{A_1,A_2,\dots,A_n\}$ be a finite collection of sets in $\RR^m$.
Consider the table
$$T=\Closure\left(\RR^m\backslash \bigcup_{i=1}^n A_i\right).$$
Assume in addition that
\begin{enumerate}[(a)]
\item\label{A-convex} Each $A_i$ is closed and convex.
\item\label{A-intersection} All $A_i$ have nonempty intersection;
that is, 
$$\bigcap_{i=1}^n A_i\not=\emptyset.$$
\item\label{A-bry} The boundaries $\partial A_1,\partial A_2,\dots,\partial A_n$ are \emph{smooth hypersurfaces}%%%???DEFINE???
.
\item\label{A-angle} There is $\alpha<\pi$,
such that for any $i$ and $j$,
the angle between outer normals to $\partial A_i$ and $\partial A_j$ at any common point $p\in \partial A_i\cap\partial A_j$ is at most $\alpha$.
\end{enumerate}

\begin{thm}{Exercise}
Assume that the collection of convex sets $A_i$ in $\RR^m$ satisfies conditions (\ref{A-intersection}) and (\ref{A-bry}) above and in addition, all $A_i$ are bounded.
Show that condition (\ref{A-angle}) is also satisfied.
\end{thm}

A billiard trajectory on such table is a unit-speed broken line 
with the break points at walls (that is, at $\partial A_i$),
such that at every break
point the left and right velocities have equal projections onto the tangent
plane to the wall.
We assume that trajectory meets one $A_i$ at the time.

The number of break
points of the trajectory will be called \emph{number of collisions} of the trajectory.

\begin{thm}{Collision theorem}\label{thm:baby-collision}
Assume 
$$T=\Closure\left(\RR^m\backslash \bigcup_{i=1}^n A_i\right)$$
is a billiard table which satisfies conditions (\ref{A-convex})--(\ref{A-angle}) above.
Then the number of collisions of any trajectory in  $T$  is bounded
by a number $N$ which depends only on $n$ and $\alpha$.

\end{thm}

In the proof we will need the following lemma in convex geometry.

\begin{thm}{Lemma}\label{lem:sina}
Let $\{A_1,A_2,\dots,A_n\}$ be a collection of sets in $\RR^m$
which satisfies conditions (\ref{A-convex})--(\ref{A-angle}).
Set $K=\bigcap_{i=1}^n A_i$, then
$$\dist{K}{}{}\ge \sin\alpha \cdot \max_i\{\dist{A_i}{}{}\}.$$

Moreover, assume 
\[e_i(x,y)=\min_{a\in A_i}\{|x-a|+|a-y|-|x-y|\}\] 
\[E(x,y)=\min_{a\in K}\{|x-a|+|a-y|-|x-y|\}\]
for the given two points $x,y\in \RR^m$, 
then
$$E(x,y)\ge \sin\alpha 
\cdot \max_i\{e_i(x,y)\}.$$

\end{thm}

\parit{Proof.}
???
\qeds


\parit{Proof of Collision theorem.}
The proof is by induction on $n$.

The base case $n=1$ is evident; the number of collisions cannot exceed $1$.  
It follows from the convexity of $A_1$ that
if the trajectory is reflected once in $\partial A_1$, 
then it cannot return to $A_1$.

The proof of the first step $n=2$ is slightly simpler than the remaining steps.
To simplify the presentation, we prove the case $n=2$, 
and then describe the necessary modifications to prove the $n=3$ case.
Once this is done, the proof of the general case should be evident.

\parit{First step; $n=2$.} To reduce number of indices, set $A=A_1$ and $B=A_2$. 

Note that any trajectory hits $A$ and $B$ in turn;
that is, the trajectory can not hit $A$ (as well as $B$)
twice in a raw.

Fix a large number $N$ and assume a trajectory has at least $2\cdot N$ collisuion.
Then it should hit $A$,
then $B$,
then $A$
and so on $2\cdot N$ times
and then finally $B$.
Denote by 
$a_1\in \partial A$, 
$b_1\in \partial B,
\dots
b_N\in \partial B$
the brake points of this trajectory.

Prepare $2\cdot N$ copies of $\RR^m$,
say $\mathcal{R}_1$, $\mathcal{R}_2,\dots,\mathcal{R}_{2\cdot N}$.
Each $\mathcal{R}_i$ contains a copy of $A$ and $B$, which will be denoted as $A_i$ and $B_i$.

Let us glue a new space say $\mathcal{R}$ out of $\mathcal{R}_i$'s
by gluing 
\begin{enumerate}
\item $\mathcal{R}_1$ to $\mathcal{R}_2$ by identifying $A_1$ and $A_2$;
\item $\mathcal{R}_2$ to $\mathcal{R}_3$  by identifying $B_2$ and $B_3$;
\item $\mathcal{R}_3$ to $\mathcal{R}_4$  by identifying $A_3$ and $A_4$;
\item and so on;
\item $\mathcal{R}_{2\cdot N-1}$ to $\mathcal{R}_{2\cdot N}$ by identifying $A_{2\cdot N-1}$ and $A_{2\cdot N}$.
\end{enumerate}
According to Reshetnyak's gluing theorem (\ref{thm:reshetnyak}), $\mathcal{R}\in\cCat{}{0}$.
Further we will view $\mathcal{R}_i$ as subsets of $\mathcal{R}$.

Denote by $f_i\:\RR^m\to\mathcal{R}$ the distance preserving map which identifies $\RR^m$ and $\mathcal{R}_i$
and let $F\:\mathcal{R} \to \RR^m$ be the natural projection which identifies all $\mathcal{R}_i$ with $\RR^m$.

Consider the broken geodesic $\gamma$ connecting points $f_1(a_1),f_2(b_2),f_3(a_3),\dots,f_{2\cdot N}$ in $\mathcal{R}$.
Note that $\gamma$ forms a local geodesic in $\mathcal{R}$.
Therefore, according to ??? $\gamma$ is a geodesic with end points in $A_1\cup A_{2\cdot N}$,
but all interior points of $\gamma$ do not belong to $A_1\cup A_{2\cdot N}$.

Hence in order to arrive to a contradiction, 
it is sufficient to prove the following claim.
 
\begin{clm}{}
The union $A_1\cup A_{2\cdot N}$ is a convex set in $\mathcal{R}$.
\end{clm}

Due to the convexity of $A_1$ and $A_{2\cdot N}$,
it is sufficient to show that for any points $p\in A_1\backslash A_{2\cdot N}$ and $q\in A_{2\cdot N}\backslash A_1$,
the geodesic $[pq]$ contain a point $x\in K\z=A_1\cap A_{2\cdot N}$.

Assume contrary.
Mark points $p=p_1,q_2,p_3,\dots,p_{2\cdot N}=q$ of $[pq]$ such that $p_i\in A_i$ and $q_j\in B_j$.
Note that $[p_{2\cdot i-1}q_{2\cdot i}]\in \mathcal{R}_{2\cdot i-1}$ 
and 
$[q_{2\cdot i}p_{2\cdot i+1}]\in \mathcal{R}_{2\cdot i}$ for each $i$.
We will construct a curve passing throug $K$ and connecting $p$ to $q$ which is shorter than $[pq]$;
this way we arrive to a contradiction. 




\parit{Second step; $n=3$.}



\qeds

The Baby collision theorem (\ref{thm:baby-collision}) admits a straightforward generalization to higher dimensions and to an arbitrary finite number of sets $B_i$.
Namely, the following result holds:

\begin{thm}{Adult collision theorem}\label{thm:adult-collision}
Let $B_1,B_2,\dots,B_n$ be a finite collection of open convex sets in $\RR^m$.
Assume each $B_i$ contains the origin $0\in\RR^m$
and for each $i$, the boundary $W_i=\partial B_i$ is a smooth hypersurface.
Further, assume there is $\eps>0$ such that for any point $p\in W_i\cap W_j$ the angle between the tangent 
hyperplane of $W_i$ and $W_j$ at $p$ is at least $\eps$.

Then 
there is a natural number $N$ which depends only on $\eps$
such that the number of collisions of any trajectory in $T$ is at most $N$.
\end{thm}

The proof of the Collision theorem can be given along the same lines as its baby case.
We glue the space in a similar way:
Take a copy of $\RR^m$
and glue to it $n$ copies of $\RR^m$, along each $B_i$.
Further glue to each copy $n-1$ copies of $\RR^m$... 

Now we will show how to apply the above theorem to prove the following. 

\section{On number of collisions of balls}

\begin{thm}{Theorem}\label{thm:balls}
Consider $n$ identical homogeneus hard balls
moving freely and colliding
elastically in empty space $\RR^3$. 
Every ball moves
along a straight line with constant speed until two balls collide, and then
the new velocities of the two balls are determined by the
laws of classical mechanics.

Then the total number of collisions cannot exceed some number $N$ which depends only on $n$.
\end{thm}

\parit{Proof.}
A position of a collection of $n$ balls can be represented by a point in $\RR^{3\cdot n}$.
If $a_i=(x_i,y_i,z_i) \in \RR^3$ is the center of the $i$-th ball
then
the corresponding point in $\RR^{3\cdot N}$ is
$$(a_1, a_2 , \dots , a_n ) = (x_1, y_1 , z_1 , x_2 , y_2 , z_2 , \dots , x_n , y_n , z_n).$$
Not every point in $\RR^{3\cdot n}$ represents a valid configuration of balls. 
We have to exclude positions where some of the balls overlap. 
The $i$-th and $j$-th ball intersect if 
$$|a_i - a_j | < 2.$$ 
This inequality defines a cylinder $C_{ij} \subset \RR^{3\cdot n}$. 
The complement
$$\RR^{3\cdot n}\backslash \bigcup_{i\ne j} C_{ij}$$ is the configuration space of our system. 
Its points correspond
to valid positions of the system of balls.

The evolution of the system
of balls traces a path in the configuration space. 
It is easy to verify that
the point representing the configuration of balls moves straight and at a
constant speed until it hits one of the cylinders $C_{ij}$ (this event corresponds
to a collision in the system of balls), and then it continues following the
standard law of billiard collision: 
the angle of reflection is equal to the angle
of incidence. 

Note that each cylinder $C_{ij}$ is an open convex set with smooth boundaries which contains a unit ball around the origin.
It is easy to check that the sets $C_{ij}$ satisfy the conditions (\ref{A-convex})--(\ref{A-angle}) on page \pageref{A-convex};
we can take $\alpha=\tfrac\pi3$ in (\ref{A-angle}).
Therefore our theorem follows 
from Theorem~\ref{thm:baby-collision}.
\qeds