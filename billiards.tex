\chapter{Billiards}
\section{Piecewise concave table}

\begin{wrapfigure}[7]{r}{37mm}
\begin{lpic}[t(-0mm),b(-0mm),r(0mm),l(0mm)]{pics/table(.5)}
\end{lpic}
\end{wrapfigure}

In this section, we shall consider a billiard table 
formed by Euclidean space with a finite collection of convex sets removed.

Let $\{A_1,A_2,\dots,A_n\}$ be a finite collection of sets in $\RR^m$.
We will consider billiard table $T$ formed by the complement
$$T=\Closure\left(\RR^m\backslash \bigcup_{i=1}^n A_i\right).$$
We will make the following assumption on $A_i$.

\begin{thm}{Table condition}\label{condition:A}
We say that a billiard table $T\subset\RR^m$ is dispersing if 
$$T=\RR^m\backslash \bigcup_{i=1}^n A_i$$
for a finite collection of sets $\{A_1,A_2,\dots,A_n\}$ in $\RR^m$ 
such that
\begin{subthm}{A-convex} Each $A_i$ is open and convex.
\end{subthm}

\begin{subthm}{A-bry} The boundaries $\partial A_1,\partial A_2,\dots,\partial A_n$ are \emph{smooth hypersurfaces}.
\end{subthm}

The surfaces $\partial A_i$ will be called \emph{walls} of $T$.

We say that $T$ is has central point 
if 

\begin{subthm}{A-intersection} All $A_i$ have nonempty intersection.
\end{subthm}

We say that $T$ has $\eps$-opened  corner if

\begin{subthm}{A-angle} For any $p\in\RR^m$ there is a vector $\nu$
such that if $p\in\partial A_i$ for some $i$ and $\nu_i$ is the outer normal vector to $\partial A_i$ at $p$ then $\mangle(\nu,\nu_i)<\tfrac\pi2-\eps$.
\end{subthm}
\end{thm}

Note that condition (\ref{SHORT.A-angle}) is nontrivial 
only for the points $p$ which lie 
on the intersection 
of two or more boundaries $\partial A_i$.

\begin{thm}{Exercise}
Assume that the collection of convex sets $A_i$ in $\RR^m$ satisfies conditions (\ref{SHORT.A-convex})--(\ref{SHORT.A-intersection}) above and in addition, all $A_i$ are bounded.
Show that condition (\ref{SHORT.A-angle}) is also satisfied for some $\eps>0$.
\end{thm}

\begin{thm}{Exercise}
Assume that the collection of convex sets $A_i$ in $\RR^m$ satisfies conditions (\ref{SHORT.A-convex})--(\ref{SHORT.A-intersection}) above and in addition, each $A_i$ is centrally symmetric; that is 
\[x\in A_i\iff -x\in A_i.\]
Show that condition (\ref{SHORT.A-angle}) is also satisfied for some $\eps>0$.
\end{thm}

A \emph{billiard trajectory} 
on this table is a unit-speed broken line $\gamma$
with the \emph{regular reflection} at the points on $\partial A_i$ the angle of reflection is equal to the angle
of incidence.
(More precisely,  if $\gamma(t_0)\in\partial A_i$ then the vector
$\gamma^+(t_0)+\gamma^-(t_0)$  is normal to the tangent
plane of $\partial A_i$ at $\gamma(t_0)$.
Recall that according to our convention  $\tfrac{\gamma(t_0-\eps)}{\eps}\to\gamma^-(t_0)$ as $\eps\to0+$.)
We assume that trajectory meets one $A_i$ at the time.

The number of break
points of the trajectory will be called \emph{number of collisions} of the trajectory.

\begin{thm}{Collision theorem}\label{thm:baby-collision}
Assume $T\subset\RR^m$
is a centered dispersing table 
with $n$ walls and $\eps$-opened corners.
Then the number of collisions of any trajectory in  $T$  is bounded
by a number $N$ which depends only on $n$ and $\eps$.
\end{thm}

\begin{thm}{Corollary}\label{cor:balls}
Consider $n$ homogeneous hard balls
moving freely and colliding
elastically in empty space $\RR^3$. 
Every ball moves
along a straight line with constant speed until two balls collide, and then
the new velocities of the two balls are determined by the
laws of classical mechanics.

Then the total number of collisions cannot exceed some number $N$ which depend on the radiuses and masses of the balls.
If the balls are identical then $N$ depends only on $n$.
\end{thm}

\parit{Proof.}
A position of a collection of $n$ balls can be represented by a point in $\RR^{3\cdot n}$.
If $a_i=(x_i,y_i,z_i) \in \RR^3$ is the center of the $i$-th ball
then
the corresponding point in $\RR^{3\cdot N}$ is
\begin{align*}
\bm{a}&=(a_1, a_2 , \dots , a_n ) =
\\
&=(x_1, y_1 , z_1 , x_2 , y_2 , z_2 , \dots , x_n , y_n , z_n).
\end{align*}
Not every point in $\RR^{3\cdot n}$ represents a valid configuration of balls. 
We have to exclude positions where some of the balls overlap. 
The $i$-th and $j$-th ball intersect if 
$$|a_i - a_j | < R_i+R_j,$$
where $R_i$ denoted the radius of the ball number $i$.
These inequality defines $\tfrac{n\cdot(n-1)}{2}$ cylinders 
\[C_{i,j}=\set{(a_1, a_2 , \dots , a_n )\in\RR^{3\cdot n}} {|a_i - a_j | < R_i+R_j}.\] 
The complement
\[\RR^{3\cdot n}\backslash \bigcup_{i\ne j} C_{i,j}\] 
is the configuration space of our system. 
Its points correspond
to valid positions of the system of balls.

The evolution of the system
of balls traces a path in the configuration space. 
It is easy to verify that
the point representing the configuration of balls moves straight and at a
constant speed until it hits one of the cylinders $C_{ij}$ (this event corresponds
to a collision in the system of balls).

Consider the norm of $\bm{a}=(a_1,\dots,a_n)\in \RR^{3\cdot n}$ defined by
\[\lVert x\rVert=M_1\cdot|a_1|^2+\dots+M_n\cdot |a_n|,\]
where $|a_i|=\sqrt{x_i^2+y_i^2+z_i^2}$ 
and $M_i$ denotes the mass of the ball number $i$.
In the metric defined by $\lVert {*}\rVert$,
the collisions follow the
standard law of billiard: 
the angle of reflection is equal to the angle
of incidence. 

Note that each cylinder $C_{ij}$ is an open convex set with smooth boundaries which contains a unit ball around the origin.
It is easy to check that the sets $C_{ij}$ satisfy the conditions (\ref{A-convex})--(\ref{A-angle}) on page \pageref{A-convex}.
(In fact if all balls are identical then we can take $\alpha=\tfrac\pi3$ in (\ref{A-angle}).)
It remains to apply Theorem~\ref{thm:baby-collision}.
\qeds

\section{Reshetnyak's puff pastry}

\begin{thm}{Proposition}
Let $(A^1,\dots,A^n)$ be an array of convex closed sets in $\RR^m$.
Assume that $\spc{R}$ is the space
obtained by
gluing successive spaces in the array of $n+1$ copies of $\RR^m$ 
along $A^1,\dots,A^n$.
Then $\spc{R}\in\cCat{}{0}$
and moreover the induced map from each copy of $\RR^m$ to  $\spc{R}$ is distance preserving.
\end{thm}

The statement follows from Reshetnyak's gluing theorem.
The space $\spc{R}$ described in the proposition will be called \emph{Reshetnyak's puff pastry} for the array $(A^1,\dots,A^n)$.
The copies of $\RR^m$ in the Reshetnyak's puff pastry $\spc{R}$
will be called \emph{levels};
they will denoted by $\spc{R}^0,\dots,\spc{R}^n$.
The point in the $\kay$-th level $\spc{R}^\kay$
corresponding to $x\in \RR^m$
will be denoted by $x^\kay$.

\parit{Proof.}
Let us use indunction on $n$
the induction on $n$.
In the base case $n=0$, we have $\spc{R}\iso\RR^m\in\cCat{}{0}$.

\parit{Step.}
By induction hypothesis, we can assume that 
$\spc{R}'\in\cCat{}{0}$ where $\spc{R}'$ is the Reshetnyak's puff pastry for $(A^1,\dots,A^{n-1})$.

Note that $\spc{R}$ can be glued from $\spc{R}'$
and $\RR^m$ along $A_n$.
By Reshetnyak's gluing theorem we get $\spc{R}_{\bm{A}}\in\cCat{}{0}$.
\qeds

\begin{thm}{Definition}
A Reshetnyak's puff pastry $\spc{R}$ is called \emph{end-to-end convex} if the lower and upper levels of $\spc{R}$ form a convex subset.
\end{thm}

\begin{thm}{Lemma}\label{lem:end-to-end-convex}
Let $A$ and $B$ two convex sets in $\RR^m$.
Assume that $A\cup B\ne\emptyset$ and 
for for some positive integer $n$
and any $p\in \partial A\cap \partial B$ 
and 
any outer subnormal vectors $u$ and $v$ to $A$ and $B$
at $p$,
we have $\mangle(u,v)\le\pi\cdot(1-\tfrac1n)$.
Then the Reshetnyak's puff pastry for the $n$-array alternating $A$ and $B$ is end-to-end convex. 
\end{thm}

\parit{Proof.}
First we construct a map 
between general Reshetnyak's puff pastries 
which will be used later in the proof.

Assume $(A^1,\dots, A^n)$ and $(\check A^1,\dots,\check A^n)$ be two arrays of convex convex sets in $\RR^m$
and $\spc{R}$ and $\check{\spc{R}}$ be the corresponding Reshetnyak's  puff pastries.

Recall that for any $x$ in $\RR^m$ we can consider the point $x^i\in \spc{R}$
which corresponds to $x$ and lies on $i$-th level of $\spc{R}$.
Analogously denote by $\check x^i$ the point on $i$-th level of $\check{\spc{R}}$
which corresponds to $x$.

Assume $A_i\subset \check A_i$ for each $i$.
In this case the map $\spc{R}\to\check{\spc{R}}$
defined as $x^i\mapsto \check x^i$ is a short map.
In particular we get the following.

\begin{clm}{}
If  $\spc{R}$ and $\check{\spc{R}}$ as above and
\[\dist{x^i}{y^j}{\spc{R}}=\dist{\check x^i}{\check y^j}{\check{\spc{R}}}\]
for some $x,y\in \RR$ and $i,j\in \{0,\dots,n\}$
then the geodesic $[\check x^i \check y^j]_{\check{\spc{R}}}$ 
is the image of geodesic $[x^i y^j]_{\spc{R}}$
under the map $x^i\mapsto \check x^i$.
\end{clm}


Let us come back to the proof.
We need to show that $\spc{R}^0\cup\spc{R}^n$ forms a convex set in $\spc{R}$.
In other words, we need to show that $\spc{R}^0\cup\spc{R}^n$
is isometric to the doubling of $\RR^m$ in $A\cap B$. 

Fix $x,y\in \RR^m$.
Choose a point $z\in A\cap B$
for which the sum $\dist{x}{z}{}+\dist{y}{z}{}$ takes minimal value.
Since $\spc{R}\in\cCat{}{0}$, it is sufficient to show that the geodesic $[x^0y^n]_\spc{R}$ pass though $z^0=z^n$.

Assume contrary.
Then there are half spaces or whole space
$A'\supset A$ and $B'\supset B$
such that $\dist{x}{z}{}+\dist{y}{z}{}$ takes minimal value
for all $z\in A'\cap B'$.

In the array ???, 
exchange each $A$ to $A'$ and each $B$ to $B'$.
The corresponding Reshetnyak's puff pastry $\spc{R}'$
splits as a product or $\RR^{m-2}$ and a puff pastry
glued from the copies of the palne $\RR^2$.
Then $z\in\partial A\cap\partial B$.
By Proposition~\ref{prop:warp-examples},
it is sufficient to show that ???
and the latter is evident.



\section{Proof of Collision theorem.}


The base case $n=1$ is evident; the number of collisions cannot exceed $1$.  
It follows from the convexity of $A_1$ that
if the trajectory is reflected once in $\partial A_1$, 
then it cannot return to $A_1$.

The proof of the first step $n=2$ is slightly simpler than the remaining steps.
To simplify the presentation, we prove the case $n=2$, 
and then describe the necessary modifications to prove the $n=3$ case.
Once this is done, the proof of the general case should be evident.

\parit{First step; $n=2$.} To reduce number of indices, set $A=A_1$ and $B=A_2$. 

Note that any trajectory hits $A$ and $B$ in turn;
that is, the trajectory can not hit $A$ (as well as $B$)
twice in a raw.

Fix a large number $N$ and assume a trajectory $\gamma$ has at least $N$ collisuion.
We may assume that the trajectory meets the bodies in order \[\bm{A}_N=(\underbrace{A,B,A, B,\dots}_{N\ \text{times}}).\]

Consider the Reshetnyak's  puff pastry $\spc{R}$
for the array $\bm{A}_N$.
The trajectory $\gamma$ can be lifted to the puff pastry as a geodesic.
It follows that $\spc{R}$ is not end-to-end convex;
the later contradicts Lemma~\ref{lem:end-to-end-convex}.

\parit{Other steps; $n>2$.}
Denote by $N_{n-1}$ a number which is bigger than the number of  collisions of any billiard trajectory on the table with $n-1$ walls.

Fix a large number $N=N_{n-1}^3$ and assume a trajectory $\gamma$ has at least $N$ collisions.
Let us write the $N$ bodies in the order our trajectory meets them; denote by $\bm{A}_N$ the obtained array.  
If one takes any $N_{n-1}$ bodies in the raw from $\bm{A}_N$
then it will have all $n$ bodies in it;
otherwise we would get a contradiction with the induction hypothesis.

It follows that one can remove some bodies from the $\bm{A}_N$.


Then it should hit $A$,
then $B$,
then $A$
and so on $2\cdot N$ times
and then finally $B$.
Denote by 
$a_1\in \partial A$, 
$b_1\in \partial B,
\dots
b_N\in \partial B$
the brake points of this trajectory.

Prepare $2\cdot N$ copies of $\RR^m$,
say $\mathcal{R}_1$, $\mathcal{R}_2,\dots,\mathcal{R}_{2\cdot N}$.
Each $\mathcal{R}_i$ contains a copy of $A$ and $B$, which will be denoted as $A_i$ and $B_i$.

Let us glue a new space say $\mathcal{R}$ out of $\mathcal{R}_i$'s
by gluing 
\begin{enumerate}
\item $\mathcal{R}_1$ to $\mathcal{R}_2$ by identifying $A_1$ and $A_2$;
\item $\mathcal{R}_2$ to $\mathcal{R}_3$  by identifying $B_2$ and $B_3$;
\item $\mathcal{R}_3$ to $\mathcal{R}_4$  by identifying $A_3$ and $A_4$;
\item and so on;
\item $\mathcal{R}_{2\cdot N-1}$ to $\mathcal{R}_{2\cdot N}$ by identifying $A_{2\cdot N-1}$ and $A_{2\cdot N}$.
\end{enumerate}
According to Reshetnyak's gluing theorem (\ref{thm:reshetnyak}), $\mathcal{R}\in\cCat{}{0}$.
Further we will view $\mathcal{R}_i$ as subsets of $\mathcal{R}$.

Denote by $f_i\:\RR^m\to\mathcal{R}$ the distance preserving map which identifies $\RR^m$ and $\mathcal{R}_i$
and let $F\:\mathcal{R} \to \RR^m$ be the natural projection which identifies all $\mathcal{R}_i$ with $\RR^m$.

Consider the broken geodesic $\gamma$ connecting points $f_1(a_1),f_2(b_2),f_3(a_3),\dots,f_{2\cdot N}$ in $\mathcal{R}$.
Note that $\gamma$ forms a local geodesic in $\mathcal{R}$.
Therefore, according to ??? $\gamma$ is a geodesic with end points in $A_1\cup A_{2\cdot N}$,
but all interior points of $\gamma$ do not belong to $A_1\cup A_{2\cdot N}$.

Hence in order to arrive to a contradiction, 
it is sufficient to prove the following claim.
 
\begin{clm}{}
The union $A_1\cup A_{2\cdot N}$ is a convex set in $\mathcal{R}$.
\end{clm}

Due to the convexity of $A_1$ and $A_{2\cdot N}$,
it is sufficient to show that for any points $p\in A_1\backslash A_{2\cdot N}$ and $q\in A_{2\cdot N}\backslash A_1$,
the geodesic $[pq]$ contain a point $x\in K\z=A_1\cap A_{2\cdot N}$.

Assume contrary.
Mark points $p=p_1,q_2,p_3,\dots,p_{2\cdot N}=q$ of $[pq]$ such that $p_i\in A_i$ and $q_j\in B_j$.
Note that $[p_{2\cdot i-1}q_{2\cdot i}]\in \mathcal{R}_{2\cdot i-1}$ 
and 
$[q_{2\cdot i}p_{2\cdot i+1}]\in \mathcal{R}_{2\cdot i}$ for each $i$.
We will construct a curve passing throug $K$ and connecting $p$ to $q$ which is shorter than $[pq]$;
this way we arrive to a contradiction. 




\parit{Second step; $n=3$.}



\qeds

The Baby collision theorem (\ref{thm:baby-collision}) admits a straightforward generalization to higher dimensions and to an arbitrary finite number of sets $B_i$.
Namely, the following result holds:

\begin{thm}{Adult collision theorem}\label{thm:adult-collision}
Let $B_1,B_2,\dots,B_n$ be a finite collection of open convex sets in $\RR^m$.
Assume each $B_i$ contains the origin $0\in\RR^m$
and for each $i$, the boundary $W_i=\partial B_i$ is a smooth hypersurface.
Further, assume there is $\eps>0$ such that for any point $p\in W_i\cap W_j$ the angle between the tangent 
hyperplane of $W_i$ and $W_j$ at $p$ is at least $\eps$.

Then 
there is a natural number $N$ which depends only on $\eps$
such that the number of collisions of any trajectory in $T$ is at most $N$.
\end{thm}

The proof of the Collision theorem can be given along the same lines as its baby case.
We glue the space in a similar way:
Take a copy of $\RR^m$
and glue to it $n$ copies of $\RR^m$, along each $B_i$.
Further glue to each copy $n-1$ copies of $\RR^m$... 

Now we will show how to apply the above theorem to prove the following. 