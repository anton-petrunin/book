\chapter{Billiards}
\section{Dispersing billiard}

\begin{wrapfigure}{r}{37mm}
\begin{lpic}[t(-0mm),b(-0mm),r(0mm),l(0mm)]{pics/table(.5)}
\end{lpic}
\end{wrapfigure}

Assume $A^1,A^2,\dots A^n$ be a finite collection of closed convex sets in $\EE^m$.
Consider the billiard table formed by the closure of the complement 
$$T=\overline{\EE^m\backslash \bigcup_{i=1}^n A^i}.$$
The sets $A^i$ will be called \emph{walls} of the table.

We will always assume that the boundary $\partial A^i$ is a smooth hypersurface.

A \emph{billiard trajectory} 
on the table $T$ is a unit-speed broken line $\gamma$
with the \emph{regular reflection} at the break points on $\partial A^i$ 
--- the angle of reflection is equal to the angle of incidence.
The break points of the trajectory will be called \emph{collisions}.
We assume that trajectory meets one $A^i$ at the time.

The billiard described above will be called \emph{semi-dispersing} since after each collision, the nearby spreading trajectories will spread at least as rapidly.

We say that the walls $A^1,\dots,A^n$ have \emph{$\eps$-wide corners} 
if at each point $p\in R^m$ 
there is a right circular cone $C_p$ with the tip $p$ and aperture $\eps$
such that 
\[p\notin A^i\iff C_p\cap A^i=\emptyset.\]
for any $i$. 

\begin{thm}{Exercise}
Assume that the walls
of a semi-dispersing billiard table $T$ 
are compact and have common interior point.
Show that the walls of $T$ have $\eps$-wide corners
for some $\eps>0$.
\end{thm}

\begin{thm}{Exercise}\label{ex:centrally-simmetric-walls}
Assume that a semi-dispersing billiard table $T$ has
are centrally symmetric walls with common center.
Show that the walls of $T$  have $\eps$-wide corners
for some $\eps>0$.
\end{thm}

\begin{thm}{Collision Theorem}\label{thm:collision}
Assume $T\subset\RR^m$
is a semi-dispersing billiard table.
Assume that the walls of $T$ have common interior point and $\eps$-wide corners.
Then the number of collisions of any trajectory in  $T$  is bounded
by a number $N$ which depends only on the number of walls $n$ and $\eps$.
\end{thm}


The theorem will be proved in the next two sections.
Let us formulate and prove its corollary.

\begin{thm}{Corollary}\label{cor:balls}
Consider $n$ homogeneous hard balls
moving freely and colliding
elastically in empty space $\RR^3$. 
Every ball moves
along a straight line with constant speed until two balls collide, and then
the new velocities of the two balls are determined by the
laws of classical mechanics.

Then the total number of collisions cannot exceed some number $N$ which depend on the radiuses and masses of the balls.
If the balls are identical then $N$ depends only on $n$.
\end{thm}

The 1 and 2-dimensional cases admit simpler proofs.
The proof below works in all dimensions.



\parit{Proof.}
A position of a collection of $n$ balls can be represented by a point in $\RR^{3\cdot n}$.
If $a_i=(x_i,y_i,z_i) \in \RR^3$ is the center of the $i$-th ball
then
the corresponding point in $\RR^{3\cdot N}$ is
\begin{align*}
\bm{a}&=(a_1, a_2 , \dots , a_n ) =
\\
&=(x_1, y_1 , z_1 , x_2 , y_2 , z_2 , \dots , x_n , y_n , z_n).
\end{align*}
Not every point in $\RR^{3\cdot n}$ represents a valid configuration of balls. 
We have to exclude positions where some of the balls overlap. 
The $i$-th and $j$-th ball intersect if 
$$|a_i - a_j | \le R_i+R_j,$$
where $R_i$ denoted the radius of the ball number $i$.
These inequality defines $\tfrac{n\cdot(n-1)}{2}$ cylinders 
\[C_{i,j}=\set{(a_1, a_2 , \dots , a_n )\in\RR^{3\cdot n}} {|a_i - a_j | \le R_i+R_j}.\] 
The closure of the complement
\[T=\overline{\RR^{3\cdot n}\backslash \bigcup_{i\ne j} C_{i,j}}\] 
is the configuration space of our system. 
Its points correspond
to valid positions of the system of balls.

The evolution of the system
of balls traces a path in the configuration space. 
It is easy to verify that
the point representing the configuration of balls moves straight and at a
constant speed until it hits one of the cylinders $C_{ij}$ (this event corresponds
to a collision in the system of balls).

Consider the norm of $\bm{a}=(a_1,\dots,a_n)\in \RR^{3\cdot n}$ defined by
\[\lVert x\rVert=M_1\cdot|a_1|^2+\dots+M_n\cdot |a_n|,\]
where $|a_i|=\sqrt{x_i^2+y_i^2+z_i^2}$ 
and $M_i$ denotes the mass of the ball number $i$.
In the metric defined by $\lVert {*}\rVert$,
the collisions follow the
standard law of billiard: 
the angle of reflection is equal to the angle
of incidence. 

In particular, the number of collisions of hard balls which we need to estimate 
is the same as the number of collisions of corresponding billiard trajectory on the table $T$.

Note that each cylinder $C_{i,j}$ are convex sets 
with smooth boundaries which 
are centrally symmetric around the origin.
By Exercise~\ref{ex:centrally-simmetric-walls} the walls have $\eps$-wide corners for some $\eps>0$ which depends on radiuses $R_i$ and the masses $M_i$.
(In fact if all balls are identical then we can take $\eps=\tfrac\pi3$.)
It remains to apply Theorem~\ref{thm:baby-collision}.
\qeds

\section{Reshetnyak's puff pastry}

In this section we discuss a construction which will be used in the next section to prove Collision Theorem~\ref{thm:collision}.

\begin{thm}{Definition}
Let $(A^1,\dots,A^N)$ be an array of convex closed sets in $\RR^m$.
Consider an array of $N+1$ copies of $\RR^m$;
assume that space $\spc{R}$ is 
obtained by
gluing successive spaces in the array  
along $A^1,\dots,A^N$ correspondingly.

The obtained space $\spc{R}$  will be called \emph{Reshetnyak's puff pastry} for the array $(A^1,\dots,A^N)$.
The copies of $\RR^m$ in the Reshetnyak's puff pastry $\spc{R}$
will be called \emph{levels};
they will denoted by $\spc{R}^0,\dots,\spc{R}^N$.
The point in the $\kay$-th level $\spc{R}^\kay$
corresponding to $x\in \RR^m$
will be denoted by $x^\kay$.
\end{thm}

Applying Reshetnyak's gluing theorem \ref{thm:gluing}
to the above definition, we get the following.

\begin{thm}{Proposition}
Let $(A^1,\dots,A^N)$ be an array of convex closed sets in $\RR^m$.
Then the corresponding Reshetnyak's puff pastry $\spc{R}$
is a $\cCat{}{0}$ space.

Assume $(\check A^1,\dots,\check A^N)$ is an other array of convex bodies in $\RR^m$ such that $\check A^\kay\supset A^\kay$ for each $\kay$.
Let $\check{\spc{R}}$ be the corresponding Reshetnyak's  puff pastries.
Then the map $\spc{R}\to\check{\spc{R}}$
defined as $x^\kay\mapsto\check x^\kay$ is short.

In particular 
if  $\spc{R}$ and $\check{\spc{R}}$ as above and
\[\dist{x^i}{y^j}{\spc{R}}=\dist{\check x^i}{\check y^j}{\check{\spc{R}}}\]
for some $x,y\in \RR$ and $i,j\in \{0,\dots,n\}$
then the geodesic $[\check x^i \check y^j]_{\check{\spc{R}}}$ 
is the image of geodesic $[x^i y^j]_{\spc{R}}$
under the map $x^i\mapsto \check x^i$.
\end{thm}

Note that in the proposition above one could assume that $\check A^\kay=\RR^m$ for some $\kay$
and $\check A^i=A^i$ for $i\ne \kay$.
In this case $\check{\spc{R}}$ 
is the Reshetnyak's  puff pastry for the 
the array $(A^1,\dots,A^N)$ with removed $A^\kay$. 

\begin{thm}{Definition}
A Reshetnyak's puff pastry $\spc{R}$ 
is called \emph{end-to-end convex} 
if the lower and upper levels of $\spc{R}$ form a convex subset.
\end{thm}

Let $A^1,\dots,A^n$ be convex bodies in $\RR^m$.
Assume an array $\bm{X}=(A^{i_1},\dots, A^{i_N})$
has each $A^i$ at least once. 
Let $\spc{R}$ be the Reshetnyak's puff pastry for $\bm{X}$.
Note that $\spc{R}$ is end-to-end convex
if and only if the union lower and upper levels
$\spc{R}_0\cup\spc{R}_N$ is isometric to the doubling of $\RR^m$ in the intersection $A^1\cap\dots\cap A^n$.

From the discussion above we get the following.

\begin{thm}{Observation}
Let an array convex bodies $\bm{X}$
consists of $n$ different bodies $A^1,\dots,A^n$
in $\RR^m$.
Assume that an other array $\bm{X}'$
obtained form $\bm{X}$ by inserting one or more bodies from the list $A^1,\dots,A^n$ in arbitrary places.
Denote by $\spc{R}$ and $\spc{R}'$ 
the Reshetnyak's puff pastries for $\bm{X}$ and $\bm{X}'$.

If $\spc{R}$ is end-to-end convex then so is $\spc{R}'$
\end{thm}



Now assume an other array $\bm{X}'$
obtained form $\bm{X}$ by insetring one or more bodies from the list $A^1,\dot,A^n$ in arbitrary place.

$\spc{R}$


\begin{thm}{Proposition}\label{prop:end-to-end-convex}
Let  $A^1,\dots,A^n$ be a collection of convex sets in $\RR^m$
with nonempty intersection 
and
$\eps$-wide corners.
Then for any positive integer $\kay$ there is a positive integer $K$
such that if 
\[\bm{A}=(A^{i_1},\dots, A^{i_K})\]
is an array such that any $\kay$ elements in the raw from $\bm{A}$ 
each $A^i$ appears at least once
then the  Reshetnyak's puff pastry for $\bm{A}$ is end-to-end convex.
\end{thm}

The proof of proposition is based on the following lemma.

\begin{thm}{Lemma}\label{lem:end-to-end-convex}
Let $A$ and $B$ two convex sets in $\RR^m$ with $\eps$-wide.
Consider the alternating array
\[\bm{X}_n=(\underbrace{A,B,A,\dots}_{\text{$n$ times}}).\]
If $n\cdot\eps\ge\pi$
then the Reshetnyak's puff pastry for $\bm{X}_n$ is end-to-end convex. 
\end{thm}

\parit{Proof.}
Let us come back to the proof.
We need to show that $\spc{R}^0\cup\spc{R}^n$ forms a convex set in $\spc{R}$.
In other words, we need to show that $\spc{R}^0\cup\spc{R}^n$
is isometric to the doubling of $\RR^m$ in $A\cap B$. 

Fix $x,y\in \RR^m$.
Choose a point $z\in A\cap B$
for which the sum 
\[\dist{x}{z}{}+\dist{y}{z}{}\] 
takes minimal value.
Since $\spc{R}\in\cCat{}{0}$, it is sufficient to show that the geodesic $[x^0y^n]_\spc{R}$ pass though $z^0=z^n$.

Assume contrary.
Then there are half spaces (or whole spaces) $A'$ and $B'$ such that
$A'\supset A$ and $B'\supset B$
and 
\[\dist{x}{z}{}+\dist{y}{z}{}\] 
takes minimal value
for all $z\in A'\cap B'$.

In the array ???, 
exchange each $A$ to $A'$ and each $B$ to $B'$.
The corresponding Reshetnyak's puff pastry $\spc{R}'$
splits as a product or $\RR^{m-2}$ and a puff pastry
glued from the copies of the palne $\RR^2$.
Then $z\in\partial A\cap\partial B$.
By Proposition~\ref{prop:warp-examples},
it is sufficient to show that ???
and the latter is evident.
\qeds
 

\begin{thm}{Observation}
Asume $A^1,A^2,\dots,A^n$ be a collection of convex bodies in $\RR^m$
with $\eps$-wide corners.
Then the collection 
$A^1\cap A^2,A^3,\dots,A^n$ has also $\eps$-wide corners.
\end{thm}

\parit{Proof.}

\qeds


\parit{Proof of Proposition~\ref{prop:end-to-end-convex}.}
Assume an array $\bm{A}'$ is obtained from $\bm{A}$ by inserting few labels which were already in $\bm{A}$.
Note denote by $\spc{R}$ and $\spc{R}'$
the  Reshetnyak's puff pastry for $\bm{A}$ and $\bm{A}'$ correspondingly.

Assume that 
$\bm{A}=(A^{i_1},\dots,A^{i_L})$ is an array which use each $A^i$
and the Reshetnyak's puff pastry for $\bm{A}$ is end-to-end convex.
Then from above it follows that any array
$\bm{A}=(A^{i_1},\dots,A^{i_{\kay\cdot L}})$
such that from any $\kay$ elements in the raq each $A^i$ appears at least once then 
$\bm{A}\succeq\bm{A}$.
Therefore the Reshetnyak's puff pastry for $\bm{A}$ is also end-to-end convex.

Let us introduce new label $X$ for the set $A^{n-1}\cap A^n$.
By induction hypothesis there is an array
from the alphabet $A^1,\dots,A^{n-2},X$ such that corresponding 
Reshetnyak's puff pastry for $\bm{A}$ is also end-to-end convex.

Now let us exachange each $X$ in this alphabet by 
\[\underbrace{A^{n-1},A^n,A^{n-1},...}_{N\ \text{times}},\]
where $N$ provided by Lemma~\ref{lem:end-to-end-convex}.
Applying lemma we get that the obtained array has end-to-end convex puff pastry
\qeds



\section{Proof of Collision theorem.}


The base case $n=1$ is evident; the number of collisions cannot exceed $1$.  
It follows from the convexity of $A^1$ that
if the trajectory is reflected once in $\partial A^1$, 
then it cannot return to $A^1$.

The proof of the first step $n=2$ is slightly simpler than the remaining steps $n\ge 3$.
To simplify the presentation we prove these steps separately. 

\parit{First step; $n=2$.} To reduce number of indices, set $A=A^1$ and $B=A^2$. 

Note that any trajectory hits $A$ and $B$ in turn;
that is, the trajectory can not hit $A$ (as well as $B$)
twice in a raw.

Fix a large number $N$ and assume a trajectory $\gamma$ has at least $N$ collisions.
We may assume that the trajectory meets the bodies in order \[\bm{A}_N=(\underbrace{A,B,A, B,\dots}_{N\ \text{times}}).\]

Consider the Reshetnyak's  puff pastry $\spc{R}$
for the array $\bm{A}_N$.
The trajectory $\gamma$ can be lifted to the puff pastry as a geodesic.
It follows that $\spc{R}$ is not end-to-end convex;
the later contradicts Lemma~\ref{lem:end-to-end-convex}.

\parit{Second step; $n=3$.} 
To reduce number of indices, set $A=A^1$, $B=A^2$ and $C=A^3$.

Fix a large number $N$ and assume a trajectory $\gamma$ has at least $N$ collisions.
Write the labels $A$, $B$ and $C$ in the order $\gamma$ meets them; denote by $\bm{A}$ the obtained array.

Let $N_2$ be the number from the previous step.
Note that if one takes $N_2$ elements in the raw from $\bm{A}$
then each label $A$, $B$ and $C$ will appear at least once;
otherwise we get a contradiction with the induction hypothesis.

Set 
\[\bm{X}=(\underbrace{A,B,A\dots}_{N_2\ \text{times}})
\eqlbl{eq:X}\]
It follows that if $N>N_2^3$ then we can remove some elements from $\bm{A}$ to obtain the array
\[(\underbrace{\bm{X},C,\bm{X},\dots}_{N_2\ \text{times}}),\]
here each $\bm{X}$ as in \ref{eq:X}.

\parit{Other steps; $n\ge3$.}
Denote by $N_{n-1}$ a number which is bigger than the number of  collisions of any billiard trajectory on the table with $n-1$ walls.

Fix a large number $N=N_{n-1}^3$ and assume a trajectory $\gamma$ has at least $N$ collisions.
Let us write the $N$ bodies in the order our trajectory meets them; denote by $\bm{A}_N$ the obtained array.  
If one takes any $N_{n-1}$ bodies in the raw from $\bm{A}_N$
then it will have all $n$ bodies in it;
otherwise we would get a contradiction with the induction hypothesis.

It follows that one can remove some bodies from the $\bm{A}_N$.


Then it should hit $A$,
then $B$,
then $A$
and so on $2\cdot N$ times
and then finally $B$.
Denote by 
$a_1\in \partial A$, 
$b_1\in \partial B,
\dots
b_N\in \partial B$
the brake points of this trajectory.

Prepare $2\cdot N$ copies of $\RR^m$,
say $\mathcal{R}_1$, $\mathcal{R}_2,\dots,\mathcal{R}_{2\cdot N}$.
Each $\mathcal{R}_i$ contains a copy of $A$ and $B$, which will be denoted as $A^i$ and $B^i$.

Let us glue a new space say $\mathcal{R}$ out of $\mathcal{R}_i$'s
by gluing 
\begin{enumerate}
\item $\mathcal{R}_1$ to $\mathcal{R}_2$ by identifying $A^1$ and $A^2$;
\item $\mathcal{R}_2$ to $\mathcal{R}_3$  by identifying $B^2$ and $B^3$;
\item $\mathcal{R}_3$ to $\mathcal{R}_4$  by identifying $A^3$ and $A^4$;
\item and so on;
\item $\mathcal{R}_{2\cdot N-1}$ to $\mathcal{R}_{2\cdot N}$ by identifying $A^{2\cdot N-1}$ and $A^{2\cdot N}$.
\end{enumerate}
According to Reshetnyak's gluing theorem (\ref{thm:reshetnyak}), $\mathcal{R}\in\cCat{}{0}$.
Further we will view $\mathcal{R}_i$ as subsets of $\mathcal{R}$.

Denote by $f_i\:\RR^m\to\mathcal{R}$ the distance preserving map which identifies $\RR^m$ and $\mathcal{R}_i$
and let $F\:\mathcal{R} \to \RR^m$ be the natural projection which identifies all $\mathcal{R}_i$ with $\RR^m$.

Consider the broken geodesic $\gamma$ connecting points $f_1(a_1),f_2(b_2),f_3(a_3),\dots,f_{2\cdot N}$ in $\mathcal{R}$.
Note that $\gamma$ forms a local geodesic in $\mathcal{R}$.
Therefore, according to ??? $\gamma$ is a geodesic with end points in $A^1\cup A^{2\cdot N}$,
but all interior points of $\gamma$ do not belong to $A^1\cup A^{2\cdot N}$.

Hence in order to arrive to a contradiction, 
it is sufficient to prove the following claim.
 
\begin{clm}{}
The union $A^1\cup A^{2\cdot N}$ is a convex set in $\mathcal{R}$.
\end{clm}

Due to the convexity of $A^1$ and $A^{2\cdot N}$,
it is sufficient to show that for any points $p\in A^1\backslash A^{2\cdot N}$ and $q\in A^{2\cdot N}\backslash A^1$,
the geodesic $[pq]$ contain a point $x\in K\z=A^1\cap A^{2\cdot N}$.

Assume contrary.
Mark points $p=p_1,q_2,p_3,\dots,p_{2\cdot N}=q$ of $[pq]$ such that $p_i\in A^i$ and $q_j\in B^j$.
Note that $[p_{2\cdot i-1}q_{2\cdot i}]\in \mathcal{R}_{2\cdot i-1}$ 
and 
$[q_{2\cdot i}p_{2\cdot i+1}]\in \mathcal{R}_{2\cdot i}$ for each $i$.
We will construct a curve passing throug $K$ and connecting $p$ to $q$ which is shorter than $[pq]$;
this way we arrive to a contradiction. 




\parit{Second step; $n=3$.}



\qeds
