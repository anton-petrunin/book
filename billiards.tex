\chapter{Billiards}
\section{Semi-dispersing billiard}

Assume $A^1,A^2,\dots A^n$ be a finite collection of closed convex sets in $\EE^m$.
Consider the billiard table formed by the closure of the complement 
$$T=\overline{\EE^m\backslash \bigcup_{i=1}^n A^i}.$$
The sets $A^i$ will be called \emph{walls} of the table.

We will always assume that the boundary $\partial A^i$ is a smooth hypersurface.

A \emph{billiard trajectory} 
on the table $T$ is a unit-speed broken line $\gamma$
with the \emph{regular reflection} at the break points on $\partial A^i$ 
--- the angle of reflection is equal to the angle of incidence.
The break points of the trajectory will be called \emph{collisions}.
We assume that trajectory meets one $A^i$ at the time.

The billiard described above will be called \emph{semi-dispersing} since after each collision, the nearby spreading trajectories will spread at least as rapidly.

We say that the walls $A^1,\dots,A^n$ have \emph{$\eps$-wide corners} 
if at each point $p\in R^m$ 
there is a right circular cone $C_p$ with the tip $p$ and aperture $\eps$
such that 
\[p\notin A^i\iff C_p\cap A^i=\emptyset.\]
for any $i$. 

\begin{thm}{Exercise}
Assume that the walls
of a semi-dispersing billiard table $T$ 
are compact and have common interior point.
Show that the walls of $T$ have $\eps$-wide corners
for some $\eps>0$.
\end{thm}

\begin{thm}{Exercise}\label{ex:centrally-simmetric-walls}
Assume that a semi-dispersing billiard table $T$ has
are centrally symmetric walls with common center.
Show that the walls of $T$  have $\eps$-wide corners
for some $\eps>0$.
\end{thm}

\begin{thm}{Collision Theorem}\label{thm:collision}
Assume $T\subset\RR^m$
is a semi-dispersing billiard table.
Assume that the walls of $T$ have common interior point and $\eps$-wide corners.
Then the number of collisions of any trajectory in  $T$  is bounded
by a number $N$ which depends only on the number of walls $n$ and $\eps$.
\end{thm}


The theorem will be proved in the next two sections.
Let us formulate and prove its corollary.

\begin{thm}{Corollary}\label{cor:balls}
Consider $n$ homogeneous hard balls
moving freely and colliding
elastically in empty space $\RR^3$. 
Every ball moves
along a straight line with constant speed until two balls collide, and then
the new velocities of the two balls are determined by the
laws of classical mechanics.

Then the total number of collisions cannot exceed some number $N$ which depend on the radiuses and masses of the balls.
If the balls are identical then $N$ depends only on $n$.
\end{thm}

The 1 and 2-dimensional cases admit simpler proofs.
The proof below works in all dimensions.



\parit{Proof.}
A position of a collection of $n$ balls can be represented by a point in $\RR^{3\cdot n}$.
If $a_i=(x_i,y_i,z_i) \in \RR^3$ is the center of the $i$-th ball
then
the corresponding point in $\RR^{3\cdot N}$ is
\begin{align*}
\bm{a}&=(a_1, a_2 , \dots , a_n ) =
\\
&=(x_1, y_1 , z_1 , x_2 , y_2 , z_2 , \dots , x_n , y_n , z_n).
\end{align*}
Not every point in $\RR^{3\cdot n}$ represents a valid configuration of balls. 
We have to exclude positions where some of the balls overlap. 
The $i$-th and $j$-th ball intersect if 
$$|a_i - a_j | \le R_i+R_j,$$
where $R_i$ denoted the radius of the ball number $i$.
These inequality defines $\tfrac{n\cdot(n-1)}{2}$ cylinders 
\[C_{i,j}=\set{(a_1, a_2 , \dots , a_n )\in\RR^{3\cdot n}} {|a_i - a_j | \le R_i+R_j}.\] 
The closure of the complement
\[T=\overline{\RR^{3\cdot n}\backslash \bigcup_{i\ne j} C_{i,j}}\] 
is the configuration space of our system. 
Its points correspond
to valid positions of the system of balls.

The evolution of the system
of balls traces a path in the configuration space. 
It is easy to verify that
the point representing the configuration of balls moves straight and at a
constant speed until it hits one of the cylinders $C_{ij}$ (this event corresponds
to a collision in the system of balls).

Consider the norm of $\bm{a}=(a_1,\dots,a_n)\in \RR^{3\cdot n}$ defined by
\[\lVert x\rVert=M_1\cdot|a_1|^2+\dots+M_n\cdot |a_n|,\]
where $|a_i|=\sqrt{x_i^2+y_i^2+z_i^2}$ 
and $M_i$ denotes the mass of the ball number $i$.
In the metric defined by $\lVert {*}\rVert$,
the collisions follow the
standard law of billiard: 
the angle of reflection is equal to the angle
of incidence. 

In particular, the number of collisions of hard balls which we need to estimate 
is the same as the number of collisions of corresponding billiard trajectory on the table $T$.

Note that each cylinder $C_{i,j}$ are convex sets 
with smooth boundaries which 
are centrally symmetric around the origin.
By Exercise~\ref{ex:centrally-simmetric-walls} the walls have $\eps$-wide corners for some $\eps>0$ which depends on radiuses $R_i$ and the masses $M_i$.
(In fact if all balls are identical then we can take $\eps=\tfrac\pi3$.)
It remains to apply Theorem~\ref{thm:baby-collision}.
\qeds

\section{Reshetnyak's puff pastry}

In this section we discuss a construction which will be used in the next section to prove Collision Theorem~\ref{thm:collision}.

Let $(A^1,\dots,A^N)$ be an array of convex closed sets in $\RR^m$.
Consider an array of $N+1$ copies of $\RR^m$.
Assume that space $\spc{R}$ is 
obtained by
gluing successive spaces in the array  
along $A^1,\dots,A^N$ correspondingly.

The obtained space $\spc{R}$  will be called \emph{Reshetnyak's puff pastry} for the array $(A^1,\dots,A^N)$.
The copies of $\RR^m$ in the Reshetnyak's puff pastry $\spc{R}$
will be called \emph{levels};
they will denoted by $\spc{R}^0,\dots,\spc{R}^N$.
The point in the $\kay$-th level $\spc{R}^\kay$
corresponding to $x\in \RR^m$
will be denoted by $x^\kay$.

The puff pastry $\spc{R}$ admits natural projection
$\spc{R}\to\RR^m$ defined as $x^\kay\mapsto x$.
Note that the restriction of this projection to each level $\spc{R}^\kay$ is an isometry.

Given $x\in \RR^m$, any point $x^\kay\in\spc{R}$ is called lifting of $x$.
Note that 
$x^i=x^j$ for some $i<j$
if and only if $x\in A^{i+1}\cap\dots\cap A^j$.


Applying Reshetnyak's gluing theorem \ref{thm:gluing}
to the above definition, we get the following.

\begin{thm}{Proposition}
Let $(A^1,\dots,A^N)$ be an array of convex closed sets in $\RR^m$.
Then the corresponding Reshetnyak's puff pastry $\spc{R}$
is a $\cCat{}{0}$ space.

Assume $(\check A^1,\dots,\check A^N)$ is an other array of convex bodies in $\RR^m$ such that $\check A^\kay\supset A^\kay$ for each $\kay$.
Let $\check{\spc{R}}$ be the corresponding Reshetnyak's  puff pastries.
Then the map $\spc{R}\to\check{\spc{R}}$
defined as $x^\kay\mapsto\check x^\kay$ is short.

In particular 
if  $\spc{R}$ and $\check{\spc{R}}$ as above and
\[\dist{x^i}{y^j}{\spc{R}}=\dist{\check x^i}{\check y^j}{\check{\spc{R}}}\]
for some $x,y\in \RR$ and $i,j\in \{0,\dots,n\}$
then the geodesic $[\check x^i \check y^j]_{\check{\spc{R}}}$ 
is the image of geodesic $[x^i y^j]_{\spc{R}}$
under the map $x^i\mapsto \check x^i$.
\end{thm}

Note that in the proposition above one could assume that $\check A^\kay=\RR^m$ for some $\kay$
and $\check A^i=A^i$ for $i\ne \kay$.
In this case $\check{\spc{R}}$ 
is the Reshetnyak's  puff pastry for the 
the array $(A^1,\dots,A^N)$ with removed $A^\kay$. 

\begin{thm}{Definition}
A Reshetnyak's puff pastry $\spc{R}$ 
is called \emph{end-to-end convex} 
if the union of lower and upper levels in $\spc{R}$ 
forms a convex set.
\end{thm}

Let $A^1,\dots,A^n$ be convex bodies in $\RR^m$.
Assume an array $\bm{A}=(A^{i_1},\dots, A^{i_N})$
has each $A^i$ at least once. 
Let $\spc{R}$ be the Reshetnyak's puff pastry for $\bm{A}$.
Note that $\spc{R}$ is end-to-end convex
if and only if the union lower and upper levels
$\spc{R}_0\cup\spc{R}_N$ is isometric to the doubling of $\RR^m$ in the intersection $A^1\cap\dots\cap A^n$.

From the discussion above we get the following.

\begin{thm}{Observation}
Let $\bm{A}$ be an array of convex bodies in $\RR^m$.
Assume that an other array $\bm{A}'$
obtained by inserting in $\bm{A}$
copies of the bodies which were already listed in $\bm{A}$.

For example we can take $\bm{A}=(A,C,B,C,A)$ 
and insert $B$ to the second place and $A$ to the 4-th place to obtain $\bm{A}'=(A,B,C,A,B,C,A)$.

Denote by $\spc{R}$ and $\spc{R}'$ 
the Reshetnyak's puff pastries for $\bm{A}$ and $\bm{A}'$.
If $\spc{R}$ is end-to-end convex then so is $\spc{R}'$.
\end{thm}


\begin{thm}{Proposition}\label{prop:end-to-end-convex}
Given $\eps>0$ and integer $n$
there is an array of integers $(i_1,\dots,i_N)$ 
such that $i\kay\in\{1,\dots,n\}$ for any $\kay$ 
and each number $1,\dots,n$ appear in the array at least one
such that the following holds.

If $A^1,\dots,A^n$ is a collection of convex sets in $\RR^m$ with common point 
and $\eps$-wide corners 
then the puff pastry for the array
$(A^{i_1},\dots,A^i_N)$ is end-to-end convex.
\end{thm}

The proof of proposition is based on the following lemma.

\begin{thm}{Lemma}\label{lem:end-to-end-convex}
Let $A$ and $B$ two convex sets in $\RR^m$ with $\eps$-wide.
Consider the alternating array
\[\bm{X}_n=(\underbrace{A,B,A,\dots}_{\text{$n$ times}}).\]
If $n\cdot\eps\ge\pi$
then the Reshetnyak's puff pastry for $\bm{X}_n$ is end-to-end convex. 
\end{thm}

\parit{Proof.}
Let us come back to the proof.
We need to show that $\spc{R}^0\cup\spc{R}^n$ forms a convex set in $\spc{R}$.
In other words, we need to show that $\spc{R}^0\cup\spc{R}^n$
is isometric to the doubling of $\RR^m$ in $A\cap B$. 

Fix $x,y\in \RR^m$.
Choose a point $z\in A\cap B$
for which the sum 
\[\dist{x}{z}{}+\dist{y}{z}{}\] 
takes minimal value.
Since $\spc{R}\in\cCat{}{0}$, it is sufficient to show that the geodesic $[x^0y^n]_\spc{R}$ pass though $z^0=z^n$.

Assume contrary.
Then there are half spaces (or whole spaces) $A'$ and $B'$ such that
$A'\supset A$ and $B'\supset B$
and 
\[\dist{x}{z}{}+\dist{y}{z}{}\] 
takes minimal value
for all $z\in A'\cap B'$.

In the array ???, 
exchange each $A$ to $A'$ and each $B$ to $B'$.
The corresponding Reshetnyak's puff pastry $\spc{R}'$
splits as a product or $\RR^{m-2}$ and a puff pastry
glued from the copies of the palne $\RR^2$.
Then $z\in\partial A\cap\partial B$.
By Proposition~\ref{prop:warp-examples},
it is sufficient to show that ???
and the latter is evident.
\qeds

\parit{Proof of Proposition~\ref{prop:end-to-end-convex}.}
The array $(i_1,\dots,i_N)$ is constructed recursively.
For $n=1$, one can take the array from one element $(1)$.

Now assume an array for $n$ is alrady constructed.
let us exchange each occurrence of $n$
to the string 
\[\underbrace{n,n+1,n,\dots}_{\text{$N$ times}}\]
where $N=\lceil\tfrac\pi\eps\rceil$.

Applying Lemma \ref{lem:end-to-end-convex}
and the following observation we get that the obtained array meets the conditions of the proposition.

\begin{clm}{}
Asume $A^1,A^2,\dots,A^n$ be a collection of convex bodies in $\RR^m$
with $\eps$-wide corners and $B=A^{n-1}\cap A^n$
Then the collection 
$A^1,\dots,A^{n-2},B$ has also $\eps$-wide corners.
\qeds
\end{clm}



\section{Proof of Collision theorem.}

We present the proof of Collision Theorem (\ref{thm:collision})
based on the results developed in the previous section.

\parit{Proof of Collision Theorem (\ref{thm:collision}).}
We will apply induction on $n$.

\parit{Base.}The base case $n=1$ is evident; the number of collisions cannot exceed $1$.  
It follows from the convexity of $A^1$ that
if the trajectory is reflected once in $\partial A^1$, 
then it cannot return to $A^1$. 

\parit{Step.}
Assume $\gamma$ is a trajectory which meets the walls in the order $A^{i_1},\dots,A^{i_N}$ for a large integer $N$.

Consider the array 
\[\bm{A}_\gamma=(A^{i_1},\dots,A^{i_N}).\]
By induction hypothesis, there is a positive integer $M$ such any $M$ elements in the raw from  $\bm{A}_\gamma$ contain each $A^i$ at least once.

Let $\spc{R}_\gamma$ be  the  Reshetnyak's puff pastry for  $\bm{A}_\gamma$.

Let us lift $\gamma(t)$ to $\gamma^\kay(t)$ 
for any moment of time $t$ between $\kay$ and $(\kay+1)$-th collisions.
Note that this way we lift $\gamma$ to a minimizing geodesic in $\spc{R}$.

Since $\gamma$ does not pass through $A_1\cap\dots\cap A_n$
the lift of $\gamma$ goes away form $\spc{R}_\gamma^0\cup \spc{R}_\gamma^N$.
It follows that $\spc{R}_\gamma$ is not end-to-end convex.

Let 
\[\bm{B}=(A^{j_1},\dots,A^{j_K})\] 
be the array provided by ???;
so $\bm{B}$ contains each $A^i$ at least once
and the puff pastry $\spc{R}_{\bm{B}}$ for $\bm{B}$ is end-to-end convex.

Assuming $N\ge K\cdot M$,
we can obtain $\bm{A}_\gamma$  
by inserting elements $A^i$ in $\bm{B}$.
By Proposition ???, 
$\spc{R}_\gamma$ is end-to-end convex,
a contradiction.
\qeds
