\section{Semi-dispersing billiard}

Let $A^1,A^2,\dots A^n$ be a finite collection of closed convex sets in $\EE^m$.
Assume that the boundary $\partial A^i$ is a smooth hypersurface.

Consider the billiard table formed by the closure of the complement 
$$T=\overline{\EE^m\backslash \bigcup_{i} A^i}.$$
The sets $A^i$ will be called \emph{walls} of the table $T$.

A \emph{billiard trajectory} 
on the table $T$ is a unit-speed broken line $\gamma$
with the \emph{regular reflection} at the break points on $\partial A^i$ 
--- the angle of reflection is equal to the angle of incidence.
The break points of the trajectory will be called \emph{collisions}.
We assume that trajectory meets one $A^i$ at the time.

The billiard described above will be called \emph{semi-dispersing} since after each collision, the nearby spreading trajectories will spread at least as rapidly.
(The canonical definition of  \emph{semi-dispersing billiard} is more general.)

We say that the closed set $A$ has  \emph{$\eps$-wide corners} 
if together with each point $p$
the set $A$ contains a small right circular cone
with tip at $p$ and
with arbitrary aperture smaller that $\eps$.

For example, 
a plane polygon 
has $\eps$-wide corners
if all its interior angles are at least $\eps$.

We will consider finite collections closed convex sets 
$A^1,\dots,A^n\subset\RR^m$ 
such that for any subset $F\subset\{1,\dots,n\}$
the intersection
$\bigcap_{i\in F}A^i$ 
has $\eps$-wide corners.
In this case we say that all intesections of $A^i$ have $\eps$-wide corners.

\begin{thm}{Exercise}
Assume that the walls
of a semi-dispersing billiard table $T$ 
are compact and have common interior point.
Show that all the intersections of the walls of $T$ have $\eps$-wide corners for some $\eps>0$.
\end{thm}

\begin{thm}{Exercise}\label{ex:centrally-simmetric-walls}
Assume that a semi-dispersing billiard table $T$ has
are centrally symmetric walls with common center.
Show that all the intersections of the walls of $T$  have $\eps$-wide corners for some $\eps>0$.
\end{thm}

\begin{thm}{Collision Theorem}\label{thm:collision}
Assume $T\subset\RR^m$
is a semi-dispersing billiard table.
Assume that the walls of $T$ have common interior point 
and all their intersections have $\eps$-wide corners.
Then the number of collisions of any trajectory in  $T$  is bounded
by a number $N$ which depends only on the number of walls $n$ and $\eps$.
\end{thm}

The Collision Theorem was proved by Burago, Ferleger and Kononenko in \cite{BFK};
we present their proof with minor improvements.

The theorem will be proved in the next two sections;
in Section~\ref{sec:puff-pastry} we discuss a recursive construction based on Reshetnyak's gluing theorem,
which then used in the proof given in Section~\ref{sec:proof-of-collision}.

First let us formulate and prove a corollary of Collision Theorem.

\begin{thm}{Corollary}\label{cor:balls}
Consider $n$ homogeneous hard balls
moving freely and colliding
elastically in empty space $\RR^3$. 
Every ball moves
along a straight line with constant speed until two balls collide, and then
the new velocities of the two balls are determined by the
laws of classical mechanics.

Then the total number of collisions cannot exceed some number $N$ which depend on the radiuses and masses of the balls.
If the balls are identical then $N$ depends only on $n$.
\end{thm}

The proof below admits a straightforward generalization to all dimensions.

\parit{Proof.}
Denote by $a_i=(x_i,y_i,z_i) \in \RR^3$ the center of the $i$-th ball.
Consider the corresponding point in $\RR^{3\cdot N}$
\begin{align*}
\bm{a}&=(a_1, a_2 , \dots , a_n ) =
\\
&=(x_1, y_1 , z_1 , x_2 , y_2 , z_2 , \dots , x_n , y_n , z_n).
\end{align*}

The $i$-th and $j$-th ball intersect if 
$$|a_i - a_j | \le R_i+R_j,$$
where $R_i$ denoted the radius of the $i$-th ball.
These inequality defines $\tfrac{n\cdot(n-1)}{2}$ cylinders 
\[C_{i,j}=\set{(a_1, a_2 , \dots , a_n )\in\RR^{3\cdot n}} {|a_i - a_j | \le R_i+R_j}.\] 
The closure of the complement
\[T=\overline{\RR^{3\cdot n}\backslash \bigcup_{i< j} C_{i,j}}\] 
is the configuration space of our system. 
Its points correspond
to valid positions of the system of balls.

The evolution of the system
of balls is described by the motion of
the point $\bm{a}\in\RR^{3\cdot n}$.
It moves straight and at a
constant speed until it hits one of the cylinders $C_{i,j}$; 
this event corresponds
to a collision in the system of balls.

Consider the norm of $\bm{a}=(a_1,\dots,a_n)\in \RR^{3\cdot n}$ defined by
\[\lVert \bm{a}\rVert=\sqrt{M_1\cdot|a_1|^2+\dots+M_n\cdot |a_n|},\]
where $|a_i|=\sqrt{x_i^2+y_i^2+z_i^2}$ 
and $M_i$ denotes the mass of the $i$-th ball.
In the metric defined by $\lVert {*}\rVert$,
the collisions follow the
standard law of billiard: 
the angle of reflection is equal to the angle
of incidence. 

In particular, the number of collisions of hard balls which we need to estimate 
is the same as the number of collisions of corresponding billiard trajectory on the table $T$.

Note that each cylinder $C_{i,j}$ is a convex set;
it has smooth boundary 
and it is centrally symmetric around the origin.
By Exercise~\ref{ex:centrally-simmetric-walls} all the intersections of the walls have $\eps$-wide corners for some $\eps>0$ which depends on radiuses $R_i$ and the masses $M_i$.
(If all balls are identical then we can take $\eps=\tfrac\pi3$.)
It remains to apply Collision Theorem~\ref{thm:collision}.
\qeds

\section{Reshetnyak's puff pastry}\label{sec:puff-pastry}

In this section we introduce \emph{Reshetnyak's puff pastry}, 
a construction which will be used in the next section to prove Collision Theorem~\ref{thm:collision}.

Let $(A^1,\dots,A^N)$ be an array of convex closed sets in $\RR^m$.
Consider an array of $N+1$ copies of $\RR^m$.
Assume that the space $\spc{R}$ is 
obtained by
gluing successive spaces in the array  
along $A^1,\dots,A^N$ correspondingly.

\begin{wrapfigure}{r}{58mm}
\begin{lpic}[t(0mm),b(4mm),r(0mm),l(0mm)]{pics/puff-pastry-example(1)}
\lbl[b]{18,24;$A$}
\lbl[b]{38,24;$B$}
\lbl[w]{53,2,-11;$\spc{R}_0$}
\lbl[w]{53,8,-11;$\spc{R}_1$}
\lbl[w]{53,12,11;$\spc{R}_2$}
\lbl[w]{53,18,11;$\spc{R}_3$}
\lbl[b]{28,-4;Puff pastry for $(A,B,A)$.}
\end{lpic}
\end{wrapfigure}

%???PIC???puff-pastry-example

The obtained space $\spc{R}$  will be called \emph{Reshetnyak's puff pastry} for the array $(A^1,\dots,A^N)$.
The copies of $\RR^m$ in the Reshetnyak's puff pastry $\spc{R}$
will be called \emph{levels};
they will denoted by $\spc{R}^0,\dots,\spc{R}^N$.
The point in the $\kay$-th level $\spc{R}^\kay$
which corresponds to $x\in \RR^m$
will be denoted by $x^\kay$.

Given $x\in \RR^m$, any point $x^\kay\in\spc{R}$ is called \emph{lifting} of $x$;
we also can consider the liftings of the maps to $\RR^m$ 
in particular the liftings of subsets in $\RR^m$.

Note that the intersection $A^1\cap\dots\cap A^N$ admits unique lifting in $\spc{R}$.
Moreover, $x^i=x^j$ for some $i<j$
if and only if $x\in A^{i+1}\cap\dots\cap A^j$.
Note also that the restriction of the natural projection $x^\kay\mapsto x$ to each level $\spc{R}^\kay$ is an isometry.

\begin{thm}{Observation}\label{obs:puff pastry is CAT}
Any Reshetnyak's puff pastry is a $\cCat{}{0}$ space.
\end{thm}

The above observation is proved by applying Reshetnyak's gluing theorem \ref{thm:gluing} few times.

\begin{thm}{Proposition}\label{prop:A-check-A}
Assume $(A^1,\dots,A^N)$ and $(\check A^1,\dots,\check A^N)$ are two arrays of convex closed sets in $\RR^m$ 
such that $ A^\kay\subset \check A^\kay$ for each $\kay$.
Let $\spc{R}$ and $\check{\spc{R}}$ be the corresponding Reshetnyak's  puff pastries.
Then the map $\spc{R}\to\check{\spc{R}}$
defined as $x^\kay\mapsto\check x^\kay$ is short.

In particular 
if  
\[\dist{x^i}{y^j}{\spc{R}}=\dist{\check x^i}{\check y^j}{\check{\spc{R}}}\]
for some $x,y\in \RR^m$ and $i,j\in \{0,\dots,n\}$
then the geodesic $[\check x^i \check y^j]_{\check{\spc{R}}}$ 
is the image of geodesic $[x^i y^j]_{\spc{R}}$
under the map $x^i\mapsto \check x^i$.
\end{thm}

\parit{Proof.}
The first statement in the proposition 
follows from the construction of Reshetnyak's  puff pastries.

Since the map is short, the image of $[x^i y^j]_{\spc{R}}$
is a geodesic joining $\check x^i$ to $\check y^j$ in $\check{\spc{R}}$.
By Observation~\ref{obs:puff pastry is CAT}, $\check{\spc{R}}\in\cCat{}{0}$. 
Therefore such geodesic is unique.
Hence the second statement follows.
\qeds

\begin{thm}{Definition}
A Reshetnyak's puff pastry $\spc{R}$ 
is called \emph{end-to-end convex} 
if the union of lower and upper levels in $\spc{R}$ 
forms a convex set.
\end{thm}

\parbf{Remark.}
Let $\spc{R}$ be the Reshetnyak's puff pastry for an array of convex sets 
\[\bm{A}=(A^{1},\dots, A^{N}).\]
Note that $\spc{R}$ is end-to-end convex
if and only if the union lower and upper levels
$\spc{R}_0\cup\spc{R}_N$ is isometric to the doubling of $\RR^m$ in the intersection $A^1\cap\dots\cap A^N$.


\begin{thm}{Observation}\label{obs:end-to-end-convex}
Let $\check{\bm{A}}$ and $\bm{A}$ be arrays of convex bodies in $\RR^m$.
Assume that 
the  array $\bm{A}$ 
obtained by inserting in $\check{\bm{A}}$
copies of the bodies which were already listed in $\check{\bm{A}}$.

For example we can take $\check{\bm{A}}=(A,C,B,C,A)$ 
and insert $B$ to the second place and $A$ to the 4-th place to obtain $\bm{A}=(A,B,C,A,B,C,A)$.

Denote by $\check{\spc{R}}$ and $\spc{R}$ 
the Reshetnyak's puff pastries for $\check{\bm{A}}$ and $\bm{A}$ correspondingly.
If $\check{\spc{R}}$ is end-to-end convex then so is $\spc{R}$.
\end{thm}

\parit{Proof.}
Without loss of generality we can assume that $\bm{A}$ 
obtained by inserting one element in $\check{\bm{A}}$,
say at the place number $\kay$.

Note that $\check{\spc{R}}$ is isometric to the puff pastry 
for $\bm{A}$ with $A^\kay$ exchanged to $\RR^m$.
It remains to apply Proposition~\ref{prop:A-check-A}.
\qeds


\begin{thm}{Proposition}\label{prop:end-to-end-convex}
Given $\eps>0$ and a positive integer $n$
there is an array of integers $\bm{j}_\eps=(j_1,\dots,j_N)$ 
such that 

\begin{subthm}{} For each $\kay$ we have $1\le j_\kay\le n$
and each number $1,\dots,n$ appear in $\bm{j}_\eps$ at least one.
\end{subthm}

\begin{subthm}{}
If $A^1,\dots,A^n$ is a collection of closed convex sets in $\RR^m$ with common point 
and all their intersections have $\eps$-wide corners 
then the puff pastry for the array
$(A^{j_1},\dots,A^{j_N})$ is end-to-end convex.
\end{subthm}

Moreover we can assume that $N\le (\lceil\tfrac\pi\eps\rceil+1)^n$.
\end{thm}

The proof of proposition is based on the following lemma.

\begin{thm}{Lemma}\label{lem:end-to-end-convex}
Let $A$ and $B$ two convex sets in $\RR^m$ and $A\cap B$ has $\eps$-wide corners.
Then the Reshetnyak's puff pastry for the array
\[(\underbrace{A,B,A,\dots}_{\text{$\lceil\tfrac\pi\eps\rceil+1$ times}}).\]
is end-to-end convex. 
\end{thm}

In turn, the proof of lemma is based on its partial case.

\begin{thm}{Sublemma}\label{sublem:end-to-end-convex}
Let $\ddot A$ and $\ddot B$ two  
half-planes in $\RR^2$ 
and $A\cap B$ is an angle with angle measure $\alpha$.
Then the Reshetnyak's puff pastry for the array \[(\underbrace{\ddot A,\ddot B,\ddot A,\dots}_{\text{$\lceil\tfrac\pi\alpha\rceil+1$ times}})\]
is end-to-end convex. 
\end{thm}

\begin{wrapfigure}{r}{52mm}
\begin{lpic}[t(0mm),b(0mm),r(0mm),l(0mm)]{pics/puff-pastry(1)}
%\lbl[t]{4,58;$a$}
\end{lpic}
\end{wrapfigure}

\parit{Proof.}
Note that the puff pastry $\ddot{\spc{R}}$ is isometric to the cone over space glued from the unit circles as shown on the diagram.

All the short arcs on the diagram have length $\alpha$;
the long arcs have length $\pi-\alpha$,
so going around along any path will take $2\cdot\pi$.

By Exercise~\ref{ex:convexity-in-cone},
the statement reduces to the $\pi$-convexity of the union of inner and outer circles 
marked by fat lines on the diagram.

The latter holds if the zigzag line has length at least $\pi$.
This line is formed by $\lceil\tfrac\pi\alpha\rceil$ arcs with length $\alpha$ each.
Hence the sublemma follows.
\qeds


\parit{Proof of Lemma~\ref{lem:end-to-end-convex}.}
Fix arbitrary $x,y\in \RR^m$.
Choose a point $z\in A\cap B$
for which the sum 
\[\dist{x}{z}{}+\dist{y}{z}{}\] 
takes the minimal value.
To show the end-to-end convexity of  $\spc{R}$,
it is sufficient to prove the following.

\begin{clm}{}\label{clm:z in xy}
The geodesic $[x^0y^n]_\spc{R}$ pass though $z^0=z^n\in \spc{R}$.
\end{clm}

Without loss of generality we can assume that $z\in\partial A\cap\partial B$.
Indeed, since the puff pastry for 1-array $(B)$ is end-to-end convex,
Proposition~\ref{prop:A-check-A} together with Observation~\ref{obs:end-to-end-convex}
imply \ref{clm:z in xy} in the case if $z$ lies in the interior of $A$.
Similarly we can treat the case $z$ lies in the interior of $B$.

The following claim is an exercise in convex geometry, we leave its proof to the reader.

\begin{clm}{}
There are half-spaces $\dot A$ and $\dot B$ such that
$\dot A\supset A$ and $\dot B\supset B$
and 
\[\dist{x}{z}{}+\dist{y}{z}{}\] 
takes minimal value
for all $z\in \dot A\cap \dot B$.
\end{clm}

\begin{wrapfigure}{r}{35mm}
\begin{lpic}[t(0mm),b(0mm),r(0mm),l(0mm)]{pics/AB-xyz(1)}
\lbl[br]{13,35;$x$}
\lbl[bl]{25,37;$y$}
\lbl[t]{18,28;$z$}
\lbl{8,22;$A$}
\lbl[t]{25,23;$B$}
\lbl[t]{3,33;$\dot A$}
\lbl[t]{31,29;$\dot B$}
\end{lpic}
\end{wrapfigure}

Since $A\cap B$ have $\eps$-wide corners,
the angle measure of  $\dot A\cap \dot B$ is at least $\eps$.

Note that $\RR^{m}$ admits isometric splitting $\RR^{m-1}\times \RR^2$ 
such that 
\begin{align*}
\dot A&=\RR^{m-1}\times \ddot A
\\
\dot B&=\RR^{m-1}\times \ddot B
\end{align*}
where $\ddot A$ and $\ddot B$ are half-planes in $\RR^2$.

Let us exchange each $A$ to $\dot A$ and each $B$ to $\dot B$
in the array, we get the array
\[(\underbrace{\dot A,\dot B,\dot A,\dots}_{\text{$\lceil\tfrac\pi\eps\rceil+1$ times}}).\]
The corresponding puff pastry $\dot{\spc{R}}$
splits as a product of $\RR^{m-2}$ and a puff pastry, 
say $\ddot{\spc{R}}$,
glued from the copies of the plane $\RR^2$ for the array
\[(\underbrace{\ddot A,\ddot B,\ddot A,\dots}_{\text{$\lceil\tfrac\pi\eps\rceil+1$ times}}).\]


According to Sublemma~\ref{sublem:end-to-end-convex} and Observation~\ref{obs:end-to-end-convex}, $\ddot{\spc{R}}$ is end-to-end convex.

Since $\dot{\spc{R}}\iso\RR^{m-2}\times\ddot{\spc{R}}$,
by Proposition~\ref{prop:warp-examples}
$\dot{\spc{R}}$ is also end-to-end convex.

It follows that the geodeisc $[\dot x^0\dot y^N]_{\dot{\spc{R}}}$ pass through $\dot z^0=\dot z^N\in\dot{\spc{R}}$.
By Proposition~\ref{prop:A-check-A}, 
the image of $[\dot x^0\dot y^N]_{\dot{\spc{R}}}$ 
under the map $\dot x^\kay\mapsto x^\kay$
is the geodesic $[x^0 y^N]_{\spc{R}}$.
Hence Claim~\ref{clm:z in xy} follows.
\qeds

\parit{Proof of Proposition~\ref{prop:end-to-end-convex}.}
The array $(j_1,\dots,j_N)$ is constructed recursively.
For $n=1$, one can take the array from one element $(1)$.

Now assume an array for $n$ is alrady constructed.
let us exchange each occurrence of $n$ by the alternating string 
\[\underbrace{n,n+1,n,\dots}_{\text{$\lceil\tfrac\pi\eps\rceil+1$ times}}.\]

Applying Lemma \ref{lem:end-to-end-convex}, 
we get that the obtained array meets the conditions of the proposition.

The upper bound on $N$ follows directly from the construction.
\qeds



\section{Proof of Collision theorem.}\label{sec:proof-of-collision}

Here we present the proof of Collision Theorem (\ref{thm:collision})
based on the results developed in the previous section.

\parit{Proof of Collision Theorem (\ref{thm:collision}).}
Let us apply induction on $n$.

\parit{Base; $n=1$.}
The number of collisions cannot exceed $1$.  
Indeed, by the convexity of $A^1$,
if the trajectory is reflected once in $\partial A^1$, 
then it cannot return to $A^1$. 

\parit{Step.}
Assume $\gamma$ is a trajectory which meets the walls in the order $A^{i_1},\dots,A^{i_N}$ for a large integer $N$.

Consider the array 
\[\bm{A}_\gamma=(A^{i_1},\dots,A^{i_N}).\]
By induction hypothesis we get the following

\begin{clm}{}\label{clm:collision-induction hypothesis}
There is a positive integer $M$ 
such that any $M$ elements in the raw from  $\bm{A}_\gamma$ contain each $A^i$ at least once.
\end{clm}

Let $\spc{R}_\gamma$ be  the  Reshetnyak's puff pastry for  $\bm{A}_\gamma$.

Let us lift $\gamma(t)$ to $\gamma^\kay(t)$ 
for any moment of time $t$ between $\kay$ and $(\kay+1)$-th collisions.
Since $\spc{R}\in\cCat{}{0}$ (see Observation~\ref{obs:puff pastry is CAT}),
this way we lift $\gamma$ to a geodesic in $\spc{R}$.

Since $\gamma$ does not pass through $A_1\cap\dots\cap A_n$
the lift of $\gamma$ goes away form $\spc{R}_\gamma^0\cup \spc{R}_\gamma^N$.
It follows that $\spc{R}_\gamma$ is not end-to-end convex.

Let 
\[\bm{B}=(A^{j_1},\dots,A^{j_K})\] 
be the array provided by Proposition~\ref{prop:end-to-end-convex};
so $\bm{B}$ contains each $A^i$ at least once
and the puff pastry $\spc{R}_{\bm{B}}$ for $\bm{B}$ is end-to-end convex.

Assuming $N\ge K\cdot M$,
the induction hypothesis \ref{clm:collision-induction hypothesis}
implies that $\bm{A}_\gamma$ can be obtained  
by inserting a finite number of $A^i$'s in $\bm{B}$.

By Observation~\ref{obs:end-to-end-convex}, 
$\spc{R}_\gamma$ is end-to-end convex,
a contradiction.
\qeds
