%%!TEX root = all.tex
\chapter{Curves and Lipschtz maps}


\section{Curves in CBB spaces}

Let
$\spc{L}$ be a complete length $\Alex{}$ space
and $\alpha\:\II\to \spc{L}$ be a curve.
We say that $v\in\T_{\alpha(t_0)}$ is a \emph{rigth} or \emph{left} \emph{derivative} of $\alpha$ at $t_0\in\II$ (briefly, $v=\alpha^+(t_0)$ or correspondingly $v=\alpha^-(t_0)$)
if 
for any $q\not=\alpha(t_0)$, we have
\[(\distfun{q}{}{}\circ\alpha)^\pm(t_0)=(\d_{\alpha(t_0)}\distfun{q}{}{})(v).\]

%???This seems to be sufficient for us, but it is kind of weak def.For example: there is $\spc{L}$ and a curve $\alpha\:\RR\to \spc{L}$ such that $\alpha^+(0)=\0$ but \[\liminf_{t\to 0}\frac{\dist{\alpha(0)}{\alpha(t)}{}}{|t|}>0.\]Also this def. is good together with Kuratowski tangent cone, and it is not that good with the standard definition.A.???


The following lemma gives an equivalent desription of rigth and left derivatives in finite dimensional case.


\begin{thm}{Lemma}
Let $\spc{L}$ be an $m$-dimensional complete length $\Alex{}$ space and $\alpha\:\II\to \spc{L}$ be a curve.
Assume for some $t_0\in \II$ we have $v=\alpha^\pm(t_0)$, then 
for any choice of geodesics $[\alpha(t_0)\alpha(t)]$, we have
\[v=\lim_{t\to t_0\pm}\tfrac{1}{|t-t_0|}\cdot\ddir{\alpha(t_0)}{\alpha(t)}.\]

\end{thm}

\parit{Proof.}???
\qeds

Recal that right/left differentiability of curves is defined in \ref{def:diff-curv}.
For $\Alex{}$ spaces, 

\begin{thm}{Definition}\label{def:diff-curv-yet}
Let $\spc{L}$ be a complete length $\Alex{}$ space  
and $\alpha\:\II\to \spc{X}$ be a curve.

We say that $\alpha$ is \emph{differentiable} at $t_0\in\II$ if 
both $\alpha^+(t_0)$ and $\alpha^-(t_0)$ are defined and $\alpha^+(t_0)+\alpha^-(t_0)=0$; that is,  $|\alpha^+(t_0)|=|\alpha^-(t_0)|$ and 
$\mangle(\alpha^+(t_0),\alpha^-(t_0))=\pi$ if $|\alpha^\pm(t_0)|\ne 0$.
\end{thm}





Let us remind that our sign convension is not standard (see also Section~\ref{sec:conv-real}). 
In particular, for $\alpha\:\RR\to \RR$, $\alpha(t)=t$ we have $\alpha^+\equiv 1,\ \ \alpha^-\equiv -1$.

If $\alpha$ right/left differentiable at $t_0$.
Then for any point $q\not=\alpha(t_0)$
\[(\distfun{q}{}{}\circ\alpha)^\pm(t_0)
=
(\d_{\alpha(t_0)}\distfun{q}{}{})(\alpha^\pm(t_0)).\]
Thus, both derivatives $(\distfun{q}{}{}\circ\alpha)^\pm(t_0)$ are also defined.
The following lemma shows that converse holds for finite-dimensional Alexandrov's spaces.


\begin{thm}{Lemma}\label{lem:count-der}
Let $\spc{L}$ be an $m$-dimensional complete length $\Alex\kappa$ space,
$\II$ be a real inteval,
$\alpha\:\II\to \spc{L}$ be a curve,
$t_0\in\II$ 
and $Q\subset \spc{L}$ be a subset which is dense in a neighborhood of $\alpha(t_0)$.
For $q\in Q$, set $f_q=\distfun{q}{}{}\circ\alpha$.
Then 
\begin{subthm}{lem:count-der:rl}
If right (left) derivative $f^+_q(t_0)$ ($f^-_q(t_0)$) is defined for each $q\in Q\backslash \{\alpha(t_0)\}$, then $\alpha$ is right (left) differentiable at $t_0$.
\end{subthm}

\begin{subthm}{} If moreover $f'_q(t_0)$ is defined for each $q\in Q$, then $\alpha$ is differentiable at $t_0$.
\end{subthm}
\end{thm}



\parit{Proof.}
Without loss of generality we can assume $t_0=0$. Set $p=\alpha(0)$.

Assume right derivatives $f^+_q(0)$ are defined for each $q\in Q$.

Let us show that for any sequence $t_n\to 0+$, the sequence $\frac{\dist{p}{\alpha(t_n)}{}}{t_n}$ is bounded.
Assume the contrary;
let us pass to a subsequence of $t_n$ so that $\dir p{\alpha(t_n)}\to\xi$.
Take a point $q\in Q$ such that $\mangle(\xi,\dir p q)<\tfrac\pi4$.
Then for large $n$ we have 
\[\frac{\dist{p}{\alpha(t_n)}{}}{t_n}
<
2\cdot\frac{f_q(0)-f_q(t_n)}{t_n}
<
4\cdot|f^+_q(0)|,\] a 
contradiction.

Thus, to prove the first part of the lemma, 
it is sufficient to show that if $v_1$ and $v_2$ are two partial limit of $\ddir{p}{\alpha(t)}$ for $t\to 0+$, then $v_1=v_2$.

For both, $v_1$ and $v_2$, we have
\[f_q^+(0)=(\d_p\distfun{q}{}{})(v_1)=(\d_p\distfun{q}{}{})(v_2).\]
We can choose $q$ so that $|v_1|\cdot\d_p\distfun{q}{}{}$ is arbitrary close to function $w\mapsto -\<v_1,w\>$.
Thus $\<v_1,v_2\>=\<v_1,v_1\>$ and similarly, $\<v_1,v_2\>=\<v_2,v_2\>$. 
Hence $v_1=v_2$.
\qeds

According to Rademacher's theorem, any Lipscitz real-to-real function is differentiable almost everywhere. Here its extension for Lipschitz curves in $\Alex{}$ spaces.

\begin{thm}{Rademacher's theorem}\label{thm:velocity}
Let $\II$ be a real interval, $\spc{L}$ be an $m$-dimensional complete length $\Alex\kappa$ space
and $\alpha\:\II\to \spc{L}$ be a Lipschitz curve.
Then 

\begin{subthm}{vel-exist} the tangent vectors $\alpha$ is differentiable almost everywhere in $\II$;
\end{subthm}

\begin{subthm}{vel-length}
$\length\alpha=\int_{\II}|\alpha^\pm(t)|\cdot\d t$.
\end{subthm}

\end{thm}

\parit{Proof.}
Choose a countable everywhere dense subset $Q\subset \spc{L}$.
Note that for any $q\in Q$, the function $f_q(t)=\dist{q}{\alpha(t)}{}$ is Lischitz.
Thus, according to standard Rademacher's theorem for real-to-real function \cite[3.1.6]{federer}, the derivatives $f_q'(t)$ are defined for almost all $t\in\II$ and each $q\in Q$.
Thus, (\ref{SHORT.vel-exist}) follows directly from the Lemma~\ref{lem:count-der}.

Part (\ref{SHORT.vel-length}) follows from \cite[2.7.4]{BBI}
???should we write it here???
\qeds

\section{Rademacher's theorem}

The original Rademacher's theorem (see \cite[3.1.6]{federer})
states that a Lipschitz map from Euclidean space to Euclidean space is differentiable at almost all points.
Here we give a versions of this theorem for maps with targets
general metric space
and also Alexandrov spaces, both $\Alex{}$ and $\CAT{}$.
These generalizations appear in \cite{kirchheim} and \cite{lytchak:diff}.

First let us define a differentiability of map 
with target in arbitrary metric space.

\begin{thm}{Definition}
Let $m$ be positive integer and $A\subset \RR^m$.

A pseudometric $\dist{}{}{}$ on $A$ is said to be \emph{differentiable}\index{differentiable pseudometric}
at $a\in A$ if there is a prenorm $\|{*}\|$ on $\RR^m$ such that
\[\dist{x}{y}{}=\|x-y\|+o(|x-a|+|y-a|).
\eqlbl{eq:met-diff}\]
???IT SHOULD BE EUCLID METRIC IN o-SMALL.

Let $\spc{X}$ be a metric space.
A map $\map\:A\to\spc{X}$ is called \emph{metrically differentiable}\index{metrically differentiable map} at $a\in A$ if the pulled back pseudometric 
\[\dist{x}{y}{A}
\df
\dist{\map(x)}{\map(y)}{\spc{X}}\]
is differentiable at $a$.
In this case, the norm $\|{*}\|$ which satisfies \ref{eq:met-diff} is called \emph{metric differentaial} at $a$.
\end{thm}

Note that there are maps which are metrically differentiable,
but not differentiable at a given point,
even if the target space is Euclidean.
For example, consider map $\RR\to\RR^2$,
defined as $t\mapsto (t\cdot\sin\alpha(t), t\cdot\cos\alpha(t))$,
where $t\mapsto \alpha(t)$ is a smooth function such that 
$\alpha(t)=\alpha(-t)$ for any $t$,
$\tfrac{\alpha'(t)}t\to 0$ as $t\to 0$, 
but $\alpha(t)\to \infty$ as $t\to 0+$.

The following variation of Rademacher's theorem 
is proved by Kirchheim in \cite{kirchheim}.

\begin{thm}{Rademacher's theorem}\label{thm:rademacher}
Let $\spc{X}$ be a metric space 
and $\map\:\EE^m\subto\spc{X}$ be a locally Lipschitz submap.
Then $\Phi$ is metrically differentiable almost everywhere.
\end{thm}

\parit{Proof.}

\qeds

\begin{thm}{Rademacher's theorem for $\Alex{}$ spaces}
Let $\spc{L}$ be a complete length $\Alex{}$ space 
and $\map\:\EE^m\subto\spc{L}$ be a locally Lipschitz submap.
Let $A\subset \Dom \map$ be the set where $\map$ is differentiable;
that is, $a\in A$ if and only if $\d_a\map\:\EE^m\to\T_{\map(a)}$ is defined and linear.
Then 
\begin{subthm}{}
Then $A$ is a subset of full measure in $\Omega$.
\end{subthm}

\begin{subthm}{}
For any $a\in A$,
the map $\EE^m\to\RR$ defined by $v\mapsto |\d_a(v)|$ is metric differential at $a$.
\end{subthm}

\begin{subthm}{}
For any $a\in A$,
the map $\EE^m\to\RR$ defined by $v\mapsto |\d_a(v)|^2$ is a quadratic form.
\end{subthm}

\begin{subthm}{}
For any $a\in A$, the tangent space
$\T_{\map(a)}$ admits splitting
\[\T_{\map(a)}=L\oplus L^\bot,\]
where $L=\Im\d_a\map$.
\end{subthm}
\end{thm}

\begin{thm}{Rademacher's theorem for $\CAT{}$ spaces}\label{thm:rademacher-cba}
Let $\spc{U}$ be a complete length $\CAT{}$ space
and $\map\:\EE^m\subto\spc{U}$ be a locally Lipschitz submap.
Let $A\subset \Dom \map$ be the set where $\map$ is differentiable;
that is, $a\in A$ if and only if $\d_a\map\:\EE^m\to\T_{\map(a)}$ is defined and linear.
Then 
\begin{subthm}{}
Then $A$ is a subset of full measure in $\Omega$.
\end{subthm}

\begin{subthm}{}
For any $a\in A$,
the map $\EE^m\to\RR$ defined by $v\mapsto |\d_a(v)|$ is metric differential at $a$.
\end{subthm}

\begin{subthm}{}
For any $a\in A$,
the map $\EE^m\to\RR$ defined by $v\mapsto |\d_a(v)|^2$ is a quadratic form.
\end{subthm}

\begin{subthm}{thm:rademacher-cba:splitting}
If in addition $\spc{U}$ is separable, then for almost all $a\in A$, 
the tangent space
$\T_{\map(a)}$ admits splitting
\[\T_{\map(a)}=L\oplus L^\bot,\]
where $L=\Im\d_a\map$.
\end{subthm}
\end{thm}

Note that if $\spc{U}$ is the $\RR$-tree
(see ???), then any isometric embedding $\RR\to \spc{U}$ violates the conclusion in (\ref{SHORT.thm:rademacher-cba:splitting}).
In particular, the assumption that $\spc{U}$ is separable is necessary in (\ref{SHORT.thm:rademacher-cba:splitting}).


\begin{thm}{Corollary}\label{cor:rademacher-cbb-sourse}
Let $\spc{L}$ be an $m$-dimensional complete length $\Alex{}$ space and $\spc{N}$ be either $\CAT{}$ or $\Alex{}$ space.
Then any locally Lipschitz submap $f\:\spc{L}\subto\spc{N}$ is linearly differentiable almost everywhere in $\Dom f$. 
\end{thm}

\parit{Proof of the corollary.}
Consider a distance chart $\bm{x}\:\spc{L}\subto \RR^m$.
Since $\bm{x}$ is bi-Lipschitz,
the map $f\circ\bm{x}^{-1}\:\RR^m\subto\spc{N}$ Lipschitz.
According to Rademacher's theorem,
$h=f\circ\bm{x}^{-1}$ is linearly differentiable almost everywhere
in $\Dom h=\Dom\bm{x}^{-1}\cap \bm{x}(\Dom f)$.

First let us show the following.

\begin{clm}{}
$\bm{x}$ is linearly differentiable almost everywhere in $\Dom \bm{x}$.
\end{clm}


Indeed, by Rademacher's theorem,
$\bm{x}^{-1}$ is differentiable almost everywhere in $\Dom\bm{x}^{-1}=\Im \bm{x}$.
Since $\bm{x}$ is bi-Lipschitz,  
if $\d_{\bm{x}(p)}\bm{x}^{-1}$ is defined then it has to be invertable.
Thus the right hand side of the following expression is defined
\[\d_p\bm{x}
=
(\d_{\bm{x}(p)}\bm{x}^{-1})^{-1}.\]
Hence the claim follows. 
\claimqeds

Set $\Omega=\Dom h\circ \bm{x}=\Dom f\cap \Dom \bm{x}$.
Note that $h\circ \bm{x}=(f|\Omega)$ is linearly differentiable almost everywhere in $\Omega$.

According to ???, almost all points in $\spc{L}$ can be covered by distance charts;
hence the result.
\qeds


\begin{thm}{Lemma}
Let $\spc{X}$ be a metric space 
and $\alpha\:[t_{\min},t_{\max}]\to\spc{X}$ be unit-speed??? curve.
Then there is a subset $A\subset \II$ of full measure 
such that for any $a\in A$ we have 
\[\dist{\alpha(x)}{\alpha(y)}{}=|x-y|+o(|x-a|+|y-a|).
\eqlbl{eq:lem:rademacher-1}\]
\end{thm}

\parit{Proof.}
Note that
\[\dist{\alpha(x)}{\alpha(y)}{}\le|x-y|
\eqlbl{eq:short}
\]
for any $x,y\in\II$.
Therefore, it is sufficient to prove \ref{eq:lem:rademacher-1} only for $x\le a\le y$.

Given $\eps>0$, denote by $Z_{\eps}$
the set of all $a\in\II$ for which 
there are $x$ and $y$ arbitrary close to $a$
such that 
\[\dist{\alpha(x)}{\alpha(y)}{}<(1-\eps)\cdot|x-y|
\ \t{and}\ 
x\le a\le y
\eqlbl{eq:vitali}.\]
Denote by $\ell_\eps$ the Lebegues measure of $Z_\eps$.

Assume $\ell_\eps>0$ for some $\eps>0$. 
Choose $\eps_0>0$ so small that for
any partition $t_{\min}= t_0 < y_1 <\dots <t_n = t_{\max}$ such that $t_i-t_{i-1}<\eps_0$ for any $i$, 
we have
\[
\length\alpha
-
\sum_{i=1}^n\dist{\alpha(t_{i-1})}{\alpha(t_i)}{}
<
\ell_\eps???
\]
Consider the set $\mathfrak B$ of all intervals of
the form $[x,y]$ such that $x$ and $y$ satisfy \ref{eq:vitali} for some $a$.
Applying Vitali's covering theorem (see ???)
we can extract from $\mathfrak B$ a countable collection of disjoint
intervals $[x_i,y_i]$, $i\in\NN$ that covers $Z_\eps$ up to a set of zero measure;
moreover we can assume that for each $i$ we have $y_i-x_i<\eps_0$. 
Hence for a sufficiently large $n$, 
\[\sum_{i=1}^n|y_i-x_i|>\tfrac12\cdot\ell_\eps\]

Since the intervals $[x_i,y_i]$ are disjoint, 
they can be included
in a partition all whose intervals are shorter than $\eps_0$. 
By the choice of $\eps_0$, we have ???
In the left-hand sum above all terms are nonnegative and those for which for some ??? satisfy
Therefore
This contradiction proves the theorem.
\qeds

\begin{thm}{Lemma}\label{lem:lytchak-diff}
Let $\spc{X}$ be a metric space,
$\alpha\:[t_{\min},t_{\max}]\to\spc{X}$ be unit-speed??? curve 
and $S\subset [t_{\min},t_{\max}]$ be a countable set.
For any $s\in S$, set $h_s(t)=\dist{\alpha(s)}{\alpha(t)}{}$.
Then there is a subset $A\subset \II$ of full measure 
such that 
\begin{subthm}{lem:lytchak-diff:a}
For any $s\in S$, the function $h_s$ is differentiable at any $a\in A$
\end{subthm}

\begin{subthm}{lem:lytchak-diff:b}
If for some $a\in A$ there is a sequence $s_n\in S$ such that $s_n\to a\pm$, then $h'_{s_n}(a)\to\pm1$.
\end{subthm}

\end{thm}

\parit{Proof.}
Note that $h_s$ is a 1-Lipschitz function, 
Therefore part \ref{SHORT.lem:lytchak-diff:a} follows from the ordinary Rademacher's theorem.

Denote by $N_\eps$ the set of all points $z\in[t_{\min},t_{\max}]$
for which there is a sequence $s_n\to z-$ such that $h_{s_n}'(z)<1-\eps$.
Assume for some $\eps>0$, the set $N_\eps$ has positve measure.
let $a$ be a Lebegue point of $N$.


To prove the second part of lemma, choose aarbitrary contable dense set $S$ in $[t_{\min},t_{\max}]$.
Note that the function $h_s\:t\mapsto\dist{\alpha(s)}{\alpha(t)}{}$ is $1$-Lipschitz.
Thus according to ordinary Rademacher's theorem, the functions $h_s$ is differentiable at a set of full measure in $[t_{\min},t_{\max}]$.
Consider set 


\qeds

The proof relies on the following lemma in measure theory.
Part (\ref{SHORT.lem:diff-1-dim:metr-diff}) is classical, ???

\begin{thm}{Lemma}\label{lem:diff-1-dim}
Let $\dist{}{}{}$ be a pseudometric on an open interval $\II$ and $T\subset\II$ be a countable subset.
Assume that 
\[\dist{x}{y}{}\le|x-y|\]
for any $x,y\in\II$.
Then there is a subset $A$ of full measure in $\II$ such that 

\begin{subthm}{lem:diff-1-dim:h'}
For any $t\in T$ the function $h_t\:x\mapsto\dist{t}{x}{}$ is differentiable at each $a\in A$.
\end{subthm}

\begin{subthm}{lem:diff-1-dim:metr-diff}
For any $a\in A$, there is $r_a\ge0$ such that
\[\dist{x}{y}{}=r_a\cdot|x-y|+o(|x-a|+|y-a|)\] 
\end{subthm}

\begin{subthm}{lem:diff-1-dim:semi-diff}
For any $a\in A$ and any sequence $t_n$ in $T$, such that $t_n\to a\pm$,
we have $h'_t(a)\to\mp r_a$.
\end{subthm}
\end{thm}

\parit{Proof.}
Cearly $h_t$ is 1-Lipschitz.
Thus part (\ref{SHORT.lem:diff-1-dim:h'}) follows from the classical Rademacher's theorem.

To prove part (\ref{SHORT.lem:diff-1-dim:semi-diff}),
assume for beggining that $\II$ is length  
To prove part 
\qeds

\section{???}

The following lemma is a baby case of the main theorem in \cite{hitzelberger-lytchak}

\begin{thm}{Lemma}\label{lem:affine=>isom}
Let $(\RR^m,\dist{}{}{})$ be a geodesic space such that 
\begin{subthm}{lem:affine=>isom:lip}
The metric $\dist{}{}{}$ is bi-Lipshitz equivalent to the canonical metric on $\RR^m$. 
\end{subthm}
\begin{subthm}{lem:affine=>isom:speed}
Any geodesic in $(\RR^m,\dist{}{}{})$ is formed by segments of 
staight line with constant-speed parametization.
\end{subthm}
Then the metric $\dist{}{}{}$ is induced by a norm on $\RR^m$.
\end{thm}

\parit{Proof.}
From uniqueness of straight segment between given points in $\RR^m$,
it follows that any straight line in $\RR^m$ forms a geodesic in $(\RR^m,\dist{}{}{})$.

Set 
\[\|\bm{v}\|_{\bm{x}}=\distfun{\bm{x}}{(\bm{x}+\bm{v})}.\]
The condition \ref{SHORT.lem:affine=>isom:speed} implies that 
\[ \|\lam\cdot\bm{v}\|_{\bm{x}}
=
|\lam|\cdot\|\bm{v}\|_{\bm{x}}
\eqlbl{eq:lem:affine=>isom:speed}\]
for any $\bm{x},\bm{v}\in\RR^m$ and $\lam\in\RR$.
Further, the condition \ref{SHORT.lem:affine=>isom:lip} implies that 
\[
\|\lam\cdot\bm{v}\|_{\bm{x}}
-
\|\lam\cdot\bm{v}\|_{\bm{x}'}
\le 
\Const\cdot \|\bm{x}-\bm{x'}\|_2
\eqlbl{eq:lem:affine=>isom:bound}\]
for any $\bm{x},\bm{x'},\bm{v}\in\RR^m$, 
$\lam\in\RR$
and some fixed $\Const\in\RR$.
Applying \ref{eq:lem:affine=>isom:speed} and \ref{eq:lem:affine=>isom:bound},
for $\lam\to\infty$, we get
 that
$\|\bm{v}\|_{\bm{x}}$ does not depend on $\bm{x}$.
Equivalently, there is a norm $\|{*}\|$ on $\RR^m$ such that
\[\dist{\bm{x}}{\bm{y}}{}=\|\bm{x}-\bm{y}\|.\]
\qedsf



\section{Exercises}

\begin{thm}{Exercise}
Construct a infinite-dimesnional complete length $\Alex{\kappa}$ space $\spc{L}$ and a curve $\alpha\:\RR\to \spc{L}$ such that 
$\alpha^+(0)=\0$ 
but $\liminf_{t\to 0}\dist{\alpha(0)}{\alpha(t)}{}/|t|>0$.
\end{thm}













