%%!TEX root = the-dim.tex
%arXiv
\chapter{Dimension %ready
}

In Section~\ref{sec:prelim:dim}, we give definitions of different type of dimension-like invariants which have sense for metric spaces and state general relations between them.
In sections \ref{sec:dim>m} and \ref{sec:dim=m},
we prove equivalence of all reasonable dimension-like invariants for $\Alex{}$ spaces.
In Section~\ref{sec:dim-cba} we discuss the dimension of  $\CAT{}$ spaces.

\section{Preliminaries.}\label{sec:prelim:dim}

The proof of most of the statements in this section can be found in the book of Witold Hurewicz and Henry Wallman \cite{top-dim}, 
the rest follows directly from the definitions.

\begin{thm}{Hausdorff dimension}
\label{def:HausDim}\index{dimension!Hausdorff dimension}
Let $\spc{X}$ be a metric space. 
Its Hausdorff dimension is defined as
\[\HausDim\spc{X}=\sup\bigl\{\alpha\in\RR\ \mid \HausMes_\alpha(\spc{X})>0\bigr\},\]
 where $\HausMes_\alpha$ denotes $\alpha$-dimensional Hausdorff measure;
it is defined in Section \ref{sec:mes+balls}.
\end{thm}

Let $\spc{X}$ be a metric space and $\{V_\beta\}_{\beta\in\IndexSet[2]}$
 be its open cover.
Let us recall two notions in general topology:
\begin{itemize}
\item The \emph{order}\index{order} of $\{V_\beta\}$ is supremum of all integers $n$ such that there is a collection of $n+1$ elements of $\{V_\beta\}$ which has nonempy intersection.
\item An open cover $\{W_\alpha\}_{\alpha\in\IndexSet}$ of $\spc{X}$ is called \emph{refinement}\index{refinement} of  $\{V_\beta\}_{\beta\in\IndexSet[2]}$ if for any $\alpha\in\IndexSet$ there is $\beta\in\IndexSet[2]$ such that $W_\alpha\subset V_\beta$.
\end{itemize}

\begin{thm}{Topological dimension}
\label{def:TopDim}
\index{dimension!topological dimension}
Let $\spc{X}$ be a metric space. 
The topological dimension of $\spc{X}$ is defined as the minimum of nonnegative integers $n$, 
such that for any finite open cover of $\spc{X}$ there is a finite open refinement with order~$n$.

If no such $n$ exists, the topological dimension of $\spc{X}$ is infinite.

The topological dimension of $\spc{X}$ will be denoted by $\TopDim\spc{X}$.
\end{thm}

The invariants satisfying the following two statements \ref{dim-axiom-norm} and \ref{dim-axiom-sigma} are commonly called ``dimension'';
by that reason we call them axioms.

\begin{thm}{Normalization axiom}
\label{dim-axiom-norm} For any $m\in\ZZ_{\ge0}$,
\[\TopDim\EE^m=\HausDim\EE^m=m.\]

\end{thm}

\begin{thm}{Cover axiom}\label{dim-axiom-sigma} 
If $\{A_n\}_{n=1}^\infty$ is a countable closed cover of $\spc{X}$, then
\begin{align*}
\TopDim \spc{X}&=\sup\nolimits_n\{\TopDim A_n\},
\\
\HausDim \spc{X}&=\sup\nolimits_n\{\HausDim A_n\}.
\end{align*}

\end{thm}

\parbf{On product spaces.} Let us mention that the following two inequalities
\begin{align*}
\TopDim  (\spc{X}\times\spc{Y})
&\le 
\TopDim \spc{X}+ \TopDim\spc{Y},
\intertext{and}
\HausDim (\spc{X}\times\spc{Y})
&\ge 
\HausDim \spc{X}+ \HausDim\spc{Y},
\end{align*}
hold for any pair of metric spaces $\spc{X}$ and $\spc{Y}$.

These inequalities in might be strict.
For the topological dimension it holds for apair of Pontryagin surfaces constructed in \cite{pontyagin-surface}.
For Hausdorff dimension, an example was constructed by Abram Besicovitch and Pat Moran in \cite{besicovitch-moran}.

\medskip
 
The following theorem follows from \cite[theorems V 8 and VII 2]{top-dim}.

\begin{thm}{Szpilrajn's theorem}\label{thm:szpilrajn} 
Let $\spc{X}$ be a separable metric space.
Assume $\TopDim\spc{X}\ge m$. Then $\HausMes_m \spc{X}>0$.

In particular, 
$\TopDim\spc{X}\le\HausDim\spc{X}$.
\end{thm}

In fact it is true that for any separable metric space $\spc{X}$ we have
\[\TopDim\spc{X}=\inf\{\HausDim\spc{Y}\},\]
where infimum is taken over all metric spaces $\spc{Y}$ which are homeomorphic to $\spc{X}$.

\begin{thm}{Definition}
Let $\spc{X}$ be a metric space
and $F\:\spc{X}\to\RR^m$ be  a continuous map.
A point $\bm{z}\in \Im F$ is called a  \emph{stable value}\index{stable value} of $F$
if there is $\eps>0$ such that $\bm{z}\in\Im F'$ 
for any continuous map $F'\:\spc{X}\to\RR^m$ which is $\eps$-close to $F$;
that is, $|F'(x)-F(x)|<\eps$ for all $x\in \spc{X}$.
\end{thm}

The following theorem follows from \cite[theorems VI 1$\&$2]{top-dim}.
(This theorem also holds for non-separable metric spaces, 
see \cite[???]{nagata} or \cite[3.2.10]{engelking}).

\begin{thm}{Stable value theorem}\label{thm:stable-value}
Let $\spc{X}$ be a separable metric space.
Then $\TopDim\spc{X}\ge m$ if and only if there is a map $F\:\spc{X}\to\RR^{m}$ with a stable value.
\end{thm}



\begin{thm}{Proposition}\label{thm:HausDim+Lip}
Suppose $\spc{X}$ and $\spc{Y}$ are metric spaces 
and $\map \:\spc{X}\to \spc{Y}$ satisfies
\[\dist{\map (x)}{\map (x')}{}\ge \eps\cdot\dist[{{}}]{x}{x'}{}\]
for some fixed $\eps>0$ and any pair $x,x'\in \spc{X}$.
Then
\[\HausDim \spc{X}\le \HausDim \spc{Y}.\]

In particular, if there is a Lipschitz onto map $\spc{Y}\to \spc{X}$, then  
\[\HausDim \spc{X}\le \HausDim \spc{Y}.\]

\end{thm}

In addition to $\HausDim$ and $\TopDim$, 
we will use yet another one, which we call \emph{linear dimension}.

Recall that \emph{cone map} is a map between cones which respects the cone multiplication.

\begin{thm}{Definition of linear dimension}\label{def:lin-dim}\index{dimension!Linear dimension}
Let $\spc{X}$ be a metric space. 
The \emph{linear dimension} of $\spc{X}$ (denoted by $\LinDim\spc{X}$\index{$\LinDim$}) is defined by
\begin{multline*}
\LinDim\spc{X}=\sup\{m\in\ZZ_{\ge0}\mid\exists\,p\in \spc{X}\ \ \text{which admits an}\\ 
\text{isometric cone embedding}\ \  \EE^m\hookrightarrow \T_p\spc{X}\},
\end{multline*}
where $\EE^m$ denotes $m$-dimensional Euclidean space 
and $\T_p\spc{X}$ denotes tangent space of $\spc{X}$ at $p$ (defined in Section~\ref{sec:tangent-space+directions}).
\end{thm}

Note that $\LinDim$ takes values in $\ZZ_{\ge0}\cup\{\infty\}$.
 
For general metric spaces, $\LinDim$ has almost no  relations to $\HausDim$ and $\TopDim$.
Also, $\LinDim$ does not satisfy the cover axiom
 (\ref{dim-axiom-sigma}).
For $\LinDim$, an inverse of the product inequality holds; that is,
\[\LinDim(\spc{X}\times \spc{Y})
\ge
\LinDim\spc{X}+ \LinDim\spc{Y}
\eqlbl{eq:inverse-product-axiom}\] 
for any two metric spaces $\spc{X}$ and $\spc{Y}$; 
that is easy to check. 
According to the following exercise, he inequality \ref{eq:inverse-product-axiom} might be strict for some spaces, see also \cite{schroeder-foetch}.

\begin{thm}{Exercise}\label{ex:schroeder-foetch}
Construct two norms on $\RR^{10}$ such that 
non of the corresponding metric spaces $\spc{X}$ and $\spc{Y}$
have a isometric copy of $\EE^2$ but
$\spc{X}\times\spc{Y}$ has isometric copy of $\EE^{10}$.
\end{thm}

The linear dimension will be applied only to  Alexandrov spaces and to their open subsets (in both cases of curvature bounded below and above).
As we shall see, in all these cases, $\LinDim$  behaves nicely and  is easy to work with.

\parbf{Remarks.}
Linear dimension was first introduced by Conrad Plaut in \cite{plaut:survey}
under the name \emph{local dimension}\index{local dimension}\index{dimension!local dimension}. 
\emph{Geometric dimension}\index{geometric dimension}\index{dimension!geometric dimension} introduced in by Bruce Kleiner in \cite{kleiner} is closely related; 
it coincides %should we explain why???
 with the linear dimension for $\Alex{}$ and $\CAT{}$ spaces.

One could modify the definition of linear dimension by taking an arbitrary $n$-dimensional Banach space instead of the Euclidean $n$-space.
Such a definition makes more sense for general metric spaces.
This dimension is additive with respect to the direct product (that is, we always have equality in \ref{eq:inverse-product-axiom}). 

For Alexandrov spaces (either CBB or CAT) this modification is equivalent to the definition we use in this book.













\section{Dimension theorem for CBB %???
spaces}\label{sec:dim>m}

As the main dimension-like invariant, we will use  the linear dimension $\LinDim$; 
see Definition~\ref{def:lin-dim}, in other words by default \emph{dimension} means \emph{linear dimension}. 


The following theorem is the main result of this section.


\begin{thm}{Theorem}\label{thm:dim-infty}
Let $\spc{L}$ be a complete length $\Alex{\kappa}$ space, 
$q\in \spc{L}$, 
$R>0$ 
and $m\in \ZZ_{\ge0}$.
Then the following statements are equivalent:
\begin{subthmA}{LinDim}  $\LinDim\spc{L}\ge m$.
\end{subthmA}

\begin{subthmA}{thm:dim-infty:rank}
There is a point $p\in\spc{L}$ which admits a $\kappa$-strutting array $(b,a^1,\dots,a^m)\in\spc{L}^{m+1}$.
\end{subthmA}

\begin{subthmA}{LinDim+} The set 
\begin{center}
$\Euk^m=\{p\in \spc{L}\mid$ there is an isometric cone embedding $\EE^m\hookrightarrow \T_p\}$            \end{center} 
\noi contains a dense G-delta set in $\spc{L}$.
\end{subthmA}

\begin{subthmA}{TopDim} There is a $C^{\frac{1}{2}}$-embedding; that is, an embedding which is bi-H\"older with exponent $\tfrac{1}{2}$,
\[\cBall[1]_{\EE^m}\hookrightarrow \oBall(q,R).\]
\end{subthmA}.

\begin{subthmA}{pack} 
\[\pack_\eps \oBall(q,R)>\frac{\Const}{\eps^m}\]
for some fixed $\Const>0$ and any $\eps>0$.
\end{subthmA}

\medskip

In particular:
\begin{enumerate}[(i)]
\item If $\LinDim\spc{L}=\infty$, then all the statements (\ref{SHORT.LinDim+}), (\ref{SHORT.TopDim}) and (\ref{SHORT.pack}) are satisfied for all $m\in\ZZ_{\ge0}$. 
\item 
 If the statement (\ref{SHORT.TopDim}) or (\ref{SHORT.pack}) is satisfied for some choice of $q\in \spc{L}$ and $R>0$, then it also is satisfied for any other choice of $q$ and $R$.
\end{enumerate}
\end{thm}

For finite-dimensional spaces, Theorem~\ref{thm:dim-finite} gives a stronger version 
of the theorem above.

The proof of the above theorem with exception of  statement~(\ref{SHORT.TopDim}) was given in \cite{plaut:dimension}.
At that time, it was not known whether for any complete length $\Alex\kappa$ space $\spc{L}$,
\[\LinDim\spc{L}=\infty\ \ \Rightarrow\ \ \TopDim\spc{L}=\infty.\]
The latter implication was proved by Grigory Perelman;
it was done by combining an idea of Conrad Plaut with the technique of gradient flow; see \cite{perelman-petrunin:qg}.
The statement \ref{TopDim} which we prove is somewhat stronger.


To prove Theorem \ref{thm:dim-infty}  we will need the following three propositions.


\begin{thm}{Proposition}\label{E=T}
Let $\spc{L}$ be a complete length $\Alex{\kappa}$ space and $p\in \spc{L}$.
Assume there is an isometric cone embedding $\iota\:\EE^{m}\hookrightarrow \T_p\spc{L}$.  Then either
\begin{subthm}{}
 $\Im\iota=\T_p\spc{L}$, or
\end{subthm}

\begin{subthm}{} there is a point $p'$ arbitrarily close to $p$ such that there is an isometric cone embedding $\iota':\EE^{m+1}\hookrightarrow \T_{p'}\spc{L}$.
\end{subthm}
\end{thm}


\parit{Proof.}
Assume $\iota(\EE^{m})$ is a proper subset of $\T_p\spc{L}$.
Equivalently, there is a direction $\xi \in \Sigma_p\backslash\iota(\SS^{m-1})$,
where $\SS^{m-1}\subset \EE^m$ is the unit sphere. 

Fix $\eps>0$ so that $\mangle(\xi,\sigma)>\eps$ for any $\sigma\in \iota(\SS^{m-1})$. 
Choose a \emph{maximal $\eps$-packing} in $\iota(\SS^{m-1})$;
that is, an array of directions $\zeta^1,\zeta^2,\dots,\zeta^n\in \iota(\SS^{m-1})$ so that $n=\pack_\eps \SS^{m-1}$ and $\mangle(\zeta^i,\zeta^j)>\eps$ for any $i\not=j$.

Choose an array of points $x,z^1,z^2,\dots,z^n\in\spc{L}$ so that
$\dir p x\approx\xi$, $\dir p{z^i}\approx\zeta^i$;
here write ``$\approx$'' for ``sufficiently close''.
We can choose this array on such a way that 
$\angk{\kappa}p x{z^i}>\eps$ for all $i$ 
and $\angk{\kappa}p{z^i}{z^j}>\eps$ for all $i\not=j$.
Applying Corollary \ref{cor:euclid-subcone}, we can find a point $p'$ arbitrary close to  $p$ 
so that all directions $\dir{p'}x$, $\dir{p'}{z^1}$, $\dir{p'}{z^2},\dots,\dir{p'}{z^n}$
belong to an isometric copy of $\SS^{\kay-1}$ in $\Sigma_{p'}$.
In addition, we can assume that $\angk{\kappa}{p'}x{z^i}>\eps$ and $\angk{\kappa}{p'}{z^i}{z^j}>\eps$.
From hinge comparison (\ref{angle}),
$\mangle(\dir{p'}x,\dir{p'}{z^i})>\eps$ 
and $\mangle(\dir{p'}{z^i},\dir{p'}{z^j})>\eps$;
that is, 
\[\pack_\eps \SS^{\kay-1}\ge n+1>\pack_\eps \SS^{m-1}.\] 
Hence $\kay>m$.
\qeds


\begin{thm}{Proposition}\label{pack-homogeneus}
Let $\spc{L}$ be a complete length $\Alex{\kappa}$ space, then 
 for any two poins $p,\bar p\in \spc{L}$ and any $R,\bar R>0$ there is a contsant $\delta=\delta(\kappa,R,\bar R,\dist{p}{\bar p}{})>0$ such that
\[\pack_{\delta\cdot\eps}\oBall(\bar p,\bar R)\ge \pack_{\eps}\oBall(p,R).\]

\end{thm}

\begin{wrapfigure}{r}{45mm}
\begin{lpic}[t(-8mm),b(0mm),r(0mm),l(0mm)]{pics/two-balls(1)}
\lbl[r]{9,11;$\bar p$}
\lbl[r]{33,11;$p$}
\lbl[r]{10,5;$\bar R$}
\lbl[l]{35,5;$R$}
\lbl[l]{37,17;$x^i$}
\lbl[b]{16,13.5;$\bar x^i$}
\end{lpic}
\end{wrapfigure}

\parit{Proof.} According to \ref{cor:CAT>k-sence}, we can assume that $\kappa\le 0$.

Let $n=\pack_{\eps}\oBall(p,R)$ and $x^1,x^2,\dots, x^n\in\oBall(p,R)$ be a maximal $\eps$-packing;
that is, $\dist{x^i}{x^j}{}>\eps$ for all $i\not=j$.
Without loss of generality we can assume that $x^1,x^2,\dots, x^n\in \Str(\bar p)$.
Thus, for each $i$ there is a unique geodesic $[\bar p x^i]$ (see \ref{thm:almost.geod}).
Choose factor $s>0$, so that $\bar r>s\cdot(\dist{p}{\bar p}{}+R)$.
For each $i$, take $\bar x^i\in[\bar p x^i]$ so that 
$\dist{\bar p}{\bar x^i}{}=s\cdot(\dist{p}{x^i}{})$.
From \ref{cor:monoton:2-sides},
\[\angkk\kappa {\bar p}{\bar x^i}{\bar x^j}{}\ge\angk\kappa {\bar p}{x^i}{x^j}.\]
Applying cosine rule, we get a constant $\delta=\delta(\kappa,R,\bar R,\dist{p}{\bar p}{})>0$ such that 
\[\dist{\bar x^i}{\bar x^j}{}>\delta\cdot(\dist{x^i}{x^j}{})>\delta\cdot\eps\] 
for all $i\not=j$.
Hence the statement follows.
\qeds


\begin{thm}{Proposition}\label{E-comeagre} 
Let $\spc{L}$ be a complete length $\Alex{\kappa}$ space, 
$r\z<\varpi\kappa$ 
and $p\in \spc{L}$.
Assume that 
\[\pack_{\eps} \oBall(p,r)
>\pack_{\eps}\cBall[r]_{\Lob{m}{\kappa}}
\eqlbl{eq:pack>pack}\]
for some $\eps>0$.
Then there is a G-delta set $A\subset \spc{L}$,
which is dense in a neighborhood of $p$,
such that $\dim\Lin_q>m$ for any $q\in A$.
\end{thm}

\parit{Proof.} 
Choose a maximal $\eps$-packing in $\oBall(p,r)$;
that is, an array of points $x^1,x^2,\dots, x^n\in\oBall(p,r)$ so that $n=\pack_\eps \oBall(p,r)$ and $\dist{x^i}{x^j}{}>\eps$ for any $i\not=j$.
Choose a neighborhood $\Omega\ni p$,
such that $\dist{q}{x^i}{}<r$ for any $q\in \Omega$ and all $i$.
Set 
\[A= \Omega\cap\Str(x^1,x^2,\dots,x^n).\]
According to Theorem \ref{thm:almost.geod}, $A$ is a G-delta set which is dense in $\Omega$.

Assume $\kay=\dim\Lin_q\le m$ for some $q\in A$.
Consider an array of vectors $v^1,v^2,\dots,v^n\in \Lin_q$,
$v^i=\ddir{q}{x^i}$.
Clearly 
\[|v^i|=\dist{q}{x^i}{}<r\] 
and from hinge comparison (\ref{angle})
we have 
\[\side\kappa \hinge o{v^i}{v^j}\ge \dist{x^i}{x^j}{}>\eps.\]
Note that the ball $\oBall(\0,r)_{\Lin_q}$ equipped with the metric $\rho(v,w)\z=\side\kappa \hinge o{v}{w}$ is isometric to 
$\cBall[r]_{\Lob{\kay}{\kappa}}$.
Thus,
\[
\pack_\eps\cBall[r]_{\Lob{\kay}{\kappa}}
\ge
\pack_\eps \oBall(p,r),
\]
which contradicts $\kay\le m$ and \ref{eq:pack>pack}.
\qeds

The proof is essentially done in \ref{E=T}, \ref{pack-homogeneus}, \ref{E-comeagre}, \ref{thm:inverse-function}, \ref{thm:right-inverse-function}; 
here we only assemble the proof from these parts.

We will prove implications 
\[\textrm{(\ref{SHORT.LinDim+}) 
$\Rightarrow$ 
(\ref{SHORT.LinDim}) 
$\Rightarrow$ 
(\ref{SHORT.thm:dim-infty:rank}) 
$\Rightarrow$ 
(\ref{SHORT.pack}) 
$\Rightarrow$ 
(\ref{SHORT.LinDim+}) 
$\Rightarrow$ 
(\ref{SHORT.TopDim}) 
$\Rightarrow$ 
(\ref{SHORT.pack}).}\]

\parit{Proof of \ref{thm:dim-infty}.} 
The implication (\ref{SHORT.LinDim+})$\Rightarrow$(\ref{SHORT.LinDim}) is trivial.
The proof of (\ref{SHORT.TopDim})$\Rightarrow$(\ref{SHORT.pack}) is valid for general metric spaces;
it is based on general relations between topological dimension, Hausdorff measure and $\pack_\eps$. 

\parit{(\ref{SHORT.LinDim})$\Rightarrow$(\ref{SHORT.thm:dim-infty:rank}).}
Choose a point $p\in\spc{L}$ such that $\dim\Lin_p\ge m$.
Clearly one can choose an array of directions $\xi^0,\xi^1,\dots,\xi^m\in\Lin_p$ so that $\mangle(\xi^i,\xi^j)>\tfrac\pi2$ for all $i\not=j$.
Choose an array of points $x^0,x^1,\dots,x^m\in\spc{L}$ so that each $\dir{p}{x^i}$ is sufficiently close to $\xi^i$;
in particular, we have $\mangle\hinge{p}{x^i}{x^j}>\tfrac\pi2$.
Choose points $a^i\in\mathopen{]}p x^i\mathclose{]}$ sufficiently close to $p$.
One can do it so that each $\angk\kappa p{a^i}{a^j}$ is arbitrary close to $\mangle\hinge p{a^i}{a^j}$,
in particular, $\angk\kappa p{a^i}{a^j}>\tfrac{\pi}{2}$.
Finally, set $b=a^0$.




\parit{(\ref{SHORT.thm:dim-infty:rank})$\Rightarrow$(\ref{SHORT.pack}).} 
Let $p\in \spc{L}$ be a point which admits a $\kappa$-strutting array $b,a^1,\dots, a^m\in \spc{L}$.
The right-inverse mapping map theorem (\ref{thm:right-inverse-function:open-map})
implies that the distance map $\distfun{\bm{a}}{}{}\:\spc{L}\to\RR^m$,
\[\distfun{\bm{a}}{}{}\:x\mapsto(\dist{a^1}{x}{},\dist{a^2}{x}{},\dots,\dist{a^n}{x}{})\]
is open in a neighborhood of $p$.
Since the distance map $\distfun{\bm{a}}{}{}$ is Lipschitz, 
for any $r>0$, there is $\Const>0$ such that
\[\pack_\eps \oBall(p,r)>\frac{\Const}{\eps^m}.\]
Applying \ref{pack-homogeneus}, we get similar inequality for any other ball in $\spc{L}$;
that is, for any point $q\in\spc{L}$, $R>0$ there is $\Const'>0$ such that 
\[\pack_\eps \oBall(q,R)>\frac{\Const'}{\eps^m}.\]


\parit{(\ref{SHORT.pack})$\Rightarrow$(\ref{SHORT.LinDim+}).} 
Note that for any $q'\in\spc{L}$ and $R'>\dist{q}{q'}{}+R$ we have
\begin{align*}
\pack_\eps\oBall(q',R')
&\ge
\pack_\eps\oBall(q,R)
\ge
\\
&\ge
\frac{\Const}{\eps^m}
>
\\
&>
\pack_\eps\cBall[R']_{\Lob{m-1}{\kappa}}.
\end{align*}
for all sufficiently small $\eps>0$.
Applying \ref{E-comeagre}, we get that
\begin{center}
$\Euk^m=\{p\in \spc{L}\mid$ there is an isometric cone embedding $\iota:\EE^m\hookrightarrow \T_p\}$
\end{center} 
contains a G-delta set which is dense in a neighborhood of any point $q'\in\spc{L}$.

\parit{(\ref{SHORT.LinDim+})$\Rightarrow$(\ref{SHORT.TopDim}).} 
Since $\Euk^m$ contains a dense G-delta set in $\spc{L}$, we can choose $p\in \oBall(q,R)$ with an isometric cone embedding $\iota\:\EE^m\hookrightarrow \T_p$.

Repeating the construction in (\ref{SHORT.LinDim})$\Rightarrow$(\ref{SHORT.thm:dim-infty:rank}), 
we get a $\kappa$-strutting array $b,a^1,\dots, a^m\in \spc{L}$ for $p$.

Applying the right-inverse theorem (\ref{thm:right-inverse-function}),
we obtain a $C^{\frac{1}{2}}$-submap 
\[\map\:\RR^m\subto \oBall(q,R)\]
which is a right inverse for $\distfun{\bm{a}}{}{}\:\spc{L}\to\RR^m$ and $\map(\dist{\bm{a}}{p}{})=p$.
In particular, $\map$ is a $C^{\frac{1}{2}}$-embedding of $\Dom\map$.


\parit{(\ref{SHORT.TopDim})$\Rightarrow$(\ref{SHORT.pack}).}
Let $W\subset\oBall(q,R)$ be the image of the embedding.
Since $\TopDim W=m$,
Szpilrajn's theorem (\ref{thm:szpilrajn}) implies that
\[\HausMes_m W>0.\]

Given $\eps>0$, consider a maximal $\eps$-packing of $W$;
that is, an array of points $(x^1,x^2,\dots,x^n)$ in $W$ such that $n=\pack_\eps W$ and $\dist{x^i}{x^j}{}>\eps$ for all $i\not=j$.
Note that $W$ is covered by balls $\oBall(x^i,2\cdot\eps)$.

By the definition of Hausdorff measure,
\[\pack_\eps W
\ge
\frac{\Const}{\eps^m}\cdot\HausMes_m W\]
for a fixed constant $\Const>0$ and all small $\eps>0$.
Hence (\ref{SHORT.pack}) follows.
\qedsf








\section{Finite dimensional CBB spaces}\label{sec:dim=m}

In this section we will show that all reasonable notions of dimension coincide on the class of Alexandrov spaces with curvature bounded below.

First we will prove a  stronger version of Theorem \ref{thm:dim-infty} for the finite-dimensional case.

\begin{thm}{Theorem}\label{thm:dim-finite}
Let $\spc{L}$ be a complete length $\Alex{\kappa}$ space, 
$m$ be a nonnegative integer,
$0<R\le \varpi\kappa$ and
$q\in \spc{L}$.
Then the following statements are equivalent:
\begin{subthm}{LinDim-f}  $\LinDim\spc{L}= m$;
\end{subthm}

\begin{subthm}{thm:dim-finite:rank}
$m$ is the maximal integer, such that there is a point $p\in\spc{L}$ which admits a $\kappa$-strutting array $b,a^1,\dots,a^m$.
\end{subthm}

\begin{subthm}{LinDim+-f} $\T_p\iso \EE^m$ for any point $p$ in a dense G-delta set of $\spc{L}$.
\end{subthm}

\begin{subthm}{TopDim-f} There is an open bi-Lipschitz embedding 
\[\cBall[1]_{\EE^m}\hookrightarrow \oBall(q,R)\subset \spc{L};\]
\end{subthm}

\begin{subthm}{pack-f} For any $\eps>0$,
\[\pack_\eps\cBall[R]_{\Lob{m}{\kappa}} \ge\pack_\eps \oBall(q,R),\]
moreover, there is $\Const=\Const(q,R)>0$  such that 
\[\pack_\eps \oBall(q,R)>\frac\Const{\eps^m}.\]
\end{subthm}

\end{thm}

The above theorem was essentially proved in \cite{BGP}.

Using theorems \ref{thm:dim-infty} and \ref{thm:dim-finite}, 
one can show that linear dimension is equal to many different types of dimensions, such 
as \emph{small} and \emph{big inductive dimension} 
and \emph{upper} and  \emph{lower box counting dimensions}
(also known as \emph{Minkowski dimensions}), 
\emph{homological dimension} and so on.

The following two corollaries follow from \ref{pack-f}.

\begin{thm}{Corollary}\label{cor:dim>proper}
Any $m$-dimensional complete length $\Alex{}$ space is proper and geodesic.
\end{thm}


\begin{thm}{Corollary} Let $(\spc{L}_n)$ be a sequence of length $\Alex\kappa$ spaces and $\spc{L}_n\to \spc{L}_\o$ as $n\to\o$.
Assume $\LinDim L_n\le m$ for all $n$.
Then $\LinDim L_\o\le m$.
\end{thm}

\begin{thm}{Corollary}\label{dim=dim} 
Let $\spc{L}$ be a complete length $\Alex{\kappa}$ space. 
Then for any open $\Omega\subset \spc{L}$, we have
\[
\LinDim \spc{L}=
\LinDim\Omega =
\TopDim\Omega=
\HausDim\Omega,
\]
where $\TopDim$ and $\HausDim$ denotes  topological dimension (\ref{def:TopDim}) and Hausdorff dimension (\ref{def:HausDim}) correspondingly.

In particular, $\spc{L}$ is dimension homogeneous; that is, any open set has the same linear dimension.
\end{thm}

\parit{Proof of \ref{dim=dim}.} 
The equality
\[\LinDim \spc{L}= \LinDim\Omega\]
follows from \ref{LinDim}$\&$\ref{SHORT.LinDim+}.

If $\LinDim \spc{L}=\infty$, then
applying  \ref{TopDim}, for $\oBall(q,R)\subset \Omega$, we get that there is a compact subset $K\subset \Omega$ with arbitrary large $\TopDim K$ therefore
\[\TopDim\Omega=\infty.\] 
By Szpilrajn's theorem (\ref{thm:szpilrajn}),
$\HausDim K\ge \TopDim K$.
Thus, we also have 
\[\HausDim\Omega=\infty.\]

If $\LinDim \spc{L}=m<\infty$, then first inequality in \ref{pack-f} 
implies that \[\HausDim \oBall(q,R)\le m.\] 
According to Corollary~\ref{cor:dim>proper}, 
$\spc{L}$ is proper and in particular it has countable base. 
Thus applying Szpilrajn theorem  again, we get
\[\TopDim\Omega\le \HausDim \Omega\le m.\]
Finally, \ref{TopDim-f} implies that $m\le\TopDim\Omega$.
\qeds


\parit{Proof of Theorem~\ref{thm:dim-finite}.} The equivalence (\ref{SHORT.LinDim-f})$\Leftrightarrow$(\ref{SHORT.thm:dim-finite:rank}) follows from \ref{thm:dim-infty}.

\parit{(\ref{SHORT.LinDim-f})$\Rightarrow$(\ref{SHORT.LinDim+-f}).}
If $\LinDim\spc{L}=m$, then from Theorem~\ref{thm:dim-infty}, 
we get that $\Euk^m$ contains a dense G-delta set in $\spc{L}$.
From \ref{E=T}, it follows that $\T_p$ is isometric to $\EE^m$ for any $p\in \Euk^m$.

\parit{(\ref{SHORT.LinDim+-f})$\Rightarrow$(\ref{SHORT.TopDim-f}).} It is proved exactly the same way as implication \textit{(\ref{SHORT.LinDim+})$\Rightarrow$(\ref{SHORT.TopDim})} of theorem \ref{thm:dim-infty}, 
but we have to apply existence of distance chart (\ref{thm:inverse-function}) instead of the right-inverse theorem%\ref{thm:right-inverse-function}
.

\parit{(\ref{SHORT.TopDim-f})$\Rightarrow$(\ref{SHORT.pack-f}).} 
From (\ref{SHORT.TopDim-f}), it follows that, there is a point $p\in\oBall(q,R)$ and $r>0$ such that
$\oBall(p,r)\subset \spc{L}$ is bi-Lipschit homeomorphic to a bounded open set of $\EE^m$.
Thus, there is $\Const>0$ such that 
\[\pack_\eps \oBall(p,r)>\frac{\Const}{\eps^m}.\eqlbl{eq:thm:dim-finite*}\]
Applying \ref{pack-homogeneus}, we get that inequality \ref{eq:thm:dim-finite*}, with different constants, holds for any other ball, in particular for $\oBall(q,R)$.

Applying \ref{E-comeagre}, we get the first inequality.

\parit{(\ref{SHORT.pack-f})$\Rightarrow$(\ref{SHORT.LinDim-f}).} 
From theorem \ref{thm:dim-infty}, we have $\LinDim\spc{L}\ge m$. 
Applying theorem \ref{thm:dim-infty} again, we get that if $\LinDim\spc{L}\ge m+1$, then for some $\Const>0$ and any $\eps>0$
\[\pack_\eps \oBall(q,R)\ge \frac{\Const}{\eps^{m+1}},\]
but
\[\frac{\Const'}{\eps^m}\ge\pack_\eps \oBall(q,R),\] 
for any $\eps>0$,
a contradiction.
\qeds

The following exercise was suggested by Alexander Lytchak.

\begin{thm}{Exercise} 
Let $\spc{L}$ be a complete length $\Alex{}$ space and $\Sigma_p\spc{L}$ is compact for any $p\in\spc{L}$.
Prove that $\spc{L}$ is finite dimensional.
\end{thm}




















\section{One-dimensional CBB spaces}

\begin{thm}{Theorem}\label{thm:dim=1.CBB} 
Let $\spc{L}$ be an one-dimensional complete length $\Alex\kappa$ space.
Then $\spc{L}$ is isometric to a connected complete Riemannian $1$-dimensional manifold with possibly non-empty boundary.
\end{thm}



\parit{Proof.}
Clearly $\spc{L}$ is connected.
It remains to show the following. 
\begin{clm}{}\label{clm:1-dim-all}
For any point $p\in\spc{L}$
there is $\eps>0$ such that $\oBall(p,\eps)$ 
is isometric to either $[0,\eps)$ or $(-\eps,\eps)$.
\end{clm}

First let us show that
\begin{clm}{}\label{clm:1-dim-mid}
If $p\in\l]x y\r[$ for some $x$, $y\in\spc{L}$ and $\eps<\min\{\dist{p}{x}{},\dist{p}{y}{}\}$
then $\oBall(p,\eps)\subset\l]x y\r[$.
In particular
$\oBall(p,\eps)\iso(-\eps,\eps)$.
\end{clm}
\begin{wrapfigure}{r}{30mm}
\begin{lpic}[t(-3mm),b(-3mm),r(0mm),l(0mm)]{pics/dim=1(1)}
\lbl[t]{2,.5;$x$}
\lbl[t]{9,.5;$p$}
\lbl[t]{14,.5;$q$}
\lbl[t]{27,.5;$y$}
\lbl[b]{14,9;$z$}
\end{lpic}
\end{wrapfigure}

Assume the contrary;
that is, there is 
$$z\in \oBall(p,\eps)\backslash\l]x y\r[.$$
Consider a geodesic $[p z]$, let $q\in[p z]\cap[x y]$ be the point which maximizes the distance $\dist{p}{q}{}$.
At  $q$, we have three distinct directions: 
to $x$, to $y$ and to $z$.
Moreover, $\mangle\hinge{q}{x}{y}=\pi$.
Thus, according to Proposition~\ref{E=T}, 
$\LinDim\spc{L}>1$, a contradiction.

Now assume there is no geodesic passing thru $p$. 
Since $\LinDim\spc{L}=1$ there is a point $x\not=p$.

Fix a positive value $\eps<\dist{p}{x}{}$.
Let us show that 

{

\begin{wrapfigure}{r}{30mm}
\begin{lpic}[t(-0mm),b(-3mm),r(0mm),l(0mm)]{pics/dim=1(1)}
\lbl[t]{2,.5;$p$}
\lbl[t]{9,.5;$w$}
\lbl[t]{14,.5;$q$}
\lbl[t]{27,.5;$y$}
\lbl[b]{14,9;$z$}
\end{lpic}
\end{wrapfigure}

\begin{clm}{}\label{clm:1-dim-end}
$\oBall(p,\eps)\subset [p x]$;
in particular $\oBall(p,\eps)\iso[0,\eps)$.
\end{clm}

Assume the contrary;
let $z\in \oBall(p,\eps)\backslash[p y]$.
Choose a point $w\in \l] p y \r[$ such that 
\[\dist{p}{w}{}+\dist{p}{z}{}<\eps.\]
Consider geodesic $[w z]$, let $q\in[p y]\cap[w z]$  be the point which maximizes the distance $\dist{w}{q}{}$.
Since no geodesics pass thru $p$, we have $p\not=q$.
As above, $\mangle\hinge{q}{p}{y}=\pi$ 
and $\dir{q}{z}$ is distinct from $\dir{q}{p}$ and $\dir{q}{p}$.
Thus, according to Proposition~\ref{E=T}, 
$\LinDim\spc{L}>1$, a contradiction.

}

Clearly $\t{\ref{clm:1-dim-mid}}+\t{\ref{clm:1-dim-end}}\Rightarrow\t{\ref{clm:1-dim-all}}$;
hence the result.
\qeds

\begin{comment}
\parit{Proof 2.}
According to \ref{LinDim+-f}, there is a point $p\in\spc{L}$, such that $\T_p\iso\RR$;
that is, $\Sigma_p$ consists of two directions $\xi_+$, $\xi_-$ on angle $\pi$.
Both of directions $\xi_+$, $\xi_-$ have to be geodesic;
otherwise the set of geodesic directions $\Sigma_p'$ would not be dense in $\Sigma_p$.
Let $\gamma_+$ and $\gamma_-$ be maximal geodesics in directions $\xi_+$ and $\xi_-$ correspondingly.

The geodesics $\gamma_+$ and $\gamma_-$ cover $\spc{L}$ completely;
that follows sinse $\spc{L}$ is geodesic (\ref{cor:dim>proper}) and since geodesics in $\spc{L}$ do not bifurcate (\ref{thm:g-split}).
It is easy to see that only following cases can occur
\begin{itemize}
\item both $\gamma_+$ and $\gamma_-$ are defined on $[0,\infty)$.
In this case $\spc{L}\iso\RR$.
\item $\gamma_+$ is defined on closed interval, say $[0,a]$ and $\gamma_-$ is defined of an infinite interval $[0,\infty)$ (or vise versa).
In this case $\spc{L}\iso\RR_{\ge0}$.
\item both $\gamma_+$ and $\gamma_-$ are defined on closed intervals, say $[0,a]$ and $[0,b]$ and $\gamma_+(a)\not=\gamma_-(b)$.
In this case $\spc{L}\iso[0,a+b]$.
\item both $\gamma_+$ and $\gamma_-$ are defined on one closed interval, say $[0,a]$ and $\gamma_+(a)=\gamma_-(a)$.
In this case, $\spc{L}\iso\tfrac{a}{\pi}\blow\SS^1$.
\end{itemize}
Hence the result.\qeds
\end{comment}





\section{Dimension of CBA spaces}\label{sec:dim-cba}

We start with two examples.

The first example shows that the dimension of a complete length $\CAT{}$ spaces is not local;
that is, such spaces might have open sets with different linear dimension.

Such an example can be constructed by gluing at one point two Euclidean spaces of different dimensions.
According to Reshetnyak's gluing theorem (\ref{thm:gluing}), this construction gives a $\CAT{0}$ spaces.

The second example provides a complete length $\CAT{}$ space 
with topological dimension one and arbitrary large Hausdorff dimension.
Thus, for complete length $\CAT{}$ spaces, one should not expect any relations between topological and Hausdorff dimensions except for the one provided by Szpilrajn's theorem (\ref{thm:szpilrajn}).

To construct the second type of examples,
note that the completion of any metric tree has topological dimension one and is $\CAT\kappa$ for any $\kappa$.
Start with a binary tree $\Gamma$, and a sequence $\eps_n>0$ such that $\sum_n\eps_n<\infty$.
Define the metric on $\Gamma$
by prescribing length of edge from level $n$ to the level $n+1$ to be  $\eps_n$.
For an appropriately chosen sequence $\eps_n$, the completion of $\Gamma$ will contain a Cantor set of arbitrary large dimension.

\medskip

The following is a  version of the theorem proved by Bruce Kleiner in \cite{kleiner} with an improvement made by Alexander Lytchak in \cite{lytchak:diff}.

\begin{thm}{Theorem}\label{thm:dim-infty-CBA}
For any complete length $\CAT\kappa$ space $\spc{U}$, the following statements are equivalent:

\begin{subthm}{LinDim-CBA}  $\LinDim\spc{U}\ge m$;
\end{subthm}

\begin{subthm}{thm:dim-infty-CBA:bary} 
For some $z\in \spc{U}$ there is an array of $m+1$ balls $B^i\z=\oBall(a^i,r^i)$ with $a^0,a^1,\dots,a^m\in \oBall(z,\frac{\varpi\kappa}2)$  
such that 
\[\bigcap_i B^i=\emptyset,\quad\bigcap_{i\not=j} B^i\not=\emptyset\]
for each $j$.
\end{subthm}


\begin{subthm}{thm:dim-infty-CBA:mnfld} 
There is a $C^{\frac{1}{2}}$-embedding $\map\:\cBall[1]_{\EE^m}\hookrightarrow \spc{U}$;
that is, $\map$ is bi-H\"older with exponent $\tfrac{1}{2}$.
\end{subthm}

\begin{subthm}{thm:dim-infty-CBA:TopDim}
There is a closed separable set $K\subset\spc{U}$ such that 
\[\TopDim K\ge m.\]
\end{subthm}

\end{thm}

\parbf{Remarks.}
Theorem \ref{thm:loc-lip-inverse} gives a stronger version of part (\ref{SHORT.thm:dim-infty-CBA:mnfld}) in finite dimensional case.
Namely, a complete length $\CAT{}$ space with linear dimension $m$ 
admits a bi-Lipschitz embedding $\map$ of an open set of $\RR^m$.
Moreover, the Lipschitz constants of $\map$ can be made arbitrary close to $1$.

\begin{thm}{Corollary}\label{cor:dim-CBA}
For any separable complete length $\CAT{}$ space $\spc{U}$, we have
\[\TopDim\spc{U}=\LinDim\spc{U}.\]

\end{thm}


To prove Theorem~\ref{thm:dim-infty-CBA} we will need the following lemma.

Any simplicial complex can be equipped with a length-metric
such that each $\kay$-simplex  
is isometric to the standard simplex
\[\Delta^\kay
=
\set{(x_0,\dots,x_\kay)\in \RR^{\kay+1}}
{x_i\ge 0,\quad x_0+\dots+x_\kay=1}\]
with the metric induced by the $\ell^1$-norm on $\RR^{\kay+1}$.
This metric will be called the  \emph{$\ell^1$-metric}\index{$\ell^1$-metric} on the simplicial complex.

\begin{thm}{Lemma}\label{lem:approximation-cba}
Let $\spc{U}$ be a complete length $\CAT\kappa$ space
and $\rho\:\spc{U}\to\RR$ be a continuous positive function.
Then there is a simplicial complex $\spc{N}$ equiped with $\ell^1$-metric,
a locally Lipschitz map $\map\:\spc{U}\to \spc{N}$ 
and a Lipschitz map $\map[2]\:\spc{N}\to\spc{U}$ such that

\begin{subthm}{lem:approximation-cba:displacement}
The \emph{displacement} of composition $\map[2]\circ\map\:\spc{U}\to\spc{U}$ is bounded by $\rho$;
that is,
\[\dist{x}{\map[2]\circ\map(x)}{}<\rho(x)\] 
for any $x\in\spc{U}$
\end{subthm}

\begin{subthm}{lem:approximation-cba:im}
If $\LinDim\spc{U}\le m$ 
then the $\map[2]$-image of any closed simplex in $\spc{N}$ 
coincides with the image of its $m$-skeleton.
\end{subthm}

\end{thm}

\parit{Proof.}
Without loss of generality, we can assume that  for any $x$ we have $\rho(x)\z<\rho_0$
for some fixed $\rho_0<\tfrac{\varpi\kappa}{2}$.

By Stone's theorem%ref???
, any metric space is paracompact.
Thus, we can choose a locally finite covering $\set{\Omega_\alpha}{\alpha\in\IndexSet}$ of $\spc{U}$ such that $\Omega_\alpha\subset \oBall(x,\tfrac{1}{3}\cdot\rho(x))$ for any $x\in \Omega_\alpha$. 

Denote by $\spc{N}$ the nerve of the covering  $\{\Omega_\alpha\}$;
that is, $\spc{N}$ is an abstract simplicial complex with set of vertexes formed by $\IndexSet$ such that
$\{\alpha^0,\alpha^1,\dots,\alpha^n\}\subset\IndexSet$ 
are vertexes of a simplex  if and only if
$\Omega_{\alpha^0}
\cap
\Omega_{\alpha^1}
\cap\dots\cap
\Omega_{\alpha^n}
\not=
\emptyset$.

Fix a Lipschitz partition of unity 
$\phi_\alpha\:\spc{U}\to [0,1]$,  subordinate to $\{\Omega_\alpha\}$.
Consider the map $\map\:\spc{U}\to \spc{N}$ such that the barycentric coordinate of $\map(p)$ is $\phi_\alpha(p)$.
Note that $\map$ is locally Lipschitz. 
Clearly $\map$-preimage of any open simplex in $\spc{N}$ lies in $\Omega_\alpha$ for some $\alpha\in\IndexSet$.

For each $\alpha\in\IndexSet$, 
choose $x_\alpha\in\Omega_\alpha$.
Let us extend the map $\alpha\mapsto x_\alpha$
to a map $\map[2]\:\spc{N}\to\spc{U}$ which is $\kappa$-barycentric on each simplex;
according to Theorem~\ref{thm:cat-bary-web:Lip}, this extension exists, 
$\map[2]$ is Lipschtz
and its Lipschitz constant depends only on $\rho_0$ and $\kappa$.

\parit{(\ref{SHORT.lem:approximation-cba:displacement})}
Fix $x\in\spc{U}$.
Denote by $\Delta$ the minimal simplex which contains $\map(x)$;
and let $(\alpha^0,\alpha^1,\dots,\alpha^n)$ be the vertexes of $\Delta$.
Note that $\alpha$ is a vertex of $\Delta$ if and only if $\phi_{\alpha}(x)>0$.
Thus,
\[\dist{x}{x_{\alpha^i}}{}<\tfrac{1}{3}\cdot\rho(x)\]  
for any $i$.
Therefore 
\[\diam\map[2](\Delta)
\le
\max_{i,j}\{\dist{x_{\alpha^i}}{x_{\alpha^j}}{}\}
<
\tfrac{2}{3}\cdot\rho(x).\]
In particular 
\[\dist{x}{\map[2]\circ\map(x)}{}\le\dist{x}{x_{\alpha^0}}{}+\diam \map[2](\Delta) <\rho(x).\]

\parit{(\ref{SHORT.lem:approximation-cba:im})}
Assume the contrary;
that is, $\map[2](\spc{N})$ is not inculded in the $\map[2]$-image of $m$-skeleton of $\spc{N}$.
Then for some $\kay>m$,
there is a $\kay$-simplex $\Delta^\kay$ in $\spc{N}$,
such that the baricentric simplex $\sigma=\map[2]|\Delta^\kay$ is nondegenrate; 
that is, 
$$W=\map[2](\Delta^\kay)\backslash\map[2](\partial\Delta^\kay)\not=\emptyset.
$$
Applying Corollary~\ref{cor:LinDim>bary},
we get that $\LinDim\spc{U}\ge \kay$, a contradiction.
\qeds






\parit{Proof of Theorem~\ref{thm:dim-infty-CBA}.} 
Note that
\begin{itemize}
\item The implication (\ref{SHORT.thm:dim-infty-CBA:bary})$\Rightarrow$(\ref{SHORT.thm:dim-infty-CBA:mnfld})
follows directly from Lemma~\ref{lem:nondeg-test-with-balls}
and Theorem~\ref{thm:cat-bary-web:mnfld}.
\item The implication 
(\ref{SHORT.thm:dim-infty-CBA:mnfld})$\Rightarrow$(\ref{SHORT.thm:dim-infty-CBA:TopDim}) 
is trivial.
\end{itemize}
 
\parit{(\ref{SHORT.thm:dim-infty-CBA:TopDim})$\Rightarrow$(\ref{SHORT.LinDim-CBA}).}
According to Theorem~\ref{thm:stable-value}, 
there is a continuous map $f\:K\to \RR^{m}$ with a stable value.
By Tietze extension theorem%ref???
, it is possible to extend $f$ 
to a continuous map $F\:\spc{U}\to \RR^{m}$.

Fix $\eps>0$.
Since $F$ is continuous, there is a continuous positive function $\rho$ defined on $\spc{U}$ such that 
\[\dist{x}{y}{}<\rho(x)\ \ \Rightarrow\ \ | F(x)- F(y)|<\tfrac13\cdot\eps.\]
Apply Lemma~\ref{lem:approximation-cba} for the function $\rho$.
For the obtained simplicial complex $\spc{N}$, 
 and maps $\map\:\spc{U}\to \spc{N}$,  $\map[2]\:\spc{N}\to \spc{U}$, we have
\[|F\circ \map[2]\circ\map(x)-F(x)|<\tfrac13\cdot\eps\] 
for any $x\in \spc{U}$.

According to Lemma~\ref{lem:lip-approx},
there is a locally Lipschitz map $F_\eps\:\spc{U}\to \RR^{m+1}$ 
such that $|F_\eps(x)-F(x)|<\tfrac13\cdot\eps$ for any $x\in \spc{U}$.

Note that
$\map(K)$ is contained in a countable subcomplex of $\spc{N}$, say $\spc{N}'$.
Indeed, since $K$ is separable, there is a countable dense collection of points $\{x_n\}$ in $K$.
Denote by $\Delta_n$ the minimal simplex of $\spc{N}$ which contains $\map(x_n)$.
Then $\map(K)\subset\bigcup_i\Delta_n$.

Arguing by contradiction,
assume $\LinDim\spc{U}<m$.
By Lemma~\ref{lem:approximation-cba:im},
the image $F_\eps\circ\map[2]\circ\map(K)$ lies in the $F_\eps$-image of $(m-1)$-skeliton of $\spc{N}'$;
in particular it can be covered 
by countable collection of Lipschitz images of $(m-1)$-simplexes.
Hence
$\bm{0}\in \RR^m$ is not a stable value of the restriction $F_\eps\circ\map[2]\circ\map|K$.
Since $\eps>0$ is arbitrary, 
we get that $\bm{0}\in \RR^m$ is not a stable value of $f=F|K$, a contradiction.

\parit{(\ref{SHORT.LinDim-CBA})$\Rightarrow$(\ref{SHORT.thm:dim-infty-CBA:bary}).} 
The following claim follows from the definition of tangent space.

\begin{clm}{}\label{clm:finite-config-cba}
Let $q\in \spc{U}$ and $\dot x^1,\dot x^2,\dots,\dot x^n\in \T_q$.
Then given $\delta>0$,
there is a point array $x^1,x^2,\dots,x^n$ in $\spc{U}$ 
such that 
\[\mangle(\dot x^i,\ddir q{x^i})<\delta\]
and for some fixed $\lam>0$ we have
\[\tfrac1\lam\cdot\dist[{{}}]{q}{x^i}{}=|\dot x^i|\ \ 
\t{and}\ \  |\tfrac1\lam\cdot\dist[{{}}]{x^i}{x^j}{}-\dist{\dot x^i}{\dot x^j}{}|<\delta\] 
for all $i$ and $j$.

Moreover the value $\lam$ can be taken arbitrary small.
\end{clm}

\parit{Proof of the Claim.} For each $i$ choose a geodesic $\gamma^i$ 
from $q$ which goes almost in the directions of $\dot x^i$.
Then take the point $x^i$ on $\gamma^i$ on the distance $\lam\cdot|\dot x^i|$ from $q$.
\claimqeds


Choose $q\in \spc{U}$ such that $\T_q$ contains a subcone $E$ isometric to $m$-dimensional Euclidean space.
Note that one can choose $\eps>0$ 
and a point arrray $(\dot a^0,\dot a^1,\dots,\dot a^m)$ in $E\subset \T_q$ 
such that 
$\bigcap_i\cBall[\dot a^i,1+\eps]=\emptyset$
and $\bigcap_{i\not=j}\cBall[\dot a^i,1\z-\eps]\z{\not=}\emptyset$ for each $j$.

Applying Claim \ref{clm:finite-config-cba}, we get a point array 
$(a^0,a^1,\dots,a^m)$ in $\spc{U}$
such that $\bigcap_i\cBall[a^i,\lam]=\emptyset$
and $\bigcap_{i\not=j}\cBall[a^i,\lam]\not=\emptyset$ for each $j$.
Since $\lam>0$ can be  chousen arbitrary small, 
it implies (\ref{SHORT.thm:dim-infty-CBA:bary}).
\qeds







 













\section{Remarks and open problems}

The following conjecture (in an equivalent form)
appers in \cite{kleiner}, see also \cite[p.~133]{gromov:asymt-inv}.

\begin{thm}{Conjecture}
For any complete length $\CAT{}$ space $\spc{U}$, we have
\[\TopDim\spc{U}
=
\LinDim\spc{U}.\]

\end{thm}

By Corollary~\ref{cor:dim-CBA} it holds for separable spaces





